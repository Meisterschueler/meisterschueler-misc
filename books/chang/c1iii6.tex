% File: c1iii6

\subsection{Auswendiglernen}
\label{c1iii6} 

\subsubsection{Warum auswendig lernen?}
\label{c1iii6a}

Die Gründe für das Auswendiglernen sind so zwingend, daß es überraschend ist, wie vielen Menschen diese nicht bewußt waren.
\textbf{Fortgeschrittene Pianisten müssen wegen des hohen Grades an technischer Fertigkeit, der erwartet wird, aus dem Gedächtnis spielen.}
Von fast allen Schülern (einschließlich denen, die von sich selber glauben, daß sie schlechte Merkfähigkeiten haben) werden die meisten schwierigen Passagen aus dem Gedächtnis gespielt.
Wer nicht auswendig spielt, muß zwar eventuell zur psychologischen Unterstützung und für einen kleinen Wink ab und zu die Notenblätter vor sich haben, spielt aber in Wirklichkeit schwierige Passagen fast völlig aus dem \enquote{\hyperref[c1iii6d]{Hand-Gedächtnis}\index{Hand-Gedächtnis}}.
\textbf{Wegen dieser Notwendigkeit, aus dem Gedächtnis zu spielen, hat sich das Auswendiglernen zu einem wissenschaftlichen Vorgang entwickelt, der untrennbar mit jedem stichhaltigen Prozeß des Klavierstudiums verbunden ist.}

\textbf{Auswendiglernen ist ein Weg, neue Stücke schnell zu lernen.}
Auf lange Sicht lernen Sie technisch bedeutsame Stücke viel schneller durch Auswendiglernen als durch das Benutzen der Noten.
Auswendiglernen gestattet dem Klavierspieler, irgendwo mitten in einem Stück mit dem Spielen zu beginnen, es ist eine Methode, mit der man über Gedächtnisblockaden und Spielfehler hinwegkommt oder sie sogar völlig vermeidet, und  mit seiner Hilfe entwickelt man ein besseres Verständnis der Musik.
Es erlaubt auszugsweises Spielen (kleine Auszüge aus einer Komposition spielen), eine sehr nützliche Fähigkeit für zwangloses Vorspielen, zum Unterrichten und zum Lernen wie man vorspielt.
Wenn Sie 10 Stunden Repertoire auswendig gelernt haben, was ohne weiteres erreichbar ist, erkennen Sie den Vorteil davon, daß Sie nicht Ihre ganzen Noten mit sich herumtragen müssen oder sie durchsuchen müssen, um ein Musikstück oder einen Auszug zu finden.
Wenn Sie von Auszug zu Auszug springen möchten, wäre die Suche danach in einem Stapel Noten unpraktisch.
Bei Flügeln stört der Notenständer den Ton, d.h. man kann sich selbst nicht spielen hören, wenn der Notenständer aufgestellt ist.
Dieser Effekt ist in einer Konzerthalle oder einem Zuschauerraum mit guter Akustik besonders dramatisch -- der Flügel kann praktisch unhörbar werden.
Aber vor allem \textbf{können Sie sich durch das Auswendiglernen zu 100\% auf die Musik konzentrieren.}
Klavierspielen ist eine darstellende Kunst, und ein auswendig gespielter Vortrag ist für das Publikum lohnender, weil es die Fähigkeit zum Auswendigspielen als ein besonderes Talent ansieht.
Ja, wenn Sie auswendig lernen, werden Sie zu einem dieser genialen Künstler, die von \hyperref[memorizer]{Nichtauswendiglernenden}\index{Nichtauswendiglernenden} beneidet werden!

Der Gewinn aus diesem Buch vervielfacht sich, weil es ein Komplettpaket ist; d.h. das Ganze ist \textit{viel} größer als die Summe seiner Teile.
Auswendiglernen ist ein gutes Beispiel.
Um das zu verstehen, lassen Sie uns diejenigen Schüler betrachten, die nicht auswendig lernen.
Sobald ein neues Stück \enquote{gelernt} aber noch nicht perfektioniert ist, verlassen diese Schüler üblicherweise das Stück und gehen zum nächsten; teilweise, weil es so lange dauert, neue Stücke zu lernen und teilweise, weil die Noten zu lesen dem Aufführen schwieriger Stücke nicht förderlich ist.
In der Regel lernen Schüler, die nicht auswendig lernen, niemals ein Stück wirklich gut, und dieses Handicap begrenzt die technische Entwicklung.
\textbf{Wenn sie nun in der Lage wären, gleichzeitig schnell zu lernen und auswendig zu lernen, würden sie \textit{für den Rest ihres Lebens} mit den fertigen Stücken Musik machen!}
Wir sprechen nicht nur darüber, ein Stück auswendig zu lernen oder nicht auswendig zu lernen -- wir sprechen über einen \textit{lebenslangen} Unterschied in Ihrer Entwicklung als Künstler und darüber, ob Sie wirklich eine Chance haben zu musizieren.
Es ist der Unterschied zwischen einem darstellenden Künstler und einem Schüler, der niemals ein vorführbares Stück hat.
Erst wenn Sie mit einem Stück technisch fertig sind, können Sie überhaupt daran denken, es wirklich musikalisch zu spielen.
Wie schade, daß Schüler, die nicht richtig informiert sind, sich den besten Teil davon entgehen lassen, was es bedeutet, ein Pianist zu sein, und sich die Gelegenheit entgehen lassen, sich als Künstler zu entwickeln.

\textbf{Auswendiglernen nützt der Entwicklung des Gehirns in der Jugend und verlangsamt seinen altersbedingten Verfall.} Das Auswendiglernen von Klaviermusik wird nicht nur Ihr Gedächtnis im täglichen Leben -- außerhalb des Klavierspielens -- verbessern, sondern auch den Gedächtnisverlust im Alter verlangsamen und sogar die Leistungsfähigkeit des Gehirns für das Auswendiglernen verbessern.
Sie werden Methoden lernen, mit denen man das Gedächtnis verbessern kann, und ein Verständnis der Funktion des menschlichen Gedächtnisses entwickeln.
Sie werden zu einem \enquote{Gedächtnisexperten}, was Ihnen Vertrauen in Ihr Erinnerungsvermögen verleiht; ein Mangel an Selbstvertrauen ist ein wichtiger Grund sowohl für ein schlechtes Gedächtnis als auch für viele andere Probleme, wie z.B. ein geringes Selbstwertgefühl.
Das Gedächtnis beeinflußt die Intelligenz in hohem Maße, und ein gutes Gedächtnis erhöht den effektiven IQ.

In meiner Jugend schien das Leben so kompliziert zu sein, daß ich mich, um es zu vereinfachen, intuitiv dem \enquote{Prinzip des geringsten Wissens} anschloß, welches besagt: \enquote{Je weniger unnötige Information man in sein Gehirn stopft, desto besser.}
Diese Theorie ist der für Plattenspeicher in einem Computer analog: \enquote{Je mehr Müll man löscht, desto mehr Speicher hat man zur Nutzung übrig.}
Ich weiß nun, daß dieser Ansatz Faulheit und einen Minderwertigkeitskomplex, daß man kein guter Auswendiglernender sei, erzeugt und schädlich für das Gehirn ist, weil es so ist, als ob man sagt, daß man um so stärker wird, je weniger Muskeln man benutzt, weil mehr Energie übrig bleibt.
Das Gehirn hat die Kapazität, viel mehr zu speichern als jemand in seinem ganzen Leben hineinstecken könnte.
Wenn man aber nicht lernt, es zu benutzen, wird man nie von seinem ganzen Potential profitieren.
Ich habe durch meinen früheren Fehler viel gelitten.
Ich fürchtete mich davor, zum Bowling zu gehen, weil ich meinen Punktestand nicht so wie die anderen im Kopf behalten konnte.
Seit ich meine Philosophie geändert habe, so daß ich nun versuche alles zu behalten, hat sich mein Leben dramatisch verbessert.
Ich versuche nun sogar, mir die Neigung und die Unebenheiten auf jedem Golfgrün, das ich spiele, zu merken.
Das kann einen großen Effekt auf den Score haben.
Natürlich war der entsprechende Nutzen für meine Laufbahn als Klavierspieler unbeschreiblich.


\label{assoziativ}

\textbf{Das Gedächtnis ist eine assoziative Funktion des Gehirns}.
Bei einer assoziativen Funktion wird ein Objekt mit einem anderen in eine bestimmte Beziehung gesetzt.
Praktisch alles, was wir erleben, wird in unserem Gehirn gespeichert, ob wir das wollen oder nicht.
Wenn das Gehirn diese Informationen vom Kurzzeitgedächtnis in das permanente Gedächtnis überträgt (ein automatischer Vorgang, der gewöhnlich 2 bis 5 Minuten dauert), verbleiben sie dort im Grunde ein Leben lang.
Wenn wir auswendig lernen, ist deshalb das Speichern der Informationen nicht das Problem -- das Abrufen der Informationen ist das Problem, weil unser Gedächtnis nicht wie ein Computerspeicher arbeitet, bei dem alle Daten eine Adresse haben, sondern mit einem Verfahren, das wir noch nicht verstehen.
Der Vorgang, den man am besten versteht, ist der assoziative Prozeß: Um uns an \hyperref[johndoe]{Otto}s Telefonnummer zu erinnern, denken wir zuerst an Otto, dann erinnern wir uns daran, daß er verschiedene Telefone besitzt, und dann fällt uns ein, daß seine Mobiltelefonnummer 0xxx-1234567 ist.
Das heißt, die Nummer ist mit dem Mobiltelefon verknüpft, das mit Otto verknüpft ist.
Jede Ziffer der Telefonnummer hat eine umfangreiche Reihe von Verknüpfungen zu unseren Erfahrungen mit Zahlen, angefangen mit den ersten Zahlen, die wir als kleines Kind gelernt haben.
Ohne diese Assoziationen hätten wir keinerlei Vorstellung davon, was Zahlen sind, und wären deshalb nicht in der Lage, uns überhaupt an sie zu erinnern.
\enquote{Otto} hat ebenfalls viele Assoziationen (wie sein Haus, seine Familie usw.), und das Gehirn muß diese alle herausfiltern und nur der Assoziation \enquote{Telefon} folgen, um die Nummer zu finden.
Wegen der großen Leistungsfähigkeit des Gehirns bei der Informationsverarbeitung ist der Abrufprozeß effizienter, wenn mehr Assoziationen existieren, und die Anzahl der Assoziationen wächst schnell, wenn mehr Informationen gespeichert werden, weil sie untereinander vernetzt werden können.
Deshalb ist das menschliche Gedächtnis zum Computerspeicher fast diametral verschieden: Je mehr man auswendig lernt, desto leichter wird es, mehr auswendig zu lernen, weil man mehr Assoziationen erzeugen kann.
Leider nutzen die meisten von uns die Vorteile dieser unglaublich effizienten Methode des Gedächtnisses nicht vollständig aus; wir erzeugen nur eine kleine Zahl von Assoziationen und unser Gehirn filtert diese nicht immer erfolgreich, um bei einer bestimmten gespeicherten Information anzukommen.
Die Kapazität unseres Gedächtnisses ist so groß, daß sie im Grunde unendlich ist.
Sogar gute Auswendiglernende \enquote{sättigen} ihr Gedächtnis nie, bis die Zeichen des Alters anfangen, Ihren Tribut zu fordern.
Wenn mehr Material in das Gedächtnis gestellt wird, erhöht sich die Anzahl der Assoziationen geometrisch.
Dieser geometrische Zuwachs erklärt teilweise den enormen Unterschied in der Speicherkapazität von guten und schlechten Auswendiglernenden.
Somit sagt uns alles, was wir über das Gedächtnis wissen, daß uns Auswendiglernen nur nützlich sein kann.


\subsubsection{Wer kann auswendig lernen, was und wann?}
\label{c1iii6b}

\textbf{Jeder kann lernen auswendig zu lernen, wenn man ihm die richtigen Methoden dafür beibringt.}
Wir zeigen hier, daß man die für das Auswendiglernen erforderliche Zeit auf ein vernachlässigbares Maß reduzieren kann, wenn man das Auswendiglernen mit dem anfänglichen Lernen des Musikstücks kombiniert.
Tatsächlich kann \textbf{die richtige Integration der Verfahren für das Lernen und das Auswendiglernen die für das Lernen erforderliche Zeit reduzieren und dem Auswendiglernen praktisch einen negativen Zeitanteil zuweisen.
Es stellt sich heraus, daß fast alle für das Auswendiglernen erforderlichen Elemente auch erforderliche Elemente für das Lernen sind.
Wenn man diese Prozesse voneinander trennt, muß man am Ende dieselbe Prozedur zweimal durchlaufen.
Niemand würde eine solche Tortur auf sich nehmen (oder zumindest wenige); das erklärt, warum diejenigen, die nicht bereits während des ersten Lernens auswendig lernen, niemals gut auswendig lernen.}

Da Auswendiglernen der schnellste Weg zu lernen ist, sollten Sie jedes lohnende Stück, das sie spielen, auswendig lernen.
Das Auswendiglernen ist ein kostenloses Nebenprodukt des Prozesses, ein neues Musikstück zu lernen.
\textbf{Deshalb sind die Anweisungen für das Auswendiglernen im Prinzip trivial: Befolgen Sie einfach die in diesem Buch angegebenen Lernregeln -- mit der zusätzlichen Anforderung, daß Sie während dieser Lernvorgänge alles aus dem Gedächtnis heraus durchführen.}
Lernen Sie z.B., während Sie eine LH-Begleitung Takt für Takt lernen, diese Takte auswendig.
Da ein Takt üblicherweise aus 6 bis 12 Noten besteht, ist es einfach, diesen auswendig zu lernen.
Sie werden dann in Abhängigkeit von der Schwierigkeit diese Abschnitte 10, 100, oder mehr als 1000mal wiederholen müssen, bevor Sie das Stück spielen können -- das sind viel mehr Wiederholungen als zum Auswendiglernen benötigt werden.
Sie können gar nicht anders als es auswendig zu lernen!
Warum also eine solch unbezahlbare und einmalige Gelegenheit versäumen?

Wir haben in den Abschnitten I und II gesehen, daß es der Schlüssel zum schnellen Lernen der Technik ist, die Musik auf völlig einfache Teilmengen zu reduzieren; dieselben Prozeduren vereinfachen auch das Auswendiglernen dieser Teilmengen.
Auswendiglernen kann eine enorme Menge an Übungszeit einsparen.
Sie müssen nicht jedesmal auf die Noten sehen, so daß Sie einen RH-Ausschnitt einer Beethoven-Sonate und einen LH-Ausschnitt eines Chopin-Scherzos mit HS üben und beliebig von Ausschnitt zu Ausschnitt springen können.
Sie können sich auf das Lernen der Technik konzentrieren, ohne sich jedesmal durch das Nachsehen der Noten ablenken lassen zu müssen.
Das beste von allem ist, daß die Vielzahl der Wiederholungen, die Sie benötigen, um das Stück zu üben, das Stück ohne zusätzlichen Zeitaufwand und in einer Art und Weise an das Gedächtnis übergibt, die mit keiner anderen Prozedur erreicht wird.
Das sind einige der Gründe, warum das Auswendiglernen vor dem Lernen der einzige Weg ist.

\textbf{Schließlich führt Auswendiglernen, und das ist der \enquote{entscheidende Faktor}, zu \hyperref[c1iii6tastatur]{mentalem Spielen}\index{mentalem Spielen} (s. \autoref{c1iii6tastatur}), das der Schlüssel zu einem \hyperref[c1iii12]{absoluten Gehör}\index{absoluten Gehör}, einem höheren effektiven IQ, reduzierter \hyperref[c1iii15]{Nervosität}\index{Nervosität} bzw. Streß, zum Komponieren und zur Fähigkeit, mit Leichtigkeit und ohne Fehler vorzuspielen, ist.}
Beim mentalen Spielen können Sie das ganze Stück in Gedanken, ohne Klavier, spielen.
Um auf die Stufe eines Konzertpianisten zu gelangen, müssen Sie das mentale Spielen lernen.
Alle großen Pianisten und Komponisten wurden auf diese Art zu dem, was sie waren.
Praktisch jeder vollendete Klavierspieler komponiert am Ende Musik; Auswendiglernen, absolutes Gehör und mentales Spielen sind entscheidende Elemente für ein erfolgreiches Komponieren.


\subsubsection{Auswendiglernen und Pflege des Gelernten}
\label{c1iii6c}

\textbf{Ein auswendig gelerntes Repertoire erfordert zwei Zeitinvestitionen: die erste für das anfängliche Auswendiglernen des Stücks und eine zweite \enquote{Pflegekomponente}, um das Gedächtnis dauerhafter zu verankern und um etwaige vergessene Abschnitte zu reparieren.}
Während der Lebensspanne eines Pianisten ist die zweite Komponente die bei weitem größere, weil die anfängliche Investition null oder sogar negativ ist.
Die Pflege ist ein Grund, warum einige das Auswendiglernen aufgeben: \enquote{Warum auswendig lernen, wenn ich es sowieso wieder vergesse?}
Die Pflege kann die Größe des Repertoires begrenzen, denn wenn man z.B. fünf bis zehn Stunden Musik auswendig gelernt hat, dann schließen die Erfordernisse der Pflege in Abhängigkeit von der Person eventuell das Auswendiglernen zusätzlicher Stücke aus.
Es gibt mehrere Wege, Ihr Repertoire über diese pflegebedingte Grenze hinaus zu erweitern.
Ein offensichtlicher Weg ist, die auswendig gelernten Stück wegzulegen und sie später erneut auswendig zu lernen, wenn es notwendig ist.
\textbf{Stücke, die hinreichend gut auswendig gelernt wurden, können sehr schnell wieder aufpoliert werden, selbst wenn man sie jahrelang nicht gespielt hat.}
Es ist fast wie Fahrradfahren; hat man erst einmal gelernt, wie man Fahrrad fährt, muß man nie wieder alles erneut lernen.
Wir werden im folgenden verschiedene Pflegeverfahren besprechen, die Ihr auswendig gelerntes Repertoire bedeutend vergrößern können. 

\textbf{Lernen Sie so viele Stücke wie möglich auswendig, bevor Sie 20 Jahre alt sind.
Stücke, die man in diesen frühen Jahren lernt, werden praktisch nie vergessen, und selbst wenn sie vergessen werden, kann man sie sich am leichtesten wieder in Erinnerung rufen.}
Deshalb sollten junge Menschen ermutigt werden, alle Stücke ihres Repertoires auswendig zu lernen.
Stücke, die später als mit 40 Jahren gelernt werden, erfordern einen höheren Aufwand für das Auswendiglernen und die Pflege, obwohl viele Menschen über 60 keine Probleme damit haben, neue Stücke auswendig zu lernen (wenn auch langsamer als vorher).
Beachten Sie das Wort \enquote{lernen} in den vorangegangenen Sätzen; die Stücke müssen nicht unbedingt\footnote{in jüngeren Jahren} auswendig gelernt worden sein, und Sie können sie trotzdem später im Vergleich zu Stücken, die sie im späteren Alter gelernt oder auswendig gelernt haben, mit besserer Merkfähigkeit auswendig lernen.

Es gibt Gelegenheiten, bei denen Sie nicht auswendig lernen müssen, wie z.B. wenn Sie eine große Zahl leichter Stücke, besonders Begleitungen, lernen möchten, bei denen es zu lange dauern würde, sie auswendig zu lernen und zu pflegen.
Eine weitere Gruppe von Musikstücken, die Sie nicht auswendig lernen sollten, sind jene, die Sie zum Üben des Blattspiels benutzen.
\hyperref[c1iii11]{Vom Blatt zu spielen}\index{Vom Blatt zu spielen} ist eine gesonderte Fertigkeit, die in \autoref{c1iii11} behandelt wird.
\textbf{Jeder sollte ein auswendig gelerntes Repertoire haben, sowie ein vom Blatt zu spielendes Repertoire}, um die Fertigkeit im Blattspiel zu verbessern.

\textbf{Wenn Sie ein Stück gut spielen können, es aber nicht auswendig gelernt haben, kann es sehr frustrierend sein, zu versuchen, das Stück auswendig zu lernen.}
Zu viele Schüler sind aufgrund dieser Schwierigkeiten davon überzeugt, daß Sie schlecht auswendig lernen können.
Das geschieht, weil der Teil der Motivation zum Auswendiglernen, der aus der Zeitersparnis während des ersten Lernens des Stücks resultiert, entfällt, wenn man das Stück bereits mit der vorgegebenen Geschwindigkeit spielen kann.
Die einzige übrigbleibende Motivation ist die Annehmlichkeit, aus dem Gedächtnis vorzuspielen.
Mein Vorschlag an diejenigen, die glauben, sie seien schlechte Auswendiglernende: Lernen Sie ein völlig neues Stück, das Sie nie zuvor studiert haben, indem Sie es von Anfang an mit den Methoden dieses Buchs auswendig lernen.
Sie werden angenehm überrascht sein, wie gut Sie beim Auswendiglernen sind.
Die meisten Fälle von \enquote{schlechtem Gedächtnis} resultieren aus der Lernmethode, nicht aus der Speicherfähigkeit des Gehirns.
Wegen der Wichtigkeit des Themas \enquote{\hyperref[c1iii6l]{Blattspieler und Auswendiglernende}\index{Blattspieler und Auswendiglernende}} wird es später noch einmal behandelt.
 

\subsubsection{Hand-Gedächtnis}
\label{c1iii6d}

\textbf{Eine große Komponente Ihres anfänglichen Gedächtnisses wird das Hand-Gedächtnis sein, das vom wiederholten Üben kommt.
Die Hand spielt einfach weiter, ohne daß Sie sich wirklich an jede einzelne Note  erinnern.}
Obwohl wir weiter unten alle bekannten Arten des Gedächtnisses besprechen werden, fangen wir zunächst mit der Analyse des Hand-Gedächtnisses an, weil es früher häufig als die einzige und beste Gedächtnismethode angesehen wurde, obwohl es in Wirklichkeit die unwichtigste ist.
Das Hand-Gedächtnis besteht aus mindestens zwei Komponenten: einer reflexartigen Handbewegung, die aus der Berührung der Tasten resultiert, und einem Reflex im Gehirn auf den Klang des Klaviers.
Beide dienen als Stichwort für Ihre Hand, sich in einer bestimmten vorprogrammierten Weise zu bewegen.
Der Einfachheit halber werden wir sie zusammenfassen und als Hand-Gedächtnis bezeichnen.
Das Hand-Gedächtnis ist nützlich, weil es Ihnen hilft, während des Übens gleichzeitig auswendig zu lernen.
Tatsächlich muß jeder allgemeine Konstrukte -- wie Tonleitern, Arpeggios, Alberti-Begleitungen usw. -- aus dem Hand-Gedächtnis heraus üben, so daß Ihre Hände sie automatisch spielen können, ohne daß Sie an jede einzelne Note denken müssen.
Deshalb gibt es, wenn Sie anfangen ein neues Stück auswendig zu lernen, keinen Grund, das Hand-Gedächtnis bewußt zu vermeiden.
Einmal erworben, wird man das Hand-Gedächtnis niemals verlieren, und wir zeigen unten, wie man es benutzen kann, um nach einem Hänger den Faden wiederzufinden.

Der biologische Mechanismus, durch den die Hände das Hand-Gedächtnis erwerben, wird nicht so gut verstanden, aber meine Hypothese ist, daß er Nervenzellen außerhalb des bewußten Teils des Gehirns, wie z.B. Nervenzellen im Rückenmark, zusätzlich zum Gehirn einbezieht.
Die Zahl der Nervenzellen außerhalb des Gehirns ist wahrscheinlich der Zahl derer im Gehirn vergleichbar.
Obwohl die Befehle für das Klavierspielen aus dem Gehirn stammen müssen, ist es sehr wahrscheinlich, daß die schnellen Spielreflexe nicht den ganzen Weg zum bewußten Gehirn hinauf zurücklegen.
\textbf{Deshalb muß das Hand-Gedächtnis eine Art Reflex sein, der viele Nervenzelltypen einbezieht.
Als Antwort auf das Spielen der ersten Note spielt der Reflex die zweite Note, was die dritte Note anregt, usw.}
Das erklärt, warum Ihnen das Hand-Gedächtnis nicht dabei hilft, neu zu starten, wenn Sie hängengeblieben sind, solange Sie nicht bis zur ersten Note zurückgehen.
Tatsächlich ist das Neustarten eines Stücks an einer willkürlichen Stelle ein hervorragender Test, ob Sie aus dem Hand-Gedächtnis spielen oder ob Sie für den Notfall noch eine andere Gedächtnismethode haben.
Da es nur eine konditionierte Antwort ist, ist das Hand-Gedächtnis kein wirkliches Gedächtnis und hat zahlreiche schwerwiegende Nachteile.

Wenn wir über das Hand-Gedächtnis sprechen, meinen wir im allgemeinen ein HT-Gedächtnis.
\textbf{Da das Hand-Gedächtnis nur nach vielen Wiederholungen erworben wird, ist es eines der am schwersten zu löschenden oder zu ändernden Gedächtnisse.}
Das ist einer der Hauptgründe für das HS-Üben -- zu vermeiden, sich inkorrekte HT-Angewohnheiten anzueignen, die man nur sehr schwer ändern kann.
Das HS-Gedächtnis ist vom HT-Gedächtnis grundlegend verschieden.
Das HS-Spielen ist einfacher und kann direkt vom Gehirn gesteuert werden.
Beim HT-Gedächtnis braucht man eine Art Rückkopplung, um die Hände (und wahrscheinlich die beiden Gehirnhälften) bis zu der Genauigkeit zu koordinieren, die für die Musik erforderlich ist.
Deshalb ist das HS-Üben die effektivste Methode zur Vermeidung der Abhängigkeit vom Hand-Gedächtnis.

Es ist nicht möglich, eine klare Trennlinie zwischen Technik und Gedächtnis zu ziehen.
Ein Klavierspieler mit mehr Technik kann schneller auswendig lernen.
Ein Grund, warum man das Gedächtnis nicht von der Technik trennen kann, ist, daß beides zum Spielen notwendig ist, und solange man nicht spielen kann, kann man weder Technik noch Gedächtnis demonstrieren.
Es gibt deshalb (neben der bloßen Annehmlichkeit Zeit zu sparen usw.) eine tiefere biologische Basis, die der Methode dieses Buchs zugrunde liegt, durch die Gedächtnis und Technik gleichzeitig erworben werden.
 


