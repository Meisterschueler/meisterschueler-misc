% File: c1iii16d

\subsection{Einige Elemente des Klavierunterrichts}\hypertarget{c1iii16d}{}

Der Klavierunterricht sollte kein Routineablauf sein, bei dem der Schüler das Unterrichtsstück spielt und der Lehrer ein neues Stück zuweist.
\textbf{Beim Beginnen eines neuen Stücks ist es die Aufgabe des Lehrers, es in Abschnitten durchzugehen, den Fingersatz zu untersuchen, die Noten zu analysieren und im Grunde den Schüler während der Unterrichtsstunde auf die endgültige Geschwindigkeit zu bringen, zumindest mit HS oder abschnittsweise.}
Nachdem die technischen Probleme gelöst sind, ändert sich die Aufgabe zum musikalischen Spielen hin: den musikalischen Inhalt untersuchen, den Ausdruck hervorbringen, die Eigenschaften des Komponisten (Mozart unterscheidet sich von Chopin usw.), die Farbe usw.
Ein guter Lehrer kann den Schülern eine enorme Menge Zeit sparen, indem er ihnen alle notwendigen Elemente der Technik demonstriert.
Es sollte nicht dem Schüler überlassen werden, diese durch Versuch und Irrtum herauszufinden.
Aufgrund dieser Erfordernisse können Unterrichtsstunden jenseits der Anfängerstufe sehr intensiv und zeitaufwendig werden.
\textbf{Tonleitern sollten Anfängern mit dem Daumenuntersatz gelehrt werden, aber innerhalb eines Jahres sollte ihnen auch der \hyperlink{c1iii5b}{Daumenübersatz} beigebracht werden.}
Obwohl die meisten Übungen wie \hyperlink{c1iii7h}{Hanon} heute als nicht hilfreich angesehen werden, ist es sehr wichtig, in der Lage zu sein, \hyperlink{c1iii5}{Tonleitern und Arpeggios} (in allen Tonarten)
gut zu spielen; das wird mehrere Jahre harter Arbeit erfordern.

Jeden zweiten bis dritten Tag 30 Minuten zu üben, ist das absolut notwendige Minimum, um überhaupt Fortschritte zu machen.
Eine halbe Stunde täglich ist bei Kindern für einen bedeutenden Fortschritt angemessen.
Wenn sie älter werden, brauchen sie stetig mehr Zeit.
Das sind die minimalen Übungszeiten; für einen schnelleren Fortschritt wird mehr Zeit benötigt.
Wenn die Übungsmethoden effizient sind und die Schüler gute Fortschritte machen, wird die Frage, wieviel Übungszeit genug ist, bedeutungslos - es gibt soviel Musik und macht soviel Spaß, daß nie genug Zeit vorhanden ist.

\textbf{Der beste Weg, Schüler zum Üben zu motivieren, und die beste Art, die Kunst Musik zu machen zu lehren, ist, \hyperlink{c1iii14}{Konzerte} abzuhalten.}
Wenn die Schüler auftreten müssen, bekommen alle Anweisungen des Lehrers, die notwendige Übungszeit usw. eine völlig neue Bedeutung und Dringlichkeit.
\textbf{Die Schüler werden dadurch selbstmotiviert.}
Es ist ein Fehler, Klavier ohne jegliches Programm zum Auftreten zu lehren.
Es gibt zahlreiche Möglichkeiten für solche Programme, und erfahrene Lehrer sind in der Lage, für jeden Schüler jeder Stufe ein angemessenes zu entwickeln.
Formelle Konzerte und Musikwettbewerbe sind voller Fallen und müssen mit Sorgfalt und jeder Menge Planung angegangen werden.
Lehrer können jedoch informelle Konzerte in weniger streßbeladenen Formaten organisieren, die für die Schüler einen enormen Nutzen haben.

\textbf{Obwohl Konzerte und Wettbewerbe wichtig sind, ist es noch wichtiger, ihre Gefahren zu vermeiden.}
Die Hauptgefahr ist, daß Konzerte selbstzerstörerisch sein können, weil der Streß, die Nervosität, der zusätzliche Aufwand, die zusätzliche Zeit und das Gefühl des Versagens auch nach kleinen Fehlern beim Formen der Fähigkeit und der psychologischen Grundlage des Schülers zum Auftreten in jedem Alter mehr Schaden anrichten als Gutes tun können.
Deshalb \textbf{müssen Lehrer ein klar definiertes Programm bzw. eine Vorgehensweise haben, die Kunst des Auftretens zusätzlich zur Kunst des Spielens zu lehren.}
Die vorbereitenden Methoden für Konzerte, die oben in \hyperlink{c1iii14}{Abschnitt 14} besprochen wurden, sollten Teil dieses Programms sein.
Pop-Musik oder Musik, \enquote{die Spaß macht}, ist besonders für das Auftrittstraining geeignet.
Vor allem muß das Programm so gestaltet sein, daß es eine belohnende Atmosphäre der Leistungsfähigkeit erzeugt und keine wettbewerbsorientierte, bei der, wenn der Schüler die schwierigsten Stücke spielt, die er bewältigen kann, alles was geringer ist als unglaubliche Perfektion ein Versagen ist.
Für Wettbewerbe muß den Schülern bereits früh beigebracht werden, daß die Beurteilung nie perfekt oder fair ist; daß es nicht der Sieg sondern der Prozeß der Teilnahme ist, der wegen seines pädagogischen Werts am wichtigsten ist.
Ein entspannter und weniger nervöser Schüler wird das gleiche vorgegebene Stück besser ausführen und eine bessere Einstellung zum Auftreten entwickeln.
Die Schüler müssen verstehen, daß der Prozeß das endgültige Ziel eines Wettbewerbs ist, nicht daß man am Ende gewinnt.
Eine der wichtigsten Komponenten dieses Ziels ist es, die Fähigkeit zu entwickeln, die Erfahrung zu genießen, anstatt nervös zu werden.
Eine der wichtigsten Gefahren der meisten Wettbewerbe ist die Betonung des schwierigsten Materials, daß der Schüler spielen kann.
Der korrekte Schwerpunkt sollte die Musik sein, nicht die Akrobatik.

Natürlich müssen wir danach streben, Wettbewerbe zu gewinnen und fehlerlose Konzerte zu spielen.
Es gibt aber streßbeladene und weniger streßbeladene Herangehensweisen für diese Ziele.
\textbf{Es ist die Aufgabe des Lehrers, die Streßkontrolle zu lehren.}
Leider ignoriert die Mehrheit der Lehrer heutzutage völlig die Kontrolle des Stresses bei Auftritten oder schlimmer noch, Eltern und Lehrer tun häufig so, als ob es so etwas wie Nervosität nicht gäbe, sogar wenn sie selbst nervös sind.
Das kann den Effekt haben, ein dauerhaftes Problem mit der Nervosität zu erzeugen.
Sehen Sie dazu oben in Abschnitt 15 eine Besprechung über die \hyperlink{c1iii15}{Kontrolle der Nervosität}.

\textbf{Es ist wichtig, einem Schüler zunächst alles über Nervosität und Streß beizubringen und
ihn nicht auf die Bühne zu schubsen, um ohne Vorbereitung aufzutreten, in der vergeblichen Hoffnung, daß er irgendwie von selbst lernen wird, wie man auftritt.}
Solch ein Vorgehen ist so ziemlich das gleiche wie jemanden in der Mitte eines tiefen Sees ins Wasser zu werfen, um ihm das Schwimmen beizubringen; diese Person kann für den Rest ihres Lebens Angst vor dem Wasser haben.
In jeder Unterrichtsstunde für den Lehrer zu spielen, ist zwar ein guter Anfang aber eine beklagenswert ungenügende Vorbereitung.
Deshalb sollte der Lehrer einen Plan für ein \enquote{Auftrittstraining} entwickeln, bei dem der Schüler schrittweise in die Auftritte eingeführt wird.
Dieses Training muß während der ersten Unterrichtsstunden beginnen.
Verschiedene Fertigkeiten, wie über Gedächtnisblockaden hinwegkommen bzw. sie vermeiden, Fehler kaschieren, Fehler erahnen bevor sie auftreten, Auszüge-Spielen, an einer beliebigen Stelle im Stück anfangen, die Auswahl der aufzuführenden Stücke, Kommunikation mit dem Publikum usw., sollten gelehrt werden.
Vor allem müssen sie das \hyperlink{c1ii12mental}{mentale Spielen} lernen.
Wir haben gesehen, daß HS-Üben, \hyperlink{c1ii17}{langsames Spielen} und \hyperlink{c1iii6g}{\enquote{kalt} spielen} die wichtigen Komponenten der Vorbereitung sind.
Die meisten Schüler wissen nicht, welche \enquote{fertigen} Stücke sie zufriedenstellend aufführen können, bevor sie es nicht wirklich mehrere Male tun; deshalb wird jeder Schüler, auch unter den fertigen Stücken, ein \enquote{aufführbares} und ein \enquote{nicht aufführbares} Repertoire haben.
\textbf{Eine der besten Möglichkeiten, für Auftritte zu trainieren, ist, die fertigen Stücke des Schülers \hyperlink{c1iii13}{aufzunehmen} und ein Album des fertigen Repertoires herzustellen, das regelmäßig auf den neuesten Stand gebracht wird, wenn der Schüler Fortschritte macht.}
Das sollte von Beginn des Unterrichts an durchgeführt werden, damit diese Fertigkeit so früh wie möglich entwickelt wird.
Der erste Fehler, den die meisten Klavierspieler begehen, ist, zu denken: \enquote{Ich bin noch ein Anfänger, deshalb ist mein Spiel es nicht wert, aufgenommen zu werden.}
Wenn man das am Anfang glaubt, wird man es schließlich für den Rest des Lebens befolgen, weil es zu einer sich selbst erfüllenden Prophezeiung wird.
Diese Behauptung ist falsch, weil Musik das Höchste ist - leichte Kompositionen, die musikalisch gespielt werden, sind kaum zu übertreffen; Horowitz kann \enquote{Alle meine Entchen} auch nicht besser spielen als ein gut unterrichteter Anfänger.

Ohne ein Auftrittstraining werden sogar gute Künstler nicht entsprechend ihrer besten Fähigkeiten vorspielen, und die Mehrheit der Schüler wird am Ende glauben, daß ein Auftritt am Klavier die reine Hölle ist und die Musik oder das Klavier daran schuld ist.
Hat sich diese Einstellung bereits während der Jugend verfestigt, wird sie in das Erwachsenenalter übernommen.
In Wahrheit sollte es genau das Gegenteil sein.
Das Auftreten sollte das endgültige Ziel sein, die endgültige Belohnung für all die harte Arbeit.
Es ist die Demonstration der Fähigkeit, ein Publikum zu beherrschen, die Fähigkeit, die größten Ideen der größten musikalischen Genies, die je gelebt haben, zu übermitteln.
\textbf{Ein sicheres \hyperlink{c1ii12mental}{mentales Spielen} ist die effektivste Methode, das Lampenfieber zu reduzieren.}

Eine Möglichkeit, Schüler in das Auftreten bei Konzerten einzuführen, ist, simulierte Konzerte unter den Schülern abzuhalten und sie Ihre Befürchtungen, Schwierigkeiten, Schwächen und Stärken diskutieren zu lassen, um sie alle mit den wichtigsten Punkten vertraut zu machen.
\hyperlink{c1ii12mental}{Wie spielt man in Gedanken}?
Tut man es ständig?
Benutzt man das \hyperlink{c1iii6foto}{fotografische Gedächtnis} oder das \hyperlink{c1iii6tastatur}{Tastatur-Gedächtnis} oder nur das \hyperlink{c1iii6musik}{Musik-Gedächtnis}?
Geschieht es automatisch oder macht man es zu bestimmten Zeiten?
Sie werden die einzelnen Punkte besser verstehen, wenn sie sie tatsächlich erfahren können und sie dann mit ihren Mitschülern offen besprechen.
Jeder Streß oder \hyperlink{c1iii15}{Nervosität}, die sie fühlen könnten, wird weniger angsteinflößend, wenn sie erkennen, daß jeder dieselben Dinge erlebt, daß Nervosität absolut natürlich ist, und daß es verschiedene Möglichkeiten gibt, sie zu bekämpfen oder sogar einen Vorteil daraus zu ziehen.
Insbesondere wird der ganze Prozeß viel weniger mysteriös und furchterregend, wenn sie erst einmal durch den kompletten Prozeß vom Anfang bis zum Ende eines simulierten Konzerts hindurchgegangen sind.
\textbf{Schülern muß beigebracht werden, daß zu lernen, Spaß am Auftreten zu haben, ein Teil der Kunst des Klavierspielens ist.
Diese \enquote{Kunst des Auftretens} erfordert, so wie die Fingertechnik, ebenfalls Studium und Übung}.
In einer Gruppe von Schülern gibt es immer einen, der gut im Auftreten ist.
Die anderen können durch die Beobachtung der guten Schüler lernen und durch die Diskussion darüber, wie diese die einzelnen Aufgaben bewältigen.
Es gibt aber auch Schüler, die auf der Bühne einfach nur erstarren - diese benötigen besondere Hilfe, wie z.B. sehr einfache Stücke zum Aufführen zu lernen, während eines Konzerts mehrere Gelegenheiten zum Auftreten zu bekommen oder in einer Gruppe oder einem Duo aufzutreten.

\textbf{Eine andere Möglichkeit, Schüler in das Auftreten einzuführen und gleichzeitig etwas Spaß zu haben, ist, ein informelles Konzert anzusetzen, in dem die Schüler das Spiel \enquote{Wer kann am schnellsten spielen?} spielen.}
Bei diesem Spiel spielt jeder Schüler das gleiche Stück aber der Zeitraum zum Üben ist begrenzt, z.B. auf drei Wochen.
Beachten Sie, daß bei dieser List die verborgene Tagesordnung ist, den Schülern beizubringen, wie man Konzerte genießt, nicht ihnen beizubringen, wie man schnell spielt.
Die Schüler stimmen selbst darüber ab, wer der Sieger ist.
Zunächst gibt der Lehrer keine Anweisungen; die Schüler müssen ihre eigenen Übungsmethoden auswählen.
Nach dem ersten Konzert hält der Lehrer eine Gruppenstunde, in der die Schüler ihre Übungsmethoden diskutieren und der Lehrer nützliche Informationen hinzufügt.
Selbstverständlich müssen Klarheit, Genauigkeit und die Musik bei der Wahl des Gewinners berücksichtigt werden.
Man kann die Musik schneller klingen lassen, indem man langsamer aber genauer spielt.
Es wird große Unterschiede in den Übungsmethoden und den erzielten Ergebnissen bei den einzelnen Schülern geben, und auf diese Art werden sie voneinander lernen und die Grundlagen besser verstehen.
Während die Schüler an einem \enquote{Wettbewerb} teilnehmen, muß der Lehrer sicherstellen, daß es eine freudige Erfahrung ist, eine Möglichkeit, die Freude am Auftreten zu erfahren, eine Möglichkeit, die \hyperlink{c1iii15}{Nervosität} völlig zu vergessen.
Fehler erzeugen Gelächter, es sollte nicht die Nase über sie gerümpft werden.
Und nachher könnten Erfrischungen gereicht werden.
Der Lehrer darf nicht vergessen, neben den Anweisungen zum Lernen der \enquote{Wettbewerbsfähigkeiten} auch hin und wieder Anweisungen zum Lernen des Auftretens einzustreuen.

Wie sollten Konzerte organisiert sein, nachdem den Schülern die Grundlagen des Auftretens beigebracht wurden?
Sie sollten so gestaltet sein, daß sie die Fähigkeit aufzutreten verstärken.
\textbf{Eines der schwersten Dinge ist, dieselbe Komposition mehrere Male am gleichen Tag oder an aufeinanderfolgenden Tagen aufzuführen.}
Deshalb bieten solche wiederholten Auftritte das beste Training für die Verstärkung der Fähigkeit aufzutreten.
Für Lehrer oder Schulen mit genügender Schülerzahl ist der folgende Plan gut zu verwenden.
Teilen Sie die Schüler in Gruppen für Anfänger, Mittelstufe und Fortgeschrittene auf.
Halten Sie am Freitag ein Konzert für die Anfänger ab, mit ihren Eltern und Freunden als Publikum.
Anfänger sollten ab ihrem ersten Unterrichtsjahr, bereits in einem Alter von 4 oder 5 Jahren, an Konzerten teilnehmen.
Am Ende dieses Konzerts spielen die fortgeschrittenen Schüler ebenfalls, was es für das Publikum wirklich lohnend macht, das Konzert zu besuchen.
Am Samstag spielen die Mittelstufenschüler, mit ihren Eltern und Freunden als Publikum; wieder spielen am Ende die fortgeschrittenen Schüler.
Am Sonntag halten die fortgeschrittenen Schüler ihr Konzert ab, mit ihren Eltern als Publikum; einige besondere Gäste könnten eingeladen werden.
Auf diese Weise müssen die fortgeschrittenen Schüler dasselbe Stück an drei Tagen hintereinander aufführen.
Das Sonntagskonzert der fortgeschrittenen Schüler sollte aufgenommen und auf CD überspielt werden, da diese großartige Souvenirs sind.
Wenn diese Art von Konzert zweimal im Jahr abgehalten wird, dann hat jeder fortgeschrittene Schüler jedes Jahr sechs Konzerte \enquote{in der Tasche}.
Wenn diese Schüler auch zu Wettbewerben geschickt werden (was üblicherweise eine Ausscheidung, ein Finale und wenn man gewinnt noch ein Abschlußkonzert bedeutet), dann haben sie ein angemessenes Auftrittstraining (mindestens 9 Auftritte im Jahr).
Da die meisten Stücke nicht \enquote{sicher} sind, bis sie dreimal aufgeführt wurden, dient dieser Konzertplan auch dazu, das Konzertstück \enquote{sicher} zu machen, so daß es nun, nach nur einem Konzertwochenende, in das \enquote{aufführbare} Repertoire aufgenommen werden kann.

\textbf{Lehrer sollten gewillt sein, mit anderen Lehrern zu kommunizieren, Ideen auszutauschen und voneinander zu lernen.}
Es gibt nichts potentiell schädlicheres für einen Schüler als einen Lehrer, dessen Lehrmethoden inflexibel und an einem gewissen Zeitpunkt stehengeblieben sind.
In diesem Informationszeitalter gibt es so etwas wie geheime Methoden, das Klavierspielen zu unterrichten, nicht, und der Erfolg des Lehrers hängt von der offenen Kommunikation ab.
Ein wichtiger Punkt der Kommunikation ist der Austausch der Schüler.
Die meisten Schüler können in hohem Maß davon profitieren, daß sie von mehr als einem Lehrer unterrichtet werden.
Lehrer von Anfängern sollten ihre Schüler, sobald sie so weit sind, an Lehrer der höheren Stufen weiterreichen.
Natürlich werden die meisten Lehrer versuchen, ihre besten Schüler zu behalten und so viele Schüler zu unterrichten wie sie können.
Eine Möglichkeit, dieses Problem zu lösen, ist für die Lehrer, eine Gruppe von Lehrern mit verschiedenen Spezialgebieten zu bilden, so daß die Gruppe eine komplette Schule bildet.
Das hilft auch den Lehrern, weil es für sie viel einfacher wird, Schüler zu finden.
Für Schüler, die gute Lehrer suchen, ist es aufgrund dieser Überlegungen klar, daß es am besten ist, eher nach Lehrergruppen zu suchen als nach Lehrern, die einzeln arbeiten.
Lehrer können ebenfalls davon profitieren, wenn sie sich zusammenschließen und die Schüler und die Kosten für die Einrichtungen teilen.

\textbf{Lehrer, die gerade beginnen, haben oft Schwierigkeiten, ihre ersten Schüler zu finden.}
Sich einer Gruppe von Lehrern anzuschließen, ist eine gute Möglichkeit anzufangen.
Auch müssen viele etablierte Lehrer oft Schüler aus Zeitmangel abweisen, besonders wenn der Lehrer in seinem Einzugsgebiet einen guten Ruf hat.
Diese Lehrer sind gute Quellen für Schüler.
Eine Möglichkeit, den Vorrat an potentiellen Schülern zu erhöhen, ist, den Schülern anzubieten, sie bei ihnen zu Hause zu unterrichten.
Zumindest für die ersten paar Jahre könnte dies ein guter Ansatz für das Vergrößern des potentiellen Schülerreservoirs sein.


\subsection{Warum die größten Pianisten nicht unterrichten konnten}\hypertarget{c1iii16e}{}

Sehr wenige der großen Pianisten waren gute Lehrer.
Das ist vollkommen natürlich, weil Künstler ihr ganzes Leben trainieren Künstler zu sein und nicht Lehrer.
Ich habe als Physikstudent an der Cornell University eine ähnliche Situation erlebt. Ich nahm Kurse bei Professoren, die auf das Unterrichten spezialisiert waren, und besuchte auch wöchentliche Vorlesungen berühmter Physiker, darunter zahlreiche Nobelpreisgewinner.
Einige dieser berühmten Physiker konnten gewiß spannende Vorlesungen halten, die großes Interesse hervorriefen, aber ich lernte die meisten Fertigkeiten, die notwendig waren, um einen Job als Physiker zu finden, von den unterrichtenden Professoren, nicht von den Nobelpreisträgern.
Dieser Unterschied in der Fähigkeit zu unterrichten zwischen den unterrichtenden und praktizierenden Wissenschaftlern verblaßt - wegen der Natur der wissenschaftlichen Disziplin (s. \hyperlink{c3_1}{Kapitel 3}) - im Vergleich zu der Kluft, die in der Welt der Kunst besteht.
Lernen und Unterrichten sind integrale Bestandteile davon, ein Wissenschaftler zu sein.
Im Gegensatz dazu waren die größten Pianisten entweder widerstrebend oder aus wirtschaftlicher Notwendigkeit zum Unterrichten gezwungen, ohne eine bedeutende Ausbildung dafür erhalten zu haben.
Deshalb gibt es viele Gründe, warum große Künstler u.U. keine guten Lehrer waren.

Leider haben wir in der Vergangenheit bei den berühmten Künstlern eine Anleitung in der Annahme gesucht, daß wenn sie es können, sie auch in der Lage sein sollten, uns zu zeigen wie es geht.
Typische historische Berichte zeigen, daß wenn man einen berühmten Pianisten fragt, wie man eine bestimmte Passage spielen muß, er sich an das Klavier setzen und sie spielen wird, weil die Sprache des Pianisten mit den Händen und dem Klavier und nicht mit dem Mund gesprochen wird.
Derselbe große Künstler hat vielleicht nur eine geringe Vorstellung davon, wie die Finger sich bewegen oder wie sie die Klaviertasten handhaben.
Um die Hände auf die richtige Art zu bewegen, muß man lernen, eine Vielzahl von Muskeln und Nerven zu kontrollieren, und dann die Hände darauf trainieren, diese Bewegungen auszuführen.
Es gibt unter den Möglichkeiten, Technik zu erwerben, zwei Extreme.
Ein Extrem ist der analytische Ansatz, bei dem jede Bewegung, jeder Muskel und jede physiologische Information analysiert wird.
Das andere Extrem ist der künstlerische Ansatz, bei dem man sich einfach ein bestimmtes musikalisches Ergebnis vorstellt, und der Körper reagiert auf verschiedene Arten, bis das gewünschte Resultat erreicht ist.
Dieser künstlerische Ansatz kann nicht nur eine schnelle Vereinfachung sein, sondern auch zu unerwarteten Ergebnissen führen, welche die ursprüngliche Idee übersteigen können.
Er hat auch den Vorteil, daß ein \enquote{Genie} ohne analytische Ausbildung Erfolg haben kann.
Der Nachteil ist, daß es keine Garantie für den Erfolg gibt.
Technik, die auf diese Art erworben wird, kann nicht analytisch gelehrt werden, außer indem man sagt, daß \enquote{man die Musik auf diese Art fühlen muß}, um sie zu spielen.
Leider ist diese Art der Anweisung für diejenigen, die noch nicht wissen, wie man etwas spielt, wenig hilfreich, außer um zu zeigen, daß es möglich ist.
Es reicht auch nicht, die Übungsmethoden zu kennen.
Man braucht die richtige Erklärung, warum sie funktionieren.
Diese Erfordernis liegt oft außerhalb der Fachkenntnisse des Künstlers oder Klavierlehrers.
Deshalb gibt es ein grundlegendes Hindernis für die richtige Entwicklung der Werkzeuge für den Klavierunterricht: Künstler und Klavierlehrer haben nicht die Ausbildung, um solche Werkzeuge zu entwickeln; auf der anderen Seite haben Wissenschaftler und Ingenieure, die über eine solche Ausbildung verfügen können, nicht genügend Erfahrung mit dem Klavier, um die Methoden für das Klavierspielen zu erforschen.

Die alten Meister waren selbstverständlich Genies und hatten sowohl eine bemerkenswerte Einsicht und Erfindungsgabe, als auch ein intuitives Gespür für Mathematik und Physik, das sie auf ihr Klavierspiel anwandten.
Deshalb ist es nicht richtig, zu schließen, daß sie keine analytische Herangehensweise an die Technik hatten; praktisch jede analytische Lösung für das Klavierüben, die wir heute kennen, wurde durch diese Genies viele Male erneut erfunden oder zumindest von ihnen benutzt.
Es ist deshalb unglaublich, daß niemand jemals daran gedacht hat, diese Ideen systematisch zu dokumentieren.
Es ist sogar noch erstaunlicher, daß anscheinend sowohl die Lehrer als auch die Schüler nicht einmal in groben Zügen erkannten, daß die Übungsmethoden der Schlüssel für den Erwerb der Technik waren.
Ein paar gute Lehrer haben immer gewußt, daß Talent eher erzeugt wird als angeboren ist (siehe \hyperlink{reference}{Quellenverzeichnis}).
Die größte Schwierigkeit scheint die Unfähigkeit des künstlerischen Ansatzes gewesen zu sein, die korrekte theoretische Basis (Erklärung) dafür zu bestimmen, warum diese Übungsmethoden funktionieren.
Ohne eine solide theoretische Erklärung oder Basis kann sogar eine korrekte Methode von verschiedenen Lehrern mißbraucht, mißverstanden, verändert oder herabgewürdigt werden, so daß sie möglicherweise nicht immer funktioniert und als unzuverlässig oder nutzlos angesehen wird.
Diese historischen Tatsachen verhinderten jegliche geordnete Entwicklung der Lehrmethoden für das Klavierspielen.
Deshalb ist das Verständnis - oder die Erklärung, warum eine Methode funktioniert - mindestens so wichtig wie die Methode selbst.
Diese Situation wurde dadurch, daß das \enquote{Talent} als Weg zum Erfolg gepriesen wurde, noch verschlimmert.
Das war eine bequeme Ausrede für erfolgreiche Pianisten, die mehr Anerkennung bekamen als sie verdienten und gleichzeitig von der Verantwortung für ihre Unfähigkeit, die \enquote{weniger talentierten} zu unterrichten, befreit wurden.
Und natürlich trug das Attribut \enquote{Talent} zu ihrem wirtschaftlichen Erfolg bei.

Außerdem neigten die Klavierlehrer dazu, insofern wenig mitteilsam zu sein, als sie ihre Vorstellungen vom Unterrichten kaum mit anderen teilten.
Nur an großen Konservatorien gab es einen bedeutenden Austausch von Ideen, so daß die Qualität des Unterrichts an den Konservatorien besser war als irgendwo sonst.
Die im vorangegangenen Abschnitt beschriebenen Probleme verhinderten jedoch sogar an diesen Organisationen jegliche wirklich systematische Entwicklung der Lehrmethoden.
Ein zusätzlicher Faktor war die Einteilung der Lernenden in Anfänger und fortgeschrittene Schüler.
Konservatorien akzeptierten im allgemeinen nur fortgeschrittene Schüler; ohne eine den Konservatorien entsprechende Ausbildung erreichten jedoch nur wenige Schüler die fortgeschrittenen Stufen, die notwendig waren, um akzeptiert zu werden.
Das verlieh dem Klavierspielenlernen den Ruf, viel schwieriger zu sein als es tatsächlich ist.
Der Engpaß, der durch den Mangel an guten Lehrmethoden erzeugt wurde, wurde in der Vergangenheit dem Mangel an \enquote{Talent} zugeschrieben.
Wenn alle diese historischen Fakten zusammengetragen werden, ist leicht zu verstehen, warum die großen Meister nicht unterrichten konnten und warum sogar hingebungsvolle Klavierlehrer nicht alle Werkzeuge hatten, die sie benötigten.

Anfangs schrieb ich dieses Buch nur als Sammlung einiger bemerkenswert effektiver Lehrwerkzeuge; es hat sich jedoch zu einem Projekt weiterentwickelt, das die historischen Schwächen, die für die meisten Schwierigkeiten beim Erwerb der Technik verantwortlich sind, direkt behandelt.
Das Schicksal hat die Zukunft des Klaviers plötzlich in ein weites, offenes, unbekanntes Land mit unbegrenzten Möglichkeiten verwandelt.
Wir kommen in eine schöne, neue, aufregende Ära, die von jedem genossen werden kann.



