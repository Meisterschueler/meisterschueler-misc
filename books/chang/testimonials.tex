% File: testimonials

\label{testimonials}

<h2><br>\underline{Leserkommentare}</h2>

Diese Leserkommentare veranschaulichen sowohl die Hoffnungen, Versuche und den Kummer als auch die Erfolge von Klavierspielern und Klavierlehrern.
Die Leserkommentare sind nicht nur eine Sammlung schmeichelhafter Lobeshymnen, sondern eine offene Diskussion darüber, was es bedeutet, das Klavierspielen zu lernen.
Die Zahl der Lehrer, die mir Kommentare gegeben haben, und ihre Berichte, daß sie durch die Benutzung dieser Art von Methoden mehr Erfolg mit ihren Schülern haben, sind eine Ermutigung für mich.
Es scheint unausweichlich, daß Lehrer, die Nachforschungen anstellen und ihre Lehrmethoden verbessern, erfolgreicher sind.
Zahlreiche Klavierspieler haben erwähnt, daß sie durch ihre vorherigen Lehrer völlig falsch unterrichtet wurden.
Viele, die ihre Lehrer mochten, merkten an, daß diese Lehrer Methoden benutzten, die denen in diesem Buch ähnlich sind.
Es gibt eine fast einmütige Übereinstimmung darüber, was richtig und was falsch ist; wenn man dem wissenschaftlichen Ansatz folgt, kommt man deshalb nicht in die Situation, bei der man sich nicht darüber einigen kann, was richtig ist.
Ich war beeindruckt, wie schnell einige Leser diese Methoden aufgenommen haben.

Die Auszüge wurden kaum bearbeitet, irrelevante Details jedoch weggelassen.
Eintragungen in [...] sind meine Kommentare.
Ich nehme die Gelegenheit wahr, jedem zu danken, der mir geschrieben hat; Sie haben mir geholfen, das Buch zu verbessern.
Ich kann nicht über die Tatsache hinweggehen, daß die Leser das Buch für mich weiterschreiben (d.h., ich könnte ihre Anmerkungen in mein Buch einfügen und sie würden perfekt passen!).
Im folgenden habe ich nicht nur die schmeichelhaften Bemerkungen ausgewählt; ich wählte Material, das bedeutend (lehrreich) erschien, egal ob es positiv oder kritisch ist.


<ol type=1>

\item \label{testimonials01}
[Von einem Pfarrer:]<br>
Dieses Buch ist die Klavierbibel.
Ich habe solch enorme Fortschritte gemacht, seit ich es gekauft habe [die erste Ausgabe].
Ich werde es weiterhin anderen empfehlen.


\item \label{testimonials02}
[Jan. 2003 erhielt ich diese E-Mail (mit freundlicher Genehmigung):]<br>
Mein Name ist Marc, und ich bin 17 Jahre alt.
Ich habe gerade vor einem Monat mit dem Klavierspielen angefangen und habe Ihr Buch \enquote{The Fundamentals of Piano Practice} gelesen. . .
Ich habe noch keinen Lehrer, aber ich bin dabei, einen zu suchen. . .
[Gefolgt von einer Reihe Fragen, die bei jemand so jungem mit so wenig Klaviererfahrung von früher Reife zeugen.
Ich beantwortete die Fragen so gut ich - damals - konnte.]

[Mai 2004 erhielt ich diese erstaunliche E-Mail:]<br>
Ich erwarte nicht, daß Sie sich an mich erinnern, aber ich sandte Ihnen vor ungefähr einem Jahr eine E-Mail. . .
Ich möchte Sie wissen lassen, wie ich mit Hilfe Ihrer Methode mit dem Klavierspielen zurechtgekommen bin.
Ich begann mit dem Klavierspielen ungefähr Weihnachten 2002 und benutzte Ihre Methode von Anfang an.
Mitte März 2003 nahm ich am Konzertwettbewerb meiner High-School teil - aus Spaß und der Erfahrung wegen, nicht in der Hoffnung, ihr 500\$-Stipendium zu gewinnen.
Ich kam - in einer Konkurrenz mit reiferen Klavierspielern, die seit bis zu 10 Jahren spielten - unerwartet auf den ersten Platz.
Es war ein Schock für die Richter, als ich ihnen sagte, daß ich seit 3 Monaten spiele.
Vor ein paar Tagen gewann ich auch den diesjährigen Wettbewerb.
Mit anderen Worten: Der Fortschritt ist sehr schnell gekommen.
Ein solcher Fortschritt ist einer der größten Motivatoren (neben der generellen Liebe zur Musik), 
so daß ich nun für den Rest meines Lebens Klavier spielen und mich darin verbessern kann.
Und obwohl ich meinen Lehrern ebenfalls Anerkennung zollen muß, ist Ihre Methode meine Grundlage, auf der sie aufbauen, und ich glaube, daß die Methode der Hauptgrund für meinen Fortschritt ist.
Trotzdem halte ich mich immer noch für einen Anfänger. . .
Meine Website (www.mtm-piano.tk) enthält alle Aufnahmen, die ich bis heute (18) gemacht habe. . .
Vor kurzem habe ich Chopins Regentropfen-Präludium, Scarlattis K.466 und Bachs F-Dur-Invention aufgenommen. . .
Meine nächste Aufnahme wird Bachs e-Moll-Sinfonie\footnote{dreistimmige Invention} sein, und ich möchte das bis Ende nächster Woche fertig haben.
Ihr Buch ist weit mehr als ein Liebhaber der Musik und des Klaviers erwarten könnte, und ich kann Ihnen nicht genug für die Hilfe danken, die Sie mir und so vielen anderen aufstrebenden Klavierspielern gegeben haben. . .
[Gehen Sie auf seine Website und hören Sie sich diese erstaunlichen Aufnahmen an!
Sie können ihn sogar auf der Music-Download-Website (music.download.com) finden; suchen Sie nach \enquote{Marc McCarthy}.]


\item \label{testimonials03}
[Von einer angesehenen, erfahrenen Klavierlehrerin:]<br>
Ich habe gerade Ihren neuen Abschnitt überflogen [über Übungen für parallele Sets] und dachte, ich sollte Ihnen meine erste Reaktion mitteilen.
Als Prinzregentin der Übungshasser habe ich mich lautstark für die Kriminalisierung von Hanon und anderen eingesetzt und hatte zunächst bestürzt angenommen, daß Sie sich den unterdrückten Massen der pseudo-voodoo-haft Übenden angeschlossen hätten - hoffnungslos, hilflos, wiederholend, wiederholend, . . .
Jedenfalls, um auf den Punkt zu kommen, sehe ich einen Punkt des Verdienstes in Ihrem Ansatz, WENN WENN WENN der Schüler Ihre GANZEN Anweisungen befolgt und die beschriebenen Schlüsselkombinationen als Diagnosewerkzeug benutzt - NICHT um jede einzelne Kombination als tägliche Routine zu wiederholen.
Als Diagnosewerkzeug und daraus folgendes Heilmittel; das ist Ihnen wunderbar gelungen!
Es war etwas vertrautes in Ihren Übungen, deshalb habe ich heute im Studio herumgesucht und fand die \enquote{Technische Studien} von Louis Plaidy, Edition Peters, die ca. 1850 das erste Mal gedruckt wurden.
Obwohl sich Plaidys Philosophie bezüglich des Gebrauchs seiner Übungen sehr von Ihrer unterscheidet, folgen die tatsächlich abgedruckten Noten buchstabengetreu (ich sollte notengetreu sagen) dem, was Sie in Ihrem Übungskapitel beschreiben.
Plaidys Übungen waren in den späten 1800er Jahren in Europa hoch angesehen und wurden während dieser Zeit am Konservatorium in Leipzig benutzt.
Plaidy selbst war ein ziemlich vielgefragter Lehrer, von dessen Protegés einige in Liszts innerem Kreis akzeptiert wurden und/oder einigen Erfolg auf den Konzertbühnen hatten.
Sie sind in großartiger Gesellschaft!


\item \label{testimonials04}
Ich möchte gerne wissen, ob Sie die Arbeit von Guy Maier kennen.
Geht sein Ansatz des \enquote{Impuls}-Übens von 5-fingrigen Mustern in die gleiche Richtung wie die von Ihnen besprochenen \enquote{parallelen Sets}?
Maier benutzt das Prinzip, eine Note mit jedem Finger zu wiederholen, während die anderen ruhig auf der Tastenoberfläche gehalten werden, als eine der 5-fingrigen Übungen.
\textit{Thinking Fingers} war eines der Übungsbücher, das Maier mit Herbert Bradshaw Anfang der 1940er Jahre schrieb.
Eine seiner ersten 5-fingrigen Übungen, die widerzuspiegeln scheint, was Sie über \enquote{Quadrupel}-Wiederholungen auf einer Note gesagt haben, geht folgendermaßen:<br>
a. Einzelne Finger mit wiederholten 1, 2, 3, 4, 8, und 16 Noten-Anschlägen.<br>
b. Üben Sie jeden Finger einzeln, drücken Sie die anderen Tasten leicht herunter oder halten Sie die Finger still in der Position auf den Tasten.<br>
c. Benutzen Sie CDEFG mit der rechten Hand, setzen Sie 5 Finger auf diese Noten eine Oktave oberhalb des mittleren C, mit dem Daumen der rechten Hand auf dem C.<br>
d. Genauso mit der linken Hand, eine Oktave unter dem mittleren C, mit dem fünften Finger auf dem C.<br>
e. Üben Sie die Hände getrennt; beginnen Sie mit dem Daumen der rechten Hand, spielen Sie einen Anschlag C, lassen Sie los, dann zwei Anschläge usw. bis zu 16.
Wiederholen Sie das mit jedem Finger, benutzen Sie dann die linke Hand.<br>
[Sehen Sie dazu meine Übungen in \hyperref[c1iii7b]{\autoref{c1iii7b}}; es ist erstaunlich, daß wir unabhängig voneinander für diese Übung zu Gruppen von \enquote{Quadrupeln} (vier Wiederholungen) gekommen sind und bis zu 4 Quadrupeln (16 Wiederholungen), was fast mit meiner Übung \#1 identisch ist.]<br>
f. Anfänger werden die Anschläge langsam ausführen und sich an die volle Geschwindigkeit heranarbeiten müssen (und hier kommen glaube ich Ihre \enquote{Quadrupel} ins Spiel - so viele Wiederholungen je Sekunde sind das Ziel).<br> Maier bezeichnet 16 als seine Grenze.
Er gibt eine Vielzahl von Mustern dafür, diese Vorgehensweise bei 5-fingrigen Anschlagsübungen zu verwenden, in Buch 1 und 2 von \textit{Thinking Fingers}, 1948 herausgegeben von Belwin Mills Inc., NY.
Ich glaube, Maier war bemüht, den Schülern dabei zu helfen, die nötige Gewandtheit ohne die endlosen Wiederholungen von Hanon, Pischna und anderen zu bekommen.


\item \label{testimonials05}
Bitte senden Sie mir Ihr Buch - ich bin seit mehr als 50 Jahren Klavierlehrer und immer noch bereit dazuzulernen.


\item \label{testimonials06}
[Diese Zuschrift öffnet einem die Augen: Sie zeigt uns eines der am meisten falsch diagnostizierten Probleme, das uns daran hindert, schnell zu spielen.]<br>
In jungen Jahren begann ich mit dem Klavier und hörte dann wieder auf.
Als Teenager ging ich dann auf ein [berühmtes] Konservatorium und versuchte jahrelang, Technik zu erwerben, aber versagte kläglich und machte am Ende Karriere als Ingenieur.
Jahre später kehrte ich zum Klavier (Clavinova) zurück und versuche nun, das zu tun, was mir Jahre zuvor nicht gelang.
Einer der Gründe, warum ich mit dem Üben aufhörte, ist, daß meine Frau und mein Sohn ärgerlich wurden, wenn sie mich Passagen ständig wiederholen hörten; das Clavinova erlaubt mir, ohne Schuldgefühle zu jeder Stunde zu üben\footnote{mit Kopfhörern!}.
Ich las Ihre Website und war fasziniert.
Ich wünschte, ich hätte einige Jahre zuvor an einige Ihrer Ideen gedacht.
Ich habe eine Frage und kann scheinbar keine sinnvolle Antwort darauf finden, und doch ist es eine solch grundlegende Frage.
Man hat mir beigebracht, daß man, wenn man Klavier spielt, das Gewicht des Arms auf jeden Finger der spielt überträgt.
Schwerkraft.
Man drückt niemals nach unten, man muß entspannt sein.
Deshalb fragte ich meine Lehrer, wie man pianissimo spielt.
Die Antwort war, daß man näher an den Tasten spielt.
Das funktioniert bei mir nicht.
[Lange Diskussion verschiedener Methoden zu versuchen, pianissimo mit dem Armgewicht zu spielen und warum sie nicht funktionieren.
Anscheinend kann er nur pianissimo spielen, wenn er bewußt die Hände von den Tasten hebt.
Auch ist, da alles oftmals als forte herauskommt, die Geschwindigkeit ein Problem.]
Würden Sie mir diese Frage bitte beantworten?
Was macht man mit dem Armgewicht, wenn man pianissimo spielt?
Ich habe viele Bücher über das Klavierspielen gelesen und mit vielen vollendeten Pianisten gesprochen.
Es ist eine Sache, zu wissen wie man alles spielt, und es ist eine ziemlich andere Sache, in der Lage zu sein, jemand anderem beizubringen, wie man spielt.
[Ich hätte das nicht besser sagen können!]
Ihre Schriften sind brillant und auf vielfältige Weise revolutionär, ich wußte instinktiv, daß wenn irgend jemand mir helfen könnte, dann Sie.<br>
[Nach solch einem Kompliment mußte ich etwas tun, weshalb ich den Bericht seiner Schwierigkeiten sorgfältig las und zu dem Schluß kam, daß er nach so vielen Jahren der Versuche unwissentlich auf das Klavier herunterdrücken mußte, fast so als wenn er hypnotisiert wäre.
Ich sagte ihm, er solle einen Weg finden, um festzustellen, ob er tatsächlich herunterdrückte - keine leichte Aufgabe.
Dann kam diese Antwort:]<br>
Danke für Ihre Antwort.
Die Wahrheit wird am besten anhand von Extremen untersucht.
Ihr Vorschlag brachte mich auf die Idee, daß ich vielleicht IMMER so spielen sollte, wie ich MEIN pianissimo spiele - indem ich meine Hände von den Tasten hebe.
Ich eilte zu meinem Hanon, und JA!
Ich kann viel schneller spielen!
Ich eilte schnell zum Bach-Präludium II, das ich niemals mit der richtigen Geschwindigkeit (144) spielen konnte, und ich hatte immer Schwierigkeiten, die Finger gleichzeitig landen zu lassen, wenn ich schnell spielte, und bei Geschwindigkeiten über 120 landeten die Finger zusammen, wie eine einzige Note.
Kein Herumfingern, keine Anstrengung.
Nicht nur das, ich kann piano oder forte so schnell spielen wie ich möchte.
Es fühlt sich so unglaublich LEICHT an!
Ich habe es gerade entdeckt!
Ich kann es nicht glauben.
[Lange Diskussion darüber, wie er über die Jahre dazu kam, Armgewicht mit herunterdrücken gleichzusetzen, was hauptsächlich durch die Angst, die Lehrerin nicht zu verstehen, verursacht wurde, die eine strikte Verfechterin des Armgewichts war.
Das ist tatsächlich etwas, was mir sehr verdächtig bei der Armgewichtsmethode vorkam: daß ein so starkes Betonen des Armgewichts und übermäßig strikte Disziplin eine Art Neurose oder Mißverstehen verursachen könnten - vielleicht sogar eine Art Hypnose.]
Eine große Mauer ist gerade eingestürzt und nun, nach so vielen Jahren des Nachdenkens und Stunden des Übens (ich übte am Konservatorium bis zu 10 Stunden täglich, und immer noch lernte ich nur die Musik auswendig, ohne jemals meine Technik zu verbessern), kann ich dahinter sehen.
Ich entdeckte, daß ich die Fähigkeit habe, schneller zu spielen als ich je geträumt hätte (habe gerade die C-Dur-Tonleiter versucht und war schockiert, daß ich das war, der spielte) und mit dem vollen Dynamikbereich, den ich möchte, OHNE ANSPANNUNG.
[Eine lange Beschreibung all der neuen Dinge, die er nun kann, und ein Vergleich mit seinen vorherigen Jahren des Kämpfens und der Kritik von anderen.]
Ich muß Ihnen dafür danken.
Ihr Buch war das einzige, das ich jemals gelesen habe, das genug Variationen von der Hauptlinie bot, um meinen Geist schließlich von einem großen Irrtum zu befreien.
Ich drückte herunter, ließ nicht los.
Meine Arme wiegen einfach keine Tonne, sondern sie sind frei.
Weil ich Angst vor meiner Lehrerin hatte und einzig auf das Gewicht meiner Arme achtete, drückte ich unbewußt nach unten.
Ich wagte nie, PPP für sie zu spielen.
Ich wußte wie, aber ich war sicher, daß es die falsche Technik war.
[Ich fürchte, daß das jungen Menschen häufig passiert; sie verstehen den Lehrer nicht aber fürchten sich davor zu fragen und nehmen am Ende das Falsche an.]
Was sie mir hätte sagen sollen, war: \enquote{DRÜCKE NIEMALS HERUNTER!}; statt dessen fixierte ich mich auf das Gewicht meines Arms als den Schlüssel für alles.
[Ein junger Mensch muß herunterdrücken, um \enquote{Gewicht} auf seine Arme zu geben!
Wie erklärt man einem Kind, das keinen Physikunterricht hatte, daß das falsch ist?]
Sie erlaubte mir nie, schnell zu spielen.
[Das ist ein weiterer Kommentar, den ich von Schülern strikter Armgewichtslehrer gehört habe - Geschwindigkeit kommt nicht in Frage, bis bestimmte Meilensteine erreicht sind; obwohl wir vorsichtig sein müssen, wenn wir für die Geschwindigkeit üben, ist langsamer zu spielen nicht der schnellste Weg zur Geschwindigkeit.]
Weil ich verspannt war, und sie sagte, ich würde niemals schnell spielen, wenn ich verspannt wäre.
In Ihrem Buch sagen Sie, daß wir schnell spielen müssen, um die Technik zu entdecken.
Es wurde mir nie erlaubt!
Ihr Buch und Ihre E-Mail befreiten mich von den Ketten in meinem Geist, die mich all die Jahre gefangen gehalten hatten.
Ich danke Ihnen vielmals.
Ich kann nicht beschreiben, wie dankbar ich Ihnen und Ihren Einblicken bin.

[Obwohl meine obigen Kommentare gegen die Armgewichtsschule gerichtet scheinen, ist das nicht der Fall - ähnliche Schwierigkeiten gelten für jede Lehre, die auf ungenügendem Wissen basiert und sich in den Händen eines strengen disziplinarischen Lehrers befindet.
Unglücklicherweise hat in der Vergangenheit eine große Zahl Klavierlehrer aufgrund eines Mangels an theoretischem Verständnis und rationaler Erklärungen inflexible Lehrmethoden übernommen.
Eine systematische Behandlung der Geschwindigkeit finden Sie in \hyperref[c1ii13]{\autoref{c1ii13}} und besonders in Abschnitt III.7i.]


\item \label{testimonials07}
Ich fand Ihr Buch im Internet und schätze mich sehr glücklich.
Danke vielmals, daß Sie eine solch große Anstrengung unternommen haben, die Klaviertechnik und sinnvolle Übungsgewohnheiten zu beschreiben.
Ich bin Klavierlehrer.
Ich habe erst angefangen, Ihr Buch zu lesen, und habe bereits einige Übungstechniken bei meinen Schülern angewandt.
Sie mochten es und ich mag es ebenfalls.
Das Üben wird dadurch viel interessanter.
Kennen Sie das Buch \enquote{The Amateur Pianist's Companion} von James Ching, herausgegeben 1956 von Keith Prowse Music Publishing Co. in London?
Dieses Buch wird vielleicht nicht mehr gedruckt, aber ich habe es gebraucht gefunden.
Es könnte Sie interessieren, denn \enquote{die Details der korrekten Haltungen, Bewegungen und Bedingungen, wie sie in diesem Buch skizziert werden, sind das Ergebnis von ausgedehnten Forschungen auf dem Gebiet der physiologischen Mechanik der Klaviertechnik, die vom Autor in Zusammenarbeit mit Professor H. Hartridge, Professor der Physiologie, und H. T. Jessop, Dozent in Mechanik und Angewandter Mathematik an der Universität von London, durchgeführt wurden}.


\item \label{testimonials08}
Ich bin so dankbar, daß ich Ihre Website gefunden habe.
Ich bin ein erwachsener Klavierspieler, der als junger Mensch auf die falsche Art unterrichtet wurde.
Ich versuche immer noch, meine schlechten Techniken und Angewohnheiten abzutrainieren.
Ich nehme nun Stunden bei einem sehr guten Lehrer.


\item \label{testimonials09}
Vor ein paar Wochen habe ich mir Ihr Buch aus dem Internet heruntergeladen und es ausprobiert.
Ich bin ungefähr zur Hälfte damit durch und weit davon entfernt, alles völlig anzuwenden, aber ich bin bis jetzt so erfreut über die Resultate, daß ich dachte, ich gebe Ihnen ein paar spontane Rückmeldungen.<br>
Zunächst ein paar Hintergrundinformationen.
Ich lernte Klavier bis zu einer fortgeschrittenen Stufe und begann ein Musikstudium, das ich nach einem Jahr abbrach, um Mathematik zu studieren.
Nach dem Diplom war ich ein enthusiastischer Amateur, aber während der letzten 20 Jahre spielte ich immer weniger, hauptsächlich wegen meiner Frustration über den Mangel an Fortschritt und weil ich überzeugt war, daß ich nie soviel Zeit für das Üben haben würde, wie nötig wäre, um besser zu spielen.<br>
Ich suchte nach ein paar Tips für den Kauf eines Klaviers und kam zu Ihrer Website.
Nachdem ich ein paar Kapitel gelesen hatte, lud ich mir alles herunter und fing an es auszuprobieren.
Das ist nicht das erste Mal, daß ich versucht habe, mich mit einem Buch oder dem Rat eines Lehrers zu verbessern.
Hier sind meine Erfahrungen nach drei Wochen.
[Beachten Sie, wie schnell man diese Methoden lernen und sie nutzen kann.]<br>
Ich habe mich darauf konzentriert, 4 Stücke zu lernen, die ich sehr gern mag:<br>
- Ravels Prélude<br>
- Chopins Prélude Nr. 26 in As-Dur<br>
- Poulencs Novelette Nr. 1<br>
- Ravels Alborada del Graziosa aus Miroirs<br>
Die Ravel-Prélude ist ein kleines Stück mit keiner offensichtlichen technischen Schwierigkeit.
Das ist ein Stück, das ich immer vom Blatt gespielt habe, aber nie richtig gut.
Es gibt in der Mitte einen Abschnitt mit gekreuzten Händen mit einer exquisiten Dissonanz, die einige Schwierigkeiten bereitet, aber das war es dann.
Ich wandte die Übungsmethoden in diesem Buch auf das Stück an, und es wurde plötzlich weitaus nuancierter lebendig als ich ihm jemals zugetraut hätte.
Es ist alles andere als das beiläufig spielbare Stück, für das ich es immer gehalten hatte, aber ohne die richtigen Übungsmethoden wird es immer als das erscheinen.<br>
Die Poulenc-Novelette ist eines der Stücke, die ich 20 Jahre lang mindestens einmal die Woche gespielt habe und die ich sehr mag.
Ich habe es nie wirklich ganz zu meiner Zufriedenheit gespielt, aber ich nahm immer an, daß es wegen des Mangels an Übungszeit war.
Ich begann unter Benutzung Ihrer Vorschläge zu analysieren was falsch war.
Neben einigen offensichtlichen Fehlern, die ich nie wirklich korrekt gelernt hatte, war das überraschendste Ergebnis, daß es für mich unmöglich war, mit dem Metronom im Takt zu bleiben!
Eine detailliertere Analyse offenbarte die Ursache - ein großer Teil von Poulencs Werk erfordert schnelle und merkwürdige Verschiebungen der Handposition bei Melodien, die über diese Verschiebungen hinweg durchgehalten werden müssen.
Die schlechte Angewohnheit, die ich gelernt hatte, war, während dieser Verschiebungen nach den Tasten zu \enquote{greifen} und so die Melodielinie zu zerstören und das Stück schrittweise zu beschleunigen.
Die Offenbarung für mich war, daß das Problem nicht durch Üben mit dem Metronom behoben werden konnte!
Es konnte nur durch die Analyse des Problems und die Ausarbeitung einer Strategie zur Behandlung der Verschiebungen behoben werden.
Nun bin ich sehr zufrieden mit der Art wie ich spiele und habe sogar viel Zeit übrig, um über die Musik nachzudenken.<br>
Alborada del Graziosa ist ein Fall für sich.
Das ist ein höllisch schwieriges Stück, das ich in der Vergangenheit versucht habe zu lernen, aber bei dem ich nicht in der Lage war, die meisten Passagen auf die korrekte Geschwindigkeit zu bringen.
Meine Annahme war immer, daß mehr Üben nötig war, und daß ich nie die Zeit finden konnte.
Wieder habe ich die Methoden in Ihrem Buch angewandt, um das zu lernen, und nach drei Wochen bin ich zwar noch nicht am Ziel, aber ich kann nun das meiste davon mit der richtigen Geschwindigkeit spielen und auch ziemlich musikalisch.
Ich schätze, daß ich es innerhalb von ein paar Wochen in den Fingern habe und mich auf die Musik konzentrieren kann.<br>
Zu guter Letzt die Chopin-Prélude.
Ich hatte diese für eine Prüfung gelernt als ich 16 Jahre alt war aber seit damals nie wirklich gespielt.
Ich begann, sie erneut zu lernen und machte ein paar Entdeckungen.
Als erstes hatte ich sie nie mit der endgültigen Geschwindigkeit gespielt, sogar bei der Prüfung, das war also etwas, was ich beheben mußte.
Das funktionierte jedoch nicht - ich entdeckte, daß ich aus zwei Gründen nicht schneller werden konnte.
Erstens hatte ich gelernt, das Legato mit dem Pedal nachzuahmen - aber wenn man schneller wird, erhält man nur ein Durcheinander von Tönen, und wenn ich versuchte, das Pedal richtig zu benutzen, konnte ich das Legato nicht hinbekommen.
Zweitens enthält der mittlere Abschnitt einige weit gestreckte, gebrochene Akkorde in der linken Hand, die auf jedem Schlag verschoben werden.
Langsam gespielt ist das OK aber bei der richtigen Geschwindigkeit wird das höllisch schwierig und sogar schmerzhaft zu spielen.
Im Grunde mußte ich das Stück erneut lernen - neuer Fingersatz, neue Handpositionen, anderes Pedalieren usw.
Nun kann ich das mit jeder Geschwindigkeit, die ich mag, ohne Streß spielen.
Ich fand das einen interessanten Beweis für das, was Sie im Buch sagen - das ist ein sehr kleines Stück, das ziemlich einfach erscheint aber bei der richtigen Geschwindigkeit seinen Charakter völlig ändert und jeden Schüler frustrieren wird, der die intuitive Methode benutzt, es sei denn er ist mit einer Spanne von mehr als 1,5 Oktaven gesegnet.<br>
Abschließend möchte ich Ihnen für das Schreiben dieses Buchs danken und noch mehr dafür, daß Sie es über das Internet verfügbar gemacht haben.
Ich habe in der Vergangenheit enorme Summen Geld für sehr angesehene Lehrer ausgegeben und nicht einer davon, obwohl ich nicht bezweifle, daß sie diese Techniken selbst beherrschen, konnte mich lehren, wie man übt.


\item \label{testimonials10}
Ich denke, Ihr Buch ist es wert, daß ich es lese, obwohl ich viele der Regeln (wie z.B. mit getrennten Händen üben, Akkord-Anschlag . . . ) von meinen Lehrern gelernt habe.
Sogar wenn nur eine Regel, die ich aus Ihrem Buch gelernt habe, funktioniert, dann ist das meines Erachtens weitaus mehr wert als die 15\$, die ich für die erste Ausgabe bezahlt habe.
Ich mag auch den Abschnitt über die Vorbereitung auf Konzerte.
Ich stimme zu, daß vor dem Konzert mit voller Geschwindigkeit zu spielen verboten ist.
Ich habe das mit meinem Lehrer besprochen, und wir sehen verschiedene Gründe warum [ausgedehnte Diskussionen darüber, warum am Tag des Konzerts mit voller Geschwindigkeit zu spielen, zu Problemen führen kann; hier nicht angegeben, weil ich sie nicht verstehen kann].
Deshalb ist vor dem Konzert schnell zu üben eine Situation, in der man nichts gewinnen kann.
Ich würde gerne mehr darüber sehen, wie man auf Geschwindigkeit kommt und wie man die Hände effizienter zusammenbringt.
Manche Musik (ich denke an Bachs Inventionen) ist mit getrennten Händen leicht zu spielen aber schwierig mit beiden Händen zusammen.
Alles in allem macht es mir Spaß, Ihr Buch zu lesen.


\item \label{testimonials11}
Ich empfehle jedem, das Üben mit getrennten Händen zu versuchen, wie es in Ihrem Buch erklärt wird.
Während ich bei Robert Palmieri an der Kent State University studierte, ließ er es mich als Teil meiner Übungen tun.
Es half mir, über das Amateurstadium hinauszukommen und zu einer viel besseren Technik und musikalischem Spielen.


\item \label{testimonials12}
Auf der Grundlage dessen, was ich Ihrer Website entnehmen konnte, wandte ich eines der Prinzipien - das Spielen mit getrennten Händen bei voller Geschwindigkeit - auf eine Reihe schwieriger Passagen in zwei völlig unterschiedlichen Stücken an, die ich spielte, eines ein Kirchenlied, das andere ein Jazz-Stück.
Interessanterweise fand ich, als ich gestern in der Kirche war und es Zeit wurde, die Gemeinde zu begleiten, daß die schwierigen Teile, die ich mit der Methode der getrennten Hände gelernt hatte, unter den festesten und sichersten des ganzen Lieds waren.
Es schien, daß jedesmal, wenn ich zu einem dieser schwierigen Punkte kam, ein geistiger Schalter anging, so daß mein Gehirn und Nervensystem diese Teile mit besonderer Sorgfalt und Genauigkeit ausführten.
Dasselbe gilt für den schwierigen Punkt in dem Jazz-Stück, der nun überhaupt kein Problem mehr ist.


\item \label{testimonials13}
Ungefähr vor eineinhalb Jahren kaufte ich das Buch \enquote{Fundamentals of Piano Practice} von Ihnen.
Ich wollte Ihnen einfach persönlich für Ihren Beitrag danken.
Es hat mir ziemlich viel geholfen!
Vor Ihrem Buch wußte ich nie, wie man üben soll, weil es mir nie beigebracht wurde.
Ich nahm durchaus Unterricht, aber meine Lehrer hatten mich nie gelehrt wie man übt.
Ist das nicht erstaunlich!
Ich habe den Verdacht, daß das alltäglich ist.
Der nützlichste Rat für mich ist Ihr Vorschlag, beim letzten Durchgang des Stücks, das man übt, mit viel geringerer Geschwindigkeit zu spielen.
Ich muß zugeben, daß es am schwierigsten für mich war, diese Angewohnheit zu entwickeln.
Aber ich versuche es.
Ich finde, daß langsames Üben eine große Hilfe ist.
Nur jeweils einen Takt oder zwei zu üben war auch wertvoll!
Ich wünschte, daß es mir leichter fallen würde, Noten auswendig zu lernen; wenn Sie irgendwelche neuen Ideen zum Auswendiglernen haben, lassen Sie es mich bitte wissen.
[Ich habe seit dieser Korrespondenz beträchtliches Material über das Auswendiglernen hinzugefügt.]


\item \label{testimonials14}
Danke, daß Sie meine Fragen zum Klavierüben beantwortet haben.
Ich muß Ihnen sagen, daß es eine besonders verzwickte Prélude von Chopin gibt - die in Cis-Moll.
Als ich Ihr Buch erhielt, bewältigte ich diese Prélude innerhalb eines Tages und schneller als mit ihrer hohen Geschwindigkeit.
Zugegeben, es ist eine kurze, aber viele Klavierspieler ringen damit.
Diese Erfahrung war sehr ermutigend.


\item \label{testimonials15}
Ich spiele nun seit 8 Jahren Klavier und habe Ihr Buch vor ungefähr einem Jahr gekauft.
Nachdem ich dieses Buch gelesen habe, sind meine täglichen einstündigen Übungssitzungen viel produktiver.
Ich lerne auch neue Stücke viel schneller.
Sie zeigen Einblicke in folgendes:<br>
Korrekte Übungsmethoden.<br>
Wie man ein neues Stück anfängt.<br>
Langsames Üben (wann man es benutzt und warum).<br>
Wann man schneller spielen soll als normal.<br>
Wie man sich auf ein Konzert vorbereitet.<br>
Ich stimme nicht allem zu was Sie schreiben, aber ich lese Ihr Buch ungefähr alle zwei Monate, damit ich die richtige Art zu üben nicht aus den Augen verliere.
[Das wird oft gesagt: Mein Buch ist eine solch verdichtete Zusammenfassung, daß man es mehrere Male lesen muß.]


\item \label{testimonials16}
Nach einer Woche war ich sehr zufrieden mit mir und der Methode, da ich dachte, daß ich eine ganze Seite mit HS erfolgreich AUSWENDIGGELERNT!!! hätte.
Das war eine absolut unbekannte Leistung, soweit es mich betraf.
Probleme kamen aber auf, als ich versuchte, die beiden Hände zusammenzubringen, was ich dann in der Zeit versuchte, in der ich den Rest des Stücks lernte.
Als ich versuchte, den Rest des Stücks zu lernen, fand ich auch heraus, daß ich die erste Seite falsch \enquote{auswendiggelernt} hatte, und ich machte schließlich Anmerkungen für mich selbst.
[Das geschieht wahrscheinlich häufiger als die meisten von uns zugeben - wenn Sie Schwierigkeiten haben, HT zur endgültigen Geschwindigkeit zu kommen - PRÜFEN SIE DIE NOTEN! - könnte die Ursache ein Fehler beim Notenlesen sein.
Fehler beim Rhythmus sind besonders schwer zu entdecken.]
Ihr Buch HAT mir genau das gegeben, wonach ich gesucht habe - d.h. eine Grundlage um herauszuarbeiten, wie man schneller und effizienter lernt.
Kein Lehrer war jemals in der Lage, mir einen Anhaltspunkt dafür zu geben, wie man an das Lernen eines Stücks herangeht.
Der einzige Vorschlag, den ich jemals bekam, war \enquote{Schau es Dir an, und sieh, was Du daraus machen kannst.}, und dafür, wie man die Genauigkeit und/oder die Geschwindigkeit verbessert \enquote{Üben, Üben, . . .} WAS?????
Ich habe nun die Antworten auf diese entscheidenden Fragen erhalten. Danke.


\item \label{testimonials17}
Ich habe Ihr Buch auf Ihrer Website gelesen und habe viel für mich herausgeholt.
Sie haben mich dazu inspiriert, auf die Art zu üben, von der ich immer gewußt habe, daß es die beste Art ist aber niemals die Geduld dazu hatte.
Was Sie über gleichmäßige Akkorde vor dem Versuch schnell zu spielen schreiben, hat mir sehr geholfen.
Ich glaube, daß meine Unfähigkeit, über eine bestimmte Geschwindigkeit hinaus zu spielen, von einer grundlegenden Ungleichmäßigkeit in meinen Fingern kommt, um die ich mich nie wirklich gekümmert habe.
Ich sagte immer nur: \enquote{Ich kann nicht gut schnell spielen.}
Ich habe ein kleines Stück einer Etüde mit dem Akkord-Anschlag aufgearbeitet und kann es tatsächlich ziemlich flüssig und gleichmäßig spielen!
Ich bin neugierig auf Ihre Theorien über die Entwicklung eines absoluten Gehörs.
Die Lager scheinen in bezug auf dieses Thema sehr weit auseinander zu sein: Genetik und Umwelt.
[Seit dieser Korrespondenz habe ich die Übungen für parallele Sets zum Üben der Akkorde hinzugefügt und habe einen ausgedehnten Abschnitt über das Aneignen eines absoluten Gehörs geschrieben.]


\item \label{testimonials18}
Ich wollte Sie einfach wissen lassen, wie gut meiner Familie von Musikern Ihr Buch über das Klavierspielen gefallen hat.
Ohne Zweifel haben Sie in Ihrem Buch einige innovative, unorthodoxe Ideen vorgetragen, die trotz der Tatsache, daß sie gemessen an den Standards der meisten Klavierlehrer extrem klingen, wirklich funktionieren.
[Ich stimme zu!]
Die Methode, die Hände getrennt zu üben, scheint genauso gut zu funktionieren wie die Methode nicht alles soooooo langsam zu spielen!
Auch hat es sich als nützlich erwiesen, nicht so viel Betonung auf das Metronom zu legen.
Mit Sicherheit haben Ihre Methoden geholfen, den ganzen Lernprozeß für neue Stücke zu beschleunigen, und ich kann mir nun nicht vorstellen, wie wir jemals ohne das Wissen um Ihre \enquote{musikalischen Wahrheiten} zurechtgekommen sind.
Danke nochmals, daß Sie ein solch wunderbares JUWEL von einem Buch geschrieben haben!


\item \label{testimonials19}
Ich habe die Online-Kapitel gelesen und bin der Meinung, daß von jedem Klavierlehrer verlangt werden müßte, dieses Buch gelesen zu haben.
Ich bin einer der Unglücklichen, der 7 Jahre damit verbracht hat, ohne die geringste Ahnung von Entspannung oder effizienten Übungsmethoden Tonleitern und Hanon zu üben.
Ich fing damit an, gute Übungshinweise aus Diskussionsforen im Internet und verschiedenen Büchern zu sammeln, aber Ihr Buch ist die bei weitem umfassendste und überzeugendste Quelle, die ich bisher gefunden habe.


\item \label{testimonials20}
Ich bin ein Klavierspieler der Mittelstufe.
Vor einem Monat habe ich Teile Ihres Buchs heruntergeladen, und ich muß in einem Wort sagen, daß es fabelhaft ist!
Als Wissenschaftler schätze ich die strukturierte Art, in der das Material des Themas präsentiert und auf einer grundlegenden Stufe erklärt wird.
Es hat meine Betrachtungsweise des Klavierübens verändert.
Besonders der Teil über das Auswendiglernen hat mir bereits geholfen, den Aufwand für das Auswendiglernen erheblich zu verringern.
Mein Privatlehrer (ein auftretender Solist) benutzt den einen oder anderen Teil Ihrer Methode.
Der Lehrer ist jedoch ein Czerny-Anhänger und hat noch nie vom Daumenübersatz gehört.
Sie müssen dem Daumenübersatz mehr Beachtung schenken, besonders dem flüssigen Aneinanderfügen von parallelen Sets.
Ich habe das Buch an meinen Lehrer weitergegeben und empfehle es jedem.<br>
[Ein Jahr später:]<br>
Ich habe Ihnen vor über einem Jahr einmal wegen Ihres phantastischen Buchs im Internet geschrieben.
Die Methoden funktionieren wirklich.
Durch das Benutzen Ihrer Methoden war ich in der Lage, einige Stücke viel schneller zu lernen und zu meistern.
Ihre Methoden funktionieren wirklich bei Stücken, die bekanntermaßen schwierig auswendig zu lernen sind, wie einige Mozart-Sonaten und Stücken, von denen mein Klavierlehrer sagte, daß sie schwierig auswendig zu lernen sind, wie die Bach-Inventionen oder einige Préludes von Chopin.
Total einfach, wenn man Ihre Methode benutzt.
Ich nehme nun die Fantaisie Impromptu in Angriff, und dieses scheinbar unmögliche Stück liegt offensichtlich innerhalb meiner Reichweite!
Ich mag auch Ihren Beitrag über das Unterbewußtsein.
Ich frage mich, ob Sie das Buch \enquote{The Mindbody Prescription} von J. D. Sarno kennen.
Dieses Buch behandelt das Unterbewußtsein genau wie Sie es tun.
Als ich an meiner Doktorarbeit arbeitete, löste ich meine scheinbar unlösbaren theoretischen Rätsel genau wie Sie es getan haben.
Ich fütterte mein Gehirn damit, und ein paar Tage später platzte die Lösung einfach heraus.
Was Sie schreiben ist also absolut richtig!


\item \label{testimonials21}
Ihre Vorschläge, wie man Musik auswendig lernt, indem man Assoziationen erzeugt (z.B. eine Geschichte), klangen für mich unklug.
Aber als ich übte, konnte ich nicht anders als zu fragen, was ich mit einer bestimmten musikalischen Phrase assoziieren könnte, die einen problematischen F-Akkord hatte.
\enquote{Gib Dir selbst ein F für falsch gespielt.} kam mir in den Sinn.
Ich dachte, das wäre kein sehr ermutigender Gedanke!
Aber jedesmal, wenn ich nun zu dieser Phrase komme, erinnere ich mich an das F.
Ich hab's. Meine Herren! Danke. Ihr Buch ist sehr nützlich.
Es spiegelt die Vorschläge meiner Lehrerin wider aber mit mehr Details.
Wenn ich nicht Klavierspielen kann, macht nichts mehr Spaß als etwas über das Klavierspielen zu lesen . . .
In den letzten Wochen vor meinem letzten Konzert schlug meine Lehrerin vor, während des Übens durch meine Fehler hindurchzuspielen.
Dann zurückzugehen und an den problematischen Takten zu arbeiten, größtenteils so wie Sie vorschlagen, obwohl es das einzige Mal war, daß das vorkam.
Sie sagt, daß die meisten Menschen den Fehler nicht einmal bemerken, solange er die Musik nicht unterbricht.
Ihr Punkt ist, die Musik nicht zu unterbrechen und das Problem an der Quelle zu beheben, indem man zu dem Takt zurückgeht.
Ich finde, daß ich mich sehr oft korrigiere (stottere); ich werde mich darauf konzentrieren, es nicht zu tun.
Dieser Rat ist nicht intuitiv, wie Sie wissen.
Man korrigiert Fehler wie von selbst, wenn sie auftreten.
Aber ich sehe ein, daß dies ständig zu tun in Wahrheit die Fehler verfestigt.


\item \label{testimonials22}
Ich stolperte über Ihr Online-Buch über das Klavierüben, als ich nach Artikeln über das absolute Gehör suchte.
Als ich es las, war ich von dem verwendeten wissenschaftlichen Vorgehen beeindruckt.
Besonders das Konzept der \enquote{Geschwindigkeitsbarrieren} und wie man sie überwindet half mir sehr.
Ich fand Ihr Buch gerade zur richtigen Zeit.
Viele Probleme, denen ich beim Klavierspielen begegne, werden in Ihrem Buch besprochen.
Viele Klavierlehrer haben anscheinend kein klares wissenschaftliches Konzept dafür, wie man bestimmte Probleme von Klavierspielern der Mittelstufe behandelt.
Deshalb arbeite ich mich Abschnitt für Abschnitt mit gutem Erfolg durch das Buch.
Es gibt verschiedene Dinge, die ich in Ihrem Buch vermisse.
In manchen Kapiteln wären Bilder hilfreich, wie z.B. korrekte Handpositionen, Daumenübersatz, Übungen für parallele Sets.
Etwas wie eine chronologische Tabelle für den Übungsablauf könnte nützlich sein.
\enquote{Kalt üben} wäre z.B. an der ersten Position.
Sie weisen immer auf die Wichtigkeit hin, WANN man WAS tun soll.
Könnten Sie die Übungen, die Sie erklären, in einer Weise ordnen, die sie am effizientesten macht?
Auf alle Fälle möchte ich meine tiefe Dankbarkeit für Ihr Projekt aussprechen!


\item \label{testimonials23}
Den ganzen Winter hindurch habe ich selbst weiter Klavier gelernt und ich muß sagen, daß jedes Wort in Ihrem Buch wahr ist.
Ich habe das Klavierspielen mehrere Jahre gelernt und nur einen durchschnittlichen Fortschritt gemacht.
Weil ich das Klavier und romantische Musik liebe, macht mich das manchmal verrückt und zutiefst frustriert.
Ich wende Ihre Methoden seit ungefähr einem Jahr an und machte enorme Fortschritte.
Ich arbeite nun an mehreren Stücken gleichzeitig, Kompositionen von denen ich nie zuvor gedacht hatte, daß ich sie spielen könnte.
Es ist wunderbar.
Heute habe ich ein kleines Repertoire, daß ich mit großer Befriedigung spielen kann.


\item \label{testimonials24}
Ich habe Ihr Buch der ersten Ausgabe bestellt und erhalten und habe Teile Ihrer zweiten Ausgabe gelesen.
Ich fand Ihre Information extrem wertvoll.
Ich sende Ihnen diese E-Mail in der Hoffnung, ein paar Ratschläge für mein kommendes Konzert zu bekommen.
Ich bin extrem nervös, aber nachdem ich Ihre Abschnitte über Konzerte gelesen habe, verstehe ich deren Wichtigkeit.
Ich wünschte, ich hätte Ihre Anmerkungen über das Auswendiglernen gehabt als ich anfing, weil ich extrem viel Zeit gebraucht habe, es endlich (auf die falsche Art) auswendig zu lernen.
Ich bin mir nicht sicher, wie ich das Stück beim Konzert vorspielen soll.
Bei den wenigen Gelegenheiten, bei denen ich für andere gespielt habe, stolperte ich über bestimmte Abschnitte, weil ich wegen meiner Nerven vergaß, wo ich im Stück war.
Das ist mein erstes Konzert, so daß ich nicht weiß, was mich erwartet.
Jeder Tip oder Rat über Übungsabläufe wäre mir sehr willkommen.<br>
[Nachdem wir uns ein paarmal darüber ausgetauscht hatten, was er spielte, usw., gab ich ihm ein Szenario von typischen Übungsabläufen für die Konzertvorbereitung und was er während des Konzerts erwarten sollte.
Nach dem Konzert erhielt ich die folgende E-Mail:]<br>
Ich wollte Sie nur wissen lassen, daß mein Konzert, dafür daß es das erste Mal war, extrem gut verlaufen ist.
Der Rat, den Sie mir gegeben haben, war sehr hilfreich.
Ich war nervös als ich das Stück begann, wurde dann aber extrem fokussiert (so wie Sie es sagten, daß es geschehen würde).
Ich konnte mich sogar auf die Musik konzentrieren und nicht nur die Bewegungen durchgehen.
Das Publikum war von meiner Fähigkeit, es aus dem Gedächtnis heraus zu tun, beeindruckt (wie sie sagten, daß sie es würden).
Sie hatten Recht damit, daß eine positive Erfahrung wie diese mir mit meinem Selbstvertrauen helfen würde.
Ich fühle mich aufgrund der Erfahrung großartig!
Mein Lehrer ist von [ein berühmtes Konservatorium] und lehrt Hanon und anderes technisches Material.
Deshalb war und ist Ihr Buch eine Goldmine für mich.
Ich möchte in der Lage sein, die Stücke zu spielen, die ich mag, ohne 20 Jahre damit verbringen zu müssen, sie zu lernen.
Ich fühle aber auch, daß ich einen Lehrer brauche.


\item \label{testimonials25}
[Und schließlich hunderte von Zuschriften der Art:]

Ich muß sagen, daß Ihr Buch hervorragend ist . . .

Seit ich C. C. Chang's Fundamentals of Piano Practice gelesen habe, habe ich seine Vorschläge ausprobiert; Dank an diejenigen, die es empfohlen haben und an Herrn Chang, daß er sich die Zeit genommen hat, es zu schreiben und dafür, daß er es verfügbar gemacht hat.

Ich fand Ihre Webseiten bei meinem Arbeiten für das Klavierspielen sehr nützlich.

Ihre Arbeit ist einfach wunderbar!

Das ist hilfreich und ermutigend, da ich zum Klavier zurückkehre, nachdem ich viele Jahre keines zur Verfügung hatte; danke!

Sie haben mir enorm geholfen.

Nach dem, was ich bisher gelesen habe, macht es viel Sinn, und ich bin darauf gespannt es auszuprobieren.

Usw., usw.


</ol> 



<p align=\enquote{center}>Danksagung</p>

Dieses Buch ist meiner Frau Merry gewidmet, deren Liebe, Unterstützung und grenzenlose Energie mich in die Lage versetzt haben, für dieses Projekt so viel Zeit zu verwenden.
 

\textbf{\textit{Ende der Übersetzung dieser Seite}}

<table>
 Original: & \hyperref[http://www.pianopractice.org]{http://www.pianopractice.org} (extern) \\ 
</table>





