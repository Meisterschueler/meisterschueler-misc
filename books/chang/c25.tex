% File: c25

\section{Wie man anfängt}\hypertarget{c2_5}{}

\subsection{Einleitung}\hypertarget{c2_5a}{}

\textbf{Ohne einen Lehrer können Sie nicht einfach mit dem Stimmen anfangen.}
Sie werden schnell Ihren Bezugspunkt verlieren und keine Ahnung haben, wie Sie wieder zurückkommen.
\textbf{Deshalb müssen Sie zunächst bestimmte Stimmverfahren lernen und üben, damit Sie am Ende nicht mit einem unspielbaren Klavier dastehen, das Sie nicht wiederherstellen können.}
Dieser Abschnitt ist ein Versuch, Sie auf die Stufe zu bringen, bei der Sie ein richtiges Stimmen versuchen können, ohne auf Schwierigkeiten dieser Art zu stoßen.

\textbf{Als erstes müssen Sie lernen, was man tun kann und was nicht, um zu vermeiden, daß Sie das Klavier zerstören, was leicht geschehen kann.
Wenn man eine Saite zu stark spannt, dann bricht sie.\footnote{Verletzungsgefahr!}
Die anfänglichen Anweisungen sind dafür gedacht, Saitenbrüche aufgrund von amateurhaftem Vorgehen zu minimieren, lesen Sie sie deshalb sorgfältig.}
Sie müssen im voraus planen, was Sie tun, wenn eine Saite bricht.
Eine gebrochene Saite ist, auch wenn Sie über längere Zeit nicht ersetzt wird, für sich genommen keine Katastrophe für das Klavier.
Es ist jedoch wahrscheinlich klug, die ersten Übungen zu machen, kurz bevor man die Absicht hat, seinen Stimmer zu sich zu bitten.
Wenn Sie erst wissen wie man stimmt, ist ein Saitenbruch - außer bei sehr alten oder mißhandelten Klavieren -  ein seltenes Problem.
Die Stimmwirbel werden während des Stimmens um solch kleine Beträge gedreht, daß die Saiten fast nie brechen.
Ein verbreiteter Fehler, der von Anfängern begangen wird, ist, den Stimmhammer am falschen Stimmwirbel anzusetzen.
Da das Drehen des Wirbels keine hörbare Veränderung bewirkt, wird dann weitergedreht, bis die Saite bricht.
Eine Möglichkeit, das zu vermeiden, ist, \textbf{immer damit zu beginnen, \textit{tiefer} zu stimmen, wie es unten empfohlen wird, und niemals den Wirbel zu drehen, ohne den Ton anzuhören.}

\textbf{Die wichtigste Aufgabe für einen beginnenden Stimmer ist, den Zustand des Stimmstocks zu bewahren.}
Der Druck des Stimmstocks auf die Wirbel ist enorm.
Sie dürfen das natürlich niemals tun, aber angenommen, Sie würden den Wirbel sehr schnell um 180 Grad drehen, wäre die dabei an der Fläche zwischen Wirbel und Stimmstock erzeugte Hitze ausreichend, um das Holz zu verbrennen und seine molekulare Struktur zu verändern.
Es ist klar, daß alle Drehungen des Wirbels in langsamen, kleinen Schritten ausgeführt werden müssen.
Wenn Sie den Wirbel durch Drehen entfernen müssen, drehen Sie ihn nur eine viertel Drehung (gegen den Uhrzeigersinn), warten Sie einen Moment, bis sich die Hitze von der Grenzfläche weg verteilt hat, wiederholen Sie dann den Vorgang, usw., um eine Beschädigung des Stimmstock zu vermeiden.

\textbf{Ich werde alles am Beispiel des Flügels erklären, die entsprechenden Bewegungen für \enquote{\hyperlink{upright}{Aufrechte}} sollten aber offensichtlich sein.}
\textbf{Es gibt beim Stimmen zwei grundlegende Bewegungen.
Die erste ist die Drehung des Wirbels, so daß die Saite entweder angezogen oder entspannt wird.\footnote{Mit \enquote{wiegen} ist im folgenden kein wildes Hin- und Herschaukeln des Stimmwirbels gemeint, sondern jeweils das behutsame Ziehen von der Saite weg bzw. behutsame Nachgeben zur Saite hin!}
Die zweite ist, den Wirbel rückwärts zu Ihnen hin zu wiegen (um an der Saite zu ziehen) oder ihn vorwärts zu wiegen, in Richtung der Saite, um sie nachzulassen.}
Wenn die wiegende Bewegung extrem ausgeführt wird, vergrößert sie das Loch und beschädigt den Stimmstock.
Beachten Sie, daß das Loch an der Oberseite des Stimmstocks ein wenig elliptisch ist, weil die Saite den Wirbel in Richtung der Hauptachse der Ellipse zieht.
Darum vergrößert ein kleines Wiegen rückwärts die Ellipse nicht, weil der Wirbel durch die Saite immer in das vordere Ende der Ellipse gezogen wird.
Auch ist der Wirbel nicht gerade, sondern wird durch den Zug der Saite elastisch zur Saite hin gebogen.
Deshalb kann die wiegende Bewegung für das Bewegen der Saite sehr effektiv sein.
Sogar ein geringes Maß an Vorwärtswiegen, innerhalb der Elastizität des Holzes, ist unschädlich.
Anhand dieser Überlegungen wird deutlich, daß \textbf{Sie die Drehung benutzen müssen, wann immer sie möglich ist, und die wiegende Bewegung nur, wenn sie absolut notwendig ist}.
Nur sehr kleine wiegende Bewegungen sollten angewandt werden.
Bei den höchsten Noten (die zwei obersten Oktaven), ist die für das Stimmen der Saite notwendige Bewegung so gering, daß Sie eventuell nicht in der Lage sind, sie angemessen durch das Drehen des Wirbels zu kontrollieren.
Das Wiegen bietet eine viel feinere Kontrolle und kann für diese \hyperlink{c2_5_infi}{abschließende, winzige Bewegung} benutzt werden, um die Saite in perfekte Stimmung zu bringen.

Was ist nun der einfachste Weg, mit dem Üben zu beginnen?
Lassen Sie uns zunächst die am einfachsten zu stimmenden Noten auswählen.
Diese liegen in der \hyperlink{Noten}{C3-C4}-Oktave.
Tiefere Noten sind wegen ihres hohen harmonischen Gehalts schwieriger zu stimmen, und die höheren Noten sind schwierig, weil das für das Stimmen notwendige Maß der Wirbeldrehung mit steigender Tonhöhe abnimmt.
Beachten Sie, daß C4 für das mittlere C steht; das H direkt darunter ist H3, und das D direkt über dem mittleren C ist D4.
Die Oktavnummer 1, 2, 3, . . . ändert sich somit beim C, nicht beim A.
Wählen wir das G3 als unsere Übungsnote, und fangen wir mit dem Numerieren der Saiten an.
Jede Note in diesem Bereich hat 3 Saiten.
Von der linken Seite beginnend, numerieren wir die Saiten 123 (für G3), 456 (für G\#3), 789 (für A3), usw.
Fügen Sie zwischen den Saiten 3 und 4 einen Keil ein, um die Saite 3 zu dämpfen, so daß nur 1 und 2 schwingen können, wenn Sie G3 spielen.
Plazieren Sie den Keil ungefähr in der Mitte zwischen Steg und Agraffe.

\textbf{Es gibt zwei grundlegende Arten zu stimmen: unisono und harmonisch.}
Beim Unisono werden die beiden Saiten identisch gestimmt.
Beim harmonischen Stimmen wird eine Saite harmonisch zur anderen gestimmt, z.B. im Abstand einer Terz, Quarte, Quinte oder Oktave.
Die drei Saiten jeder Note unisono zu stimmen, ist einfacher, als harmonisch zu stimmen; lassen Sie uns also damit beginnen.
 

\hypertarget{c2_5b}{}
\subsection{Einsetzen und Bewegen des Stimmhammers}\hypertarget{c2_5_hamm}{} 

Wenn Ihr Stimmhammer eine justierbare Länge hat, ziehen Sie ihn ungefähr 3 Zoll\footnote{7,5cm} heraus, und stellen Sie ihn fest.
Halten Sie den Griff des Stimmhammers in Ihrer RH und den Einsatz mit Ihrer LH, und setzen Sie den Einsatz oben am Wirbel an.
Richten Sie den Griff so aus, daß er ungefähr senkrecht zu den Saiten steht und nach rechts zeigt.
Wackeln Sie mit Ihrer RH leicht mit dem Griff um den Stimmwirbel, und schieben Sie den Einsatz mit Ihrer LH über den Wirbel nach unten, so daß der Einsatz sicher so weit eingeschoben ist wie es geht.
\textbf{Entwickeln Sie vom ersten Tag an die Angewohnheit, mit dem Einsatz zu wackeln, so daß er sicher eingeschoben ist.}
An diesem Punkt ist der Griff wahrscheinlich nicht perfekt senkrecht zu den Saiten; wählen Sie einfach die Position des Einsatzes so, daß der Griff so gut wie möglich senkrecht steht.
Finden Sie nun eine Position, in der Sie Ihre RH so abstützen, daß Sie einen festen Druck auf den Hammer ausüben können.
Sie können z.B. die Spitze des Griffs mit dem Daumen und einem oder zwei Fingern greifen und den Arm auf dem hölzernen Klavierrahmen abstützen oder den kleinen Finger auf den Stimmwirbeln direkt unter dem Griff abstützen.
Wenn der Griff näher an der Platte (dem Metallrahmen) über den Saiten ist, könnten Sie Ihre Hand an der Platte abstützen.
Sie sollten den Griff nicht so greifen, wie Sie einen Tennisschläger halten, und ziehen bzw. drücken, um den Wirbel zu drehen - das gibt Ihnen nicht genügend Kontrolle.
Sie werden vielleicht nach mehreren Jahren Übung dazu in der Lage sein, aber am Anfang ist eine exakte Kontrolle zu schwierig, wenn man den Griff packt und drückt, ohne sich an etwas abzustützen.
\textbf{Entwickeln Sie deshalb die Angewohnheit, je nach der Griffposition gute Stellen zum Abstützen zu finden.}
Üben Sie diese Positionen und stellen Sie sicher, daß Sie einen kontrollierten, konstanten und kräftigen Druck auf den Griff ausüben können, aber drehen Sie noch keine Wirbel.

Der Hammergriff muß nach rechts zeigen, so daß Sie, wenn Sie ihn zu sich hin drehen (die Saite wird gespannt), gegen die Kraft der Saite arbeiten und den Wirbel aus der Vorderseite des Lochs (zur Saite hin) befreien.
Das gestattet - wegen der reduzierten Reibung - dem Wirbel, sich freier zu drehen.
Wenn Sie \textit{tiefer} stimmen, versuchen sowohl Sie als auch die Saite, den Wirbel in die gleiche Richtung zu drehen.
Der Wirbel würde dabei zu leicht drehen, wenn nicht sowohl Ihr Druck als auch der Zug der Saite den Wirbel gegen die Vorderseite des Lochs drücken, somit den Druck (Reibung) erhöhen und es verhindern würden.
Würden Sie den Griff nach links stellen, bekämen Sie sowohl bei der Bewegung zum Höher- als auch zum Tieferstimmen Probleme.
Beim Höherstimmen drücken sowohl Sie als auch die Saite den Wirbel gegen die Vorderseite des Lochs, was es um so schwieriger macht, den Wirbel zu drehen, und das Loch beschädigt.
Beim Tieferstimmen neigt der Hammer dazu, den Wirbel von der Vorderkante des Lochs abzuheben und reduziert die Reibung.
Außerdem drehen sowohl der Hammer als auch die Saite den Wirbel in die gleiche Richtung.
Jetzt dreht sich der Wirbel zu leicht.
Der Hammergriff muß bei \enquote{Aufrechten} nach links zeigen.
Wenn man von oben auf den Stimmwirbel schaut, sollte der Hammer bei Flügeln nach 3 Uhr und bei \enquote{Aufrechten} nach 9 Uhr zeigen.
In beiden Fällen befindet sich der Hammer auf der Seite der letzten Windung der Saite.

Professionelle Stimmer benutzen diese Hammerpositionen nicht.
Die meisten benutzen 1-2 Uhr für Flügel und 10-11 Uhr für \enquote{Aufrechte}, und Reblitz empfiehlt 6 Uhr für Flügel und 12 Uhr für \enquote{Aufrechte}.
Um zu verstehen warum, betrachten wir zunächst das Einsetzen des Hammers bei einem Flügel bei 12 Uhr (bei 6 Uhr ist es ähnlich).
Nun ist die Reibung des Wirbels mit dem Stimmstock beim Höher- und Tieferstimmen die gleiche.
Beim Höherstimmen arbeiten Sie jedoch gegen die Saitenspannung und beim Tieferstimmen hilft Ihnen die Saite.
Deshalb ist die Differenz der benötigten Kraft zwischen Höher- und Tieferstimmen viel größer als die Differenz ist, wenn der Hammer auf 3 Uhr steht, was ein Nachteil ist.
Anders als bei der 3-Uhr-Position, wiegt der Wirbel während des Stimmens nicht vor und zurück, so daß, wenn Sie den Druck auf den Stimmhammer nachlassen, der Wirbel nicht zurückspringt - er ist stabiler - und Sie können eine höhere Genauigkeit erreichen.

Die 1-2-Uhr-Position ist ein guter Kompromiß, der sowohl die Vorteile der 3-Uhr-Position als auch der 12-Uhr-Position ausnutzt.
Anfänger haben nicht die Genauigkeit, um den vollen Vorteil aus der 1-2-Uhr-Position zu ziehen; mein Vorschlag ist deshalb, mit der 3-Uhr-Position anzufangen, was zunächst einfacher sein sollte, und zur 1-2-Uhr-Position überzugehen, wenn Ihre Genauigkeit steigt.
Wenn Sie gut werden, kann die höhere Genauigkeit der 1-2-Uhr-Position Ihr Stimmen beschleunigen, so daß Sie jede Saite in wenigen Sekunden stimmen können.
Bei der 3-Uhr-Position werden Sie raten müssen wieviel der Wirbel zurückspringt und um diesen Betrag überstimmen müssen, was mehr Zeit benötigt.
Klar wird es wichtiger, wo Sie den Hammer plazieren, sobald Sie besser werden.
 

\hypertarget{c2_5c}{}
\subsection{Den Wirbel einstellen}\hypertarget{c2_5_wirb}{} 

\textbf{Es ist wichtig, den Stimmwirbel richtig \enquote{einzustellen}, damit die Stimmung hält.}
Wenn man von oben auf den Wirbel schaut, kommt die Saite von der rechten Seite des Wirbels (bei Flügeln - sie ist bei \enquote{Aufrechten} auf der linken Seite) und ist um ihn herumgewickelt.
Deshalb stimmen Sie \textit{höher}, wenn Sie den Wirbel im Uhrzeigersinn drehen, und \textit{tiefer}, wenn Sie den Wirbel gegen den Uhrzeigersinn drehen.
Die Saitenspannung versucht immer, den Wirbel gegen den Uhrzeigersinn zu drehen (oder \textit{tiefer}).
Normalerweise verstimmt sich ein Klavier \textit{tiefer}, wenn man es spielt.
Da der Stimmstock den Wirbel so stark umklammert, ist der Wirbel jedoch niemals gerade sondern verdreht.

Wenn man ihn im Uhrzeigersinn dreht und anhält, wird die Oberseite des Wirbels in bezug auf den Boden im Uhrzeigersinn verdreht.
In dieser Position möchte die Oberseite des Wirbels gegen den Uhrzeigersinn drehen (der Wirbel möchte sich zurückdrehen), aber er kann nicht, weil er vom Stimmstock gehalten wird.
Erinnern Sie sich daran, daß die Saite ebenfalls versucht, ihn gegen den Uhrzeigersinn zu drehen.
Die beiden Kräfte zusammen können genügen, um das Klavier schnell \textit{tiefer} zu verstimmen, wenn man etwas laut spielt.

Wenn der Wirbel gegen den Uhrzeigersinn gedreht wird, geschieht das Gegenteil - der Wirbel will sich im Uhrzeigersinn zurückdrehen, was der Saitenkraft entgegenwirkt.
Das reduziert das Nettodrehmoment am Wirbel, was die Stimmung stabiler macht.
Tatsächlich kann man den Wirbel so weit gegen den Uhrzeigersinn verdrehen, daß die zurückdrehende Kraft viel größer als die Saitenkraft ist, und das Klavier kann sich dann beim Spielen selbst \textit{höher} verstimmen.
Klar muß man den Wirbel richtig einstellen, damit man eine stabile Stimmung erzeugt.
Diese Erfordernis wird bei den folgenden Stimmanweisungen berücksichtigt.
 

\hypertarget{c2_5d}{}
\subsection{Unisono stimmen}\hypertarget{c2_5_unis}{}

Stecken Sie nun den Stimmhammer auf den Wirbel für Saite 1.
Wir werden die Saite 1 nach Saite 2 stimmen.
\textbf{Die Stimmbewegung, die Sie üben werden, ist:}

\begin{enumerate}[label={\arabic*.}] 
 \item \textbf{\textit{tiefer}}
 \item \textbf{\textit{höher}}
 \item \textbf{\textit{tiefer}}
 \item \textbf{\textit{höher}}
 \item \textbf{\textit{tiefer}}
 \end{enumerate}
Außer bei (1) muß jede Bewegung kleiner als die vorhergehende sein.
Wenn Sie besser werden, werden Sie Schritte passend hinzufügen oder weglassen.
Wir nehmen an, daß die beiden Saiten fast gestimmt sind.
Während Sie stimmen, müssen Sie zwei Regeln beachten:

\begin{itemize} 
 \item \textbf{Drehen Sie nie einen Wirbel, wenn Sie nicht gleichzeitig auf den Ton hören.}
 \item \textbf{Lassen Sie nie den Druck auf den Griff des Stimmhammers nach, bis diese Bewegung komplett ist.}
 \end{itemize}
Fangen wir z.B. mit Bewegung (1) \textit{tiefer} an: spielen Sie die Note alle ein oder zwei Sekunden mit der LH, so daß es einen dauernden Ton gibt, während Sie das Ende des Hammergriffs mit dem Daumen und dem Zeigefinger von sich weg drücken.
Spielen Sie die Note so, daß Sie einen fortwährenden Ton aufrecht erhalten.
Heben Sie die Taste nicht an, egal wie lang, da dies den Ton stoppt.
Halten Sie die Taste unten, und spielen Sie mit einer schnellen Auf- und Abbewegung, so daß der Ton nicht unterbrochen wird.
Der kleine Finger und der Rest Ihrer RH sollten gegen das Klavier abgestützt werden.
Die erforderliche Bewegung des Hammers beträgt nur ein paar Millimeter.
Zunächst werden Sie einen steigenden Widerstand spüren, und dann wird der Wirbel anfangen sich zu drehen.
Bevor der Wirbel anfängt sich zu drehen, sollten Sie eine Veränderung im Ton hören.
Hören Sie beim Drehen des Wirbels darauf, wie die Saite 1 \textit{tiefer} wird und eine Schwebung mit der mittleren Saite erzeugt; die Schwebungsfrequenz nimmt während Sie drehen zu.
Hören Sie bei einer Schwebungsfrequenz von 2 bis 3 je Sekunde auf.
Das äußere Ende des Hammergriffs sollte sich weniger als einen cm bewegen.
Erinnern Sie sich daran, daß Sie nie den Wirbel drehen, wenn kein Ton zu hören ist, weil Sie sonst in bezug auf die Änderung der Schwebungen sofort die Orientierung verlieren.
Halten Sie aus demselben Grund immer einen konstanten Druck auf den Hammer aufrecht, bis die Bewegung abgeschlossen ist.

Was ist die rationale Erklärung für die o.a. 5 Bewegungen?
Angenommen, die beiden Saiten sind vernünftig gestimmt, dann stimmen Sie bei Schritt (1) die Saite 1 \textit{tiefer}, um sicherzustellen, daß Sie in Schritt (2) den Stimmpunkt passieren\footnote{d.h. den Punkt, an dem die Saite genau richtig gestimmt ist}.
Das schützt auch gegen die Möglichkeit, daß Sie den Hammer auf den falschen Stimmwirbel gesetzt haben; solange Sie \textit{tiefer} stimmen, werden Sie niemals eine Saite zerbrechen.

Nach (1) sind Sie mit Sicherheit \textit{tiefer}, so daß Sie in Schritt (2) auf den Stimmpunkt hören können, während Sie durch ihn hindurchkommen.
Gehen Sie darüber hinaus, bis Sie eine Schwebungsfrequenz von ungefähr 2 bis 3 je Sekunde auf der \textit{höher}en Seite hören und stoppen Sie.
Sie wissen nun, wo der Stimmpunkt ist und wie er klingt.
Der Grund dafür, so weit über den Stimmpunkt hinaus zu gehen, ist, daß Sie den Wirbel wie oben erklärt einstellen möchten.

Kehren Sie nun zu \textit{tiefer} zurück, Schritt (3), aber stoppen Sie dieses Mal direkt hinter dem Stimmpunkt, sobald Sie irgendwelche einsetzenden Schwebungen hören können.
Der Grund, warum man nicht zu weit hinter den Stimmpunkt kommen möchte, ist, daß man nicht das \enquote{Einstellen des Wirbels} aus Schritt (2) rückgängig machen möchte.
Achten Sie wieder genau darauf, wie der Stimmpunkt klingt.
Er sollte perfekt sauber und rein klingen.
Dieser Schritt stellt sicher, daß Sie den Wirbel nicht zu weit eingestellt haben.

Führen Sie nun das endgültige Stimmen durch, indem Sie in Richtung \textit{höher} gehen (Schritt 4), dabei so wenig wie möglich über die perfekte Stimmung hinausgehen und die Saite dann durch Drehen nach \textit{tiefer} (Schritt 5) in die endgültige Stimmung bringen.
Beachten Sie, daß Ihre letzte Bewegung immer \textit{tiefer} sein muß, um den Wirbel einzustellen.
Wenn Sie gut darin werden, könnten Sie in der Lage sein, das Ganze in drei Bewegungen (\textit{tiefer, höher, tiefer}) durchzuführen.

Idealerweise sollten Sie von Schritt (1) bis zur endgültigen Stimmung den Ton ohne Unterbrechung aufrechterhalten, immer Druck auf den Griff ausüben und niemals den Hammer loslassen.
Am Anfang werden Sie das wahrscheinlich Bewegung für Bewegung ausführen müssen.
Wenn Sie es beherrschen, wird die ganze Durchführung nur ein paar Sekunden dauern.
Aber zunächst wird es \textit{viel} länger dauern.
Bis Sie Ihre \enquote{Stimmuskeln} entwickelt haben, werden Sie schnell ermüden und von Zeit zu Zeit aufhören müssen, um sich zu erholen.
Das gilt nicht nur für die Hand- und Armmuskeln, auch die erforderliche Konzentration des Geistes und des Gehörs auf die Schwebungen kann eine große Anstrengung sein und schnell Ermüdung verursachen.
Sie müssen schrittweise eine \enquote{Stimmausdauer} entwickeln.
Die meisten kommen besser zurecht, wenn Sie nur mit einem statt mit beiden Ohren hören; drehen Sie deshalb Ihren Kopf, um festzustellen, welches Ohr besser ist.

\textbf{Der häufigste Fehler, den Anfänger in diesem Stadium begehen, ist, bei dem Versuch, die Schwebungen zu hören, die Stimmbewegung zu unterbrechen.}
Schwebungen sind schwer zu hören, wenn sich nichts ändert.
Wenn der Wirbel nicht gedreht wird, ist schwer zu entscheiden, welche der vielen Dinge, die man hört, die Schwebung ist, auf die man sich konzentrieren muß.
\textbf{Stimmer bewegen den Hammer weiter und hören dann auf \underline{die Veränderungen der Schwebungen}.}
Wenn die Schwebungen sich ändern, ist es einfacher, die einzelne Schwebung zu identifizieren, die man für das Stimmen dieser Saite benutzt.
Deshalb wird es nicht einfacher, wenn man die Stimmbewegung verlangsamt.
Somit bewegt sich der Anfänger auf einem schmalen Grat.
Wenn man den Wirbel zu schnell dreht, bricht die Hölle los und man verliert die Orientierung.
Wenn man auf der anderen Seite zu langsam dreht, wird es schwierig, die Schwebungen zu identifizieren.
Arbeiten Sie deshalb daran, den Bereich der Bewegung zu bestimmen, den Sie benötigen, um die Schwebungen zu erkennen und die richtige Geschwindigkeit, mit der Sie den Wirbel beständig drehen können, um die Schwebungen entstehen und verschwinden zu lassen.
Falls Sie sich hoffnungslos verirrt haben, dämpfen Sie die Saiten 2 und 3, indem Sie einen Keil zwischen sie setzen, spielen Sie die Note, und hören Sie, ob Sie eine andere Note auf dem Klavier finden, die der Note nahe kommt.
Wenn die andere Note tiefer ist als G3, dann müssen Sie \textit{höher} stimmen, um zurückzukommen, und umgekehrt.

Wenn Sie nun die Saite 1 mit Saite 2 gleich gestimmt haben, bringen Sie den Keil in eine neue Position, so daß Saite 1 gedämpft wird und die Saiten 2 und 3 frei schwingen können.
Stimmen Sie Saite 3 nach Saite 2.
Wenn Sie zufrieden sind, entfernen Sie den Keil und hören Sie, ob das G nun frei von Schwebungen ist.
Sie haben eine Note gestimmt!
Wenn das G ziemlich gut gestimmt war, bevor Sie angefangen haben, haben Sie nicht viel erreicht; finden Sie eine Note in der Nähe, die aus der Stimmung ist, um zu sehen, ob Sie sie \enquote{reinigen} können.
Beachten Sie, daß Sie bei diesem Schema immer eine einzelne Saite nach einer anderen einzelnen Saite stimmen.
Im Prinzip sind, wenn Sie wirklich gut sind, die Saiten 1 und 2 perfekt gestimmt, nachdem Sie mit dem Stimmen von 1 fertig sind, so daß Sie den Keil nicht mehr brauchen.
Sie sollten in der Lage sein, Saite 3 nach den zusammen schwingenden 1 und 2 zu stimmen.
In der Praxis funktioniert das nicht, bis Sie es wirklich beherrschen.
Das kommt von einem Phänomen, das man \hyperlink{mitschwingung}{Mitschwingung} nennt.
 


