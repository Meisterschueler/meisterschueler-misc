% File: c35

\chapter{Was Träume erzeugt und Methoden zu ihrer Kontrolle}
\label{c3_5}

\section{Einleitung}
\label{c3_5a}

Dieser Abschnitt hat nichts mit Klavieren zu tun.
Er ist hier eingefügt, weil er ein wenig Klarheit darüber bringt, wie das Gehirn funktioniert.
Ich kenne keine Untersuchungen über die Ursachen von Träumen und Methoden für ihre Kontrolle, wie ich sie unten beschreibe.
Wenn Sie eine solche Quelle kennen, schicken Sie mir bitte eine Mail.

Haben Sie jemals wiederkehrende Träume gehabt und sich gefragt, was sie verursacht?
Oder Alpträume, die Sie gerne losgeworden wären?
Es scheint so, als hätte ich Antworten auf diese beiden Fragen gefunden und bei dem Prozeß einige Einsichten darüber gewonnen, wie das Gehirn während wir schlafen funktioniert.

Die meisten Traumdeuter sind heutzutage wie Handleser.
Sie bemühen sich, Ihre Zukunft vorherzusagen und schreiben Träumen magische Kräfte oder Botschaften zu, die wundervoll wären, wenn sie wahr wären, aber leider so realistisch sind wie Séancen oder Kaffeesatzlesen.
Ich habe gefunden, daß eine Interpretation der Träume, die auf körperlichen Anzeichen basiert, uns eine Menge darüber sagen kann, wie unser Gehirn funktioniert.
Ich bespreche hier vier Arten von Träumen, die ich hatte und für die ich eine körperliche Erklärung gefunden habe.
In Diskussionen mit Freunden habe ich entdeckt, daß viele ähnliche Träume haben und diese, fast mit Sicherheit, ähnliche Ursachen.
Im letzten Abschnitt bespreche ich, was diese Träume uns darüber sagen, wie unser Gehirn funktioniert.
Ich bin zu dem Schluß gekommen, daß dieses Herangehen an Träume viel lohnender ist als das der Wahrsager und ähnlicher Traumdeuter.
Die vier Träume, die unten besprochen werden sind:

\begin{itemize} 
 \item \hyperref[c3_5b]{fallen},
 \item \hyperref[c3_5c]{unfähig sein zu laufen},
 \item \hyperref[c3_5d]{zu spät zu Besprechungen oder Prüfungen kommen oder unfähig sein, das Ziel zu finden}, und
 \item \hyperref[c3_5e]{ein langer, komplexer Traum, der für mich spezifisch ist}.
\end{itemize}

Die ersten drei sind, so glaube ich, ziemlich verbreitete Träume, die viele Menschen haben.
 

\section{Der Fall-Traum}
\label{c3_5b}

In diesem Traum falle ich, nicht von einem bestimmten Ort oder hinunter auf einen bestimmten Platz, aber ich falle definitiv und habe Angst.
Und ich bin absolut unfähig, den Fall zu stoppen.
Stets bin ich, wenn ich lande, unverletzt.
Es gibt keine Schmerzen.
Tatsächlich fühlt es sich, obwohl ich auf dem Boden aufgeschlagen bin, wie eine weiche Landung an, und der Traum hört immer auf, sobald ich lande.
Die weiche Landung ist besonders seltsam, denn bei jedem Fall auf fast jede Fläche gibt es im allgemeinen am Ende irgendeine Art von Katastrophe.
Was würde alle diese Details jenes Traums erklären?
Ich habe eines Tages die körperliche Ursache dieses Traums entdeckt, als ich unmittelbar nach dem Traum aufwachte und feststellte, daß meine Knie heruntergefallen waren.
Ich hatte auf dem Rücken geschlafen, hatte beide Knie hochgestellt und als ich meine Beine streckte, führte das Gewicht der Bettdecke dazu, daß meine Füße ausrutschten und die Knie herunterfielen.
Diese fallenden Knie brachten mein Gehirn dazu, den Fall-Traum zu erzeugen!
Zunächst war das nur eine hypothetische Erklärung und eine offensichtlich dumme noch dazu.
Warum konnte mein Gehirn nicht erkennen, daß meine Knie gefallen waren?
Nachdem die Hypothese aber erst einmal aufgestellt war, konnte ich sie jedesmal überprüfen, wenn ich diesen Traum hatte (über einen Zeitraum von mehreren Jahren), und es gelang mir mehrere Male, sie zu bestätigen.
Beim Aufwachen konnte ich mich deutlich daran erinnern, daß meine Knie gerade eben heruntergefallen waren.
Die Tatsache, daß die Knie auf das weiche Bett fallen, erklärt die weiche Landung, und da hinterher nichts passiert, endet der Traum.
Warum bin ich unfähig, den Fall der Knie zu stoppen?
Wie weiter unten wiederholt gezeigt wird, haben wir während wir schlafen manchmal eine sehr geringe Kontrolle über unsere Muskeln.
Nicht nur das, das schlafende Gehirn kann nicht einmal die einfache Tatsache erkennen, daß das Knie fällt.
Zusätzlich denkt es sich das aus, was normalerweise ein unglaubliches Szenario eines Falls sein sollte, und tatsächlich glaube ich es am Ende.
Dieser letzte Teil ist der absurdeste, weil ich mich doch tatsächlich selbst hereinlege!


\section{Der Unfähig-zu-laufen-Traum}
\label{c3_5c}

Das ist ein sehr frustrierender Traum.
Ich möchte laufen, aber ich kann es nicht.
Es macht keinen Unterschied, ob mich jemand verfolgt oder ob ich bloß schnell irgendwo hinlaufen will; ich kann nicht laufen.
Wenn man läuft, muß man sich vorwärts beugen.
Deshalb versuche ich im Traum, mich nach vorne zu beugen, aber ich kann es nicht.
Irgend etwas schiebt mich fast zurück.
Im Traum habe ich sogar überlegt, daß wenn ich nicht vorwärts laufen oder mich vorwärts beugen kann, warum dann nicht zurücklehnen oder rückwärts laufen?
Auf diese Art kann ich mich zumindest bewegen.
Ich kann mich auch nicht zurücklehnen, meine Füße sind wie gelähmt, und ich komme weder vorwärts noch rückwärts richtig voran.
Wenn man läuft, muß man zunächst seine Knie nach vorne und oben bringen, so daß man nach hinten treten kann, aber ich kann auch das nicht.
Was würde ein solches Gefühl auslösen während ich schlafe?
Ich habe die Ursache dieses Traumes entdeckt, nachdem ich den \hyperref[c3_5b]{Fall-Traum} gelöst hatte, so daß die Erklärung leichter zu finden war.
Wieder kam ich auf die Erklärung, als ich unmittelbar nach dem Traum aufwachte und mich selber mit dem Gesicht nach unten auf dem Bauch liegend fand. Heureka!
Wenn man auf dem Bauch liegt, kann man den Winkel des Körpers zum Bett nicht verändern; man kann sich nicht vorwärts lehnen.
Man kann auch nicht die Knie nach oben ziehen, weil das Bett im Weg ist.
Man kann sich nicht zurücklehnen, weil man von der Schwerkraft nach unten gedrückt wird.
Man kann nicht rückwärts gehen, weil das Bettzeug im Weg ist.
Das zeigt erneut, daß man während man schläft keine große Kontrolle über die Muskeln hat, denn wenn man wach wäre, dann wäre das Hochziehen der Knie nicht so schwierig, sogar wenn man mit dem Gesicht nach unten liegen würde.
Nachdem ich die Erklärung gefunden hatte, konnte ich sie wieder mehrmals bestätigen; d.h., als ich wach wurde, lag ich mit dem Gesicht nach unten.
An diesem Punkt fing ich an zu erkennen, daß vielleicht die meisten unserer Träume eine körperliche Erklärung haben.
Das Ganze machte jedoch irgendwie keinen Sinn - warum sollte mein Gehirn nicht wissen, daß meine Knie herunterfallen, oder daß ich mit dem Gesicht nach unten schlafe?
Wie kann mein Gehirn einen so komplexen Traum träumen und trotzdem nicht in der Lage sein, solch einfache Dinge zu erkennen?
Und wieder hat sich mein Gehirn eine Geschichte ausgedacht und sie mich erfolgreich glauben lassen, während ich träumte.


\section{Der Zu-spät-zur-Prüfung-kommen- oder Sich-verlaufen-Traum}
\label{c3_5d}

Dies ist ein weiterer frustrierender Traum.
Können Sie sehen, wie ein Muster zum Vorschein kommt?
Ich werde weiter unten spekulieren, warum Träume dazu neigen, negativ oder alptraumhaft zu sein.
Dies ist kein bestimmter Traum, sondern eine ganze Klasse von Träumen, in denen ich versuche, zu einer Prüfung oder irgendwo anders hin zu kommen, aber spät dran bin und nicht hingelangen oder es nicht finden kann.
Ich muß z.B. einen steilen Hang überwinden oder um Gebäude herumlaufen.
Oder wenn ich in einem Gebäude bin, gehe ich durch einen Irrgarten aus Rampen, Treppen, Türen, Aufzügen usw., aber ich kann noch nicht einmal zum Ausgangspunkt zurück.
Tatsächlich wird es immer schlimmer und komplexer.
Nach einer Weile kann ich ziemlich erschöpft sein.
Dieser Traum könnte auftreten, wenn ich in einer ungünstigen oder unbequemen Position schlafe, aus der ich nicht leicht herauskomme, wie z.B. auf meiner Hand schlafend oder im Laken oder Bettzeug eingewickelt.
Jede Art von Schlafposition, die unbequem ist, aus der ich gerne herauskommen möchte, es aber nicht leicht tun kann während ich schlafe.
Wenn ich in den Laken eingewickelt bin, kann ich mich nicht so leicht daraus befreien während ich schlafe, und je mehr ich damit kämpfe, desto mehr verwickle ich mich darin, und es kann sehr anstrengend werden.
Ich war bisher nicht in der Lage, diese Traumfamilie oder eines seiner Mitglieder direkt mit einer bestimmten Ursache zu verbinden, wie bei den anderen drei Träumen.
Ich habe jedoch eine leichte Schlafapnoe und das erste Auftreten dieser Art von Träumen fällt damit zusammen, was ich für das erste Auftreten der Schlafapnoe halte.
Somit könnte der Traum durch meine Unfähigkeit zu atmen verursacht worden sein.

Was auch immer die genaue Ursache ist, ob eine unbequeme Position oder Schlafapnoe, so ist klar, daß ich, wenn ich wach gewesen wäre, leicht eine Lösung gefunden hätte.
Somit ist das Muster, das zum Vorschein kommt, daß mein logisches Denkvermögen und meine Fähigkeit zur Lösung von Problemen stark beeinträchtigt sind; sehr einfache Probleme können mich in die Klemme bringen während ich schlafe.
 

\section{Die Lösung für meinen langen und komplexen Traum}
\label{c3_5e}

Nachdem ich die Lösung für die drei oben genannten Träume gefunden hatte, war ich überzeugt, daß ein anderer wiederkehrender Traum ebenfalls eine körperliche Ursache hatte.
Dieser Traum war lang und komplex aber immer derselbe.
Er fängt angenehm an.
Ich gehe für eine Klettertour nach draußen, und vor mir ist eine sanfter Hügel oder eine wogende Wiese, die in der Ferne zu einem Berg führen.
Das erste Anzeichen, daß etwas nicht stimmt, kommt von diesem Berg.
Er geht mit steilen Felswänden nach oben, und die Spitze ist so hoch, daß ich sie kaum sehen kann.
Ich mache mich trotzdem auf den Weg, aber sofort tritt eine furchterregende Situation ein: Ich bin an der Kante einer vertikalen Felswand, und ich kann nicht einmal den Boden darunter sehen!
Ich bekomme Angst, drehe mich sofort um und versuche zurück zu gehen, aber der Vorsprung, auf dem ich weitergehe, wird schmaler, als ob ich auf einem Schwebebalken gehen würde.
Schließlich merke ich, daß ich fast am Ende bin aber eine letzte Hürde nehmen muß: einen Fluß!
Bevor ich über Felsen springe, um über den Fluß zu kommen, prüfe ich ihn mit der Hand, und das Wasser ist kalt und tief.
Ungefähr in diesem Stadium endet der Traum.
Wie würde ich solch einen komplexen Traum erklären?
Ich löste das Rätsel wieder, nachdem ich unmittelbar nach dem Traum erwachte.
Ich hatte am Rand des Betts geschlafen, und eine Hand schaute unter der Bettdecke hervor und hing herunter.
Nun konnte ich jedes Detail meines Traums erklären!
Mein Traum beginnt offensichtlich damit, daß ich auf dem Bauch schlafe, mit meinem Kinn auf dem Bett, und ich schaue auf das Kissen vor mir (die wogende Wiese).
Hinter dem Kissen ist das vertikale hölzerne Kopfende, aus kanadischer Walnuß hergestellt, das wie eine steile Felswand aussieht, welches der Berg ist.
Mit meinem Kinn auf dem Bett kann ich kaum die Spitze des Kopfendes sehen.
Bis hierher ist interessant, daß ich offensichtlich Sachen ansehe während ich schlafe.
Da ich an der Bettkante schlafe, fällt eine Hand über die Kante, und das ist die Kante der Felswand, an der ich stehe.
%Ungefähr sieben Zoll\footnote{\lt 20 cm} von meinem Bett entfernt steht mein Nachttisch mit einer schmalen abgestuften Kante wie die Oberseite eines Schwebebalkens (schwer zu beschreiben).
Ungefähr sieben Zoll\footnote{WTF!!! 20 cm} von meinem Bett entfernt steht mein Nachttisch mit einer schmalen abgestuften Kante wie die Oberseite eines Schwebebalkens (schwer zu beschreiben).
Meine Hand tastet also offensichtlich herum.
Da meine Hand nun nicht mehr unter der Bettdecke liegt, fühlt sie sich kalt an (der kalte Fluß). Das ist es!
Diese Erklärungen tragen jedem Detail meines Traums Rechnung!
Diese Erklärungen haben mich überzeugt, daß Träume interpretiert werden KÖNNEN, und daß die meisten von ihnen körperliche Ursachen haben.
Wenn das alles wahr ist, dann sollten wir in der Lage sein, die Ursachen und Erklärungen zu benutzen, um daraus abzuleiten, was das Gehirn während des Schlafens tut.
Das ist eine aufregende Aussicht, von deren Verwirklichung nicht einmal die Wahrsager und Traumdeuter träumen konnten.


\section{Die Kontrolle der Träume}
\label{c3_5f}

Das erstaunlichste an der Erklärung dieser Träume war, daß ich etwas Kontrolle über sie entwickelte.
Nachdem ich völlig überzeugt war, daß jede Erklärung richtig ist, verschwanden diese Träume!
Ich konnte mich nicht mehr selbst hereinlegen!
Zu denken, daß fallende Knie dasselbe sind wie von einem Dach oder einer Klippe zu stürzen, heißt ganz klar, mich selbst hereinzulegen.
Wenn der Mechanismus erst einmal verstanden wird, dann wird das Gehirn nicht mehr getäuscht.
Obwohl das Gehirn hinreichend abgeschaltet ist, so daß es während des Schlafs leicht getäuscht werden kann, hat es demnach genügend Kapazität, um die Wahrheit zu erkennen, wenn der Mechanismus erst bekannt ist.

Trotzdem erschien es mir irgendwie weit hergeholt, daß ich mich selbst hereingelegt hatte.
Um mich selbst davon zu überzeugen, daß diese Art von Täuschung möglich ist, mußte ich ein Beispiel aus dem richtigen Leben finden.
Glücklicherweise habe ich eins gefunden.
Es ist das, was Magier tun.
Wenn man einen Zaubertrick beobachtet, weiß man, daß es keine Zauberei ist, man fällt aber jedesmal in dem Sinn darauf herein, daß es völlig verwirrend und sehr aufregend ist.
Nun ändert sich die Geschichte gänzlich, wenn Ihnen jemand erklären sollte, wie der Trick funktioniert.
Dann verschwindet plötzlich das Mysterium und die Spannung, und man konzentriert sich am Ende darauf, wie der Magier den Trick ausführt.
Man kann nicht dazu verleitet werden zu denken, daß es Zauberei ist.
Somit kann unser Gehirn in einem Traum so lange getäuscht werden, wie es nicht weiß, daß es getäuscht wird.
Da die meisten Menschen die Erklärung für den Traum nicht kennen, sind sie sich der stattfindenden Täuschung offensichtlich nicht bewußt, und die Träume gehen weiter.
Kennt man erst einmal die Ursache des Traums, weiß man auch, daß das Gehirn getäuscht wurde; deshalb ist es dann für das Gehirn viel einfacher, die Wahrheit herauszufinden, und der Traum verschwindet.
Bevor man die Wahrheit herausgefunden hat, wußte das Gehirn nicht einmal, daß es getäuscht wurde, so daß es keinen Grund hatte, nach der Wahrheit zu suchen.
Nun scheint alles Sinn zu machen.


\section{Was uns diese Träume über das Gehirn lehren}
\label{c3_5g}

Diese vier Beispiele lassen darauf schließen, daß die meisten Träume einen konkreten körperlichen Ursprung haben.
Ich habe diese Art von Erklärung niemals zuvor gesehen, trotzdem erscheint alles vernünftig.
Soweit ich weiß, ist der \hyperref[c3_5b]{Fall-Traum} weit verbreitet - viele haben diesen Traum.
Bei mir war es das fallende Knie; bei jemand anderem mag es ein Arm oder ein Bein sein, daß über die Bettkante gleitet.

Die obigen Resultate bieten eine Unmenge an Möglichkeiten, über die Funktionsweise des Gehirns zu spekulieren.
Dazu ein paar Ideen.
Während des Schlafs ist der größte Teil des Gehirns abgeschaltet, so daß es nicht überraschend ist, wenn das Gehirn leicht getäuscht werden kann.
Es scheint, daß die höheren Funktionen vollständiger abgeschaltet sind, so daß das logische Denken am stärksten beeinträchtigt ist.
Es kann sein, daß Angst das Gefühl ist, das beim Einschlafen als letztes abgeschaltet und beim Aufwachen als erstes eingeschaltet wird - wahrscheinlich aus Gründen des Überlebens.
Wenn ein Feind während des Schlafs angreift, ist Angst das erste Gefühl, das aufgeweckt werden muß.
Das läßt darauf schließen, daß die meisten Träume tendenziell alptraumhaft sein könnten.
Aber selbstverständlich kann das von Person zu Person unterschiedlich sein, und einige Menschen können hauptsächlich angenehme Träume haben, je nach der Veranlagung der Person.
In meinem Fall legen die Anhaltspunkte nahe, daß die Träume, die ich entschlüsselt habe, unmittelbar bevor ich aufgewacht bin aufgetreten sind.
Das läßt darauf schließen, daß die meisten Träume während der kurzen Zeit zwischen Schlaf und Erwachen auftreten.
Obwohl es Schlafwandler gibt, die ihre starken Muskeln während des Schlafs kontrollieren können, zeigt das oben gesagte, daß die Bemühungen, während eines Traums die Muskeln zu bewegen, nicht gut in tatsächliche Bewegungen umgesetzt werden.
Das \hyperref[c3_5e]{vierte Beispiel} zeigt jedoch, daß man sich während des Schlafs viel bewegt - zusätzlich zu den normalen Bewegungen, die notwendig sind, um den Körper periodisch in eine andere Lage zu bringen, damit man einen ausgedehnten Durchblutungsmangel an den Stellen, an denen man aufliegt, vermeidet usw.
Somit ist die Bewegung des Körpers während des Schlafs ein völlig normaler Prozeß als Antwort auf die Schmerzen, die entstehen, wenn man zu lange in einer Position bleibt.
Eine Minderheit scheint in der Lage zu sein, im Grunde die ganze Nacht in einer Position zu schlafen; solche Menschen müssen eine Methode haben, die Auflagepunkte mit Sauerstoff usw. zu versorgen, so daß keine wunden Stellen entstehen (vielleicht bewegen sie sich unmerklich nach der einen oder anderen Seite, um den Druck zeitweilig zu verringern).

Ich glaube, daß ich hier ein paar überzeugende Beispiele dafür aufgeführt habe, wie Träume eher auf der Wirklichkeit basierend interpretiert werden können als auf den falschen übernatürlichen Kräften, die historisch bedingt mit der Traumdeutung verbunden werden.
Dieser Ansatz scheint eine Einsicht in die Arbeitsweise des Gehirns während des Schlafs zu liefern.
Eine mögliche Anwendung von Träumen, die mit der Realität verbunden werden können, ist, daß sie zu nützlichen Diagnosewerkzeugen für Störungen wie z.B. Schlafapnoe werden können.
Sie können uns viel über unsere Bewegungen während des Schlafs sagen und darüber, wie man etwas ändern kann, damit man besser schlafen kann.



