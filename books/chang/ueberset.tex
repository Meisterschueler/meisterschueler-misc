% File: ueberset 

\chapter{Anmerkungen zur Übersetzung}
\label{ueberset}

Ich hatte das Buch zunächst satzweise übersetzt und dabei den Sinn der von Chuan C. Chang benutzten Worte soweit wie möglich erhalten.
Mir ist bewußt, daß der Text dadurch an manchen Stellen etwas ungewöhnlich klang, aber insbesondere bei den Anweisungen zur Spieltechnik war mir die \enquote{Werktreue} wichtiger als die Sprachgewohnheiten.
Nachdem das Buch nun (mit Ausnahme der fortlaufenden Ergänzungen durch Chuan C. Chang) komplett übersetzt ist und ich inzwischen wesentlich besser verstehe, worauf es in den einzelnen Passagen ankommt, passe ich die Wortwahl während der Überarbeitung zunehmend an den alltäglichen Sprachgebrauch an, wenn ich mir sicher bin, den tieferen Sinn eines Abschnitts nicht zu verfälschen.

Wie übersetzt man das \textbf{you}?
 Da ich -- von ein paar Ausnahmen abgesehen -- auch nicht möchte, daß mich einfach jemand duzt, stand es für mich außer Frage, daß ich in meiner Übersetzung \textbf{you} in der Regel mit \textbf{Sie} übersetze.
Damit der Text dadurch nicht zu förmlich wird, habe ich mich dazu entschieden, nur bei Anleitungen, Empfehlungen usw.
den Leser durch die Verwendung von \textbf{Sie} direkt anzusprechen.
Bei allgemeinen Ausführungen, Erklärungen und Hintergrundinformationen habe ich dagegen oft \textbf{man} benutzt.


Ursprünglich hatte ich an manchen Stellen des Texts eine Mischung aus der weiblichen und der männlichen Form eines Wortes oder Ausdrucks benutzt.
Bei der Überarbeitung ersetze ich diese nun durch die derzeit immer noch häufiger benutzte männliche Form.
Das bedeutet nicht, daß ich jetzt zum Macho mutiert bin, sondern dient der flüssigeren Lesbarkeit, weil dadurch Konstrukte wie z.B. \enquote{die/der Klavierlehrer/in} entfallen.
Vielleicht gibt es ja bei der nächsten Rechtschreibreform auch zu diesem Thema etwas Neues.
Bis dahin bleibe ich jedenfalls bei der alten Rechtschreibung.


\label{Pedale}

Das \textbf{linke Pedal} wird meistens \textbf{Dämpfer- bzw. Dämpfungspedal oder Sordino} genannt und das \textbf{rechte Pedal} als \textbf{Haltepedal oder Verlängerungspedal} bezeichnet.
Neuerdings gibt es aber auch Quellen, in denen die Namen der Pedale an die englischen Ausdrücke angeglichen sind.
Darin wird nun das rechte Pedal als Dämpfer- bzw. Dämpfungspedal (damper pedal) bezeichnet und das linke u.a. als Pianopedal (soft pedal).
Es ist mir bisher nicht gelungen, den Ursprung dieser neuen Bezeichnungen ausfindig zu machen.
Solange ich nicht weiß, ob es sich hierbei wirklich um eine neue Wortbildung in der deutschen Sprache handelt oder die Änderung nur durch eine weitergetragene Falschübersetzung der Ausdrücke \enquote{damper pedal} und \enquote{soft pedal} entstanden ist, werde ich die bisherigen Begriffe im Text beibehalten.

Teilweise gibt es englische Wörter, für die es zwar im Deutschen ein Wort gibt, das jedoch negativ belegt ist, wie z.B. \enquote{exposed = ausgesetzt}.
Wurden solche Wörter in einem positiven Sinn gebraucht, habe ich versucht sie so zu umschreiben, daß der Sinn \underline{und} der positive Eindruck erhalten bleiben.<br>
\label{memorizer}
Außerdem gibt es Wörter, für die es -- ähnlich wie für das \enquote{Hier-kommt-der-Einkauf-des-nächsten-Kunden-Holz} an der Supermarktkasse -- keinen vernünftigen Ausdruck im Deutschen gibt.
So hatte ich \textbf{memorizer} zunächst mit \enquote{Merker} übersetzt. Nach fast 2 Jahren habe ich mich dann entschieden, das teilweise negativ belegte Wort \enquote{Merker} (z.B. Blitzmerker) zu ersetzen und \enquote{memorizer} mit \textbf{Auswendiglernender} und \enquote{non-memorizer} mit \textbf{Nichtauswendiglernender} zu übersetzen. Die etwas kürzere Fassung \enquote{Auswendiglerner} wird oft abwertend gebraucht (z.B. Kassenbon-Auswendiglerner), weshalb ich sie nicht benutze.<br>
\label{reversepsychology}
Für den Begriff \textbf{reverse psychology} scheint es außer der für mich unbefriedigenden Eindeutschung \textbf{umgekehrte Psychologie} keinen Ausdruck zu geben.
Damit ist gemeint, jemandem etwas so zu sagen, daß er hinterher das Gegenteil tut, und man genau das erreichen wollte.


\label{HsHt}

Die Abkürzungen wie \textbf{HS} (hands seperated) und \textbf{HT} (hands together) habe ich absichtlich nicht übersetzt.
Wenn Übersetzer aus anderen Ländern das gleiche tun, dann haben Leser aus unterschiedlichen Ländern weniger Probleme, sich untereinander über die in diesem Buch vorgestellten Methoden auszutauschen, weil es z.B. keinen Deutschen gibt, der von GH (getrennte Hände) und keinen Franzosen, der von MS (mains séparées) spricht.


\label{Motherboard}

Können Sie sich etwas unter einem \textit{Mutterbrett} vorstellen? Wie wäre es mit \textit{Hauptplatine}?
Schon besser oder?
Oder unter einer \textit{Klangkarte}? Da ist es leichter.<br>
In den Abschnitten über \hyperref[c1iii13MIDI]{MIDI, Digitalpianos usw.} hat es mich ja förmlich in den Fingern gejuckt aber ich habe die Begriffe \textbf{Motherboard, Soundkarte usw.} kommentarlos dringelassen, da jemand, der sich mit PCs auskennt, weiß was gemeint ist.
Trotzdem fände ich es gut, wenn die schleichende Anglisierung der deutschen Sprache ein wenig gebremst würde.
Dann bekomme ich vielleicht auch mal wieder wohlschmeckende Haferflocken statt crispiger Cerealien zum Frühstück.


\label{Noten}

Die \textbf{Noten} werden im Text gemäß der Konvention der Klavierstimmergilde bezeichnet, d.h. die 88 Tasten des Klaviers tragen die Bezeichnungen \textit{A0} bis \textit{C8}.
\textit{C4} steht somit für das \textit{mittlere C = c'}.
Die meisten Sequenzer-Programme verwenden jedoch einen anderen Wertebereich.
Der MIDI-Wert 0 entspricht dabei \textit{C-2}, was zu \textit{C3} für das mittlere C führt.
Um die Verwirrung zu komplettieren, beginnen manche Programme mit \textit{MIDI 0 = C0}, was zu \textit{C5} für das mittlere C führt.


\label{ueb-canonic}
Chuan C. Chang benutzt im Original ein Wortspiel mit dem aus der Thermodynamik stammenden Begriff \hyperref[canonic]{\textbf{canonical ensemble}} und der musikalischen Bedeutung der Begriffe \textit{canonical} und \textit{ensemble}, das mit der korrekten deutschen Übersetzung \textit{kanonische Gesamtheit} leider nicht mehr funktioniert.


\label{ueb-KV525}
Der Text des Wortwechsels der männlichen und weiblichen Stimme am Anfang von Mozarts \hyperref[KV525]{\enquote{Eine Kleine Nachtmusik} (KV525)} stammt nicht aus der Sekundärliteratur.
Er ist dem Originaltext von Chuan C. Chang angelehnt (\enquote{Hey, are you coming?} und \enquote{OK, OK, I'm coming!}) und berücksichtigt das 9-notige Schema sowie die Sprachgewohnheiten in Opern der damaligen Zeit.
Wenn Sie eine Quelle mit dem genauen Wortlaut kennen, lassen Sie es mich bitte wissen.


\label{upright}
Im allgemeinen ist aus dem Zusammenhang ersichtlich, ob es sich bei dem Wort \textbf{Klavier} um den Sammelbegriff (piano) handelt oder um ein Klavier im engeren Sinne, d.h. um ein aufrecht stehendes Klavier (upright) im Gegensatz zum Flügel (grand).
In der Übersetzung der Datei über \hyperref[c2_1]{das Stimmen} war aber zunächst nicht immer sofort zu erkennen, welche der beiden Bedeutungen jeweils gemeint war.
Da diese Eindeutigkeit jedoch beim Stimmen von besonderer Bedeutung ist, habe ich \textbf{upright} in dieser Datei ausnahmsweise mit \textbf{das \enquote{Aufrechte}} übersetzt.
In manchen Quellen findet man dafür auch die Bezeichnung \textit{Piano}, was aber als veraltet gilt.
Hin und wieder findet man auch \textit{Pianino}, aber das ist genaugenommen ein \enquote{Aufrechtes} mit verringertem Tonumfang, d.h. weniger als 88 Tasten.


\label{pin}
Ist ein \textbf{tuning pin} nun ein \textit{Stimmwirbel}, weil er zum Stimmen der Saite gedreht wird, oder ein \textit{Stimmnagel}, weil er eine relativ glatte Oberfläche hat und manchmal zur Rettung der Stimmbarkeit in den Stimmstock gehämmert wird?
Im Internet findet man \textit{Stimmwirbel} am meisten, der Begriff \textit{Stimmnagel} kommt allerdings auch recht häufig vor.
Manchmal findet man in einer Quelle sogar beide Begriffe in (un-)harmonischer Eintracht nebeneinander.
In Analogie zu anderen Instrumenten, bei denen zum Stimmen der Saiten ähnliche Teile \enquote{herumgewirbelt} werden, habe ich mich für \textbf{Stimmwirbel} entschieden. 


\label{et}
Der Begriff \textbf{gleichstufige Temperatur} bzw. \textbf{gleichmäßige Temperatur} ist korrekter als die meistens in der Literatur zu findende \textbf{gleichschwebende Temperatur}.
Die Frequenzverhältnisse der Halbtonschritte (kleine Sekunde) haben immer den gleichen Wert (100 Cent, bzw. zwölfte Wurzel aus 2, ungefähr 1,059463) und somit auch alle Intervalle (z.B. Quinte = 700 Cent).
Die exakten Werte der Schwebungsfrequenzen der Intervalle weichen jedoch voneinander ab (s. Reblitz S. 212ff.). Die Schwebungsfrequenz von zwei Saiten ist die Differenz der Grund- bzw. Obertöne der Saiten, deren Frequenzen am dichtesten beieinanderliegen. Dazu ein paar Beispiele, ausgehend von A4 = 440 Hz:

\begin{tabular}{l|l|l|l}
 Intervall & 1. Ton & 2. Ton & Schwebung \\
 \hline 
 kl. Sekunde & A4 = 440Hz & A\#4 $\approx$ 466,164Hz & $\approx$ 26,164Hz \\ 
   & A\#4 $\approx$ 466,164Hz & H4 $\approx$ 493,884 Hz & $\approx$ 27,720Hz \\ 
 Quinte & A3: 2. Oberton = 660Hz & E4: 1. Oberton $\approx$ 659,256Hz & $\approx$ 0,744Hz \\ 
   & A\#3: 2. Oberton $\approx$ 699,246Hz & F4: 1. Oberton $\approx$ 698,456Hz & $\approx$ 0,790Hz \\ 
\end{tabular}

Diese Werte gelten nur für idealisierte Saiten. Die Obertöne und somit auch die Schwebungsfrequenzen realer Saiten weichen wegen der \hyperref[c2_5_stre]{Inharmonizität} davon ab.

\label{mitschwingung}
Das korrekte Wort für \textbf{sympathetic vibration} ist tatsächlich \textbf{Mitschwingung}, da es hier um das Verhalten zweier oder mehrerer gleichzeitig angeschlagener Saiten geht, die ungefähr mit der gleichen Frequenz schwingen.
Das scheinbar naheliegende Wort \textit{Resonanz} beschreibt dagegen, daß eine nicht angeschlagene Saite, deren Frequenz z.B. das ganzzahlige Vielfache der Frequenz der angeschlagenen Saite ist, zur Schwingung angeregt wird. 


\label{moden}
\textbf{Schwingungsmoden} ist der physikalische Begriff für die verschiedenen Schwingungszustände einer Saite, deren Enden befestigt sind.
Vereinfacht könnte man sagen, sie sind die einzelnen Frequenzen, aus denen die Schwingung der Saite zusammengesetzt ist, d.h. der Grundton und die Obertöne.


\label{transversal}
\textbf{Transversale Wellen} haben ihre Amplitude (\enquote{Ausschlag}) quer zur Fortpflanzungsrichtung, wie z.B. an der Wasseroberfläche oder eben in Klaviersaiten.
Wenn eine Klaviersaite also so fest eingespannt wäre, daß sie sich nicht (oder kaum) quer zu ihrer Längsrichtung bewegen könnte, bliebe das Klavier ziemlich stumm.
Wellen, die ihre Amplitude in Bewegungsrichtung haben, heißen \textbf{longitudinale Wellen}.
Man sieht sie z.B. an der Oberseite eines Kornfelds, durch das der Wind weht.
Schallwellen sind ebenfalls longitudinale Wellen.


\label{johndoe}
Vielleicht habe ich es bei der Übersetzung des \hyperref[assoziativ]{assoziativen Prozesses zum Abrufen der Informationen aus dem Gedächtnis} ein wenig übertrieben, aber ich wollte \textbf{John} nicht einfach nur mit \enquote{Johann} oder \enquote{Johannes} übersetzen, weil \enquote{\textbf{Otto} Normalverbraucher} die deutsche Entsprechung von \enquote{John Doe, the average American} ist.




\label{AbkFarben}

<h2 align=\enquote{center}>Im Text verwendete Abkürzungen und Farben</h2>
LH = Left Hand = Linke Hand<br>
RH = Right Hand = Rechte Hand<br>
HS = Hands Separate = (im Wechsel) nur mit der LH oder RH spielen<br>
HT = Hands Together = mit beiden Händen zusammen spielen<br>
TO = Thumb Over = Daumenübersatz<br>
TU = Thumb Under = Daumenuntersatz<br>
FI = Fantaisie Impromptu, Op. 66, von Frederic Chopin /
 \hyperref[FI]{(1)},
 \hyperref[c1iii2]{(2)},
 \hyperref[c1iii5wagen]{(3)}<br>
FPD = Fast Play Degradation = \hyperref[fpd]{Abbau von Fähigkeiten durch zu schnelles Spielen}<br>
NG = Nucleation Growth = \hyperref[ng]{Kernbildung-Wachstums-Mechanismus}<br>
PPI = Post Practice Improvement = \hyperref[c1ii15]{automatische Verbesserung der Fähigkeiten nach dem Üben}<br>
ET = Equal Temperament = gleichschwebende, gleichstufige, bzw. gleichmäßige Temperatur
 (\hyperref[et1]{1}),
 (\hyperref[c2_6_et]{2})<br>
WT = Well Temperament(s) = \hyperref[c2_2_wtk2]{Wohltemperierte Stimmung(en)}<br>
HT = Historical Temperament(s) = \hyperref[c2_2_hist]{Historische Temperatur(en)}<br>
K-II = Kirnberger II
 (\hyperref[c2_2_wtk2]{1}),
 (\hyperref[c2_6_kirn]{2})<br>
<br>



%<table>
% \footnote{Blauer kursiver Text in eckigen Klammern} & Vom Übersetzer eingefügte Anmerkungen, die im Original nicht enthalten sind \\ 
% Grüner normaler Text & Zukünftige Links auf Abschnitte, die noch übersetzt werden \\ 
% \textbf{\textit{Grüner fetter kursiver Text}} & Hinweise auf Änderungen \\ 
%  (extern)  & Kennzeichnung von Links auf Seiten außerhalb von FOPPDE \\ 
%</table>


