% File: c23

\subsection{Werkzeuge zum Stimmen}
\label{c2_3}

\textbf{Sie werden einen Stimmhammer, mehrere Gummikeile, einen Filzstreifen zum Dämpfen, eine oder zwei Stimmgabeln und Ohrstöpsel oder Ohrenschützer benötigen.}
Professionelle Stimmer benutzen heutzutage auch eine elektronische Stimmhilfe; wir werden diese aber hier nicht berücksichtigen, weil sie für den Amateur nicht rentabel ist und ihre richtige Anwendung ein fortgeschrittenes Wissen über die Feinheiten des Stimmens erfordert.
Die Stimmethode, die wir hier behandeln, wird aurales Stimmen genannt -- Stimmen nach Gehör.
Alle guten professionellen Stimmer müssen gute aurale Stimmer sein, auch wenn sie oft elektronische Stimmhilfen verwenden.

Für Flügel brauchen Sie größere Gummikeile zum Dämpfen, während \enquote{\hyperref[upright]{Aufrechte}} kleinere mit Metallgriffen erfordern.
Vier Keile jeden Typs werden ausreichen.
Sie können diese per Versand bestellen oder Ihren Stimmer bitten, den ganzen Satz Werkzeuge, den Sie benötigen, für Sie zu kaufen.

Die verbreitetsten Dämpfungsstreifen sind aus Filz, ungefähr 4ft lang und 5/8`` breit\footnote{ca. 1,22m x 1,6cm}.
Sie werden benutzt, um die 2 Nebensaiten der 3-saitigen Noten in der Oktave zu dämpfen, die für das \enquote{\hyperref[c2_4]{Einstellen des Bezugspunkts}} benutzt wird (s.u.).
Es gibt die Streifen auch als verbundene Gummikeile, aber diese funktionieren nicht genauso gut.
Die Streifen gibt es auch in Gummi, aber Gummi dämpft nicht so gut und ist nicht so stabil wie Filz (die Streifen können sich während des Stimmens verschieben oder herausspringen).
Der Nachteil der Filzstreifen ist, daß sie auf dem Resonanzboden eine Filzfaserschicht hinterlassen, die abgesaugt werden muß.

Ein Stimmhammer hoher Qualität besteht aus einem verlängerbaren Griff, einem an der Spitze des Griffs befestigten Kopf und einem auswechselbaren, in den Kopf geschraubten Einsatz.
Es ist eine gute Idee, einen Stimmwirbel zu haben, den Sie in den Einsatz stecken können, so daß Sie den Einsatz mit Hilfe eines Schraubstocks fest in den Kopf schrauben können.
Ansonsten könnten Sie den Einsatz verkratzen, wenn Sie ihn mit dem Schraubstock fassen.
Wenn der Einsatz nicht fest im Kopf sitzt, wird er sich während des Stimmens lösen.
Die meisten Klaviere erfordern einen Einsatz \#2, es sei denn, das Klavier wurde mit größeren Stimmwirbeln neu besaitet.
Der Standardkopf ist ein 5-Grad-Kopf.
Diese \enquote{5 Grad} sind der Winkel zwischen der Einsatzachse und dem Griff.
Sowohl die Köpfe als auch die Einsätze gibt es in verschiedenen Längen, aber \enquote{Standard-} oder \enquote{mittlere} Länge wird genügen.
 

\label{c2_3_gabel}

Besorgen Sie sich zwei Stimmgabeln -- A440 und C523,3 -- von guter Qualität.
Entwickeln Sie die gute Angewohnheit, sie am schmalen Hals des Griffs zu halten, so daß Ihre Finger nicht die Schwingungen der Stimmgabeln stören.
Klopfen Sie die Spitze der Gabel fest gegen einen muskulösen Teil Ihres Knies und testen Sie die Aushaltezeit (Sustain).
Sie sollte für 10 bis 20 Sekunden deutlich zu hören sein, wenn Sie sie nahe an Ihr Ohr halten.
Die beste Art, die Gabel zu hören, ist, die Spitze des Griffs auf den dreieckigen Knorpel (Tragus) zu setzen, der zur Mitte des Ohrlochs hin herausragt.
Sie können die Lautstärke der Gabel anpassen, indem Sie den Tragus mit dem Ende der Gabel ein- oder auswärts drücken.
Benutzen Sie keine Pfeifen; diese sind zu ungenau.

Ohrenschützer sind eine notwendige Schutzvorrichtung, da Gehörschäden das Berufsrisiko eines Stimmers sind.
Wie weiter unten erklärt wird, ist es notwendig \hyperref[c2_5_infi]{die Tasten hart anzuschlagen} (auf die Tasten zu hämmern -- um den Jargon der Stimmer zu benutzen), um richtig zu stimmen, und die Klangintensität eines solchen Hämmerns kann das Ohr sehr leicht schädigen, was zu Gehörverlust und Tinnitus führt.
 

\subsection{Vorbereitung}
\label{c2_4}

Bereiten Sie das Stimmen vor, indem Sie den Notenständer entfernen, so daß sie an die Stimmwirbel herankommen (Flügel).
Für den folgenden Abschnitt brauchen Sie keine weiteren Vorbereitungen.
Um \enquote{die Bezugspunkte einzustellen} müssen Sie alle Nebensaiten der 3-fachen Saiten in der \enquote{Bezugsoktave} mit den Dämpfungsstreifen dämpfen, so daß, wenn Sie eine Note in dieser Oktave spielen, nur die mittlere Saite vibriert.
Sie werden wahrscheinlich je nach Stimmalgorithmus fast zwei Oktaven dämpfen müssen.
Probieren Sie zunächst den ganzen Stimmalgorithmus aus, um die höchste und die tiefste Note zu bestimmen, die Sie dämpfen müssen.
Dämpfen Sie dann alle Noten dazwischen.
Benutzen Sie das gerundete Ende des Drahtgriffs eines Dämpfungskeils für \enquote{Aufrechte}, um den Filz in den Raum zwischen den äußeren Saiten zweier nebeneinander liegenden Noten zu pressen.
 


