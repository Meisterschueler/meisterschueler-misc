% File: c1iii7b

\subsubsection{Parallele Sets}
\label{c1iii7b}

\textbf{Das Hauptziel von Übungen ist das Aneignen der Technik, was sich für alle Absichten und Zwecke auf Geschwindigkeit, Kontrolle und Klang reduzieren läßt.}
Damit Übungen nützlich sind, müssen sie dafür geeignet sein, Ihre Schwächen zu identifizieren und diese Fähigkeiten zu stärken.
Um das zu erreichen, \textbf{müssen wir einen vollständigen Satz Übungen haben, und sie müssen in einer logischen Reihenfolge angeordnet sein, so daß man leicht eine Übung ausfindig machen kann, die für einen bestimmten Zweck notwendig ist.}
Solch eine Übung muß deshalb auf einem grundlegenden Prinzip des Klavierspielens basieren, das alle Aspekte abdeckt.
Und wie \textit{identifizieren} wir unsere einzelnen Schwächen?
Die Tatsache, daß man etwas nicht spielen kann, sagt einem noch nicht, warum das so ist oder wie man das Problem lösen kann.

\textbf{Das Konzept der parallelen Sets stellt meines Erachtens den Rahmen für einen universellen Satz von Übungen zur technischen Entwicklung zur Verfügung.}
Das kommt daher, daß jede beliebige musikalische Passage aus Kombinationen paralleler Sets konstruiert werden kann (d.h. aus Gruppen von Noten, die unendlich schnell gespielt werden können).
Ich beschreibe im folgenden eine vollständige Gruppe von Übungen für parallele Sets, die alle diese Anforderungen erfüllt.
In \hyperref[c1ii11]{\autoref{c1ii11}} finden Sie eine Einführung zu den parallelen Sets.
Natürlich stellen parallele Sets alleine keinen vollständigen Satz von Übungen dar; Verbindungen, Wiederholungen, Sprünge, Dehnungen usw. werden auch benötigt.
Diese Themen werden hier ebenfalls behandelt.
Offenbar lehrte Louis Plaidy im späten 19. Jahrhundert Übungen, die den Übungen für parallele Sets ähnlich waren.

Alle Übungen für parallele Sets sind HS-Übungen, wechseln Sie deshalb häufiger die Hände.
Sie können sie jedoch jederzeit HT üben und in jeder miteinander vereinbaren Kombination, sogar 2 Noten gegen 3, usw.
Tatsächlich sind diese Übungen vielleicht der beste Weg, um solche ungleichen RH-LH-Kombinationen zu üben.
Probieren Sie zunächst jede der Übungen ein wenig aus, und lesen Sie dann in \hyperref[c1iii7c]{Abschnitt (c)}, wie man sie benutzt.
Wenn man sie erweitert, gibt es eine unendliche Zahl (was sie sein sollten, wenn sie vollständig sind), so daß Sie sie niemals alle üben werden.
Sie werden sowieso niemals alle benötigen, und wahrscheinlich sind mehr als die Hälfte davon redundant.
Benutzen Sie diese Übungen nur, wenn Sie sie brauchen (Sie werden sie \textit{ständig} brauchen!).
Zu diesem Zeitpunkt ist es nur erforderlich, daß Sie genügend vertraut mit ihnen werden, so daß Sie sofort eine bestimmte Übung abrufen können, wenn es notwendig wird.
So verschwenden Sie niemals Zeit mit unnötigen Übungen.

Ich wiederhole es noch einmal: \textbf{Übungen für parallele Sets sind nicht dafür gedacht, daß man sie wie \hyperref[c1iii7h]{Hanons Übungen}\index{Hanons Übungen} jeden Tag übt; sie sollen dazu benutzt werden, Ihre Schwächen zu diagnostizieren und sie zu korrigieren.}
Wenn die Schwierigkeiten beseitigt sind, benötigen Sie diese Übungen für parallele Sets nicht mehr.

\textbf{Die Übungen für parallele Sets werden stets Ihre Technik auf die Probe stellen.
Wenn Sie ein Anfänger ohne Technik sind, werden sie wahrscheinlich alle nicht ausführen können}.
Es wird für alle Übungen unmöglich sein, sie mit den erforderlichen Geschwindigkeiten zu spielen.
Die meisten Schüler werden zunächst keine Ahnung haben, wie man sie korrekt spielt.
Wenn Sie die Übungen noch nie durchgeführt haben, wäre es sehr hilfreich, wenn Sie sich ein paar von jemandem zeigen lassen könnten.
Mittelstufenschüler mit 2 bis 5 Jahren Unterricht sollten in der Lage sein, mehr als die Hälfte davon zufriedenstellend zu spielen.
Somit liefern diese Übungen ein Mittel zum Messen Ihres Fortschritts.
Das ist die völlige Entwicklung der Technik und schließt deshalb die Klangkontrolle und das musikalische Spielen ein, wie in Kürze erklärt wird.
Fortgeschrittene Schüler werden die Übungen immer noch benötigen, aber -- anders als noch in der Entwicklung befindliche Schüler -- nur kurz, oftmals nur für ein paar Sekunden während des Übens.


\paragraph{Übung \#1}
\label{c1iii7b1}

Diese Übung baut die grundlegende Bewegung auf, die für alle folgenden Übungen benötigt wird.
\textbf{Spielen Sie nur eine Note}, z.B. Finger 1 (Daumen der RH oder LH), als vier Wiederholungen: 1111.
In dieser Übung lernen wir nur, wie man eine \enquote{Sache} rasch wiederholt; dann ersetzen wir die \enquote{Sache} durch ein paralleles Set, so daß wir Zeit sparen können, indem wir so viele parallele Sets wie möglich innerhalb einer kurzen Zeitspanne spielen.
Denken Sie daran, daß ein wichtiger Grund für das Durchführen von Übungen das Zeitsparen ist.

Sie können das 1111 als Quadrupel gleicher Stärke spielen oder als Einheiten eines 4/4- oder 2/4-Takts.
\textbf{Die Idee ist, sie so schnell zu spielen wie Sie können, bis zu Geschwindigkeiten von mehr als einem Quadrupel je Sekunde.}
Wenn Sie ein Quadrupel zu Ihrer Zufriedenheit spielen können, versuchen Sie zwei: 1111,1111.
Das Komma repräsentiert eine Pause von willkürlicher Länge, die verkürzt werden sollte, wenn Sie Fortschritte machen.
Wenn Sie zwei schaffen, hängen Sie vier in schneller Folge aneinander: 1111,1111,1111,1111.
Sie \enquote{meistern} diese Übung, wenn sie vier Quadrupel in Folge mit ungefähr einem Quadrupel je Sekunde und ohne Pause zwischen den Quadrupeln spielen können.
Spielen Sie sie leise, entspannt und nicht staccato, wie weiter unten detaillierter erklärt wird.
Die Übung gilt als bewältigt, wenn Sie die Quadrupel so lange und so schnell spielen können wie Sie möchten und zwar mit völliger Kontrolle und ohne Ermüdung.
Diese scheinbar triviale Bewegung ist viel wichtiger als sie auf den ersten Blick erscheint, weil sie die Grundlage für alle geschwindigkeitsbezogenen Bewegungen ist.
Das wird offensichtlich, wenn wir zu parallelen Sets mit mehreren Fingern kommen, wie z.B. solchen in schnellen Alberti-Begleitungen oder Tremolos.
Deshalb widmen wir im folgenden dieser Übung so viele Absätze.

Wenn sich beim Aneinanderhängen der Quadrupel Streß aufbaut, arbeiten Sie weiter daran, bis Sie vier Quadrupel schnell und streßfrei spielen können.
Beachten Sie, daß jeder Teil Ihres Körpers einbezogen sein muß: Finger, Hand, Arm, Schulter usw., nicht nur die Finger.
Das bedeutet nicht, daß jeder Teil Ihres Körpers sich in einem sichtbaren Maß bewegen muß -- sie mögen stillstehend erscheinen, aber sie müssen teilhaben.
Ein großer Teil des \enquote{Einbeziehens} wird das bewußte \hyperref[c1ii14]{Entspannen}\index{Entspannen} sein, weil das Gehirn dazu tendiert, sogar für die kleinste Aufgabe zu viele Muskeln zu benutzen.
Versuchen Sie, nur die notwendigen Muskeln für die Bewegung zu isolieren, und entspannen Sie alle anderen.
Die endgültige Bewegung mag den Anschein erwecken, daß sich nur der Finger bewegt.
Aus ein paar Metern Entfernung werden wenige Menschen eine Bewegung von 1 mm erkennen; wenn jeder Teil Ihres Körpers sich weniger als 1 mm bewegt, kann sich das leicht zu den mehreren mm aufaddieren, die für den Anschlag notwendig sind, auch wenn dabei kein Finger bewegt wird.
Experimentieren Sie deshalb  für ein optimales Spielen mit unterschiedlichen Positionen Ihrer Hand, des Handgelenks usw.

Wenn Ihre Geschwindigkeit steigt, werden die Finger, Hände und Arme automatisch ideale Positionen einnehmen; ansonsten werden Sie nicht in der Lage sein, mit diesen Geschwindigkeiten zu spielen.
Diese Positionen gleichen denen berühmter Pianisten während des Spielens bei einem Konzert -- letzten Endes ist das der Grund, daß sie es spielen können.
Deshalb ist es wichtig, beim Besuch eines Konzerts sein Opernglas mitzubringen und sich die Details der Bewegungen professioneller Pianisten anzusehen.
Die Positionen und Bewegungen der Hände und des Körpers sind für die unten vorgestellten fortgeschrittenen Übungen besonders wichtig.
Beginnen Sie deshalb \textit{jetzt} damit, das Erkennen dieser Verbesserungen zu trainieren.

Anfänger werden im ersten Jahr ihres Klavierunterrichts nicht in der Lage sein, ein Quadrupel je Sekunde zu spielen, und sollten mit geringeren Geschwindigkeiten zufrieden sein.
Zwingen Sie sich nicht, mit Geschwindigkeiten zu üben, die Sie nicht handhaben können.
Periodische kurze Exkursionen zu Ihrer schnellsten Geschwindigkeit sind jedoch für Erkundungszwecke notwendig.
Sogar Schüler mit mehr als fünf Jahren Unterricht werden Teile dieser Übungen schwierig finden.
Wenn Sie das hier zum ersten Mal lesen, üben Sie \hyperref[c1iii7b1]{Übung \#1}\index{Übung \#1} für eine Weile und dann \hyperref[c1iii7b2]{Übung \#2}\index{Übung \#2} (s.u.).
Wenn Sie mit \#2 Probleme bekommen, dann können Sie diese lösen, indem Sie wieder \#1 üben (Versuchen Sie es, es funktioniert!).
Sie können auch einen kurzen Blick auf die anderen Übungen werfen.
Es ist aber nicht notwendig, sie jetzt schon alle durchzuführen.
Wenn Sie mit dem Üben schwieriger Kompositionen beginnen, wird es oft genug notwendig werden, die Übungen durchzuführen, so daß Sie genügend Gelegenheit dazu haben werden.

\textbf{Üben Sie, bis jeglicher Streß verschwindet und Sie fühlen können, wie die Schwerkraft Ihren Arm nach unten zieht.}
Sobald sich Streß aufbaut, werden Sie den Zug durch die Schwerkraft nicht mehr fühlen.
Versuchen Sie nicht zu viele Quadrupel auf einmal, wenn Sie nicht das Gefühl haben, daß Sie die völlige Kontrolle haben.
Zwingen Sie sich nicht, unter Streß weiter zu üben, weil Üben unter Streß zur Gewohnheit werden kann, bevor man es merkt.
Wenn Sie unter Streß üben, werden Sie sogar anfangen langsamer zu werden.
Das ist ein deutliches Zeichen dafür, daß Sie entweder langsamer werden oder die Hände wechseln müssen.
Sie verbessern sich am schnellsten bei den höchsten Geschwindigkeiten, mit denen Sie zurechtkommen.
Spielen Sie zuerst nur ein Quadrupel sehr gut, bevor Sie ein weiteres hinzufügen -- auf diese Weise werden Sie schneller vorankommen, als wenn Sie viele Quadrupel auf einmal herunterspielen.
Bei vier Quadrupeln kann man aufhören, denn wenn man vier spielen kann, dann kann man üblicherweise eine beliebige Anzahl hintereinander spielen.
Wie viele man \textit{genau} benötigt, bevor man eine unendliche Anzahl in Folge spielen kann, hängt jedoch von jedem einzelnen ab.
Wenn Sie bereits nach zwei Quadrupeln eine unendliche Anzahl mit jeder gewünschten Geschwindigkeit spielen können, dann haben Sie den Test für Übung \#1 ebenfalls bestanden, und Sie müssen sie \textit{niemals} mehr durchführen!
Diese Übungen unterscheiden sich völlig von den \hyperref[c1iii7h]{Hanon-Übungen}\index{Hanon-Übungen}, die man jeden Tag wiederholen muß; sobald Sie den Test bestanden haben, machen Sie mit etwas schwierigerem weiter -- das ist Fortschritt.
An den ersten paar Übungstagen sollten Sie während des Übens einige Verbesserungen bemerken, da Sie schnell neue Bewegungen lernen und falsche eliminieren.
Um weitere Fortschritte zu machen, müssen Sie wegen des notwendigen Wachstums von Muskeln und Nerven die \hyperref[c1ii15]{PPI}\index{PPI} benutzen.
Anstatt \textit{während} des Übens auf die Geschwindigkeit zu drängen, warten Sie, bis die Hand die Schnelligkeit von selbst entwickelt, so daß Sie schneller spielen, wenn Sie das \textit{nächste Mal} üben; das kann geschehen, wenn Sie die Hände wechseln oder wenn Sie am folgenden Tag erneut üben.
\textbf{Nach der ersten oder zweiten Woche wird sich die Schnelligkeit überwiegend zwischen den Übungen und nicht während des Übens entwickeln}; lassen Sie sich deshalb nicht entmutigen, wenn Sie sich scheinbar auch nach einem harten Training nicht viel verbessern -- das ist normal.
Der größte Teil des Wachstums der Muskeln und Nerven scheint im Schlaf zu geschehen, wenn die Ressourcen des Körpers nicht für die Tagaktivitäten benötigt werden.
Deshalb sollten Sie nach einem guten Nachtschlaf die beste PPI bemerken.
Beim Versuch, während des Übens eine sichtbare Verbesserung zu erzielen, zuviel zu üben, ist eine der Hauptursachen für \hyperref[c1iv2b]{Geschwindigkeitsbarrieren}\index{Geschwindigkeitsbarrieren}, \hyperref[c1iii10hand]{Verletzungen}\index{Verletzungen} und Streß.
\textbf{Ihre Aufgabe während des Übens ist die Konditionierung der Hand für eine maximale PPI.}
Die Konditionierung erfordert nur ungefähr hundert Wiederholungen; darüber hinaus beginnt der Gewinn je Wiederholung zu sinken, und die Konditionierung für die PPI wird nur unbedeutend gesteigert.

Es geht hier um das Aneignen von Technik, nicht um Muskelaufbau.
\textbf{Technik bedeutet Musik machen, und diese Übungen sind für die Entwicklung eines musikalischen Spielens wertvoll.}
Hämmern Sie nicht drauflos wie ein Wilder.
Wenn Sie den Klang einer Note nicht kontrollieren können, wie wollen Sie ihn dann bei mehreren Noten kontrollieren?
Ein wesentlicher Trick, um den Klang zu kontrollieren, ist, leise zu üben.
Indem Sie leise spielen, bringen Sie sich selbst aus dem Übungszustand, in dem Sie die Natur des Klangs völlig ignorieren, draufloshämmern und nur versuchen, die Wiederholungen zu erreichen.
Drücken Sie die Taste völlig nieder, und halten Sie sie einen Moment unten (ganz kurz -- nur den Bruchteil einer Sekunde).
Das stellt sicher, daß der Fänger den Hammer greift und die Schwingungen stoppt, die dieser aufnimmt, wenn er von den Saiten abspringt.
Wenn diese Schwingungen nicht beseitigt werden, kann man den nächsten Anschlag nicht kontrollieren.
\textbf{Lesen Sie Abschnitt III.4b über das \hyperref[c1iii4b]{Spielen mit flachen Fingern}\index{Spielen mit flachen Fingern}; dieser Abschnitt ist obligatorischer Stoff, bevor Sie eine der Übungen für parallele Sets ernsthaft ausführen.}

\textbf{Halten Sie den spielenden Finger zu jeder Zeit so nah wie möglich über der Taste}, um die Geschwindigkeit und Genauigkeit zu steigern und den Klang zu kontrollieren.
Wenn der Finger die Taste nicht hin und wieder berührt, verlieren Sie die Kontrolle.
Lassen Sie die Finger nicht die ganze Zeit auf den Tasten ruhen, sondern berühren Sie die Tasten so leicht Sie können, so daß Sie wissen, wo sie sind.
Das wird Ihnen ein zusätzliches Gefühl dafür geben, wo all die anderen Tasten sind, und wenn die Zeit kommt sie zu spielen, werden Ihre Finger nicht die falschen Tasten anschlagen, weil Sie wissen, wo die richtigen Tasten sind.
\textbf{Ermitteln Sie den minimalen Tastenhub, der für die Wiederholung notwendig ist, und üben Sie, mit dem geringstmöglichen Tastenhub zu spielen.}
Der Tastenhub ist für Klaviere größer als für Flügel.
Mit kleineren Tastenhüben können Sie schnellere Geschwindigkeiten erzielen.

\textbf{Es ist wichtig, das Handgelenk in die Bewegung für die Wiederholung einzubeziehen.}
Das Handgelenk bestimmt alle drei Ziele, die wir verfolgen: Geschwindigkeit, Kontrolle und Klang.
Erinnern Sie sich daran, daß \enquote{das Handgelenk einbeziehen} keine übertriebene Bewegung des Handgelenks bedeutet; seine Bewegung ist eventuell nicht wahrnehmbar, da Sie den Impuls und nicht die Bewegung des Handgelenks benötigen.

\textbf{Arbeiten Sie an der Kontrolle und dem Klang, anstatt immer an der Geschwindigkeit zu arbeiten.
Wiederholungen, die für die Kontrolle und den Klang geübt werden, zählen ebenfalls für die Konditionierung der Geschwindigkeit.}
Sowohl Kontrolle als auch Geschwindigkeit erfordern dieselbe Fertigkeit: Genauigkeit.
Schnelles Üben unter Streß konditioniert lediglich für ein gestreßtes Spielen, was die Bewegung sogar verlangsamt.

Sie müssen ständig nachforschen und experimentieren.
Ist der Klang unterschiedlich, wenn Sie Ihre Fingerspitze an einem Punkt halten oder wenn Sie sie leicht über die Taste gleiten lassen?
Üben Sie, Ihre Finger vorwärts (zum Klavier hin) und rückwärts (zu Ihrem Körper hin) gleiten zu lassen.
Der Daumen ist eventuell der am leichtesten gleitende Finger.
\textbf{Spielen Sie mit der Spitze des Daumens, nicht mit dem Gelenk}; das wird Sie in die Lage versetzen, den Daumen gleiten zu lassen und Ihre Hand oder Ihr Handgelenk anzuheben, und somit die Chancen verringern, daß die anderen Finger aus Versehen einige Tasten anschlagen.
Mit der Spitze zu spielen erhöht auch die effektive Reichweite und Geschwindigkeit der Daumenbewegung, d.h. bei unveränderter Bewegung des Daumens bewegt sich die Spitze weiter und schneller als das Gelenk.
Wenn Sie wissen, wie man die Finger gleiten läßt, können Sie auch dann mit Selbstvertrauen spielen, wenn die Tasten schlüpfrig oder vom Schwitzen feucht sind.
\textbf{Lassen Sie Ihre Fähigkeit zu spielen nicht von der Griffigkeit der Tastenoberfläche abhängig werden, da diese nicht immer gegeben ist.}
Mit angehobenem Handgelenk zu spielen führt dazu, daß die Finger zu Ihnen hin gleiten.
Wenn Sie das Handgelenk absenken, tendieren die Finger dazu, von Ihnen weg zu gleiten, besonders die Finger 2 bis 5.
Üben Sie jede dieser gleitenden Bewegungen: Üben Sie für eine Weile alle fünf Finger mit angehobenem Handgelenk; wiederholen Sie es dann mit gesenktem Handgelenk.
Bei einer mittleren Höhe des Handgelenks werden die Finger nicht gleiten, auch nicht wenn die Tasten schlüpfrig sind.

Experimentieren Sie mit der Kontrolle des Klangs, indem Sie absichtlich etwas gleiten.
Gleiten steigert die Kontrolle, weil man mit einer größeren Bewegung einen geringen Tastenweg erzeugt.
Das Ergebnis ist, daß jegliche Fehler in der Bewegung um das Verhältnis von Tastenweg zu totaler Bewegung vermindert werden, was immer kleiner als 1 ist.
Deshalb können Sie mittels Gleiten gleichmäßigere Quadrupel spielen, als wenn Sie gerade herunterkommen.
Sie können auch leiser spielen.
Das Gleiten vereinfacht auch die Fingerbewegung, weil der Finger nicht gerade herunter kommen muß -- jede Bewegung mit einer abwärts gerichteten Komponente wird ausreichen, was Ihre Optionen vermehrt.

Wiederholen Sie es mit allen anderen Fingern.
Schüler, die diese Übung das erste Mal ausführen, sollten finden, daß einige Finger (typischerweise 4 und 5) schwieriger als die anderen sind.
Das ist ein Beispiel, wie man diese Übungen als Diagnosewerkzeug benutzt, um schwache Finger aufzuspüren.

Das \hyperref[c2_7_hamm]{Intonieren des Klaviers (der Hämmer)}\index{Intonieren des Klaviers (der Hämmer)} ist für die richtige Ausführung dieser Übungen entscheidend.
Das gilt sowohl für den Erwerb neuer Fertigkeiten als auch für das Vermeiden des unmusikalischen Spielens.
Mit abgenutzten Hämmern, die intoniert werden müssen, ist es unmöglich, leise (oder kraftvolle oder tiefe) musikalische Töne zu erzeugen.


\paragraph{Übung \#2}
\label{c1iii7b2}

\textbf{Übungen für parallele Sets mit 2 Fingern:} spielen Sie 12 (RH: Daumen auf C, gefolgt vom Zeigefinger auf D) so schnell Sie können, wie eine Vorschlagsnote.
Die Idee ist, es schnell aber völlig kontrolliert zu spielen.
Offensichtlich werden hier die Methoden von Abschnitt I und II gebraucht.
Wenn z.B. die RH eine Übung leicht ausführen kann aber die entsprechende Übung für die LH schwierig ist, \hyperref[c1ii20]{benutzen Sie die RH um die LH zu unterrichten}\index{benutzen Sie die RH um die LH zu unterrichten}.
Üben Sie sowohl mit dem Schlag auf der 1 als auch mit dem Schlag auf der 2.
Wenn das zufriedenstellend ist, spielen Sie ein Quadrupel wie in Übung \#1: 12,12,12,12.
Wenn Sie Schwierigkeiten damit haben, ein Quadrupel aus parallelen Sets mit 12 zu beschleunigen, spielen Sie die beiden Noten gleichzeitig als \enquote{Akkord} und üben Sie dieses Akkord-Quadrupel so wie Sie das aus einer Note bestehende Quadrupel in \hyperref[c1iii7b1]{Übung \#1}\index{Übung \#1} geübt haben.
Bringen Sie das Quadrupel wieder auf Geschwindigkeit, d.h. ungefähr ein Quadrupel pro Sekunde.
Verbinden Sie dann vier Quadrupel in Folge.
Wiederholen Sie die ganze Übung jeweils mit 23, 34 und 45.
Dann nach unten: 54, 43 usw.
\textbf{Alle Anmerkungen darüber, wie man für Übung \#1 übt, gelten hier ebenfalls.}

Bei dieser und den folgenden Übungen sind die Anmerkungen der vorangegangenen Übungen fast immer auf die nachfolgenden Übungen anwendbar und werden im allgemeinen nicht wiederholt.
Auch werde ich nur repräsentative Mitglieder einer Familie von Übungen auflisten und es dem Leser überlassen, alle anderen Familienmitglieder herauszufinden.
Die gesamte Anzahl der Übungen ist viel größer als man am Anfang denkt.
Wenn man z.B. versuchen würde, verschiedene Übungen für parallele Sets HT zu kombinieren, würde die Zahl der Möglichkeiten schnell die Vorstellungskraft übersteigen.
Für Anfänger, die Schwierigkeiten mit dem HT-Spielen haben, könnten diese Übungen die beste Art darstellen, das HT-Spielen zu üben.

Alle Noten eines parallelen Sets müssen so schnell wie möglich gespielt werden, weil parallele Sets hauptsächlich für die Entwicklung der Geschwindigkeit benutzt werden.
Ein Zweck paralleler Sets ist, das Gehirn das Konzept der extremen -- fast bis zur unendlichen -- Geschwindigkeit zu lehren.
Es stellt sich heraus, daß sobald das Gehirn sich an eine bestimmte Maximalgeschwindigkeit gewöhnt hat, alle langsameren Geschwindigkeiten einfacher auszuführen sind.

Führen Sie die Übungen zunächst nur unter Benutzung der weißen Tasten aus.
Haben Sie alle Übungen mit den weißen Tasten durchgeführt, arbeiten Sie mit ähnlichen Übungen, bei denen Sie die schwarzen Tasten einschließen.

Am Anfang werden Sie vielleicht in der Lage sein, die zwei Noten hintereinander sehr schnell aber ohne viel unabhängige Kontrolle zu spielen.
Sie können am Anfang \enquote{schummeln} und die Geschwindigkeit erhöhen, indem Sie die beiden Finger \enquote{phasenkoppeln}, z.B. die Finger in einer festen Position halten und einfach die Hand senken, um die zwei Noten zu spielen.
Erinnern Sie sich daran, daß der \hyperref[c1iv2a]{Phasenwinkel}\index{Phasenwinkel} der Abstand zwischen aufeinanderfolgenden Fingern beim parallelen Spielen ist.
\textbf{Aber irgendwann müssen Sie mit Unabhängigkeit der Finger spielen.
Die anfängliche Phasenkoppelung wird nur benutzt, um schnell auf Geschwindigkeit zu kommen.
Das ist ein Grund, warum einige Lehrer das parallele Spielen nicht lehren, weil sie der Meinung sind, daß paralleles Spielen mit Phasenkopplung gleichzusetzen ist, was die Musik zerstören würde.}
Der Grund für dieses Problem ist, daß beide Finger nach dem Spielen mit Phasenkopplung auf ihren Tasten liegenbleiben und die beiden Noten sich überlappen.
Es ist genauso wichtig, den Finger zu einem bestimmten Zeitpunkt zu heben, wie ihn zu einem bestimmten Zeitpunkt zu senken.
Zum Spielen mit unabhängigen Fingern muß der Finger genau dann gehoben werden, wenn der zweite Finger spielt, so daß aufeinanderfolgende Noten eindeutig getrennt werden.
Den Finger zu senken und ihn unten zu behalten führt dazu, daß der Fänger den Hammer greift und ihn so unter Kontrolle hält.
Wenn man dann den Finger hebt, kommt der Dämpfer herunter und stoppt den Ton, was die Dauer des Tons kontrolliert.
Deshalb werden die parallelen Sets in Quadrupeln gespielt; am Ende des Quadrupels muß man den letzten Finger heben und kontrolliert somit das Ende des Quadrupels präzise.

Können Sie erst einmal entspannte, schnelle parallele Sets spielen, müssen Sie langsamer werden und daran arbeiten, jede einzelne Note korrekt zu spielen.
Anfänger werden Schwierigkeiten damit haben, die Finger zum richtigen Zeitpunkt anzuheben, um die Notendauer zu kontrollieren.
In diesem Fall können Sie entweder abwarten, bis sich Ihre Technik weiterentwickelt oder die Übungen zum Anheben der Finger unten in \hyperref[c1iii7d]{Abschnitt d)} durchführen.


\paragraph{Übung \#3}
\label{c1iii7b3}

\textbf{Größere parallele Sets:} z.B. 123 und seine Familie, 234 usw.
Wiederholen Sie alle Prozeduren wie in \hyperref[c1iii7b2]{Übung \#2}\index{Übung \#2}.
Arbeiten Sie dann mit der 1234-Gruppe und schließlich mit den 12345-Sets.
Bei diesen großen Sets müssen Sie vielleicht die Wiederholungsgeschwindigkeit der Quadrupel ein wenig reduzieren.
\textbf{Die Zahl der möglichen Übungen ist für diese größeren Sets sehr groß.}
Der Schlag kann auf jeder Note sein, und Sie können mit jeder Note beginnen.
123 kann z.B. als 231 und 312 geübt werden.
Wenn man abwärts spielt, kann die 321 als 213 oder 132 gespielt werden; alle sechs sind unterschiedlich, weil Sie feststellen werden, daß einige leicht und andere schwierig sind.
Wenn man die Schlagvarianten einschließt, gibt es bereits für drei Finger auf den weißen Tasten 18 Übungen.


\paragraph{Übung \#4}
\label{c1iii7b4}

\textbf{Erweiterte parallele Sets:} beginnen Sie mit den zweinotigen Sets 13, 24 usw. (die Terz-Gruppe).
Diese Sets schließen die Gruppen vom Typ 14 (Quarte) und 15 (Quinte und Oktave) ein.
Dann gibt es die dreinotigen erweiterten parallelen Sets: 125-, 135-, 145-Gruppen (Quinte und Oktave).
Hier haben Sie eventuell mehrere Möglichkeiten für die mittlere der drei Noten.
Außerdem gibt es erweiterte Sets, die mit 12 gespielt werden: Terzen, Quarten, Quinten usw.; diese können auch mit 13 usw. gespielt werden.


\paragraph{Übung \#5}
\label{c1iii7b5}

\textbf{Die zusammengesetzten parallelen Sets:} 1.3,2.4, wobei 1.3 und 2.4 jeweils zwei gleichzeitig gespielte Noten, z.B. CE, darstellen.
Üben Sie dann die 1.4,2.5 Gruppe.
Ich habe oft Sets gefunden, die leicht aufwärts aber schwer abwärts zu spielen sind oder umgekehrt.
1.3,2.4 ist für mich z.B. einfacher als 2.4,1.3.
\textbf{Diese zusammengesetzten Sets erfordern einiges an Geschicklichkeit.}
Solange Sie nicht mindestens einige Jahre Unterricht hatten, sollten Sie nicht erwarten, sie mit einer gewissen Fertigkeit spielen zu können.

Das ist das Ende der wiederholenden Quadrupel-Übungen, die auf \hyperref[c1iii7b1]{Übung \#1}\index{Übung \#1} basieren.
Im Prinzip sind die Übungen \#1 bis \#5 die einzigen Übungen, die Sie benötigen, weil man sie benutzen kann, um die im folgenden besprochenen parallelen Sets zu konstruieren.
Die Übungen \#6 und \#7 sind zu komplex, um sie in schnellen Quadrupeln zu wiederholen.


\paragraph{Übung \#6}
\label{c1iii7b6}

\textbf{Komplexe parallele Sets:} Diese werden am besten einzeln geübt anstatt als schnelle Quadrupel.
In den meisten Fällen sollten sie in einfachere parallele Sets aufgeteilt werden, die als Quadrupel geübt werden können -- zumindest am Anfang.
\enquote{Abwechselnde Sets} sind vom Typ 1324, und \enquote{gemischte Sets} sind vom Typ 1342, 13452 usw., d.h. Mischungen aus abwechselnden und normalen Sets.
Es gibt natürlich eine große Zahl davon.
\textbf{Die meisten der komplexen parallelen Sets, die technisch wichtig sind, können in Bachs Unterrichtsstücken gefunden werden, besonders in den \hyperref[c1iii20]{zweistimmigen Inventionen}\index{zweistimmigen Inventionen}} (s. Abschnitt III.20).
Deshalb sind Bachs Unterrichtsstücke -- im Gegensatz zu \hyperref[c1iii7h]{Hanon}\index{Hanon} -- einige der besten Übungsstücke für das Erwerben der Technik.


\paragraph{Übung \#7}
\label{c1iii7b7}

Üben Sie nun \textbf{verbundene parallele Sets}, z.B. 1212, die eine oder mehrere Verbindungen enthalten.
Das kann entweder ein Triller (CDCD) oder ein Lauf (CDEF, bei dem Sie den \hyperref[c1iii5a]{Daumen übersetzen}\index{Daumen übersetzen} müssen) sein.
Der 1212 Triller unterscheidet sich von \hyperref[c1iii7b2]{Übung \#2}\index{Übung \#2}, weil das 12 Intervall in jener Übung so schnell wie möglich gespielt werden muß, während das nachfolgende 21 Intervall langsamer sein kann.
Hier muß das Intervall zwischen den Noten immer das gleiche sein.
Nun können diese Sets nicht unendlich schnell gespielt werden, weil die Geschwindigkeit durch Ihre Fähigkeit, die Sets zu verbinden, begrenzt ist.
Das Ziel ist hier immer noch Geschwindigkeit -- wie schnell Sie sie akkurat und entspannt spielen können und wie viele Sie zusammenbinden können.
Das ist eine Übung, um zu lernen, wie man Verbindungen spielt.
Spielen Sie während einer Bewegung der Hand so viele Noten wie möglich.
Üben Sie z.B., 1212 mit einer Abwärtsbewegung der Hand zu spielen.
Üben Sie dann, zwei davon mit einer Abwärtsbewegung zu spielen usw., bis sie vier aufeinanderfolgend mit einer Bewegung schaffen.

\textbf{Für ein schnelles Spielen sind die ersten beiden Noten die wichtigsten}; sie müssen mit der richtigen Geschwindigkeit begonnen werden.
Es mag hilfreich sein, nur die ersten beiden Noten phasengekoppelt zu spielen, um sicherzustellen, daß diese korrekt anfangen.
Wenn die ersten beiden Noten mit hoher Geschwindigkeit begonnen werden, dann folgt der Rest meistens leichter.

\textbf{Benutzen Sie bei Sets, die den Daumen beinhalten, den \hyperref[c1iii5a]{Daumenübersatz}\index{Daumenübersatz}, um sie zu verbinden, außer in besonderen Situationen, in denen der Daumenuntersatz benötigt wird (sehr wenige).}
Erforschen Sie verschiedene Verbindungsbewegungen, um zu sehen, welche am besten funktionieren.
Eine kleine \hyperref[Rollung]{Rollung}\index{Rollung} des Handgelenks ist eine nützliche Bewegung.
Um Sets zu verbinden, die nicht den Daumen einbeziehen, kreuzen Sie fast immer darüber, nicht darunter.
Viele dieser daumenlosen Überkreuzbewegungen sind jedoch von fragwürdigem Wert, weil man sie selten benötigt.

\textbf{Verbundene parallele Sets sind das Hauptübungselement in Bachs zweistimmigen Inventionen.}
Halten Sie deshalb in diesen Inventionen nach einigen der einfallsreichsten und technisch wichtigsten verbundenen parallelen Sets Ausschau.
Wie in \hyperref[c1iii19c]{\autoref{c1iii19c}} erklärt, ist es für viele Schüler oft extrem schwierig, bestimmte Kompositionen von Bach auswendig zu lernen und sie jenseits einer bestimmten Geschwindigkeit zu spielen.
Deshalb werden Stücke von Bach oft nicht so gerne gespielt, und diese wertvolle Quelle für das Erwerben der Technik wird nur begrenzt benutzt.
Wenn man sie jedoch im Hinblick auf die parallelen Sets analysiert und gemäß der Methoden dieses Buchs lernt, sind solche Kompositionen im allgemeinen leicht zu lernen.
Deshalb sollte dieses Buch die Beliebtheit der Bachstücke in hohem Maß steigern.
Sehen Sie dazu in \hyperref[c1iii19c]{\autoref{c1iii19c}} weitere Erklärungen darüber, wie man Bach übt.

Die nahezu unendliche Zahl an notwendigen Übungen für parallele Sets zeigt, wie beklagenswert unzulänglich die älteren Übungen sind (z.B. \hyperref[c1iii7h]{Hanon}\index{Hanon} -- ich werde Hanon als einen gattungsmäßigen Vertreter davon benutzen, was hier als die \enquote{falsche} Art von Übung angesehen wird).
Es gibt jedoch einen Vorteil der Übungen vom Hanon-Typ: Sie beginnen mit den am meisten anzutreffenden Fingersätzen und den leichtesten Übungen, d.h. sie sind gut geordnet.
Die Chancen sind jedoch fast 100\%, daß sie wenig hilfreich sind, wenn man in einem beliebigen Musikstück auf einen schwierigen Abschnitt trifft.
Das Konzept der parallelen Sets erlaubt uns, die einfachste mögliche Serie von Übungen zu identifizieren, die einen vollständigeren Satz bilden, der auf praktisch alles anwendbar ist, dem man begegnen kann.
Sobald diese Übungen jedoch ein wenig komplex werden, wird ihre Anzahl unüberschaubar groß.
Wenn Sie zur Komplexität selbst der einfachsten Hanon-Übung kommen, wird die Zahl der möglichen Übungen für parallele Sets widerspenstig groß.
Sogar Hanon erkannte diese Unzulänglichkeit und schlug Variationen vor, wie z.B. die Übungen in allen möglichen Transpositionen zu üben.
Das ist sicher hilfreich, aber es fehlen immer noch ganze Kategorien, wie z.B. \hyperref[c1iii7b1]{Übung \#1}\index{Übung \#1} und \hyperref[c1iii7b2]{Übung \#2}\index{Übung \#2} (die grundlegendsten und nützlichsten) oder die unglaublichen Geschwindigkeiten, die wir ohne weiteres mit den Übungen für parallele Sets erreichen können.
\textbf{Beachten Sie, daß die Übungen \#1 bis \#4 einen kompletten Satz von rein parallelen Übungen (ohne Verbindungen) bilden.}
Intervalle, die größer sind als das, was man als Akkord erreichen kann, fehlen auf der Liste der hier beschriebenen parallelen Sets, weil sie nicht unendlich schnell gespielt werden können und den Sprüngen zugeordnet werden müssen.
Methoden für das \hyperref[c1iii7f]{Üben von Sprüngen}\index{Üben von Sprüngen} werden weiter unten in Abschnitt (f) besprochen.

Es ist leicht, Hanon zu erstaunlichen Geschwindigkeiten zu bringen, indem man die Methoden dieses Buchs benutzt.
Sie könnten es zum Spaß versuchen, werden sich aber schnell fragen, wozu es gut sein soll.
Sogar diese hohen Geschwindigkeiten können nicht an das heranreichen, was man ohne weiteres mit parallelen Sets erreichen kann, weil jede Hanon-Übung mindestens eine Verbindung enthält und deshalb nicht unendlich schnell gespielt werden kann.
\textbf{Das ist klarerweise der größte Vorteil der Übungen für parallele Sets: Es gibt keine Geschwindigkeitsbegrenzung, weder in der Theorie noch in der Praxis, und gestattet Ihnen deshalb, die Geschwindigkeiten in ihrem gesamten Umfang ohne Beschränkungen und ohne Streß zu erforschen.}
Wie bereits gesagt, ist die Hanon-Reihe aufsteigend nach Schwierigkeit geordnet, und diese Steigerung wird hauptsächlich durch das Einschließen von mehr Verbindungen und schwierigeren parallelen Sets erzeugt.
Bei den Übungen für parallele Sets werden diese einzelnen \enquote{Elemente der Schwierigkeit} ausdrücklich auseinandergehalten, so daß man sie getrennt üben kann.

Stellen Sie sich zur Verdeutlichung der Nützlichkeit dieser Übungen vor, daß Sie einen zusammengesetzten vierfingrigen Triller basierend auf \hyperref[c1iii7b5]{Übung \#5}\index{Übung \#5} üben möchten (z.B. C.E,D.F,C.E,D.F, . . .).
Indem Sie die Übungen in der Reihenfolge von \#1 bis \#7 befolgen, haben Sie nun ein schrittweises Rezept für das Diagnostizieren Ihrer Schwierigkeiten und um sich diese Fertigkeit anzueignen.
Stellen Sie zunächst sicher, daß Ihre zweinotigen Akkorde gleichmäßig sind, indem Sie die Übungen \#1 und \#2 anwenden.
Versuchen Sie dann 1.3,2 und dann 1.3,4.
Wenn diese zufriedenstellend sind, dann versuchen Sie 1.3,2.4.
Arbeiten Sie dann an den umgekehrten Richtungen: 2.4,1 und 2.4,3. Und schließlich 2.4,1.3.
Der Rest sollte offensichtlich sein, wenn Sie bis hierhin gelesen haben.
Das kann ein hartes Training sein; denken Sie deshalb daran, oft die Hände zu wechseln, bevor eine Ermüdung einsetzt.

\textbf{Es soll hier noch einmal betont werden, daß in den Methoden dieses Buchs kein Platz für stupide wiederholende Übungen ist.}
Solche Übungen haben einen weiteren heimtückischen Nachteil.
Viele Klavierspieler benutzen sie, um \enquote{sich aufzuwärmen} und für das Spielen in gute Form zu kommen.
Das kann jemandem den falschen Eindruck vermitteln, daß diese wunderbare Form eine Konsequenz des stupiden Übens sei.
Sie ist es nicht; die \enquote{aufgewärmte Form} ist dieselbe, unabhängig davon, wie man zu ihr gelangt ist.
Deshalb kann man die Fallen der stupiden Übungen vermeiden, indem man nützlichere Möglichkeiten benutzt, um die Hände aufzuwärmen.
\hyperref[c1iii5a]{Tonleitern}\index{Tonleitern} sind für das Lockern der Finger hilfreich und \hyperref[c1iii5e]{Arpeggios}\index{Arpeggios} für das Lockern der Handgelenke.
Und sie dienen dem Lernen einiger grundlegender Fertigkeiten, wie wir oben in \hyperref[c1iii5]{\autoref{c1iii5}} gesehen haben.


\subsubsection{Wie verwendet man die Übungen für parallele Sets (Appassionata, 3. Satz)?}
\label{c1iii7c}

\textbf{Die Übungen für parallele Sets sind nicht dafür gedacht, \hyperref[c1iii7h]{Hanon}\index{Hanon}, Czerny usw. oder irgendeine Art von Übung zu ersetzen.}
Die Philosophie dieses Buchs ist, daß die Zeit besser damit verbracht werden kann, \enquote{wahre} Musik zu üben als \enquote{Übungsmusik}.
Die Übungen für parallele Sets wurden eingeführt, weil es keinen bekannten schnelleren Weg gibt, um Technik zu erwerben.
Deshalb sind technische Stücke, wie Liszt- oder Chopin-Etüden oder die Bach-Inventionen, keine \enquote{Übungsmusik} in diesem Sinne.
\textbf{Die Übungen für parallele Sets sind folgendermaßen zu benutzen:}

\begin{enumerate}[label={\roman*.}] 
\item \textbf{Zu Diagnosezwecken:} Indem Sie diese Übungen systematisch durchgehen, können Sie viele Ihrer Stärken und Schwächen erkennen.
Wenn Sie zu einer Passage kommen, die Sie nicht spielen können, haben Sie mit den Übungen eine Methode, um genau zu ermitteln, warum Sie die Passage nicht spielen können.
Im nachhinein erscheint es offensichtlich, daß man ein gutes Diagnosewerkzeug braucht, wenn man ein technisches Detail verbessern möchte.
Ansonsten ist es so, als ob man für eine Operation in eine Klinik gehen würde, ohne zu wissen welche Krankheit man hat.
Gemäß dieser medizinischen Analogie ist Hanon zu üben so, als wenn man in ein Krankenhaus gehen und täglich die einfachsten Routineuntersuchungen durchführen lassen würde.
Die Fähigkeit der parallelen Sets zur Diagnose ist am nützlichsten, wenn man eine schwierige Passage übt.
Es hilft Ihnen, genau zu bestimmen welche Finger schwach, langsam, unkoordiniert usw. sind.


\item \textbf{Für das Erwerben von Technik:} Die in (i) gefundenen Schwächen können nun durch die Benutzung genau derselben Übungen korrigiert werden, die zu ihrer Diagnose benutzt wurden.
Sie arbeiten einfach an den Übungen, die die Probleme offenbart haben.
Im Prinzip hören diese Übungen niemals auf, weil die Obergrenze der Geschwindigkeit offen ist.
In der Praxis enden sie jedoch bei Geschwindigkeiten um ein Quadrupel je Sekunde, weil wenige Stücke (wenn überhaupt eines) höhere Geschwindigkeiten erfordern.
In den meisten Fällen können Sie diese hohen Geschwindigkeiten nicht benutzen, sobald Sie nur eine Verbindung hinzufügen.
Das zeigt das Gute dieser Übungen, indem sie Ihnen erlauben, mit Geschwindigkeiten zu üben, die schneller sind als das, was Sie benötigen werden, und Ihnen so diesen zusätzlichen Spielraum an Sicherheit und Kontrolle verleihen.
Sie sollten diese Übungen während des HS-Übens am meisten benutzen, wenn Sie die Geschwindigkeit höher als die endgültige Geschwindigkeit bringen.


\end{enumerate}
\textbf{Die Prozeduren (i) und (ii) sind alles, was Sie brauchen, um die meisten Probleme beim Spielen von schwierigem Material zu lösen.}
Haben Sie sie erst erfolgreich auf mehrere zuvor \enquote{unmögliche} Situationen angewandt, werden Sie die Gewißheit erlangen, daß -- innerhalb eines vernünftigen Rahmens -- nichts unbezwingbar ist.

Nehmen Sie als Beispiel eine der schwierigsten Passagen des dritten Satzes von Beethovens Appassionata: die LH-Begleitung zu dem höhepunkthaften Lauf der RH in Takt 63 und ähnliche nachfolgende Passagen.
Wenn Sie sich Aufnahmen davon sorgfältig anhören, werden Sie feststellen, daß sogar die berühmtesten Pianisten Schwierigkeiten mit dieser LH haben und dazu neigen, sie langsam zu beginnen und dann zu beschleunigen oder sogar die Noten vereinfachen.
Diese Begleitung besteht aus den zusammengesetzten parallelen Sets 2.3,1.5 und 1.5,2.3, wobei 1.5 eine Oktave ist.
Die erforderliche Technik zu erwerben reduziert sich einfach dazu, diese parallelen Sets zu perfektionieren und sie dann zu verbinden.
Für viele wird eines der beiden parallelen Sets schwierig sein, und dieses müssen Sie meistern.
Zu versuchen es zu lernen, indem man es HT erst langsam spielt und dann beschleunigt, würde viel länger dauern.
Tatsächlich garantiert bloßes Wiederholen nicht, daß Sie jemals erfolgreich sein werden, weil es zu einem Rennen zwischen dem Erfolg und dem Aufbau einer Geschwindigkeitsbarriere wird, wenn Sie versuchen schneller zu werden.
Sie müssen HS üben und häufig die Hände abwechseln, um Streß und Ermüdung zu vermeiden.
Sie müssen am Anfang auch leise üben, damit Sie lernen zu \hyperref[c1ii14]{entspannen}\index{entspannen}.
Ohne parallele Sets besteht eine hohe Wahrscheinlichkeit, daß Sie streßbeladene Angewohnheiten entwickeln und eine Geschwindigkeitsbarriere erzeugen.
Ist diese Geschwindigkeitsbarriere erst einmal aufgebaut, können Sie Ihr ganzes Leben üben, ohne eine Verbesserung zu erzielen.
Wenn Sie ohne parallele Sets üben, können Sie zudem nicht herausfinden, welcher Teil dieses Abschnitts Sie davon abhält, Fortschritte zu machen.

Ein weiterer Grund für die Geschwindigkeitsbarriere ist das \enquote{Spielen unter Zwang}.
Der Rest des dritten Satzes läßt sich größtenteils einfach auf Geschwindigkeit bringen, weshalb man dazu neigt, ihn mit der endgültigen Geschwindigkeit zu spielen, und dann zu versuchen, sich ohne die erforderliche Technik durch diese schwierige Passage \enquote{hindurchzuzwängen}.
Das daraus resultierende streßbeladene Spielen erzeugt eine Geschwindigkeitsbarriere.
Dieses Beispiel zeigt, wie wichtig es ist, die schwierigen Teile als erste zu üben.
In dieser Situation wird jemand, der weiß, wie man üben muß, langsamer werden und somit nicht jenseits seiner Fähigkeiten spielen müssen.

\textbf{Um es zusammenzufassen: Die Übungen für parallele Sets bilden einen der Hauptpfeiler der Methoden dieses Buchs.}
Sie sind einer der Gründe für die Behauptung, daß nichts zu schwierig zum Spielen ist, wenn man weiß, wie man üben muß.
Sie dienen sowohl als Diagnosewerkzeug als auch als ein Werkzeug zur Entwicklung der Technik.
Praktisch die ganze Technik sollte durch den Gebrauch von parallelen Sets während des HS-Übens erworben werden, um die Geschwindigkeit nach oben zu bringen, das Entspannen zu lernen und Kontrolle zu erlangen.
Sie bilden einen vollständigen Satz, so daß Sie wissen, daß Sie alle notwendigen Werkzeuge besitzen.
Anders als Hanon usw. können sie sofort zur Hilfe herbeigeholt werden, wenn Sie auf \textit{irgendeine} schwierige Passage treffen, und sie erlauben das Üben mit jeder Geschwindigkeit, einschließlich von Geschwindigkeiten, die weit höher sind als Sie sie jemals benötigen werden.
Sie sind ideal dafür, das Spielen ohne Streß und mit Klangkontrolle zu üben.
Insbesondere ist es wichtig, sich anzugewöhnen, mit den Fingern über die Tasten zu gleiten und die Tasten zu erfühlen, bevor man sie spielt.
Das Gleiten der Finger (über die Tasten zu streichen) verleiht Klangkontrolle und das Erfühlen der Tasten verbessert die Genauigkeit.
Ohne eine schwierige Passage in einfache parallele Sets aufzuteilen ist es unmöglich, diese zusätzlichen Feinheiten in Ihr Spielen aufzunehmen.
Wir kommen nun zu mehreren anderen nützlichen Übungen.



