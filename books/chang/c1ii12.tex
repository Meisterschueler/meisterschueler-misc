% File: c1ii12

\label{c1ii12}

% zuletzt geändert 20.09.2009

\subsection{Lernen, Auswendiglernen und mentales Spielen}

\textbf{Es gibt keinen schnelleren Weg auswendig zu lernen, als es gleich zu tun, wenn Sie ein Stück das erste Mal lernen, und für ein schwieriges Stück gibt es keinen schnelleren Weg, es zu lernen, als es auswendig zu lernen.}
Beginnen Sie das Auswendiglernen, indem Sie lernen, wie die Musik klingen sollte: Melodie, Rhythmus usw.
Benutzen Sie dann die Notenblätter, um für jede Note die zugehörige Taste zu finden und sich zu merken; das nennt man \hyperref[c1iii6tastatur]{Tastatur-Gedächtnis} - Sie merken sich, wie Sie dieses Stück auf dem Klavier spielen, mit dem Fingersatz, den Handbewegungen usw.
Einige Klavierspieler benutzen das \hyperref[c1iii6foto]{fotografische Gedächtnis} bei dem Sie sich das Notenblatt als komplettes Bild merken.
Wenn man ein Notenblatt nehmen und versuchen sollte, sich jede einzelne Note zu merken, wäre das - sogar für einen Konzertpianisten - unsagbar schwer.
Wenn man jedoch die Musik kennt (Melodie, Akkordstruktur usw.), wird es für jeden einfach!
Das wird in \hyperref[c1iii6]{\autoref{c1iii6}} erklärt, in dem das Auswendiglernen detaillierter behandelt wird.
Ich ziehe das Tastatur-Gedächtnis dem fotografischen Gedächtnis vor, weil es dabei hilft, die Noten auf dem Klavier zu finden, ohne dass man in Gedanken das Notenblatt \enquote{lesen} muss.
Prägen Sie sich deshalb jeden Abschnitt ein, den Sie für die Technik üben, während Sie diesen viele Male \hyperref[c1ii7]{mit getrennten Händen} in kleinen Segmenten wiederholen.
\textbf{Die Prozeduren für das Einprägen sind im Grunde mit denen für das Aneignen der Technik identisch.}
Zum Beispiel sollte das Einprägen zunächst mit getrennten Händen erfolgen, für \hyperref[c1ii5]{die schwierigen Abschnitte zuerst} usw.
Wenn sie erst später auswendig lernen, müssen Sie die gleiche Prozedur noch einmal ausführen.
Es mag einfacher erscheinen, ein zweites Mal durch die gleiche Prozedur zu gehen.
Ist es aber nicht.
Auswendiglernen ist ein komplexer Vorgang (sogar nachdem Sie das Stück gut spielen können); Schüler, die versuchen, ein Stück nach dem Lernen auswendig zu lernen, geben aus diesem Grund entweder auf oder sie lernen es niemals völlig auswendig.
Das ist verständlich; der zum Einprägen erforderliche Aufwand kann schnell den Punkt abnehmender Ertragszuwächse erreichen, wenn man das Stück bereits spielen kann.

Zwei wichtige Punkte, die Sie auswendig lernen müssen, sind die Taktart (siehe \hyperref[c1iii1b]{\autoref{c1iii1b}}) und die Vorzeichnung (siehe \hyperref[c1iii5d]{\autoref{c1iii5d}}).
Die Taktart ist leicht zu verstehen und hilft Ihnen dabei, mit dem korrekten Rhythmus zu spielen.
Die Vorzeichnung (wie viele Kreuze und Be's) ist komplexer, weil sie Ihnen nicht die genaue Tonart (C-Dur usw.) verrät, in der das Stück steht.
Wenn Sie wissen, ob die Komposition in einer Dur- oder Molltonart steht, dann gibt Ihnen die Vorzeichnung die Tonart; wenn die Vorzeichnung zum Beispiel keine Kreuze und Be's hat (wie in \enquote{Für Elise}), dann  ist es entweder C-Dur oder a-Moll (siehe III.5d).
Die meisten Schüler kennen die Dur-Tonleitern; Sie werden mehr über die Theorie wissen müssen, um die Moll-Tonleitern herauszufinden;
deshalb sollten nur diejenigen mit genügenden theoretischen Kenntnissen sich die Tonart merken.
Wenn Sie nicht sicher sind, lernen Sie nur die Vorzeichnung.
Diese Tonart ist die Tonika der Musik, um die herum der Komponist Akkordprogressionen dazu benutzt, die Tonart zu ändern.
Die meisten Kompositionen beginnen und enden mit der Tonika, und die Akkorde schreiten in der Regel entlang des Quintenzirkels fort (siehe Kapitel 2, Abschnitt 2b). Bis jetzt wissen wir, dass \enquote{Für Elise} entweder in C-Dur oder a-Moll steht.
Da es etwas melancholisch ist, vermuten wir eine Moll-Tonart.
Die ersten zwei Takte sind wie eine Fanfare, die das erste Thema einführt, der Hauptteil des Themas beginnt in Takt 3, der A ist, die Tonika von a-Moll!
Zudem ist der letzte Akkord ebenfalls die Tonika von a-Moll.
Wir sind deshalb fast sicher, dass es in a-Moll steht.
Das einzige Vorzeichen in a-Moll ist G\# (siehe \hyperref[tablemoll]{Tabelle der Moll-Tonleitern}), das wir in Takt 4 finden; daraus schließen wir, dass es in a-Moll steht.
Wenn Sie diese Details verstehen, dann können Sie \textit{wirklich} gut auswendig lernen.

Kehren wir zur Taktart zurück, die 3/8 ist: drei Schläge je Takt und ein Achtel je Schlag.
Somit ist es im Format eines Walzers\footnote{fast, weil der Walzer im 3/4-Takt ist}, aber musikalisch sollte es nicht wie ein Tanz gespielt werden, sondern viel sanfter, weil es melancholisch und eindringlich romantisch ist.
Die Taktart sagt uns, dass Takte wie Takt 3 nicht wie zwei Triolen gespielt werden dürfen, weil es drei Schläge gibt.
Man muss den Akzent des ersten Schlags eines jeden Takts jedoch nicht überbetonen wie in einem Wiener Walzer.
Die Taktart ist für das musikalische und korrekte Spielen eindeutig  nützlich.
Ohne die Taktart können sie sich schnell einen falschen Rhythmus angewöhnen, der Ihr Spielen für Experten amateurhaft klingen lässt.

\textbf{Haben Schüler erst einmal die für sie passenden Abläufe zum Lernen und Auswendiglernen entwickelt, werden die meisten von ihnen der Meinung sein, dass gleichzeitiges Lernen und Auswendiglernen für schwierige Passagen weniger Zeit benötigt als das Lernen alleine.}
Das geschieht, weil man den Vorgang eliminiert, auf die Noten zu schauen, sie zu interpretieren und die Befehle von den Augen zum Gehirn und danach zu den Händen zu geben.
Material, das auswendig gelernt wurde, solange man jung ist (ungefähr bevor man 20 Jahre alt wird), wird fast nie vergessen.
Deshalb ist es so wichtig, schnelle Methoden für das Aneignen der Technik zu lernen und so viele Stücke wie möglich auswendig zu lernen, bevor man das späte Teenageralter erreicht.
Es ist einfacher, etwas auswendig zu lernen, wenn man es schnell spielen kann; machen Sie sich deshalb keine Sorgen, wenn Sie am Anfang Schwierigkeiten haben, etwas bei langsamer Geschwindigkeit auswendig zu lernen; es wird einfacher, wenn Sie schneller werden.


\label{c1ii12mental}

\textbf{Die einzige Möglichkeit, gut auswendig zu lernen, ist, das \hyperref[c1iii6tastatur]{mentale Spielen} lernen.}
Tatsächlich ist das mentale Spielen das logische und endgütige Ziel aller Übungsmethoden, die wir besprechen, weil Technik alleine Sie nicht in die Lage versetzt, fehlerfrei, musikalisch und ohne nervös zu werden vorzuspielen.
Lesen Sie \hyperref[c1iii6j]{\autoref{c1iii6j}} für mehr Details über das mentale Spielen.
Beim mentalen Spielen lernen Sie, das Klavier in Gedanken - ohne Klavier - zu spielen, einschließlich des richtigen Fingersatzes und Ihrer Vorstellung davon, wie die Musik klingen soll.
Sie können das \hyperref[c1iii6tastatur]{Tastatur-Gedächtnis} oder das \hyperref[c1iii6foto]{fotografische Gedächtnis} für das mentale Spielen benutzen, aber ich empfehle für Anfänger das Tastatur-Gedächtnis, weil es effizienter ist;
für fortgeschrittene Spieler sind Tastatur-Gedächtnis und  fotografisches Gedächtnis dasselbe, denn wenn man das eine beherrscht, kommt das andere wie von selbst.
Wann immer Sie einen kleinen Abschnitt auswendig lernen, schließen Sie die Augen, und prüfen Sie, ob sie ihn in Gedanken spielen können, ohne ihn auf dem Klavier zu spielen.
Haben Sie ein ganzes Stück mit getrennten Händen auswendig gelernt, sollten Sie es auch mit getrennten Händen in Ihrem Kopf spielen können.
Das ist der Zeitpunkt, die Struktur des Stücks zu analysieren, wie es aufgebaut ist und wie die Themen sich mit dem Fortgang der Musik entwickeln.
Wenn Sie geübt sind, werden Sie feststellen, dass es nur eine geringe Investition an Zeit erfordert, sich das mentale Spielen anzueignen.
Das Beste ist: Sie werden auch entdecken, dass Ihr Gedächtnis mit dem Aufbau eines soliden mentalen Spielens so gut wie nur irgend möglich wird; Sie werden darauf vertrauen, dass Sie in der Lage sind, ohne Fehler, Gedächtnisblockaden usw. zu spielen, und Sie werden sich auf die Musik konzentrieren können.
Mentales Spielen hilft auch der Technik; es ist zum Beispiel viel einfacher, mit hoher Geschwindigkeit zu spielen, wenn Sie mit dieser Geschwindigkeit in Gedanken spielen können; die Unfähigkeit schnell zu spielen hat ihren Ursprung sehr oft im Gehirn.
Ein Vorteil des mentalen Spielens ist, dass Sie es jederzeit und überall üben und ihre effektive Übungszeit in hohem Maß steigern können.

\textbf{Das Gedächtnis ist ein assoziativer Prozess.
Gedächtniskünstler (einschließlich einiger Savants) und alle Konzertpianisten, die Stunden von Musik auswendig lernen können, hängen von Algorithmen ab, mit denen Sie das Gespeicherte assoziieren können (egal, ob sie es wissen oder nicht).}
Musiker haben in dieser Hinsicht besonderes Glück, weil Musik gerade ein solcher Algorithmus ist.
Trotzdem wird dieser \enquote{Gedächtnistrick}, die Musik als Algorithmus für das Auswendiglernen zu benutzen, Musikschülern selten formal gelehrt; statt dessen wird ihnen oft geraten, stets zu wiederholen, \enquote{bis die Musik in den Händen ist}, was eine der schlechtesten Gedächtnis-Methoden ist, denn wie wir in \hyperref[c1iii6d]{\autoref{c1iii6d}} sehen werden, führt Wiederholung zum \enquote{\hyperref[c1iii6d]{Hand-Gedächtnis}}, was eine falsche Art von Gedächtnis ist, die zu vielen Problemen, wie Gedächtnisblockaden, führen kann.
Beim mentalen Spielen assoziieren Sie die Musik in Gedanken damit, wie Sie sie am Klavier erzeugen.
Es ist wichtig, das mentale Spielen zu üben, ohne am Klavier zu spielen, weil sie ein \enquote{Klang-Gedächtnis} (so wie ein \enquote{Hand-Gedächtnis}) erwerben und den Klang des Klaviers als Stütze für das Abrufen benutzen können, und das Klang-Gedächtnis kann dieselben Probleme verursachen, die mit dem Hand-Gedächtnis verbunden sind.

Warum sind das Gedächtnis und das mentale Spielen so wichtig?
Sie lösen nicht nur die praktischen Probleme der Technik und des \hyperref[c1iii14]{Auftretens}, sondern bringen Sie auch als Musiker voran und steigern die Intelligenz.
So wie man einen Computer beschleunigen kann, indem man Speicher hinzufügt, so kann man seine effektive Intelligenz steigern, indem man das Gedächtnis verbessert.
Tatsächlich ist Gedächtnisverlust eines der ersten Zeichen eines geistigen Verfalls, zum Beispiel bei Alzheimer.
\textbf{Es ist nun klar, dass viele dieser \enquote{erstaunlichen Kunststücke} großer Musiker wie Mozart einfach Nebenprodukte eines starken mentalen Spielens waren, und dass solche Fertigkeiten erlernt werden können.}


\subsection{Spielgeschwindigkeit beim Üben}
\label{c1ii13}

\textbf{Kommen Sie so schnell wie möglich auf Geschwindigkeit.} Erinnern Sie sich daran, dass wir immer noch \hyperref[c1ii7]{mit getrennten Händen} üben.
So schnell zu spielen, dass man anfängt Stress zu empfinden und Fehler zu machen, verbessert die Technik nicht, weil man nur die Fehler übt und sich schlechte Angewohnheiten aneignet.
Die Finger zu zwingen, auf dieselbe Art schneller zu spielen, ist nicht der Weg, die Geschwindigkeit zu erhöhen.
Wie beim \hyperref[c1ii11]{parallelen Spielen} gezeigt wurde, brauchen Sie neue Arten zu spielen, die automatisch die Geschwindigkeit erhöhen und den Stress reduzieren.
Beim parallelen Spielen ist es oft sogar einfacher, schnell als langsam zu spielen.
Erarbeiten Sie Handpositionen und -bewegungen, die automatisch die Geschwindigkeit erhöhen.
Diese Themen sind die Hauptbeiträge dieses Buchs und werden später im Einzelnen behandelt, da sie zu umfangreich sind, um hier schon behandelt zu werden; dazu gehören so spezifische Fertigkeiten, wie der \hyperref[c1iii5b]{Daumenübersatz}, die \hyperref[c1iii5c]{Glissandobewegung}, die \hyperref[c1ii14]{Entspannung}, die \hyperref[c1iii4b]{flachen Fingerhaltungen}, die Bewegungen der Arme und Handgelenke sowie die Benutzung der \hyperref[c1ii15]{automatischen Verbesserung nach dem Üben}.
Wenn Sie innerhalb weniger Minuten keinen bedeutenden Fortschritt erzielen, machen Sie wahrscheinlich etwas falsch - denken Sie sich etwas Neues aus.
Schüler, die die intuitive Methode benutzen, haben sich damit abgefunden, dieselbe Sache stundenlang mit geringer sichtbarer Verbesserung zu wiederholen.
Diese Mentalität muss vermieden werden, damit man schneller lernt.
Wenn man die Geschwindigkeit erhöht, kann man in zwei Arten von Situationen kommen.
Die eine betrifft die technischen Fertigkeiten, die Sie bereits besitzen; Sie sollten in der Lage sein, diese innerhalb von Minuten auf Geschwindigkeit zu bringen.
Die andere betrifft neue Fertigkeiten; diese werden mehr Zeit benötigen und werden weiter unten in \hyperref[c1ii15]{Abschnitt 15} besprochen.

\textbf{Die Technik verbessert sich am schnellsten, wenn man mit einer Geschwindigkeit spielt, bei der man exakt spielen kann.}
Das stimmt insbesondere wenn man beidhändig spielt (bitte gedulden Sie sich - ich verspreche Ihnen, dass wir noch zum beidhändigen Üben kommen).
Da Sie einhändig mehr Kontrolle haben, kommen Sie einhändig zu weitaus schnellerem Spiel als beidhändig, ohne den Stress zu vergrößern oder sich schlechte Angewohnheiten anzueignen.
Somit ist es falsch, zu denken, man könne schneller Fortschritte erzielen, indem man so schnell wie möglich spielt (schließlich kann man dieselbe Passage zweimal so oft spielen, wenn man doppelt so schnell spielt!).
Da eines der Hauptziele des einhändigen Übens das Gewinnen von Geschwindigkeit ist, kommen die Notwendigkeit schnell Geschwindigkeit zu erreichen und das exakte Üben miteinander in Konflikt.
Die Lösung ist, die Geschwindigkeit beim Üben ständig zu ändern; bleiben Sie nicht zu lange bei einer Geschwindigkeit.
Es gibt für sehr schwierige Passagen, die Fertigkeiten erfordern, die Sie noch nicht besitzen, keine Alternative für das stufenweise Erhöhen der Geschwindigkeit.
Benutzen Sie dazu versuchsweise zu hohe Geschwindigkeiten, um herauszufinden, was geändert werden muss, damit Sie mit solchen Geschwindigkeiten spielen können.
Werden Sie dann langsamer, und üben Sie diese neuen Bewegungen.

Um die Geschwindigkeit zu variieren, gehen Sie zunächst zu einer handhabbaren \enquote{Maximalgeschwindigkeit}, bei der Sie exakt spielen können.
Werden Sie dann schneller (indem Sie, wenn notwendig, \hyperref[c1ii11]{parallele Sets} usw. benutzen), und achten Sie darauf, wie das Spielen geändert werden muss (machen Sie sich nichts daraus, wenn Sie an diesem Punkt nicht exakt spielen, da Sie es nicht viele Male wiederholen).
Benutzen Sie dann diese Bewegung und spielen Sie mit der vorhergehenden \enquote{exakten Maximalgeschwindigkeit}.
Es sollte nun spürbar einfacher sein.
Üben Sie eine Weile mit dieser Geschwindigkeit, versuchen Sie dann langsamere Geschwindigkeiten, um sicherzustellen, dass Sie völlig entspannt sind und exakt spielen.
Wiederholen Sie dann die ganze Prozedur.
Auf diese Art schrauben Sie die Geschwindigkeit in gut zu bewältigenden Schritten hoch und arbeiten an jeder benötigten Fähigkeit gesondert.
In den meisten Fällen sollten Sie in der Lage sein, das meiste des neuen Stücks - zumindest in kleinen Segmenten und einhändig - während der ersten Sitzung in der endgültigen Geschwindigkeit zu spielen.
Am Anfang mag es unmöglich erscheinen, die endgültige Geschwindigkeit während der ersten Sitzung zu erreichen, aber mit Übung kann jeder Schüler dieses Ziel erstaunlich schnell erreichen.
 

\subsection{Wie man entspannt}
\label{c1ii14}

\textbf{Das Wichtigste zum Erreichen der vorgegebenen Geschwindigkeit ist, zu entspannen.}
Entspannen bedeutet, dass man nur die Muskeln benutzt, die zum Spielen benötigt werden.
Dadurch kann man so hart arbeiten wie man möchte und entspannt sein.
Der entspannte Zustand ist beim einhändigen Üben besonders leicht zu erreichen.
Es gibt zwei Denkschulen zur Entspannung.
Eine Schule behauptet, dass es auf lange Sicht besser sei, nicht zu üben als mit dem leichtesten Anflug von Spannung zu üben.
Diese Schule unterrichtet, indem sie zeigt, wie man eine Note entspannt spielt, dann vorsichtig weitergeht und nur das leichte Material präsentiert, das man entspannt spielen kann.
Die andere Schule argumentiert, dass Entspannung sicherlich ein notwendiger Aspekt der Technik sei, aber dass es nicht der optimale Ansatz ist, die ganze Übungsphilosophie der Entspannung unterzuordnen.
Der zweite Ansatz sollte der bessere sein, vorausgesetzt, Ihnen sind die Fallen bewusst.

Das menschliche Gehirn kann ziemlich verschwenderisch sein.
Sogar für die einfachsten Aufgaben benutzt das untrainierte Gehirn die meisten Muskeln des Körpers.
Und wenn die Aufgabe schwierig ist, neigt das Gehirn dazu, den ganzen Körper in einer Masse angespannter Muskeln einzusperren.
Um zu entspannen, müssen Sie eine bewusste Anstrengung unternehmen, um alle unnötigen Muskeln abzuschalten.
Das ist nicht einfach, weil es den natürlichen Neigungen des Gehirns entgegensteht.
Sie müssen das Entspannen genauso viel üben wie das Bewegen der Finger zum Spielen der Tasten.
Entspannen bedeutet nicht, \enquote{alle Muskeln erschlaffen zu lassen}; es bedeutet, dass die nicht benötigten Muskeln sogar dann entspannt sind, wenn die notwendigen unter voller Last arbeiten.
Diese Fähigkeit zur Koordination verlangt viel Übung.

Wenn die Entspannung für Sie etwas Neues ist, können Sie mit den einfacheren Stücken, die Sie gelernt haben, anfangen und das Hinzufügen der Entspannung üben.
Die \hyperref[c1iii7b]{Übungen für parallele Sets} von Abschnitt III.7 können Ihnen helfen, das Entspannen zu üben.
Eine Möglichkeit, die Entspannung zu spüren, ist, ein paralleles Set zu üben, es zu beschleunigen bis man Stress aufbaut und dann zu versuchen zu entspannen.
Sie werden neue Bewegungen und Positionen der Arme, Handgelenke usw. finden müssen, die das erlauben; wenn Sie diese gefunden haben, werden Sie spüren, wie der Stress in der Hand während des Spielens verschwindet.

Entspannen Sie alle die unterschiedlichen Körperfunktionen, wie das Atmen und das periodische Schlucken.
Einige Schüler unterbrechen das Atmen beim Spielen anspruchsvoller Passagen, weil sie sich auf das Spielen konzentrieren.
Wenn Sie entspannt sind, sollten Sie in der Lage sein, alle normalen Körperfunktionen auszuführen und sich trotzdem gleichzeitig auf das Spielen zu konzentrieren.
\hyperref[c1ii21]{Abschnitt 21} weiter unten erklärt, wie man das Zwerchfell für die richtige Atmung benutzt.
Wenn Ihre Kehle nach schwerem Üben trocken ist, haben Sie das Schlucken vergessen.
Das alles sind Anzeichen von Stress.

Viele Schüler, denen das Entspannen nicht gelehrt wurde, glauben, dass langes wiederholtes Üben irgendwie die Hand so verwandelt, dass sie spielen kann.
In Wahrheit ist es oft so, dass die Hand zufällig über die richtige Bewegung für die Entspannung stolpert.
Deshalb werden manche Fähigkeiten schnell erworben, während andere ewig brauchen, und deshalb erwerben manche Schüler bestimmte Fähigkeiten schnell, während andere Schüler mit denselben Fähigkeiten kämpfen.
Entspannung ist ein Zustand des instabilen Gleichgewichts: Indem man lernt zu entspannen, wird das Spielen leichter, was das Entspannen vereinfacht usw.
Das erklärt, warum die Entspannung für manche ein größeres Problem ist, während sie für andere völlig normal ist.
Aber das ist eine der wunderbarsten Informationen - sie bedeutet, dass jeder das Entspannen lernen kann, wenn er richtig unterrichtet wird.

Entspannung ist das Einsparen von Energie.
Es gibt mindestens zwei Möglichkeiten zum Einsparen:

\begin{enumerate}[label={\arabic*.}] 
\item Benutzen Sie keine unnötigen Muskeln, insbesondere die gegensinnigen Muskeln\footnote{Antagonisten}.
\item Schalten Sie die arbeitenden Muskeln ab, sobald diese ihre Arbeit verrichtet haben.
 \end{enumerate}
Lassen Sie uns dies mit dem einfingrigen \hyperref[c1ii10]{Freien Fall} demonstrieren.
\enquote{1} ist das leichteste; erlauben Sie einfach der Schwerkraft, den Fall völlig zu kontrollieren, während der ganze Körper bequem auf der Bank ruht.
Wer angespannt ist, wird beide Muskeln zusammenziehen: diejenigen für das Heben und diejenigen für das Senken der Hand.
Für \enquote{2} müssen Sie eine neue Angewohnheit lernen, wenn Sie sie noch nicht haben (wenige haben sie am Anfang).
Das ist die Angewohnheit, alle Muskeln zu entspannen, sobald sie den unteren Punkt des Tastenwegs erreicht haben.
Während des Freien Falls lassen Sie den Arm durch die Schwerkraft nach unten ziehen, aber am Ende des Tastenwegs müssen Sie den Finger für einen Moment anspannen, um die Hand zu stoppen.
Danach müssen Sie alle Muskeln schnell entspannen.
Heben Sie nicht die Hand, lassen Sie die Hand bequem auf dem Klavier ruhen und zwar mit gerade so viel Kraft auf dem Finger, die genügt, das Gewicht des Arms zu unterstützen.
Stellen Sie sicher, dass Sie\footnote{die Tasten} nicht herunterdrücken.
Das ist schwieriger als man zunächst annimmt, weil der Ellenbogen mitten in der Luft schwebt und dieselben Muskelbündel, die benutzt werden, um die Finger für die Unterstützung des Armgewichts zu spannen, auch benutzt werden, um\footnote{die Tasten} herunterzudrücken.

Zueinander gegensinnige Muskeln gleichzeitig anzuspannen ist ein Hauptgrund der Verspannung.
Wenn der Klavierspieler es nicht merkt, kann es außer Kontrolle geraten und zu Verletzungen führen.
So wie wir lernen müssen, die einzelnen Finger der Hand unabhängig zu kontrollieren, müssen wir auch lernen, jeden der gegensinnigen Muskeln, wie Beuger und Strecker, unabhängig zu kontrollieren.
Die schlimmste Auswirkung von Stress ist, dass er Sie in einen Kampf zwingt, den Sie nicht gewinnen können, weil Sie gegen einen Gegner kämpfen, der genau so stark ist, wie Sie es sind - nämlich Sie selbst.
Es sind Ihre eigenen Muskeln, die gegen Ihren Körper arbeiten. 
Und je mehr Sie üben, um so schlimmer wird das Problem.
Wenn es schlimm genug wird, kann es zu Verletzungen führen, weil die Muskeln stärker werden als die Materialbelastbarkeit Ihres Körpers ist.

Ohne Training denken wenige Menschen daran, ihre Muskeln gezielt abzuschalten; normalerweise vergisst man sie einfach, wenn ihre Arbeit getan ist.
Wenn die Finger schnell arbeiten, müssen Sie jedoch schnell entspannen; ansonsten bekommen die Finger niemals eine Pause oder sind niemals bereit für die nächste Note.
Eine gute Übung für das schnelle Entspannen ist, mit einer gedrückten Taste anzufangen und einen schnellen, mäßig lauten Ton mit demselben Finger zu spielen.
Nun müssen Sie eine aufwärts oder abwärts gerichtete Kraft aufbringen \textit{und} den Muskel abschalten.
Wenn Sie ihn abschalten, müssen Sie zu dem Gefühl zurückkehren, das Sie am Ende eines Freien Falls hatten.
Sie werden herausfinden, dass es um so länger dauert zu entspannen, je härter Sie die Note spielen.
Üben Sie, diese Zeit zum Entspannen zu verkürzen.

Das Wunderbare an diesen Entspannungsmethoden ist, dass sie, nachdem Sie sie für eine kurze Zeit praktiziert haben (vielleicht ein paar Wochen), zunehmend von selbst in Ihr Spielen einfließen - sogar in Stücke, die Sie bereits gelernt haben -, solange Sie auf die Entspannung achten.
\textbf{Entspannung (den ganzen Körper einbeziehen), Armgewicht (Freier Fall) und die Vermeidung von stupiden, wiederholenden Übungen waren Schlüsselelemente in Chopins Lehren.}
Entspannung ist nutzlos, solange sie nicht von musikalischem Spielen begleitet wird; Chopin bestand sogar auf musikalischem Spielen vor dem Erwerben von Technik, weil er wusste, dass Entspannung, Musik und Technik untrennbar sind.
Das mag der Grund sein, warum die meisten von Chopins Kompositionen (anders als die von Beethoven) mit einer weiten Spanne von Geschwindigkeiten gespielt werden können.



