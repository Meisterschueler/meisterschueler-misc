% File: c1iii6l2

\label{c1iii6l2}

\textbf{Bachs Invention \#8, erster Tag.}
Die Taktart ist 3/4, d.h. es kommt jeweils ein Schlag auf eine Viertelnote, und jeder Takt hat drei Schläge.
Die Tonartenvorzeichnung hat ein Be, was die Tonart im Quintenzirkel einen Schritt gegen den Uhrzeigersinn von C-Dur zu F-Dur verschiebt.
Beginnen Sie, indem Sie die Takte 2 bis 4 der LH einschließlich der ersten beiden Noten von Takt 5 (Verbindung) auswendig lernen.
Es sollte weniger als eine Minute dauern, sich das zu merken; fangen Sie dann an, es mit der richtigen Geschwindigkeit zu spielen.
Heben Sie Ihre Hände vom Klavier, schließen Sie die Augen, und spielen Sie diesen Abschnitt in Gedanken, wobei Sie sich jede Note und jede Taste, die Sie spielen, bildlich vorstellen.
Machen Sie dann das gleiche mit der RH, Takte 1 bis 4, einschließlich der ersten vier Noten von Takt 5.
Kehren Sie nun zur LH zurück, und sehen Sie, ob Sie es ohne die Noten spielen können, und genauso mit der RH.
Wenn Sie es können, sollten Sie diesen Teil der Noten nie wieder nachsehen müssen, es sei denn, Sie hängen fest, was hin und wieder passieren kann.
Wechseln Sie zwischen der LH und der RH hin und her, bis Sie zufrieden sind.
Das sollte nur einige weitere Minuten dauern.
Sagen wir, die ganze Prozedur dauert fünf Minuten, bei einem schnellen Lerner sind es weniger.

Lernen Sie nun die Takte 5 bis 7 einschließlich der ersten beiden Noten der LH und der ersten vier Noten der RH in Takt 8.
Das sollte in ungefähr vier Minuten abgeschlossen sein.
Das sind alles HS-Übungen; wir werden nicht mit HT anfangen, bis wir das ganze Stück HS auswendig gelernt haben.
Es steht Ihnen jedoch frei, es jederzeit mit HT zu versuchen; verschwenden Sie aber keine Zeit mit dem HT-Üben, wenn Sie nicht sofort einen schnellen Fortschritt machen, da wir einem Ablaufplan folgen müssen!
Wenn Sie mit den Takten 5 bis 7 beginnen, machen Sie sich keine Sorgen darüber, daß Sie die zuvor auswendig gelernten Takte vergessen - Sie sollten nicht mehr an diese denken.
Das wird nicht nur die geistige Anspannung und Verwirrung verringern (da Sie nicht verschiedene auswendig gelernte Abschnitte vermischen), sondern auch dafür sorgen, daß Sie den zuvor auswendig gelernten Abschnitt teilweise vergessen, gezwungen sind, ihn erneut auswendig zu lernen, und ihn somit besser behalten.
Wenn Sie mit den Takten 5-7 zufrieden sind, verbinden Sie die Takte 1-7 einschließlich der Verbindung in Takt 8.
Das wird weitere 3 Minuten für beide Hände (HS) benötigen.
Wenn Sie die Takte 2-4 vergessen haben, während Sie 5-7 lernten, wiederholen Sie den Lernprozeß - es wird Ihnen schnell wieder einfallen, und die Erinnerung wird dauerhafter sein.
Vergessen Sie nicht, jeden Abschnitt in Gedanken zu spielen.

Lernen Sie als nächstes die Takte 8-11 auswendig, und fügen Sie sie den vorhergehenden Abschnitten hinzu.
Lassen Sie uns für diesen Teil 8 Minuten veranschlagen, was insgesamt 20 Minuten macht, um die Takte 1-11 auswendig zu lernen und HS auf die vorgegebene Geschwindigkeit zu bringen.
Wenn Sie mit einigen Teilen technische Schwierigkeiten haben, machen Sie sich keine Sorgen, wir werden später daran arbeiten.
Es wird nicht erwartet, daß Sie jetzt schon etwas perfekt spielen können.

Als nächstes verlassen wir die Takte 1-11 und arbeiten nur an den Takten 12-23 (machen Sie sich keine Gedanken darüber, ob Sie sich an die Takte 1-11 erinnern werden - es ist wichtig, daß Sie alle Befürchtungen zerstreuen und das Gehirn frei halten, um sich auf das Auswendiglernen zu konzentrieren).
Teilen Sie diesen Abschnitt in folgende Teilstücke (die Verbindungen sollten offensichtlich sein): 12-15, 16-19 und 19-23.
Takt 19 wird zweimal geübt, weil das zusätzliche Zeit dafür bedeutet, den schwierigen vierten Finger der LH zu trainieren.
Arbeiten Sie nur an den Takten 12-23, bis Sie sie alle HS hintereinander spielen können.
Das sollte ungefähr weitere 20 Minuten dauern.

Nehmen Sie dann den Rest bis zum Ende (24-34).
Diese Takte können in folgenden Teilstücken gelernt werden: 24-25, 26-29 und 30-34.
Das wird weitere 20 Minuten erfordern, also insgesamt eine Stunde, um das Ganze auswendig zu lernen.
Sie können nun entweder aufhören und morgen weitermachen oder sich jeden der drei Abschnitte noch einmal vornehmen.
Wichtig ist hier, daß Sie sich keine Gedanken darüber machen, ob Sie sich morgen an das alles erinnern können (wahrscheinlich nicht), aber um Spaß zu haben, können Sie sogar versuchen, die drei Abschnitte zu verbinden, oder Sie spielen den Anfangsteil HT, um zu sehen, wie weit Sie kommen.
Arbeiten Sie an den Teilen, die Ihnen technische Probleme bereiten, wenn Sie versuchen die Geschwindigkeit zu steigern.
Üben Sie diese technischen Trainingseinheiten in so kleinen Teilstücken wie Sie können; das bedeutet häufig zweinotige parallele Sets.
D.h. üben Sie nur die Noten, die Sie nicht zufriedenstellend spielen können.
Springen Sie von Abschnitt zu Abschnitt.
Die gesamte am ersten Tag für das Auswendiglernen aufgewandte Zeit ist eine Stunde.
Sie können auch das zweite Stück, Invention \#1, beginnen.
Üben Sie zwischen dem ersten und zweiten Tag, in Gedanken zu spielen, wann immer Sie Zeit dazu haben.

\textbf{Zweiter Tag:} Gehen Sie jeden der drei Abschnitte noch einmal durch, und verbinden Sie sie dann.
Spielen Sie jeden Abschnitt zuerst in Gedanken, bevor Sie etwas auf dem Klavier spielen.
Es kann sein, daß Sie an manchen Stellen die Notenblätter benötigen.
Legen Sie dann die Notenblätter beiseite - Sie werden sie, außer in Notfällen und um die Genauigkeit während der Pflege noch einmal zu prüfen, selten nochmals nötig haben.
Die einzige Anforderung am zweiten Tag ist, in der Lage zu sein, das ganze Stück - sowohl am Klavier als auch in Gedanken - von Anfang bis Ende HS zu spielen.
Konzentrieren Sie sich darauf, die Geschwindigkeit zu steigern, und werden Sie so schnell Sie können ohne Fehler zu machen.
Üben Sie zu \hyperref[c1ii14]{entspannen}.
Wenn Sie anfangen Fehler zu machen, werden Sie langsamer, und zirkulieren Sie die Geschwindigkeit auf und ab.
Beachten Sie, daß es eventuell leichter ist auswendig zu lernen, wenn Sie schnell spielen, und daß Sie vielleicht auf Gedächtnislücken stoßen, wenn Sie langsam spielen.
Üben Sie deshalb mit verschiedenen Geschwindigkeiten.
Fürchten Sie sich nicht vor dem schnellen Spielen, aber achten Sie darauf, daß Sie es mit genügend mittlerer Geschwindigkeit und langsamem Spielen ausgleichen, damit Sie jeglichen \hyperref[fpd]{FPD} eliminieren.
Anfänger haben bei Akkordwechseln, die oft am Anfang eines Takts auftreten, die größten Schwierigkeiten.
Akkordwechsel erzeugen Schwierigkeiten, da man nach dem Wechsel eine neue Gruppe ungewohnter Noten spielen muß.

Wenn sie am zweiten Tag mit HS völlig zufrieden sind, könnten Sie mit HT beginnen und dazu die gleichen kleinen Teilstücke verwenden, mit denen Sie HS gelernt haben.
Die erste Note von Takt 3 ist eine Kollision der beiden Hände. Benutzen Sie nur die LH für diese Note, ebenso in Takt 18.
Spielen Sie leise, auch wenn \textit{f} angegeben ist, so daß Sie die Schlagnoten betonen können, um die beiden Hände zu synchronisieren und das \hyperref[c1ii14]{Entspannen} zu üben.
Sie werden am Anfang wahrscheinlich etwas angespannt sein, aber konzentrieren Sie sich so früh wie möglich auf das Entspannen.

\textbf{Mäßige Geschwindigkeit ist oft die Geschwindigkeit, mit der man am einfachsten aus dem Gedächtnis spielen kann, weil man den Rhythmus benutzen kann, um weiter zu machen, und man die Musik in Phrasen statt in einzelnen Noten erinnern kann.}
Richten Sie Ihre Aufmerksamkeit deshalb von Anfang an auf den \hyperref[c1iii1b]{Rhythmus}.
Werden Sie nun langsamer, und arbeiten Sie an der Genauigkeit.
Um zu vermeiden, daß das langsame Spielen schneller wird, achten Sie auf jede einzelne Note.
Wiederholen Sie diesen \enquote{Schnell-Langsam-Zyklus}, und Sie sollten mit jedem Zyklus merklich besser werden.
Die Hauptziele sind, das Stück komplett HS auswendig zu lernen und das HS-Spielen soviel wie möglich zu beschleunigen.
Wo immer Sie technische Probleme haben, benutzen Sie die \hyperref[c1iii7b]{Übungen für parallele Sets}, um die Technik schnell zu entwickeln.
Sie sollten nicht mehr als 1 Stunde benötigen.

\textbf{Dritter Tag:} Lernen Sie HT mit den drei Hauptabschnitten, wie Sie es mit HS getan haben.
Sobald Sie feststellen, daß Sie mit HT durcheinander kommen, gehen Sie zu HS zurück, um die Dinge aufzuklären.
Das ist auch ein guter Moment, um die Geschwindigkeit mit HS weiter zu steigern, schneller als die endgültige Geschwindigkeit (später mehr darüber, wie man das macht).
Natürlich werden diejenigen mit geringeren technischen Fertigkeiten langsamer spielen müssen.
Erinnern Sie sich daran, daß das Entspannen wichtiger ist als die Geschwindigkeit.
Sie werden mit HS schneller spielen als mit HT, und alle Versuche, die Geschwindigkeit zu steigern, sollten HS durchgeführt werden.
Da die Hände noch nicht richtig koordiniert sind, werden Sie wahrscheinlich ein paar Gedächtnislücken haben, und es mag schwierig sein, HT ohne Fehler zu spielen, außer wenn Sie langsam spielen.
Ab hier werden Sie sich auf die langsamere \hyperref[c1ii15]{PPI} verlassen müssen, um eine größere Verbesserung zu erzielen.
Sie haben jedoch in 3 Stunden innerhalb von 3 Tagen das Stück im Grunde auswendig gelernt und können es, vielleicht schleppend, HT spielen.
Sie können auch das ganze Stück in Gedanken spielen.

Fangen Sie nun mit dem zweiten Stück (Invention \#1) an, während Sie das erste Stück auf Hochglanz polieren.
Üben Sie die beiden Stücke abwechselnd.
Arbeiten Sie an \#1, bis Sie anfangen \#8 zu vergessen, frischen Sie dann \#8 wieder auf, und arbeiten Sie daran, bis Sie anfangen \#1 zu vergessen.
Erinnern Sie sich daran, daß Sie ein wenig vergessen \textit{wollen}, damit Sie erneut lernen können; das wird gebraucht, um ein Langzeitgedächtnis aufzubauen.
Es hat psychologische Vorteile, diese Art Programme zu benutzen, bei denen man nur gewinnen kann: Wenn man vergißt, ist es genau das, was man erreichen wollte; wenn man nicht vergessen kann, um so besser!
Dieses Programm gibt Ihnen auch ein Maß dafür, wieviel Sie in einer vorgegebenen Zeitspanne auswendig lernen können, bzw. wieviel nicht.
Jüngere Menschen sollten finden, daß die Menge, die man auf einmal auswendig lernen kann, rapide ansteigt, wenn man Erfahrung bekommt und mehr Tricks zum Auswendiglernen kennt.
Je schneller Sie auswendig lernen, desto schneller können Sie spielen, und je schneller sie spielen, desto leichter wird es auswendig zu lernen.
Gesteigertes Selbstvertrauen spielt auch eine größere Rolle.
Letzten Endes wird der hauptsächliche begrenzende Faktor Ihre technische Fertigkeitsstufe sein, nicht die Fähigkeit zum Auswendiglernen.
Wenn Sie genügend Technik haben, werden Sie innerhalb weniger Tage mit der richtigen Geschwindigkeit spielen.
Wenn Sie es nicht können, bedeutet das nur, daß Sie mehr Technik brauchen.
Es bedeutet nicht, daß Sie ein schlechter Auswendiglernender sind.

\textbf{Vierter Tag:} Es gibt nicht viel, was Sie nach zwei bis drei Tagen tun können, um das erste Stück technisch zu beschleunigen.
Beginnen Sie das Üben von \#8 einige Tage lang zunächst mit HS, und wechseln Sie dann zu HT, beides jeweils mit verschiedenen Geschwindigkeiten gemäß Ihrer augenblicklichen Laune.
Sobald Sie sich dazu bereit fühlen, üben Sie HT, aber kehren Sie zu HS zurück, wenn Sie anfangen Fehler zu machen, Gedächtnislücken mit HT bekommen oder wenn Sie technische Probleme damit haben, auf die endgültige Geschwindigkeit zu kommen.
Üben Sie, das Stück in Abschnitten HT zu spielen, springen Sie zufällig von Abschnitt zu Abschnitt durch das Stück.
Versuchen Sie, mit dem letzten kleinen Abschnitt zu beginnen und sich zum Anfang zurückzuarbeiten.

Isolieren Sie die Problemstellen, und üben Sie diese gesondert.
Die meisten Menschen haben eine schwächere LH, so daß es problematisch sein kann, die LH schneller als die endgültige Geschwindigkeit werden zu lassen.
Es mag z.B. schwer sein, die letzten vier Noten der LH in Takt 3 der Invention \#8, 4234(5), wobei (5) die Verbindung ist, schnell zu spielen.
Teilen Sie sie in diesem Fall in drei parallele Sets auf - 42, 423 und 2345 -, und üben Sie diese unter Benutzung der \hyperref[c1iii7b]{Übungen für parallele Sets}.
Verbinden Sie sie dann zu 423 und 2345.
423 ist kein paralleles Set (4 und 3 spielen dieselbe Note), so daß man es nicht so schnell wie parallele Sets spielen kann.
Bringen Sie sie zuerst auf annähernd unendliche Geschwindigkeit (fast wie ein Akkord), und lernen Sie dann, bei diesen Geschwindigkeiten zu entspannen, indem Sie in schnellen Quadrupeln spielen (s. \hyperref[c1iii7b1]{Abschnitt III.7b}).
Werden Sie dann schrittweise langsamer, um die Unabhängigkeit der Finger zu entwickeln.
Verbinden Sie die parallelen Sets paarweise, und verbinden Sie sie zum Schluß alle miteinander.
Das ist eine wirkliche Verbesserung der Technik und wird deshalb nicht über Nacht geschehen.
Sie werden während des Übens nur eine geringe Verbesserung wahrnehmen, aber Sie sollten am nächsten Tag eine eindeutige Verbesserung spüren und eine große Verbesserung nach einigen Wochen \hyperref[c1ii15]{(PPI)}.

Wenn Sie das Stück HT spielen können, beginnen Sie damit, es in Gedanken HT zu spielen.
Dieses HT-Üben sollte einen oder zwei Tage benötigen.
Wenn sie die Aufgabe, in Gedanken zu spielen, jetzt nicht erledigen, werden Sie es, wie die meisten Menschen, niemals tun.
Wenn Sie aber erfolgreich sind, wird es zu dem mächtigsten Werkzeug Ihres Gedächtnisses.

\textbf{Ungefähr am 5. oder 6. Tag} sollten Sie in der Lage sein, die Invention \#13 zu beginnen, so daß Sie im folgenden alle drei Stücke täglich üben können.
Eine weitere Vorgehensweise ist, zunächst nur die Invention \#8 gut zu lernen, und dann, nachdem man mit der ganzen Prozedur vertraut ist, mit \#1 und \#13 anzufangen.
Der Hauptgrund dafür, mehrere Stücke auf einmal zu lernen, ist, daß diese Stücke so kurz sind, daß man zu viele Wiederholungen an einem Tag spielt, wenn man nur eins übt.
Erinnern Sie sich daran, daß Sie ab dem ersten Tag mit der richtigen Geschwindigkeit spielen (HS), und ab dem zweiten Tag sollten Sie zumindest einige Abschnitte schneller als die endgültige Geschwindigkeit spielen.
Auch dauert es länger, diese drei Stücke nacheinander zu lernen, als sie zusammen zu lernen.

Wie schnell Sie Fortschritte machen, hängt nach dem zweiten oder dritten Tag mehr von Ihrer Fertigkeitsstufe als von den Fähigkeiten Ihres Gedächtnisses ab.
Wenn Sie erst das ganze Stück HS nach Belieben spielen können, sollten Sie das Stück als auswendig gelernt ansehen.
Wenn Sie über die mittlere Stufe hinaus sind, werden Sie sehr schnell in der Lage sein, es HT zu spielen.
Wenn Sie aber nicht derart fortgeschritten sind, werden die technischen Schwierigkeiten jeder Hand den Fortschritt verlangsamen.
Das Gedächtnis wird nicht der begrenzende Faktor sein.
Für die Arbeit am HT werden Sie offensichtlich an der Koordination der beiden Hände arbeiten müssen.
Bach hat die Inventionen so konstruiert, daß man gleichzeitig das Koordinieren der Hände und mit beiden Händen unabhängig voneinander zu spielen lernt.
Das ist der Grund, daß es zwei Stimmen gibt und sie sich überlagern; auch spielt in \#8 eine Hand staccato während die andere legato spielt.\footnote{Hier unterscheiden sich die verschiedenen Editionen.
Im Original von Invention \#8 ist kein Staccato angegeben.
Bach weist bereits in seiner Einleitung zur Reinschrift von 1723 darauf hin, daß die Stücke u.a. für die \enquote{Lehrbegierigen} geschrieben wurden.
Die zweistimmigen Inventionen und dreistimmigen Sinfonien sind demnach Lehrstücke für den Klavierspieler, und es liegt im Ermessen des Lehrers und Spielers, sie dem augenblicklichen Zweck entsprechend anzupassen; so wurden z.B. bei Invention \#1 die Figuren mit vier Sechzehntelnoten in Bachs Reinschrift nachträglich teilweise mit Durchgangsnoten ergänzt - wahrscheinlich um einem Schüler zu zeigen, daß man an diesen Stellen statt dessen auch zwei Sechzehnteltriolen spielen kann.
Gleichzeitig sind die Inventionen und Sinfonien Lehrstücke für das Komponieren (und deshalb auch gut komponierte Musik) - Zitat aus der Einleitung: \enquote{... einen starcken Vorschmack von der Composition zu überkommen})}

Alle drei oben besprochenen Stücke sollten in eins bis zwei Wochen vollständig auswendig gelernt sein, und Sie sollten anfangen, zumindest mit dem ersten Stück gut zurechtkommen.
Nehmen wir an, Sie haben sich in den zwei Wochen nur auf das Auswendiglernen dieser drei Stücke konzentriert.
Wenn Sie nun zu alten Stücken zurückkehren, die Sie zuvor auswendig gelernt haben, werden Sie feststellen, daß Sie sich weniger gut daran erinnern können.
Das ist ein guter Zeitpunkt, um sie wieder aufzupolieren und diese \hyperref[c1iii6f]{Pflegeaufgabe} mit dem weiteren Verbessern Ihrer neuen Bach-Stücke abzuwechseln.
Sie sind im Grunde fertig.
Herzlichen Glückwunsch!

Wie gut Sie aus dem Gedächtnis spielen können, hängt sowohl von Ihrer Technik als auch davon ab, wie gut Sie etwas auswendig gelernt haben.
\textbf{Es ist wichtig, den Mangel an Technik nicht mit der Unfähigkeit zum Auswendiglernen zu verwechseln.}
Die meisten Menschen mit Schwierigkeiten beim Auswendiglernen haben ein ausreichendes Gedächtnis aber eine unzulängliche Technik.
Deshalb benötigen Sie Methoden für das Testen Ihrer Technik und Ihres Gedächtnisses.
Wenn Ihre Technik ausreichend ist, sollten Sie in der Lage sein, HS entspannt ungefähr mit dem 1,5-fachen der endgültigen Geschwindigkeit zu spielen.
Bei \#8 ist die Geschwindigkeit ungefähr MM = 100 auf dem Metronom, so daß Sie in der Lage sein sollten, mit beiden Händen ungefähr mit 150 HS zu spielen.
Bei 150 haben Sie Glenn Gould geschlagen (wenn auch HS - er spielte ungefähr 140)!
Wenn Sie oberhalb von 100 nicht gut HS spielen können, dann müssen Sie Ihre Technik verbessern, bevor Sie erwarten können, irgend etwas HT nahe an 100 zu spielen.
Der beste Test für das Gedächtnis ist, festzustellen, ob Sie das Stück in Gedanken spielen können.
Wenn Sie diese Tests durchführen, können Sie herausfinden, ob Sie an der Technik oder am Gedächtnis arbeiten müssen.

Die meisten Menschen haben eine schwächere LH; bringen Sie die LH-Technik so nah an die RH-Stufe wie möglich.
Benutzen Sie, wie oben für Takt 3 der LH veranschaulicht, die Übungen für parallele Sets, um an Ihrer Technik zu arbeiten.
Bach ist besonders nützlich, um die Techniken der LH und RH auszubalancieren, weil beide Hände ähnliche Passagen spielen.
Deshalb wissen Sie sofort, daß die LH schwächer ist, wenn sie nicht auf die gleiche Geschwindigkeit kommen kann wie die RH.
Bei anderen Komponisten, wie z.B. Chopin, ist die LH gewöhnlich viel einfacher und bietet keinen guten Test für die LH.
Schüler mit unzulänglicher Technik müssen eventuell wochenlang HS üben, bevor Sie darauf hoffen können, diese Inventionen HT mit der vorgegebenen Geschwindigkeit zu spielen.
Spielen Sie in diesem Fall HT nur mit zufriedenstellenden langsamen Geschwindigkeiten und warten Sie darauf, daß sich Ihre HS-Technik entwickelt, bevor Sie die HT-Geschwindigkeit steigern.

Bachs Musik ist dafür berüchtigt, schwer schnell zu spielen und hochanfällig für \hyperref[fpd]{FPD} (\enquote{Schnellspiel-Abbau}, s. Abschnitt II.25) zu sein.
Die intuitive Lösung für dieses Problem war, geduldig langsamer zu üben.
Man muß bei vielen Kompositionen von Bach nicht sehr schnell spielen, um unter FPD zu leiden.
Wenn Ihre maximale Geschwindigkeit MM = 20 ist, während die vorgeschlagene Geschwindigkeit 100 ist, dann ist für Sie 20 schnell, und der FPD kann  bereits bei dieser Geschwindigkeit sein schreckliches Haupt erheben.
Deshalb erzeugt langsames HT-Spielen und der Versuch, es zu beschleunigen, nur mehr Verwirrung und FPD.
Wir kennen nun den Grund für den Ruf von Bachs Musik: Die Schwierigkeit resultiert aus zu vielen Wiederholungen mit langsamem HT-Spielen, die lediglich die Verwirrung vergrößern, ohne Ihr Gedächtnis oder Ihre Technik zu unterstützen.
Die bessere Lösung ist das abschnittsweise HS-Üben.
Für diejenigen, die das nie zuvor getan haben: Sie werden bald mit Geschwindigkeiten spielen, die Sie nie für möglich gehalten haben.


\label{ruhig}

\textbf{Ruhige Hände.}
Viele Lehrer bezeichnen zu Recht \enquote{ruhige Hände} als ein wünschenswertes Ziel.
\textbf{Dabei spielen hauptsächlich die Finger, und die Hände bewegen sich so wenig wie möglich.
Ruhige Hände sind der Lackmustest für den Erwerb der Technik.}
Das Eliminieren unnötiger Bewegungen gestattet nicht nur ein schnelleres Spielen, sondern erhöht auch die Kontrolle.
Viele von Bachs Stücken wurden für das Üben ruhiger Hände entwickelt.
Einige der unerwarteten Fingersätze, die auf dem Notenblatt verzeichnet sind, wurden gewählt, um für das Spielen mit ruhigen Händen passend zu sein oder es zu erleichtern.
Einige Lehrer drängen alle Schüler - auch Anfänger - dazu, jederzeit mit ruhigen Händen zu spielen, eine solche Vorgehensweise ist jedoch kontraproduktiv, da man \enquote{ruhige Hände} nicht langsam spielen kann; es gibt also keine Möglichkeit, ruhige Hände bei niedriger Geschwindigkeit zu lehren.
Der Schüler fühlt nichts und fragt sich, warum das jetzt gut sein soll.
Wenn man langsam spielt oder wenn der Schüler nicht über genügend Technik verfügt, ist eine gewisse zusätzliche Bewegung unvermeidlich und angemessen.
Die Hände unter diesen Bedingungen zur Bewegungslosigkeit zu zwingen, würde nur das Spielen erschweren und Streß erzeugen.
Diejenigen, die bereits die Technik der ruhigen Hände besitzen, können beim langsamen oder schnellen Spielen ohne Schaden viel Bewegung hinzufügen.
Manche Lehrer versuchen, ruhige Hände zu lehren, indem sie eine Münze auf die Hand legen, um zu sehen, ob die Hand ruhig genug ist, so daß die Münze nicht herunterfällt.
Diese Methode zeigt sicherlich, daß der Lehrer die Bedeutung der ruhigen Hände erkannt hat, bringt aber dem Schüler nichts.
Wenn man Bach mit der vollen Geschwindigkeit mit ruhigen Händen spielt, dann wird eine auf die Hand gelegte Münze sofort davonfliegen.
Nur wenn man jenseits einer bestimmten Geschwindigkeit spielt, werden ruhige Hände für den Klavierspieler offensichtlich und notwendig.
Wenn Sie das erste Mal ruhige Hände bekommen, ist es absolut nicht zu verkennen, machen Sie sich also keine Sorgen, daß Sie es nicht mitbekommen könnten.
Die beste Zeit, den Schüler zu lehren was ruhige Hände bedeuten, ist, wenn er genügend schnell spielt, so daß er die ruhigen Hände fühlen kann.
Wenn Sie die Technik erworben haben, dann können Sie sie auf das langsame Spielen anwenden; Sie sollten nun das Gefühl haben, daß Sie über viel mehr Kontrolle verfügen und zwischen den Noten mehr freie Zeit haben.
Somit sind ruhige Hände keine besondere Bewegung der Hand, sondern ein Gefühl der Kontrolle und der nahen völligen Abwesenheit von Geschwindigkeitsbarrieren.

Im Fall der hier besprochenen Stücke von Bach werden die ruhigen Hände bei Geschwindigkeiten nahe der endgültigen Geschwindigkeit zur Notwendigkeit; offensichtlich wurden die Geschwindigkeiten im Hinblick auf die ruhigen Hände ausgewählt.
Ohne ruhige Hände werden Sie bei den empfohlenen Geschwindigkeiten auf Geschwindigkeitsbarrieren treffen.
HS-Üben ist für ruhige Hände wichtig, weil sie viel leichter zu erwerben und zu fühlen sind, wenn man HS spielt, und weil HS-Spielen es erlaubt, viel schneller als mit HT zu den Geschwindigkeiten für ruhige Hände zu gelangen.
Tatsächlich ist es am besten, erst mit HT anzufangen, wenn man mit jeder der beiden Hände ruhig spielen kann, weil das die Wahrscheinlichkeit verringert, daß man schlechte Angewohnheiten verfestigt.
Das bedeutet, HT mit oder ohne ruhige Hände macht einen Unterschied, so daß man sich nicht angewöhnen sollte, HT ohne ruhige Hände zu spielen.
Diejenigen mit ungenügender Technik benötigen eventuell zu lange, um ruhige Hände zu erreichen, so daß solche Schüler eventuell HT ohne ruhige Hände anfangen müssen; sie können die ruhigen Hände später schrittweise erwerben, indem sie mehr HS üben.
Das erklärt, warum diejenigen mit genügender Technik diese Inventionen so viel schneller lernen können als diejenigen ohne.
Solche Schwierigkeiten sind einige der Gründe dafür, nicht zu versuchen, Stücke zu lernen, die zu schwierig für Sie sind, und bieten nützliche Tests dafür, ob die Komposition zu schwierig oder für Ihre Fertigkeitsstufe angemessen ist.
Diejenigen mit ungenügender Technik werden mit Sicherheit riskieren, Geschwindigkeitsbarrieren aufzubauen.
Obwohl manche behaupten, daß die Bach-Inventionen \enquote{mit jeder Geschwindigkeit} gespielt werden können, stimmt das nur für den musikalischen Gehalt; diese Kompositionen müssen mit ihrer empfohlenen Geschwindigkeit gespielt werden, um den vollen Nutzen aus der Klavierlektion zu ziehen, die Bach im Sinn hatte.
In diesem Abschnitt wird die Geschwindigkeit wegen der Notwendigkeit, ruhige Hände zu erklären und zu erwerben, überbetont;
üben Sie jedoch nicht die Geschwindigkeit um der Geschwindigkeit willen, da dies wegen des Stresses und den schlechten Angewohnheiten nicht funktionieren wird; musikalisches Spielen ist immer noch der beste Weg, die Geschwindigkeit zu steigern - siehe Abschnitt III.7i.

Bei denjenigen mit einer stärkeren RH, wird die RH auch als erste ruhig; wenn Sie das Gefühl erst einmal kennen, können Sie es schneller auf die LH übertragen.
Wenn es einsetzt, werden Sie plötzlich merken, daß es leichter wird, schnell zu spielen.
Deshalb funktioniert das HT-Üben beim Lernen von neuen Bach-Stücken nicht - es gibt keine Möglichkeit, schnell zu ruhigen Händen zu kommen.

Bach hat diese Inventionen für die technische Entwicklung geschrieben.
Deshalb hat er beiden Händen gleich schwieriges Material gegeben; das stellt für die LH eine größere Herausforderung dar, weil die Baß-Hämmer und -Saiten schwerer sind.
Bach hätte sich geärgert, wenn er Übungen wie die \hyperref[c1iii7h]{Hanon-Serie} gesehen hätte, weil er wußte, daß Übungen ohne Musik eine Zeitverschwendung sind. Dies wird durch den Aufwand deutlich, den er in diese Kompositionen steckte, um die Musik einzubeziehen.
Die Menge des technischen Materials, das er in diese Kompositionen packte, ist unglaublich: Unabhängigkeit der Finger (ruhige Hände, Kontrolle, Geschwindigkeit), sowohl Koordination als auch Unabhängigkeit der beiden Hände (mehrere Stimmen, staccato gegen legato, kollidierende Hände, Verzierungen), Harmonien, Musik erzeugen, sowohl die LH als auch die schwachen Finger (4 und 5) stärken, alle hauptsächlichen parallelen Sets, Benutzungsarten des Daumens, Standard-Fingersätze usw.
Beachten Sie, daß die Verzierungen \hyperref[c1iii7b]{Übungen für parallele Sets} und nicht nur musikalische Verzierungen, sondern ein integraler Bestandteil der technischen Entwicklung sind.
Durch die Verzierungen verlangt Bach von Ihnen, daß Sie mit der einen Hand parallele Sets üben, während Sie gleichzeitig mit der anderen Hand einen anderen Teil spielen und mit dieser Kombination Musik erzeugen!

Achten Sie darauf, Bach nicht zu laut zu spielen, auch wenn \textit{f} angezeigt ist.
Die Instrumente seiner Zeit erzeugten viel weniger Klang als moderne Klaviere, so daß Bach Musik schreiben mußte, die mit Klang gefüllt ist und wenige Pausen hat.
Ein Zweck der zahlreichen Verzierungen und Triller war zu Bachs Zeiten, den Klang auszufüllen.
Deshalb neigt seine Musik dazu, zu viel Klang zu haben, wenn sie auf modernen Klavieren laut gespielt wird.
Besonders bei den Inventionen und Sinfonien, in denen der Schüler versucht, alle die konkurrierenden Melodien herauszubringen, besteht die Neigung, jede folgende Melodie lauter zu spielen, was zu lauter Musik führt.
Die verschiedenen Melodien müssen auf der Basis des musikalischen Konzepts miteinander konkurrieren, nicht durch die Lautstärke.
Leiser zu spielen trägt auch zum Erreichen einer völligen Entspannung und der wahren Unabhängigkeit der Finger bei.

Wenn Sie eine der Sinfonien (dreistimmigen Inventionen) lernen möchten, könnten Sie \#15 probieren, die leichter als die meisten anderen ist.
Sie ist sehr interessant und hat einen Abschnitt in der Mitte, in dem die beiden Hände kollidieren und oft dieselben Noten spielen.
Wie alle Bach-Kompositionen enthält diese viel mehr als man auf den ersten Blick sieht.
Gehen Sie deshalb vorsichtig zu Werke.
Vor allem ist es allegro vivace!
Die Taktbezeichnung ist ein fremdartiges 9/16, was bedeutet, daß die Gruppen von sechs 32tel-Noten im dritten Takt als 3 Schläge gespielt werden müssen und nicht als 2 (drei Notenpaare anstelle von zwei Triolen).
Diese Taktbezeichnung führt zu den drei wiederholten Noten (zwei im dritten Takt), die thematischen Wert haben und in einer für Bach charakteristischen Weise über die Tastatur wandern.
Wenn die beiden Hände in Takt 28 kollidieren, heben Sie die RH und lassen Sie die LH daruntergleiten.
Spielen Sie alle Noten mit beiden Händen.
Falls die Kollision der Daumen problematisch ist, können Sie den RH-Daumen weglassen und nur mit dem LH-Daumen spielen.
Achten Sie darauf, daß Sie in Takt 36 den richtigen Fingersatz für die RH benutzen: (5),(2,3),(1,4),(3,5),(1,4),(2,3).

\textbf{Lassen Sie uns zum Schluß den letzen notwendigen Schritt zum Auswendiglernen besprechen: das Analysieren der Struktur bzw. der \enquote{Geschichte} hinter der Musik.}
Der Merkprozeß ist so lange unvollständig, bis Sie die Geschichte hinter der Musik verstehen.
Wir werden Invention \#8 benutzen.
Die ersten 11 Takte bilden die \enquote{Einführung}.
Hierbei spielen die RH und die LH im Grunde dasselbe, die LH jeweils einen Takt verzögert, und das Hauptthema wird eingeführt.
Der \enquote{Hauptteil} besteht aus den Takten 12 bis 28, in denen die Rollen der beiden Hände zunächst vertauscht sind, die LH führt die RH, gefolgt von einigen faszinierenden Entwicklungen.
Das \enquote{Ende} beginnt mit Takt 29 und bringt das Stück zu einem ordnungsgemäßen Abschluß, bei dem die ursprüngliche Rolle der RH erneut bekräftigt wird.
Beachten Sie, daß das Ende das gleiche ist wie das Ende der Einführung - das Stück endet effektiv zweimal, was das Ende überzeugender macht.
Beethoven hat dieses Mittel, ein Stück mehrfach zu beenden, weiterentwickelt und in unglaubliche Höhen geführt.

Wir präsentieren nun einige Erklärungen dafür, warum die Entwicklung einer solchen \enquote{Geschichte} - alle großen Musiker haben Ihre Musik auf eine bestimmte Weise aufgebaut - die beste und vielleicht einzig zuverlässige Art ist, eine Komposition dauerhaft auswendig zu lernen.



