% File: c31

\chapter{Wissenschaftliche Methode, Theorie des Lernens und das Gehirn}
\label{c3_1}

\subsection{Einleitung}

Der erste Teil dieses Kapitels beschreibt meine Vorstellung davon, was eine wissenschaftliche Methode ist, und wie ich sie benutzt habe, um dieses Buch zu schreiben.
Dieser wissenschaftliche Ansatz ist der Hauptgrund, warum sich dieses Buch von allen anderen Büchern über das Thema des Klavierspielenlernens unterscheidet.

Die anderen Abschnitte behandeln Themen des Lernens im allgemeinen, und die Gleichung für die Lernrate wird hergeleitet.
Das ist die Gleichung, die benutzt wurde, um die Lernraten in \hyperref[c1iv5]{\autoref{c1iv5}} zu berechnen.
Ich bespreche auch Themen, die das Gehirn betreffen, weil das Gehirn offensichtlich ein integraler Bestandteil des Spielmechanismus ist.
Mit Ausnahme der anfänglichen Diskussion darüber, wie sich das Gehirn im Laufe des Älterwerdens entwickelt und wie diese Entwicklung das Lernen des Klavierspielens beeinflußt, haben die Themen über das Gehirn jedoch nur eine geringe direkte Verbindung zum Klavier.
Die Rolle des Gehirns beim Lernen des Klavierspielens muß natürlich viel mehr erforscht werden.
Ich habe auch eine Diskussion über die Interpretation von Träumen eingefügt, die mehr Licht in die Frage bringt, wie das Gehirn arbeitet.
Zum Schluß beschreibe ich meine Erfahrungen mit meinem Unterbewußtsein, welches mir in zahlreichen Fällen gute Dienste geleistet hat.
 

\subsection{Der wissenschaftliche Ansatz}
\label{c3_2}

\subsubsection{Einleitung}
\label{c3_2a}

Dieses Buch wurde mit dem besten mir möglichen wissenschaftlichen Ansatz geschrieben, wobei ich benutzt habe, was ich während meiner 31-jährigen Karriere als Wissenschaftler lernte.
Ich befaßte mich nicht nur mit Grundlagenforschung (es wurden mir sechs Patente erteilt), sondern auch mit Materialwissenschaft (Mathematik, Physik, Chemie, Biologie, Maschinenbau, Elektronik, Optik, Akustik, Metallen, Halbleitern, Isolatoren), industrieller Problemlösung (Fehlermechanismen, Ausfallsicherheit, Fertigung) und wissenschaftlichen Veröffentlichungen (ich habe über 100 gegengeprüfte Artikel in den meisten großen Wissenschaftsmagazinen veröffentlicht).
Sogar nachdem ich meinen Doktortitel in Physik von der Cornell University verliehen bekam, investierten meine Arbeitgeber im Laufe meiner Karriere über eine Million Dollar, um meine Ausbildung zu fördern.
Zurückblickend war diese ganze wissenschaftliche Ausbildung für das Schreiben dieses Buchs unentbehrlich.
Diese Notwendigkeit, die wissenschaftliche Methode zu verstehen, läßt darauf schließen, daß es den meisten Klavierspielern schwerfallen würde, das gleiche Ergebnis zu erzielen.
Ich erkläre unten genauer, daß die Ergebnisse wissenschaftlicher Anstrengungen für jeden nützlich sind, nicht nur für Wissenschaftler.
\textbf{Deshalb bedeutet die Tatsache, daß dieses Buch von einem Wissenschaftler geschrieben wurde, daß jeder in der Lage sein sollte, es leichter zu verstehen, als wenn es nicht von einem Wissenschaftler geschrieben wäre.}
Ein Ziel dieses Abschnitts ist es, diese Botschaft zu erläutern.


\subsubsection{Lernen}
\label{c3_2b}

Klavier, Algebra, Bildhauerei, Golf, Physik, Biologie, Quantenmechanik, Tischlerei, Kosmologie, Medizin, Politik, Wirtschaftswissenschaft usw. - was haben diese gemeinsam?
Sie sind alle wissenschaftliche Disziplinen und haben deshalb eine große Zahl grundlegender Prinzipien gemeinsam.
In den folgenden Abschnitten \textbf{werde ich viele der wichtigen Prinzipien der wissenschaftlichen Methode erklären und zeigen, wie sie für das Erzeugen eines nützlichen Produkts benötigt werden}, z.B. für ein Handbuch zum Lernen des Klavierspielens.
Diese Erfordernisse für ein Klavierbuch unterscheiden sich nicht von den Erfordernissen für das Schreiben eines fortgeschrittenen Lehrbuchs über Quantenmechanik; die Erfordernisse sind ähnlich, obwohl die Inhalte Welten voneinander entfernt sind.
Ich beginne mit der Definition der wissenschaftlichen Methode, weil sie von der Öffentlichkeit so oft mißverstanden wird.
Danach beschreibe ich den Beitrag der wissenschaftlichen Methode zum Schreiben dieses Buchs.
Bei dieser Gelegenheit stelle ich heraus, wann die Klavierlehre in der Vergangenheit wissenschaftlich oder unwissenschaftlich war.
Während der letzten Jahrhunderte gab es durch das Anwenden der wissenschaftlichen Methode auf fast alle wichtigen Disziplinen enorme Fortschritte; ist es nicht an der Zeit, daß wir dasselbe mit dem Lernen und Unterrichten des Klavierspielens tun?

Dieser Abschnitt wurde hauptsächlich geschrieben, um die wissenschaftliche Methode zu skizzieren, in der Hoffnung, anderen dabei zu helfen, sie auf den Klavierunterricht anzuwenden.
Ein weiteres Ziel ist, zu erklären, warum es einen Wissenschaftler wie ich einer bin erforderte, um zu einem solchen Buch zu kommen.
Warum konnten Musiker ohne wissenschaftliche Ausbildung nicht bessere Bücher über das Klavierlernen schreiben?
Schließlich sind sie die führenden Experten auf diesem Gebiet!
Ich werde unten ein paar der Antworten darauf geben.

Ich vermute, Sie werden beim Lesen der folgenden Abschnitte Konzepte finden, die sich von Ihren Vorstellungen von der Wissenschaft unterscheiden.
\textbf{Wissenschaft an sich besteht nicht aus Mathematik, Physik oder Gleichungen.
Sie befaßt sich mit menschlichen Interaktionen, die andere Menschen zu etwas befähigen} (s.u.).
Ich habe viele \enquote{Wissenschaftler} gesehen, die nicht verstehen was Wissenschaft ist, und deshalb in ihrer eigenen Berufung versagten (d.h. entlassen wurden).
So wie täglich 8 Stunden zu üben Sie nicht notwendigerweise zu einem vollendeten Pianisten werden läßt, macht Sie das Bestehen aller Physik- und Chemieexamen nicht zu einem Wissenschaftler; Sie müssen etwas mit diesem Wissen vollbringen.
Ich war von vielen Klaviertechnikern besonders beeindruckt, die ein praktisches Verständnis der Physik haben, obwohl sie kein Wissenschaftsdiplom haben.
Diese Techniker müssen wissenschaftlich sein, weil das Klavier so tief in der Physik verwurzelt ist.
So definieren Mathematik, Physik usw. nicht die Wissenschaft (ein verbreitetes Mißverständnis); diese Gebiete haben sich lediglich als nützlich für Wissenschaftler erwiesen, weil sie in einer absolut vorhersagbaren Weise befähigen.
\textbf{Ich habe vor, Ihnen im folgenden die Ansicht eines Insiders darüber zu zeigen, wie Wissenschaft ausgeführt wird.}

Kann jemand, der keinerlei wissenschaftliche Ausbildung besitzt, das folgende lesen und sofort damit beginnen, den wissenschaftlichen Ansatz zu benutzen?
Wahrscheinlich nicht.
Es gibt keinen anderen Weg, als Wissenschaft zu studieren.
Sie werden sehen, daß die Erfordernisse und Komplexitäten der wissenschaftlichen Methode die meisten Menschen vor unüberwindbare Schwierigkeiten stellen.
Das ist natürlich eine Erklärung dafür, daß dieses Buch so einmalig ist.
Sie werden aber zumindest eine Vorstellung davon bekommen, was einige der nützlichen Vorschläge sind, wenn Sie den wissenschaftlichen Ansatz verfolgen möchten.

Lassen Sie uns, bevor wir die Definition der Wissenschaft in Angriff nehmen, ein verbreitetes Beispiel dafür untersuchen, wie Menschen die Wissenschaft mißverstehen, weil uns das ermitteln hilft, warum wir eine Definition benötigen.
Sie können einen Klavier- oder Tanzlehrer sagen hören, daß er ein Gefühl beschreibt, den Flug eines Vogels oder die Bewegung einer Katze, und seine Schüler bekommen sofort auf eine Art eine Vorstellung davon, wie sie spielen oder tanzen müssen, die der Lehrer unmöglich erreicht hätte, wenn er die Bewegung der Knochen, Muskeln, Arme usw. beschrieben hätte.
Der Lehrer behauptet dann, daß die Vorgehensweise des Künstlers besser ist als die wissenschaftliche.
Dieser Lehrer bemerkt nicht, daß er wahrscheinlich eine sehr gute wissenschaftliche Methode benutzt hat.
Indem man eine Analogie herstellt oder das Endprodukt der Musik beschreibt, kann man oft viel mehr Informationen übermitteln als durch das detaillierte Beschreiben jeder Komponente der Bewegung.
Es ist so, als ob man von Schmalband- zu Breitbandkommunikation übergeht und ist ein gültiges wissenschaftliches Vorgehen; es hat wenig mit der Unterscheidung zwischen Wissenschaft und Kunst zu tun.
Diese Art von Mißverständnis entsteht oft, weil die Menschen glauben, daß Wissenschaft schwarz oder weiß ist - daß etwas entweder wissenschaftlich ist oder nicht; die meisten Dinge im richtigen Leben sind mehr oder weniger wissenschaftlich, es ist nur eine Frage des Ausmaßes.
Was diese Lehrmethoden wissenschaftlicher macht oder nicht, hängt davon ab, wie gut sie die notwendigen Informationen übermitteln.
In dieser Hinsicht sind viele berühmte Künstler, die gute Lehrer sind, Meister dieser Art von Wissenschaft.
Ein weiteres häufiges Mißverständnis ist, daß Wissenschaft zu schwierig für Künstler sei.
Das verwundert doch sehr.
Die geistigen Prozesse, die Künstler beim Erzeugen der höchsten Stufen von Musik oder anderen Künsten durchlaufen, sind mindestens so komplex wie jene von Wissenschaftlern, die über den Ursprung des Universums nachdenken.
Das Argument, daß die Menschen mit unterschiedlichen Talenten für Kunst oder Wissenschaft geboren werden, mag teilweise gültig sein; ich stimme dieser Ansicht jedoch nicht zu - für den größten Teil der Menschen gilt, daß sie Künstler oder Wissenschaftler sein können, je nachdem in welchem Ausmaß sie, besonders in früher Kindheit, mit jedem Gebiet in Berührung gekommen sind.
Deshalb haben die meisten Menschen, die gute Musiker sind, die Fähigkeit, große Wissenschaftler zu sein.
Wenn man sein ganzes Leben Kunst studieren würde, hätte man nicht viel Zeit Wissenschaft zu studieren, wie kann man also beides miteinander kombinieren?
So wie ich es verstehe, sind Kunst und Wissenschaft komplementär; die Kunst hilft den Wissenschaftlern und umgekehrt.
Künstler, die der Wissenschaft aus dem Weg gehen, schaden sich nur selbst, und Wissenschaftler, die der Kunst aus dem Weg gehen, neigen dazu, weniger erfolgreiche Wissenschaftler zu sein.
Was mich in meiner Zeit am College am meisten beeindruckte, war die große Zahl von Wissenschaftsstudenten, die Musiker waren.
 


