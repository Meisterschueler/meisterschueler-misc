% File: c1iv5

\subsection{Berechnung der Lernrate}
\label{c1iv5}

Es folgt mein grober Entwurf zur Berechnung der Lernrate der Methoden dieses Buchs.
Das Ergebnis deutet darauf hin, daß sie ungefähr 1000 mal schneller sind als die intuitive Methode.
Der große Faktor von 1000 macht es unnötig, den genauen Wert zu ermitteln, um zu zeigen, daß ein großer Unterschied besteht.
Das Ergebnis erscheint angesichts der Tatsache plausibel, daß viele Schüler, die ihr ganzes Leben lang hart gearbeitet und die intuitive Methode benutzt haben, nicht in der Lage sind, irgend etwas bedeutsames vorzuspielen, während ein glücklicher Schüler, der die richtigen Lernmethoden benutzt, in weniger als 10 Jahren ein Konzertpianist werden kann.
Es ist klar, daß der Unterschied in den Übungsmethoden den Unterschied zwischen einem Leben voller Frustrationen und einer lohnenden Karriere am Klavier ausmachen kann.
Nun bedeutet \enquote{1000 mal schneller} nicht, daß man innerhalb einer Millisekunde ein Pianist werden kann; es bedeutet, daß die intuitiven Methoden 1000 mal \textit{langsamer} sind als die guten Methoden.
Der Schluß, den wir daraus ziehen sollten, ist, daß unsere Lernrate mit den richtigen Methoden sehr nahe an denen von berühmten Komponisten wie Mozart, Beethoven, Liszt und Chopin liegen sollte.
Erinnern Sie sich daran, daß wir bestimmte Vorteile haben, die sich diesen verstorbenen Genies nicht boten.
Sie hatten nicht diese wundervollen Beethoven-Sonaten, Liszt- und Chopin-Etüden, usw., mit denen wir heute die Technik erwerben, oder die Kompositionen von Mozart, mit denen wir vom \enquote{Mozart-Effekt} profitieren, oder Bücher wie dieses, mit einer geordneten Liste von Übungsmethoden.
Zudem gibt es nun hunderte bewährter Methoden, um diese Kompositionen für den Technikerwerb zu nutzen (Beethoven hatte oft Schwierigkeiten, seine eigenen Kompositionen zu spielen, weil niemand die richtigen oder falschen Methoden kannte, sie zu üben).
Eine faszinierende historische Anmerkung ist, daß das einzige allgemein verfügbare Material zum Üben für alle diese großen Pianisten Bachs Kompositionen waren.
Das führt uns zu dem Gedanken, daß Bach zu studieren ausreichend sein mag, um die meisten grundlegenden Fertigkeiten des Klavierspielens zu erwerben.

Ich werde versuchen, eine detaillierte Berechnung durchzuführen, indem ich mit den fundamentalsten Prinzipien anfange und bis zum Endresultat voranschreite ohne unbekannte Schritte auszulassen.
Auf diese Weise können eventuelle Fehler in dieser Berechnung bereinigt werden, wenn wir unser Verständnis darüber, wie wir die Technik erwerben, verbessern.
Das ist, offensichtlich, der wissenschaftliche Ansatz.
Es gibt nichts neues in diesen Berechnungen, außer daß sie auf das musikalische Lernen angewandt werden.
Das mathematische Material ist einfach ein Rückgriff auf bekannte Algebra und Infinitesimalrechnung.

Die Mathematik kann benutzt werden, um Probleme auf die folgende Weise zu lösen.
Als erstes definiert man die Bedingungen, die die Natur des Problems bestimmen.
Wenn diese Bedingungen korrekt bestimmt wurden, gestatten Sie es, Differentialgleichungen aufzustellen; diese sind getreue, mathematische Aussagen über die Bedingungen.
Wenn die Differentialgleichungen aufgestellt sind, bietet die Mathematik Methoden sie zu lösen, um eine Funktion zur Verfügung zu stellen, die die Antworten auf die Probleme durch die Parameter beschreibt, die diese Antworten bestimmen.
Die Lösungen der Probleme können dann durch Einsetzen der geeigneten Parameterwerte in die Funktion berechnet werden.

Das physikalische Prinzip, das wir benutzt haben, um unsere Lerngleichung herzuleiten, ist die Linearität mit der Zeit.
Solch ein abstraktes Konzept mag so erscheinen, als hätte es nichts mit dem Klavier zu tun und ist sicherlich unbiologisch, es stellt sich aber heraus, daß das genau das ist, was wir brauchen.
Lassen Sie mich also das Konzept der \enquote{Linearität mit der Zeit} erklären.
Es bedeutet einfach proportional zur Zeit.
Wenn wir z.B. in der Zeit $T$ eine Menge Technik $L$ lernen ($L$ steht für Lernen), dann sollten wir, wenn wir diesen Prozeß ein paar Tage später wiederholen, eine weitere Menge $L$ in der gleichen Zeit $T$ lernen.
Deshalb sagen wir, daß $L$ in bezug auf $T$ in dem Sinne linear ist, daß sie proportional sind; in $2T$ sollten wir $2L$ lernen.
Natürlich wissen wir, daß Lernen in hohem Maß nichtlinear ist.
Wenn wir denselben kurzen Abschnitt 4 Stunden lang üben, dann werden wir wahrscheinlich während der ersten 30 Minuten mehr erreichen als während der letzten 30 Minuten.
Wir sprechen aber über eine optimierte Übungssitzung, die einen Durchschnitt vieler Übungssitzungen darstellt, die über einen Zeitraum von mehreren Jahren ausgeführt wurden (in einer optimierten Übungssitzung werden wir nicht dieselben 4 Noten 4 Stunden lang üben!).
Wenn wir den Durchschnitt über all diese Lernprozesse bilden, dann neigen sie dazu ziemlich linear zu sein.
Innerhalb eines Faktors von 2 oder 3 ist Linearität sicherlich eine gute Näherung, und dieses Maß an Genauigkeit ist alles was wir brauchen.
Beachten Sie, daß die Linearität in der ersten Näherung nicht davon abhängt, ob man ein schneller oder langsamer Lerner ist; das ändert nur die Proportionalitätskonstante.
Deshalb kommen wir zur ersten Gleichung

\begin{equation}
 \label{eq:Lkt}
 L = kT
\end{equation}
%<h5>(Gleichung 1.1)</h5>

wobei $L$ die Zunahme des Lernens im Zeitintervall $T$ und $k$ die Proportionalitätskonstante ist.
Wir versuchen, die Abhängigkeit von $L$ von der Zeit zu finden, oder $L(t)$, wobei $t$ die Zeit ist (im Gegensatz zu $T$, was ein Zeitintervall ist).
Genauso ist $L$ eine Zunahme des Lernens, während $L(t)$ eine Funktion ist.

Nun kommt das erste interessante neue Konzept.
Wir können $L$ kontrollieren; wenn wir $2L$ möchten, üben wir einfach zweimal.
Aber das ist nicht das $L$, das wir behalten, weil wir im Laufe der Zeit etwas $L$ \textit{verlieren}, nachdem wir üben.
Leider können wir um so mehr vergessen, je mehr wir wissen; d.h. die Menge, die wir vergessen, ist zu der ursprünglichen Menge Wissen L(U) proportional.
Angenommen, wir haben $L(U)$ erworben, ist die Menge $L$, die wir in $T$ verlieren, deshalb:

\begin{equation}
 \label{eq:LkTLU}
 L = -kTL(U)
\end{equation}
%<h5>(Gleichung 1.2)</h5>

wobei die \enquote{k}s in den Gleichungen \ref{eq:Lkt} und \ref{eq:LkTLU} unterschiedlich sind, aber wir benennen sie der Einfachheit halber nicht um.
Beachten Sie, daß $k$ ein negatives Vorzeichen hat, weil wir $L$ verlieren.
\autoref{eq:LkTLU} führt zu der Differentialgleichung

\begin{equation}
 \label{eq:dLtdt}
 dL(t)/dt = -kL(t)
\end{equation}
%<h5>(Gleichung 1.3)</h5>

wobei $d$ für Differential steht (das ist alles normale Infinitesimalrechnung), und die Lösung dieser Differentialgleichung ist:

\begin{equation}
\label{eq:LtKekt}
 L(t)=Ke^{-kt}
\end{equation}
%<h5>(Gleichung 1.4)</h5>

wobei $e$ die Basis des natürlichen Logarithmus ist (ca. 2,71828), und $K$ ist eine neue Konstante, die mit $k$ verbunden ist (der Einfachheit halber haben wir einen weiteren Term in der Lösung ignoriert, der im jetzigen Stadium unwichtig ist).
\autoref{eq:LtKekt} sagt uns, daß wir, wenn wir $L$ gelernt haben, sofort anfangen, es exponentiell zur Zeit zu vergessen, wenn der Prozeß des Vergessens linear zur Zeit ist.

Da der Exponent nur eine Zahl ist, hat $k$ in \autoref{eq:LtKekt} die Einheit $1/Zeit$.
Wir werden $k = 1/T(k)$ setzen, wobei $T(k)$ die charakteristische Zeit genannt wird.
Hier bezieht sich $k$ auf einen spezifischen Prozeß des Lernens und Vergessens.
Wenn wir Klavierspielen lernen, dann lernen wir durch eine Unzahl von Prozessen, von denen die meisten nicht vollständig verstanden werden.
Deshalb ist es im allgemeinen nicht möglich, für jeden Prozeß genaue Werte für $T(k)$ zu bestimmen, so daß wir in den Berechnungen ein paar \enquote{intelligente Schätzungen} machen müssen.
Beim Klavierüben müssen wir schwieriges Material viele Male wiederholen, bevor wir es gut spielen können, und wir müssen jeder Wiederholung beim Üben eine Zahl (sagen wir $i$) zuordnen.
Dann wird \autoref{eq:LtKekt} zu

\begin{equation}
\label{eq:Litk}
 L(i,t,k)=K(i)e^{-t(i)/T(k)}
\end{equation}
%<h5>(Gleichung 1.5)</h5>

für jede Wiederholung $i$ und jeden \enquote{Lernen/Vergessen}-Prozeß $k$.
Lassen Sie uns ein paar relevante Beispiele untersuchen.
Angenommen, Sie üben nacheinander 4 Noten eines parallelen Sets, spielen schnell und wechseln die Hände, usw. 10 Minuten lang.
Wir weisen der Ausführung eines parallelen Sets, das nur ungefähr eine halbe Sekunde dauert $i = 0$ zu.
Sie haben das eventuell zehn- oder hundertmal während der zehnminütigen Übungseinheit wiederholt.
Sie haben $L(U)$ nach dem ersten parallelen Set gelernt.
Wir müssen aber den Betrag von $L(U)$ berechnen, den wir nach der zehnminütigen Übungseinheit behalten.
Da wir viele Male wiederholen, müssen wir in Wahrheit das kumulierte Lernen von allen berechnen.
Gemäß \autoref{eq:Litk}, ist dieser kumulative Effekt durch die Summe aller \enquote{L}s über alle Wiederholungen des parallelen Sets gegeben:

\begin{equation}
\label{eq:sumL}
 L(Total)=\sum_i K(i)e^{t(i)/T(k)}
\end{equation}
%<h5>(Gleichung 1.6)</h5>

Lassen Sie uns nun ein paar Werte in \autoref{eq:sumL} einsetzen, damit wir ein paar Antworten bekommen.
Nehmen Sie eine Passage, die Sie (mit der intuitiven Methode) langsam in ungefähr 100 Sekunden HT spielen können.
Diese Passage enthält vielleicht 2 oder 3 parallele Sets, die schwierig sind und die Sie in weniger als einer Sekunde schnell spielen können, so daß Sie die Sets in diesen 100 Sekunden (mit den Methoden dieses Buchs) mehr als hundertmal wiederholen können.
Typischerweise sind diese 2 oder 3 Stellen die einzigen, die sie ausbremsen, so daß Sie, wenn Sie diese gut spielen können, die ganze Passage mit der endgültigen Geschwindigkeit spielen können.
Natürlich werden Sie die Stellen auch mit der intuitiven Methode viele Male wiederholen, aber lassen Sie uns den Unterschied im Lernen für jede der 100 Sekunden dauernden Wiederholungen vergleichen.
Für diesen schnellen Lernprozeß ist unsere Neigung das Gelernte zu \enquote{verlieren} ebenfalls schnell, so daß wir eine \enquote{Vergessenszeitkonstante} von ungefähr 30 Sekunden annehmen können; d.h. alle 30 Sekunden vergessen Sie fast 30\% von dem, was Sie von einer Wiederholung gelernt haben.
Beachten Sie, daß Sie auch nach langer Zeit niemals alles vergessen, weil der Prozeß des Vergessens exponentiell ist -- exponentielle Abfälle erreichen niemals den absoluten Nullpunkt.
Auch können Sie mit parallelen Sets innerhalb kurzer Zeit viele Wiederholungen ausführen, so daß sich diese Lernereignisse schnell ansammeln.
Diese Vergessenszeitkonstante von 30 Sekunden hängt von dem Mechanismus des Lernens und Vergessens ab, und ich habe eine relativ kurze für schnelle Wiederholungen ausgewählt; wir werden unten eine viel längere untersuchen.

Eine Wiederholung von einem parallelen Set pro Sekunde angenommen, ist die Lernmenge durch die erste Wiederholung $e^{-100/30} \approx 0,04$ (Sie haben 100 Sekunden, um die erste Wiederholung zu vergessen), während Ihnen die letzte Wiederholung $e^{-1/30} \approx 0,97$ gibt und die durchschnittliche Lernmenge ungefähr dazwischen liegt: ungefähr 0,4 (wie wir sehen werden, müssen wir nicht genau sein). Wir kommen so mit Hilfe der parallelen Sets bei mehr als 100 Wiederholungen auf eine gelernte Menge von mehr als 40.
Bei der intuitiven Methode haben wir eine einzige Wiederholung oder $e^{-100/30} \approx 0,04$.
Der Unterschied ist ein Faktor von 40/0,04 = 1.000!
Bei einem solch großen Faktor brauchen wir keine große Genauigkeit, um zu zeigen, daß es einen großen Unterschied gibt.
Die tatsächliche Differenz in der Lernmenge kann sogar noch größer sein, weil die Wiederholung mit der intuitiven Methode mit niedriger Geschwindigkeit stattfindet, während die Wiederholrate der parallelen Sets mit der endgültigen Geschwindigkeit oder sogar einer noch höheren Geschwindigkeit erzielt wird.

Die Zeitkonstante von 30 Sekunden, die oben benutzt wurde, war für einen \enquote{schnellen} Lernprozeß, wie er mit dem Lernen \textit{während} einer einzelnen Übungseinheit verbunden ist.
Es gibt viele andere, wie den Technikerwerb durch die \hyperref[c1ii15]{Automatische Verbesserung nach dem Üben (PPI)}\index{Automatische Verbesserung nach dem Üben (PPI)}.
Nach jeder strengen Konditionierung wird sich Ihre Technik durch die PPI für eine Woche oder mehr verbessern.
Die Rate des Vergessens, oder des Technikverlusts, beträgt bei einem solch langsamen Prozeß nicht 30 Sekunden sondern viel mehr, wahrscheinlich mehrere Wochen.
Um die gesamte Differenz in der Lernrate zu berechnen, müssen wir deshalb die Differenz für alle bekannten Methoden des Technikerwerbs berechnen, wobei wir die entsprechende Zeitkonstante benutzen, die von Methode zu Methode beträchtlich abweichen kann.
Die PPI wird in hohem Maß durch das Konditionieren bestimmt, und das Konditionieren ist der oben berechneten Wiederholung der parallelen Sets ähnlich.
Deshalb sollte der Unterschied in der PPI ebenfalls ungefähr 1000 betragen.

Wenn wir die wichtigsten Raten wie oben beschrieben berechnen, können wir die Ergebnisse verfeinern, indem wir andere Faktoren berücksichtigen, die die endgültigen Ergebnisse beeinflussen.
Es gibt Faktoren, die die Methoden dieses Buchs langsamer machen (das Auswendiglernen kann zunächst länger dauern als das Spielen vom Blatt, oder HS kann länger dauern als HT, weil man jede Passage dreimal statt einmal lernen muß, usw.) und Faktoren, die sie schneller machen (wie das Lernen in kurzen Abschnitten, schnell auf Geschwindigkeit kommen, Geschwindigkeitsbarrieren vermeiden, usw.).
Es gibt viele weitere Faktoren, die die intuitiven Methoden langsamer machen, so daß das obige Resultat \enquote{1000 mal schneller} eine Unterschätzung sein kann.
Es ist jedoch wahrscheinlich nicht möglich, den vollen Vorteil aus dem 1000fachen Faktor zu ziehen, da die meisten Schüler bereist einige der Ideen dieses Buchs benutzen werden.

Die Auswirkungen der Geschwindigkeitsbarrieren sind schwer zu berechnen, weil Geschwindigkeitsbarrieren von jedem Klavierspieler künstlich erzeugt werden und ich nicht weiß, wie ich eine Gleichung dafür schreiben soll.
Die Erfahrung zeigt uns, daß die intuitive Methode für Geschwindigkeitsbarrieren anfällig ist.
Die Methoden dieses Buchs bieten viele Möglichkeiten, sie zu vermeiden.
Zudem werden die Geschwindigkeitsbarrieren hier klar definiert, so daß es möglich ist, sie während des Übens von vornherein zu vermeiden.
Parallele Sets sind das mächtigste Werkzeug, um sie zu vermeiden, weil Geschwindigkeitsbarrieren im allgemeinen nicht entstehen, wenn man die Geschwindigkeit von hoher Geschwindigkeit aus verringert.
Deshalb verzögern Geschwindigkeitsbarrieren die Lernrate für intuitive Methoden in hohem Maß.
Einige Lehrer, die die Geschwindigkeitsbarrieren nicht ausreichend verstehen, verbieten ihren Schülern, etwas gewagtes und schnelles zu üben, was den Fortschritt noch mehr verlangsamt, sogar wenn dieses langsame Spielen die Geschwindigkeitsbarrieren erfolgreich völlig vermeidet.
Wenn alle diese Faktoren berücksichtigt werden, kommen wir zu dem Schluß, daß das Ergebnis \enquote{bis zu 1000 mal schneller} im Grunde korrekt ist.
Wir sehen auch, daß der Gebrauch der parallelen Sets, schwierige Abschnitte zuerst üben, kurze Abschnitte üben und schnell auf Geschwindigkeit kommen die Hauptfaktoren sind, die das Lernen beschleunigen.
HS-Üben, Entspannung und frühzeitiges Auswendiglernen sind einige der Wegzeuge, die uns in die Lage versetzen, den Gebrauch dieser beschleunigenden Methoden zu optimieren.



