% File: c1iii4b

\subsubsection{Mit flachen (gestreckten) Fingern spielen}
\label{c1iii4b}

Wir stellten in \hyperref[c1ii2]{\autoref{c1ii2}} fest, dass die anfängliche Fingerform für das Lernen des Klavierspielens die teilweise gebogene Haltung ist.
Viele Lehrer lehren die gebogene Haltung als \enquote{korrekte} Haltung für das Klavierspielen, und dass die flachen Fingerhaltungen irgendwie falsch seien.
V. Horowitz demonstrierte jedoch, dass die flache, oder gestreckte, Haltung der Finger sehr nützlich ist.
\textbf{Wir besprechen hier, warum die flache Haltung der Finger nicht nur nützlich ist, sondern auch ein entscheidender Teil der Technik ist und alle vollendeten Pianisten sie benutzen.}

Wir werden zunächst den Begriff \enquote{\textbf{flache Fingerhaltung}} als die Haltung definieren, bei der die Finger gerade von der Hand weg gestreckt sind, um die Diskussion zu vereinfachen.
Wir werden diese Definition später dahingehend verallgemeinern, dass sie besondere Arten von \enquote{nicht gebogenen} Haltungen bedeutet; diese Haltungen sind wichtig, weil sie ein Teil des Repertoires an Fingerhaltungen sind, das Sie benötigen, um ein richtiger Klavierspieler zu werden.

\textbf{Die wichtigsten Vorteile der flachen Haltung sind, dass sie die Bewegung der Finger vereinfacht und eine völlige \hyperref[c1ii14]{Entspannung} ermöglicht}, das heißt die Zahl der Muskeln, die gebraucht werden, um die Fingerbewegung zu kontrollieren, ist kleiner als bei der gebogenen Haltung, weil alles was man tun muss, das Drehen des gesamten Fingers um den Knöchel ist.
Bei der gebogenen Haltung muss man jeden Finger jedesmal wenn man eine Note anschlägt genau im richtigen Maß strecken, damit man mit dem Finger den korrekten Winkel zur Oberfläche der Taste aufrechterhält.
Bei der Bewegung mit der flachen Fingerhaltung werden nur die Hauptmuskeln benutzt, die nötig sind, um die Tasten herunterzudrücken.
\textbf{Mit flachen Fingern zu üben kann die Technik verbessern, weil man nur die wichtigsten Muskeln und Nerven trainiert.}
Versuchen Sie, um die Komplexität der gebogenen Haltung zu demonstrieren, das folgende Experiment.
Strecken Sie zunächst den Zeigefinger Ihrer rechten Hand gerade aus (flache Fingerhaltung) und wackeln Sie mit ihm schnell auf und ab, wie Sie es beim Klavierspielen tun würden.
Behalten Sie nun diese Auf- und Abwärtsbewegung bei, und krümmen Sie den Finger schrittweise so weit Sie können.
Sie werden feststellen, dass es, wenn Sie den Finger krümmen, schwieriger wird, die Fingerspitze auf und ab zu bewegen, bis es unmöglich wird, wenn der Finger komplett eingerollt ist.
Ich habe dieses Phänomen \textbf{\enquote{Krümmungslähmung}} genannt.
Wenn es Ihnen gelingt, die Fingerspitze zu bewegen, können Sie es, verglichen mit der gestreckten Haltung, nur sehr langsam tun, weil Sie eine ganz neue Muskelgruppe benutzen müssen.
Tatsächlich ist die einfachste Art, Ihre Fingerspitze in der gebogenen Haltung schnell auf und ab zu bewegen, die ganze Hand zu bewegen.

Deshalb brauchen Sie mit der gebogenen Fingerhaltung mehr Geschicklichkeit als mit der flachen Haltung, um mit derselben Geschwindigkeit zu spielen.
\textbf{Entgegen der Meinung vieler Klavierspieler kann man mit der flachen Haltung schneller spielen als mit der gebogenen, weil jegliche Krümmung ein bestimmtes Maß an Krümmungslähmung nach sich zieht.}
Das wird besonders wichtig, wenn die Geschwindigkeit und/oder ein Mangel an Technik während des Übens von etwas schwierigem Stress erzeugen.
Das Maß an Stress ist bei der gebogenen Haltung größer und dieser Unterschied kann ausreichend sein, um eine Geschwindigkeitsbarriere zu erzeugen.

In der Literatur (Jaynes, Kapitel 6) wird behauptet, dass die tiefen Hohlhandmuskeln (die Mm. lumbricales unter und die Mm. interossei zwischen den Mittelhandknochen) für das Klavierspielen wichtig seien.
Es gibt jedoch keine Untersuchungen, die diese Behauptungen stützen, und man weiß nicht, ob diese Muskeln bei der flachen Fingerhaltung eine Rolle spielen.
Im Allgemeinen glaubt man, dass diese Muskeln hauptsächlich dazu benutzt werden, die Krümmung der Finger zu kontrollieren, sodass bei der flachen Fingerhaltung nur die Muskeln in den Armen benutzt werden, um die Finger zu bewegen, und die Mm. lumbricales die Finger nur in Position halten (gebogen oder flach), was die Bewegung vereinfachen und bei der flachen Haltung größere Kontrolle und Geschwindigkeit erlauben soll.
Man ist also heute nicht sicher, ob die Mm. lumbricales eine höhere Geschwindigkeit erlauben oder Krümmungslähmung verursachen.

Obwohl die flache Haltung einfacher ist, \textbf{sollten alle Anfänger zuerst die gebogene Haltung lernen und die flache Haltung nicht lernen, bis sie benötigt wird}.
Wenn Anfänger mit der einfacheren flachen Haltung anfangen, werden sie die gebogene Haltung nie richtig gut lernen.
Anfänger, die versuchen, mit der flachen Haltung schnell zu spielen, werden wahrscheinlich mit \hyperref[c1ii11]{parallelen Sets} mit festen Phasen spielen anstatt mit unabhängigen Fingern.
Das führt zu Kontrollverlust und ungleichmäßigen Geschwindigkeiten.
Haben sich diese schlechten Angewohnheiten erst einmal gebildet, ist es schwierig, die Unabhängigkeit der Finger zu lernen.
Aus diesem Grund verbieten viele Lehrer ihren Schülern, mit flachen Fingern zu spielen, was ein schrecklicher Fehler ist.
\hyperref[Sandor]{Sandor} nennt die flachen Fingerhaltungen \enquote{falsche Haltungen}, aber \hyperref[Fink]{Fink} empfiehlt bestimmte Haltungen, die klar flache Fingerhaltungen sind (wir werden unten verschiedene flache Haltungen behandeln).
\textbf{\hyperref[c1iii3]{Triller} erfordern oft wegen ihrer komplexen Art die gebogene Haltung.}

Die meisten Klavierspieler, die für sich selbst lernen, benutzen meistens flache Fingerhaltungen.
Sehr junge Kinder (unter 4 Jahren) haben üblicherweise Schwierigkeiten, ihre Finger zu krümmen.
Aus diesem Grund benutzen Jazz-Pianisten die flachen Fingerhaltungen öfter als klassische Pianisten (weil viele sich das Klavierspielen zunächst selbst beigebracht haben), und klassische Lehrer weisen zu Recht darauf hin, dass die frühen Jazz-Pianisten eine unterlegene Technik hatten.
Tatsächlich wies der frühe Jazz viel weniger technische Schwierigkeiten als die klassische Musik auf.
Dieser Mangel an Technik resultierte jedoch aus einem Mangel an Unterricht, nicht daraus, dass sie flache Fingerhaltungen benutzten.
Somit sind die flachen Fingerhaltungen nichts Neues, ziemlich intuitiv (nicht alles Intuitive ist schlecht) und eine natürliche Art zu spielen.
Deshalb ist der Weg zu guter Technik eine sorgfältige Balance zwischen dem Üben mit gebogenen Fingern und dem Wissen, wann die flachen Haltungen zu benutzen sind.
\textbf{\textit{Neu in diesem Abschnitt ist das Konzept, dass die gebogene Haltung nicht von Natur aus überlegen ist, und dass die flachen Fingerhaltungen ein notwendiger Teil der fortgeschrittenen Technik sind.}}

Der vierte Finger ist für die meisten Menschen besonders problematisch.
Ein Teil dieser Schwierigkeiten erwächst aus der Tatsache, dass er der Finger ist, der am schwierigsten zu heben ist, was es schwierig macht, schnell zu spielen und zu vermeiden, versehentlich zusätzliche Noten zu treffen.
Diese Probleme sind wegen der Komplexität der Bewegung und der Krümmungslähmung eng mit der gebogenen Haltung verbunden.
In der vereinfachten Anordnung mit flachen Fingern sind diese Schwierigkeiten reduziert, sodass der vierte Finger unabhängiger wird und leichter anzuheben ist.
Wenn Sie Ihre Hand in der gebogenen Haltung auf eine glatte Fläche legen und den vierten Finger heben, wird er eine bestimmte Strecke aufwärts gehen; wenn Sie nun die gleiche Prozedur mit der flachen Fingerhaltung wiederholen, wird die Fingerspitze \textit{zweimal} so weit hochgehen.
Deshalb ist es einfacher, die Finger, und besonders den vierten Finger, in der flachen Haltung anzuheben.
Die Leichtigkeit des Anhebens reduziert den Stress beim schnellen Spielen.
Wenn man versucht, schwierige Passagen mit der gebogenen Haltung schnell zu spielen, werden sich einige Finger (besonders 4 und 5) manchmal zu viel krümmen, was noch mehr Stress erzeugt und die Notwendigkeit, diese Finger \enquote{von sich zu werfen}, um eine Note zu spielen.
Diese Probleme können vermieden werden, indem man die flachen Fingerhaltungen benutzt.

Ein weiterer Vorteil der flachen Fingerhaltung ist, dass sie Ihre Reichweite vergrößert, weil die Finger weiter ausgestreckt sind.
Aus diesem Grund verwenden die meisten Klavierspieler (besonders jene mit kleinen Händen) bereits die flache Haltung für das Spielen \hyperref[c1iii7e]{großer Akkorde} usw., oftmals ohne es zu merken.
Solche Menschen fühlen sich jedoch manchmal wegen des Mangels an Krümmung \enquote{schuldig} und versuchen, soviel Krümmung wie möglich einzubauen, was Stress erzeugt.

Noch ein Vorteil der flachen Fingerhaltung ist, dass die Finger die Tasten mit den Fingerpolstern statt mit den Fingerspitzen herunterdrücken.
Dieser fleischige Teil reagiert empfindlicher auf Druck, und die Fingernägel stören weniger.
Wenn jemand etwas anfasst, um es zu fühlen, benutzt er immer diesen Teil des Fingers, nicht die Fingerspitze.
\textbf{Dieses zusätzliche Polster und die zusätzliche Empfindlichkeit können mehr Gefühl und Kontrolle, sowie mehr Schutz vor Verletzungen bieten.}
Bei der gebogenen Haltung kommen die Finger fast senkrecht auf die Tastenoberfläche herunter, sodass man mit den Fingerspitzen spielt, wo es zwischen dem Knochen und der Tastenoberfläche das geringste Polster gibt.
Wenn man sich die Fingerspitzen durch zu hartes Üben mit der gebogenen Haltung verletzt hat, kann man ihnen eine Pause gönnen, indem man die flache Fingerhaltung benutzt.
Zwei Arten von Verletzungen können an der Fingerspitze auftreten, wenn man die gebogene Haltung benutzt, und beide können vermieden werden, wenn man flache Finger benutzt.
Die erste ist einfach eine Prellung von zu vielem Klopfen.
Die zweite ist das Lösen des Fleisches unter dem Fingernagel, was häufig daraus resultiert, dass man die Fingernägel zu kurz schneidet.
Diese zweite Art der Verletzung ist gefährlich, weil sie zu einer schmerzhaften Infektion führen kann.
Sogar wenn man ziemlich lange Fingernägel hat, kann man immer noch mit der flachen Fingerhaltung spielen.

Noch wichtiger ist, dass man mit flachen Fingern \textbf{die schwarzen Tasten spielen kann, indem man hauptsächlich die großen Bereiche an der Unterseite der Finger benutzt; diese große Fläche kann benutzt werden, um zu vermeiden, dass man die schwarzen Tasten verfehlt}, die man mit der gebogenen Haltung leicht verfehlen kann, weil sie so schmal sind.
\textbf{Spielen Sie bei schnellen Passagen und großen Akkorden die schwarzen Tasten mit flachen Fingern und die weißen Tasten mit gebogenen Fingern}; das kann Ihre Geschwindigkeit und Genauigkeit in hohem Maß steigern.

Wenn die Finger flach ausgestreckt sind, kann man weiter in Richtung der Klappe reichen.
Da man bei dieser Haltung näher an den Waagepunkten (am Waagbalkenstift) ist und somit eine kleinere Hebelwirkung erhält, erfordert es eine etwas größere Kraft, um die Tasten herunterzudrücken.
\textbf{Das resultierende (effektiv) höhere Tastengewicht gestattet es Ihnen, ein leiseres \textit{pp} zu spielen.
Somit führt die Fähigkeit, näher an die Waagepunkte heranzugehen, zur Fähigkeit, das effektive Tastengewicht zu vergrößern.}
Das höhere Tastengewicht gestattet eine größere Kontrolle und ein leiseres Pianissimo.
Obwohl die Veränderung des Tastengewichts gering ist, wird dieser Effekt bei höherer Geschwindigkeit in hohem Maß verstärkt.
\textbf{Andere argumentieren, dass die Enden der Tasten einen größeren Hebel bieten, sodass man eine größere Kontrolle für das \textit{pp} erlangt.}
Probieren Sie deshalb beide Methoden, und schauen Sie, welche für Sie am besten funktioniert. 

Mit flachen Fingern zu spielen gestattet auch ein lauteres Fortissimo, besonders bei den schwarzen Tasten.
Dafür gibt es zwei Gründe.
Erstens ist, wie oben beschrieben, die Fläche des Fingers, die für den Kontakt zur Verfügung steht, größer, und das Polster ist dicker.
Deshalb kann man eine größere Kraft mit einer geringeren Chance der Verletzung oder des Schmerzes übertragen.
Zweitens ist die gesteigerte Genauigkeit, die aus der größeren Kontaktfläche resultiert, beim Erzeugen eines zufriedenstellenden, respekteinflößenden und reproduzierbaren Fortissimos hilfreich.
Bei der gebogenen Haltung ist die Wahrscheinlichkeit, die schwarzen Tasten zu verfehlen oder von ihnen abzurutschen, manchmal für ein volles Fortissimo zu Angst einflößend.
Befürworter der gebogenen Haltung argumentieren, dass diese die einzige ist, die stark genug ist, um das lauteste Fortissimo zu spielen.
Das stimmt nicht; Athleten, die Fingerstände ausführen, tun dies mit flachen Fingerhaltungen, nicht mit den Fingerspitzen.
Tatsächlich erleiden Klavierspieler, die mit der gebogenen Haltung zu viel üben oft Verletzungen der Fingerspitzen.

Die Fähigkeit, leichter fortissimo zu spielen, legt den Schluss nah, dass die flache Fingerhaltung viel entspannter sein kann als die gebogene Haltung.
Das stellt sich als wahr heraus, aber es gibt einen zusätzlichen Mechanismus, der die Entspannung vergrößert.
Mit flachen Fingern kann man sich darauf verlassen, dass die Sehnen an der Unterseite die Finger gerade halten, wenn man auf die Tasten herunterdrückt.
Das heißt, dass man sich, anders als bei der gebogenen Haltung, kaum anstrengen muss, um die Finger gerade zu halten, wenn man die Tasten herunterdrückt, weil, außer wenn man sehr gelenkig ist, die Sehnen an der Unterseite verhindern, dass sich die Finger zurückbiegen.
Lernen Sie deshalb, wenn Sie das Spielen mit flachen Fingern üben, diese Sehnen dazu zu benutzen, Ihnen beim Entspannen zu helfen.
Seien Sie vorsichtig, wenn Sie das erste Mal damit beginnen, flache Finger für das Spielen eines Fortissimos zu benutzen.
Wenn Sie völlig entspannen, können Sie eine Verletzung dieser Sehnen durch Überdehnung riskieren, besonders bei den kleinen Fingern, weil deren Sehnen so klein sind.
Wenn Sie anfangen Schmerzen zu spüren, dann versteifen Sie entweder die Finger während des Anschlags oder hören Sie mit dem Spielen mit flachen Fingern auf und krümmen diesen Finger.
Wenn Sie mit gebogenen Fingern fortissimo spielen, müssen Sie sowohl die Streck- als auch die Beugemuskeln jedes Fingers kontrollieren, um die Finger in der gebogenen Haltung zu halten.
Bei der flachen Haltung können Sie die Streckmuskeln völlig entspannen und nur die Beugemuskeln benutzen, somit den Stress fast völlig eliminieren (der daraus resultiert, dass die beiden Muskelgruppen gegeneinander arbeiten) und den Vorgang für das Niederdrücken der Taste um mehr als 50\% vereinfachen.

Die beste Möglichkeit, mit dem Üben des Spielens mit flachen Fingern anzufangen, ist das Üben der H-Dur-Tonleiter.
Bei dieser Tonleiter spielen alle Finger außer dem Daumen und dem kleinen Finger die schwarzen Tasten.
Da diese beiden Finger im Allgemeinen in Läufen keine schwarzen Tasten spielen, ist das genau das, was Sie üben möchten.
Der Fingersatz für diese Tonleiter ist bei der rechten Hand der \hyperref[table]{Standard-Fingersatz}, aber die linke Hand muss mit dem vierten Finger auf dem H beginnen.
Sie möchten vielleicht zunächst den folgenden Abschnitt (III.5) über das \hyperref[c1iii5]{Spielen schneller Tonleitern} lesen, bevor Sie mit dieser Übung fortfahren, weil Sie wissen müssen, wie man mit Daumenübersatz spielt, wie man die Glissandobewegungen benutzt usw.
Durch das Fühlen der Tasten werden Sie keine einzige Note verfehlen, weil sie im Voraus wissen, wo die Tasten sind.
Wenn eine Hand schwächer als die andere ist, wird dieser Unterschied mit flachen Fingern dramatischer zu Tage treten. Die flache Fingerhaltung offenbart die technischen Fertigkeiten und Mängel deutlicher, weil der Hebel unterschiedlich ist (die Finger sind effektiv länger) und Ihre Finger empfindlicher sind.
Benutzen Sie in diesem Fall \hyperref[c1ii20]{die stärkere Hand, um die schwächere zu unterrichten}, wie man spielt.
Mit flachen Fingern zu üben, mag eine der schnellsten Arten sein, die schwächere Hand zu ermutigen, mit der stärkeren gleichzuziehen, weil man direkt mit den Hauptmuskeln arbeitet, die für die Technik relevant sind.

Wenn Sie beim Spielen mit der flachen Fingerhaltung auf irgendwelche Schwierigkeiten stoßen, versuchen Sie die \hyperref[c1iii7b]{Übungen für parallele Sets} mit den schwarzen Tasten.
Spielen Sie alle fünf schwarzen Tasten mit den fünf Fingern: die zweinotige Gruppe mit Daumen und Zeigefinger und die dreinotige Gruppe mit den verbleibenden drei Fingern.
Anders als bei der H-Dur-Tonleiter wird diese Übung auch den Daumen und den kleinen Finger entwickeln.
Bei dieser Übung (oder bei der H-Dur-Tonleiter) können Sie mit allen Arten von Handpositionen experimentieren.
Anders als bei der gebogenen Haltung \textbf{kann man spielen, während die Handfläche die Oberfläche der weißen Tasten berührt.
Man kann auch das Handgelenk heben, sodass sich die Finger in Wahrheit rückwärts biegen (entgegen der Richtung beim Krümmen), wie bei der \hyperref[c1iii5wagen]{Wagenradbewegung} (III.5e).
Es gibt auch eine Haltung der flachen Finger dazwischen, bei der die Finger gerade bleiben aber im Fingergelenk nach unten gebeugt sind.}
Ich nenne das die \enquote{\textbf{Pyramidenhaltung}}, weil die Hand und die Finger eine Pyramide mit den Knöcheln als Spitze bilden.
Diese Pyramidenhaltung kann für sehr schnelle Passagen sehr wirksam sein, weil sie die Vorteile der gebogenen Haltung und der gestreckten Haltung kombiniert.

Die Nützlichkeit dieser verschiedenen Fingerhaltungen macht es notwendig, dass wir die Definition des \enquote{Spielens mit flachen Fingern} erweitern.
Die gestreckte Haltung ist nur ein Extremfall, und es gibt eine beliebige Zahl von Haltungsvariationen zwischen der völlig flachen Haltung und der gebogenen Haltung.
Zusätzlich zur Pyramidenhaltung können Sie die Finger im ersten Gelenk nach den Knöcheln beugen.
Das nennen wir die \enquote{\textbf{Spinnenhaltung}}.
\textbf{\textit{Der kritische Punkt ist hierbei, dass das letzte Gelenk (vor den Fingernägeln) völlig entspannt sein und man es strecken können muss, wenn man die Taste herunterdrückt.
Deshalb ist die verallgemeinerte Definition der flachen Fingerhaltung, dass das dritte (beim Daumen das zweite) Fingerglied völlig entspannt und gestreckt ist.}}
Die Glieder sind von der Handfläche zur Fingerspitze mit 1-3 durchnummeriert (der Daumen hat nur 1 und 2).\footnote{Der Daumen hat scheinbar ebenfalls drei äußere Glieder; das \enquote{erste} gehört aber anatomisch zur Handfläche. Wenn man es mitzählen würde, dann hätten die anderen Finger analog dazu vier Glieder.}
Wir werden sowohl die Pyramiden- als auch die Spinnenhaltung \enquote{flache Fingerhaltungen} nennen, weil alle drei flachen Haltungen zwei wichtige Eigenschaften gemeinsam haben: Das dritte Glied des Fingers ist niemals gebogen und ist immer entspannt, und man spielt mit der empfindlichen Unterseite der Fingerspitze (Fotos dazu finden Sie bei \hyperref[Prokop]{Prokop} auf den Seiten 13 bis 15).
Ab jetzt benutzen wir diese weiter gefasste Definition der flachen Fingerhaltungen.
Obwohl die Finger bei vielen dieser Haltungen gebeugt sind, werden wir sie flache Haltungen nennen, um sie von der gebogenen Haltung zu unterscheiden.
Der größte Teil der Krümmungslähmung resultiert aus der Beugung des dritten Glieds.
Das kann demonstriert werden, indem man nur das dritte Glied beugt (wenn man es kann) und dann versucht, diesen Finger schnell zu bewegen.
Beachten Sie, dass die völlige Entspannung des dritten Glieds nun Teil der Definition der flachen Fingerhaltungen ist.
Die flache Haltung vereinfacht die Berechnungen im Gehirn, weil man die Beugemuskeln des dritten Glieds fast völlig ignoriert.
Das sind zehn weniger zu kontrollierende Beugemuskeln, und diese sind besonders unhandliche und langsame Muskeln; deshalb kann es die Geschwindigkeit der Finger steigern, wenn man sie ignoriert.
Wir sind bei der Erkenntnis angelangt, dass \textbf{\textit{die gebogene Haltung zum Spielen von fortgeschrittenem Material geradezu falsch ist.
Die verallgemeinerte flache Fingerhaltung ist zum Spielen mit Geschwindigkeiten, die von fortgeschrittenen Spielern gebraucht werden, genau das richtige!}}
Wie weiter unten besprochen, gibt es jedoch bestimmte Situationen, in denen man bestimmte einzelne Finger schnell krümmen muss, um eine weiße Taste zu erreichen und um zu vermeiden, dass man mit den Fingernägeln an die Klappe stößt.
Die Wichtigkeit der verallgemeinerten flachen Fingerhaltung kann nicht überbetont werden, weil sie eines der Schlüsselelemente der Entspannung ist, das oftmals völlig ignoriert wird.

Die flache Fingerhaltung bietet eine viel größere Kontrolle, weil das Polster auf der Unterseite der Fingerspitze der empfindlichste Teil des Fingers ist und das entspannte dritte Glied sich wie ein Stoßdämpfer verhält.
Das gestattet Ihnen; die Tasten zu erfühlen; bei einem Auto machen Stoßdämpfer nicht nur das Fahren bequemer, sondern halten das Rad auch für eine bessere Kontrolle auf der Straße.
Wenn Sie Schwierigkeiten damit haben, in einer Komposition die Farbe herauszubringen, wird es einfacher werden, wenn Sie die flachen Fingerhaltungen benutzen.
\textbf{\textit{In der gebogenen Haltung mit den Fingerspitzen zu spielen, ist so, als ob man ein Auto ohne Stoßdämpfer fahren oder ein Klavier mit abgenutzten Hämmern spielen würde.
Der Ton neigt dann dazu, schriller zu sein; man ist effektiv auf eine Tonfarbe beschränkt.}}
Indem Sie die flachen Fingerhaltungen benutzen, können Sie die Tasten besser fühlen und den Ton und die Farbe leichter kontrollieren.
Da Sie das dritte Fingerglied völlig entspannen und auch einige der Streckmuskeln ignorieren können, sind die flachen Fingerbewegungen einfacher, und Sie können schneller spielen, besonders bei schwierigem Material wie schnellen \hyperref[c1iii3]{Trillern}.
\textbf{\textit{Wir sind deshalb bei einem sehr wichtigen allgemeinen Konzept angekommen: Wir müssen uns selbst von der \enquote{Tyrannei} der einen festen gebogenen Haltung befreien.}}
Wir müssen lernen, alle verfügbaren Fingerhaltungen zu benutzen, weil jede ihre Vorteile hat.

Sie möchten vielleicht die Bank niedriger stellen, um mit dem flachen Teil der Finger spielen zu können.
Wenn die Bank niedriger gestellt wird, wird es üblicherweise notwendig, sie weiter weg vom Klavier zu stellen, damit man zwischen dem Körper und dem Klavier mehr Platz zum Bewegen der Arme und Ellbogen hat.
Mit anderen Worten: Viele Klavierspieler sitzen zu hoch und zu nahe am Klavier, was nicht wahrnehmbar ist, wenn man mit gebogenen Fingern spielt.
Deshalb bieten Ihnen die flachen Fingerhaltungen eine präzisere Möglichkeit, die Höhe und die Position der Sitzbank zu optimieren.
Bei diesen niedrigeren Höhen könnten manchmal die Handgelenke und sogar die Ellbogen beim Spielen unter die Höhe der Tastatur fallen; das ist durchaus zulässig.
Weiter weg vom Klavier zu sitzen, bietet Ihnen auch mehr Platz, um sich zum Fortissimo-Spielen vorzulehnen.

Sämtliche flachen Fingerhaltungen können auf einem Tisch geübt werden.
Legen Sie für die völlig flache Haltung einfach alle Finger und die Handfläche flach auf einen Tisch, und üben Sie, jeden Finger - besonders Finger 4 - unabhängig von den anderen anzuheben.
Üben Sie die Pyramiden- und die Spinnenhaltung, indem Sie nur die fleischigen Polster der Fingerspitzen auf dem Tisch halten und beim Herunterdrücken das dritte Glied völlig entspannen, so dass es sogar nach hinten gebogen wird.
Bei der Pyramidenhaltung wird das zu einer Art Streckübung für alle Beugesehnen, und die beiden letzten Glieder sind entspannt.
Sie werden auch feststellen, dass die flache Fingerhaltung beim Tippen auf einer Schreibmaschinen- oder Computertastatur gut funktioniert.

Der vierte Finger bereitet im Allgemeinen jedem Probleme, und es gibt eine Übung zum Verbessern seiner Unabhängigkeit, die man mit der Spinnenhaltung ausführen kann.
Setzen Sie auf dem Klavier die Finger 3 und 4 auf das C\# und das D\# und die restlichen Finger auf weiße Tasten.
Drücken Sie alle fünf Tasten herunter.
Die erste Übung ist, mit Finger 4 zu spielen und ihn dabei so weit wie möglich anzuheben.
Bei allen Übungen müssen Sie die nicht spielenden Finger unten behalten.
Die zweite Übung ist, die Finger 3 und 4 abwechselnd zu spielen (343434 usw.), wobei Finger 4 wieder so weit wie möglich angehoben wird, aber Finger 3 nur hoch genug, um die Note zu spielen und so, dass der Finger ständig im Kontakt mit der Tastenoberfläche bleibt (ziemlich schwierig, besonders wenn man versucht, es zu beschleunigen).
Die meisten können den vierten Finger in der Spinnenhaltung am höchsten anheben, was zeigt, dass das eventuell im Allgemeinen die beste Haltung zum Spielen ist.
Spielen Sie mit dem Finger 3 lauter als mit dem Finger 4 (Betonung auf die 3).
Wiederholen Sie es mit den Fingern 4 und 5, mit der Betonung auf der 5, und halten Sie die 5 so weit wie möglich auf den Tasten.
Spielen Sie in der letzten Übung parallele Sets - (3,4), (4,3), (5,4) und (4,5) -, wobei alle anderen Finger ihre Tasten vollständig herunterdrücken; wählen Sie beliebige Tasten, die für Sie bequem sind.
Diese Übungen mögen zunächst schwierig erscheinen, Sie werden aber überrascht sein, wie schnell Sie (innerhalb von ein paar Tagen) in der Lage sein werden, sie zu spielen.
Hören Sie aber nicht auf, nur weil Sie diese Übungen ausführen können.
Üben Sie weiter, bis Sie sie mit völliger Kontrolle und Entspannung sehr schnell ausführen können, weil Sie sonst keinen Nutzen davon haben.
Diese Übungen simulieren die schwierige Situation, in der Sie mit den Fingern 3 und 5 spielen, während Sie den Finger 4 über den Tasten halten.

Die zusätzliche Reichweite, die große Kontaktfläche und das zusätzliche Polster unter den Fingern machen das Legato-Spielen mit flachen Fingern einfacher und anders als das Legato mit der gebogenen Haltung.
Die flache Fingerhaltung vereinfacht es auch, zwei Noten mit einem Finger zu spielen, besonders weil man spielen kann, ohne dass die Finger parallel zu den Tasten sind und man eine sehr große Fläche unter den Fingern benutzen kann, um mehr als eine Taste unten zu halten.
Da Chopin für sein Legato bekannt war, gut mehrere Noten mit einem Finger spielen konnte und empfahl, die H-Dur-Tonleiter zu üben, benutzte er wahrscheinlich das Spielen mit flachen Fingern.
Yvonne Combe, die den ersten Anstoß zu diesem Buch gab, lehrte die flachen Fingerhaltungen und wies darauf hin, dass sie besonders nützlich sind, um Chopin zu spielen.
Ein Legato-Trick, den sie lehrte, war, mit der flachen Haltung anzufangen und dann den Finger zur gebogenen Haltung einzurollen, sodass man die Hand bewegen kann, ohne die Finger von den Tasten zu heben.
Man kann es auch umgekehrt machen, wenn man von den schwarzen Tasten zu den weißen heruntergeht\footnote{das heißt man lässt die Finger zunächst auf den schwarzen Tasten und streckt die Finger während man die Hand zu den weißen Tasten nach unten bewegt}.

Sie können die Nützlichkeit der flachen Fingerhaltung demonstrieren, indem Sie sie auf irgend etwas anwenden, das Ihnen Schwierigkeiten bereitet.
So hatte ich zum Beispiel ein paar Stress-Probleme beim Beschleunigen von \hyperref[c1iii20]{Bachs Inventionen}, weil sie die Unabhängigkeit der Finger erfordern, besonders der Finger 3, 4 und 5.
Während ich nur mit der gebogenen Haltung übte, fühlte ich, dass ich begann, bei ein paar Passagen, für die ich nicht genug Fingerunabhängigkeit hatte, eine Geschwindigkeitsbarriere aufzubauen.
Als ich die flache Fingerhaltung anwandte, wurde es viel leichter, sie zu spielen.
Das erlaubte mir schließlich, mit höheren Geschwindigkeiten und mit größerer Kontrolle zu spielen.
Die Bach-Inventionen sind gute Stücke zum Üben der flachen Fingerhaltungen, was nahelegt, dass Bach eventuell an die flachen Fingerhaltungen dachte, als er sie komponierte.

Eine Diskussion des Spielens mit flachen Fingern wäre ohne eine Diskussion darüber, warum man die gebogene Haltung benötigt, sowie einiger ihrer Nachteile unvollständig.
Diese Haltung ist nicht wirklich eine mit Absicht gebogene Haltung, sondern eine entspannte Haltung, bei der es bei den meisten Menschen eine natürliche Krümmung gibt.
Diejenigen, deren entspannte Haltung zu gestreckt ist, müssen eventuell eine leichte Krümmung hinzufügen, um die ideale gebogene Haltung zu erreichen.
Bei dieser Haltung berühren alle Finger die Tasten in einem Winkel zwischen 45 und 90 Grad (der Daumen mag einen etwas kleineren Winkel bilden).
Es gibt bestimmte für das Klavierspielen absolut notwendige Bewegungen, die die gebogene Haltung erfordern.
Einige davon sind: bestimmte weiße Tasten spielen (wenn die anderen Finger schwarze Tasten spielen), zwischen den schwarzen Tasten spielen und vermeiden, mit den Fingernägeln an die Klappe zu stoßen.
Besonders für Klavierspieler mit großen Händen ist es notwendig, die Finger 2, 3 und 4 zu krümmen, wenn die Finger 1 und 5 die schwarzen Tasten spielen, um zu verhindern, dass man mit den Fingern 2, 3 und 4 die Klappe trifft.
\textbf{\textit{Deshalb ist die Freiheit, mit einem willkürlichen Maß an Krümmung zu spielen, eine notwendige Freiheit.
Einer der größten Nachteile der gebogenen Haltung ist, dass die Streckmuskeln nicht genügend trainiert werden.
Das führt dazu, dass die Beugemuskeln ihnen kraftmäßig überlegen sind und somit Probleme bei der Kontrolle entstehen.
Bei den flachen Fingerhaltungen sind die ungenutzten Beugemuskeln entspannt; die zugehörigen Sehnen werden gestreckt, was die Finger flexibler macht.
Es gibt zahlreiche Berichte über die außerordentliche Flexibilität der Finger von Franz Liszt.}}

Die falsche Vorstellung, dass die flache Haltung schlecht für die Technik sei, kommt von der Tatsache, dass sie zu \hyperref[c1ii22]{schlechten Angewohnheiten} führen kann, die mit dem falschen Gebrauch der \hyperref[c1ii11]{parallelen Sets} zusammenhängen.
Das geschieht, weil es mit flachen Fingern eine einfache Sache ist, die Finger zu positionieren und sie alle als paralleles Set, das sich als schneller Lauf maskiert, auf das Klavier herunter prasseln zu lassen.
Das kann zu ungleichmäßigem Spiel führen, und Anfänger könnten es als eine Möglichkeit benutzen, schnell zu spielen, ohne Technik zu entwickeln.
Indem man zuerst die gebogene Haltung lernt, und lernt, wie man die parallelen Sets richtig benutzt, kann man dieses Problem vermeiden.
Bei meinen zahlreichen Kontakten mit Lehrern habe ich festgestellt, dass die besten Lehrer mit der Nützlichkeit der flachen Fingerhaltungen vertraut sind.
Das gilt besonders für die Gruppe der Lehrer, deren Unterrichtsmethode auf Liszt zurückgeht, weil Liszt diese Haltung benutzte.
Liszt war Czernys Schüler, folgte aber nicht immer Czernys Lehren und benutzte die flachen Haltungen, um den Klang zu verbessern (Boissier, Fay, Bertrand).
Es ist tatsächlich schwer vorstellbar, dass es fortgeschrittene Klavierspieler gibt, die nicht wissen, wie man die flachen Fingerhaltungen benutzt.
Wenn Sie das nächste Mal ein Konzert besuchen oder ein Video ansehen, schauen Sie einmal zum Beweis nach, ob Sie diese flachen Haltungen erkennen können - Sie werden sehen, dass jeder vollkommene Klavierspieler sie benutzt.
Da traditionell hauptsächlich die gebogene Haltung gelehrt wurde, werden Sie jedoch vielleicht feststellen, dass einige Klavierspieler die gebogene Haltung zu oft benutzen.
Es ist erfreulich, dass der berühmteste Pianist sich\footnote{in dieser Hinsicht} oft dazu entschloss, seinen eigenen Lehrer, Czerny, zu ignorieren.

Wenn man Ihnen die ganze Zeit nur die gebogene Haltung beigebracht hat, mag es zunächst merkwürdig erscheinen, die flachen Fingerhaltungen zu lernen, weil einige wichtige Sehnen sich verkürzt haben.
Einige Lehrer sehen die flachen Fingerhaltungen als eine Art Betrug und als Hinweis auf mangelnde Fertigkeiten mit gebogenen Fingern an, aber das stimmt nicht; die flachen Haltungen sind eine notwendige Fertigkeit.
Beginnen Sie das Üben der flachen Haltungen mit Vorsicht, weil manche Sehnen der Finger vielleicht erst gedehnt werden müssen.
Alle Sehnen müssen von Zeit zu Zeit gedehnt werden, aber die gebogene Haltung gestattet das nicht.

Was ist die Reihenfolge der Wichtigkeit all dieser Haltungen - was ist die flache \enquote{Standardhaltung}, die wir am meisten benutzen sollten?
Die Spinnenhaltung ist die wichtigste.
Das Insektenreich hat diese Haltung nicht ohne guten Grund übernommen; sie haben in hunderten Millionen von Jahren herausgefunden, dass sie am besten funktioniert.
Beachten Sie, dass die Unterscheidung zwischen der Spinnenhaltung und der gebogenen Haltung subtil sein kann, und viele Klavierspieler, die glauben, sie würden die gebogene Haltung benutzen, verwenden in Wahrheit etwas, das den flachen Fingerhaltungen näher kommt.
Die zweitwichtigste Haltung ist die flach ausgestreckte Haltung, weil sie zum Spielen \hyperref[c1iii7e]{weiter Akkorde} und Arpeggios gebraucht wird.
Die dritte Haltung ist die gebogene Haltung, die zum Spielen der weißen Tasten benötigt wird, und die Pyramidenhaltung kommt an vierter Stelle.
Bei der Pyramidenhaltung wird während des Anschlags nur ein Beugemuskel je Finger benutzt, bei der Spinnenhaltung zwei und bei der gebogenen Haltung alle drei sowie die Streckmuskeln.
Die endgütige Wahl der Fingerhaltung ist aber eine persönliche Angelegenheit und muss dem Klavierspieler überlassen bleiben.

Im Allgemeinen kann man folgende Regel anwenden, um zu entscheiden, welche Fingerhaltung man benutzt:
Spielen Sie die schwarzen Tasten mit der völlig flachen Haltung, und benutzen Sie die gebogene und die Pyramidenhaltung für die weißen Tasten.
Die Spinnenhaltung ist vielseitig, wenn Sie sie sich in jungen Jahren angeeignet haben, und man kann mit ihr sowohl schwarze als auch weiße Tasten spielen.
Beachten Sie, dass es im Allgemeinen vorteilhaft ist, zwei Arten von Fingerhaltungen zu benutzen, wenn man innerhalb einer Gruppe von Noten sowohl schwarze als auch weiße Tasten spielen muss.
Das mag zunächst als eine zusätzliche Schwierigkeit erscheinen, aber bei hohen Geschwindigkeiten könnte das die einzige Möglichkeit sein.
Es gibt natürlich zahlreiche Ausnahmen: So benötigen Sie zum Beispiel in Passagen mit dem vierten Finger eventuell mehr flache als gebogene Haltungen, auch wenn die meisten oder alle Tasten weiß sind, um das Heben des vierten Fingers zu vereinfachen.

Obwohl diese Ausführungen über das Spielen mit flachen Fingern umfangreich sind, so sind sie keineswegs vollständig.
In einer detaillierteren Abhandlung müssen wir besprechen, wie man das Spielen mit flachen Fingern auf spezielle Fertigkeiten anwendet, wie zum Beispiel Legato oder das Spielen von zwei Noten mit einem Finger, wobei man jede Note einzeln kontrolliert.
Chopins Legato ist als etwas ganz besonderes dokumentiert, genauso wie sein Staccato.
Ist sein Staccato mit der flachen Fingerhaltung verbunden?
Beachten Sie, dass wir in allen flachen Haltungen einen Vorteil aus dem \enquote{Federeffekt} des entspannten dritten Glieds ziehen können, was beim Staccato-Spielen nützlich sein kann.
Klar müssen wir mehr Nachforschungen anstellen, damit wir lernen, wie man die flachen Fingerhaltungen benutzt.
Es gibt insbesondere eine Kontroverse darüber, ob man hauptsächlich mit der gebogenen Haltung spielen und die flache Fingerhaltung hinzufügen sollte, wann immer es notwendig ist, so wie von den meisten Lehrern gelehrt wurde, oder anders herum, wie Horowitz es getan hat und es hier empfohlen wird.
Die flachen Fingerhaltungen sind auch mit der \hyperref[c1ii3]{Höhe der Sitzbank} verbunden.
Es ist leichter, mit flachen Fingern zu spielen, wenn die Bank niedriger ist.
Es gibt zahlreiche Berichte von Pianisten, die entdeckt haben, dass sie bei einer niedrigeren Bankposition viel besser spielen können (Horowitz und Glenn Gould sind Beispiele).
Sie behaupten, dass sie eine bessere Kontrolle erhalten, insbesondere für das Pianissimo und die Geschwindigkeit, aber niemand hat bisher eine Erklärung dafür gegeben, warum das so ist.
Meine Erklärung ist, dass die niedrigere Höhe der Bank es ihnen erlaubte, mehr flache Fingerhaltungen zu benutzen.
Es scheint aber keinen guten Grund dafür zu geben, übermäßig tief zu sitzen, wie Glenn Gould es getan hat, weil man immer das Handgelenk senken kann, um denselben Effekt zu erzielen.

Zusammengefasst hatte Horowitz gute Gründe, mit flachen Fingern zu spielen, und die obige Diskussion legt nahe, dass ein Teil seiner hohen technischen Stufe daraus resultierte, dass er die flachen Fingerhaltungen mehr als die anderen benutzte.
\textbf{Die wichtigste Botschaft dieses Abschnitts ist, dass wir lernen müssen, das dritte Glied des Fingers zu entspannen und mit dem berührungsempfindlichen Teil der Fingerspitze spielen sowie die Beweglichkeit der Finger trainieren müssen.}
Die Aversion gegen oder sogar das Verbot des Spielens mit flachen Fingern einiger Lehrer stellt sich als Fehler heraus; in Wahrheit wird jedes Einkrümmen der Finger zu einem gewissen Maß an Krümmungslähmung führen.
Anfänger müssen jedoch die gebogene Haltung als erste lernen, weil sie oft benötigt wird und schwieriger als die flachen Haltungen ist.
Wenn Schüler die leichtere Methode mit flachen Fingern als erstes lernen, werden sie die gebogene Haltung später eventuell niemals ausreichend gut lernen.
Das Spielen mit flachen Fingern ist für die Geschwindigkeit nützlich, das Vergrößern der Reichweite, Spielen mehrerer Noten mit einem Finger, Vermeiden von Verletzungen, \enquote{Fühlen der Tasten}, Legato, Entspannung, pianissimo oder fortissimo spielen und das Hinzufügen von Farbe.
Obwohl die gebogene Haltung notwendig ist, ist die Behauptung \enquote{Man braucht die gebogene Haltung, um technisch schwieriges Material zu spielen.} irreführend - Sie benötigen dafür bewegliche Finger.
Mit flachen Fingern zu spielen, gibt uns die Freiheit, viele notwendige und vielseitige Haltungen zu benutzen.
Wir wissen nun, wie man all diese schwarzen Tasten spielen kann und keine einzige Note verfehlt.
Ich danke Ihnen, Johann Sebastian Bach, Frederic Chopin, Vladimir Horowitz und Yvonne Combe.



