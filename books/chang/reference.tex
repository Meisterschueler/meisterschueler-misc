% File: reference 

\label{reference}

<h2><br>\underline{Quellenverzeichnis}</h2>

<h3><br>\underline{Buchbesprechungen}</h3>

\paragraph{Allgemeine Schlußfolgerungen aus den besprochenen Büchern}
\label{allgemein}

\begin{enumerate}[label={\arabic*.}] 
\item 
Die Klavierliteratur hat sich in den letzten 100 Jahren von der Aufmerksamkeit auf die Finger und Fingerübungen zum Gebrauch des ganzen \hyperref[c1iii4c]{Körpers}\index{Körpers}, zur \hyperref[c1ii14]{Entspannung}\index{Entspannung} und der \hyperref[c1iii14d]{musikalischen Aufführung}\index{musikalischen Aufführung} hin entwickelt.
Deshalb enthalten die älteren Veröffentlichungen oftmals Konzepte, die nun angezweifelt werden.
Das bedeutet nicht, daß Mozart, Beethoven, Chopin und Liszt nicht die richtige Technik hatten, sondern daß in der Literatur hauptsächlich die großen Auftritte aufgezeichnet wurden aber nicht das, was man tun mußte, um so gut zu werden.
Kurz gesagt: Die Klavierliteratur war bis in die heutige Zeit in beklagenswerter Weise unzulänglich.

\item Ein Konzept, das sich nicht geändert hat, ist, daß musikalische Gesichtspunkte, wie \hyperref[c1iii1b]{Rhythmus}\index{Rhythmus}, \hyperref[c1iii1]{Klang}\index{Klang}, Phrasierung usw., nicht von der Technik getrennt werden können.

\item Fast jedes Buch beschäftigt sich mit einer Auswahl derselben Themen; die Hauptunterschiede liegen im Ansatz und im Detailreichtum.
Fast alle behandeln nur einzelne Aspekte und sind unvollständig.
Sie behandeln zunächst den menschlichen Geist und die menschliche Anatomie sowie ihr Verhältnis zum Klavier: geistige Haltung und Vorbereitung, \hyperref[c1ii3]{Sitzposition, Bankhöhe}\index{Sitzposition, Bankhöhe}, Rolle der \hyperref[c1ii10]{Arme}\index{Arme}, \hyperref[c1iii4]{Hände}\index{Hände} und \hyperref[c1iii4b]{Finger}\index{Finger} -- oft mit entsprechenden Übungen und einer Besprechung von \hyperref[c1iii10hand]{Verletzungen}\index{Verletzungen}.
Sodann Konzepte der Technik und Musikalität: Anschlag, \hyperref[c1iii1]{Klang}\index{Klang}, \hyperref[c1iii5a]{Daumen}\index{Daumen}, Legato, \hyperref[c1iii1c]{Staccato}\index{Staccato}, \hyperref[c1ii18]{Fingersätze}\index{Fingersätze}, \hyperref[c1iii5a]{Tonleitern}\index{Tonleitern}, \hyperref[Arpeggios]{Arpeggios}\index{Arpeggios}, Oktaven, \hyperref[c1iii7e]{Akkorde}\index{Akkorde}, wiederholte Noten, \hyperref[c1ii13]{Geschwindigkeit}\index{Geschwindigkeit}, Glissando, \hyperref[c1ii23]{Pedal}\index{Pedal}, Übungszeit, \hyperref[c1iii6]{Auswendiglernen}\index{Auswendiglernen} usw.
Es gibt erstaunlich wenig Literatur über das \hyperref[c1iii11]{Blattspiel}\index{Blattspiel}.
\label{c030530}\footnote{Eine sehr aufschlußreiche und verständliche Arbeit zum Vom-Blatt-Spielen mit einer eigenen Konzeption ist die Dissertation \enquote{Zur Methodik des elementaren Prima-Vista-Spiels} (2001/2002, 448 S.) von Dr. Bernd Sommer.
Eine Zusammenfassung der Arbeit und die Möglichkeit zum Herunterladen der Dissertation finden Sie \hyperref[http://www.dissertation.de/buch.php3?buch=1405]{hier} (extern) (mit freundlicher Genehmigung von Dr. Sommer).}

\item Von ein paar älteren Ausnahmen abgesehen, raten die meisten vom Gebrauch des \hyperref[c1iii5a]{Daumenuntersatzes}\index{Daumenuntersatzes} zum Spielen von Tonleitern ab; der Daumenuntersatz ist jedoch für einige bestimmte Anwendungen eine wertvolle Bewegung.
Chopin bevorzugte das Untersetzen des Daumens für sein Legato, lehrte aber das \hyperref[c1iii5a2]{Übersetzen}\index{Übersetzen} wo es technisch vorteilhaft war.

\item Der Mangel an Quellenangaben in vielen Büchern spiegelt die Tatsache wieder, daß die Lehrmethoden für das Klavierspielen niemals ausreichend oder richtig dokumentiert wurden.
Jeder Autor mußte im Prinzip jedesmal das Rad neu erfinden.
Das zeigt sich auch in den aktuellen Lehrmethoden.
Die Lehrmethoden für das Klavier wurden im Grunde durch die Worte aus dem Mund des Lehrers an den Schüler weitergegeben, was an die Art erinnert, in der prähistorische Menschen ihre Überlieferungen und medizinische Praktiken über Generationen hinweg weitergaben.
Dieser grundlegende Makel brachte die Entwicklung der Lehrmethoden fast zum Stillstand, und sie haben sich im Grunde über Jahrhunderte hinweg nicht verändert.

\hyperref[Whiteside]{Whitesides Buch} wurde deshalb weithin anerkannt, weil es der erste wirkliche Versuch war, mit einer wissenschaftlichen Vorgehensweise die besten Übungsmethoden zu entdecken.
Gemäß den Überlieferungen wurden jedoch die meisten ihrer \enquote{Entdeckungen} von Chopin gelehrt; offenbar stand diese Information Whiteside nicht zur Verfügung.
Es mag jedoch mehr als bloßer Zufall sein, daß sie Chopins Musik ausgiebig in Ihren Lehren benutzte.
Whitesides Buch versagte kläglich, denn obwohl sie Experimente durchführte und ihre Ergebnisse dokumentierte, benutzte sie keine klare Sprache, ordnete ihre Ergebnisse nicht, führte keine Analyse von Ursache und Wirkung durch usw., was für ein gutes wissenschaftliches Projekt notwendig ist.
Trotzdem war ihr Buch zur Zeit seiner Veröffentlichung -- wegen der minderen Qualität aller anderen -- eines der besten erhältlichen Bücher.

Eine ungeheure Anzahl Lehrer behauptet, die Liszt-Methode zu lehren, aber es gibt nur eine fragmentarische und herzlich wenig Dokumentation darüber, was diese Methode ist.
Es gibt reichlich Literatur darüber, wohin Liszt reiste, wen er traf und unterrichtete, was er spielte und welche wunderbaren Meisterleistungen des Klavierspiels er vollführte, aber es gibt praktisch keine Aufzeichnungen davon, was ein Schüler tun muß, damit er in der Lage ist so zu spielen.

\item \hyperref[Chang]{Changs Buch}\index{Changs Buch} (besonders diese zweite Ausgabe) ist das einzige, das die Übungsmethoden dafür zur Verfügung stellt, bestimmte anfängliche technische Probleme (Überwindung von \hyperref[c1iv2b]{Geschwindigkeitsbarrieren}\index{Geschwindigkeitsbarrieren}, \hyperref[c1ii14]{Entspannung}\index{Entspannung}, \hyperref[c1ii21]{Ausdauer}\index{Ausdauer}, \hyperref[c1iii6]{Auswendiglernen}\index{Auswendiglernen}, \hyperref[c1iii6h]{langsames}\index{langsames} gegenüber \hyperref[c1ii13]{schnellem}\index{schnellem} Üben usw.) zu lösen, die im Anfängerstadium gelernt werden sollten aber nicht immer gelehrt werden.
Die anderen Bücher behandeln hauptsächlich die \enquote{höheren} Stufen des Klavierspielens und nehmen an, daß der Schüler die grundlegenden Techniken auf irgendeine magische Weise erworben hat.
Offensichtlich ist es wichtig, diese Fertigkeiten der \enquote{höheren Stufen} von Anfang an zu lernen, so daß Changs Buch eine große Lücke in der Literatur über das Erwerben von Technik füllt.

 \end{enumerate}
<br>\textbf{Format der Darstellung:} Autor, Titel, Erscheinungsjahr, Anzahl der Seiten und ob Quellen im Buch angegeben werden.<br>
Die Quellen sind ein Indiz dafür, wie wissenschaftlich das Buch ist.
Nach diesem Kriterium ist Changs erste Ausgabe nicht wissenschaftlich; dieser Mangel wurde in der zweiten Ausgabe beseitigt.
Die Besprechungen erheben nicht den Anspruch der Objektivität und sind nicht umfassend; sie befassen sich hauptsächlich damit, wie relevant diese Bücher für den Klavierschüler sind, der sich für Klaviertechnik interessiert.
Das meiste \enquote{irrelevante} Material wurde ignoriert.


\label{Bree}

<br>\textbf{Bree, Malwine}, \enquote{The Leschetizky Method}. 1997 (1913), 92 S., keine Quellenangaben.<br>
Obwohl dieses Buch 1997 erschienen ist, ist es eine Wiederveröffentlichung des Materials von 1913.<br>
Abstammungslinie der Unterrichtsmethode: Beethoven-Czerny-Leschetizky-Bree.<br>  Buch mit Übungen für das Entwickeln der Technik, Fotos der Fingerpositionen.
Befürwortet den Daumenuntersatz.
Handposition, Übungen für die Unabhängigkeit der Finger, Tonleitern, Akkorde, Anschlag, Glissando, Pedal, Auftritte usw., eine ziemlich vollständige Abhandlung.
Lesen Sie das, um etwas über die älteren \enquote{etablierten} Methoden herauszufinden.


\label{Bruser}

<br>\textbf{Bruser, Madeline}, \enquote{The Art of Practicing}. 1997, 272 S., Quellenangaben (artofpracticing.com).<br>
Basiert darauf, zunächst den Geist (Meditation) und Körper (Dehnungsübungen) vorzubereiten, und geht dann zu einigen nützlichen Einzelheiten der Fertigkeiten für das Klavierspielen über.
Die Menge der Anweisungen für das Klavierspielen wird leider dadurch reduziert, daß ebenfalls Anweisungen für andere Instrumente (hauptsächliche Saiten- und Blasinstrumente) gegeben werden.
Obwohl körperliches Training (leichte Gymnastik) gut ist, sind Übungen wie Tonleitern nicht hilfreich.
Enthält wenige nützliche Informationen.


\label{Chang}

<br>\textbf{Chang, Chuan C.}, \enquote{\hyperref[http://www.pianopractice.org]{Fundamentals of Piano Practice}} (extern), erste Ausgabe. 1994, 130 S., keine Quellenangaben (2. Ausgabe mit Quellenangaben).<br> Abstammungslinie der Unterrichtsmethode: Long-Combe<br>
Lehrt die grundlegendsten Methoden für das schnelle Erwerben der Technik (Üben mit getrennten Händen, Akkord-Anschlag [parallele Sets], schwierige Passagen kürzen, Auswendiglernen, Entspannung, Geschwindigkeitsbarrieren eliminieren usw.).
Kein anderes Buch bespricht alle dieser wesentlichen Elemente, die für einen schnellen Fortschritt und eine korrekte Technik notwendig sind.
Behandelt auch das Spielen vom Blatt, Vorbereitung auf Konzerte, Kontrolle der Nervosität, Anschlag mit freiem Fall, welche Übungen gut sind und welche nutzlos oder schädlich, Lernen des absoluten Gehörs, Konturieren usw.
Enthält ein Kapitel über das Klavierstimmen für den Amateur, erklärt die chromatische Tonleiter und das Temperieren.
Die auf dieser Website \hyperref[Inhalt]{zum freien Download vorliegende zweite Ausgabe}\index{zum freien Download vorliegende zweite Ausgabe} ist eine aktualisierte und erweiterte Fassung der ersten Ausgabe.
\textbf{Muß man gelesen haben.}


\label{Eigeldinger}

<br>\textbf{Eigeldinger, Jean-Jacques,}\enquote{Chopin, pianist and teacher as seen by his pupils}. 1986, 324 S., Quellenangaben.<br>
Die wissenschaftlichste und vollständigste Zusammenfassung relevanten Materials über Chopin in bezug auf Unterricht, Technik, Interpretation und Geschichte.
Wegen des Mangels an direkter Dokumentation zu Chopins Zeit ist praktisch das ganze Material anekdotenhaft.
Trotzdem scheint die Genauigkeit wegen der umfassenden Dokumentation, dem Fehlen jeglicher erkennbarer systematischer Fehler und der offensichtlichen Tatsache, daß solch ein tiefes Verständnis nur von Chopin selbst gekommen sein kann, unzweifelhaft zu sein -- die Ergebnisse sind in verblüffender Übereinstimmung mit dem besten heute verfügbaren Material.
Eigeldinger hat die Themen in hilfreichen Gruppen angeordnet (Technik, Interpretation, Zitate, kommentierte Notenblätter und Fingersätze, Chopins Stil).
Ich würde mir wünschen, daß es mehr Übungsmethoden enthielte, aber wir müssen alle zur Kenntnis nehmen, daß der Mangel an Dokumentation zu Chopins Zeit zum Verlust eines großen Teils dessen was er gelehrt hat führte.
Im Fall von Franz Liszt ist die Situation weitaus schlechter.

Die technischen Lehren werden kurz und prägnant auf den Seiten 23-64 präsentiert.
Diese Lehren sind fast in völliger Übereinstimmung mit denen der besten Quellen, von \hyperref[Walker]{Liszt}\index{Liszt} und \hyperref[Whiteside]{Whiteside} bis zu \hyperref[Fink]{Fink}, \hyperref[Sandor]{Sandor}, \hyperref[Suzuki]{Suzuki}\index{Suzuki} und \hyperref[Chang]{diesem Buch (Chang)}\index{diesem Buch (Chang)}.
Die Präsentation steht in starkem Kontrast zu Whiteside; hier ist sie autoritativ (Whiteside nimmt manchmal ihre eigenen Ergebnisse zurück), kurz (nur 41 Seiten verglichen mit 350 Seiten bei Whiteside!), organisiert und klar, wobei ein ähnlicher Themenbereich abgedeckt wird.
Der zweite Teil, Seiten 65-89, behandelt die Interpretation und enthält deshalb viel weniger Informationen über Technik, ist jedoch genauso informativ wie der erste Abschnitt.
Er befaßt sich (sehr!) kurz damit, wie man jede einzelne von Chopins Hauptkompositionen interpretiert.
Die verbleibenden 200 Seiten widmen sich der Dokumentation, Illustrationen, Chopins Anmerkungen zu seinen eigenen Kompositionen und Fingersätzen und einem 10 Seiten umfassenden \enquote{Entwurf} von grundlegendem Material für den Anfängerunterricht.

Anmerkungen zur Technik: Chopin war Autodidakt; es ist wenig darüber bekannt, wie er lernte als er jung war, außer, daß er von seiner Mutter unterrichtet wurde, die eine vollendete Pianisten war.
Chopin glaubte nicht an Drill und Übungen (er empfahl nicht mehr als 3 Stunden Übung pro Tag).
Chopins Methoden stehen nicht in dem Maß, wie es zunächst erscheinen mag, im Gegensatz zu Liszts, obwohl Liszt häufig mehr als 10 Stunden täglich übte und Übungen \enquote{bis zur Erschöpfung} empfahl.
Chopin schrieb wie Liszt Etüden, und Liszts \enquote{Übungen} waren keine stupiden Wiederholungen, sondern zum Erwerb der Technik bestimmte Methoden.

Man soll lernen, Musik zu machen, \textit{bevor} man Technik erlernt.
Der ganze Körper muß einbezogen werden, und der Gebrauch des Armgewichts (Freier Fall) ist ein Schlüsselelement der Technik.
Er lehrte sowohl den Daumenübersatz (besonders wenn die passierte Note schwarz ist!) als auch den Daumenuntersatz und gestattete es sogar, jeden Finger über jeden anderen rollen zu lassen, wann immer es vorteilhaft war -- der Daumen war nicht einzigartig und mußte \enquote{frei} sein.
Jeder Finger war jedoch unterschiedlich.
Das Übersetzen des Daumens (genau wie das anderer Finger) war besonders nützlich bei beidhändigen chromatischen Tonleitern (Terzen usw.).
Bei Chopin mußte das Klavier sprechen und singen; für Liszt war es ein Orchester.
Da die C-Dur-Tonleiter schwieriger ist, benutzte er die H-Dur-Tonleiter, um Entspannung und Legato zu lehren; ironischerweise ist es besser, die Tonleiter zunächst staccato zu lernen, um die schwierigen Probleme mit dem Legato zu eliminieren, obwohl er am Ende immer zu seiner Spezialität zurückkommt -- dem Legato.
Große Arpeggios erfordern eher eine geschmeidige Hand als eine große Reichweite.
Beim Rubato wird der Rhythmus streng eingehalten, während die Zeit im Verlauf der Melodie ausgeliehen und wieder zurückgegeben wird.
[Meiner Meinung nach wird diese Definition oft falsch zitiert und falsch verstanden; nur weil er das ein paarmal gesagt hat, bedeutet es nicht, daß er es auf alles angewandt hat.
Diese Definition des Rubato gilt besonders für die Situation, in der die RH rubato spielt, während die LH im strikten Zeitmaß bleibt.
Chopin hat sicherlich auch erlaubt, daß Rubato eine Freiheit vom strikten Tempo zugunsten des Ausdrucks war.]
Chopin bevorzugte den Pleyel, ein Klavier mit sehr leichter Mechanik.
Seine Musik ist auf modernen Instrumenten eindeutig schwieriger zu spielen, besonders das Pianissimo und Legato.
\textbf{Muß man gelesen haben}.


\label{Fink}

<br>\textbf{Fink, Seymour}, \enquote{Mastering Piano Technique}, 1992, 187 S., ausgezeichnetes Quellenverzeichnis; Video ebenfalls erhältlich.<br>
Das wissenschaftlichste der hier aufgeführten Bücher, wie es sich für einen Universitätsprofessor gebührt.
Wissenschaftliche Abhandlung, die die korrekte Terminologie benutzt (im Gegensatz zu \hyperref[Whiteside]{Whiteside}, die häufig nichts von der Standardterminologie wußte), leicht verständlich, beginnt mit der menschlichen Anatomie und ihrer Beziehung zum Klavier, gefolgt von einer Auflistung der Bewegungen, die ins Spielen einbezogen sind, einschließlich des Pedals.
Tonleitern dürfen nicht mit Daumenuntersatz gespielt werden, aber der Daumenuntersatz ist eine wichtige Bewegung (S. 115).
Veranschaulicht jede Bewegung und die zugehörigen Klavierübungen.
Gute Beschreibung des Freien Falls.
Strikter mechanischer Ansatz, aber das Buch betont das Erzeugen eines volleren Tons und das emotionale Spielen.
Die Bewegungen sind aus den Diagrammen schwer zu entnehmen und machen den Kauf des Videos wünschenswert.
Sie müssen entweder Fink oder \hyperref[Sandor]{Sandor} lesen; vorzugsweise beide, da sie ähnliche Themen von verschiedenen Standpunkten aus angehen.
Einige Leser werden den einen mögen und den anderen ablehnen.
Fink basiert auf Übungen, Sandor mehr auf Beispielen von klassischen Kompositionen.

Die erste Hälfte ist eine Abhandlung aller grundlegenden Bewegungen und von Übungen für diese Bewegungen.
Diese schließen ein: Pronation, Supination, Wegführen, Heranziehen, Handpositionen (ausgestreckt, Handfläche, krallen), Fingerschläge, Bewegungen des Unterarms, Oberarms, der Schulter (Schub, Zug, Zirkulieren) usw.
Der zweite Teil wendet diese Bewegungen auf Beispiele aus berühmten Klassikern von Ravel, Debussy und Rachmaninoff bis Chopin, Beethoven, Mozart und vielen anderen an. \textbf{Man muß entweder dieses Buch oder Sandor gelesen haben}.


\label{Gieseking}

<br>\textbf{Gieseking, Walter und Leimer, Karl}, \enquote{Modernes Klavierspiel}, 2 Bücher in einem, 1972, keine Quellenangaben.<br>
Abstammungslinie der Unterrichtsmethode: Leimer-Gieseking.
\textbf{Erstes Buch}: Gieseking, \enquote{Modernes Klavierspiel}, 77 S.<br>
Wichtigkeit des Zuhörens, \enquote{Ganzkörper}-Methode (wie bei der Armgewichtsschule), Konzentration, präzises Üben, Aufmerksamkeit auf die Details.
Hervorragende Behandlung, wie man eine Komposition für das Üben und Auswendiglernen analysiert.
Dieses Buch ist ein Vertreter der meisten Bücher, die von jenen großen Künstlern geschrieben wurden.
Ein typischer Rat in bezug auf die Technik ist \enquote{Konzentration, präzises Üben und Aufmerksamkeit auf die Details führt automatisch zur Technik.} oder \enquote{Benutzen Sie Ihre Ohren.} oder \enquote{Alle Noten eines Akkords müssen zusammen klingen.} ohne jeden Hinweis darüber, wie man jede einzelne Fertigkeit tatsächlich erlangt.

Führt Ihnen vor, wie man Bachs zweistimmige C-Dur-Invention (\#1) übt, die dreistimmige C-Dur-Invention (\#1) und Beethovens Sonate \#1, aber mehr von der Analyse und Interpretation her als vom Standpunkt der technischen Fertigkeiten.
Er führt Sie durch die ersten 3 Sätze von Beethovens Sonate, entläßt aber den technisch anspruchsvollsten 4. Satz mit \enquote{bringt keine weiteren neuen Probleme}!
Beachten Sie, daß dieser letzte Satz einen starken, schwierigen und sehr schnellen 5,2,4-Fingersatz, gefolgt von einem absteigenden Arpeggio mit Daumenübersatz in der LH sowie schnelle und akkurate weite Akkordsprünge in der RH erfordert.
Bei diesen hätten wir uns ein wenig Rat von Gieseking gewünscht.
Changs Buch schließt diese Lücke, indem es die Anleitung in \hyperref[c1iii8]{Kapitel 1, Abschnitt III.8} liefert.
Lesenswert, sogar wenn es nur wegen der speziellen Führung durch die oben angegebenen Stücke ist.

\textbf{Zweites Buch}: Leimer, \enquote{Rhythmik, Dynamik, Pedal und andere Probleme des Klavierspiels}, 56 S.<br>
Wichtigkeit von Rhythmus, Zählen, akkuratem Timing, Phrasieren.
Ausgezeichneter Abschnitt über den Gebrauch des Pedals.
Enthält einige spezielle Informationen, die woanders schwer zu finden sind.


\label{Green}

<br>\textbf{Green, Barry, und Gallwey, Timothy}, \enquote{The Inner Game of Music}, 1986, 225 S., keine Quellenangaben.<br>
 Mentales Herangehen an Musik; Entspannung, Bewußtsein, Vertrauen.
Fast keine technischen Anweisungen zum Klavierspielen.
Nur für diejenigen, die glauben, daß geistige Haltung der Schlüssel zum Klavierspielen ist.
Wer an bestimmten Rezepten für das Üben interessiert ist, wird wenige nützliche Informationen finden.


\label{Hofman}

<br>\textbf{Hofman, Josef}, \enquote{Piano Playing, With Piano Questions Answered}, 1909, 183 S., keine Quellenangaben.<br>
Abstammungslinie der Unterrichtsmethode: Moszkowki, Rubinstein.<br>
Die erste Hälfte behandelt sehr nützliche allgemeine Regeln, und die zweite Hälfte ist in Frage-und-Antwort-Form.
Der größte Teil des Buchs bespricht generelle Konzepte; nicht viele detaillierte technische Anweisungen.
Kein unentbehrliches Buch für Technik, ist aber gut nebenbei zu lesen.


\label{Lhevine}

<br>\textbf{Lhevine, Josef}, \enquote{Basic Principles in Piano Playing}, 1972, 48 S., keine Quellenangaben.<br> Ausgezeichnete Behandlung wie man einen guten Klang erzeugt.
Kurze Besprechung von: Grundwissen der Tonarten, Tonleitern usw., Rhythmus, Gehörtraining, leise und laut, Genauigkeit, Staccato, Legato, Auswendiglernen, Übungszeit, Geschwindigkeit, Pedal.
Meistens oberflächlich -- das Buch ist zu kurz.
Gute allgemeine Zusammenfassung, es fehlen aber nähere Einzelheiten und es enthält kein Material, das man nicht auch woanders findet.


\label{Prokop}

<br>\textbf{Prokop, Richard}, \enquote{Piano Power, a Breakthrough Approach to Improving your Technique}, 1999, 108 S., sehr wenige Quellenangaben.<br>
Die Einführung liest sich, als ob das das Buch wäre, auf das jeder gewartet hat.
Je mehr man jedoch liest, desto desillusionierter wird man.
Der Autor -- Pianist, Klavierlehrer und Komponist -- begann das Klavierspielen mit der \enquote{intuitiven Methode} (s. \hyperref[c1ii1]{\autoref{c1ii1}} von Chang) zu lernen, und seine Lehren bestehen immer noch zu 50\% daraus.
Er kennt z.B. nicht den Daumenübersatz und trifft deshalb auf viele \enquote{Probleme}.
Die Lehren bestehen aus \enquote{Theoremen}, die er \enquote{beweist}.
Wenn man nur ein paar solcher Theoreme liest, zeigt sich, daß man bei der Klaviertechnik Theoreme nicht wie in der Mathematik beweisen kann, was im Grunde das ganze Buch widerlegt.
Er bringt ein paar nützliche Ideen zur Sprache:

\begin{enumerate} 
 \item Wichtigkeit der Streckmuskeln (Anheben der Finger); akkurates Anheben der Finger (und Pedale) ist genauso wichtig wie ein akkurater Anschlag.
Er stellt Übungen für das Anheben jedes Fingers zur Verfügung und gibt die beste Beschreibung der Knochen, Sehnen und Muskeln der Finger, Hände und Arme und wie bzw. welche Bewegungen durch sie kontrolliert werden.
 \item Detaillierte Analyse der Vor- und Nachteile von kleinen, mittelgroßen und großen Händen.
\end{enumerate}
Da gute Ideen mit den falschen vermischt sind, kann dieses Buch den weniger informierten Schüler in die Irre führen oder verwirren.
Es gibt keine \enquote{Durchbrüche} (s. Titel); empfehlenswert nur für diejenigen, die nützliche Ideen von den falschen unterscheiden können.


\label{Richman}

<br>\textbf{Richman, Howard,} \enquote{Super Sight-Reading Secrets}, 1986, 48 S., keine Quellenangaben.<br>
Dieses Buch ist das beste Buch über das Notenlesen.
Es enthält alle Grundlagen; sie werden in allen Einzelheiten beschrieben, und wir lernen die ganze korrekte Terminologie und Methodik.
Es beginnt damit, wie man Noten liest (für den Anfänger), und geht auf logische Weise vorwärts bis zu fortgeschrittenen Stufen des Blattspiels; es ist für den Anfänger besonders hilfreich.
Es ist auch kurz und prägnant; Sie sollten deshalb das ganze Buch einmal durchlesen, bevor Sie mit dem eigentlichen Üben anfangen.
Beginnt damit, wie man psychologisch an das Notenlesen herangeht.
Grundlegende Komponenten des Notenlesens sind Tonhöhe, Rhythmus und Fingersatz.
Nach einer hervorragenden Einführung in die Notation werden geeignete Übungen gezeigt.
Anschließend wird der Vorgang des Spielens vom Blatt in die einzelnen Schritte der visuellen, neuralen, muskulären und auralen\footnote{also Auge, Gehirn/Nerven, Muskeln, Ohren} Prozesse zerlegt, die mit dem Notenblatt beginnen und als Musik enden.
Übungen für das Lernen der \enquote{Orientierung auf der Tastatur} (Noten finden, ohne auf die Tastatur zu sehen) und \enquote{visuelle Wahrnehmung} (sofort erkennen, was zu spielen ist) schließen sich an.
In Abhängigkeit von der Person kann es von 3 Monaten bis zu 4 Jahren dauern, es zu lernen; Sie sollten es täglich üben.
Schließlich ungefähr eine Seite Gedanken über fortgeschrittenes Blattspiel.
\textbf{Muß man gelesen haben}.


\label{Sandor}

<br>\textbf{Sandor, Gyorgy}, \enquote{On Piano Playing}, 1995, 240 S., keine Quellenangaben.<br>  Abstammungslinie der Unterrichtsmethode: Bartok-Kodaly-Sandor.<br>  Das vollständigste, wissenschaftlichste und teuerste Buch.
Enthält den größten Teil des Materials aus \hyperref[Fink]{Fink}, betont die \enquote{Armgewichts}-Methoden.
Bespricht: Freier Fall, Tonleitern (Daumenübersatz; hat die detaillierteste Beschreibung von Tonleiter- und Arpeggiospiel, S. 52-78), Drehung, Staccato, Schub, Pedale, Klang, Üben, Auswendiglernen, Auftritte.
Begleitet Sie durch das Lernen der gesamten Waldstein-Sonate (Beethoven).

Zahlreiche Beispiele, wie man die Grundsätze des Buchs auf Kompositionen von Chopin, Bach, Liszt, Beethoven, Haydn, Brahms, Schumann und vielen anderen anwendet.
Dieses Buch ist ziemlich vollständig; es behandelt Themen von der Auswirkung der Musik auf Emotionen bis zu Besprechungen des Klaviers, der menschlichen Anatomie und grundlegenden Spielbewegungen, sowie Auftritte und Aufnahmen; viele Themen werden jedoch nicht hinreichend ausführlich behandelt.
\textbf{Muß man gelesen haben}, aber \hyperref[Fink]{Fink} wird Ihnen ähnliche Informationen zu geringeren Kosten bieten.


\label{Sherman}

<br>\textbf{Sherman, Russell}, \enquote{Piano Pieces}, 1997, keine Quellenangaben.<br>  Besteht aus 5 Abschnitten, die das Spielen, Unterrichten, kulturelle Gesichtspunkte, Notenblätter und \enquote{alles andere} behandeln.
Die Inhalte sind in keiner besonderen Reihenfolge angeordnet, mit keinen wirklichen Lösungen oder Schlußfolgerungen.
Bespricht die Politik der Künste (Musik), Meinungen, Urteile und Beobachtungen, zu denen Pianisten einen Bezug haben; ob Nicht-Pianisten diese Gedankengänge verstehen können, ist fraglich, sie gewähren aber Einblicke.
Sitzposition, Daumen dient als Impulsausgleich.
Finger = Truppen; Körpermitte = Versorgungslinie, Unterstützung, Transportschiff und Herstellung.
Finger gegen Körper = Vertrieb gegen CEO; deshalb resultiert die Kontrolle der Finger nicht in Musik.
Leichte Stücke sind wertvoll zum Lernen, wie man Musik macht.
Was hat man davon, Klavierspielen zu lernen?
Es ist, finanziell gesehen, nicht einmal eine gute Karriere.
Sollte man den Finger gleiten lassen?
Was spielt eine Rolle für die Schönheit oder den Charakter des Klavierklangs?
Wie wichtig sind Qualitätsklaviere und gute Stimmer?
Pro und Kontra von Wettbewerben (hauptsächlich Kontras): Vorbereitung auf Wettbewerbe ist kein Musikmachen und wird oftmals zu einem athletischen Wettbewerb; ist es den Streß und den Aufwand wert?; Urteile sind niemals perfekt.

Handelt von Themen, denen sich Pianisten, Lehrer und Eltern gegenübersehen; beschreibt viele der Hauptprobleme, präsentiert aber wenige Lösungen.
Dieses Buch berührt viele Themen, ist aber so ziellos wie der Titel.
Lesen Sie es nur, wenn Sie Zeit totzuschlagen haben.


\label{Suzuki}

<br>\textbf{Suzuki, Shinichi (et al)}, zwei Bücher (es gibt mehr): \enquote{The Suzuki Concept: An Introduction to a Successful Method for Early Music Education}, 1973, 216 S., keine Quellenangaben, hat eine große, ausgezeichnete Bibliographie.<br>
Hauptsächlich für einen Geigenunterricht, der in jungen Jahren beginnt.
Nur ein kurzes Kapitel (7 Seiten) über Methoden des Klavierunterrichts.

\enquote{HOW TO TEACH SUZUKI PIANO}, 1993, 21 S., keine Quellenangaben.<br>  Eine kurze allgemeine Skizzierung der Suzuki-Klavier-Methoden.
Die von Chang beschriebenen Methoden sind in genereller Übereinstimmung mit den Suzuki-Methoden.
Lassen Sie ein Baby zuhören; kein Beyer, Czerny, Hanon oder Etüden (sogar Chopin!); Auftritte sind ein Muß; Lehrer müssen einheitliche Lehrmethoden haben und offene Diskussionen führen (Forschungsgruppen); Auswendiglernen und Blattspiel müssen ausgewogen sein, aber Auswendiglernen ist wichtiger.
Lehrern wird eine kleine Anzahl abgestufter Stücke angegeben, auf denen Ihr Unterricht basieren soll.
Suzuki ist eine zentral kontrollierte Schule; als solche hat sie viele der Vorteile von den Fakultäten der etablierten Musikhochschulen und Colleges, aber die akademische Stufe ist im allgemeinen niedriger.
Suzuki-Lehrer sind mindestens eine Klasse besser als der durchschnittliche Privatlehrer, weil sie bestimmten Mindestanforderungen genügen müssen.
Beschreibt viele allgemeine Vorgehensweisen beim Unterrichten aber wenig Einzelheiten darüber, wie man Klavier für die Technik übt.
Klassisches Beispiel dafür, wie ein autoritäres System schlechte Lehrer durch das Festlegen von Mindestanforderungen ausschließen kann.


\label{Walker}

<br>\textbf{Walker, Alan}, \enquote{Franz Liszt, The Virtuoso Years, 1811-1847}, 1983, 481 S., Quellenangaben.<br>
Das ist das erste von drei Büchern; es behandelt die Periode von Liszts Geburt bis zu der Zeit, als er sich mit 36 Jahren entschied, nicht mehr aufzutreten.
Das zweite Buch behandelt die Jahre 1848-1861, in denen er sich hauptsächlich dem Komponieren widmete.
Das dritte Buch behandelt die Jahre 1861-1886, seine letzten Jahre.
Ich bespreche hier nur das erste Buch.

Liszt ist als der größte Pianist aller Zeiten bekannt.
Deshalb würden wir erwarten, von ihm am meisten darüber zu lernen, wie man Technik erwerben kann.
Unglücklicherweise ist jedes Buch über Liszt in dieser Hinsicht eine absolute Enttäuschung.
Meine Vermutung ist, daß Technik zu Liszts Zeit so etwas wie ein \enquote{Berufsgeheimnis} war und seine Lektionen nie dokumentiert wurden.
Paganini übte völlig im Geheimen und stimmte sogar seine Geige verdeckt anders, um Resultate zu erzielen, die kein anderer erzielen konnte.
Chopin hingegen war ein Komponist und professioneller Lehrer -- das waren die Quellen seiner Einkünfte, und es gibt zahlreiche Berichte über seinen Unterricht.
Liszts Grundlage seines Ruhms waren seine Auftritte.
Sein Erfolg in dieser Hinsicht spiegelt sich in der Tatsache wieder, daß praktisch jedes Buch über Liszt eine endlose und wiederholte Chronik seiner unglaublichen Auftritte ist.
Meine Vermutung über diese Heimlichtuerei würde erklären, warum so viele der Pianisten dieser Zeit behaupten, Schüler von Liszt gewesen zu sein, obwohl sie Liszts Lehrmethoden selten in nützlicher Weise detailliert beschreiben.
Wenn man jedoch diesen Details unter den heutigen Lehrern der \enquote{Liszt-Schule} nachspürt, findet man heraus, daß sie ähnliche Methoden benutzen (getrennte Hände, schwierige Passagen kürzen, Akkord-Anschlag usw.).
Was immer die wahren Gründe sind, Liszts Lehrmethoden wurden nie ausreichend dokumentiert.
Ein Vermächtnis, daß Liszt uns hinterlassen hat, ist die gut dokumentierte Tatsache, daß die Art Meisterleistungen, die er vorführte, menschlich möglich sind.
Das ist wichtig, weil es bedeutet, daß wir alle ähnliche Dinge tun können, wenn wir wieder entdecken, wie er es gemacht hat.
Vielen Pianisten ist das gelungen, und ich hoffe, daß mein Buch ein Schritt in die richtige Richtung dafür ist, die besten bekannten Übungsmethoden zu dokumentieren.

Walkers Buch ist typisch für die anderen Bücher über Liszt, die ich gelesen habe, und ist im Grunde eine Chronik von Liszts Leben und kein Lehrbuch darüber, wie man Klavier lernt.
Als solches ist es eine der besten Biographien und enthält zahlreiche Besprechungen einzelner Kompositionen mit besonderen pianistischen Ansprüchen und Schwierigkeiten.
Leider lehrt uns die Beschreibung einer unmöglichen Passage als \enquote{die mit größter Leichtigkeit ausgeführt wurde} nicht, wie man es machen muß.
Dieses Fehlen der technischen Lehrinformation ist, angesichts der Tatsache, daß die Zahl der bibliographische Berichte über Liszt 10.000 bei weitem übersteigt, überraschend!
Tatsächlich muß jede technische Information, die wir aus diesem Buch entnehmen können, mit Hilfe unseres eigenen Klavierwissens aus dem Inhalt abgeleitet werden (s. u. das Beispiel über die Entspannung).
Der Abschnitt mit der Überschrift \enquote{Liszt und die Tastatur} (S. 285-318) enthält ein paar Fingerzeige darauf, wie man spielt.
Wie in allen drei Büchern, wird Liszt als ein Halbgott verehrt, der nicht fehlgehen kann, ja sogar mit Superhänden ausgestattet ist, die für das Klavier irgendwie ideal gestaltet sind -- er konnte eine Dezime leicht erreichen.
Diese Voreingenommenheit vermindert die Glaubwürdigkeit, und die unablässigen, wiederholten Berichte der übermenschlichen Darbietungen erzeugen eine Langeweile, die von der enormen Menge an aufschlußreichen und faszinierenden historischen Details in diesen Büchern ablenkt.

Vom Standpunkt der Klaviertechnik ist vielleicht der interessanteste Punkt, daß Liszt von früher Jugend an ein dünner, kränklicher Mann war.
Tatsächlich wurde er im Alter von drei Jahren nach einer Krankheit als so gut wie tot aufgegeben, und man hatte sogar schon einen Sarg bestellt.
Er fing mit dem Klavierspielen erst mit sechs Jahren an und hatte nicht einmal ein eigenes Klavier zum Üben, bis er sieben war, weil seine Familie so arm war.
Er wurde von seinem Vater unterrichtet, einem talentierten Musiker und passablen Pianisten und wurde von Geburt an mit Musik getränkt.
Czerny war sein erster \enquote{richtiger} Lehrer, im Alter von 11, und Czerny behauptet, er habe Franz dessen ganzen grundlegenden Fertigkeiten beigebracht.
Er gibt jedoch zu, daß Franz bereits ein offensichtliches Wunderkind war, als sie einander vorgestellt wurden -- was verdächtig widersprüchlich scheint.
Franz rebellierte sogar gegen Czernys Drill, machte aber von Übungen für seine technische Entwicklung ausgiebigen Gebrauch.
Was er übte waren die Grundlagen: Läufe, Sprünge, wiederholte Noten.
Meine Interpretation ist, daß dieses keine stupiden Wiederholungen für den Muskelaufbau waren, sondern Übungen mit bestimmten Zielen hinsichtlich der Fertigkeiten, und wenn die Ziele erreicht waren, wandte er sich neuen zu.

Aber wie führt ein schwächlicher Mensch \enquote{unmögliche} Übungen bis zur Erschöpfung aus?
Indem er entspannt!
Liszt könnte, aus der Notwendigkeit heraus, der größte Experte der Welt für Entspannung gewesen sein.
Was die Entspannung angeht, mag es kein Zufall sein, daß Paganini ebenfalls ein kränklicher Mann war.
Zu der Zeit als er berühmt wurde, in seinen Dreißigern, hatte er Syphilis, und seine Gesundheit verschlechterte sich aufgrund seiner Leidenschaft für das Glücksspiel und der Ansteckung mit Tuberkulose.
Und doch waren diese beiden Männer mit schlechter Gesundheit die beiden größten Meister ihrer Instrumente.
Die Tatsache, daß beide körperlich schwach waren, zeigt, daß die Energie für übernatürliche Darbietungen nicht von einer athletischen Muskelkraft kommt, sondern eher von einer völligen Beherrschung der Entspannung.
Chopin war ebenfalls eher schwächlich und steckte sich mit Tuberkulose an.
Eine traurige historische Notiz sind -- zusätzlich zu Paganinis schlechter Gesundheit und den Auswirkungen der primitiven chirurgischen Versuche dieser Zeit -- die Umstände seines schrecklichen Todes, da es eine Verzögerung bei seinem Begräbnis gab und man ihn in einem betonierten Wasserbehälter verwesen ließ.

Ein weiterer bemerkenswerter Lehrer von Liszt war Salieri, der ihn Komposition und Theorie lehrte.
Damals war Salieri über 70 Jahre alt und hatte jahrelang unter dem Verdacht gelitten, er hätte Mozart aus Eifersucht vergiftet.
Liszt verbesserte sich im Alter von 19 Jahren immer noch.
Seinen Meisterleistungen wird die Popularisierung des Klaviers zugeschrieben.
Ihm wird die Erfindung des Klavierkonzerts zugeschrieben (indem er es aus dem Salon in die Konzerthalle brachte).
Eines seiner Mittel war sowohl der Gebrauch mehrerer Klaviere als auch der Auftritt mehrerer Pianisten.
Er spielte sogar Konzerte mit mehreren Klavieren mit Chopin und anderen Koryphäen seiner Zeit.
Dies gipfelte in phantastischen Kompositionen mit bis zu 6 Klavieren, die als \enquote{Konzert mit 60 Fingern} beworben wurden.
Innerhalb von 10 Wochen spielte er 21 Konzerte und 80 Werke, 50 aus dem Gedächtnis.
Daß er sein Publikum so begeistern konnte, war das überraschendere, weil angemessene Klaviere (Steinway, Bechstein) bis in die 1860er Jahre nicht verfügbar waren -- fast 20 Jahre nachdem er aufhörte, Konzerte zu geben.

Ich habe dieses Buch mit der Absicht gelesen, Information darüber herauszuziehen, wie man Klavier übt.
 Wie Sie sehen, können wir heute vom größten Pianisten aller Zeiten fast nichts darüber erfahren,
 wie man Klavier übt, obwohl seine Lebensgeschichte ein faszinierender Lesestoff ist.


\label{Werner}

<br>\textbf{Werner, Kenney}, \enquote{Effortless Mastery}, 191 S., plus Meditations-CD, ein paar Quellenangaben und viele Vorschläge für Material zum Anhören.<br>
Mentaler bzw. spiritueller Ansatz zum Musizieren; fast keine Beschreibungen der Mechanik des Spielens, Schwerpunkt ist die Meditation.
Das Buch ist wie eine Autobiographie, und die Lektionen werden so gelehrt, wie er sie während seines Lebens gelernt hat.
In derselben Kategorie wie \hyperref[Green]{Green und Gallwey}\index{Green und Gallwey} aber ein anderer Ansatz.


\label{Whiteside}

<br>\textbf{Whiteside, Abby}, \enquote{On Piano Playing}, 2 Bücher in einem, 1997, keine Quellenangaben.<br>
Das ist eine Neuauflage von \enquote{Indispensables of Piano Playing} (1955), und \enquote{Mastering Chopin Etudes and Other Essays} (1969).<br>
Abstammungslinie der Unterrichtsmethode: Ganz-Whiteside.

\textbf{Erstes Buch}: \enquote{Indispensables of Piano Playing}, 155 S.<br>
Benutzt kein Standard-Englisch, verworrene Logik, biblische Phrasen, unnötig langatmig.
Die Inhalte sind ausgezeichnet aber der fürchterliche Schreibstil macht das Lernen unproduktiv.
Viele der Ideen, die sie beschreibt, tauchen in anderen Büchern auf, aber es kann sein, daß sie die meisten hervorgebracht (oder wiederentdeckt) hat.
Obwohl ich Schwierigkeiten hatte, das Buch zu lesen, haben andere behauptet, daß es leichter zu verstehen sei, wenn man es schnell lesen kann.
Das kommt zum Teil daher, daß sie sich ständig wiederholt und oft einen Absatz oder sogar eine Seite braucht, um etwas zu beschreiben, das man in einem Satz sagen kann.

Fast das ganze Buch ist so (S. 54): \enquote{Frage: Kann Gewicht -- ein regloser Druck -- dabei helfen, Gewandtheit zu erlangen?
Antwort: Es ist genau der reglose Druck des Gewichts, der nicht für die Geschwindigkeit benutzt werden kann.
Worte sind beim Unterrichten wichtig.
Worte der Bewegung sind notwendig, um die Koordination für die Geschwindigkeit zu bewirken.
Gewicht bewirkt nicht die Muskelaktivität, die das Gewicht des Arms bewegt.
Es bewirkt einen reglosen Druck.}
Ich habe diesen Abschnitt nicht deshalb gewählt, weil er besonders verworren wäre -- ich habe ihn zufällig ausgewählt, indem ich das Buch mit geschlossenen Augen geöffnet habe.

Inhalt: Man muß ihren Methoden religiös folgen; warum Rhythmus wichtig ist; es gibt unendlich viele Möglichkeiten für die Kombination des Körpers, der Arme, Hände und Finger, von denen uns die meisten nicht bewußt sind; Tonleitern mit Daumenuntersatz werden geschmäht; Funktionen jedes Teils der Anatomie für das Klavierspielen (horizontale, Einwärts-, Auswärts-, vertikale Bewegungen); Besprechung des Erzeugens von Emotionen, Auswendiglernen, Pedale, Phrasieren, Triller, Tonleitern, Oktaven, Lehrmethoden.
Stellt die Wichtigkeit des Rhythmus für die Musik heraus und wie man diesen durch Konturieren erreicht (S. 141).
Czerny und Hanon sind nutzlos oder schlimmer.

Das folgende ist ihr Angriff auf das Passieren mit Daumenuntersatz für das Spielen von Tonleitern (in verständlicherer Sprache), herausgezogen aus über zwei Seiten; die ( ) sind meine Klarstellungen:

\enquote{\textit{Passieren.} Hier sehen wir uns beim traditionellen Unterricht hinsichtlich der exakten Bewegungen, die mit den Fingern und den Daumen stattfinden sollten, mit einem Wust von Streß konfrontiert ...
Wenn ich diese Konzepte einfach hinwegblasen könnte, würde ich nicht zögern es zu tun.
Für so fehlerhaft und übel halte ich sie.
Sie können einen Pianisten buchstäblich verkrüppeln ...
Wenn es (perfekte Tonleitern spielen) hoffnungslos unmöglich erscheint und Sie keinen Schimmer einer Idee haben, wie es vollbracht werden kann, dann versuchen Sie es mit einer Koordination, die eine Tonleiter tatsächlich zu einer unmöglichen Meisterleistung macht.
Es bedeutet, daß der Daumen unter die Handfläche klappt und man nach der Position sucht; und die Finger versuchen, über den Daumen zu reichen und suchen nach der Verbindung der Tasten für das Legato.
Es ist egal, ob der Künstler, der bereits die  raschen und schönen Tonleitern und Arpeggios erreicht hat, Ihnen erzählt, daß er nur das (Daumenuntersatz) macht -- es ist nicht wahr.
Ich will nicht unterstellen, daß er lügt, sondern daß er die Koordination, die er gelehrt bekam als die Gelegenheit, die es unangemessen machte, auftrat, erfolgreich verdrängt hat.
Sie (die Daumenuntersatz-Spieler) müssen wieder körperlich zu einem neuen Koordinationsmuster ausgebildet werden; und diese erneute Ausbildung kann für sie eine Periode des schrecklichen Elends bedeuten ...
Die Bewegung (für das Passieren mit Daumenübersatz) kann mittels des Schultergelenks in jede Richtung ausgeführt werden.
Der Oberarm kann sich so bewegen, daß das Ellbogenende des Humerus\footnote{d.h. des Oberarmknochens} einen Kreisabschnitt beschreiben kann und zwar sowohl auf- und abwärts als auch ein- und auswärts, vor und zurück oder um sich selbst herum ... (usw., eine ganze Seite mit Anweisungen dieser Art, wie man mit Daumenübersatz spielt) ...
Mit Kontrolle von der Mitte funktioniert die ganze Koordination so, daß es einfacher wird, einen Finger in dem Moment zur Verfügung zu haben, in dem er benötigt wird ...
Der beste Beweis für diese Aussage ist eine schöne Tonleiter oder Arpeggio, die mit völliger Nichtbeachtung jeglichen konventionellen Fingersatzes gespielt werden.
Das geschieht oftmals bei einem begabten, nicht ausgebildeten Klavierspieler ...
Beim Passieren (Daumenübersatz) agiert der Oberarm als Drehpunkt für all die}anderen Techniken\enquote{und bezieht den Unterarm und die Hand mit ein; Beugung und Streckung am Ellbogen, Drehbewegung und seitliche Bewegung beim Handgelenk und zu guter Letzt seitliche Bewegungen der Finger und Daumen ...
Durch die Drehbewegung und abwechselnde Bewegung wird das Passieren so leicht gemacht, wie es aussieht, wenn der Experte es macht.}

\textbf{Zweites Buch}: \enquote{Mastering the Chopin Etudes and Other Essays}, 206 S.<br>
Zusammenfassung von bearbeiten Manuskripten Whitesides; viel lesbarer, weil sie von ihren Schülern bearbeitet wurden, und enthält die meisten der Ideen des ersten Buchs, basierend auf dem Spielen der Chopin-Etüden, welche sowohl aufgrund ihres unerreichten musikalischen Gehalts als auch wegen ihrer technischen Herausforderung ausgewählt wurden.
Das ist wie ein Katechismus zur obigen Bibel; es mag eine gute Idee sein, dieses Buch zu lesen, bevor man das erste o.a. Buch liest.
Beschreibt das Konturieren einigermaßen detailliert: S. 54-61 grundlegende Beschreibung und S. 191-193 grundlegende Definition, mit mehr Beispielen auf den S. 105-107 und S. 193-196.
Obwohl das Konturieren dazu benutzt werden kann, technische Schwierigkeiten zu überwinden, ist es wertvoller dafür, das musikalische Konzept der Komposition kennenzulernen oder es spielen zu lernen.

Diese beiden Bücher \textit{sind} eine Diamantenmine an praktischen Ideen; aber wie bei einer Diamantenmine muß man tief schürfen, und man weiß nie, wo sie verborgen sind.
Die Verwendung der Chopin-Etüden stellt sich hier nicht als zufällige Wahl heraus; die meisten von Whitesides Grundsätzen wurden bereits von Chopin gelehrt (s. \hyperref[Eigeldinger]{Eigeldinger}); Eigeldingers Buch wurde jedoch lange Zeit nach Whitesides Buch geschrieben, und ihr waren wahrscheinlich viele von Chopins Methoden nicht bekannt.

Es gibt keinen Mittelweg -- Sie werden Whitesides Buch entweder für die Fundgrube an Information lieben oder es hassen, weil es unlesbar, eintönig und ungeordnet ist.
 

\label{American}

<br>\textbf{Scientific American, Jan. 1979, S. 118-127, \textit{The Coupled Motions of Piano Strings}, von G. Weinreich}

Das ist ein guter Artikel über die Bewegungen von Klaviersaiten, wenn man die absoluten Grundlagen lernen muß.
Der Artikel ist jedoch nicht gut geschrieben, und die Experimente sind nicht sorgfältig ausgeführt; aber wir sollten die begrenzten Mittel berücksichtigen, die der Autor wahrscheinlich zur Verfügung hatte.
Noch weiter gehende Untersuchungen wurden sicherlich lange vor 1979 von Klavierherstellern und Akustikwissenschaftlern durchgeführt.
Ich werde im folgenden einige der Mängel, die ich in diesem Artikel gefunden habe, in der Hoffnung besprechen, daß die Kenntnis dieser Mängel den Leser in die Lage versetzt, hilfreichere Informationen aus dieser Publikation zu entnehmen und zu vermeiden, in die Irre geführt zu werden.

Es gibt keinerlei Information über die Frequenzen der Noten, die untersucht wurden.
Da das Verhalten von Klaviersaiten so frequenzabhängig ist, ist diese fehlende Information von entscheidender Bedeutung.
Behalten Sie dies im Gedächtnis, wenn Sie den Artikel lesen, da viele der Ergebnisse ohne die Kenntnis der Frequenzen, bei denen die Experimente ausgeführt wurden, schwer zu interpretieren und deshalb von fragwürdigem Wert sind.

Das mittlere Diagramm in der unteren Reihe der Abbildungen auf S. 121 (es gibt im ganzen Artikel keine Abbildungsnummern!) wird nicht ausreichend erklärt.
Der Artikel schlägt später vor, daß die vertikalen \hyperref[moden]{Moden}\index{Moden} den Anschlagsklang erzeugen.
Die Abbildung könnte deshalb so interpretiert werden, daß sie das Ausschwingen einer einzelnen Saite zeigt.
Ich kenne keine Note auf einem Flügel, bei der die Ausschwingzeit einer einzelnen Saite weniger als 5 Sekunden beträgt, wie es durch die Abbildung suggeriert wird.
Die linke Abbildung der oberen Reihe der Diagramme für eine einzelne Saite zeigt, in Übereinstimmung mit meinen flüchtigen Messungen bei einem Flügel, ein Ausschwingen von mehr als 15 Sekunden.
Somit scheinen sich die beiden Diagramme für einzelne Saiten zu widersprechen.
Für das obere Diagramm wurde der Schalldruck gemessen, für das untere hingegen die Verschiebung der Saite, so daß sie nicht richtig miteinander vergleichbar sind, aber man hätte es gerne gesehen, wenn der Autor zumindest diese offensichtliche Diskrepanz ein wenig erklärt hätte.
Ich habe den Verdacht, daß für die beiden Diagramme Saiten mit sehr verschiedenen Frequenzen benutzt wurden.

In bezug auf diese Diagramme gibt es den Satz \enquote{Ich benutzte eine empfindliche elektronische Sonde, um die vertikalen und horizontalen Bewegungen einer einzelnen Saite getrennt zu messen.} ohne weitere Informationen.
Nun wäre jeder Forscher auf diesem Gebiet sehr daran interessiert, wie der Autor es gemacht hat.
In richtigen wissenschaftlichen Veröffentlichungen ist es gängige (allgemein \textit{verlangte}) Praxis, die Ausrüstung zu benennen (üblicherweise inkl. der Hersteller und Modellnummern) und sogar wie sie benutzt wurde.
Die resultierenden Daten sind einige der wenigen neuen Informationen, die in diesem Papier präsentiert werden, und sind deshalb in diesem Artikel von größter Wichtigkeit.
Zukünftige Forscher werden wahrscheinlich dieser Vorgehensweise folgen müssen, indem sie die Saitenverschiebungen detaillierter messen und werden diese Information über die Ausrüstung benötigen.

Auf die vier Abbildungen auf Seite 122 wird nirgends im Artikel verwiesen.
Somit bleibt es uns überlassen, zu vermuten, welche Teile des Artikels dazu gehören.
Auch ist meine Vermutung, daß die beiden unteren Diagramme, die Oszillationen zeigen, nur schematische Darstellungen sind und nichts repräsentieren, das nahe an tatsächliche Daten herankommt.
Ansonsten wäre der Anschlagsklang gemäß dieser Diagramme ungefähr nach 1/40 Sekunde zu Ende.
Die in diesen beiden Diagrammen gezeichneten Kurven sind nicht nur schematisch, sondern zusätzlich rein imaginär.
Es gibt keine Daten, die sie untermauern.
Tatsächlich präsentiert der Artikel keine weiteren neuen Daten und die Diskussionen auf den darauffolgenden 5 Seiten (aus einem achtseitigen Artikel) sind im Grunde eine Übersicht bekannter akustischer Prinzipien.
Deshalb sollten die Beschreibungen der federnden, massiven und standfesten Enden genau wie die der \hyperref[mitschwingung]{Mitschwingungen}\index{Mitschwingungen} qualitativ gültig sein.

Die Hauptthese dieses Artikels ist, daß das Klavier einmalig ist, weil es einen Nachklang hat, und daß das richtige Stimmen des Nachklangs das Wesentliche einer guten Stimmung ist und die einmalige Klaviermusik erzeugt.
Meine Schwierigkeit mit dieser These ist, daß der Anschlagsklang typischerweise mehr als 5 Sekunden dauert.
Sehr wenige Klaviernoten werden derart lange gespielt.
Deshalb wird im Grunde die ganze Klaviermusik nur mit dem Anschlagsklang gespielt.
Tatsächlich benutzen Klavierstimmer hauptsächlich den Anschlagsklang (so wie er hier definiert wird) zum Stimmen.
Außerdem ist der Nachklang mindestens um 30 db schwächer; er beträgt nur ein paar Prozent des anfänglichen Klangs.
Er wird in den anderen Noten eines jeden Musikstücks völlig untergehen.
In Wirklichkeit ist es so, daß was immer die Qualität des Klavierklangs kontrolliert, sowohl den Anschlags- als auch den Nachklang kontrolliert, und was wir brauchen ist eine Abhandlung, die Licht in diesen Mechanismus bringt.

Schließlich brauchen wir eine Publikation mit richtigen Quellenangaben, so daß wir wissen können, was bereits untersucht wurde und was nicht.
(Zur Verteidigung des Autors: Scientific American erlaubt keine Quellenangaben außer zu bereits im Scientific American veröffentlichten Artikeln.
Das macht es notwendig, Artikel zu schreiben, die \enquote{selbstbezüglich} sind, was dieser Artikel nicht ist.
Gemäß Reblitz [S. 14], gibt es einen Artikel im Scientific American von 1965 über \enquote{The Physics of the Piano}, aber auf diesen Artikel wird in diesem Bericht nicht verwiesen.)


\label{Lectures}

<br>\textbf{Five Lectures on the Acoustics of the Piano}
<br>(www.speech.kth.se/music/5_lectures/contents.html)

Eine sehr moderne Vorlesungsreihe darüber, wie das Klavier seinen Klang erzeugt.
Die Einführung bespricht die Geschichte des Klaviers und präsentiert die Terminologie und Hintergrundinformation, die notwendig sind, um die Vorlesungen zu verstehen.

Die erste Vorlesung bespricht Faktoren des Klavierdesigns, die den Klang und die akustische Leistung beeinflussen.
Hämmer, Resonanzboden, Rahmen, Platte, Saiten, Stimmwirbel und wie sie zusammenarbeiten.
Stimmer stimmen die transversalen Schwingungsmoden der Saite, aber die longitudinalen \hyperref[moden]{Moden}\index{Moden} sind durch den Aufbau der Saite und der Tonleiter festgelegt und können vom Stimmer nicht gesteuert werden, haben aber hörbare Effekte.

Die zweite Vorlesung konzentriert sich auf den Klang des Klaviers.
Der Hammer hat zwei Biegemoden, eine Schaftbiegungsmode und eine schnellere Vibrationsmode.
Die erste wird durch die rasche Beschleunigung des Hammers verursacht, ähnlich der Biegung des Golfschlägers.
Die zweite ist am ausgeprägtesten, wenn der Hammer von den Saiten zurückspringt, kann aber auch auf seinem Weg zu den Saiten angeregt werden.
Klar ist der Fänger ein wichtiges Werkzeug, das der Klavierspieler benutzen kann, um diese zusätzlichen Hammerbewegungen zu reduzieren oder kontrollieren und dadurch den Klang zu kontrollieren.
Die tatsächliche zeitabhängige Saitenbewegung ist völlig anders als die Bewegung schwingender Saiten, wie sie in Lehrbüchern gezeigt wird, mit Grundschwingungen und harmonischen Obertönen, die ganzzahlige Bruchteile der Wellenlängen sind, die fein säuberlich zwischen die befestigten Enden der Saiten passen.
Sie ist in Wahrheit eine Gruppe von wandernden Wellen, die durch den Hammer in Richtung der Brücke und der Agraffe entsendet werden.
Diese wandern so schnell, daß der Hammer für einige Durchläufe -- vorwärts und rückwärts -- auf den Saiten \enquote{steckenbleibt}, und es ist die Energie einer dieser Wellen, die auf den Hammer treffen, die ihn schließlich in Richtung Fänger zurückwirft.
Wie werden nun die Grund- und Partialschwingungen erzeugt?
Einfach -- sie sind nur die Fourier-Komponenten der wandernden Wellen!
Nichtmathematisch ausgedrückt: Die einzigen in diesem System möglichen wandernden Wellen sind Wellen, die hauptsächlich die Grund- und Partialschwingungen enthalten, weil das System durch die festen Enden eingegrenzt wird.
Das Ausklingen und die harmonische Verteilung reagieren auf die genauen Eigenschaften des Hammers, wie Größe, Gewicht, Form, Härte usw., sehr empfindlich.

Die Saiten übertragen ihre Schwingungen über die Brücke auf den Resonanzboden (RB) und die Effektivität dieses Vorgangs kann durch Messen der Übereinstimmung der akustischen Impedanz bestimmt werden.
Diese Energieübertragung wird durch die Resonanzen kompliziert, die im RB durch seine  Eigenschwingungen erzeugt werden, weil die Resonanzen Spitzen und Täler in der Impedanz/Frequenz-Kurve erzeugen.
Die Effizienz der Klangerzeugung ist bei niedrigen Frequenzen gering, weil die Luft einen \enquote{Schlußspurt} um das Klavier herum machen kann, so daß eine Druckwelle über dem RB das Vakuum unter ihm aufheben kann, wenn der RB aufwärts vibriert (und umgekehrt, wenn er sich abwärts bewegt).
Bei einer hohen Frequenz erzeugen die Vibrationen des RBs zahlreiche kleine Gebiete, die sich in verschiedene Richtungen bewegen.
Wegen ihrer Nähe kann komprimierte Luft in einem Bereich das Vakuum in einem angrenzenden Bereich aufheben, was zu einem geringeren Schall führt.
Das erklärt, warum ein kleiner Anstieg in der Klaviergröße, besonders bei niedrigen Frequenzen, die Schallerzeugung stark erhöhen kann.
Diese Komplikationen machen klar, daß es eine monumentale Aufgabe ist, die akustischen Effizienzen über alle Noten des Klaviers hinweg aufeinander abzustimmen und erklärt, warum gute Klaviere so teuer sind.

Das obige ist mein Versuch einer kurzen Übersetzung von hochtechnischem Material und ist wahrscheinlich nicht 100\% richtig.
Mein Hauptzweck ist, dem Leser eine gewisse Vorstellung vom Inhalt der Vorlesungen zu vermitteln.
Diese Website enthält eindeutig sehr wissenschaftliches Material.


<h3><br>\underline{Weitere Quellen}</h3>

\begin{itemize} 
 \item Bach Bibliography (www.music.qub.ac.uk/tomita/bachbib/).
 \item Bertrand, OTT., \textit{Liszt et la Pédagogie du Piano, Collection Psychology et Pédagogie de la Musique}, (1978) E. A. P. France.
 \item Boissier, August., \textit{A Diary of Franz Liszt as Teacher 1831-32}, übersetzt von Elyse Mach.
 \item Chan, Angela, \textit{Comparative Study of the Methodologies of Three Distinguished Piano Teachers of the Nineteenth Century: Beethoven, Czerny and Liszt} (www.geocities.com/Paris/Metro/5453/maped.htm).
 \item Fay, Amy, \textit{Music Study in Germany.}
 \item Fine, Larry, \textit{The Piano Book}, Brookside Press, 4. Ausgabe, Nov. 2000.
 \item Fischer, J. C., \textit{Piano Tuning}, Dover, N.Y., 1975.
 \item Howell, W. D., \textit{Professional Piano Tuning}, New Era Printing Co., Conn. 1966.
 \item Jaynes, E. T., \textit{The Physical Basis of Music} (bayes.wustl.edu/etj/music.html).
Erklärt, warum Liszt nicht unterrichten konnte.
Beste Erklärung des \hyperref[c1iii5a]{Daumenübersatzes}\index{Daumenübersatzes} in der 
Literatur.
 \item Jorgensen, Owen H, \textit{Tuning}, Michigan St. Univ. Press, 1991.
 \item Reblitz, Arthur, \textit{Piano Servicing, Tuning, and Rebuilding,}2. Ausgabe, 1993.
 \item Moscheles, \textit{Life of Beethoven.}
 \item Sethares, William A., \textit{Adaptive tunings for musical scales}, J. Acoust. Soc. Am. 96 (1), Juli 1994, P. 10.
 \item Tomita, Yo, \textit{J. S. Bach: Inventions and Sinfonia} (www.music.qub.ac.uk/~tomita/essay/inventions.html), 1999.
 \item White, W. B., \textit{Piano Tuning and Allied Arts}, Tuners' Supply Co., Boston, Mass, 1948.
 \item Young, Robert W., \textit{Inharmonicity of Plain Wire Piano Strings}, J. Acoust. Soc. Am., 24 (3), 1952.
 \item \footnote{Zahlreiche Beiträge des Usenet-Forums rec.music.makers.piano.}
 \end{itemize}

\label{Websites}

<h3><br>\underline{Websites, Bücher, Videos}</h3>

\footnote{Wegen der unklaren deutschen Rechtslage hinsichtlich der Mitverantwortung für Inhalte von Seiten, zu denen man Links in seinen Seiten anbietet, führe ich die Links aus der Originalseite hier nicht mit auf.
Wer also wissen möchte, welche Websites Chuan C. Chang für weitergehende Informationen empfiehlt, den verweise ich auf das \hyperref[http://www.pianopractice.org]{Original dieser Seite} (extern).
Ich erspare mir (und den LeserInnen) auch die Wiederholung der umfangreichen Liste der Bücher und Videos und verweise wiederum auf das Original.<br>
Zu den einzelnen dort angeführten Büchern und Videos kann ich nicht viel sagen, da ich die meisten nicht kenne.
In meinem Bücherregal stehen (das soll jetzt keine Empfehlung sein, sondern nur als Beispiel dienen, und ich verdiene auch nichts damit, daß ich das hier schreibe!):<br>
- aus den Anfängen \enquote{Der junge Pianist}<br>
- die deutsche Ausgabe von \hyperref[Gieseking]{Leimer/Gieseking}\index{Leimer/Gieseking}<br>
- die amerikanische Ausgabe von Arthur Reblitz: \enquote{Piano Servicing, Tuning, and Rebuilding} (habe ich bisher noch nicht in Deutsch gesehen)<br>
- für den Einstieg in die Musiktheorie \enquote{der Ziegenrücker}<br>
- ein paar Bücher mit Analysen<br>
- und natürlich Noten, Noten, Noten}
 





