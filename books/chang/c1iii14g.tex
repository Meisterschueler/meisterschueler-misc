% File: c1iii14g

\subsubsection{Während des Konzerts}
\label{c1iii14g}

Die \hyperref[c1iii15]{Nervosität} ist im allgemeinen unmittelbar bevor man anfängt zu spielen am größten.
Haben Sie erst einmal angefangen, werden Sie so mit dem Spielen beschäftigt sein, daß die Nervosität normalerweise vergessen ist und weniger wird.
Dieses Wissen kann sehr beruhigend sein, so daß nichts falsch daran ist, mit dem Spielen anzufangen, sobald Sie sich für das Konzert an das Klavier gesetzt haben.
Einige verzögern das Starten, indem sie die Bank justieren oder etwas an der Kleidung richten, um genügend Zeit zu haben, das Anfangstempo usw. noch einmal mit Hilfe des \hyperref[c1ii12mental]{mentalen Spielens} zu prüfen.

Nehmen Sie nicht an, daß Sie keine Fehler machen werden; das wird Sie nur in zusätzliche Schwierigkeiten bringen, da Sie sonst auf verlorenem Posten stehen werden, wenn ein Fehler auftritt.
Seien Sie bereit, bei jedem Fehler richtig zu reagieren, oder wichtiger noch, einen drohenden Fehler vorherzusehen, den Sie vermeiden können.
Es ist erstaunlich, wie oft man einen drohenden Fehler erahnen kann, bevor er auftritt, besonders wenn man das mentale Spielen gut beherrscht.
Das Schlechteste, das die meisten Schüler tun, wenn sie einen Fehler machen oder wenn sie einen erwarten, ist, ängstlich zu werden und anzufangen langsamer und leiser zu spielen.
Das kann zur Katastrophe \textit{führen}.
Obwohl das \hyperref[c1iii6d]{Hand-Gedächtnis} nichts ist, von dem man abhängen möchte, ist das ein Zeitpunkt, an dem Sie einen Vorteil daraus ziehen können.
Das Hand-Gedächtnis hängt von der Gewohnheit und von Reizen ab - die Gewohnheit, viele Male geübt zu haben und die Reize durch die vorangegangenen Noten, die zu den nachfolgenden Noten führen.
Um das Hand-Gedächtnis zu verstärken, müssen Sie deshalb etwas schneller und lauter spielen; genau das Gegenteil von dem, was eine verängstigte Person während eines Konzerts tun würde (eine weitere kontraintuitive Situation!).
Das schnellere Spielen nutzt die Spielgewohnheit besser aus und läßt weniger Zeit dafür, einen falschen Muskel zu bewegen, der Sie aus den gewohnten Bahnen lenkt.
Das festere Spielen erhöht den Reiz für das Hand-Gedächtnis.
Nun sind schnelleres und lauteres Spielen während eines Konzerts angsteinflößende Dinge, die Sie deshalb zu Hause genauso wie alles andere üben sollten.
Lernen Sie, Fehler vorauszusehen und sie zu vermeiden, indem Sie diese Vermeidungsmethoden benutzen.
Eine andere Methode für das \enquote{durch Fehler hindurchspielen} ist, sicherzustellen, daß die Melodielinie nicht unterbrochen wird, auch wenn dadurch ein paar Noten der \enquote{Begleitung} ausgelassen werden.
Wenn Sie Übung darin haben, werden Sie es leichter finden als es klingt; am besten üben Sie es beim \hyperref[c1iii11]{Spielen vom Blatt}.
Eine weitere Methode ist, zumindest den Rhythmus zu halten.
Selbstverständlich ist das alles nicht notwendig, wenn Sie über ein sicheres mentales Spielen verfügen.

Falls Sie eine Gedächtnisblockade haben, versuchen Sie nicht, von dort anzufangen, wo Sie den Faden verloren haben, solange Sie nicht genau wissen, wie Sie anfangen müssen.
Fangen Sie bei einem vorangegangenen Abschnitt oder einem nachfolgenden Abschnitt an, den Sie gut kennen (vorzugsweise bei einem nachfolgenden Abschnitt, weil Fehler üblicherweise während eines Konzerts nicht korrigiert werden können und Sie deshalb denselben Fehler erneut machen).
\textbf{Ein sicheres mentales Spielen wird praktisch alle Gedächtnisblockaden eliminieren.}
Wenn Sie sich dafür entscheiden, den Teil mit der Gedächtnisblockade noch einmal zu spielen, dann spielen Sie ihn etwas schneller und lauter, nicht langsamer und leiser, weil das fast mit Sicherheit zu einer Wiederholung der Gedächtnisblockade führen wird.

In einer Konzerthalle mit guter Akustik wird der Schall des Klaviers vom Raum absorbiert, so daß man von seinem eigenen Spielen sehr wenig hört.
Es ist offensichtlich wichtig, vor der Veranstaltung auf dem Konzertklavier in der Konzerthalle zu proben.
Wenn bei einem Flügel der Notenständer aufgestellt ist, wird man sogar noch weniger hören; lassen Sie den Notenständer deshalb immer unten.
Wenn Sie Noten lesen müssen, dann legen Sie sie im Bereich der Stimmwirbel flach hin.


\subsubsection{Das ungewohnte Klavier}
\label{c1iii14h} 

Einige Schüler sind besorgt darüber, daß das Konzertklavier ein großer Flügel ist, während sie auf einem kleinen Klavier üben.
Zum Glück ist es leichter, auf einem großen Klavier zu spielen als auf einem kleinen.
Deshalb muß man sich beim typischen Schülerkonzert üblicherweise keine Gedanken über die unterschiedlichen Klaviere machen.
Größere Klaviere haben im allgemeinen eine bessere Mechanik, und sowohl lautere als auch leisere Töne können auf ihnen leichter erzeugt werden.
Vor allem sind Flügel leichter zu spielen als Klaviere, besonders bei schnellen, schwierigen Passagen.
Deshalb müssen Sie nur dann wegen des Klaviers besorgt sein, wenn das Konzertklavier entschieden minderwertiger als Ihr Übungsklavier ist.
Die schlechteste Situation ist, wenn Ihr Übungsklavier ein sehr guter Flügel ist, Sie aber auf einem qualitativ schlechten Klavier spielen müssen.
In diesem Fall wird es sehr schwierig sein, technisch schwierige Stücke auf dem minderwertigen Klavier zu spielen, und Sie werden dem eventuell dadurch Rechnung tragen müssen, daß Sie z.B. mit einem geringeren Tempo spielen, den Triller verkürzen oder verlangsamen usw.
Die Mechanik von Flügeln kann etwas schwerer als die von Klavieren sein, was einigen Anfängern Probleme bereiten kann.
Es ist immer ratsam, vor dem Konzert auf dem Konzertklavier zu üben.


\label{c1iii14Stimmung}

Ein weiterer wichtiger Faktor ist die \hyperref[c2_1]{Stimmung des Klaviers}.
Ein gut gestimmtes Klavier ist leichter zu spielen als ein verstimmtes.
Deshalb ist es eine gute Idee, das Konzertklavier direkt vor dem Konzert zu stimmen.
Im Gegensatz dazu ist es keine gute Idee, das Übungsklavier direkt vor dem Konzert zu stimmen, außer wenn es stark verstimmt ist.
Wenn das Konzertklavier verstimmt ist, ist es vielleicht am besten, ein wenig schneller und lauter zu spielen als Sie beabsichtigten.


\subsubsection{Nach dem Konzert}
\label{c1iii14i}

Gehen Sie nach dem Konzert die Ergebnisse durch, und ermitteln Sie Ihre Stärken und Schwächen, so daß Sie die Art und Weise des Übens und Ihrer Vorbereitungen auf die Konzerte verbessern können.
Einige wenige Schüler werden in der Lage sein, stets ohne hörbare Fehler zu spielen.
Die meisten anderen werden jedesmal wenn sie spielen ein paar Fehler machen.
Einige werden dazu neigen, auf das Klavier einzuhämmern, während andere schüchtern sind und zu leise spielen.
Es gibt ein Mittel gegen jedes Problem.
Diejenigen, die Fehler machen, haben wahrscheinlich noch nicht gelernt, ausreichend musikalisch zu spielen, und können fast immer nicht \hyperref[c1ii12mental]{in Gedanken spielen}.
Diejenigen, die in der Regel fehlerfrei spielen, haben ohne Ausnahme das mentale Spielen gelernt, egal ob sie es bewußt gebrauchen oder nicht.

Wie bereits an anderer Stelle gesagt, \textbf{ist es das Schwerste, mehrere Konzerte hintereinander zu spielen.
Wenn Sie es aber müssen, dann müssen Sie die Konzertstücke unmittelbar nach dem Konzert überholen.
Spielen Sie sie mit wenig oder keinem Ausdruck und mittlerer Geschwindigkeit, danach langsam.}
Wenn bestimmte Abschnitte oder Stücke während des Konzerts nicht zufriedenstellend waren, arbeiten Sie an diesen, aber nur in kleinen Abschnitten.
Wenn Sie mit voller Geschwindigkeit am Ausdruck arbeiten möchten, tun Sie das ebenfalls in kleinen Abschnitten.



