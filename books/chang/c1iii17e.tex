% File: c1iii17e

\subsubsection{Ein akustisches Klavier kaufen}
\label{c1iii17e}

Ein akustisches Klavier zu kaufen, kann für die Nichteingeweihten eine anstrengende Erfahrung sein, egal ob sie ein neues oder ein gebrauchtes kaufen.
Wenn man einen Händler mit einem guten Ruf finden kann, ist es gewiß sicherer ein neues zu kaufen, aber auch dann sind die Kosten für die anfängliche Wertminderung hoch.
Viele Klaviergeschäfte werden Ihnen ein Klavier mit einer Vereinbarung leihen, daß die Miete auf den Kaufpreis angerechnet wird, wenn Sie sich dafür entscheiden, es zu behalten.
In diesem Fall sollten Sie über den besten Kaufpreis verhandeln \textit{bevor} Sie über die Miete reden; wenn Sie sich bereits auf das Mieten geeinigt haben, haben Sie wenig Verhandlungsmöglichkeiten.
Sie werden am Ende einen höheren Anfangspreis haben, so daß der endgültige Preis, auch wenn Sie die Miete abziehen, kein günstiges Angebot ist.
Auch bei teuren Klavieren finden es viele Händler zu teuer, sie in gutem Zustand und gestimmt zu halten.
Bei diesen Händlern ist es schwierig, das Klavier durch Spielen zu testen.
Deshalb werden Klaviere oft aufs Geratewohl gekauft.
Bei in Massen produzierten Klavieren wie Yamaha oder Kawai ist die Qualität der neuen Klaviere meistens einheitlich, so daß man ziemlich genau weiß, was man bekommt.
Die Klangqualität der teureren \enquote{handgefertigten} Klaviere kann spürbar variieren, so daß es schwieriger ist, solch ein Klavier zu kaufen, wenn Sie ein gutes finden möchten.

Gute gebrauchte akustische Klaviere findet man nur schwer in Klaviergeschäften, weil spielbare Klaviere als erstes verkauft werden und die meisten Geschäfte auf einem ausgedehnten Inventar an unspielbaren sitzenbleiben.
\textbf{Offensichtlich findet man die besten Schnäppchen unter den privaten Angeboten.
Jemand, der sich nicht auskennt, wird sich einen Klavierstimmer oder -techniker engagieren müssen, um ein gebrauchtes Klavier aus dem privaten Markt zu bewerten.}
Man braucht auch jede Menge Geduld, weil gute private Angebote nicht immer dann zur Verfügung stehen, wenn man sie braucht.
Das Warten kann sich jedoch rentieren, weil das gleiche Klavier im privaten Verkauf voraussichtlich nur die Hälfte des Preises in einem Geschäft kosten wird (oder weiniger).
Es gibt eine ständige Nachfrage nach guten Klavieren, die einen vernünftigen Preis haben.
Das bedeutet, daß es nicht leicht ist, gute Angebote an gut zugänglichen Orten, wie Internet-Klaviermärkten, zu finden, weil gute Klaviere schnell verkauft sind.
Umgekehrt sind solche Orte hervorragend zum Verkaufen, besonders wenn man ein gutes Klavier anbietet.
Der beste Ort, um gute Angebote zu finden, ist die Kleinanzeigensparte von Zeitungen, insbesondere in Großstädten.
Die meisten solcher Anzeigen werden am Freitag, Samstag oder Sonntag aufgegeben.

Nur wenige Markenklaviere \enquote{behalten ihren Wert}, wenn man sie viele Jahre besitzt.
Der Rest verliert schnell an Wert, so daß es keine lohnende Alternative ist, zu versuchen, sie Jahre nach dem Kauf wieder zu verkaufen.
\enquote{Ihren Wert behalten} bedeutet, daß ihr Wiederverkaufswert mit der Inflation Schritt hält; das bedeutet nicht, daß man sie mit Gewinn verkaufen kann.
D. h. wenn Sie ein Klavier für 1.000 Euro gekauft haben und es 30 Jahre später für 2.500 Euro verkaufen, haben Sie keinen Gewinn erzielt, wenn die Inflation über diese 30 Jahre hinweg im Durchschnitt etwas über 3\% betragen hat.
Außerdem haben Sie noch die Kosten für das Stimmen und die Wartung.
So ist es z.B. billiger, alle 30 bis 40 Jahre einen nagelneuen 7ft-Flügel von Yamaha zu kaufen, als einen neuen Steinway M zu kaufen und ihn alle 30 bis 40 Jahre zu restaurieren; deshalb ist die Wahl, welches Klavier sie kaufen, keine Frage der Wirtschaftlichkeit, sondern hängt davon ab, welche Art von Klavier Sie benötigen.
Von wenigen Ausnahmen abgesehen, sind Klaviere keine gute Investition; man muß ein erfahrener Klaviertechniker sein, um auf dem Gebrauchtklaviermarkt ein Schnäppchen zu finden, das mit Gewinn verkauft werden kann.
Selbst wenn Sie so ein Schnäppchen finden, ist Klaviere zu verkaufen eine zeit- und arbeitsintensive Aufgabe.
Ziehen Sie für nähere Einzelheiten darüber, wie man ein Klavier kauft, das Buch von Larry Fine zu Rate.
Auch bei den berühmtesten Marken wird ein neu gekauftes Klavier bei der Auslieferung bereits 20 bis 30\% seines Kaufpreises verlieren, und wird im allgemeinen nach 5 Jahren nur noch die Hälfte eines vergleichbaren neuen Klaviers wert sein.
Als grobe Regel wird ein gebrauchtes Klavier in einem Klaviergeschäft ungefähr die Hälfte eines neuen Klaviers desselben Modells kosten und von Privat ungefähr ein Viertel.

Die Preise der Klaviere lassen sich grob danach ordnen, ob die Klaviere es wert sind, neu aufgebaut zu werden.
Jene die es wert sind, kosten meistens das Doppelte, wenn sie neu sind.
Praktisch alle Klaviere und alle Flügel, die in Massen produziert werden (Yamaha, Kawai usw.), werden nicht wieder aufgebaut, weil die Kosten ungefähr genauso hoch sind wie der Preis für ein neues Klavier desselben Modells.
Solche Klaviere wieder aufzubauen ist oft unmöglich, weil der Handel und die notwendigen Teile für den Wiederaufbau nicht existieren.
Klaviere, bei denen sich der Wiederaufbau lohnt, sind die von Steinway, Bösendorfer, Bechstein, Mason und Hamlin, einige von Knabe und ein paar andere.
Grob gesagt kostet der Wiederaufbau ungefähr 1/4 des Preises eines neuen Klaviers, und der Wiederverkaufswert ist ungefähr die Hälfte eines neuen; deshalb können sich die Kosten sowohl für den Restaurator als auch für den Käufer rechnen.


\subsubsection{Pflege und Wartung des Klaviers}
\label{c1iii17f}

Alle neuen Klaviere müssen nach dem Kauf mindestens ein Jahr speziell gepflegt und gestimmt werden, damit die Spannung der Saiten nicht mehr nachläßt und die Mechanik und die Hämmer sich ausbalancieren.
Die meisten Klavierhändler versuchen, die Kosten für die Pflege des Klaviers nach der Auslieferung zu minimieren.
Das setzt voraus, daß das Klavier vor der Lieferung gut vorbereitet wurde.
Viele Händler verschieben einen großen Teil der vorbereitenden Arbeiten auf die Zeit nach dem Kauf, und wenn der Käufer nichts darüber weiß, lassen sie einige Schritte eventuell ganz weg.
In dieser Hinsicht ist es bei den weniger teueren Modellen leichter, eines von Yamaha, Kawai, Petroff und ein paar anderen zu kaufen, weil das meiste der vorbereitenden Arbeiten bereits in der Fabrik durchgeführt wird.
Ein neues Klavier muß im ersten Jahr mindestens viermal gestimmt werden, damit sich die Spannung der Saiten stabilisiert.

Alle Klaviere erfordern zusätzlich zum regelmäßigen Stimmen eine Wartung.
Je besser die Qualität des Klaviers ist, desto leichter ist es im allgemeinen, die Verschlechterung, die durch normalen Verschleiß verursacht wird, zu erkennen, und deshalb sollte es auch mehr gewartet werden.
D.h. teurere Klaviere sind teurer im Unterhalt.
Typische Wartungsarbeiten sind: die Tasten richten, die Reibung reduzieren (z.B. die Piloten polieren), zusätzliche Töne eliminieren, die Hämmer in Form bringen und sie intonieren (nadeln), die unzähligen Buchsen überprüfen usw.
Die \hyperref[c2_7_hamm]{Hämmer zu intonieren} ist wahrscheinlich die am meisten vernachlässigte Wartungsarbeit.
Abgenutzte, harte Hämmer können einen Saitenbruch, den Verlust der musikalischen Kontrolle und ein erschwertes leises Spielen verursachen (die letzten zwei Punkte sind schlecht für die technische Entwicklung).
Sie ruinieren auch die Klangqualität des Klaviers, machen es schrill und unangenehm für das Ohr.
Wenn die Mechanik genügend abgenutzt ist, braucht sie eventuell eine Generalüberholung, d.h. alle Teile der Mechanik werden wieder gemäß der ursprünglichen Spezifikation hergerichtet.

Wenn die drahtumwickelten Baßsaiten rostig sind, kann das diese Noten absterben lassen.
Diese Saiten zu ersetzen kann sich sehr lohnen, wenn diese Baßnoten schwach sind und keinen Sustain haben.
Die oberen, nicht umwickelten Saiten müssen im allgemeinen nicht ersetzt werden, auch wenn sie rostig sind.
Bei extrem alten Klavieren können diese Saiten jedoch so auseinandergezogen sein, daß sie ihre ganze Elastizität verloren haben.
Solche Saiten sind anfällig für Brüche, können nicht richtig schwingen, erzeugen einen blechernen Klang und sollten ersetzt werden.

Klavierspieler sollten sich mit etwas Grundwissen über das \hyperref[c2_1]{Stimmen} vertraut machen, wie z.B. den Teilen eines Klaviers, Stimmungen, Stabilität der Stimmung und Auswirkungen von Temperatur- und Luftfeuchtigkeitsänderungen, damit sie in der Lage sind, sich mit einem Stimmer zu unterhalten und zu verstehen, was er tun muß.
Zu viele Klavierbesitzer wissen nichts über diese Grundlagen; infolgedessen frustrieren sie den Stimmer und arbeiten in Wahrheit gegen ihn, mit dem Ergebnis, daß das Klavier nicht richtig gewartet wird.
Einige Besitzer gewöhnen sich so sehr an ihr \enquote{verfallenes} Klavier, daß Sie, wenn der Stimmer eine gute Arbeit dabei leistet, dem Klavier wieder seinen ursprünglichen Glanz zu verleihen, sehr unglücklich mit dem fremdartigen neuen Klang und Gefühl des Klaviers sind.
Abgenutzte Hämmer neigen dazu, übermäßig helle und laute Töne zu erzeugen; das hat den unerwarteten Effekt, daß sich die Mechanik leicht anfühlt.
Deshalb können richtig intonierte Hämmer am Anfang den Eindruck erwecken, daß die Mechanik nun schwerer ist und weniger gut anspricht.
Natürlich hat der Stimmer nicht die Kraft geändert, die notwendig ist, um die Tasten niederzudrücken.
Haben sich die Besitzer erst einmal an die neu intonierten Hämmer gewöhnt, werden sie finden, daß sie eine viel bessere Kontrolle über den Ausdruck und den Ton haben, und daß sie nun sehr leise spielen können.

Klaviere müssen mindestens einmal im Jahr gestimmt werden; besser wäre zweimal, während des Frühjahrs und im Herbst, wenn die Temperatur und die Luftfeuchtigkeit in der Mitte zwischen ihren jährlichen Extremen sind.
Viele fortgeschrittene Klavierspieler lassen sie sogar öfter stimmen.
Zusätzlich zu den offensichtlichen Vorteilen, daß man in der Lage ist bessere Musik zu erzeugen und seine Musikalität schärft, gibt es viele zwingende Gründe, das Klavier gestimmt zu halten.
Einer der wichtigsten ist, daß es Ihre technische Entwicklung beeinflussen kann.
Verglichen mit einem verstimmten Klavier spielt sich ein gut gestimmtes Klavier wie von selbst - Sie werden es überraschend leichter finden, es zu spielen.
Deshalb kann ein gestimmtes Klavier tatsächlich Ihre technische Entwicklung beschleunigen.
Ein verstimmtes Klavier kann zu Spielfehlern führen und zur Angewohnheit zu stottern, d.h. bei jedem Fehler anzuhalten.
Viele wichtige Aspekte des Ausdrucks lassen sich nur auf einem gut gestimmten Klavier richtig herausarbeiten.
Da wir stets darauf bedacht sein müssen, \hyperref[c1iii14d]{musikalisch zu üben}, macht es keinen Sinn, auf einem Klavier zu üben, das keine richtige Musik erzeugen kann.
Das ist einer der Gründe, warum ich \hyperref[c2_2_wtk2]{wohltemperierte Stimmungen} (mit ihren kristallklaren Intervallen) der \hyperref[c2_6_et]{gleichmäßigen Stimmung} vorziehe, in welcher nur die Oktaven rein sind.
Sehen Sie dazu in \hyperref[c2_1]{Kapitel 2} mehr über die Vorzüge der verschiedenen Stimmungen.
Klaviere höherer Qualität haben einen eindeutigen Vorteil, weil sie nicht nur die Stimmung besser halten, sondern auch genauer gestimmt werden können.
Klaviere niedrigerer Qualität haben oft zusätzliche Schwebungen und Töne, die ein genaues Stimmen unmöglich machen.

Diejenigen, die ein \hyperref[c1iii12]{absolutes Gehör} haben, haben mit verstimmten Klavieren große Schwierigkeiten.
Wenn man das absolute Gehör hat, können sehr verstimmte Klaviere den altersbedingten schrittweisen Verlust des absoluten Gehörs beschleunigen.
Babys und sehr junge Kinder können das absolute Gehör automatisch erwerben, wenn sie den Klang des Klaviers oft genug hören, auch wenn sie keine Vorstellung davon haben, was das absolute Gehör ist.
Damit sie das richtige absolute Gehör erwerben, muß das Klavier gestimmt sein.

Wenn Sie immer auf einem gestimmten Klavier üben, werden Sie es schwer haben, auf einem verstimmten zu spielen.
Die Musik kommt nicht heraus, man macht unerwartete Fehler und hat Gedächtnisblockaden.
Das trifft auch dann zu, wenn man nichts über das Stimmen weiß und nicht sagen kann, ob eine einzelne Note verstimmt ist.
Für einen Klavierspieler ohne Erfahrung im Stimmen ist ein Stück zu spielen der beste Weg, die Stimmung zu testen.
Eine gute Stimmung ist für jeden Klavierspieler phantastisch.
Durch das Spielen eines Musikstücks können die meisten Klavierspieler leicht den Unterschied zwischen einer schlechten und einer ausgezeichneten Stimmung hören, sogar wenn sie nicht den Unterschied durch das Spielen einzelner Noten oder das Testen von Intervallen angeben können (unter der Annahme, daß sie nicht auch Klavierstimmer sind).
Deshalb muß jeder Klavierspieler, neben der technischen Entwicklung, lernen, die Vorteile einer guten Stimmung zu hören.
Es ist vielleicht eine gute Idee, ab und zu auf einem verstimmten Klavier zu üben, damit man weiß, was einen erwartet, wenn man gebeten wird, auf einem Klavier mit zweifelhafter Stimmung zu spielen.
Bei Konzerten sollte das Konzertklavier direkt vor dem Konzert gestimmt werden, so daß das Konzertklavier eine bessere Stimmung hat als das Übungsklavier.
Versuchen sie, den umgekehrten Fall zu vermeiden, bei dem das Übungsklavier besser gestimmt ist als das Konzertklavier.
Das ist ein weiterer Grund, warum Schüler, die auf preisgünstigen Klavieren üben, wenig Probleme damit haben, auf großen, ungewohnten Flügeln zu spielen, solange die Flügel gestimmt sind.

Insgesamt gesehen sind Flügel für die technische Entwicklung ungefähr bis zur Mittelstufe nicht notwendig, obwohl sie in jeder Stufe nützlich sind.
Oberhalb der Mittelstufe werden die Argumente, die Flügel gegenüber Klavieren favorisieren, stichhaltiger.
Flügel sind besser, weil ihre Mechanik schneller ist, sie genauer gestimmt werden können, einen größeren Dynamikumfang haben, über ein wahres Dämpferpedal verfügen, mehr Kontrolle über Ausdruck und Klang gestatten können (man kann den Deckel öffnen) und so eingestellt werden können, daß sie eine größere Gleichmäßigkeit der Noten bieten (durch den Gebrauch der Schwerkraft statt von Federn).
Diese Vorteile sind jedoch zunächst verglichen mit der Liebe des Schülers zur Musik, seinem Fleiß und den korrekten Übungsmethoden gering.
Flügel werden für fortgeschrittene Schüler wünschenswerter, weil technisch herausforderndes Material auf einem Flügel leichter auszuführen ist.
Für diese fortgeschrittenen Klavierspieler werden das richtige \hyperref[c2_1]{Stimmen}, das Einstellen des Klaviers und das \hyperref[c2_7_hamm]{Intonieren der Hämmer} wesentlich, denn wenn die Wartung des Klaviers vernachlässigt wird, gehen die ganzen Vorteile praktisch verloren.
 

\label{c1iii17g}
\label{digital}

\footnote{Die folgenden Absätze sind wieder eine Einfügung, die ich wegen der besseren Lesbarkeit in normaler Schrift gelassen habe.}


\subsubsection{Anmerkungen zu Digitalpianos}

An dieser Stelle muß ich noch einmal eine Lanze für die Digitalpianos brechen.
Klar hat Chuan C. Chang Recht, daß fortgeschrittene Techniken nur auf einem akustischen Klavier richtig zu lernen und anzuwenden sind.
Wer also höhere Ambitionen hat, der sollte auf alle Fälle ein qualitativ hochwertiges akustisches Klavier oder besser einen Flügel kaufen und regelmäßig stimmen und warten lassen.
Für alle anderen Klavierspieler (mich eingeschlossen) ist ein gutes - wohlgemerkt ein gutes! - Digitalpiano völlig ausreichend.
Es gibt mittlerweile einige Digitalpianos, die hinsichtlich des Ansprechverhaltens und Spielgefühls (Stichwort \textit{gewichtete Hammertastatur}) nicht oder kaum noch von den akustischen Klavieren zu unterscheiden sind, die man sich im allgemeinen gönnt.
Bis jetzt hatte ich jedenfalls nie Schwierigkeiten, bei Bekannten \enquote{mal was vorzuspielen}.

Vorteile von Digitalpianos sind z.B. die sehr geringen Unterhaltskosten (kein Stimmen, normalerweise selten bis nie Wartung oder Reparaturen, im Grunde nur ein paar Cent für den Strom), daß man mit Kopfhörern üben kann ohne jemanden zu stören und sie fast alle MIDI-fähig sind (s. Anmerkungen zum \hyperref[c1iii13MIDI]{Aufnehmen}).
Der Preis eines Digitalpianos ist meistens wesentlich niedriger als der eines akustischen Klaviers, hängt aber auch stark von der Optik ab.
Ein Digitalpiano im hochglanzpolierten Holz(imitat)gehäuse ist gewöhnlich teurer als ein \enquote{Stage-Piano} auf einem möglichst stabilen Keyboardständer aber deshalb nicht zwangsläufig auch technisch besser.

Digitalpianos bieten mehrere Klänge: neben diversen Klavieren, Flügeln und Orgeln teilweise auch völlig andere Instrumente wie z.B. Streich- und Blasinstrumente, Synthesizerklänge und Schlagzeuge.
Die Qualität der einzelnen Klänge ist sehr unterschiedlich.
Teilweise sind sie wirklich erstklassig, teilweise von den Herstellern anscheinend nur als Zugabe gedacht; letzteres gilt vor allem für die Klänge, die keine Klaviere oder Flügel sind.
Bei manchen Geräten lassen sich die Klänge auch noch verändern (Hüllkurven, Effektgeneratoren usw.), bzw. \enquote{layern}, d.h. die einzelnen Noten werden mit mehreren verschiedenen Klängen gleichzeitig wiedergegeben.
Dadurch kann man interessante Effekte erzielen und zum Teil sogar die Klänge verbessern.
Sehr gut ist, wenn man dann seine ermittelten Einstellungen noch als \enquote{Preset} speichern kann, damit man beim nächsten Einschalten nicht wieder von vorne anfangen muß, bzw. auf der Bühne schnell umschalten kann.


\footnote{Ende der Einfügung.}



