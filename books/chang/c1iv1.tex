% File: c1iv1

\chapter{Mathematische Theorien des Klavierspielens}\hypertarget{c1iv1}{}

\section{Wozu braucht man mathematische Theorien?}

Jede Disziplin kann von einer grundlegenden mathematischen Theorie profitieren, wenn eine gültige Theorie formuliert werden kann.
Jedes Feld, das erfolgreich mathematisch aufbereitet wurde, hat zwangsläufig sprunghafte Fortschritte gemacht.
Wenn die Theorie erst einmal korrekt formuliert ist, dann können die mächtigen mathematischen Werkzeuge und Schlußfolgerungen mit großer Sicherheit angewandt werden.
Im folgenden finden Sie meinen ersten Versuch einer solchen Formulierung für das Klavierspielen.
Soweit ich weiß, ist es der erste seiner Art in der Geschichte der Menschheit.
Solche unerforschten Gebiete haben in der Vergangenheit sehr schnell einen enormen Nutzen erfahren.
Ich war selbst überrascht, wie viele nützliche, und manchmal bisher unbekannte, Schlußfolgerungen wir aus sehr rudimentären Theorien ziehen können.
Egal welche Mathematik ich im folgenden benutzen werde, es wird eine wirklich einfache Mathematik sein.
Bereits in diesem frühen Stadium können wir mit den einfachsten Konzepten viel erreichen.
Weitere Fortschritte sind offensichtlich durch die Anwendung höherer Mathematik möglich.
Ich werde auch ein paar dieser Möglichkeiten besprechen.

Es wird kaum in Frage gestellt, daß die Kunst des Klavierspielens unter einem totalen Mangel an mathematischer Analyse leidet.
Außerdem bezweifelt niemand, daß Geschwindigkeit, Beschleunigung, Impuls, Kraft, usw. beim Klavierspielen eine entscheidende Rolle spielen.
Unabhängig davon, welch ein Genie in dem Künstler steckt, muß die Kunst durch Fleisch und Knochen und durch eine mechanische Vorrichtung aus Holz, Filz und Metall übermittelt werden.
Deshalb befassen wir uns hier nicht nur mit einem mathematischen, sondern auch mit einem völlig wissenschaftlichen Ansatz, der die menschliche Physiologie, Psychologie, Mechanik und Physik einbezieht, die vereint das repräsentieren, was wir am Klavier tun.

Die Notwendigkeit für ein solches Vorgehen zeigt sich anhand der Tatsache, daß es viele Fragen gibt, auf die wir immer noch keine Antworten wissen.
Was ist eine Geschwindigkeitsbarriere?
Wie viele gibt es?
Was verursacht sie?
Gibt es eine Formel für das Überwinden von Geschwindigkeitsbarrieren?
Was tun Klavierspieler, wenn sie schrill bzw. sanft spielen oder flach bzw. tief?
Ist es möglich, zwei verschiedenen Klavierspielern beizubringen, dieselbe Passage in genau derselben Art zu spielen?
Gibt es irgendeine Möglichkeit, die verschiedenen Fingerbewegungen in der gleichen Weise wie die Gangarten der Pferde zu klassifizieren?
Wir werden alle diese Fragen im folgenden beantworten.

Die Vorteile einer mathematischen Theorie sind offensichtlich.
Wenn wir z.B. die Frage, was eine Geschwindigkeitsbarriere ist (oder was Geschwindigkeitsbarrieren sind - wenn die Theorie davon ausgeht, daß es mehr als eine gibt!), mathematisch beantworten können, dann sollte die Theorie uns sofort mögliche Lösungen dafür zur Verfügung stellen, diese Geschwindigkeitsbarriere(n) zu überwinden.
Heute weiß niemand, wie viele Geschwindigkeitsbarrieren es gibt.
Zu wissen, wie viele es gibt, wäre schon ein sagenhafter Fortschritt.
Es kann wichtig sein, mathematisch zu beweisen, daß zwei Klavierspieler niemals dasselbe Stück auf genau die gleiche Art spielen können (oder sogar ein einzelner Klavierspieler dasselbe Stück nicht zweimal auf die gleiche Art spielen kann).
Wenn das der Fall ist, kann es nicht schädlich sein, jemand anderem beim Spielen zuzuhören, weil man es sowieso nicht exakt imitieren kann (unter der Annahme, daß exakte Imitation nicht wünschenswert ist), und es ist dann als unmöglich bewiesen, einem Schüler beizubringen, einen berühmten Künstler exakt zu imitieren.
Das wird sicher einen Einfluß darauf haben, wie Lehrer die Schüler darin unterrichten, Beispiele von Aufnahmen berühmter Künstler zu benutzen.

Bis vor kurzem machten sich die Chemiker über die Physiker lustig, die zwar in der Lage waren, ihre Gleichungen auf viele Dinge anzuwenden, aber einfache chemische Reaktionen nicht einmal annähernd erklären konnten.
Die Biologie und die Medizin entwickelten sich anfangs ebenfalls eigenständig - mit wenig Mathematik und mit Methoden, die von fundamentaler Wissenschaft weit entfernt waren.
Medizin, Biologie und Chemie begannen ursprünglich als reine Kunst.
Mittlerweile sind alle drei Disziplinen äußerst mathematisch und basieren auf den fortgeschrittensten wissenschaftlichen Prinzipien.
Die sich daraus ergebenden Leistungen auf diesen Gebieten sind zu zahlreich, um sie hier alle aufzuführen.
Ein Beispiel: In der Chemie wurde die absolute Grundlage der Chemiker, das Periodensystem der Elemente, von den Physikern mit Hilfe der Quantenmechanik erklärt.
Als Ergebnis davon, daß sie wissenschaftlicher wurden, sind alle drei Disziplinen enorm erfolgreich und erzielen große Fortschritte.
Die \enquote{Verwissenschaftlichung} jeder Disziplin ist unvermeidlich; es ist wegen des möglichen zu erwartenden Nutzens nur eine Frage der Zeit.
Diese Verwissenschaftlichung wird sich auch für die Musik als nützlich erweisen.

Wie wenden wir also die exakte Wissenschaft der Mathematik auf etwas an, das als Kunst wahrgenommen wird?
Sicherlich wird das Ergebnis anfangs etwas grob sein, aber Verfeinerungen werden garantiert folgen.
Klaviertechniker wissen bereits, daß das Klavier selbst in seinem Design ein Wunder im Gebrauch der physikalischen Grundlagen ist.
Klaviertechniker müssen mit einem enormen Maß an Wissenschaft, Mathematik und Physik vertraut sein, um ihrer \enquote{Kunst} nachzugehen.
Eine mathematische Theorie des Klavierspielens muß mit einem wissenschaftlichen Ansatz beginnen, in dem jedes zur Diskussion stehende Element klar als Objekt definiert und klassifiziert wird; s. \enquote{\hyperlink{c3_2}{Der wissenschaftliche Ansatz}} in Kapitel 3.
Ist das erst einmal erreicht, suchen wir nach allen relevanten Beziehungen zwischen diesen Objekten.
Diese Prozeduren bilden den Kern der Gruppentheorie. Sie ist elementar! Lassen Sie uns anfangen.



