% File: c1ii23

\subsection{Haltepedal}
\label{c1ii23}

\textbf{Üben Sie jedes neue Stück ohne Pedal -- erst \hyperref[c1ii7]{mit getrennten Händen}, dann beidhändig -- bis sie beidhändig bei der endgültigen Geschwindigkeit gut zurechtkommen.
Das ist eine entscheidende Übungsmethode, die alle guten Lehrer bei all ihren Schülern benutzen.}
Es mag zunächst schwierig erscheinen, dort, wo das Pedal benötigt wird, ohne das Pedal \hyperref[c1iii14d]{musikalisch zu spielen};
das ist jedoch der beste Weg, die präzise Kontrolle zu lernen, sodass man musikalischer spielen kann, wenn man das Pedal schließlich hinzufügt.
Schüler, die von Anfang an mit dem Pedal üben, werden zu nachlässigen Spielern, entwickeln zahlreiche \hyperref[c1ii22]{schlechte Angewohnheiten} und werden nicht einmal das Konzept der präzisen Kontrolle oder der wahren Bedeutung der Musikalität lernen.

Reine Amateure benutzen das Haltepedal häufig zu oft.
\textbf{Die offensichtliche Regel ist: Wenn die Noten kein Pedal anzeigen, dann benutzen Sie es nicht.}
Bei einigen Stücken mag es so erscheinen, als wären sie mit Pedal leichter zu spielen (besonders dann, wenn man langsam und beidhändig anfängt), aber das ist eine der schlimmsten Fallen, in die ein Anfänger tappen kann, und wird die Entwicklung behindern.
Die Mechanik fühlt sich mit getretenem Haltepedal leichter an, weil der Fuß anstelle der Finger die Dämpfer von den Saiten weghält.
Deshalb fühlt sich die Mechanik schwerer an, wenn das Pedal angehoben ist, besonders bei schnellen Abschnitten.
Einige Schüler merken nicht, dass es an den Stellen, an denen kein Pedal angezeigt wird, gewöhnlich unmöglich ist, die Musik mit der vorgegebenen Geschwindigkeit korrekt zu spielen, wenn man das Pedal benutzt.

Benutzen Sie bei \enquote{Für Elise} das Pedal nur für die großen gebrochenen Akkord-Begleitungen der linken Hand (Takt 3 und ähnliche), die Takte 83-96 und die Arpeggio-Passage der rechten Hand (Takte 100-104).
Praktisch die ganze erste schwierige Unterbrechung sollte ohne das Pedal gespielt werden.
Natürlich sollte alles zunächst ohne das Pedal geübt werden, bis Sie mit dem Stück im Grunde fertig sind.
Das wird die gute Angewohnheit fördern, die Finger nahe bei den Tasten zu halten und die schlechte Angewohnheit unterbinden, mit zu häufigem Springen und Heben der Hände zu spielen und nicht fest in die Tasten zu drücken.
Ein wichtiger Grund, das Pedal am Anfang nicht zu benutzen, ist, dass die Technik sich ohne das Pedal am schnellsten verbessert, weil man ohne die Störung durch zuvor gespielte Noten genau hören kann, was man spielt.
Sie sollten den Klang aktiv kontrollieren.

Das Pedal und die Hände richtig zu koordinieren, ist keine leichte Aufgabe.
Deshalb enden Schüler, die ein Stück von Anfang an beidhändig mit dem Pedal lernen, ausnahmslos mit inkonsistenten und schlechten Pedalangewohnheiten.
Die korrekte Prozedur ist, erst einhändig ohne Pedal zu üben, dann einhändig mit Pedal, danach beidhändig ohne Pedal und zum Schluss beidhändig mit Pedal.
Auf diese Weise können Sie sich auf jede einzelne neue Fertigkeit konzentrieren, während Sie diese in Ihr Spiel einführen.

Unaufmerksamkeit dem Pedal gegenüber kann die technische Entwicklung viel mehr verzögern als vielen Schülern bewusst ist; umgekehrt kann Aufmerksamkeit dem Pedal gegenüber hilfreich für die technische Entwicklung sein, indem sie die Genauigkeit erhöht und der Musikalität eine weitere Dimension hinzufügt.
Wenn Sie eine Sache falsch machen, wird es schwierig, alle anderen Dinge richtig zu tun.
Wenn man mit dem Pedal etwas falsch macht, kann man noch nicht einmal die korrekte Fingertechnik üben, weil die Musik auch dann falsch klingt, wenn die Fingertechnik korrekt ist.

Das Pedal existierte vor Mozarts Zeit praktisch nicht;
so wird zum Beispiel im gesamten Werk von Johann Sebastian Bach kein Pedal benutzt.
Mozart hat nie ein Pedal angegeben, aber heutzutage wird in einigen seiner Kompositionen ein wenig Pedal als optional angesehen, und viele Herausgeber haben seinen Noten Pedalzeichen hinzugefügt.
Das Pedal war zu Beethovens Zeit zwar im Grunde voll entwickelt aber als ernsthaftes musikalisches Werkzeug noch nicht völlig akzeptiert.
Beethoven benutzte es mit großem Erfolg als besonderen Effekt (dritter Satz der Waldstein-Sonate);
deshalb benutzte er es oft in hohem Maß (gesamter erster Satz der Mondschein-Sonate) oder überhaupt nicht (gesamte Pathétique, erster und zweiter Satz der Waldstein-Sonate). Chopin benutzte das Pedal ausgiebig, um seiner Musik eine zusätzliche Logikebene hinzuzufügen und nutzte die verschiedenen Arten des Pedalgebrauchs vollständig aus.
Deshalb kann man Chopin (und viele spätere Komponisten) ohne ein entsprechendes Pedaltraining nicht korrekt spielen.

Schauen Sie in den \hyperref[reference]{Quellen} nach all den unterschiedlichen Arten, die Pedale zu benutzen, wann sie benutzt werden, und wie man die Bewegungen übt (\hyperref[Gieseking]{Gieseking und Leimer}, \hyperref[Fink]{Fink}, \hyperref[Sandor]{Sandor}, \textit{Pedaling the Modern Pianoforte} von Bowen und \textit{The Pianist's Guide to Pedaling} von Banowetz).
Versuchen Sie, alle diese Bewegungen zu beherrschen, bevor Sie das Pedal mit einem tatsächlichen Musikstück benutzen.
In den Quellen gibt es einige sehr hilfreiche Übungen für den richtigen Pedalgebrauch.
Wenn Sie das Pedal benutzen, müssen Sie genau wissen, welche Bewegung Sie benutzen und warum.
Wenn Sie zum Beispiel möchten, dass so viele resonante Saiten wie möglich mitschwingen, treten Sie das Pedal bevor Sie die Note spielen.
Wenn Sie jedoch nur eine klare Note aushalten möchten, treten Sie das Pedal nachdem Sie die Note spielen; je länger Sie das Pedal verzögern, desto weniger resonante Schwingungen werden Sie bekommen.
Im Allgemeinen sollten Sie sich angewöhnen, das Pedal einen Sekundenbruchteil nach dem Spielen der Note zu treten.
Sie können einen Legato-Effekt ohne zu viel Verschwimmen erzielen, indem Sie jedes Mal, wenn sich der Akkord ändert, das Pedal schnell anheben und wieder treten.
Wie bei den Tasten ist es genauso wichtig zu wissen, wann das Pedal angehoben werden muss, wie wann es getreten werden muss.
\textbf{Das Pedal muss genauso sorgfältig \enquote{gespielt} werden, wie man die Tasten spielt.}


\subsection{Dämpferpedal, Timbre und Eigenschwingungen vibrierender Saiten}
\label{c1ii24}

\textbf{Das Dämpferpedal wird bei einem Flügel benutzt, um die Stimmung des Klangs von einem perkussiven hin zu einem mehr gelassenen und sanften (bei getretenem Dämpferpedal) zu ändern.}
Es sollte nicht einzig zum Reduzieren der Lautstärke benutzt werden, weil es auch das Timbre ändert.
Um pianissimo zu spielen, muss man nur lernen, wie man leiser spielt.
Man kann bei getretenem Dämpferpedal sehr laute Töne erzeugen.
Eine Schwierigkeit mit dem Gebrauch des Dämpferpedals ist, dass es oft nicht angezeigt wird (una corda, oder richtiger due corda für den modernen Flügel), sodass die Entscheidung es zu benutzen oft dem Klavierspieler überlassen wird.
\textbf{Bei aufrecht stehenden Klavieren macht es den Klang hauptsächlich leiser.}
Das Dämpferpedal hat bei den meisten Klavieren nur einen unbedeutend kleinen Einfluss auf das Timbre.
Anders als der Flügel kann das Klavier mit getretenem Dämpferpedal keine lauten Klänge erzeugen.

\textbf{Viele Klavierspieler verstehen nicht, wie wichtig das richtige \hyperref[c2_7_hamm]{Intonieren der Hämmer} für das Funktionieren des Dämpferpedals ist.}
Wenn Sie dazu neigen, das Dämpferpedal zum leisen Spielen zu benötigen, oder wenn es deutlich leichter ist, pianissimo zu spielen, wenn der Deckel des Flügels geschlossen ist, dann ist es fast sicher, dass die Hämmer intoniert werden müssen.
Sehen Sie dazu den Abschnitt über das Intonieren in Abschnitt 7 von Kapitel 2.
Mit richtig intonierten Hämmern sollten Sie in der Lage sein, das leise Spielen ohne das Dämpferpedal in jedem gewünschten Maß zu kontrollieren.
Mit abgenutzten, verdichteten Hämmern ist leises Spielen unmöglich, und das Dämpferpedal hat eine geringere Auswirkung auf die Veränderung des Tons.
In den meisten Fällen können die ursprünglichen Eigenschaften des Hammers durch das Intonieren (Form erneuern, Nadeln usw.) wieder hergestellt werden.
Die Mechanik muss ebenfalls gut eingestellt sein, mit einem richtig minimierten Abgang, um \textit{ppp} zu ermöglichen.

\textbf{Der Gebrauch des Dämpferpedals ist umstritten, weil zu viele Klavierspieler nicht wissen, wie es funktioniert.}
Viele benutzen es zum Beispiel, um pianissimo zu spielen, was falsch ist.
Wie in Abschnitt 7 von Kapitel 2 gezeigt, ist der Energietransfer vom Hammer zur Saite beim Auftreffen am effizientesten, bevor die Saite anfängt, sich zu bewegen.
Ein verdichteter Hammer überträgt seine Energie innerhalb einer extrem kurzen Zeitspanne beim Auftreffen, und der Hammer springt sofort wieder von den Saiten zurück.
Diese hohe Effizienz der Energieübertragung erweckt den Eindruck, dass die Mechanik sehr leicht ist.
Deshalb gibt es alte Flügel, die sich federleicht anfühlen.
Weiche Hämmer auf demselben Klavier (ohne dass etwas anderes geändert wird) würden dazu führen, dass sich die Mechanik schwerer anfühlt.
Das deshalb, weil der Hammer wegen des weicheren Aufprallpunkts länger auf der Saite bleibt und die Saite aus ihrer ursprünglichen Position gehoben wird, bevor die ganze Energie des Hammers auf die Saite übertragen wurde.
In dieser Position ist der Energietransfer ineffizienter (siehe Abschnitt 7 in Kapitel 2), und der Spieler muss kräftiger drücken, um einen Ton mit derselben Lautstärke zu erzeugen.
\textbf{So kann das Intonieren das scheinbare Tastengewicht wirkungsvoller ändern als Bleigewichte.}
Klar wird das \textit{effektive} Tastengewicht nur teilweise von der zum Niederdrücken der Taste erforderlichen Kraft kontrolliert, da es auch von der Kraft abhängt, die notwendig ist, um eine bestimmte Tonstärke zu erzeugen.
Der Klavierspieler weiß nicht, welcher Faktor (Bleigewichte oder weicher Hammer) das effektive Tastengewicht beeinflusst.
Der Klaviertechniker muss einen Kompromiss eingehen.
Einerseits muss der Hammer genügend weich intoniert sein, um einen gefälligen Ton zu erzeugen, andererseits muss er genügend hart sein, um einen angemessenen Klang zu erzeugen.
Bei allen Flügeln und Klavieren, außer denen höchster Qualität, muss der Hammer eher hart sein, um einen genügend lauten Ton zu erzeugen und damit sich die Mechanik leicht beweglich anfühlt, was es erschwert, solche Klaviere leise zu spielen.
Das kann wiederum \enquote{gerechtfertigen}, das Dämpferpedal zu benutzen, wo es nicht benutzt werden sollte.
Klavierbesitzer, die das Intonieren vernachlässigen, können die Arbeit des Klavierstimmers erschweren, denn nachdem die Hämmer richtig intoniert sind, wird sich der Besitzer beschweren, dass die Mechanik nun zum Spielen zu schwer sei.
In Wahrheit hat sich der Besitzer daran gewöhnt, mit einer federleichten Mechanik zu spielen, und nie gelernt, wie man mit wahrer Kraft spielt, um diesen großartigen Klavierklang zu erzeugen.

Bei den meisten Klavieren bewirkt das Dämpferpedal, dass alle Hämmer näher zu den Saiten hin bewegt werden und so die Hammerbewegung begrenzt und die Lautstärke verringert wird.
Anders als bei Flügeln, können bei Klavieren keine lauten Töne erzeugt werden, wenn das Dämpferpedal getreten ist.
Ein Vorteil der Klaviere ist, dass ein teilweise getretenes Dämpferpedal die entsprechende Wirkung hat;
bei Flügeln ist das teilweise getretene Dämpferpedal ein komplexes Thema, das im folgenden behandelt wird.
Es gibt ein paar hochwertige Klaviere, bei denen das Dämpferpedal ähnlich funktioniert wie das der Flügel.

\textbf{Bei modernen Flügeln bewirkt das Dämpferpedal eine Verschiebung der gesamten Mechanik (einschließlich der Hämmer) nach rechts, sodass die Hämmer im dreisaitigen Abschnitt eine Saite auslassen.}
Dadurch treffen die Hämmer jeweils nur auf zwei Saiten, was eine herrliche Transformation im Klangcharakter verursacht, wie im Folgenden beschrieben.
Die Verschiebung beträgt genau den halben Abstand zwischen benachbarten Saiten im dreisaitigen Abschnitt);
dadurch treffen die beiden aktiven Saiten die weniger benutzten Bereiche des Hammers zwischen den Saitennuten, was einen noch weicheren Klang erzeugt.
Die horizontale Bewegung darf nicht einen ganzen Saitenabstand betragen, weil sonst die Saiten in die Hammernuten der benachbarten Saiten fallen würden.
Da die Saitenabstände und die Verschiebung nicht hinreichend genau kontrolliert werden können, würde dies dazu führen, dass einige Saiten genau in die Nuten fallen, während andere sie verpassen, was einen unausgewogenen Klang erzeugen würde.

Warum ändert sich das Timbre, wenn zwei statt drei Saiten angeschlagen werden?
Das Timbre wird hierbei von mindestens vier Faktoren bestimmt:

\begin{enumerate} 
 \item der Existenz der nicht angeschlagenen Saite,
 \item dem Verhältnis von Anschlagston und Nachklang,
 \item dem harmonischen Gehalt und
 \item der Polarisation der Schwingung der Saiten.
\end{enumerate}

Die nicht angeschlagene Saite dient als Reservoir, in das die beiden anderen Saiten ihre Energie abladen können, und erzeugt viele neue Effekte.
Da die Schwingung der dritten Saite in Gegenphase zu den angeschlagenen Saiten ist (eine angeregte Saite ist immer in Gegenphase zu dem \enquote{Anreger}), nimmt sie dem anfänglichen Anschlagsklang die Spitze (siehe unten), und zur gleichen Zeit erregt sie Schwingungen, die sich von denen unterscheiden, die sich ergeben, wenn alle drei vereint angeschlagen werden.
Deshalb funktioniert das Dämpferpedal in Klavieren nicht so gut\footnote{wie in Flügeln} -- auch beim Treten des Dämpferpedals werden alle Saiten angeschlagen, und das Timbre kann sich nicht ändern.

Das Klavier erzeugt zunächst einen Anschlagston und dann einen anhaltenden Nachklang;
in den im \hyperref[reference]{Quellenverzeichnis} aufgeführten Artikeln \hyperref[Lectures]{Five Lectures on the Acoustics of the Piano} und aus dem \hyperref[American]{Scientific American} finden Sie mehr Details zu den in diesem Abschnitt besprochenen Themen.
Im Gegensatz zu dem vereinfachten Bild von Grund- und Obertönen, das wir beim \hyperref[c2_1]{Klavierstimmen} benutzen, bestehen die realen Saitenschwingungen aus einer komplexen zeitabhängigen Folge von Ereignissen, die immer noch nicht vollständig verstanden werden.
In solchen Situationen sind die tatsächlichen Daten existierender Klaviere von größerem praktischen Wert, aber diese Daten sind gut gehütete Betriebsgeheimnisse der Klavierhersteller.
Deshalb fasse ich hier das auf der Physik des Klavierklangs basierende allgemeine Wissen zusammen.
Die Schwingungen der Saiten können entweder parallel zum Resonanzboden oder senkrecht dazu polarisiert sein.
Wenn die Saiten angeschlagen werden, werden vertikal polarisierte wandernde Wellen erzeugt, die sich vom Hammer in beide Richtungen bewegen: zu den Agraffen (Capotasto) und zum Steg.
Diese Wellen wandern so schnell, dass sie mehrere hundert Mal von beiden Enden der Saiten reflektiert werden und den Hammer passieren, bevor der Hammer von den Saiten zurückspringt;
es sind in Wahrheit diese Wellen, die den Hammer zurückwerfen.
Durch die vertikalen Wellen werden horizontal polarisierte Wellen erzeugt, weil das Klavier nicht symmetrisch ist.
Diese wandernden Wellen werden zu stehenden Wellen reduziert, die aus dem Grundton und harmonischen Obertönen bestehen, weil die stehenden Wellen \enquote{Eigenschwingungen} (siehe ein Lehrbuch der Mechanik) sind, die langsam Energie zum Resonanzboden übertragen und deshalb lange anhalten.
Das Konzept von Grundtönen und Obertönen bleibt jedoch von Anfang an gültig, weil die Fourier-Koeffizienten (siehe ein Lehrbuch der Mathematik oder Physik) der Grund- und Obertonfrequenzen immer groß sind, sogar für die wandernden Wellen.
Das ist leicht zu verstehen, weil die Enden der Saiten sich nicht bewegen, besonders bei gut konstruierten, großen, schweren Klavieren.
Mit anderen Worten: Wenn die Enden fest sind, werden hauptsächlich Wellenlängen mit Knoten (Punkte ohne Bewegung) an beiden Enden erzeugt.
Das erklärt, warum Stimmer trotz der wandernden Wellen genau stimmen können, indem sie nur die Frequenzen der Grund- und Obertöne benutzen.
Die vertikal polarisierten Wellen übertragen die Energie effizienter auf den Resonanzboden als die horizontal polarisierten Wellen, erzeugen deshalb einen lauteren Ton, fallen aber schneller ab und erzeugen den Anschlagston.
Die horizontal polarisierten stehenden Wellen erzeugen den Nachklang, der dem Klavier sein langes Sustain verleiht.
Wenn das Dämpferpedal getreten wird, können nur zwei Saiten den Anschlagston erzeugen, aber irgendwann tragen alle drei Saiten zum Nachklang bei.
Deshalb ist das Verhältnis von Anschlagston und Nachklang kleiner als für drei Saiten, und der Klang ist weniger perkussiv.

Der harmonische Gehalt ist ebenfalls unterschiedlich, weil die Energie des Hammers nur auf zwei statt auf drei Saiten übertragen wird.
Das wirkt so, als ob die Saite mit einem schwereren Hammer angeschlagen wird, und es ist bekannt, dass schwerere Hämmer stärkere Grundtöne erzeugen.
Die Polarisation der Saiten ändert sich durch das Dämpferpedal ebenfalls, weil die dritte Saite mehr horizontal polarisiert wird, was zu dem sanfteren Klang beiträgt.

Dieses Wissen hilft uns dabei, das Dämpferpedal richtig zu benutzen.
Wenn das Pedal getreten wird, \textit{bevor} eine Note gespielt wird, werden die anfänglichen zeitabhängigen wandernden Wellen alle Saiten anregen, was ein sanftes Dröhnen im Hintergrund erzeugt.
Das heißt, dass beim Anschlagston die nicht harmonischen Fourier-Koeffizienten ungleich Null sind.
Wenn man seinen Finger auf irgendeine Saite legt, kann man fühlen, wie diese vibriert.
Oktav- und harmonische Saiten werden jedoch mit größeren Amplituden schwingen als die dissonanten Saiten.
Das ist eine Folge der höheren Fourier-Koeffizienten der Obertöne.
Dadurch schließt das Klavier nicht nur selektiv die Obertöne ein, sondern erzeugt sie auch selektiv.
Wenn nun das Pedal getreten wird, \textit{nachdem} die Note angeschlagen wird, werden die Oktav- und die harmonischen Saiten resonant mitschwingen aber alle anderen Saiten fast ganz still sein, weil die stehenden Wellen nur reine Obertöne enthalten.
Das erzeugt eine klare, ausgehaltene Note.
Die Lektion ist hier, dass das Pedal im Allgemeinen unverzüglich nach dem Anschlagen der Note getreten werden sollte, nicht vorher, um Dissonanzen zu vermeiden.
Das ist eine gute Angewohnheit, die es zu entwickeln gilt.

Ein teilweise getretenes Dämpferpedal funktioniert bei einem Klavier; aber kann man ein halbes Dämpferpedal bei einem Flügel benutzen?
Das sollte nicht umstritten sein, ist es aber, weil sogar einige fortgeschrittene Klavierspieler glauben, dass wenn ein vollständig getretenes Dämpferpedal einen bestimmten Effekt erzeugt, dann erzeugt ein teilweise getretenes Dämpferpedal einen teilweisen Effekt, was aber nicht stimmt.
Wenn man das Dämpferpedal teilweise tritt, wird man natürlich einen neuen Klang erhalten.
Es gibt keinen Grund, warum einem Klavierspieler nicht erlaubt sein sollte, dieses zu tun, und wenn es einen interessanten neuen Effekt erzeugt, der dem Klavierspieler gefällt, ist daran nichts falsch.
Diese Art zu spielen wurde jedoch nicht mit Absicht in das Klavier konstruiert, und ich weiß von keinem Komponisten, der etwas für teilweise getretenes Dämpferpedal auf dem Flügel komponiert hätte, insbesondere da es nicht auf mehreren Klavieren und bei mehreren Noten innerhalb eines Klaviers in gleicher Weise reproduziert werden kann.
Der übermäßige Gebrauch des teilweisen Dämpferpedals auf dem Flügel führt dazu, dass einige Saiten eine Seite der Hämmer abrasieren, was das System aus der Einstellung bringt.
Auch ist es für den Klaviertechniker unmöglich, alle Hämmer und Saiten so genau auszurichten, dass jeweils die dritte Saite bei allen dreisaitigen Noten den Hammer bei derselben Pedalstellung verpasst.
Dadurch wird der Effekt des teilweisen Dämpferpedals ungleichmäßig und von Klavier zu Klavier unterschiedlich.
Deshalb ist das halbe Treten des Dämpferpedals auf einem Flügel nicht empfehlenswert, es sei denn, Sie haben damit experimentiert und versuchen, damit einen fremdartigen und nicht reproduzierbaren neuen Effekt zu erzeugen.
Nichtsdestoweniger zeigen anekdotenhafte Berichte, dass es den Gebrauch des teilweisen Dämpferpedals auf einem Flügel gibt; fast immer wegen der Unwissenheit des Klavierspielers über die Funktionsweise dieses Teils.
Die einzige Möglichkeit, ein teilweises Dämpferpedal mit reproduzierbaren Ergebnissen zu benutzen, ist, es nur ganz wenig zu treten; in diesem Fall werden alle Saiten auf die Seiten der Nuten in den Hämmern treffen.
Aber sogar dieses Verfahren funktioniert nicht wirklich, weil es nur den dreisaitigen Abschnitt beeinflusst und zu einem misstönenden Übergang vom zweisaitigen zum dreisaitigen Abschnitt führt.

Im ein- und zweisaitigen Abschnitt haben die Saiten einen viel größeren Durchmesser, sodass die Saiten die Seitenwände der Nuten treffen, wenn sich die Mechanik seitwärts bewegt, was ihnen eine horizontale Bewegung verleiht und die Nachklangkomponente vergrößert, indem die horizontal polarisierten Schwingungen der Saiten verstärkt werden.
Dadurch ist die Veränderung des Timbres der im dreisaitigen Abschnitt ähnlich.
Dieser Mechanismus ist geradezu genial!

Zusammenfassend ist der Name Dämpferpedal beim Flügel eine unzutreffende Bezeichnung.
Seine hauptsächliche Wirkung ist die Veränderung des Timbres des Klangs.
Wenn Sie einen lauten Ton mit getretenem Dämpferpedal spielen, wird er fast so laut sein wie ohne Dämpferpedal.
Das kommt daher, dass Sie ungefähr die gleiche Energiemenge in die Erzeugung des Tons gesteckt haben.
Auf der anderen Seite ist es auf den meisten Flügeln leichter leise zu spielen, wenn man das Dämpferpedal benutzt, weil die Saiten die weniger benutzten, weichen Teile der Hämmer treffen.
Vorausgesetzt, das Klavier ist gut eingestellt und die Hämmer sind richtig \hyperref[c2_7_hamm]{intoniert}, sollten Sie ohne Dämpferpedal in der Lage sein, genauso leise zu spielen.
\textbf{Ein teilweise getretenes Dämpferpedal wird unvorhersehbare, ungleichmäßige Wirkungen erzielen und sollte auf einem akustischen Flügel nicht benutzt werden.}
Ein teilweises Dämpferpedal funktioniert auf den meisten Klavieren und allen Digitalpianos.


% zuletzt geändert 05.12.2009


