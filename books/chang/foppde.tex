<!DOCTYPE HTML PUBLIC \enquote{-//W3C//DTD HTML 4.01 Transitional//EN}>
<html> 
<head>
 <title>Klavier spielen - Grundlagen, Übungen, Praxistipps: Das komplette Buch (1,5 MB)</title>
 <meta name=\enquote{author} content=\enquote{Edgar Lins}>
 <meta name=\enquote{content-language} content=\enquote{de}>
 <meta name=\enquote{keywords} content=\enquote{piano, klavier, klavierspielen, flügel, schule, kurs, unterricht, praxis, tipp, tipps, praxistipps, ueben, lernen, klavierschule, klavierunterricht, kostenlos, online, buch, onlinebuch, online-buch, chuan, chang, stimmen, intonieren, klavierstimmen, klavierstimmer, hammer, hämmer}>
 <meta name=\enquote{description} content=\enquote{Deutsche Übersetzung des Buchs}Fundamentals of Piano Practice\enquote{(2. Ausgabe) von Chuan C. Chang}>
 <meta name=\enquote{robots} content=\enquote{index, follow}>
 <meta name=\enquote{revisit-after} content=\enquote{2 days}>
 <meta http-equiv=\enquote{content-type} content=\enquote{text/html; charset=ISO-8859-1}>
</head> 
<body bgcolor=\enquote{\#ffffff} text=\enquote{\#000000} link=\enquote{\#0000ff} vlink=\enquote{\#800080} alink=\enquote{\#ff0000}>

<h1 align=\enquote{center}>Fundamentals of Piano Practice</h1>



<h1 align=\enquote{center}>Klavier spielen</h1>
<h4 align=\enquote{center}>- Grundlagen, Übungen, Praxistipps -<br>
von Chuan C. Chang (2. Ausgabe)<br>
Übersetzung: Edgar Lins<br>
</h4>

<h4 align=\enquote{center}>\textit{Letzte Änderung: 08. April 2012}</h4>



\label{copy}

\copyright Copyright 1991-2012.
Kein Teil dieses Dokuments darf ohne die Namen des Autors, \textbf{Chuan C. Chang}, und des Übersetzers, \textbf{Edgar Lins}, sowie dieses Copyrightvermerks heruntergeladen oder kopiert werden. Die Vervielfältigung in jeglicher Form ist nur für den persönlichen Gebrauch zulässig.

08.04.2012: Ich habe mich endgültig dafür entschieden, meine Übersetzung, wie in den vergangenen Jahren, auch in Zukunft ausschließlich in der hier vorliegenden Form, kostenlos und nur für den persönlichen Gebrauch zur Verfügung zu stellen. Falls also eine, wie auch immer beschaffene,  Version meines Texts oder eine auf meinem  Text basierende Weiterentwicklung auf irgendeine Art kommerziell vertrieben werden sollte, dann wurde jenes Produkt weder von mir autorisiert, noch  von mir selbst oder in meinem Auftrag hergestellt!


\hyperref[http://www.uteedgar-lins.de/index.html]{Zur Homepage von www.uteedgar-lins.de}<br>
\hyperref[./index.html]{Zur Homepage von FOPPDE}<br>


\textbf{\textit{[Die Datei befindet sich zurzeit im Umbau. Bis zum Abschluss kann es zu Inkonsistenzen, zum Beispiel verwaisten Links, kommen. Ich bitte, das zu entschuldigen.]}}





<!-- contents.html -->

\label{Inhalt}

<h2 align=\enquote{center}>Inhaltsverzeichnis</h2>

Das Buch ist in folgenden Sprachen verfügbar:
 \hyperref[http://www.pianopractice.org]{Englisch} (extern; das Original),
 \hyperref[http://bbs.popiano.org/viewthread.php?tid=81448&amp;extra=page\%3D3]{Einfaches Chinesisch} (extern),
 \hyperref[http://www.pianogarden.tw]{Traditionelles Chinesisch} (extern),
 \hyperref[./index.html]{Deutsch},
 \hyperref[http://pagesperso-orange.fr/musico/documents/textes/pianopratique/tabledesmatieres_fr.htm]{Französisch} (extern),
 \hyperref[http://web.tiscali.it/pianobook]{Italienisch} (extern),
 \hyperref[http://pianofundamental.sakura.ne.jp]{Japanisch} (extern),
 \hyperref[http://pianoart.eu.interia.pl]{Polnisch} (extern),
 \hyperref[http://www.pianopractice.org/spanish.pdf]{Spanisch} (extern)
 .

Ich suche Freiwillige, die das Buch in jede andere Sprache übersetzen - siehe \hyperref[HinUeber]{Hinweise für Übersetzer} am Ende dieses Buchs.
Senden Sie bitte eine E-Mail an \hyperref[mailto:cc88m@aol.com?subject=foppde:\%20Translation\%20request]{cc88m@aol.com}, um die Einzelheiten zu besprechen.

\footnote{Wenn diese Website neu für Sie ist, sollten Sie zunächst die ersten beiden Abschnitte der \hyperref[anmerkungen]{Anmerkungen} lesen.}


<h2><br>\hyperref[preface]{Vorwort} \textbf{\textit{[09.08.2009]}}</h2>

<h2><br>\hyperref[c1i1]{Kapitel 1: Klaviertechnik}</h2>

\section{Einführung \textbf{\textit{[15.08.2009]}}}

\begin{enumerate} 
 \item \hyperref[c1i1]{Zweck dieses Buchs}
 \item \hyperref[c1i2]{Was ist Klaviertechnik?}
 \item \hyperref[c1i3]{Technik, Musik und mentales Spielen}
 \item \hyperref[c1i4]{Generelles Vorgehen, Interpretation, Musikunterricht, Absolutes Gehör}
\end{enumerate}

\section{Grundlegende Verfahren des Klavierübens}

\begin{enumerate} 
 \item \hyperref[c1ii1]{Der Übungsablauf} \textbf{\textit{[22.08.2009]}}
 \item \hyperref[c1ii2]{Position der Finger}
 \item \hyperref[c1ii3]{Höhe der Sitzbank und ihr Abstand zum Klavier}
 \item \hyperref[c1ii4]{Ein neues Stück - Anhören und analysieren (\enquote{Für Elise})}
 \item \hyperref[c1ii5]{Die schwierigen Abschnitte zuerst üben}  \textbf{\textit{[05.09.2009]}}
 \item \hyperref[c1ii6]{Schwierige Passagen kürzen - In kleinen Portionen üben (taktweise)}
 \item \hyperref[c1ii7]{Die Hände getrennt (einhändig, HS) üben - Erlernen der Spieltechnik}
 \item \hyperref[c1ii8]{Die Kontinuitätsregel}
 \item \hyperref[c1ii9]{Der Akkord-Anschlag}
 \item \hyperref[c1ii10]{Freier Fall, Akkord-Übung und Entspannung}
 \item \hyperref[c1ii11]{Parallele Sets}
 \item \hyperref[c1ii12]{Lernen, Auswendiglernen und mentales Spielen} \textbf{\textit{[20.09.2009]}}
 \item \hyperref[c1ii13]{Spielgeschwindigkeit beim Üben}
 \item \hyperref[c1ii14]{Wie man entspannt}
 \item \hyperref[c1ii15]{Automatische Verbesserung nach dem Üben (PPI)}
 \item \hyperref[c1ii16]{Gefahren des langsamen Spielens - Fallstricke der \enquote{Intuitiven Methode}}
 \item \hyperref[c1ii17]{Die Wichtigkeit des langsamen Spielens} \textbf{\textit{[04.10.2009]}}

 \item \hyperref[c1ii18]{Fingersatz}
 \item \hyperref[c1ii19]{Akkurates Tempo und das Metronom}
 \item \hyperref[c1ii20]{Die schwache linke Hand - Eine Hand unterrichtet die andere}
 \item \hyperref[c1ii21]{Ausdauer aufbauen, Atmung}

 \item \hyperref[c1ii22]{Schlechte Angewohnheiten: Der größte Feind des Pianisten} \textbf{\textit{[31.10.2009]}}
 \item \hyperref[c1ii23]{Haltepedal}\footnote{Anmerkungen zu den Bezeichnungen der Pedale finden Sie \hyperref[Pedale]{hier}}
 
 \item \hyperref[c1ii24]{Dämpferpedal, Timbre und Eigenschwingungen vibrierender Saiten}
 \item \hyperref[c1ii25]{Mit beiden Händen zusammen (HT) üben und mental spielen} \textbf{\textit{[05.12.2009]}}
  \begin{enumerate}[label={\alph*.}] 
   <li>\hyperref[c1ii25a]{Einführung}
   \item \hyperref[c1ii25b]{Beethovens Mondschein-Sonate, 1. Satz, Op. 27, No. 2}
   \item \hyperref[c1ii25c]{Mozarts Rondo Alla Turca, aus Sonate K300 (KV331)}
   \item \hyperref[c1ii25d]{Chopins Fantaisie-Impromptu, Op. 66}
  \end{enumerate}
 </li>
 \item \hyperref[c1ii26]{Zusammenfassung}
 \end{enumerate}

\section{Ausgewählte Themen des Klavierübens}

\begin{enumerate} 
 \item \hyperref[c1iii1]{Klang, Rhythmus, Legato, Staccato} \textbf{\textit{[14.02.2010]}}
  \begin{enumerate}[label={\alph*.}] 
   <li>\hyperref[c1iii1a]{Was ist ein \enquote{Guter Klang}?}
   <ul type=\enquote{circle}>
    <li>\hyperref[c1iii1a1]{Der Basisanschlag}
    \item \hyperref[c1iii1a2]{Klang: Einzelne gegenüber mehreren Noten, Pianissimo, Fortissimo}
   </ul>
   </li>
   \item \hyperref[c1iii1b]{Was ist Rhythmus? (Beethovens Sturm-Sonate und Appassionata)}
   \item \hyperref[c1iii1c]{Legato, Staccato}
  \end{enumerate}
 </li>
 \item \hyperref[c1iii2]{Zyklisch spielen (Chopins Fantaisie Impromptu, Op. 66)} \textbf{\textit{[21.03.2010]}}

 \item \hyperref[c1iii3]{Triller und Tremolos (Beethovens Pathétique, 1. Satz)}
  <ol type=\enquote{a}>
   <li>\hyperref[c1iii3]{Triller}
   \item \hyperref[c1iii3b]{Tremolos (Beethovens Pathétique, 1. Satz)}
   \end{enumerate}
 </li>
 \item \hyperref[c1iii4]{Bewegungen der Hand, der Finger und des Körpers} \textbf{\textit{[24.10.2010]}}
  \begin{enumerate}[label={\alph*.}] 
   <li>\hyperref[c1iii4]{Bewegungen der Hand}
   \item \hyperref[c1iii4b]{Mit flachen (gestreckten) Fingern spielen}
   \item \hyperref[c1iii4c]{Bewegungen des Körpers}
   \end{enumerate}
 </li>
 \item \hyperref[c1iii5]{Schnell spielen: Tonleitern, Arpeggios und chromatische Tonleitern (Chopins Fantaisie Impromptu und Beethovens Mondschein-Sonate, 3. Satz)} \textbf{\textit{[21.08.2011]}}
  \begin{enumerate}[label={\alph*.}] 
   <li>\hyperref[c1iii5a]{Tonleitern: Daumenuntersatz, Daumenübersatz}
   \item \hyperref[c1iii5b]{Daumenübersatz: Bewegung, Erklärung und Video}
   \item \hyperref[c1iii5c]{Daumenübersatz üben: Geschwindigkeit, Glissandobewegung}
   \item \hyperref[c1iii5d]{Tonleitern: Herkunft, Namensgebung, Fingersätze}
    <ul type=\enquote{circle}>
     <li>\hyperref[table]{Fingersatztabelle}
    </ul>
   </li>
   \item \hyperref[c1iii5e]{Arpeggios (Chopin, Wagenradbewegung, \enquote{gespreizte} Finger)}
   \item \hyperref[c1iii5f]{Schub und Zug, Beethovens Mondschein-Sonate, 3. Satz}
   \item \hyperref[c1iii5g]{Der Daumen: Der vielseitigste Finger}
   \item \hyperref[c1iii5h]{Schnelle chromatische Tonleitern}
   \end{enumerate}<br><br>[Ab hier wird der Text noch überarbeitet.]<br><br>
 </li>
 \item \hyperref[c1iii6]{Auswendiglernen}
  \begin{enumerate}[label={\alph*.}] 
   <li>\hyperref[c1iii6a]{Warum auswendig lernen?}
   \item \hyperref[c1iii6b]{Wer kann auswendig lernen, was und wann?}
   \item \hyperref[c1iii6c]{Auswendiglernen und Pflege des Gelernten}
   \item \hyperref[c1iii6d]{Hand-Gedächtnis}
   \item \hyperref[c1iii6e]{Wie fängt man an?}
   \item \hyperref[c1iii6f]{Auffrischung des Gedächtnisses}
   \item \hyperref[c1iii6g]{Kaltstart}
   \item \hyperref[c1iii6h]{Langsam spielen}
   \item \hyperref[c1iii6i]{Vorausschauend spielen}
   \item \hyperref[c1iii6j]{Langzeitgedächtnis aufbauen}
    <ul type=\enquote{circle}>
     <li>\hyperref[c1iii6hand]{Hand-Gedächtnis}
     \item \hyperref[c1iii6musik]{Musik-Gedächtnis}
     \item \hyperref[c1iii6foto]{Fotografisches Gedächtnis}
     \item \hyperref[c1iii6tastatur]{Tastatur-Gedächtnis}
     \item \hyperref[c1iii6theorie]{Theoretisches Gedächtnis}
    </ul>
   </li>
   \item \hyperref[c1iii6k]{Pflege}
   \item \hyperref[c1iii6l]{Blattspieler und Auswendiglernende (Bachs Inventionen)}
    <ul type=\enquote{circle}>
     <li>\hyperref[c1iii6l]{Blattspieler und Auswendiglernende}
     \item \hyperref[c1iii6l2]{Bachs Inventionen}
     \item \hyperref[ruhig]{Ruhige Hände}
    </ul>
   </li>
   \item \hyperref[c1iii6m]{Funktion des menschlichen Gedächtnisses}
   \item \hyperref[c1iii6n]{Ein guter Auswendiglernender werden}
   \item \hyperref[c1iii6o]{Zusammenfassung}
   \end{enumerate}
 </li>
 \item \hyperref[c1iii7]{Übungen}
  \begin{enumerate}[label={\alph*.}] 
   <li>\hyperref[c1iii7a]{Einführung}
    <ul type=\enquote{circle}>
      <li>\hyperref[c1iii7aMuskeln]{Schnelle und langsame Muskeln}
    </ul>
   </li> 
   \item \hyperref[c1iii7b]{Parallele Sets}
   \item \hyperref[c1iii7c]{Wie verwendet man die Übungen für parallele Sets (Appassionata, 3. Satz)?}
   \item \hyperref[c1iii7d]{Tonleitern, Arpeggios, Unabhängigkeit der Finger und Anheben der Finger}
   \item \hyperref[c1iii7e]{(Große) Akkorde spielen, Dehnung der Handflächen}
   \item \hyperref[c1iii7f]{Sprünge}
   \item \hyperref[c1iii7g]{Weitere Übungen}
   \item \hyperref[c1iii7h]{Probleme mit Hanons Übungen}
   \item \hyperref[c1iii7i]{Die Geschwindigkeit steigern}
    <ul type=\enquote{circle}>
     <li>\hyperref[c1iii7iAnschlag]{Schneller Anschlag, Entspannung}
     \item \hyperref[c1iii7iAndere]{Andere Geschwindigkeitsmethoden}
     \item \hyperref[c1iii7iMusik]{Geschwindigkeit und Musik}
    </ul>
   </li>
   \end{enumerate}
 </li>
 \item \hyperref[c1iii8]{Konturieren (Beethovens Sonate \#1)}
 \item \hyperref[c1iii9]{Ein Stück auf Hochglanz bringen - Fehler beseitigen}
 \item \hyperref[c1iii10]{Kalte Hände, rutschende Finger, Krankheiten, Handverletzungen, Gehörschäden}
  \begin{enumerate}[label={\alph*.}] 
   <li>\hyperref[c1iii10]{Kalte Hände}
   \item \hyperref[c1iii10rutschen]{Rutschende (trockene oder schwitzende) Finger}
   \item \hyperref[c1iii10krank]{Krankheiten}
   \item \hyperref[c1iii10ungesund]{Gesundes und ungesundes Üben}
   \item \hyperref[c1iii10hand]{Verletzungen der Hand}
   \item \hyperref[c1iii10gehoer]{Gehörschäden}
   \end{enumerate}
 </li>
 \item \hyperref[c1iii11]{Blattspiel}
 \item \hyperref[c1iii12]{Absolutes Gehör und relatives Gehör (vom Blatt singen)}
   <ul type=\enquote{circle}>
    <li>\hyperref[c1iii12tonhoehe]{Verfahren zum Lernen der relativen und absoluten Tonhöhenerkennung}
    \item \hyperref[c1iii12blatt]{Vom Blatt singen und komponieren}
   </ul>
 </li>
 \item \hyperref[c1iii13]{Filmen und Aufnehmen des eigenen Spielens,\footnote{MIDI, Digitalpianos, Keyboards usw.}}
     <ul type=\enquote{circle}>
      <li>\hyperref[c1iii13MIDI]{\footnote{MIDI, Digitalpianos, Keyboards usw.}}
    </ul>
 </li>
 \item \hyperref[c1iii14]{Vorbereitung auf Auftritte und Konzerte}
  \begin{enumerate}[label={\alph*.}] 
   <li>\hyperref[c1iii14a]{Nutzen und Risiken von Auftritten und Konzerten}
   \item \hyperref[c1iii14b]{Grundlagen fehlerfreien Vorspielens}
   \item \hyperref[c1iii14c]{Für Auftritte üben}
   \item \hyperref[c1iii14d]{Musikalisch üben}
   \item \hyperref[c1iii14e]{Zwangloses Vorspielen}
   \item \hyperref[c1iii14f]{Vorbereitung auf Konzerte}
   \item \hyperref[c1iii14g]{Während des Konzerts}
   \item \hyperref[c1iii14h]{Das ungewohnte Klavier}
   \item \hyperref[c1iii14i]{Nach dem Konzert}
   \end{enumerate}
 </li>
 \item \hyperref[c1iii15]{Ursachen und Kontrolle von Nervosität}
 \item \hyperref[c1iii16]{Unterrichten}
  \begin{enumerate}[label={\alph*.}] 
   <li>\hyperref[c1iii16a]{Lehrer}
   \item \hyperref[c1iii16b]{Kinder unterrichten, Eltern einbeziehen}
   \item \hyperref[c1iii16c]{Blattspiel, Auswendiglernen, Theorie}
   \item \hyperref[c1iii16d]{Einige Elemente des Klavierunterrichts}
   \item \hyperref[c1iii16e]{Warum die größten Pianisten nicht unterrichten konnten}
   \end{enumerate}
 </li>
 \item \hyperref[c1iii17]{Klaviere, Flügel und elektronische Klaviere; Kauf und Wartung}
  \begin{enumerate}[label={\alph*.}] 
   <li>\hyperref[c1iii17a]{Flügel, akustisches oder elektronisches Klavier?}
   \item \hyperref[c1iii17b]{Elektronische Klaviere}
   \item \hyperref[c1iii17c]{Klaviere}
   \item \hyperref[c1iii17d]{Flügel}
   \item \hyperref[c1iii17e]{Ein akustisches Klavier kaufen}
   \item \hyperref[c1iii17f]{Pflege und Wartung des Klaviers}
   \item \hyperref[c1iii17g]{\footnote{Anmerkungen zu Digitalpianos}}
   \end{enumerate}
 </li>
 \item \hyperref[c1iii18]{Wie man das Klavierspielenlernen beginnt - vom jüngsten Kind bis zum ältesten Erwachsenen}
     \begin{enumerate}[label={\alph*.}] 
        <li>\hyperref[c1iii18a]{Benötigt man einen Lehrer?}
        \item \hyperref[c1iii18b]{Bücher für Anfänger; Keyboards}
        \item \hyperref[c1iii18c]{Anfänger im Alter von 0 bis über 65}
            <ul type=\enquote{circle}>
               <li>\hyperref[c1iii18c0]{Von 0 bis 6}
               \item \hyperref[c1iii18c3]{Von 3 bis 12}
               \item \hyperref[c1iii18c13]{Von 13 bis 19}
               \item \hyperref[c1iii18c20]{Von 20 bis 35}
               \item \hyperref[c1iii18c35]{Von 35 bis 45}
               \item \hyperref[c1iii18c45]{Von 45 bis 65}
               \item \hyperref[c1iii18c65]{Über 65}
            </ul>
        </li>
      \end{enumerate}
 </li>
 \item \hyperref[c1iii19]{Der \enquote{ideale} Übungsablauf (Bachs Invention \#4)}
     \begin{enumerate}[label={\alph*.}] 
        <li>\hyperref[c1iii19a]{Die Regeln lernen}
        \item \hyperref[c1iii19b]{Ein neues Stück lernen (Invention \#4)}
        \item \hyperref[c1iii19c]{\enquote{Normale} Übungsabläufe und Bachs Lehren}
      \end{enumerate}
 </li>
 \item \hyperref[c1iii20]{Bach: der größte Komponist und Lehrer (15 Inventionen)}
   <ul type=\enquote{circle}>
    <li>\hyperref[c1iii20ps]{Liste der parallelen Sets in den einzelnen Inventionen (für die RH)}
   </ul>
 </li>
 \item \hyperref[c1iii21]{Klavierspielen und die Psychologie}
 \item \hyperref[c1iii22]{Zusammenfassung der Methoden}

</ol>

\section{Mathematische Theorien des Klavierspielens}

\begin{enumerate} 
 \item \hyperref[c1iv1]{Wozu braucht man mathematische Theorien?}
 \item \hyperref[c1iv2]{Die Theorie der Fingerbewegung}
   \begin{enumerate}[label={\alph*.}] 
    <li>\hyperref[c1iv2a]{Serielles und paralleles Spielen}
    \item \hyperref[c1iv2b]{Geschwindigkeitsbarrieren}
    \item \hyperref[c1iv2c]{Die Geschwindigkeit steigern}
   \end{enumerate}
 </li>
 \item \hyperref[c1iv3]{Die Thermodynamik des Klavierspielens}
 \item \hyperref[c1iv4]{Mozarts Formel, Beethoven und Gruppentheorie}
 \item \hyperref[c1iv5]{Berechnen der Lernrate}
 \item \hyperref[c1iv6]{Noch zu erforschende Themen}
   <ol type=\enquote{a}>
    <li>\hyperref[c1iv6a]{Impulstheorie des Klavierspielens}
    \item \hyperref[c1iv6b]{Die Physiologie der Technik}
    \item \hyperref[c1iv6c]{Gerhirnforschung (HS- und HT-Spielen usw.)}
    \item \hyperref[c1iv6d]{Was verursacht Nervosität?}
    \item \hyperref[c1iv6e]{Ursachen von und Mittel gegen Tinnitus}
    \item \hyperref[c1iv6f]{Was ist Musik?}
    \item \hyperref[c1iv6g]{In welchem Alter soll bzw. darf man mit dem Klavierspielen anfangen?}
    \item \hyperref[c1iv6h]{Die Zukunft des Klavierspielens}
    \item \hyperref[c1iv6i]{Die Zukunft des Unterrichts}
    \end{enumerate}
 </li>
</ol>

<h2><br>Kapitel 2: Stimmen des Klaviers</h2> 

\begin{enumerate} 
 \item \hyperref[c2_1]{Einleitung}
 \item \hyperref[c2_2]{Chromatische Tonleiter und Temperaturen}
  \begin{enumerate}[label={\alph*.}] 
   <li>\hyperref[c2_2a]{Einleitung}
   \item \hyperref[c2_2b]{Mathematische Behandlung}
   \item \hyperref[c2_2c]{Temperatur und Musik}
  \end{enumerate}
 </li>
 \item \hyperref[c2_3]{Werkzeuge zum Stimmen}
 \item \hyperref[c2_4]{Vorbereitung}
 \item \hyperref[c2_5]{Wie man anfängt}
  <ol type=\enquote{a}>
   <li>\hyperref[c2_5a]{Einleitung}
   \item \hyperref[c2_5_hamm]{Einsetzen und Bewegen des Stimmhammers}
   \item \hyperref[c2_5_wirb]{Den Wirbel einstellen}
   \item \hyperref[c2_5_unis]{Unisono stimmen}
   \item \hyperref[c2_5_mits]{Mitschwingung}
   \item \hyperref[c2_5_infi]{Diese letzte infinitesimale Bewegung ausführen}
   \item \hyperref[c2_5_span]{Ausgleich der Saitenspannung}
   \item \hyperref[c2_5_disk]{Wiegen im Diskant}
   \item \hyperref[c2_5_bass]{Grollen im Bass}
   \item \hyperref[c2_5_harm]{Harmonisches Stimmen}
   \item \hyperref[c2_5_stre]{Was ist Streckung?}
   \item \hyperref[c2_5_prae]{Präzision, Präzision, Präzision}
   \end{enumerate}
 </li>
 \item \hyperref[c2_6]{Stimmverfahren}
  \begin{enumerate}[label={\alph*.}] 
   <li>\hyperref[c2_6a]{Einleitung}
   \item \hyperref[c2_6_gabe]{Das Klavier nach der Stimmgabel stimmen}
   \item \hyperref[c2_6_kirn]{Kirnberger II}
   \item \hyperref[c2_6_et]{Gleichschwebende Temperatur (gleichstufige Temperatur, gleichmäßige Temperatur)}
   \end{enumerate}
 </li>
 \item \hyperref[c2_7]{Kleinere Reparaturen durchführen}
  \begin{enumerate}[label={\alph*.}] 
   <li>\hyperref[c2_7_hamm]{Intonieren der Hämmer}
   \item \hyperref[c2_7_pilo]{Polieren der Piloten}
   \end{enumerate}
 </li>
</ol> 

<h2><br>\hyperref[c3_1]{Kapitel 3: Wissenschaftliche Methode, Theorie des Lernens, Das Gehirn}</h2>

\footnote{Abschnitt 4 ist im Original zurzeit (26.5.2003) noch \enquote{preliminary draft} also ein \enquote{Rohentwurf}.}


\begin{enumerate} 
 \item \hyperref[c3_1]{Einleitung}
 \item \hyperref[c3_2]{Der wissenschaftliche Ansatz}
  \begin{enumerate}[label={\alph*.}] 
   <li>\hyperref[c3_2a]{Einleitung}
   \item \hyperref[c3_2b]{Lernen}
  \end{enumerate}
 </li>
 \item \hyperref[c3_3]{Was ist die Wissenschaftliche Methode?}
  <ol type=\enquote{a}>
   <li>\hyperref[c3_3a]{Einleitung}
   \item \hyperref[c3_3b]{Definition}
   \item \hyperref[c3_3c]{Forschung}
   \item \hyperref[c3_3d]{Dokumentation und Kommunikation}
   \item \hyperref[c3_3e]{Konsistenzprüfungen}
   \item \hyperref[c3_3f]{Grundlegende Theorie}
   \item \hyperref[c3_3g]{Dogma und Lehre}
   \item \hyperref[c3_3h]{Schlussfolgerungen}
   \end{enumerate}
 </li>
 \item \hyperref[c3_4]{Theorie des Lernens}
 \item \hyperref[c3_5]{Was Träume erzeugt und Methoden zu ihrer Kontrolle}
  \begin{enumerate}[label={\alph*.}] 
   <li>\hyperref[c3_5a]{Einleitung}
   \item \hyperref[c3_5b]{Der Fall-Traum}
   \item \hyperref[c3_5c]{Der Unfähig-zu-laufen-Traum}
   \item \hyperref[c3_5d]{Der Zu-spät-zur-Prüfung-kommen- oder Sich-verlaufen-Traum}
   \item \hyperref[c3_5e]{Die Lösung für meinen langen und komplexen Traum}
   \item \hyperref[c3_5f]{Die Kontrolle der Träume}
   \item \hyperref[c3_5g]{Was uns diese Träume über das Gehirn lehren}
   \end{enumerate}
</li>
 \item \hyperref[c3_6]{Das Unterbewusstsein}
  \begin{enumerate}[label={\alph*.}] 
   <li>\hyperref[c3_6a]{Einleitung}
   \item \hyperref[c3_6b]{Emotionen}
   \item \hyperref[c3_6c]{Das Unterbewusstsein benutzen}
   \end{enumerate}
</li>
</ol>

<h2><br>\hyperref[reference]{Quellenverzeichnis}</h2>

<h3>\underline{\hyperref[reference]{Buchbesprechungen}}</h3>
\begin{itemize} 
 \item \hyperref[allgemein]{Allgemeine Schlussfolgerungen}
 \item \hyperref[Bree]{Bree, Malwine: \textit{The Leschetizky Method}}
 \item \hyperref[Bruser]{Bruser, Madeline: \textit{The Art of Practicing}}
 \item \hyperref[Chang]{Chang, Chuan C.: \textit{Fundamentals of Piano Practice}, erste Ausgabe}
 \item \hyperref[Eigeldinger]{Eigeldinger, Jean-Jacques: \textit{Chopin, pianist and teacher as seen by his pupils}}
 \item \hyperref[Fink]{Fink, Seymour: \textit{Mastering Piano Technique}}
 \item \hyperref[Gieseking]{Gieseking, Walter und Leimer, Karl: \textit{Modernes Klavierspiel}}
 \item \hyperref[Green]{Green, Barry, und Gallwey, Timothy: \textit{The Inner Game of Music}}
 \item \hyperref[Hofman]{Hofman, Josef: \textit{Piano Playing, With Piano Questions Answered}}
 \item \hyperref[Lhevine]{Lhevine, Josef: \textit{Basic Principles in Piano Playing}}
 \item \hyperref[Prokop]{Prokop, Richard: \textit{Piano Power, a Breakthrough Approach to Improving your Technique}}
 \item \hyperref[Richman]{Richman, Howard: \textit{Super Sight-Reading Secrets}}
 \item \hyperref[Sandor]{Sandor, Gyorgy: \textit{On Piano Playing}}
 \item \hyperref[Sherman]{Sherman, Russell: \textit{Piano Pieces}}
 \item \hyperref[Suzuki]{Suzuki, Shinichi (et al): \textit{The Suzuki Concept: An Introduction to a Successful Method for Early Music Education} und<br>\textit{HOW TO TEACH SUZUKI PIANO}}
 \item \hyperref[Walker]{Walker, Alan: \textit{Franz Liszt, The Virtuoso Years, 1811-1847}}
 \item \hyperref[Werner]{Werner, Kenney: \textit{Effortless Mastery}}
 \item \hyperref[Whiteside]{Whiteside, Abby: \textit{On Piano Playing}}
 \item \hyperref[American]{Weinreich, G.:\textit{The Coupled Motions of Piano Strings}}
 \item \hyperref[Lectures]{Verschiedene: \textit{Five Lectures on the Acoustics of the Piano}}
 \end{itemize}


<h3>\underline{\hyperref[Websites]{Websites, Bücher, Videos}}</h3>

\footnote{Im \hyperref[http://www.pianopractice.org]{Original} (extern) folgt hier unter anderem eine Zusammenfassung der Links.
Da diese Liste in der übersetzten Seite wegen der unklaren deutschen Rechtslage nicht wiedergegeben wird, ist hier die Zusammenfassung nicht aufgeführt.}


<h2><br>\hyperref[anmerkungen]{Anmerkungen}</h2>

<h2><br>\hyperref[testimonials]{Leserkommentare}</h2>
Probleme, Sorgen und Erfolge von Klavierspielern; hilfreiche Kommentare von Lehrern und Lesern.


<h2><br>Anhang</h2>

\begin{itemize} 
 \item \hyperref[ueberset]{Anmerkungen zur Übersetzung}
 \item \hyperref[AbkFarben]{Im Text verwendete Abkürzungen und Farben}
 \item \hyperref[Danke]{Danke!}
 \end{itemize}




<!-- preface.html -->

\section*{Vorwort}
\label{preface}

% zuletzt geändert 09.08.2009

\textbf{Dieses ist das beste Buch, das jemals darüber geschrieben wurde, wie man am Klavier übt!}
Das Buch offenbart, dass es hocheffiziente Übungsmethoden gibt, die Ihre Lernrate beschleunigen können - bis zu einem Faktor von 1000, wenn Sie die effizientesten Übungsmethoden bisher nicht gelernt haben (siehe \hyperref[c1iv5]{Berechnen der Lernrate}).
Das Überraschende ist, dass diese Methoden, obwohl sie seit der frühesten Zeit des Klaviers bekannt sind, selten gelehrt wurden, weil nur wenige Lehrer sie kannten und diese sachkundigen Lehrer sich nie die Mühe gemacht haben, ihr Wissen zu verbreiten.

\textbf{Ich erkannte in den 1960ern, dass es kein gutes Buch darüber gab, wie man am Klavier übt.}
Das Beste, das ich finden konnte, war Whitesides Buch, was aber eine völlige Enttäuschung war (siehe die Besprechung des Buchs im \hyperref[Whiteside]{Quellenverzeichnis}).
Als Student der Cornell University, der bis 2:00 Uhr morgens lernte, damit er mit einigen der klügsten Studenten aus aller Welt mithalten konnte, hatte ich wenig Zeit, Klavier zu üben.
Ich musste wissen, was die besten Übungsmethoden sind, besonders weil alles, was ich benutzte, nicht funktionierte, obwohl ich während meiner Jugend 7 Jahre lang eifrig Klavierstunden genommen hatte.
Wie Konzertpianisten so spielen können, wie sie es tun, war ein absolutes Mysterium für mich.
War es einfach eine Frage der genügenden Anstrengung, der Zeit und des Talents, wie viele Menschen anscheinend meinen?
Wenn die Antwort \enquote{Ja} wäre, dann wäre es niederschmetternd für mich gewesen, weil es bedeutet hätte, dass meine musikalische Talentstufe so niedrig war, dass ich ein hoffnungsloser Fall war, weil ich, zumindest während meiner Jugend, genügend Anstrengung und Zeit hineingesteckt hatte, indem ich an Wochenenden bis zu 8 Stunden täglich geübt hatte.

Die Antworten kamen mir schrittweise in den 1970ern, als ich bemerkte, dass die Klavierlehrerin unserer beiden Töchter einige erstaunlich effiziente Übungsmethoden lehrte, die sich von den von der Mehrheit der Klavierlehrer gelehrten Methoden unterschieden.
\textbf{Ich habe diese effizienten Übungsmethoden über einen Zeitraum von mehr als 10 Jahren verfolgt und kam zu der Erkenntnis, dass der wichtigste Faktor für das Lernen des Klavierspielens die \textit{Übungsmethoden} sind.}
Anstrengung, Zeit und Talent waren bloße zweitrangige Faktoren!
In Wahrheit ist \enquote{Talent} schwierig zu definieren und unmöglich zu messen; es ist ein nebulöses Wort, das wir häufig benutzen, das aber keine definierbare Bedeutung hat.
Tatsächlich \textbf{können die richtigen Übungsmethoden praktisch jeden in einen \enquote{talentierten} Musiker verwandeln!}
Ich habe das jedes Mal bei hunderten von Schülerkonzerten und Klavierwettbewerben gesehen, die ich besucht habe.

\textbf{Es wird nun zunehmend erkannt, dass \enquote{Talent}, \enquote{Wunderkinder} und \enquote{Genialität} eher erzeugt werden als angeboren sind} (siehe Olson) - Mozart ist vielleicht das prominenteste Beispiel des \enquote{Mozart-Effekts}.
Einige haben diesen in \enquote{Beethoven-Effekt} umbenannt, was angebrachter sein mag, weil Mozart ein paar Schwächen in der Persönlichkeit usw. hatte, die manchmal seine ansonsten herrliche Musik beeinträchtigten, während psychologisch gesehen, Beethoven die am meisten erleuchtende Musik komponierte.
Sich Musik anzuhören ist nur eine Komponente des komplexen Mozart-Effekts.
Für Klavierspieler hat \textit{Musik zu machen} eine größere Auswirkung auf die geistige Entwicklung.
Deshalb werden gute Übungsmethoden nicht nur die Lernrate beschleunigen, sondern auch dabei helfen, das musikalische Gehirn zu entwickeln sowie das Ausmaß der Intelligenz zu erhöhen, besonders bei jungen Menschen.
Die Lernrate wird, verglichen mit der bei den langsameren Methoden, beschleunigt (es ist wie der Unterschied zwischen einem Fahrzeug, das beschleunigt, und einem, das mit konstanter Geschwindigkeit fährt).
Darum werden Schüler ohne die richtigen Übungsmethoden innerhalb weniger Jahre hoffnungslos zurückbleiben.
Das lässt die Schüler mit guten Übungsmethoden weitaus talentierter erscheinen als sie wirklich sind, weil sie in Minuten oder Tagen das lernen können, wofür die anderen Monate oder Jahre benötigen.
\textbf{Der wichtigste Aspekt des Klavierspielenlernens ist die Entwicklung des Gehirns und eine größere Intelligenz.
Das Gedächtnis ist eine Komponente der Intelligenz, und wir wissen, wie man das Gedächtnis verbessern kann (siehe \hyperref[c1iii6]{Abschnitt III.6}).
Dieses Buch lehrt uns auch, wie man die Musik in Gedanken spielt - das wird \enquote{\hyperref[c1ii12mental]{mentales Spielen}} genannt, das natürlich zum \hyperref[c1iii12]{absoluten Gehör} und zur Fähigkeit führt, Musik zu \hyperref[c1iii12blatt]{komponieren}.}
Das sind die Fertigkeiten, durch die sich die größten Musiker von anderen unterschieden, und weswegen wir sie als Genies bezeichneten; wir zeigen hier aber, dass diese Fertigkeiten nicht schwierig zu erlernen sind.
Bis jetzt war die Welt der Musiker für die wenigen \enquote{begabten} Künstler reserviert; wir wissen nun, dass sie ein Universum ist, in dem wir uns alle bewegen können.

\textbf{Die Übungsmethoden können bei jungen Schülern, die mit Leib und Seele dabei sind, innerhalb von weniger als 10 Jahren den Unterschied zwischen einer lebenslangen Zwecklosigkeit und einem Konzertpianisten ausmachen.}
Wenn man die richtigen Übungsmethoden benutzt, dauert es für einen fleißigen Schüler eines beliebigen Alters nur ein paar Jahre, bis er bedeutende Werke berühmter Komponisten spielen kann.
Die traurige Wahrheit der letzten beiden Jahrhunderte ist, dass, obwohl die meisten dieser Übungsmethoden entdeckt und tausende Male wieder entdeckt wurden, diese niemals dokumentiert wurden und die Schüler entweder gezwungen waren, sie selbst wieder zu entdecken oder, wenn sie Glück hatten, von Lehrern zu lernen, die einige dieser Methoden kannten.
Das beste Beispiel für diesen Mangel an Dokumentation sind die \enquote{Lehren} von Franz Liszt.
Es gibt ein Dutzend Franz-Liszt-Gesellschaften, und sie haben hunderte von Publikationen herausgegeben.
\textbf{Zahlreiche Bücher wurden über Liszt geschrieben (siehe \hyperref[Eigeldinger]{Eigeldinger} usw. im Quellenverzeichnis), und tausende von Lehrern haben - unter Angabe der Abstammungslinie - behauptet, die \enquote{Franz-Liszt-Methode} zu lehren.
Und doch gibt es keine einzige Publikation, die beschreibt, was diese Methode ist!}
Es gibt endlose Berichte von Liszts Fähigkeiten und technischem Können, jedoch gibt es keine einzige Quelle über die Einzelheiten, wie er dazu kam.
Zeugnisse in der Literatur zeigen, dass Liszt nicht beschreiben konnte, wie er die Technik erworben hatte; er konnte nur demonstrieren, wie er spielte.
Da es der Klavierpädagogik nicht gelungen ist, zu verfolgen, wie der größte Pianist seine Technik erlangte, wundert es wenig, dass wir kein Lehrbuch für das Klavierspielenlernen hatten.
Können Sie sich vorstellen, Mathematik, Wirtschaftswissenschaften, Physik, Geschichte, Biologie oder irgendetwas anderes ohne ein Lehrbuch zu lernen, und das (wenn Sie Glück haben) nur mit dem Gedächtnis Ihres Lehrers als Führung?
Ohne Lehrbücher und Dokumentation wäre unsere Zivilisation nicht über die der Dschungelstämme hinausgekommen, deren Wissensbasis durch wörtliche Überlieferung weitergegeben wurde.
Das ist im Grunde die Stufe, auf der sich die Klavierpädagogik während 200 Jahren befunden hat!

Es gibt viele Bücher über das Klavierspielenlernen (siehe \hyperref[reference]{Quellenverzeichnis}), jedoch kann keines davon als Lehrbuch für Übungsmethoden gelten, was man als Schüler aber benötigt.
Diese Bücher sagen Ihnen, welche Fertigkeiten Sie benötigen (\hyperref[c1iii5]{Tonleitern, Arpeggios}, \hyperref[c1iii3]{Triller} usw.), und die weiter fortgeschrittenen Bücher beschreiben die Fingersätze, Handpositionen, Bewegungen usw., mit denen man sie spielt, aber keines davon liefert einen hinreichend vollständigen, systematischen Satz Anweisungen darüber, wie man übt.
Die meisten Musikbücher für Anfänger bieten ein paar solcher Anweisungen, aber einige dieser Anweisungen sind falsch - ein gutes Beispiel ist die amateurhafte Anpreisung in der Einführung zur Hanon-Serie, wie man \enquote{mit 60 Übungen ein Virtuose wird} (siehe \hyperref[c1iii7h]{Abschnitt III.7h in Kapitel 1)}.
\textbf{Bevor dieses Buch hier geschrieben wurde, fehlte in der Klavierpädagogik das wichtigste Werkzeug für den Schüler: ein grundlegender Satz an Anweisungen dafür, wie man übt.}

Ich erkannte nicht, wie revolutionär die Methoden dieses Buch waren, bevor ich nicht 1994 die erste Ausgabe meines Buchs beendet hatte.
Die Methoden waren besser als jene, die ich zuvor benutzt hatte, und jahrelang hatte ich sie mit guten, aber nicht bemerkenswerten Resultaten angewandt.
Das Erwachen kam erst nachdem ich dieses Buch beendet hatte, als ich mein eigenes Buch wirklich las, die Methoden systematisch befolgte und ihre unglaubliche Effizienz erlebte.
Was war nun der Unterschied zwischen dem Kennen der Bestandteile der Methode und dem Lesen des Buchs?
Beim Schreiben des Buchs musste ich die verschiedenen Teile nehmen und sie in eine organisierte Struktur bringen, die einem bestimmten Zweck diente und bei der keine wesentlichen Komponenten fehlten.
Ich wusste, dass das Material in einer logischen Struktur anzuordnen der einzige Weg war, ein nützliches Handbuch zu schreiben.
Es ist in der Wissenschaft wohlbekannt, dass die meisten Entdeckungen während des Schreibens der Forschungsberichte gemacht werden und nicht während der Durchführung der Forschung.
Es war so, als wenn ich die meisten Teile eines sagenhaften Autos gehabt hätte, aber ohne einen Mechaniker, der das Auto zusammenbaut, die fehlenden Teile findet und das Auto richtig einstellt, eigneten sich diese Teile nicht gut für den Transport.
Ich wurde von dem Potential dieses Buchs überzeugt, den Klavierunterricht zu revolutionieren, und entschied mich 1999, es der Welt kostenlos im Internet zur Verfügung zu stellen;
so kann es aktualisiert werden, wenn meine Untersuchungen voranschreiten, und das Geschriebene ist sofort der Öffentlichkeit zugänglich.
Zurückblickend ist dieses Buch die Zusammenfassung von mehr als 50 Jahren Forschung, die ich seit meinen ersten Klavierstunden über das Klavierüben durchgeführt habe.

Warum sind diese Übungsmethoden so revolutionär?
Für detaillierte Antworten werden Sie das Buch lesen müssen.
Ich werde hier eine kurze Übersicht geben, wie diese wunderbaren Resultate erreicht werden, und erklären, warum die Methoden funktionieren.
\textbf{Die meisten grundlegenden Ideen in diesem Buch stammen nicht von mir.}
Sie wurden während der letzten 200 Jahre unzählige Male von allen erfolgreichen Pianisten erfunden oder wieder erfunden; diese hätten sonst nicht einen solchen Erfolg gehabt.
\textbf{Das Grundgerüst für dieses Buch wurde unter Verwendung der Lehren von Yvonne Combe erstellt}, der Lehrerin unserer beiden Töchter, die vollendete Klavierspielerinnen wurden (sie haben viele erste Preise bei Klavierwettbewerben gewonnen, und jede hat viele Jahre lang im Durchschnitt an mehr als 10 Konzerten teilgenommen; beide haben ein \hyperref[c1iii12]{absolutes Gehör} und \hyperref[c1iii12blatt]{komponieren} gerne).
Andere Teile dieses Buchs wurden aus der Literatur und aus den Ergebnissen meiner Nachforschungen im Internet zusammengestellt.
\textbf{Mein Beitrag ist das Zusammentragen dieser Ideen, sie in eine Struktur zu bringen und etwas zum Verständnis beizutragen, warum sie funktionieren.
Dieses Verständnis ist für den Erfolg der Methode entscheidend.}
Klavierspielen wurde oft wie Religion gelehrt: Glaube, Hoffnung, Liebe (Wohltätigkeit).
Glaube daran, dass wenn man von einem \enquote{Meister}-Lehrer vorgeschlagene Verfahren befolgt, diese auch funktionieren.
Hoffnung, dass \enquote{üben, üben, üben} ins Paradies führt.
Wohltätigkeit, sodass die gebrachten Opfer und die geleisteten Beiträge Wunder wirken.
Dieses Buch ist anders - \textbf{eine Methode ist nicht akzeptabel, solange die Schüler nicht verstehen, warum sie funktioniert, und sie deshalb nicht an ihre besonderen Bedürfnisse anpassen können}.
Das richtige Verständnis zu finden ist nicht einfach, weil man nicht bloß eine Erklärung aus der Luft greifen kann (sie wird falsch sein) - man braucht genügend Fachkenntnis auf diesem Gebiet, um zu der richtigen Erklärung zu kommen.
Indem man eine korrekte Erklärung bietet, filtert man automatisch die falschen Methoden heraus.
Das mag erklären, warum sogar erfahrene Klavierlehrer, deren Ausbildung stark auf die Musik ausgerichtet war, Schwierigkeiten damit haben können, das richtige Verständnis zu vermitteln, und oftmals sogar die falschen Erklärungen für die richtigen Methoden liefern werden.
In dieser Hinsicht waren mein Beruf und meine Ausbildung für die Lösung technischer Probleme, in Materialwissenschaften (Metalle, Halbleiter, Isolatoren), Optik, Akustik, Physik, Elektronik, Chemie, meine wissenschaftlichen Veröffentlichungen (ich habe über 100 geprüfte Artikel in den großen Wissenschaftsmagazinen veröffentlicht, und es wurden mir sechs Patente erteilt) usw. für das Schreiben dieses Buchs von unschätzbarem Wert.
Diese verschiedenen Erfordernisse könnten erklären, warum sonst niemand diese Art von Buch geschrieben hat.
Als Wissenschaftler habe ich mir den Kopf darüber zerbrochen, wie man \enquote{Wissenschaft} präzise definiert, und endlos mit Wissenschaftlern und Nichtwissenschaftlern über diese Definition debattiert.
Da der wissenschaftliche Ansatz für dieses Buch so grundlegend ist, habe ich einen Abschnitt (Kapitel 1, IV.2) darüber hinzugefügt.
Die Wissenschaft ist nicht bloß die theoretische Welt der intelligentesten Genies; sie ist der effizienteste Weg, unser Leben zu vereinfachen.
Wir brauchen Genies, um die Wissenschaft voranzubringen; wenn sie jedoch entwickelt wurden, sind es die Massen, die von den Fortschritten profitieren.

Was sind einige dieser zauberhaften Ideen, von denen erwartet wird, dass sie den Klavierunterricht revolutionieren?
Lassen Sie uns mit der Tatsache anfangen, dass wenn man berühmten Pianisten beim Auftritt zusieht, sie zwar unglaublich schwierige Stücke spielen, diese aber so aussehen lassen, als wenn sie einfach wären.
Wie machen sie das?
Tatsache ist, dass sie für sie leicht \textit{sind}!
Deshalb sind viele der hier besprochenen Lerntricks Methoden dafür, Schweres leicht zu machen - nicht nur leicht, sondern oft trivial einfach.
Das wird dadurch erreicht, dass man \hyperref[c1ii7]{mit beiden Händen getrennt übt} und \hyperref[c1ii6]{kleine Abschnitte zum Üben herausgreift}, manchmal bis zu einer oder zwei Noten herab.
Man kann die Dinge nicht einfacher machen als das!
Vollendete Pianisten können auch unglaublich schnell spielen - wie üben \textit{wir}, um in der Lage zu sein, schnell zu spielen?
Einfach!
Indem wir den \enquote{\hyperref[c1ii9]{Akkord-Anschlag}} (Abschnitt II.9) benutzen.
\textbf{Ein Schlüssel zum Erfolg der hier besprochenen Methoden ist deshalb die Anwendung einfallsreicher \textit{Lerntricks}, die zur Lösung bestimmter Probleme notwendig sind.}

Auch mit den hier beschriebenen Methoden müssen Sie eventuell schwierige Passagen hunderte Male und manchmal bis zu 10.000-mal üben, bevor Sie die schwierigsten Passagen mit Leichtigkeit spielen können.
Wenn Sie nun eine Beethoven-Sonate, sagen wir mit halber Geschwindigkeit (Sie lernen sie gerade), üben müssten, würde es ungefähr eine Stunde dauern, sie durchzuspielen.
Deshalb würde es 30 Jahre oder fast ein halbes Leben dauern, sie 10.000-mal zu wiederholen, wenn Sie eine Stunde täglich zum Üben hätten und 7 Tage die Woche nur diese Sonate üben würden.
Klar ist das nicht die richtige Art, die Sonate zu lernen, obwohl viele Schüler Übungsmethoden benutzen, die sich nicht sehr davon unterscheiden.
Dieses Buch beschreibt Methoden dafür, nur die wenigen Noten zu identifizieren, die man üben muss, und diese dann im Bruchteil einer Sekunde zu spielen, sodass man sie innerhalb weniger Wochen (oder bei leichterem Material sogar Tage) 10.000-mal wiederholen kann, und das bei einer Übungszeit von nur ungefähr 10 Minuten täglich an 5 Tagen die Woche - wir haben die Übungszeit von einem halben Leben auf ein paar Wochen reduziert.

Dieses Buch bespricht viele weitere Effizienz-Prinzipien, zum Beispiel gleichzeitig zu üben und \hyperref[c1iii6a]{auswendig zu lernen}.
\textbf{Während des Übens muss man jede Passage viele Male wiederholen, und Wiederholung ist die beste Art auswendig zu lernen; deshalb macht es keinen Sinn, während des Übens nicht auswendig zu lernen, insbesondere da dies der schnellste Weg zum Lernen ist.}
Haben Sie sich je gefragt, wie jeder Konzertpianist ein Repertoire von mehreren Stunden auswendig lernen kann?
Die Antwort ist ziemlich einfach.
\textbf{Studien mit Gedächtniskünstlern (wie denjenigen, die ganze Telefonbuchseiten auswendig lernen können) haben gezeigt, dass sie auswendig lernen können, weil sie Gedächtnisalgorithmen entwickelt haben, auf die sie das auswendig zu lernende Material schnell abbilden können.
Für Klavierspieler ist die Musik ein solcher Algorithmus.}
Sie können das beweisen, indem Sie einen Klavierspieler bitten, nur eine Seite zufälliger Noten auswendig zu lernen und sich jahrelang daran zu erinnern.
Das ist (ohne einen Algorithmus) unmöglich, obwohl dieser Klavierspieler vielleicht keine Schwierigkeit damit hat, sich mehrere 20 Seiten lange Beethoven-Sonaten zu merken und sie 10 Jahre später immer noch spielen zu können.
So stellt sich das, was wir für ein besonderes Talent der Konzertpianisten hielten, als etwas heraus, das jeder kann.
Schüler, die die Methoden dieses Buches benutzen, lernen - außer wenn sie das \hyperref[c1iii11]{Spielen vom Blatt} üben - alles auswendig, was sie lernen.
Darum empfiehlt dieses Buch keine Übungen wie \hyperref[c1iii7h]{Hanon} und Czerny, die nicht dazu gedacht sind, aufgeführt zu werden; aus demselben Grund sind die Chopin-Etüden allerdings empfehlenswert.
\textbf{Etwas zu üben, das nicht zur Aufführung gedacht ist, ist nicht nur eine Zeitverschwendung, sondern zerstört auch jeden Sinn für die Musik, den man ursprünglich hatte.}
Wir besprechen alle wichtigen Methoden des Auswendiglernens, die den Klavierspieler dazu befähigen, Kunststücke vorzuführen, die die meisten Menschen nur von \enquote{begnadeten Musikern} erwarten, wie 
\hyperref[c1ii12mental]{die Komposition im Kopf zu spielen}, ohne Klavier, oder sogar die ganze Komposition aus dem Gedächtnis niederzuschreiben.
Wenn Sie jede Note der Komposition aus dem Gedächtnis spielen können, gibt es keinen Grund, warum Sie sie nicht alle aufschreiben können!
Solche Fähigkeiten dienen nicht der Show oder zur Prahlerei, sondern sie sind für das \hyperref[c1iii14]{Vorspielen} ohne Fehler und Gedächtnislücken entscheidend, und sie ergeben sich fast als automatisches Nebenprodukt dieser Methoden, sogar für uns gewöhnliche Sterbliche mit einem gewöhnlichen Gedächtnis.
Viele Schüler können komplette Kompositionen spielen, sie aber nicht niederschreiben oder in Gedanken spielen - solche Schüler haben die Kompositionen nur zum Teil und auf eine Art auswendiggelernt, die für Auftritte unzureichend ist.
Ein unzureichendes Gedächtnis und ein Mangel an Selbstvertrauen sind die Hauptursachen für \hyperref[c1iii15]{Nervosität}.
Sie fragen sich, warum sie Lampenfieber bekommen und warum das fehlerfreie Vorspielen eine solch entmutigende Aufgabe ist, während Mozart sich einfach hinsetzen und spielen konnte.

\textbf{Weitere Beispiele hilfreichen Wissens sind die \hyperref[c1ii14]{Entspannung} und der Gebrauch der Schwerkraft.}
Das Gewicht des Arms ist nicht nur als Basis für gleichmäßiges und ausgeglichenes Spielen wichtig (die Schwerkraft ist immer konstant), sondern auch zum Testen des Grades der Entspannung.
\textbf{Das Klavier wurde mit der Schwerkraft als Referenz konstruiert (\hyperref[c1ii10]{Kapitel 1, Abschnitt II.10}), weil der menschliche Körper sich genau passend zur Schwerkraft entwickelte}, was bedeutet, dass die zum Klavierspielen notwendige Kraft ungefähr dem Gewicht des Arms entspricht.
Wenn wir schwierige Tätigkeiten ausführen, zum Beispiel eine anspruchsvolle Klavierpassage zu spielen, ist es unsere natürliche Neigung, uns zu verspannen, sodass der ganze Körper zu einer einzigen zusammengezogenen Muskelmasse wird.
Zu versuchen, die Finger unter solchen Bedingungen unabhängig voneinander und schnell zu bewegen, ist so, als ob man einen Sprint mit Gummibändern um beide Beine versuchen wollte.
Wenn Sie alle unnötigen Muskeln entspannen können und nur die erforderlichen Muskeln bloß für die Augenblicke benutzen, in denen sie gebraucht werden, dann können Sie längere Zeit ohne Anstrengung, ohne zu ermüden und mit mehr Kraftreserven als notwendig sind, um die lautesten Töne zu erzeugen, extrem schnell spielen. 

\textbf{Wir werden sehen, dass viele \enquote{etablierte Lehrmethoden} Mythen sind, die dem Schüler unbeschreibliches Elend verursachen können.}
Solche Mythen überleben aufgrund eines Mangels an gründlicher wissenschaftlicher Untersuchung.
Diese Methoden sind unter anderem: die gebogene Fingerhaltung, der Daumenuntersatz zum Spielen von Tonleitern, die meisten Fingerübungen, eine hohe Sitzposition, \enquote{ohne Fleiß kein Preis}, die Geschwindigkeit langsam steigern und der großzügige Gebrauch des Metronoms.
Wir erklären nicht nur, warum sie schädlich sind, sondern zeigen auch die korrekten Alternativen auf, welche jeweils folgende sind: \hyperref[c1iii4b]{flache Fingerhaltungen}, \hyperref[c1iii5a]{Daumenübersatz}, \hyperref[c1ii11]{parallele Sets}, eine niedrigere \hyperref[c1ii3]{Sitzposition}, \hyperref[c1ii14]{Entspannung}, schnelles Erreichen der Geschwindigkeit durch ein Verständnis der \enquote{Geschwindigkeitsbarrieren} und Aufzeigen besonders nützlicher Anwendungen des \hyperref[c1ii19]{Metronoms}.
\textbf{Auf \textit{Geschwindigkeitsbarrieren} trifft man, wenn man versucht, eine Passage schneller zu spielen, aber eine Maximalgeschwindigkeit erreicht, die man nicht mehr steigern kann, egal wie hart man übt.}
Was verursacht Geschwindigkeitsbarrieren, wie viele gibt es, und wie vermeidet oder eliminiert man sie?
Die Antworten: \textbf{Geschwindigkeitsbarrieren sind das Resultat Ihrer Versuche, das Unmögliche zu tun (das heisst Sie errichten die Geschwindigkeitsbarrieren selbst, indem Sie die falschen Übungsmethoden benutzen!), es gibt im Grunde eine unendliche Anzahl, und man vermeidet sie, indem man die richtigen Übungsmethoden benutzt.}
Eine Möglichkeit, Geschwindigkeitsbarrieren zu vermeiden, ist, sie gar nicht erst aufzubauen, indem man ihre Ursachen kennt (Stress, falscher Fingersatz oder \hyperref[c1iii1b]{Rhythmus}, Mangel an Technik, \hyperref[c1ii13]{zu schnelles Üben}, mit beiden Händen zusammen üben, bevor man dazu bereit ist, usw.).
\textbf{Eine weitere Möglichkeit ist, von unendlicher Geschwindigkeit aus mit der Geschwindigkeit abwärts zu gehen, indem man die \hyperref[c1ii11]{parallelen Sets} (oder den \hyperref[c1ii9]{Akkord-Anschlag}) benutzt, anstatt die Geschwindigkeit schrittweise zu steigern.}
Wenn Sie mit Geschwindigkeiten beginnen können, die oberhalb der Geschwindigkeitsbarrieren liegen, dann gibt es keine Geschwindigkeitsbarriere, wenn Sie die Geschwindigkeit verringern.

Dieses Buch behandelt oft einen wichtigen Punkt - dass die besten Übungsmethoden für das Klavierspielen überraschend kontraintuitiv sind.
Dieser Punkt ist in der Klavierpädagogik von größter Wichtigkeit, weil er der hauptsächliche Grund dafür ist, warum die falschen Übungsmethoden oft von den Schülern und den Lehrern benutzt werden.
Wenn sie nicht so kontraintuitiv wären, dann wäre dieses Buch nicht notwendig gewesen.
Folglich behandeln wir nicht nur, was man tun sollte, sondern auch, was man nicht tun sollte.
Diese negativen Abschnitte sind nicht dazu gedacht, diejenigen zu kritisieren, die die falschen Methoden benutzen, sondern sind notwendige Komponenten des Lernprozesses.
Der Grund, warum die Intuition falsch liegt, ist, dass die Aufgaben beim Klavierspielen so komplex sind und es so viele Möglichkeiten gibt, sie zu erfüllen, dass die Wahrscheinlichkeit, die richtige Methode zu treffen, nahe null ist, wenn man die einfachsten, offensichtlichen auswählt.
Dazu vier Beispiele kontraintuitiver Übungsmethoden:

\begin{enumerate}[label={\arabic*.}] 
\item \hyperref[c1ii7]{Die Hände beim Üben zu trennen} ist kontraintuitiv, weil man zunächst mit jeder einzelnen Hand üben muss, dann \hyperref[c1ii25]{mit beiden zusammen}, sodass es so aussieht, als müsste man dreimal üben anstatt nur einmal mit beiden Händen zusammen.
Warum soll man die Hände getrennt üben, also etwas, das man zum Schluss nie benutzen wird?
Ungefähr 80\% dieses Buchs handeln davon, warum man die Hände getrennt üben \textit{muss}.
\textbf{Die Hände getrennt zu üben ist der einzige Weg, schnell die Geschwindigkeit und die Kontrolle zu steigern, ohne in Schwierigkeiten zu kommen.}
Es erlaubt Ihnen, 100\% der Zeit bei jeder Geschwindigkeit ohne Ermüdung, Stress oder Verletzungen hart zu arbeiten, weil diese Methode darauf basiert, die Hände zu wechseln, sobald die arbeitende Hand anfängt müde zu werden.
\textbf{Die Hände getrennt zu üben ist die einzige Möglichkeit, wie Sie experimentieren können, um die korrekten Handbewegungen für die Geschwindigkeit und den Ausdruck zu finden, und es ist der schnellste Weg, um zu lernen \hyperref[c1ii14]{wie man entspannt}.}
Zu versuchen, sich die Technik mit beiden Händen zusammen anzueignen, ist die Hauptursache für Geschwindigkeitsbarrieren, schlechte Angewohnheiten, Verletzungen und Stress.

\item Man neigt intuitiv dazu, langsam \hyperref[c1ii25]{mit beiden Händen zusammen} zu üben und die Geschwindigkeit schrittweise zu steigern, das ist aber eine der schlechtesten Arten zu üben, weil sie so viel Zeit verschwendet und man die Hände dazu trainiert, langsame Bewegungen auszuführen, die sich von denen unterscheiden, die Sie bei der endgültigen Geschwindigkeit brauchen.
\textbf{Einige Schüler verschlimmern das Problem, indem sie das Metronom ständig  als Richtschnur benutzen, um die Geschwindigkeit zu steigern oder den \hyperref[c1iii1b]{Rhythmus} zu halten.
Das ist einer der schwersten Fälle von Missbrauch des Metronoms.
Ein Metronom sollten Sie nur kurz benutzen, um das Timing (Geschwindigkeit und Rhythmus) zu prüfen.}
Wenn Sie es zu viel benutzen, kann das zum Verlust Ihres internen Rhythmus, zum Verlust der Musikalität und zu biophysikalischen Schwierigkeiten führen, die entstehen können, wenn man starren Wiederholungen zu lange ausgesetzt ist (das Gehirn kann sogar anfangen, dem Metronomklick entgegenzuwirken, und man hört das Klicken eventuell nicht oder zur falschen Zeit).
\textbf{Die für die Geschwindigkeit notwendige Technik eignet man sich durch das Entdecken von neuen Handbewegungen an, nicht indem man eine langsame Bewegung beschleunigt}; das heisst die Handbewegungen für langsames Spielen und für schnelles Spielen unterscheiden sich voneinander.
Deshalb führt der Versuch, eine langsame Bewegung zu beschleunigen zu Geschwindigkeitsbarrieren - weil man versucht, das Unmögliche zu tun.
Langsames Spielen zu beschleunigen ist genau so, als ob man ein Pferd dazu bringen wollte, das Gehen auf die Geschwindigkeit des Galopps zu bringen - es kann es nicht.
Ein Pferd muss die Bewegung vom Gehen zum Trott, Kanter und dann zum Galopp ändern.
Wenn man das Pferd dazu zwingt, mit der Geschwindigkeit des Kanters zu gehen, dann wird es auf eine Geschwindigkeitsbarriere treffen und sich wahrscheinlich dadurch verletzen, dass es sich selbst die Hufe zertritt.

\item Um richtig auswendig zu lernen und in der Lage zu sein, gut vorzuspielen, muss man langsam üben, sogar nachdem man das Stück leicht mit der endgültigen Geschwindigkeit spielen kann.
Das ist kontraintuitiv, weil man beim Auftritt immer mit der endgültigen Geschwindigkeit spielt; warum soll man also langsam üben und so viel Zeit verschwenden?
Schnell zu spielen kann sowohl für das Vorspielen als auch für das Gedächtnis schädlich sein.
Schnell zu spielen kann \enquote{\hyperref[fpd]{Schnellspiel-Abbau}} verursachen, und die beste Möglichkeit, Ihr Gedächtnis zu testen, ist, langsam zu spielen.
\textbf{Deshalb wird es zu einem schlechten Auftritt führen, wenn man die Konzertstücke am Tag des Konzerts mit voller Geschwindigkeit übt.}
Das ist eine der kontraintuitivsten Regeln und deshalb schwer zu befolgen.
Wie oft haben Sie schon den Satz gehört: \enquote{Ich habe während der Unterrichtsstunde schrecklich gespielt, obwohl ich heute Morgen so gut gespielt habe!}?
Obwohl ein großer Teil dieses Buchs darauf ausgerichtet ist, zu lernen, mit der richtigen Geschwindigkeit zu spielen, ist deshalb der richtige Gebrauch der langsamen Geschwindigkeit für ein genaues Gedächtnis und ein fehlerfreies Vorspielen entscheidend.
Langsam zu üben ist jedoch gar nicht so einfach, weil man nicht langsam üben sollte, bevor man nicht schnell spielen kann!
Ansonsten hätten Sie keine Vorstellung davon, ob Ihre Bewegungen beim langsamen Spielen richtig oder falsch sind.
Dieses Problem wird gelöst, indem man die Hände getrennt übt und schnell die endgültige Geschwindigkeit erreicht.
Nachdem man die Handbewegungen für das schnelle Spielen kennt, kann man jederzeit langsam üben.

\item Den meisten Menschen fällt es schwer, etwas auswendig zu lernen, das sie nicht spielen können, weshalb sie instinktiv ein Stück zuerst lernen und \textit{dann} versuchen, es auswendig zu lernen.
Es stellt sich heraus, dass \textbf{man jede Menge Zeit sparen kann, indem man zuerst auswendig lernt und dann aus dem Gedächtnis heraus übt} (wir sprechen über technisch anspruchsvolle Musik, die zu schwierig ist, um sie vom Blatt zu spielen).
Außerdem behalten jene, die auswendig lernen nachdem sie das Stück gelernt haben, aus Gründen, die im Buch erläutert werden, niemals so gut.
Sie werden stets von Gedächtnisproblemen geplagt.
Deshalb müssen gute Methoden zum Auswendiglernen ein integraler Bestandteil jeder Übungsprozedur sein; Auswendiglernen ist eine Notwendigkeit, kein Luxus.

 \end{enumerate}
Diese vier Beispiele sollten dem Leser eine gewisse Vorstellung davon geben, was ich mit kontraintuitiven Übungsmethoden meine.
Das Überraschende ist, dass \textit{die Mehrzahl} der guten Übungsmethoden für die meisten Menschen kontraintuitiv ist.
Glücklicherweise haben die Genies, die vor uns kamen, bessere Übungsmethoden gefunden.

Warum führt die Tatsache, dass die korrekten Methoden kontraintuitiv sind, zur Katastrophe?
Sogar Schüler, die die korrekten Methoden gelernt haben (denen aber nie beigebracht wurde, was man nicht tun darf), können in die intuitiven Methoden zurückfallen, weil ihnen ihr Gehirn einfach weiterhin sagt, sie sollten die intuitiven Methoden benutzen (das ist die \textit{Definition} von intuitiven Methoden).
Das geschieht Lehrern natürlich genauso.
Eltern tappen jedes Mal in diese Falle!
Dadurch kann eine Beteiligung der Eltern manchmal kontraproduktiv sein; die Eltern müssen ebenfalls \textit{informiert} sein.
Deshalb unternimmt dieses Buch jede Anstrengung, um die Torheiten der intuitiven Methoden zu ermitteln und herauszustellen.
Darum raten viele Lehrer von einer Beteiligung der Eltern ab, es sei denn, die Eltern können ebenfalls am Unterricht teilnehmen.
Wenn sie sich selbst überlassen werden, zieht es die Mehrzahl der Schüler, Lehrer und Eltern zu den intuitiven (falschen) Methoden.
Das ist der Hauptgrund, warum heutzutage so viele falsche Methoden gelehrt werden und warum Schüler informierte Lehrer und vernünftige Lehrbücher brauchen.
Alle Klavierlehrer sollten ein Lehrbuch benutzen, das Übungsmethoden erklärt; das wird sie davon befreien, die Mechanismen des Übens lehren zu müssen und ihnen gestatten, sich auf die Musik zu konzentrieren, wobei die Lehrer am meisten benötigt werden.
Die Eltern sollten das Lehrbuch ebenfalls lesen, weil Eltern sehr leicht in die Fallen der intuitiven Methoden geraten.

Klavierlehrer lassen sich im Allgemeinen in drei Kategorien einteilen:

\begin{enumerate}[label={\alph*.}] 
\item private Lehrer, die nicht unterrichten können,
\item private Lehrer, die sehr gut sind, und
\item Lehrer an Universitäten und Konservatorien.
 \end{enumerate}
Die letzte Gruppe ist üblicherweise sehr gut, weil sie in ihrem Umfeld miteinander kommunizieren müssen.
Sie sind in der Lage, die schlimmsten Lehrmethoden schnell zu identifizieren und sie zu eliminieren.
Unglücklicherweise sind die meisten Schüler an Konservatorien bereits ziemlich fortgeschritten, und somit ist es zu spät, um ihnen grundlegende Übungsmethoden beizubringen.
Die Gruppe (a) besteht hauptsächlich aus \enquote{Einzelkämpfern}, die sich nicht so sehr mit anderen Lehrern austauschen und hauptsächlich die intuitiven Methoden benutzen; das erklärt, warum sie nicht unterrichten können.
Sie können viele der schlechten Lehrer umgehen, wenn Sie nur Lehrer auswählen, die eine Website unterhalten, weil diese zumindest gelernt haben zu kommunizieren.
Die Gruppen (b) und (c) sind mit den korrekten Übungsmethoden ziemlich vertraut, obwohl wenige sie alle kennen, weil es kein standardisiertes Lehrbuch gab; auf der anderen Seite wissen die meisten von Ihnen eine Menge nützlicher Details, die nicht in diesem Buch enthalten sind.
Es gibt herzlich wenige Lehrer der Gruppe (b) und die Lehrer der Gruppe (c) akzeptieren im Allgemeinen nur fortgeschrittene Schüler.
Das Problem mit dieser Situation ist, dass die meisten Schüler mit Lehrern der Gruppe (a) anfangen und nie über das Anfänger- oder Mittelstufenniveau hinauskommen und sich deshalb niemals für die Lehrer der Gruppe (c) qualifizieren.
Deshalb gibt die Mehrzahl der Anfänger frustriert auf, obwohl praktisch alle von ihnen das Potential haben, ein vollendeter Musiker zu werden.
Mehr noch, dieser Mangel an Vorwärtskommen nährt das allgemeine Missverständnis, dass Klavierspielen zu lernen ein lebenslanges fruchtloses Bemühen bedeutet, das die Mehrzahl der Eltern und Kinder davon abhält, über Klavierunterricht nachzudenken.

Es gibt eine innige Beziehung zwischen Musik und Mathematik.
Musik ist in vielerlei Hinsicht eine Form der Mathematik, und die großen Komponisten haben diese Beziehung untersucht und ausgenutzt.
Die meisten grundlegenden Theorien der Musik können mit mathematischen Termen ausgedrückt werden.
Die Harmonie ist eine Reihe von Verhältnissen, und die Harmonie führt zur chromatischen Tonleiter, die eine logarithmische Gleichung ist.
Die meisten Tonleitern sind Teilmengen der chromatischen Tonleiter, und Akkordprogressionen sind die einfachsten Beziehungen unter diesen Teilmengen.
Ich bespreche einige konkrete Beispiele für den Gebrauch der Mathematik in einigen der berühmtesten Kompositionen (\hyperref[c1iv4]{Kapitel 1, Abschnitt IV.4}) und schließe alle \hyperref[c1iv6]{Themen für zukünftige Untersuchungen} in der Musik (sowohl mathematische als auch andere) in Kapitel 1, Abschnitt IV ein.
Es macht keinen Sinn, zu fragen, ob Musik Kunst oder Mathematik ist; beide sind Eigenschaften der Musik.
Mathematik ist einfach eine Möglichkeit, etwas quantitativ zu messen; deshalb kann alles, was in der Musik quantifiziert werden kann (zum Beispiel das Taktmaß, die thematische Struktur usw.), mathematisch behandelt werden.
Obwohl die Mathematik für einen Künstler nicht notwendig ist, sind deshalb Musik und Mathematik untrennbar verbunden.
Das Wissen um diese Beziehung kann oft nützlich sein (wie von jedem großen Komponisten gezeigt wird) und wird immer nützlicher werden, je mehr sich das musikalische Verständnis der Mathematik der Musik schrittweise nähert und die Künstler lernen, einen Nutzen aus der Mathematik zu ziehen.
Die Kunst ist eine Abkürzung, bei der das menschliche Gehirn dazu benutzt wird, Ergebnisse zu erzielen, die auf anderem Wege nicht zu erzielen sind.
Ein wissenschaftliches Herangehen an die Musik beschäftigt sich nur mit den einfacheren Ebenen der Musik, die analytisch behandelt werden können: Die Wissenschaft unterstützt die Kunst.
Die Annahme ist falsch, dass die Wissenschaft irgendwann die Kunst ersetzen wird oder, das andere Extrem, dass man für Musik nur die Kunst benötigt; die Kunst sollte so frei sein, alles einzufügen, das der Künstler wünscht, und die Wissenschaft kann eine unschätzbare Hilfe bieten.

Zu viele Klavierspieler wissen nicht, wie das Klavier funktioniert und was es bedeutet, \hyperref[c2_2c]{temperiert zu stimmen}, oder was es bedeutet, ein Klavier zu \hyperref[c2_7_hamm]{intonieren}.
Das ist besonders überraschend, weil die Wartung des Klaviers einen direkten Einfluss auf die Fähigkeit Musik zu machen und auf die Entwicklung der Technik hat.
Es gibt viele Konzertpianisten, die nicht den Unterschied zwischen \hyperref[et1]{gleichschwebender Temperatur} und wohltemperierten Stimmungen kennen, obwohl einige der Kompositionen, die sie spielen (zum Beispiel Bach), den Gebrauch der einen oder der anderen ausdrücklich verlangen.
Wann man ein elektronisches Klavier benutzen soll, wann man zu einem Klavier oder Flügel höherer Qualität wechseln soll und wie man bei einem Klavier Qualität erkennt, sind in der Karriere eines jeden Klavierspielers wichtige Fragen.
Deshalb enthält dieses Buch einen Abschnitt über den \hyperref[c1iii17e]{Kauf eines Klaviers} und ein Kapitel über das \hyperref[c2_1]{Stimmen des eigenen Klaviers}.
So wie die elektronischen Klaviere bereits immer richtig gestimmt sind, so müssen auch die akustischen Klaviere in naher Zukunft dauerhaft richtig gestimmt sein, zum Beispiel indem man den temperaturabhängigen Ausdehnungskoeffizienten der Saiten benutzt, um das Klavier elektronisch zu stimmen (siehe Gilmore, http://home.kc.rr.com/eromlignod, self-tuning piano).
Heutzutage sind praktisch alle Heimklaviere fast die ganze Zeit aus der Stimmung, weil das Klavier anfängt aus der Stimmung zu gehen, sobald der Stimmer das Haus verlässt oder sich die Temperatur oder Feuchtigkeit im Raum ändert.
Das ist eine unannehmbare Situation.
Bei zukünftigen Klavieren wird man einen Schalter umlegen können, und das Klavier stimmt sich innerhalb von Sekunden selbst.
Wenn sie massenweise produziert werden, sind die Kosten eines sich selbst stimmenden Klaviers im Vergleich zu einem Qualitätsklavier gering.
Man könnte meinen, dass dies die Klavierstimmer arbeitslos machen würde, aber das wird nicht der Fall sein, weil die Zahl der Klaviere (durch dieses Buch) zunehmen wird, der Mechanismus zum Selbststimmen gewartet werden muss und bei Klavieren mit einer solch perfekten Stimmung das regelmäßige Intonieren der Hämmer und Einstellen (beides wird heute oft vernachlässigt) eine bedeutsame Verbesserung des musikalischen Ergebnisses bewirkt.
Durch die gestiegene Anzahl der fortgeschrittenen Klavierspieler entsteht eine größere Nachfrage nach diesen gehobenen Wartungsarbeiten.
Sie könnten plötzlich erkennen, dass es das Klavier war, nicht Sie selbst, das die technische Entwicklung und das musikalische Ergebnis begrenzt hat (bei abgenutzten Hämmern ist das immer der Fall!).
Was denken Sie, warum Konzertpianisten so viel Aufhebens um ihr Klavier machen?

Zusammengefasst: Dieses Buch stellt ein einmaliges Ereignis in der Geschichte der Klavierpädagogik dar und revolutioniert den Klavierunterricht.
Überraschenderweise ist wenig in diesem Buch grundlegend neu.
Wird verdanken die meisten wichtigen Konzepte Combe, Liszt, Chopin, Beethoven, Mozart, Bach usw.
Combe und Liszt gaben uns das \hyperref[c1ii7]{Üben mit getrennten Händen}, das \hyperref[c1ii6]{Üben in kleinen Portionen} und die \hyperref[c1ii14]{Entspannung};
Liszt und Chopin gaben uns den \hyperref[c1iii5b]{Daumenübersatz} und befreiten uns von \hyperref[c1iii7h]{Hanon} und Czerny;
Mozart lehrte uns das \hyperref[c1iii6]{Auswendiglernen} und das \hyperref[c1ii12mental]{mentale Spielen};
Bach wusste alles über \hyperref[c1ii11]{parallele Sets}, \hyperref[ruhig]{ruhige Hände} und die Wichtigkeit des \hyperref[c1iii14d]{musikalischen Übens},
und sie alle (besonders Beethoven) zeigten uns die Beziehungen zwischen Mathematik und Musik.
Die enorme Menge an Zeit und Anstrengung, die, um das Rad neu zu erfinden und beim nutzlosen Wiederholen von Fingerübungen, in der Vergangenheit in jeder Pianistengeneration verschwendet wurde, übersteigt alle Vorstellungen.
Indem das in diesem Buch zusammengestellte Wissen dem Schüler vom ersten Tag des Klavierunterrichts an zugänglich gemacht wird, läuten wir ein neues Zeitalter im Erlernen des Klavierspielenlernens ein.
Dieses Buch ist nicht das Ende der Straße - es ist nur ein Anfang.
Die zukünftige Erforschung der Übungsmethoden wird zweifellos Verbesserungen zu Tage fördern; das liegt in der Natur des wissenschaftlichen Vorgehens.
Es garantiert, dass wir nie wieder nützliche Informationen verlieren und immer nur voranschreiten werden, und dass jeder Lehrer Zugang zu den besten verfügbaren Informationen haben wird.
Wir verstehen bislang noch nicht die biologischen Veränderungen, die den Erwerb der Technik begleiten, und wie sich das menschliche (besonders das kindliche) Gehirn entwickelt.
Diese zu verstehen wird uns erlauben, sie direkt hervorzubringen, statt dass wir etwas 10.000-mal wiederholen müssen.
Seit Bachs Zeit gab es in der Klavierpädagogik einen Stillstand der Entwicklung; wir haben nun die Hoffnung, das Klavierspielen von einem scheinbar unerreichbaren Traum in eine Kunst zu verwandeln, an der sich jeder erfreuen kann.

PS: Dieses Buch ist mein Geschenk an die Gesellschaft.
Die Übersetzer haben ebenfalls ihre kostbare Zeit dazu beigetragen.
Zusammen leisten wir Pionierarbeit dafür, kostenlos Web-basierte Ausbildung von höchstem Format zur Verfügung zu stellen, etwas, das hoffentlich zu einem Vorboten der Zukunft wird.
Es gibt keinen Grund, warum Ausbildung nicht kostenlos sein sollte.
Eine solche Umwälzung mag so erscheinen, als ob sie die Jobs einiger Lehrer gefährden könnte, aber mit verbesserten Lehrmethoden wird das Klavierspielen viel populärer werden, was zu einer höheren Nachfrage nach Lehrern führen wird, die unterrichten können, weil Schüler mit einem guten Lehrer immer schneller lernen werden als wenn sie alleine sind.
Die ökonomischen Auswirkungen dieser verbesserten Lernmethoden können beträchtlich sein.
Dieses Buch wurde 1994 zuerst gedruckt, und die Website\footnote{das heisst die von Chuan C. Chang} startete 1999.
Ich schätze, dass bis zum Jahr 2002 mehr als 10.000 Schüler diese Methode gelernt hatten.
Nehmen wir an, dass 10.000 ernsthafte Klavierschüler durch diese Methoden 5 Stunden je Woche einsparen, dass sie 40 Wochen pro Jahr üben, und dass ihre Zeit einem Wert von 5\$ je Stunde entspricht; dann ist die gesamte jährliche Ersparnis:

(5 Stunden / (Woche * Schüler)) * (40 Wochen / Jahr) * (\$5 / Stunde) * (10.000 Schüler) = \$10.000.000 / Jahr für 2002 oder \$1.000 / Jahr und Schüler.

10 Millionen Dollar pro Jahr sind nur die Einsparung der Schüler; wir haben die Auswirkungen für die Lehrer sowie die Klavier- und Musikindustrie nicht berücksichtigt.
Jedes Mal, wenn die Übernahme wissenschaftlicher Methoden solche Sprünge in der Effektivität erzeugt hat, hat das jeweilige Gebiet in der Vergangenheit einen Aufschwung erlebt, der anscheinend grenzenlos war und jedem genutzt hat.
Bei der heutigen (2007) Weltbevölkerung von mehr als 6,6 Milliarden Menschen können wir davon ausgehen, dass der Anteil der Klavierspieler schließlich ein Prozent oder mehr als 66 Millionen betragen wird, sodass die potentielle ökonomische Auswirkung dieses Buchs mehrere Milliarden Dollar pro Jahr übersteigen könnte.
Ein solcher wirtschaftlicher Nutzen in einem kleinen Sektor war in der Vergangenheit eine unüberwindliche Kraft, und dieser Motor wird den Umschwung beim Klavierspielen weiter antreiben.
Noch wichtiger ist, dass Musik und jeder Zuwachs bei der geistigen Entwicklung eines Kindes unbezahlbar sind.
 



<!-- c1i1.html -->

\chapter{Klaviertechnik}
\label{c1i1}

\section{Einführung} 

% zuletzt geändert 15.08.2009
\subsection{Zweck dieses Buchs}

Der Zweck dieses Buchs ist es, die besten bekannten Methoden zum Üben des Klavierspielens vorzustellen.
Für Schüler bedeutet das 
Kennen dieser Methoden eine Verringerung der zum Lernen notwendigen Zeit, die einen wesentlichen Teil der Lebenszeit ausmacht, und eine Zunahme der Zeit, die für das Musizieren genutzt werden kann, anstatt sie mit dem Kampf mit der Spieltechnik zu verbringen.
Viele Schüler verbringen 100\% ihrer Zeit damit, neue Stücke zu lernen, und da dieser Vorgang so lange dauert, bleibt keine Zeit für das Üben der Kunst Musik zu machen übrig.
Dieser bedauerliche Umstand ist das größte Hindernis für die Entwicklung der Spieltechnik, weil das Musizieren für die technische Entwicklung notwendig ist.
\textbf{Das Ziel ist hier, den Lernprozess so zu beschleunigen, dass wir 10\% der Übungszeit auf die technische Arbeit und 90\% zum Musizieren verwenden.}

Wie musizieren Musiker?
\textbf{Egal ob man Musik komponiert oder ein Instrument spielt, die ganze Musik muss aus dem Gehirn des Künstlers kommen.}
Mit genügend Übung können wir sicher unser Gehirn abschalten und mechanisch aus dem Gedächtnis spielen.
Das ist absolut die falsche Art Musik zu machen, weil die resultierende Musik auf einer niedrigen Stufe sein wird.
Viele Klavierspieler nehmen irrtümlicherweise an, dass ein teurer, großer Konzertflügel seinen eigenen Klang mit seiner charakteristischen Musik erzeugt und wir deshalb zum Lernen des Klavierspielens unsere Finger trainieren müssen.
Das menschliche Gehirn ist aber hinsichtlich der Musikalität weitaus komplexer als alle mechanischen Apparate und diesen überlegen.
Das Gehirn hat nicht die Beschränkungen von Holz, Filz und Metall.
Deshalb ist es wichtiger, das Gehirn zu trainieren als die Fingermuskeln, insbesondere da jede Fingerbewegung von einem Nervenimpuls des Gehirns angeregt werden muss.
Die Antwort auf die obige Frage ist, was wir in diesem Buch als \hyperref[c1ii12mental]{mentales Spielen} bezeichnen werden.
Das mentale Spielen ist einfach der Prozess, sich die Musik in Gedanken vorzustellen oder sie sogar auf einem imaginären Klavier zu spielen.
Wir werden sehen, dass das mentale Spielen praktisch alles kontrolliert, was wir in der Musik tun, vom Lernprozess (Technik) bis zu dem \hyperref[c1iii6]{Auswendiglernen}, dem \hyperref[c1iii12]{absoluten Gehör}, dem \hyperref[c1iii14]{Auftreten}, dem \hyperref[c1iii12blatt]{Komponieren}, der \hyperref[c1iii15]{Kontrolle von Nervosität} usw.
Es ist so allumfassend, dass es nicht möglich ist, es nur in einem Abschnitt zu erklären; es wird praktisch in jedem Abschnitt dieses Buchs besprochen.
Eine ziemlich ausführliche Besprechung finden Sie in \hyperref[c1iii6tastatur]{Abschnitt III.6j}.

Das mentale Spielen machte Mozart (und alle großen Musiker) zu dem, was er war; er wird zum Teil wegen seiner Fähigkeit zum mentalen Spielen als einer der größten Genies angesehen.
Die wunderbare Nachricht ist, \textbf{dass man es lernen kann}.
Die traurige historische Tatsache ist, dass man das mentale Spielen zu vielen Schülern nie gelehrt hat;
in diesem Buch wurde dem mentalen Spielen vielleicht zum ersten Mal ein offizieller Name (Definition) gegeben, obwohl Sie, wenn Sie ein \enquote{talentierter} Musiker sind, es sich vielleicht schon irgendwie auf wundersame Weise aneignen mussten.
\textbf{Das mentale Spielen sollte bereits im ersten Jahr des Klavierunterrichts gelehrt werden und ist bei den jüngsten Kindern besonders effektiv;
die offensichtlichste Möglichkeit, das Unterrichten zu beginnen, ist, die Fertigkeiten für das Auswendiglernen und das absolute Gehör zu lehren.}
Das mentale Spielen ist die Kunst, den Geist des Publikums durch die Musik, die man spielt, zu kontrollieren, und deshalb funktioniert das am besten, wenn es von Ihrem Geist ausgeht.
Das Publikum sieht Ihre Fähigkeit, mental zu spielen, als etwas 
Außerordentliches, was nur ein paar ausgewählten begabten Musikern mit einer Intelligenz weit über dem Durchschnitt gegeben ist.
Mozart war sich fast mit Sicherheit dessen bewusst und benutzte das mentale Spielen, um sein Ansehen sehr zu vergrößern.
Das mentale Spielen hilft Ihnen auf unzählige Arten, das Klavierspielen zu lernen, wie im ganzen Buch gezeigt wird.
Da man das mentale Spielen ohne Klavier ausführen kann, können Sie zum Beispiel Ihre Übungszeit effektiv verdoppeln oder verdreifachen, 
indem Sie das mentale Spielen benutzen, wenn kein Klavier verfügbar ist.
Beethoven und Einstein erschienen oft geistesabwesend, weil sie die meiste wache Zeit mit dem mentalen Spielen verbrachten.\footnote{Wobei Einstein nicht nur Musik mental spielte, sondern zum Beispiel auch: \enquote{Was würde geschehen, wenn die Straßenbahn, in der ich gerade unterwegs bin, mit Lichtgeschwindigkeit fahren würde?}}

Somit ist das mentale Spielen nichts Neues;
nicht nur die großen Musiker und Künstler, sondern praktisch alle heutigen Spezialisten, wie Athleten, trainierte Soldaten, Geschäftsmänner usw. müssen ihr eigenes mentales Spielen pflegen, um im Wettbewerb erfolgreich zu sein.
Tatsächlich benutzt es jeder von uns ständig!
Wenn wir morgens aufstehen und die für den Tag geplanten Aktivitäten auf die Schnelle durchgehen, führen wir mentales Spielen aus, und die Komplexität dieses mentalen Spielens übersteigt wahrscheinlich die einer Mazurka von Chopin.
Trotzdem tun wir das innerhalb eines Augenblicks, sogar ohne es als mentales Spielen aufzufassen, weil wir es seit frühester Kindheit geübt haben.
Können Sie sich vorstellen, welche Desaster geschehen würden, wenn wir keinen mentalen Plan für den Tag hätten?
Aber das tun wir im Grunde, wenn wir auf eine Bühne gehen und ohne Training im mentalen Spielen ein Konzert geben.
Kein Wunder, dass man beim Auftreten so nervös wird!
Wie wir sehen werden, ist das mentale Spielen vielleicht das beste \hyperref[c1iii15]{Mittel gegen das Lampenfieber} - bei Mozart funktionierte es bestimmt.


\subsection{Was ist Klaviertechnik?}
\label{c1i2}

Wir müssen verstehen, was Technik ist, weil sie nicht zu verstehen zu falschen Übungsmethoden führt.
Wichtiger noch: Das richtige Verstehen kann uns dabei helfen, die richtigen Übungsmethoden zu entwickeln.
Das am meisten verbreitete Missverständnis ist, dass Technik eine vererbte Fingerfertigkeit sei.
Sie ist es nicht.
\textbf{Die angeborene Geschicklichkeit von vollendeten Pianisten und von Durchschnittsbürgern ist gar nicht so unterschiedlich.}
Das bedeutet, dass praktisch jeder lernen kann, gut Klavier zu spielen.
Es gibt zahlreiche Beispiele von geistig Behinderten mit eingeschränkter Koordination (Savants, Inselbegabte), die ein erstaunliches musikalisches Talent beweisen.
Viele von uns sind wesentlich geschickter, können jedoch leider die musikalischen Passagen aus einem Mangel an ein paar einfachen aber entscheidenden Informationen nicht bewältigen.
\textbf{Der Erwerb der Technik ist größtenteils ein Prozess der Entwicklung des Gehirns und der Nerven, nicht der Entwicklung von Fingerstärke.}

Technik ist die Fähigkeit, millionen verschiedener Passagen auszuführen; deshalb ist sie keine Geschicklichkeit, sondern eine Ansammlung vieler Fertigkeiten.
Das Wundersame an der Klaviertechnik und \textbf{die wichtigste Botschaft dieses Buchs ist, dass diese Fertigkeiten innerhalb kurzer Zeit erlernt werden können, wenn die richtigen Lernmethoden angewandt werden.}
Diese Fertigkeiten werden in zwei Phasen erlangt:

\begin{enumerate}[label={\arabic*.}] 
\item entdecken, wie Finger, Hände, Arme usw. bewegt werden müssen, und
\item das Gehirn, die Nerven und die Muskeln so zu konditionieren, dass sie diese Bewegungen einfach und kontrolliert ausführen können.
 \end{enumerate}
Viele Schüler denken, dass Klavierspielen zu üben aus stundenlanger Fingergymnastik besteht, weil ihnen nie die eigentliche Bedeutung der Technik beigebracht wurde.
\textbf{In Wahrheit verbessern Sie Ihr Gehirn, wenn Sie das Klavierspielen lernen!}
Sie machen sich selbst klüger und verbessern Ihr Gedächtnis; deshalb hat es so viele nützliche Auswirkungen, wenn Sie das Klavierspielen richtig lernen, zum Beispiel Erfolg in der Schule, die Fähigkeit, besser mit alltäglichen Problemen fertigzuwerden und die Fähigkeit, sich trotz zunehmenden Alters das Gedächtnis länger zu erhalten.
Deshalb ist das \hyperref[c1iii6]{Auswendiglernen} ein untrennbarer Bestandteil des Technikerwerbs.

Wir müssen unsere eigene Anatomie verstehen und lernen, wie wir die korrekte Technik entdecken und uns aneignen können.
Dies stellt eine fast unmögliche Aufgabe für das durchschnittliche menschliche Gehirn dar, es sei denn, Sie widmen ihr von Kindheit an Ihr ganzes Leben.
Selbst dann werden die meisten keinen Erfolg haben.
Der Grund ist, dass der Klavierspieler ohne die richtige Anleitung die korrekten Bewegungen usw. durch Ausprobieren herausfinden muss.
Man hängt von der geringen Wahrscheinlichkeit ab, dass die Hand bei dem Versuch, diese schwierige Passage schneller zu spielen, zufällig in eine funktionierende Bewegung verfällt.
Wenn Sie Pech haben, entdeckt Ihre Hand diese Bewegung nie, und Sie bleiben ewig hängen - ein Phänomen, das man \enquote{Geschwindigkeitsbarriere} nennt.
Die meisten Anfänger unter den Klavierschülern haben nicht die geringste Vorstellung von den komplexen Bewegungen, die die Finger, Hände und Arme ausführen können.
Zum Glück haben die vielen Genies vor uns die meisten nützlichen Entdeckungen bereits gemacht (sonst wären sie keine so großen Künstler gewesen), was zu effizienten Übungsmethoden führt.

Eine weitere falsche Vorstellung von der Technik ist, dass man, wenn die Finger erst einmal genügend geschickt sind, alles spielen kann.
Fast jede einzelne Passage, die sich von den anderen unterscheidet, ist ein neues Abenteuer; sie muss neu gelernt werden.
Erfahrene Pianisten sind \textit{scheinbar} in der Lage, fast alles zu spielen, weil

\begin{enumerate}[label={\arabic*.}] 
\item sie fast alles geübt haben, das man oft vorfindet, und
\item sie wissen, wie man Neues sehr schnell lernt.
 \end{enumerate}
Es gibt große Klassen von Passagen, wie zum Beispiel Tonleitern, die häufig auftreten.
Das Wissen, wie diese zu spielen sind, wird bedeutende Teile der meisten Kompositionen abdecken.
Wichtiger ist jedoch, dass es allgemeine Lösungen für große Problemklassen und spezielle Lösungen für spezielle Probleme gibt.


\subsection{Technik, Musik und mentales Spielen}
\label{c1i3}

Wenn wir uns nur auf die Entwicklung der \enquote{Fingertechnik} konzentrieren und die Musik während des Übens vernachlässigen, können wir unmusikalische Spielgewohnheiten annehmen.
\textbf{Unmusikalisches Spielen ist stets absolut verboten, weil es ein Fehler ist.}
Ein verbreitetes Symptom dieses Fehlers ist die Unfähigkeit, die Übungsstücke zu spielen, wenn der Lehrer (oder sonst jemand!) zuhört.
Wenn Publikum dabei ist, machen diese Schüler seltsame Fehler, die sie während des \enquote{Übens} nicht gemacht haben.
Das geschieht, weil die Schüler ohne Beachtung der Musik geübt hatten und plötzlich erkennen, dass sie nun die Musik hinzufügen müssen, weil jemand zuhört.
Leider haben sie bis zur Unterrichtsstunde niemals wirklich musikalisch geübt!
Ein weiteres Symptom unmusikalischen Übens ist, dass die Schüler sich unwohl fühlen, wenn andere sie beim Üben hören können.
\textbf{Klavierlehrer wissen, dass Schüler musikalisch üben müssen, um sich die Technik anzueignen.
Was für die Ohren und das Gehirn richtig ist, stellt sich als für den menschlichen Spielapparat richtig heraus.}
Sowohl Musikalität als auch Technik benötigen Genauigkeit und Kontrolle.
Praktisch jeder technische Makel kann in der Musik wahrgenommen werden.
Die Musik ist die schwierigste Probe, ob die Technik richtig oder falsch ist.
Wie wir das ganze Buch hindurch sehen werden, gibt es mehr Gründe, warum Musik niemals von der Technik getrennt werden sollte.
Nichtsdestoweniger neigen viele Schüler dazu, beim Üben die Musik zu vernachlässigen und ziehen es vor, zu \enquote{arbeiten}, wenn niemand dabei ist, der zuhört.
Solche Übungsmethoden erzeugen \enquote{Stille-Kämmerlein-Pianisten}, die gerne spielen aber nicht vorspielen können.
\textbf{Wenn Schülern beigebracht wird, immer musikalisch zu üben, dann wird diese Art von Problem gar nicht existieren; vorspielen und üben sind ein und dasselbe.}
Dieses Buch enthält viele Vorschläge für das Üben des \hyperref[c1iii14]{Auftretens}, wie zum Beispiel sein Spielen von Anfang an \hyperref[c1iii13]{auf Video aufzunehmen}.

\textbf{Viele Schüler denken zu Unrecht, dass die Finger die Musik kontrollieren, und sie warten darauf, dass das Klavier diesen großartigen Sound erzeugt.}
Das wird zu einer eintönigen Vorführung und unvorhersehbaren Ergebnissen führen.
Die Musik muss aus dem Geist kommen, und der Klavierspieler muss das Klavier dazu bringen, das zu erzeugen, was er möchte.
Das ist das oben vorgestellte \hyperref[c1ii12mental]{mentale Spielen};
wenn Sie das mentale Spielen noch nie geübt haben, werden Sie feststellen, dass es eine Stufe des Auswendiglernens erfordert, die Sie noch nie erreicht haben - aber \textit{genau} das brauchen Sie für einen fehlerfreien, respekteinflößenden Auftritt.
Zum Glück sind es nur ein paar Schritte von den in diesem Buch geschilderten Verfahren für das \hyperref[c1iii6]{Auswendiglernen} zum mentalen Spielen, aber es bedeutet einen großen Fortschritt in Ihren musikalischen Fertigkeiten, nicht nur für die Technik und das Musikmachen, sondern auch für das Lernen eines \hyperref[c1iii12]{absoluten Gehörs}, das Komponieren und jeden Aspekt des Klavierspielens.
So sind Technik, Musik und mentales Spielen untrennbar miteinander verflochten.
Sobald Sie sich mit dem mentalen Spielen intensiv beschäftigen, werden Sie entdecken, dass es ohne das absolute Gehör nicht wirklich funktioniert.
Diese Erörterungen bieten eine solide Grundlage für das Identifizieren der Fertigkeiten, die wir lernen müssen.
Dieses Buch liefert die Übungsmethoden, die man braucht, um sie zu lernen.


\subsection{Generelles Vorgehen, Interpretation, Musikunterricht, Absolutes Gehör}
\label{c1i4}

Die Lehrer spielen eine wichtige Rolle dabei, den Schülern zu zeigen, wie man musikalisch spielt und übt.
Zum Beispiel beginnen und enden die meisten Musikstücke mit demselben Akkord, eine etwas geheimnisvolle Regel, die eigentlich aus den Grundregeln für Akkordprogressionen resultiert.
Ein Verständnis der Akkordprogressionen ist für das \hyperref[c1iii6]{Auswendiglernen} sehr nützlich.
Eine musikalische Phrase beginnt und endet im Allgemeinen mit leiseren Noten, mit lauteren Noten dazwischen; wenn Sie im Zweifel sind, dann ist dies eine gute Grundregel.
Das ist vielleicht ein Grund, warum so viele Kompositionen mit einem unvollständigen Takt beginnen - der erste Schlag trägt in der Regel den Akzent und ist zu laut.
Es gibt viele Bücher, die sich mit der musikalischen Interpretation beschäftigen (\hyperref[Gieseking]{Gieseking}, \hyperref[Sandor]{Sandor}), und es gibt zahlreiche Beispiele in diesem Buch.

Ein musikalisches Training lohnt sich in jüngsten Jahren am meisten.
Die meisten Babys, die häufig ein perfekt gestimmtes Klavier hören, entwickeln \textit{automatisch} ein \hyperref[c1iii12]{absolutes Gehör} - das ist nichts Außergewöhnliches.
Niemand wird mit einem absoluten Gehör geboren, da es zu 100\% eine erlernte Fertigkeit ist (die exakten Frequenzen der Tonleitern sind willkürliche menschliche Konventionen - es gibt kein Naturgesetz, das besagt, dass das mittlere A bei 440 Hz sein muss; die meisten Orchester stimmen auf 442 Hz, und bevor das A standardisiert wurde, gab es einen viel größeren Bereich erlaubter Frequenzen).
Wenn dieses absolute Gehör nicht gepflegt wird, dann wird es im späteren Leben verloren gehen.
\textbf{Klavierunterricht für junge Kinder kann im Alter von ungefähr drei bis vier Jahren beginnen.
Es ist vorteilhaft, wenn Jüngere früh (ab der Geburt) klassische Musik hören, weil klassische Musik den höchsten musikalischen Gehalt (komplex, logisch) aller verschiedenen Arten von Musik hat.}
Einige Formen der zeitgenössischen Musik könnten durch das Überbetonen bestimmter beschränkter Aspekte - wie Lautstärke oder zu einfache Musikstrukturen, die das Gehirn nicht stimulieren - die musikalische Entwicklung eeinträchtigen, indem sie die Entwicklung des Gehirns stören.

Obwohl man musikalisch begabt sein muss, um Musik zu komponieren, ist die Fähigkeit, Klavier zu spielen, nicht so vom musikalischen Verstand abhängig.
In Wahrheit sind die meisten von uns musikalischer als wir uns selbst zutrauen, und es ist der Mangel an Technik, der unsere musikalische Ausdrucksfähigkeit am Klavier einschränkt.
Wir haben bereits alle die Erfahrung gemacht, berühmten Pianisten zuzuhören und zu erkennen, dass sie sich voneinander unterscheiden - das ist mehr musikalische Sensibilität als wir jemals benötigen, um mit dem Klavierspielen zu beginnen.
Man muss nicht acht Stunden täglich üben; einige berühmte Pianisten haben Übungszeiten von weniger als einer Stunde empfohlen.
Sie können Fortschritte machen, wenn Sie drei- oder viermal die Woche für jeweils eine Stunde üben.

Schließlich sollte eine umfassende musikalische Ausbildung (\hyperref[c1iii5a]{Tonleitern}, Taktarten, Hörschule - einschließlich absolutem Gehör -, Diktate, Theorie usw.) ein integraler Bestandteil davon sein, das Klavierspielen zu lernen, weil alle Teile, die man lernt, für die anderen Teile hilfreich sind.
Letzten Endes ist eine umfassende musikalische Ausbildung der einzige Weg, das Klavierspielen zu lernen.
Leider stehen den meisten angehenden Klavierspielern nicht die Mittel oder die Zeit zur Verfügung, um diesen Weg zu verfolgen.
Dieses Buch ist dazu gedacht, dem Schüler eine Ausgangsbasis zu geben, indem er lernt, wie man sich die Technik schnell aneignet, sodass er sich überlegen kann, alle die anderen nützlichen Themen zu studieren.
\textbf{In der Regel komponieren Schüler, die glänzende Klavierspieler sind, am Ende fast immer ihre eigene Musik.}
Das Studium der Kompositionslehre ist keine Voraussetzung für das Komponieren.
Einige Lehrer halten nicht viel davon, zu viel Kompositionstheorie zu lernen, bevor man mit dem Komponieren seiner eigenen Musik beginnt, weil einen das davon abhalten kann, seinen eigenen Stil zu entwickeln.

Was sind einige der herausragenden Merkmale der Methoden dieses Buchs?

\begin{enumerate}[label={\arabic*.}] 
\item Diese Methoden sind nicht so übermäßig anstrengend wie ältere Methoden, die den Schülern für den Klavierunterricht einen hingebungsvollen Lebensstil abverlangen.
Die Schüler erhalten die Möglichkeit, sich eine bestimmte Prozedur auszusuchen, mit der man ein definiertes Ziel innerhalb einer abschätzbaren Zeitspanne erreichen kann.
Wenn die Methoden \textit{wirklich} funktionieren, sollten sie kein lebenslanges blindes Vertrauen erfordern, um Können zu erlangen!

\item Jede Prozedur dieser Methoden hat eine körperliche Grundlage (wenn sie funktioniert, hat sie immer eine; die früheren Probleme der Klavierpädagogik lagen im Finden der richtigen Erklärungen); sie muss außerdem die folgenden erforderlichen Elemente enthalten:

\begin{enumerate}[label={\alph*.}] 
<li>\textbf{Ziel:} Techniken, die erworben werden sollen, das heißt wenn Sie nicht schnell genug oder keine Triller spielen können, wenn Sie auswendig spielen möchten, usw.

\item \textbf{Dann ist zu tun:} das heißt \hyperref[c1ii7]{mit getrennten Händen üben}, den \hyperref[c1ii9]{Akkord-Anschlag} benutzen, während des Übens auswendig lernen usw.

\item \textbf{Weil:} die physiologischen, psychologischen, mechanischen usw. Erklärungen dafür, warum diese Methoden funktionieren; zum Beispiel vereinfacht mit getrennten Händen zu üben schwierige Passagen.

\item \textbf{Wenn nicht:} Probleme, die entstehen, wenn auf Unkenntnis beruhende Methoden benutzt werden.
Ohne dieses \enquote{Wenn nicht} können die Schüler jede andere Methode wählen - warum also diese?
Wir müssen wissen, was wir nicht tun dürfen, denn schlechte Angewohnheiten und falsche Methoden, nicht ungenügende Übung, sind die Hauptgründe für einen Mangel an Fortschritt.


 \end{enumerate}
</li>
\item 
Dieses Buch bietet einen vollständigen, gegliederten Satz an Lernwerkzeugen, die Sie mit einem Minimum an Aufwand in das Wunderland des \hyperref[c1ii12mental]{mentalen Spielens} bringen.
Gute Reise!

 \end{enumerate}


<!-- c1ii1.html -->

\section{Grundlegende Verfahren des Klavierübens}
\label{c1ii1}

% zuletzt geändert 22.08.2009

Dieser Abschnitt enthält die minimalen Anleitungen, die Sie benötigen, bevor Sie mit dem Üben anfangen.
 

\subsection{Der Übungsablauf} 

Viele Schüler benutzen folgenden Übungsablauf:

\begin{enumerate}[label={\arabic*.}] 
\item Zunächst Tonleitern oder Fingerübungen spielen, bis die Finger aufgewärmt sind.
Zum Verbessern der Technik wird dies, insbesondere unter Verwendung von Übungen wie der Hanon-Serie, 30 Minuten durchgeführt - wenn man Zeit hat auch länger.
\item Dann nimmt man ein neues Musikstück und liest langsam eine oder zwei Seiten, während man das Stück sorgfältig mit beiden Händen vom Anfang ab spielt.
Dieses langsame Spielen wird so lange wiederholt, bis man das Stück ziemlich gut vorspielen kann, und nun wird die Geschwindigkeit schrittweise gesteigert, bis die endgültige Geschwindigkeit erreicht ist.
Für dieses schrittweise Steigern könnte ein Metronom benutzt werden.
\item Am Ende einer zweistündigen Übungseinheit fliegen die Finger, sodass die Schüler so schnell spielen können wie sie möchten und die Erfahrung genießen können, bevor sie mit dem Üben aufhören.
Nach all dem sind sie des Übens müde, entspannen sich und spielen mit Leib und Seele mit voller Geschwindigkeit.
Dies ist der Moment, in dem sie Spaß an der Musik haben!\item Wenn das Stück zufriedenstellend gespielt werden kann, wird es auswendig gelernt und dann geübt, \enquote{bis die Musik in den Händen ist}.

\item Am Tag des Konzerts oder des Unterrichts üben sie das Stück in der richtigen Geschwindigkeit (oder schneller!) so oft wie möglich, um sicherzustellen, dass es in bestem Zustand ist.
Das ist die letzte Gelegenheit, und offensichtlich gilt: je mehr Übung desto besser.
 \end{enumerate}
\textbf{Jeder Schritt dieses Ablaufs ist falsch!}
Dieser Ablauf wird mit ziemlicher Sicherheit dazu führen, dass die Schüler nicht über die Mittelstufe hinauskommen, auch wenn sie täglich mehrere Stunden üben.
Zum Beispiel gibt dieser Ablauf den Schülern keinen Hinweis, was sie tun müssen, wenn sie auf eine nicht spielbare Passage treffen, außer dass sie diese ständig - manchmal ein Leben lang - wiederholen sollen, ohne eine klare Vorstellung darüber, wann und wie die dafür notwendige Technik erworben wird.
Diese Methode überlässt die Aufgabe, das Klavierspielen zu lernen, völlig dem Schüler.
Zudem wird die Musik während des Vorspielens flach klingen und unerwartete Fehler werden fast unausweichlich sein.
Sie werden das alles verstehen, sobald sie die weiter unten beschriebenen, effizienteren Methoden kennenlernen.

\textbf{Mangel an Fortschritt ist der Hauptgrund, warum so viele Schüler mit dem Klavier aufhören}.
Schüler, insbesondere jüngere, sind clever; warum wie ein Sklave schuften und nichts dabei lernen?
Belohnen Sie die Schüler, und sie werden mehr Hingabe erzielen als jeder Lehrer erwarten kann.
Man kann Arzt sein, Wissenschaftler, Rechtsanwalt, Athlet oder was auch immer man möchte und trotzdem ein guter Pianist werden, weil es Methoden gibt, mit denen Sie die Technik rasch erwerben können, wie Sie gleich sehen werden.

\textbf{Beachten Sie, dass der obige Übungsablauf eine \enquote{intuitive} (oder \enquote{instinktive}) Methode ist.}
Wenn jemand, der durchschnittlich intelligent ist, mit nichts außer einem Klavier auf einer einsamen Insel ausgesetzt worden wäre und sich entscheiden würde zu üben, würde diese Person wahrscheinlich eine Übungsmethode wie die obige entwerfen.
Das heißt, ein Lehrer, der diese Art von Übungsmethoden lehrt, lehrt im Grunde nichts - die Methode ist intuitiv.
\textbf{Als ich zum ersten Mal damit anfing, die \enquote{richtigen Lernverfahren} zusammenzutragen, war ich am meisten davon überrascht, wie viele davon kontraintuitiv waren.}
Ich werde später erklären, warum sie so kontraintuitiv sind, aber dies bietet die beste Erklärung, warum so viele Lehrer den intuitiven Ansatz verwenden.
Diese Lehrer haben die richtigen Methoden niemals gelernt und wurden deshalb zu den intuitiven Methoden hingezogen.
Die Schwierigkeit mit kontraintuitiven Methoden ist, dass sie schwerer anzunehmen sind als intuitive; Ihr Gehirn sagt Ihnen ständig, sie seien falsch und Sie sollten zu den intuitiven zurückkehren.
Diese Botschaft des Gehirns kann vor der Unterrichtsstunde oder dem Konzert unwiderstehlich werden - versuchen Sie, (nicht informierten) Schülern zu sagen, sie sollen keinen Spaß damit haben, ihre fertigen Stücke zu spielen, bevor sie mit dem Üben aufhören, oder am Tag eines Konzerts nicht zu viel zu üben!
Es geht nicht nur um die Schüler oder Lehrer.
Es sind auch Eltern oder Freunde mit guten Absichten, die die Übungsgewohnheiten junger Schüler beeinflussen.
\textbf{Nicht informierte Eltern werden ihre Kinder stets dazu zwingen, die intuitiven Methoden zu benutzen.}
Dies ist ein Grund, warum gute Lehrer immer die Eltern bitten, ihre Kinder zu den Unterrichtsstunden zu begleiten.
Wenn die Eltern nicht informiert sind, gibt es praktisch eine Garantie dafür, dass sie die Schüler dazu zwingen, Methoden zu benutzen, die im Widerspruch zu den Anweisungen des Lehrers stehen.

Schüler, die von Anfang an mit den richtigen Methoden begannen, sind \textit{scheinbar} die Glücklichen.
Sie müssen jedoch später aufpassen, falls man ihnen nicht beigebracht hat, was die falschen Methoden sind.
Wenn sie ihren Lehrer verlassen, dann können sie plötzlich in die intuitiven Methoden verfallen und haben keine Ahnung, warum ihnen alles entgleitet.
Es ist wie ein Bär, der noch nie eine Bärenfalle gesehen hat - er wird jedes Mal gefangen.
Diese \enquote{Glücklichen} können oftmals auch nicht unterrichten, weil sie vielleicht nicht erkennen, dass viele intuitive Methoden zur Katastrophe führen können.
Die scheinbar unglücklichen Schüler, die zuerst die intuitiven Methoden gelernt haben und dann zu den besseren übergegangen sind, haben hingegen einige unerwartete Vorteile.
Sie kennen sowohl die richtigen als auch die falschen Methoden und sind oft die viel besseren Lehrer.
\textbf{Obwohl dieses Buch die richtigen Methoden lehrt, ist es deshalb genauso wichtig, zu wissen, was man \textit{nicht} tun darf und warum.}
Deshalb werden die am häufigsten benutzten falschen Methoden hier ausgiebig besprochen.

Wir beschreiben die Komponenten eines angemessenen Übungsablaufs in den folgenden Abschnitten.
Sie werden ungefähr in der Reihenfolge dargeboten, in der sie ein Schüler vom Anfang bis zum Ende eines neuen Musikstücks benutzen könnte.
\textbf{Anfänger sollten zunächst \hyperref[c1iii18]{Abschnitt III.18} lesen.}


\subsection{Position der Finger}
\label{c1ii2}

Entspannen Sie die Finger, und setzen Sie die Hand auf eine glatte Fläche, wobei alle Fingerspitzen auf der Oberfläche ruhen und das Handgelenk in gleicher Höhe wie die Knöchel ist.
\textbf{Die Hand und die Finger sollten eine Kuppel formen.
Alle Finger sollten gebogen sein.
Der Daumen sollte leicht nach unten zeigen und leicht zu den Fingern hin gebeugt sein, sodass das Nagelglied des Daumens von oben gesehen parallel zu den anderen Fingern ist.}
Diese leichte Einwärtsbeugung des Daumens ist nützlich, wenn Sie \hyperref[c1iii7e]{Akkorde mit weiter Spanne} spielen.
Dieses bringt die Daumenspitze in eine Position parallel zu den Tasten und macht es unwahrscheinlicher, dass Sie eine benachbarte Taste treffen.
Es richtet den Daumen außerdem so aus, dass die richtigen Muskeln zum Anheben und Senken des Daumens benutzt werden.
\textbf{Die Finger sind leicht gekrümmt, abwärts gebogen und treffen in einem Winkel von ungefähr 45 Grad auf die Oberfläche.}
Diese gekrümmte Haltung erlaubt es den Fingern, zwischen den schwarzen Tasten zu spielen.
Die Daumenspitze und die anderen Fingerspitzen sollten ungefähr einen Halbkreis auf der glatten Fläche bilden.
Wenn Sie dieses mit beiden Händen nebeneinander tun, dann sollten sich die beiden Daumennägel gegenüberliegen.
Benutzen Sie den Teil des Daumens direkt unter dem Daumennagel zum Spielen, nicht das Gelenk zwischen dem Nagelglied und dem mittleren Glied.
Der Daumen ist ohnehin zu kurz; spielen Sie deshalb mit seinem vorderen Teil, damit Sie mit allen Fingern möglichst gleichmäßig spielen.
Bei den anderen Fingern liegt der Knochen an den Fingerspitzen nah an der Haut.
An der Unterseite der Finger (gegenüber dem Nagel) ist das Fleisch dicker.
Dieses Polster sollte die Tasten berühren, nicht die Fingerspitze.

Das ist die Ausgangsposition.
Wenn Sie erst begonnen haben zu spielen, müssen Sie Ihre Finger eventuell fast vollständig strecken oder sie mehr krümmen, je nachdem, was Sie gerade spielen.
\textbf{Obwohl der Anfänger die ideale gekrümmte Haltung lernen muss, ist ein striktes Beibehalten der gekrümmten Haltung deshalb nicht richtig; das werden wir \hyperref[c1iii4b]{später detailliert besprechen}, insbesondere weil die gekrümmte Haltung bedeutende Nachteile hat.}


\subsection{Höhe der Sitzbank und ihr Abstand zum Klavier}
\label{c1ii3}

Die richtige Höhe der Sitzbank und ihr Abstand zum Klavier sind ebenfalls eine Frage des persönlichen Geschmacks.
Setzen Sie sich zunächst so auf die Bank, dass die Ellbogen an Ihrer Seite sind und die Unterarme geradeaus in Richtung Klavier zeigen.
\textbf{Mit den Händen in Spielposition auf den Tasten sollten die Ellbogen ein wenig unterhalb der Hände, ungefähr in Höhe der Tasten sein.}
Setzen Sie nun Ihre Hände auf die weißen Tasten - der Abstand der Sitzbank zum Klavier und Ihre Sitzposition sollten so sein, dass die Ellbogen dicht am Körper vorbeigehen, wenn Sie sie aufeinander zubewegen.
Setzen Sie sich nicht in die Mitte der Bank, sondern sitzen Sie näher zur Vorderkante, sodass Sie Ihre Füße fest auf den Boden oder die Pedale stellen können.
Die Höhe der Sitzbank und die Sitzposition sind beim Spielen lauter Akkorde am wichtigsten.
Deshalb können Sie diese Position testen, indem Sie gleichzeitig zwei Akkorde so laut Sie können auf den schwarzen Tasten spielen.
Die Akkorde sind \textit{C\#2 G\#2 C\#3} (5 2 1) für die linke Hand und \textit{C\#5 G\#5 C\#6} (1 2 5) für die rechte Hand.
Drücken Sie die Tasten mit dem vollen Gewicht Ihrer Arme und Schultern fest nieder, wobei Sie sich leicht nach vorne beugen, um einen donnernden und respekteinflößenden Klang zu erzeugen.
Vergewissern Sie sich, dass die Schultern vollkommen einbezogen sind.
Laute, eindrucksvolle Klänge können nicht durch den Einsatz der Hände und Unterarme allein erzeugt werden; die Kraft muss aus den Schultern und dem Körper kommen.
Wenn dies bequem möglich ist, sollten die Position der Bank und die Sitzposition korrekt sein.
In der Vergangenheit neigten die Lehrer dazu, ihre Schüler zu hoch sitzen zu lassen;
aus diesem Grund ist die Standardhöhe von Sitzbänken mit fester Höhe meistens einen bis zwei Zoll\footnote{2,5 - 5 cm} zu hoch, was den Schüler dazu zwingt, mehr mit den Fingerspitzen als mit den vorderen Fingerpolstern zu spielen.
Es ist deshalb wichtig, eine Bank mit verstellbarer Höhe zu haben.


\subsection{Ein neues Stück - Anhören und analysieren (\enquote{Für Elise})}
\label{c1ii4}

\textbf{Die beste Möglichkeit, mit dem Lernprozess zu beginnen, ist, sich eine Aufführung (Aufnahme) davon anzuhören.}
Der Einwand, dass das Stück als erstes anzuhören eine Art \enquote{Betrug} sei, hat keine vertretbare Grundlage.
Der angebliche Nachteil ist, dass Schüler am Ende nur noch imitieren könnten, anstatt ihre Kreativität zu benutzen.
\textbf{Es ist jedoch unmöglich, das Spiel von jemand anderem zu imitieren, weil die Spielstile so individuell sind.}
Diese Tatsache kann für einige Schüler beruhigend sein, die sich vielleicht selbst dafür die Schuld geben, dass sie es nicht schaffen, einen berühmten Pianisten zu imitieren.
Wenn möglich, hören Sie sich mehrere Aufnahmen an.
Diese können Ihnen alle Arten von neuen Ideen und Möglichkeiten eröffnen, die zu lernen mindestens genauso wichtig ist wie die Fingertechnik.
Sich nichts anzuhören ist wie zu behaupten, man dürfe nicht zur Schule gehen, weil die Schule die Kreativität zerstören wird.
Einige Schüler glauben, dass das Anhören eine Zeitverschwendung sei, weil sie niemals so gut spielen werden.
Denken Sie in diesem Fall noch einmal darüber nach.
Wenn die hier beschriebenen Methoden nicht dazu führen würden, dass Sie \enquote{so gut} spielen werden, würde ich dieses Buch nicht schreiben!
Wenn Schüler sich viele Aufnahmen anhören, geschieht meistens folgendes: Sie entdecken, dass die Vortragsweisen nicht einheitlich gut sind; dass sie sogar \textit{ihr eigenes} Spielen gegenüber dem in einigen Aufnahmen vorziehen.

\textbf{Der nächste Schritt ist, die Struktur der Komposition zu analysieren.}
Diese Struktur wird benutzt, um das Übungsprogramm zu bestimmen und die für das Lernen des Stücks benötigte Zeit zu schätzen.
\textbf{Wie jeder erfahrene Klavierlehrer weiß, ist die Fähigkeit, die zum vollständigen Lernen eines Stücks notwendige Zeit zu schätzen, für den Erfolg des Übungsablaufs von entscheidender Bedeutung.}
Lassen Sie uns Beethovens \enquote{Für Elise} als Beispiel benutzen.
\textbf{Die Analyse beginnt immer mit dem Nummerieren der Takte auf dem Notenblatt.}
Wenn die Takte noch nicht markiert sind, markieren Sie jeden zehnten Takt mit einem Bleistift direkt über der Mitte des Takts.
Ich zähle jeden unvollständigen Takt am Anfang als Takt 1; andere zählen nur die vollständigen Takte, aber das macht es schwierig, den ersten unvollständigen Takt zu identifizieren.\footnote{In der Literatur wird in der Regel ein  unvollständiger erster Takt (= Auftakt) nicht gezählt.
Wenn der erste Takt jedoch mit Hilfe von Pausenzeichen vollständig gedruckt wird, dann wird er gezählt.
Wenn die Wiederholungen mit Voltenklammern notiert werden, dann ist der Auftakt in \enquote{Für Elise} gleichzeitig das dritte Achtel des neunten Takts und wird deshalb wieder nicht gezählt.}
In \enquote{Für Elise} werden die ersten vier vollständigen Takte im Grunde fünfzehnmal wiederholt, das heißt Sie müssen nur vier Takte lernen, um 50\% des Stücks spielen zu können (es hat 124 vollständige Takte).\footnote{Der letzte Takt ist zwei Achtel lang und kann mit dem Auftakt einen vollständigen Takt bilden, sodass man \enquote{Für Elise} wie viele Lieder mit ähnlichem Aufbau ohne die Taktart zu unterbrechen mehrmals hintereinander spielen kann.}
Weitere sechs Takte werden viermal wiederholt, sodass man nur zehn Takte lernen muss, um 70\% des Stücks zu spielen.
Wenn Sie die Methoden dieses Buchs benutzen, können Sie also 70\% dieses Stücks in weniger als 30 Minuten auswendig lernen, weil diese Takte ziemlich einfach sind.
Zwischen diesen wiederholten Takten stehen zwei Unterbrechungen, die nicht einfach sind.
Ein Schüler mit ein bis zwei Jahren Unterricht sollte in der Lage sein, die erforderlichen 50 abweichenden Takte dieses Stücks in zwei bis fünf Tagen zu lernen und fähig sein, das ganze Stück nach ein bis zwei Wochen in der richtigen Geschwindigkeit und auswendig zu spielen.
Danach kann der Lehrer anfangen, mit dem Schüler den musikalischen Inhalt des Stücks zu besprechen; wie lange das dauert, hängt von den musikalischen Kenntnissen des Schülers ab.
Wir werden nun die technischen Details der schwierigen Abschnitte besprechen.

\textbf{Das Geheimnis, die Technik schnell zu erwerben, liegt darin, bestimmte Tricks dafür zu kennen, schwierige Passagen nicht nur zu spielbaren sondern oft zu trivial einfachen zu reduzieren.}
Wir können uns nun auf die wundersame Reise in die Gehirne der Genies begeben, die unglaublich effiziente Arten herausgefunden haben, das Klavierspielen zu üben!



<!-- c1ii5.html -->

\label{c1ii5}

% zuletzt geändert 05.09.2009

\subsection{Die schwierigen Abschnitte zuerst üben}

Kehren wir zu \enquote{Für Elise} zurück.
Es gibt zwei schwierige Abschnitte mit 16 und 23 Takten.
\textbf{Fangen Sie an, das Stück zu lernen, indem Sie die schwierigsten Abschnitte zuerst üben.}
Es wird am längsten dauern, diese zu lernen, deshalb sollten Sie die meiste Übungszeit darauf verwenden.
Darum fangen wir damit an, dass wir diese beiden schwierigen Abschnitte in Angriff nehmen.
Da das Ende der meisten Stücke im Allgemeinen am schwierigsten ist, werden Sie wahrscheinlich von den meisten Stücken das Ende zuerst lernen.
 

\subsection{Schwierige Passagen kürzen - In kleinen Portionen üben (taktweise)}
\label{c1ii6}

\textbf{Ein sehr wichtiger Lerntrick ist, einen kurzen Ausschnitt für das Üben zu wählen.}
Dieser Trick hat aus vielen Gründen vielleicht die größte Auswirkung auf das Reduzieren der Übungszeit.

\textbf{Innerhalb eines schwierigen Abschnitts von sagen wir zehn Takten, gibt es typischerweise nur wenige Notenkombinationen, die Sie in die Klemme bringen.
Es ist nicht notwendig, etwas anderes als diese Noten zu üben.}
Lassen Sie uns die zwei schwierigen Abschnitte in \enquote{Für Elise} untersuchen und die schwierigsten Stellen finden.
Das können der erste Takt oder die letzten fünf Takte der ersten Unterbrechung (Takte 45 bis 56\footnote{25 bis 36, wenn die Wiederholungen mit Voltenklammern notiert werden und die Takte ohne die Wiederholungen zu berücksichtigen einfach fortlaufend durchgezählt werden}) oder das letzte Arpeggio der zweiten Unterbrechung (Takte 82 bis 105\footnote{62 bis 85}) sein.
In allen schwierigen Ausschnitten ist es von entscheidender Bedeutung, die Fingersätze zu beachten.
Bei den letzten fünf Takten der ersten Unterbrechung liegt die Schwierigkeit in der rechten Hand, wobei die Finger 1 und 5 die meiste Arbeit haben.
In Takt 52\footnote{32} (mit dem Doppelschlag) ist der Fingersatz 2321231, und in Takt 53\footnote{33} ist er 251515151525.
Benutzen Sie für das Arpeggio in der zweiten Unterbrechung den Fingersatz 1231354321 usw.
Sowohl Daumenuntersatz als auch Daumenübersatz (s. \hyperref[c1iii5]{Abschnitt III.5}) wird funktionieren, weil diese Passage nicht übermäßig schnell ist, aber ich bevorzuge den Daumenübersatz, weil der Daumenuntersatz eine Bewegung des Ellbogens erfordert und diese zusätzliche Bewegung zu Fehlern führen kann.

\textbf{Kurze Ausschnitte zu üben gestattet es Ihnen, diese  innerhalb von Minuten dutzende, ja hunderte, Male zu üben.}
Der Gebrauch dieser schnellen Wiederholungen ist der schnellste Weg, um Ihrer Hand neue Bewegungen beizubringen.
Wenn die schwierigen Noten als Teil eines längeren Abschnitts gespielt werden, kann der längere Abstand zwischen den Wiederholungen und dem Spielen von anderen Noten dazwischen die Hand durcheinander bringen und dazu führen, dass Sie langsamer lernen.
Diese höhere Lerngeschwindigkeit wird in \hyperref[c1iv5]{Abschnitt IV.5} mengenmäßig berechnet, und diese Berechnung ist die Basis für die Behauptung in diesem Buch, dass diese Methoden tausendmal schneller als die intuitiven Methoden sein können.

Wir wissen alle, dass es abträglich ist, schneller zu spielen, als es Ihre Technik erlaubt.
Jedoch, \textbf{je kürzer der Ausschnitt ist, den Sie wählen, desto schneller können Sie ihn ohne schädliche Auswirkungen üben}, weil er so viel einfacher zu spielen ist.
Deshalb können Sie die meiste Zeit \textit{mit der endgültigen Geschwindigkeit oder schneller} spielen, was der Idealzustand ist, da es so viel Zeit spart.
Mit der intuitiven Methode üben Sie hingegen die meiste Zeit mit niedriger Geschwindigkeit.
 

\subsection{Die Hände getrennt (einhändig, \hyperref[HsHt]{HS}) üben - Erlernen der Spieltechnik}
\label{c1ii7}

\textbf{Im Grunde wird die Entwicklung der Technik zu 100\% durch das getrennte Üben der Hände erreicht.}
Versuchen Sie nicht, Finger- oder Hand-Technik mit beiden Händen zusammen zu entwickeln, weil es viel schwieriger, zeitaufwendiger und \textit{gefährlicher} ist, wie später im einzelnen erklärt wird.

Wählen Sie zwei kurze Passagen, jeweils eine für die rechte Hand und eine für die linke Hand.
\textbf{Üben Sie mit der rechten Hand, bis sie anfängt müde zu werden.
Wechseln Sie dann zur linken Hand.
Wechseln Sie alle 5 bis 15 Sekunden, bevor entweder die ruhende Hand abkühlt und träge wird oder die arbeitende Hand müde wird.}
Wenn Sie die Erholungspause gerade richtig wählen, werden Sie feststellen, dass die ausgeruhte Hand förmlich darauf wartet, etwas zu tun.
Üben Sie nicht, wenn die Hand müde ist, weil das zu Stress und schlechten Angewohnheiten führt.
Wer mit dem getrennten Üben der Hände nicht vertraut ist, hat im Allgemeinen eine \hyperref[c1ii20]{schwächere linke Hand}.
Geben Sie in diesem Fall der linken Hand mehr Arbeit.
Auf diese Weise können Sie 100\% der Zeit hart üben, werden aber nie mit ermüdeten Händen üben!

Üben Sie die zwei schwierigen Abschnitte von \enquote{Für Elise} mit getrennten Händen, bis Sie die Abschnitte mit jeder einzelnen Hand zufriedenstellend schneller als mit der endgültigen Geschwindigkeit spielen können, bevor Sie die Hände zusammen nehmen.
Dies kann in Abhängigkeit Ihrer Spielstärke ein paar Tage bis einige Wochen dauern.
Sobald Sie einhändig ziemlich gut spielen können, versuchen Sie es beidhändig, um zu überprüfen, dass der Fingersatz funktioniert.

\textbf{Es sollte betont werden, dass das getrennte Üben der Hände nur für schwierige Passagen gedacht ist, die Sie nicht spielen können.}
Wenn Sie die Passage angemessen beidhändig spielen können, können Sie den einhändigen Teil selbstverständlich übergehen!
Der eigentliche Zweck dieses Buchs ist, dass Sie, wenn Sie das Klavierspielen beherrschen, schnell in der Lage sind, praktisch ohne einhändig zu üben, beidhändig zu spielen.
Der Zweck ist nicht, eine Abhängigkeit vom einhändigen Spielen zu pflegen.
Spielen Sie nur einhändig, wenn es notwendig ist, und versuchen Sie, es allmählich zu reduzieren, wenn sich Ihre Technik verbessert.
Sie werden jedoch nur in der Lage sein, mit wenig einhändigem Üben beidhändig zu spielen, nachdem Sie sehr fortgeschritten sind - die meisten Schüler werden fünf bis zehn Jahre vom einhändigen Üben abhängig sein und seinen Gebrauch nie ganz aufgeben.
Der Grund dafür ist, dass die ganze Technik am schnellsten einhändig erworben wird.
Für die Option, das einhändige Üben auszulassen, gibt es eine Ausnahme.
Das ist das Auswendiglernen; aus mehreren wichtigen Gründen (siehe \enquote{\hyperref[c1iii6]{Auswendiglernen}} in Abschnitt III.6) sollten Sie alles einhändig auswendig lernen.
Obwohl Sie vielleicht nicht einhändig üben müssen, sollten Sie deshalb einhändig auswendig lernen, außer wenn Sie ein fortgeschrittener Klavierspieler mit einem guten \hyperref[c1ii12mental]{mentalen Spielen} sind.
Solche fortgeschrittenen Themen besprechen wir später.

\textbf{Anfänger sollten alles, was sie lernen, stets einhändig üben, um diese entscheidend wichtige Methode so schnell wie möglich zu beherrschen.}
Mit dem einhändigen Üben erwerben Sie die Finger- und Handtechnik; beim nachfolgenden beidhändigen Üben müssen Sie dann nur noch lernen, die beiden Hände zu koordinieren.
Indem Sie diese Aufgaben voneinander trennen, lernen Sie beides besser und schneller.
Wenn man die einhändige Methode erst einmal beherrscht, sollte man damit experimentieren, beidhändig zu spielen ohne vorher einhändig zu spielen.
Die meisten Schüler sollten in der Lage sein, die einhändigen Methoden in zwei bis drei Jahren zu beherrschen.
\textbf{Die einhändige Methode trennt nicht bloß die Hände.
Was wir im Folgenden lernen werden, sind die Myriaden von Lerntricks, die Sie benutzen können, wenn die Hände erst getrennt sind.}

\textbf{Das getrennte Üben der Hände ist lange nachdem Sie ein Stück gelernt haben wertvoll.}
Sie können Ihre Technik einhändig viel weiter vorantreiben als beidhändig.
Und es macht viel Spaß!
Sie können Finger, Hände und Arme wirklich trainieren.
Die einhändige Methode ist allem überlegen, was \hyperref[c1iii7h]{Hanon} oder andere Übungen zur Verfügung stellen können.
Das ist der Zeitpunkt, an dem Sie \enquote{unglaubliche Arten} herausfinden können, ein Stück zu spielen.
Dabei können Sie Ihre Technik \textit{wirklich} verbessern.
Das anfängliche Lernen einer Komposition dient nur dazu, die Finger mit der Musik vertraut zu machen.
Die Menge der Zeit, die man mit dem Spielen von Stücken verbringt, die man vollständig beherrscht, unterscheidet den erfahrenen Pianisten vom Amateur.
Deshalb können erfahrene Pianisten \hyperref[c1iii14]{vorspielen}, aber die meisten Amateure können nur für sich selbst spielen.


\subsection{Die Kontinuitätsregel}
\label{c1ii8}

\textbf{Wenn Sie einen Ausschnitt üben, beziehen Sie immer den Anfang des folgenden Ausschnitts mit ein.}
Diese Kontinuitätsregel stellt sicher, dass Sie zwei aufeinanderfolgende Ausschnitte, die sie gelernt haben, auch zusammen spielen können.
Sie ist für jeden Ausschnitt anwendbar, den Sie zum Üben isolieren, wie einen Takt, einen ganzen Satz oder sogar Ausschnitte kleiner als einen Takt.
\textbf{Eine Verallgemeinerung der Kontinuitätsregel ist, dass jede Passage für das Üben in kurze Ausschnitte aufgeteilt werden kann, dass diese Ausschnitte sich aber überlappen müssen.
Die überlappende Note oder Gruppe von Noten wird Verbindung genannt.}
Wenn Sie das Ende des ersten Satzes üben, dann schließen Sie einige Takte des zweiten Satzes mit ein.
Während eines Konzerts werden Sie froh sein, dass Sie so geübt haben; es könnte Ihnen sonst passieren, dass Sie plötzlich nicht mehr wissen, wie Sie den zweiten Satz anfangen müssen!

Wir können nun die Kontinuitätsregel auf diese schwierigen Unterbrechungen in \enquote{Für Elise} anwenden.
Um Takt 53 zu üben, fügen Sie die erste Note von Takt 54 hinzu (E gespielt mit Finger 1), welche die Verbindung ist.
Da alle schwierigen Abschnitte für die rechte Hand sind, sollten Sie etwas Material für die linke, sogar aus anderen Musikstücken, zum Üben finden, um der rechten durch das Abwechseln der Hände periodische Pausen zu geben.


\subsection{Der Akkord-Anschlag}
\label{c1ii9}

Angenommen, Sie möchten mit der linken Hand ein CGEG-Quadrupel (Alberti-Begleitung) viele Male sehr schnell hintereinander spielen (wie im dritten Satz von Beethovens Mondschein-Sonate).
Die Folge, die Sie üben, ist CGEGC, wobei das letzte C die Verbindung ist.
Da die Verbindung dieselbe wie die erste Note ist, können Sie dieses Quadrupel undendlich \hyperref[c1iii2]{zirkulieren} ohne aufzuhören.
Wenn Sie es einhändig langsam üben und schrittweise schneller werden, treffen Sie auf eine \enquote{Geschwindigkeitsbarriere}, eine Geschwindigkeit, nach der alles zusammenbricht und Stress entsteht.
\textbf{Die Möglichkeit, diese Barriere zu durchbrechen, ist, das Quadrupel als einen einzigen Akkord zu spielen (CEG).
Sie sind von langsamer Geschwindigkeit zu unendlicher Geschwindigkeit übergegangen!
Das wird als Akkord-Anschlag bezeichnet.}
Nun müssen Sie nur noch lernen, langsamer zu werden, was einfacher ist als schneller zu werden, weil es keine Geschwindigkeitsbarriere gibt, wenn Sie langsamer werden.
Die Kernfrage ist: Wie wird man langsamer?

\textbf{Spielen Sie zunächst den Akkord, und lassen Sie die Hand in der Frequenz auf und ab springen\footnote{gemeint ist \enquote{wie ein Ball}}, in der das Quadrupel wiederholt wird} (sagen wir zwischen ein- und zweimal je Sekunde); das lehrt die Hand, das Handgelenk, den Arm, die Schulter usw., was sie für die schnellen Wiederholungen tun müssen und trainiert die entsprechenden Muskeln.
Beachten Sie, dass die Finger jetzt in der richtigen Position für ein schnelles Spielen sind; sie ruhen bequem auf den Tasten und sind leicht gebogen.
Senken und erhöhen Sie die Spring-Frequenz (sogar über die erforderliche Geschwindigkeit hinaus!).
Beachten Sie dabei, wie Sie die Positionen und Bewegungen des Handgelenks, des Arms, der Finger usw. verändern müssen, um die Bequemlichkeit zu maximieren und Ermüdung zu vermeiden.
Wenn Sie sich nach einer Weile müde fühlen, machen Sie entweder etwas falsch oder Sie haben sich noch nicht die Technik angeeignet, die Akkorde wiederholt zu spielen.
Üben Sie es, bis Sie spielen können, ohne zu ermüden, denn wenn Sie es nicht für einen Akkord tun können, dann werden Sie es auch nie für Quadrupel können.

Behalten Sie die Finger nahe über oder auf den Tasten, wenn Sie die Geschwindigkeit steigern.
Beziehen Sie den ganzen Körper mit ein: Schultern, Ober- und Unterarme, Handgelenk.
Das Gefühl ist, aus den Schultern und Armen heraus zu spielen, nicht den Fingerspitzen.
Wenn Sie das leise, entspannt, schnell und ohne jedes Müdigkeitsgefühl spielen können, dann haben Sie Fortschritte gemacht.
Achten Sie darauf, dass die Akkorde perfekt sind (alle Noten beginnen zur gleichen Zeit), denn ohne diese Art von Empfindlichkeit werden Sie nie die Genauigkeit haben, um schnell zu spielen.\footnote{\enquote{Digital-Pianisten} haben hier zwar einen Vorteil, weil sie ihr Spiel \hyperref[c1iii13MIDI]{aufnehmen und die MIDI-Signale ansehen} können, sollten aber trotzdem die Kontrolle durch das Gehör trainieren.}
\textbf{Es ist wichtig, langsam zu üben, weil Sie so an der Genauigkeit und der \hyperref[c1ii14]{Entspannung} arbeiten können.
Die Genauigkeit verbessert sich schneller bei den geringeren Geschwindigkeiten.}
Es ist jedoch absolut wesentlich, dass Sie zu schnelleren Geschwindigkeiten kommen (selbst wenn es nur kurz ist), bevor Sie langsamer werden.
Wenn Sie dann langsamer werden, versuchen Sie, die gleichen Bewegungen beizubehalten, die bei höherer Geschwindigkeit erforderlich waren, weil sie letzten Endes \textit{diese} Bewegungen üben müssen.



<!-- c1ii10.html -->

\subsection{Freier Fall, Akkord-Übung und Entspannung}
\label{c1ii10}

\textbf{Das Spielen von exakten Akkorden zu lernen, ist der erste Schritt in der Anwendung des \hyperref[c1ii9]{Akkord-Anschlags}.}
Lassen Sie uns den obigen CEG-Akkord mit der linken Hand üben.
Die Armgewichtsmethode ist der beste Weg, Genauigkeit und Entspannung zu erreichen; dieser Ansatz wurde ausreichend in den angegebenen Quellen behandelt (\hyperref[Fink]{Fink}, \hyperref[Sandor]{Sandor}) und wird deshalb hier nur kurz angesprochen.
Setzen Sie Ihre Finger auf die Tasten, um CEG zu spielen.
Entspannen sie Ihren Arm (eigentlich den ganzen Körper), halten Sie Ihr Handgelenk flexibel, heben Sie die Hand 5 bis 20 cm über die Tasten (am Anfang die kürzere Entfernung), und lassen Sie Ihre Hand einfach frei fallen.
Lassen Sie die Hand und die Finger als eine Einheit fallen, bewegen Sie nicht die Finger.
Entspannen Sie die Hände völlig während des Fallens, dann \enquote{platzieren} Sie die Finger und das Handgelenk im Moment des Aufpralls auf die Tasten und beugen Sie das Handgelenk ein wenig, um den Stoß des Aufpralls zu mindern und die Tasten niederzudrücken.
\textbf{Indem Sie die Schwerkraft Ihre Hand absenken lassen, überantworten Sie Ihre Stärke oder Empfindlichkeit einer konstanten Kraft.}

Es mag zunächst unglaublich erscheinen, aber ein untergewichtiger Sechsjähriger und ein gigantischer Sumoringer, die ihre Hände aus derselben Höhe fallen lassen, werden einen Ton mit der gleichen Lautstärke erzeugen, wenn sie beide den Freien Fall korrekt ausführen (was nicht einfach ist, insbesondere für den Sumoringer).
Dies geschieht, weil die Geschwindigkeit des Freien Falls unabhängig von der Masse ist und der Hammer in freien Flug übergeht, sobald die Hammernuss die Stoßzunge verlässt.
Physikstudenten werden bemerken, dass bei einem elastischen Stoß (Kollision von Billardkugeln) die kinetische Energie erhalten bleibt und das oben Gesagte nicht gilt.
Bei einem solchen elastischen Stoß würde sich die Taste mit hoher Geschwindigkeit von der Fingerkuppe wegbewegen, wie ein Golfball, der von einem Betonboden abspringt.
Hier wird aber, weil die Finger entspannt und die Fingerkuppen weich sind (inelastischer Stoß), die kinetische Energie nicht erhalten und die kleine Masse (Klaviertaste) kann bei der größeren Masse (Finger, Hand und Arm) bleiben, was zu einem kontrollierten Anschlag führt.
Deshalb gilt das oben Gesagte, vorausgesetzt, dass die Klaviermechanik korrekt eingestellt ist und die effektive Masse für den Anschlag viel kleiner ist als die Masse der Hand des Sechsjährigen.
Eine Versteifung der Hand nach dem Aufprall gewährleistet eine Übertragung des gesamten Armgewichts beim Anschlag.
Versteifen Sie die Hand nicht, bevor Sie den unteren Punkt des Tastenwegs erreicht haben, weil das eine zusätzliche Kraft hinzufügen würde - wir möchten die Tasten nur mit der Schwerkraft spielen.

Genaugenommen wird der Sumoringer wegen der Impulserhaltung einen etwas lauteren Ton erzeugen, aber der Unterschied wird trotz der Tatsache, dass sein Arm vielleicht zwanzigmal schwerer ist, gering sein.
Eine weitere Überraschung ist, dass der Anschlag mit dem Freien Fall, wenn er erst einmal richtig gelernt ist, den lautesten Ton erzeugt, den dieses Kind jemals gespielt hat (bei einem hohen Fall), und dass er eine hervorragende Art ist, junge Schüler zu lehren, wie man fest spielt.
Fangen Sie bei jungen Schülern mit kurzen Fällen an, weil am Anfang ein wirklicher Freier Fall schmerzhaft sein kann, wenn die Höhe zu groß ist.
Für einen erfolgreichen Freien Fall ist es wichtig, insbesondere bei jungen Schülern, ihnen beizubringen, dass sie so tun, als ob kein Klavier da sei und die Hand durch die Tastatur hindurch fallen soll (aber durch das Klavier gestoppt wird).
Sonst werden die meisten jungen Schüler unbewusst die Hand anheben, wenn sie auf dem Klavier landet.
Mit anderen Worten: Der Freie Fall ist eine konstante Beschleunigung, und die Hand beschleunigt sogar noch während die Tasten gedrückt werden.
Am Ende ruht die Hand mit ihrem eigenen Gewicht auf den Tasten - diese Spielweise erzeugt einen angenehmen, tiefen Klang.
Beachten Sie, dass es für diesen Anschlag wichtig ist, den ganzen Weg abwärts zu beschleunigen - s. \hyperref[c1iii1]{Abschnitt III.1} über das Erzeugen eines guten Klangs.

Die bekannte \enquote{Beschleunigte Mechanik} von Steinway funktioniert, weil sie der Hammerbewegung durch eine abgerundete Stütze unter der Buchse in der Tastenmitte eine Beschleunigung hinzufügt.
Das verschiebt den Drehpunkt mit dem Niederdrücken der Taste nach vorne, verkürzt somit die vordere und verlängert die hintere Seite der Taste und bewirkt dadurch eine Beschleunigung der Pilote bei einem konstanten Niederdrücken.
Dies veranschaulicht die Bedeutung, die Klavierdesigner der Beschleunigung des Tastendrucks beimessen, und die Armgewichtsmethode stellt sicher, dass wir den vollen Nutzen aus der Gravitationsbeschleunigung ziehen, um einen guten Klang zu erzeugen.
Die Wirksamkeit der Beschleunigten Mechanik wird kontrovers diskutiert, weil es exzellente Klaviere ohne dieses Merkmal gibt.
Offensichtlich ist für den Klavierspieler wichtiger, die Beschleunigung zu kontrollieren als vom Klavier abhängig zu sein.

Die Finger müssen \enquote{gesetzt} werden, nachdem die Tasten den unteren Punkt erreichen, um die Abwärtsbewegung der Hand zu stoppen.
Dies erfordert eine kurze Kraftanwendung auf die Finger.
Lassen Sie diese Kraft weg und entspannen Sie völlig, sobald die Hand stoppt, sodass Sie fühlen können, wie die Schwerkraft Ihren Arm nach unten zieht.
Lassen Sie die Hand auf den Tasten ruhen, sodass nur die Schwerkraft die Tasten unten hält.
Sie haben soeben erreicht, dass Sie die Tasten mit der geringstmöglichen Anstrengung niederdrücken; das ist das Wesentliche der Entspannung.
\textbf{Beachten Sie, dass es ein wichtiges Element der Entspannung ist, alle Muskeln sofort zu entspannen, sobald der Freie Fall vorüber ist.}

Anfänger werden die Akkorde mit zu vielen unnötigen Kräften spielen, die nicht genau kontrolliert werden können.
\textbf{Die Benutzung der Schwerkraft kann alle unnötigen Kräfte oder Spannungen eliminieren.}
Es mag wie ein merkwürdiger Zufall erscheinen, dass die Schwerkraft die richtige Kraft ist, um Klavier zu spielen.
\textit{Das ist kein Zufall.}
\textbf{Die Menschen haben sich unter dem Einfluss der Schwerkraft entwickelt.
Unsere Kräfte zum Gehen, Heben usw. entwickelten sich, um \textit{genau} zur Schwerkraft zu passen.}
Das Klavier wurde natürlich so \textit{entworfen}, dass es zu diesen Kräften passt.
Wenn Sie wirklich entspannt sind, können Sie die Wirkung der Schwerkraft auf Ihre Hände richtig fühlen, wenn Sie spielen.
Einige Lehrer werden die Entspannung bis zu dem Punkt, an dem alles andere vernachlässigt wird, betonen, bis die \enquote{völlige} Entspannung erreicht ist; das könnte zu weit gehen - in der Lage zu sein, die Schwerkraft zu fühlen, ist ein notwendiges und ausreichendes Kriterium für die Entspannung.
\textbf{Der Freie Fall ist eine Methode, um Entspannung zu üben.
Wenn dieser entspannte Zustand erst einmal erreicht ist, muss er ein permanenter, integraler Bestandteil Ihres Klavierspiels werden.}
Eine völlige Entspannung bedeutet nicht, dass Sie zum Klavierspielen immer nur die Schwerkraft benutzen sollen.
Die meiste Zeit werden Sie Ihre eigene Kraft anwenden; \enquote{die Schwerkraft zu fühlen} ist nur eine Möglichkeit, den Grad Ihrer Entspannung zu messen.


\subsection{Parallele Sets}
\label{c1ii11}

Nun, da der CEG-Akkord mit der linken Hand zufriedenstellend ist, versuchen Sie, plötzlich vom Akkord zum Quadrupel zu wechseln.
Sie werden nun die Finger bewegen müssen, beschränken Sie die Fingerbewegungen aber auf ein Minimum.
Damit das Wechseln klappt, bauen Sie die richtigen Hand- und Armbewegungen ein (siehe \hyperref[Fink]{Fink}, \hyperref[Sandor]{Sandor}), die wir später besprechen, aber das ist ein Thema für Fortgeschrittene, lassen Sie uns deshalb einen Schritt kürzer treten und annehmen, dass Ihnen das Wechseln nicht gelingt, damit wir eine mächtige Methode zum Lösen dieses Problemtyps zeigen können.

\textbf{Die grundlegendste Art zu lernen, wie man eine schwierige Passage spielt, ist, sie mit jeweils zwei Noten aufzubauen und dabei den \hyperref[c1ii9]{Akkord-Anschlag} zu benutzen.} In unserem CGEG-Beispiel (der linken Hand) fangen wir mit den ersten beiden Noten an.
Ein zweinotiger Akkord-Anschlag (strenggenommen ein Intervall-Anschlag)!
Spielen Sie diese zwei Noten als perfektes Intervall; lassen Sie Ihre Hand und Finger (5 und 1) auf und ab springen, wie Sie es bereits beim CEG-Akkord getan haben.
Um diese zwei Noten schnell nacheinander zu spielen, senken Sie beide Finger zusammen, aber halten Sie Finger 1 etwas oberhalb von 5, sodass die 5 zuerst landet.
Es ist nur ein schnelles zweinotiges rollendes Intervall.
Da Sie beide Finger gleichzeitig nach unten bringen und nur einen leicht verzögern, können Sie sie so kurz hintereinander spielen wie Sie möchten, indem Sie die Verzögerung verringern.
\textbf{So verlangsamt man von unendlicher Geschwindigkeit!}

Ist es auf diese Art möglich, jede Kombination von Noten unendlich schnell zu spielen? Natürlich nicht.
Wie wissen wir, welche wir unendlich schnell spielen können und welche nicht?
Um diese Frage zu beantworten, müssen wir das Konzept des parallelen Spielens einführen.
Die obige Methode, die Finger zusammen zu senken, wird paralleles Spielen genannt, weil die Finger gleichzeitig gesenkt werden, also parallel.
\textbf{Ein paralleles Set ist eine Gruppe von Noten, die gleichzeitig mit einer Hand gespielt werden können.
Alle parallelen Sets können unendlich schnell gespielt werden - beim \hyperref[c1ii9]{Akkord-Anschlag} werden parallele Sets benutzt.
Die Verzögerung zwischen aufeinander folgenden Fingern wird Phasenwinkel genannt.}
In einem Akkord ist der Phasenwinkel für alle Finger Null; eine detaillierte Behandlung der parallelen Sets finden Sie in \hyperref[c1iii7b2]{Übung 2 in Abschnitt III.7b}.
Das ist einfach ein Akkord-Anschlag, aber der Begriff \enquote{parallele Sets} ist notwendig, um Irrtümer zu vermeiden, die aus der Tatsache resultieren, dass in der Musiktheorie \enquote{Akkord} und \enquote{Intervall} bestimmte Bedeutungen haben, die nicht auf alle parallelen Sets anwendbar sind.
Die höchste Geschwindigkeit der parallelen Sets wird durch die Reduzierung der Phase auf den kleinsten kontrollierbaren Wert erreicht.
Dieser Wert ist ungefähr gleich dem Fehler in Ihrem Akkordspiel.
Mit anderen Worten: Je genauer Ihre Akkorde sind, desto schneller wird Ihre maximal erreichbare Geschwindigkeit sein.
Deshalb wurde dem Üben perfekter Akkorde oben so viel Platz gewidmet.

Haben Sie erst einmal das CG gemeistert, können Sie mit dem nächsten - GE (1 3) - fortfahren, dann EG und schließlich das GC, um das Quadrupel und die Verbindung zu vervollständigen.
Verbinden Sie sie dann paarweise, CGE usw., um das Quadrupel zu vervollständigen.
Beachten Sie, dass CGE (5 1 3) ebenfalls ein paralleles Set ist.
Deshalb kann das Quadrupel plus die Verbindung aus den parallelen Sets (5 1 3) und (3 1 5) gebildet werden.
In diesem Schema ist 3 die Verbindung.
Das ist schneller, als zweinotige parallele Sets zu benutzen, aber schwieriger.
Die allgemeine Regel für die Anwendung der parallelen Sets ist: \textbf{Konstruieren Sie das Übungssegment, indem Sie die größtmöglichen parallelen Sets benutzen, die zum Fingersatz passen.}
Unterteilen Sie diese nur in kleinere parallele Sets, wenn das große parallele Set zu schwierig ist.
\hyperref[c1iii7]{Abschnitt III.7} behandelt die Einzelheiten zum Gebrauch der parallelen Sets.

Nachdem Sie ein Quadrupel gut spielen können, üben Sie, zwei hintereinander zu spielen, dann drei usw.
Schließlich werden Sie in der Lage sein, so viele hintereinander zu spielen wie Sie möchten!
Wenn Sie am Anfang den Akkord \enquote{gesprungen} haben, hat sich die Hand auf und ab bewegt.
Aber am Schluss, wenn Sie die Quadrupel in schneller Folge spielen, ist die Hand ziemlich stationär.
Sie werden auch Handbewegungen usw. hinzufügen müssen - dazu später mehr.

Der zweite schwierige Abschnitt in \enquote{Für Elise} endet mit einem Arpeggio, das drei parallele Sets enthält: 123, 135 und 432.
Üben Sie zunächst jedes parallele Set einzeln (zum Beispiel 123), fügen Sie dann die Verbindung (1231) hinzu, verbinden Sie sie dann paarweise (123135) usw., um das Arpeggio aufzubauen.

Damit jeder geübte Ausschnitt flüssig und musikalisch klingt, \textbf{müssen wir zwei Dinge vollbringen:}

\begin{enumerate}[label={\arabic*.}] 
\item \textbf{die Phasenwinkel genau kontrollieren (Unabhängigkeit der Finger), und}
\item \textbf{die parallelen Sets flüssig verbinden.}
 \end{enumerate}

Die meisten der in den Quellen beschriebenen Finger-, Hand- und Armbewegungen zielen darauf ab, diese beiden Aufgaben auf geschickte Art zu bewältigen.
Wir werden viele dieser Themen in \hyperref[c1iii1]{Abschnitt III} behandeln.
Die Quellen sind nützliche Begleiter dieses Buchs.
Um Ihnen bei der Entscheidung zu helfen, welche der Quellen Sie benutzen sollten, habe ich im \hyperref[reference]{Quellenverzeichnis} viele davon (sehr kurz) beschrieben.

Sie werden den größten Teil von \hyperref[c1iii1]{Abschnitt III} lesen müssen, damit Sie wissen, wie man die parallelen Sets am effektivsten benutzt.
Das oben beschriebene parallele Spielen nennt man \enquote{phasengekoppeltes} paralleles Spielen und ist der einfachste Weg anzufangen aber nicht das endgültige Ziel.
Um sich die Technik anzueignen, brauchen sie eine vollständige Unabhängigkeit der Finger, die mit dem Üben kommt, keine phasengekoppelten Finger.
\textbf{Parallele Sets bewirken zweierlei: Sie lehren Ihr Gehirn das Konzept des extrem schnellen Spielens und geben den Händen eine Vorstellung davon, wie sich das schnelle Spielen anfühlt.}
Für diejenigen, die noch nicht derart schnell gespielt haben, sind das völlig neue und erstaunliche Erfahrungen.
Mit dem parallelen Spielen erreichen Sie die vorgesehene Geschwindigkeit, sodass Sie mit verschiedenen Bewegungen experimentieren können, um herauszufinden, welche funktionieren.
Weil diese Methoden Ihnen hunderte von Versuchen innerhalb von Minuten gestatten, können diese Experimente rasch ausgeführt werden.
 


<!-- c1ii12.html -->

\label{c1ii12}

% zuletzt geändert 20.09.2009

\subsection{Lernen, Auswendiglernen und mentales Spielen}

\textbf{Es gibt keinen schnelleren Weg auswendig zu lernen, als es gleich zu tun, wenn Sie ein Stück das erste Mal lernen, und für ein schwieriges Stück gibt es keinen schnelleren Weg, es zu lernen, als es auswendig zu lernen.}
Beginnen Sie das Auswendiglernen, indem Sie lernen, wie die Musik klingen sollte: Melodie, Rhythmus usw.
Benutzen Sie dann die Notenblätter, um für jede Note die zugehörige Taste zu finden und sich zu merken; das nennt man \hyperref[c1iii6tastatur]{Tastatur-Gedächtnis} - Sie merken sich, wie Sie dieses Stück auf dem Klavier spielen, mit dem Fingersatz, den Handbewegungen usw.
Einige Klavierspieler benutzen das \hyperref[c1iii6foto]{fotografische Gedächtnis} bei dem Sie sich das Notenblatt als komplettes Bild merken.
Wenn man ein Notenblatt nehmen und versuchen sollte, sich jede einzelne Note zu merken, wäre das - sogar für einen Konzertpianisten - unsagbar schwer.
Wenn man jedoch die Musik kennt (Melodie, Akkordstruktur usw.), wird es für jeden einfach!
Das wird in \hyperref[c1iii6]{Abschnitt III.6} erklärt, in dem das Auswendiglernen detaillierter behandelt wird.
Ich ziehe das Tastatur-Gedächtnis dem fotografischen Gedächtnis vor, weil es dabei hilft, die Noten auf dem Klavier zu finden, ohne dass man in Gedanken das Notenblatt \enquote{lesen} muss.
Prägen Sie sich deshalb jeden Abschnitt ein, den Sie für die Technik üben, während Sie diesen viele Male \hyperref[c1ii7]{mit getrennten Händen} in kleinen Segmenten wiederholen.
\textbf{Die Prozeduren für das Einprägen sind im Grunde mit denen für das Aneignen der Technik identisch.}
Zum Beispiel sollte das Einprägen zunächst mit getrennten Händen erfolgen, für \hyperref[c1ii5]{die schwierigen Abschnitte zuerst} usw.
Wenn sie erst später auswendig lernen, müssen Sie die gleiche Prozedur noch einmal ausführen.
Es mag einfacher erscheinen, ein zweites Mal durch die gleiche Prozedur zu gehen.
Ist es aber nicht.
Auswendiglernen ist ein komplexer Vorgang (sogar nachdem Sie das Stück gut spielen können); Schüler, die versuchen, ein Stück nach dem Lernen auswendig zu lernen, geben aus diesem Grund entweder auf oder sie lernen es niemals völlig auswendig.
Das ist verständlich; der zum Einprägen erforderliche Aufwand kann schnell den Punkt abnehmender Ertragszuwächse erreichen, wenn man das Stück bereits spielen kann.

Zwei wichtige Punkte, die Sie auswendig lernen müssen, sind die Taktart (siehe \hyperref[c1iii1b]{III.1b}) und die Vorzeichnung (siehe \hyperref[c1iii5d]{III.5d}).
Die Taktart ist leicht zu verstehen und hilft Ihnen dabei, mit dem korrekten Rhythmus zu spielen.
Die Vorzeichnung (wie viele Kreuze und Be's) ist komplexer, weil sie Ihnen nicht die genaue Tonart (C-Dur usw.) verrät, in der das Stück steht.
Wenn Sie wissen, ob die Komposition in einer Dur- oder Molltonart steht, dann gibt Ihnen die Vorzeichnung die Tonart; wenn die Vorzeichnung zum Beispiel keine Kreuze und Be's hat (wie in \enquote{Für Elise}), dann  ist es entweder C-Dur oder a-Moll (siehe III.5d).
Die meisten Schüler kennen die Dur-Tonleitern; Sie werden mehr über die Theorie wissen müssen, um die Moll-Tonleitern herauszufinden;
deshalb sollten nur diejenigen mit genügenden theoretischen Kenntnissen sich die Tonart merken.
Wenn Sie nicht sicher sind, lernen Sie nur die Vorzeichnung.
Diese Tonart ist die Tonika der Musik, um die herum der Komponist Akkordprogressionen dazu benutzt, die Tonart zu ändern.
Die meisten Kompositionen beginnen und enden mit der Tonika, und die Akkorde schreiten in der Regel entlang des Quintenzirkels fort (siehe Kapitel 2, Abschnitt 2b). Bis jetzt wissen wir, dass \enquote{Für Elise} entweder in C-Dur oder a-Moll steht.
Da es etwas melancholisch ist, vermuten wir eine Moll-Tonart.
Die ersten zwei Takte sind wie eine Fanfare, die das erste Thema einführt, der Hauptteil des Themas beginnt in Takt 3, der A ist, die Tonika von a-Moll!
Zudem ist der letzte Akkord ebenfalls die Tonika von a-Moll.
Wir sind deshalb fast sicher, dass es in a-Moll steht.
Das einzige Vorzeichen in a-Moll ist G\# (siehe \hyperref[tablemoll]{Tabelle der Moll-Tonleitern}), das wir in Takt 4 finden; daraus schließen wir, dass es in a-Moll steht.
Wenn Sie diese Details verstehen, dann können Sie \textit{wirklich} gut auswendig lernen.

Kehren wir zur Taktart zurück, die 3/8 ist: drei Schläge je Takt und ein Achtel je Schlag.
Somit ist es im Format eines Walzers\footnote{fast, weil der Walzer im 3/4-Takt ist}, aber musikalisch sollte es nicht wie ein Tanz gespielt werden, sondern viel sanfter, weil es melancholisch und eindringlich romantisch ist.
Die Taktart sagt uns, dass Takte wie Takt 3 nicht wie zwei Triolen gespielt werden dürfen, weil es drei Schläge gibt.
Man muss den Akzent des ersten Schlags eines jeden Takts jedoch nicht überbetonen wie in einem Wiener Walzer.
Die Taktart ist für das musikalische und korrekte Spielen eindeutig  nützlich.
Ohne die Taktart können sie sich schnell einen falschen Rhythmus angewöhnen, der Ihr Spielen für Experten amateurhaft klingen lässt.

\textbf{Haben Schüler erst einmal die für sie passenden Abläufe zum Lernen und Auswendiglernen entwickelt, werden die meisten von ihnen der Meinung sein, dass gleichzeitiges Lernen und Auswendiglernen für schwierige Passagen weniger Zeit benötigt als das Lernen alleine.}
Das geschieht, weil man den Vorgang eliminiert, auf die Noten zu schauen, sie zu interpretieren und die Befehle von den Augen zum Gehirn und danach zu den Händen zu geben.
Material, das auswendig gelernt wurde, solange man jung ist (ungefähr bevor man 20 Jahre alt wird), wird fast nie vergessen.
Deshalb ist es so wichtig, schnelle Methoden für das Aneignen der Technik zu lernen und so viele Stücke wie möglich auswendig zu lernen, bevor man das späte Teenageralter erreicht.
Es ist einfacher, etwas auswendig zu lernen, wenn man es schnell spielen kann; machen Sie sich deshalb keine Sorgen, wenn Sie am Anfang Schwierigkeiten haben, etwas bei langsamer Geschwindigkeit auswendig zu lernen; es wird einfacher, wenn Sie schneller werden.


\label{c1ii12mental}

\textbf{Die einzige Möglichkeit, gut auswendig zu lernen, ist, das \hyperref[c1iii6tastatur]{mentale Spielen} lernen.}
Tatsächlich ist das mentale Spielen das logische und endgütige Ziel aller Übungsmethoden, die wir besprechen, weil Technik alleine Sie nicht in die Lage versetzt, fehlerfrei, musikalisch und ohne nervös zu werden vorzuspielen.
Lesen Sie \hyperref[c1iii6j]{Abschnitt III.6j} für mehr Details über das mentale Spielen.
Beim mentalen Spielen lernen Sie, das Klavier in Gedanken - ohne Klavier - zu spielen, einschließlich des richtigen Fingersatzes und Ihrer Vorstellung davon, wie die Musik klingen soll.
Sie können das \hyperref[c1iii6tastatur]{Tastatur-Gedächtnis} oder das \hyperref[c1iii6foto]{fotografische Gedächtnis} für das mentale Spielen benutzen, aber ich empfehle für Anfänger das Tastatur-Gedächtnis, weil es effizienter ist;
für fortgeschrittene Spieler sind Tastatur-Gedächtnis und  fotografisches Gedächtnis dasselbe, denn wenn man das eine beherrscht, kommt das andere wie von selbst.
Wann immer Sie einen kleinen Abschnitt auswendig lernen, schließen Sie die Augen, und prüfen Sie, ob sie ihn in Gedanken spielen können, ohne ihn auf dem Klavier zu spielen.
Haben Sie ein ganzes Stück mit getrennten Händen auswendig gelernt, sollten Sie es auch mit getrennten Händen in Ihrem Kopf spielen können.
Das ist der Zeitpunkt, die Struktur des Stücks zu analysieren, wie es aufgebaut ist und wie die Themen sich mit dem Fortgang der Musik entwickeln.
Wenn Sie geübt sind, werden Sie feststellen, dass es nur eine geringe Investition an Zeit erfordert, sich das mentale Spielen anzueignen.
Das Beste ist: Sie werden auch entdecken, dass Ihr Gedächtnis mit dem Aufbau eines soliden mentalen Spielens so gut wie nur irgend möglich wird; Sie werden darauf vertrauen, dass Sie in der Lage sind, ohne Fehler, Gedächtnisblockaden usw. zu spielen, und Sie werden sich auf die Musik konzentrieren können.
Mentales Spielen hilft auch der Technik; es ist zum Beispiel viel einfacher, mit hoher Geschwindigkeit zu spielen, wenn Sie mit dieser Geschwindigkeit in Gedanken spielen können; die Unfähigkeit schnell zu spielen hat ihren Ursprung sehr oft im Gehirn.
Ein Vorteil des mentalen Spielens ist, dass Sie es jederzeit und überall üben und ihre effektive Übungszeit in hohem Maß steigern können.

\textbf{Das Gedächtnis ist ein assoziativer Prozess.
Gedächtniskünstler (einschließlich einiger Savants) und alle Konzertpianisten, die Stunden von Musik auswendig lernen können, hängen von Algorithmen ab, mit denen Sie das Gespeicherte assoziieren können (egal, ob sie es wissen oder nicht).}
Musiker haben in dieser Hinsicht besonderes Glück, weil Musik gerade ein solcher Algorithmus ist.
Trotzdem wird dieser \enquote{Gedächtnistrick}, die Musik als Algorithmus für das Auswendiglernen zu benutzen, Musikschülern selten formal gelehrt; statt dessen wird ihnen oft geraten, stets zu wiederholen, \enquote{bis die Musik in den Händen ist}, was eine der schlechtesten Gedächtnis-Methoden ist, denn wie wir in \hyperref[c1iii6d]{Abschnitt III.6d} sehen werden, führt Wiederholung zum \enquote{\hyperref[c1iii6d]{Hand-Gedächtnis}}, was eine falsche Art von Gedächtnis ist, die zu vielen Problemen, wie Gedächtnisblockaden, führen kann.
Beim mentalen Spielen assoziieren Sie die Musik in Gedanken damit, wie Sie sie am Klavier erzeugen.
Es ist wichtig, das mentale Spielen zu üben, ohne am Klavier zu spielen, weil sie ein \enquote{Klang-Gedächtnis} (so wie ein \enquote{Hand-Gedächtnis}) erwerben und den Klang des Klaviers als Stütze für das Abrufen benutzen können, und das Klang-Gedächtnis kann dieselben Probleme verursachen, die mit dem Hand-Gedächtnis verbunden sind.

Warum sind das Gedächtnis und das mentale Spielen so wichtig?
Sie lösen nicht nur die praktischen Probleme der Technik und des \hyperref[c1iii14]{Auftretens}, sondern bringen Sie auch als Musiker voran und steigern die Intelligenz.
So wie man einen Computer beschleunigen kann, indem man Speicher hinzufügt, so kann man seine effektive Intelligenz steigern, indem man das Gedächtnis verbessert.
Tatsächlich ist Gedächtnisverlust eines der ersten Zeichen eines geistigen Verfalls, zum Beispiel bei Alzheimer.
\textbf{Es ist nun klar, dass viele dieser \enquote{erstaunlichen Kunststücke} großer Musiker wie Mozart einfach Nebenprodukte eines starken mentalen Spielens waren, und dass solche Fertigkeiten erlernt werden können.}


\subsection{Spielgeschwindigkeit beim Üben}
\label{c1ii13}

\textbf{Kommen Sie so schnell wie möglich auf Geschwindigkeit.} Erinnern Sie sich daran, dass wir immer noch \hyperref[c1ii7]{mit getrennten Händen} üben.
So schnell zu spielen, dass man anfängt Stress zu empfinden und Fehler zu machen, verbessert die Technik nicht, weil man nur die Fehler übt und sich schlechte Angewohnheiten aneignet.
Die Finger zu zwingen, auf dieselbe Art schneller zu spielen, ist nicht der Weg, die Geschwindigkeit zu erhöhen.
Wie beim \hyperref[c1ii11]{parallelen Spielen} gezeigt wurde, brauchen Sie neue Arten zu spielen, die automatisch die Geschwindigkeit erhöhen und den Stress reduzieren.
Beim parallelen Spielen ist es oft sogar einfacher, schnell als langsam zu spielen.
Erarbeiten Sie Handpositionen und -bewegungen, die automatisch die Geschwindigkeit erhöhen.
Diese Themen sind die Hauptbeiträge dieses Buchs und werden später im Einzelnen behandelt, da sie zu umfangreich sind, um hier schon behandelt zu werden; dazu gehören so spezifische Fertigkeiten, wie der \hyperref[c1iii5b]{Daumenübersatz}, die \hyperref[c1iii5c]{Glissandobewegung}, die \hyperref[c1ii14]{Entspannung}, die \hyperref[c1iii4b]{flachen Fingerhaltungen}, die Bewegungen der Arme und Handgelenke sowie die Benutzung der \hyperref[c1ii15]{automatischen Verbesserung nach dem Üben}.
Wenn Sie innerhalb weniger Minuten keinen bedeutenden Fortschritt erzielen, machen Sie wahrscheinlich etwas falsch - denken Sie sich etwas Neues aus.
Schüler, die die intuitive Methode benutzen, haben sich damit abgefunden, dieselbe Sache stundenlang mit geringer sichtbarer Verbesserung zu wiederholen.
Diese Mentalität muss vermieden werden, damit man schneller lernt.
Wenn man die Geschwindigkeit erhöht, kann man in zwei Arten von Situationen kommen.
Die eine betrifft die technischen Fertigkeiten, die Sie bereits besitzen; Sie sollten in der Lage sein, diese innerhalb von Minuten auf Geschwindigkeit zu bringen.
Die andere betrifft neue Fertigkeiten; diese werden mehr Zeit benötigen und werden weiter unten in \hyperref[c1ii15]{Abschnitt 15} besprochen.

\textbf{Die Technik verbessert sich am schnellsten, wenn man mit einer Geschwindigkeit spielt, bei der man exakt spielen kann.}
Das stimmt insbesondere wenn man beidhändig spielt (bitte gedulden Sie sich - ich verspreche Ihnen, dass wir noch zum beidhändigen Üben kommen).
Da Sie einhändig mehr Kontrolle haben, kommen Sie einhändig zu weitaus schnellerem Spiel als beidhändig, ohne den Stress zu vergrößern oder sich schlechte Angewohnheiten anzueignen.
Somit ist es falsch, zu denken, man könne schneller Fortschritte erzielen, indem man so schnell wie möglich spielt (schließlich kann man dieselbe Passage zweimal so oft spielen, wenn man doppelt so schnell spielt!).
Da eines der Hauptziele des einhändigen Übens das Gewinnen von Geschwindigkeit ist, kommen die Notwendigkeit schnell Geschwindigkeit zu erreichen und das exakte Üben miteinander in Konflikt.
Die Lösung ist, die Geschwindigkeit beim Üben ständig zu ändern; bleiben Sie nicht zu lange bei einer Geschwindigkeit.
Es gibt für sehr schwierige Passagen, die Fertigkeiten erfordern, die Sie noch nicht besitzen, keine Alternative für das stufenweise Erhöhen der Geschwindigkeit.
Benutzen Sie dazu versuchsweise zu hohe Geschwindigkeiten, um herauszufinden, was geändert werden muss, damit Sie mit solchen Geschwindigkeiten spielen können.
Werden Sie dann langsamer, und üben Sie diese neuen Bewegungen.

Um die Geschwindigkeit zu variieren, gehen Sie zunächst zu einer handhabbaren \enquote{Maximalgeschwindigkeit}, bei der Sie exakt spielen können.
Werden Sie dann schneller (indem Sie, wenn notwendig, \hyperref[c1ii11]{parallele Sets} usw. benutzen), und achten Sie darauf, wie das Spielen geändert werden muss (machen Sie sich nichts daraus, wenn Sie an diesem Punkt nicht exakt spielen, da Sie es nicht viele Male wiederholen).
Benutzen Sie dann diese Bewegung und spielen Sie mit der vorhergehenden \enquote{exakten Maximalgeschwindigkeit}.
Es sollte nun spürbar einfacher sein.
Üben Sie eine Weile mit dieser Geschwindigkeit, versuchen Sie dann langsamere Geschwindigkeiten, um sicherzustellen, dass Sie völlig entspannt sind und exakt spielen.
Wiederholen Sie dann die ganze Prozedur.
Auf diese Art schrauben Sie die Geschwindigkeit in gut zu bewältigenden Schritten hoch und arbeiten an jeder benötigten Fähigkeit gesondert.
In den meisten Fällen sollten Sie in der Lage sein, das meiste des neuen Stücks - zumindest in kleinen Segmenten und einhändig - während der ersten Sitzung in der endgültigen Geschwindigkeit zu spielen.
Am Anfang mag es unmöglich erscheinen, die endgültige Geschwindigkeit während der ersten Sitzung zu erreichen, aber mit Übung kann jeder Schüler dieses Ziel erstaunlich schnell erreichen.
 

\subsection{Wie man entspannt}
\label{c1ii14}

\textbf{Das Wichtigste zum Erreichen der vorgegebenen Geschwindigkeit ist, zu entspannen.}
Entspannen bedeutet, dass man nur die Muskeln benutzt, die zum Spielen benötigt werden.
Dadurch kann man so hart arbeiten wie man möchte und entspannt sein.
Der entspannte Zustand ist beim einhändigen Üben besonders leicht zu erreichen.
Es gibt zwei Denkschulen zur Entspannung.
Eine Schule behauptet, dass es auf lange Sicht besser sei, nicht zu üben als mit dem leichtesten Anflug von Spannung zu üben.
Diese Schule unterrichtet, indem sie zeigt, wie man eine Note entspannt spielt, dann vorsichtig weitergeht und nur das leichte Material präsentiert, das man entspannt spielen kann.
Die andere Schule argumentiert, dass Entspannung sicherlich ein notwendiger Aspekt der Technik sei, aber dass es nicht der optimale Ansatz ist, die ganze Übungsphilosophie der Entspannung unterzuordnen.
Der zweite Ansatz sollte der bessere sein, vorausgesetzt, Ihnen sind die Fallen bewusst.

Das menschliche Gehirn kann ziemlich verschwenderisch sein.
Sogar für die einfachsten Aufgaben benutzt das untrainierte Gehirn die meisten Muskeln des Körpers.
Und wenn die Aufgabe schwierig ist, neigt das Gehirn dazu, den ganzen Körper in einer Masse angespannter Muskeln einzusperren.
Um zu entspannen, müssen Sie eine bewusste Anstrengung unternehmen, um alle unnötigen Muskeln abzuschalten.
Das ist nicht einfach, weil es den natürlichen Neigungen des Gehirns entgegensteht.
Sie müssen das Entspannen genauso viel üben wie das Bewegen der Finger zum Spielen der Tasten.
Entspannen bedeutet nicht, \enquote{alle Muskeln erschlaffen zu lassen}; es bedeutet, dass die nicht benötigten Muskeln sogar dann entspannt sind, wenn die notwendigen unter voller Last arbeiten.
Diese Fähigkeit zur Koordination verlangt viel Übung.

Wenn die Entspannung für Sie etwas Neues ist, können Sie mit den einfacheren Stücken, die Sie gelernt haben, anfangen und das Hinzufügen der Entspannung üben.
Die \hyperref[c1iii7b]{Übungen für parallele Sets} von Abschnitt III.7 können Ihnen helfen, das Entspannen zu üben.
Eine Möglichkeit, die Entspannung zu spüren, ist, ein paralleles Set zu üben, es zu beschleunigen bis man Stress aufbaut und dann zu versuchen zu entspannen.
Sie werden neue Bewegungen und Positionen der Arme, Handgelenke usw. finden müssen, die das erlauben; wenn Sie diese gefunden haben, werden Sie spüren, wie der Stress in der Hand während des Spielens verschwindet.

Entspannen Sie alle die unterschiedlichen Körperfunktionen, wie das Atmen und das periodische Schlucken.
Einige Schüler unterbrechen das Atmen beim Spielen anspruchsvoller Passagen, weil sie sich auf das Spielen konzentrieren.
Wenn Sie entspannt sind, sollten Sie in der Lage sein, alle normalen Körperfunktionen auszuführen und sich trotzdem gleichzeitig auf das Spielen zu konzentrieren.
\hyperref[c1ii21]{Abschnitt 21} weiter unten erklärt, wie man das Zwerchfell für die richtige Atmung benutzt.
Wenn Ihre Kehle nach schwerem Üben trocken ist, haben Sie das Schlucken vergessen.
Das alles sind Anzeichen von Stress.

Viele Schüler, denen das Entspannen nicht gelehrt wurde, glauben, dass langes wiederholtes Üben irgendwie die Hand so verwandelt, dass sie spielen kann.
In Wahrheit ist es oft so, dass die Hand zufällig über die richtige Bewegung für die Entspannung stolpert.
Deshalb werden manche Fähigkeiten schnell erworben, während andere ewig brauchen, und deshalb erwerben manche Schüler bestimmte Fähigkeiten schnell, während andere Schüler mit denselben Fähigkeiten kämpfen.
Entspannung ist ein Zustand des instabilen Gleichgewichts: Indem man lernt zu entspannen, wird das Spielen leichter, was das Entspannen vereinfacht usw.
Das erklärt, warum die Entspannung für manche ein größeres Problem ist, während sie für andere völlig normal ist.
Aber das ist eine der wunderbarsten Informationen - sie bedeutet, dass jeder das Entspannen lernen kann, wenn er richtig unterrichtet wird.

Entspannung ist das Einsparen von Energie.
Es gibt mindestens zwei Möglichkeiten zum Einsparen:

\begin{enumerate}[label={\arabic*.}] 
\item Benutzen Sie keine unnötigen Muskeln, insbesondere die gegensinnigen Muskeln\footnote{Antagonisten}.
\item Schalten Sie die arbeitenden Muskeln ab, sobald diese ihre Arbeit verrichtet haben.
 \end{enumerate}
Lassen Sie uns dies mit dem einfingrigen \hyperref[c1ii10]{Freien Fall} demonstrieren.
\enquote{1} ist das leichteste; erlauben Sie einfach der Schwerkraft, den Fall völlig zu kontrollieren, während der ganze Körper bequem auf der Bank ruht.
Wer angespannt ist, wird beide Muskeln zusammenziehen: diejenigen für das Heben und diejenigen für das Senken der Hand.
Für \enquote{2} müssen Sie eine neue Angewohnheit lernen, wenn Sie sie noch nicht haben (wenige haben sie am Anfang).
Das ist die Angewohnheit, alle Muskeln zu entspannen, sobald sie den unteren Punkt des Tastenwegs erreicht haben.
Während des Freien Falls lassen Sie den Arm durch die Schwerkraft nach unten ziehen, aber am Ende des Tastenwegs müssen Sie den Finger für einen Moment anspannen, um die Hand zu stoppen.
Danach müssen Sie alle Muskeln schnell entspannen.
Heben Sie nicht die Hand, lassen Sie die Hand bequem auf dem Klavier ruhen und zwar mit gerade so viel Kraft auf dem Finger, die genügt, das Gewicht des Arms zu unterstützen.
Stellen Sie sicher, dass Sie\footnote{die Tasten} nicht herunterdrücken.
Das ist schwieriger als man zunächst annimmt, weil der Ellenbogen mitten in der Luft schwebt und dieselben Muskelbündel, die benutzt werden, um die Finger für die Unterstützung des Armgewichts zu spannen, auch benutzt werden, um\footnote{die Tasten} herunterzudrücken.

Zueinander gegensinnige Muskeln gleichzeitig anzuspannen ist ein Hauptgrund der Verspannung.
Wenn der Klavierspieler es nicht merkt, kann es außer Kontrolle geraten und zu Verletzungen führen.
So wie wir lernen müssen, die einzelnen Finger der Hand unabhängig zu kontrollieren, müssen wir auch lernen, jeden der gegensinnigen Muskeln, wie Beuger und Strecker, unabhängig zu kontrollieren.
Die schlimmste Auswirkung von Stress ist, dass er Sie in einen Kampf zwingt, den Sie nicht gewinnen können, weil Sie gegen einen Gegner kämpfen, der genau so stark ist, wie Sie es sind - nämlich Sie selbst.
Es sind Ihre eigenen Muskeln, die gegen Ihren Körper arbeiten. 
Und je mehr Sie üben, um so schlimmer wird das Problem.
Wenn es schlimm genug wird, kann es zu Verletzungen führen, weil die Muskeln stärker werden als die Materialbelastbarkeit Ihres Körpers ist.

Ohne Training denken wenige Menschen daran, ihre Muskeln gezielt abzuschalten; normalerweise vergisst man sie einfach, wenn ihre Arbeit getan ist.
Wenn die Finger schnell arbeiten, müssen Sie jedoch schnell entspannen; ansonsten bekommen die Finger niemals eine Pause oder sind niemals bereit für die nächste Note.
Eine gute Übung für das schnelle Entspannen ist, mit einer gedrückten Taste anzufangen und einen schnellen, mäßig lauten Ton mit demselben Finger zu spielen.
Nun müssen Sie eine aufwärts oder abwärts gerichtete Kraft aufbringen \textit{und} den Muskel abschalten.
Wenn Sie ihn abschalten, müssen Sie zu dem Gefühl zurückkehren, das Sie am Ende eines Freien Falls hatten.
Sie werden herausfinden, dass es um so länger dauert zu entspannen, je härter Sie die Note spielen.
Üben Sie, diese Zeit zum Entspannen zu verkürzen.

Das Wunderbare an diesen Entspannungsmethoden ist, dass sie, nachdem Sie sie für eine kurze Zeit praktiziert haben (vielleicht ein paar Wochen), zunehmend von selbst in Ihr Spielen einfließen - sogar in Stücke, die Sie bereits gelernt haben -, solange Sie auf die Entspannung achten.
\textbf{Entspannung (den ganzen Körper einbeziehen), Armgewicht (Freier Fall) und die Vermeidung von stupiden, wiederholenden Übungen waren Schlüsselelemente in Chopins Lehren.}
Entspannung ist nutzlos, solange sie nicht von musikalischem Spielen begleitet wird; Chopin bestand sogar auf musikalischem Spielen vor dem Erwerben von Technik, weil er wusste, dass Entspannung, Musik und Technik untrennbar sind.
Das mag der Grund sein, warum die meisten von Chopins Kompositionen (anders als die von Beethoven) mit einer weiten Spanne von Geschwindigkeiten gespielt werden können.



<!-- c1ii15.html -->

\subsection{Automatische Verbesserung nach dem Üben (PPI)}
\label{c1ii15}

\textbf{Während einer Sitzung kann man nur ein bestimmtes Maß an Verbesserung erwarten}, weil es hauptsächlich zwei Arten gibt, sich zu verbessern.
Die erste ist die offensichtliche Verbesserung, die vom Lernen der Noten und Bewegungen kommt und in sofortiger Verbesserung resultiert.
Das tritt bei Passagen auf, für die Sie bereits die Technik zum Spielen haben.
\textbf{Die zweite wird \enquote{Automatische Verbesserung nach dem Üben (PPI)} genannt\footnote{PPI = post practice improvement} und resultiert aus physiologischen Veränderungen beim Erwerben einer neuen Technik.}
Dies ist ein langsamer Veränderungsprozess, der über Wochen oder Monaten abläuft, weil er das Wachstum von Nerven- und Muskelzellen erfordert.

Deshalb sollten Sie beim Üben versuchen, Ihren Fortschritt zu bewerten, sodass Sie aufhören und mit etwas anderem weitermachen können, sobald der Punkt abnehmender Ertragszuwächse erreicht ist, also üblicherweise nach weniger als zehn Minuten.
\textbf{Wie von Zauberhand wird sich Ihre Technik nach einer guten Übung für mindestens einige Tage von selbst weiter verbessern.}
\textit{Wenn Sie alles richtig gemacht haben}, sollten Sie deshalb am nächsten Tag feststellen, dass Sie besser spielen können.
Wenn das nur an einem Tag geschieht, ist der Effekt nicht so groß. Wenn das jedoch über Monate oder Jahre geschieht, kann der kumulative Effekt enorm sein.

Es ist normalerweise profitabler, verschiedene Dinge während einer Sitzung zu üben und sie simultan verbessern zu lassen (während Sie nicht üben!), als zu hart an einer Sache zu arbeiten.
Zu viel zu üben kann sogar Ihrer Technik schaden, wenn es zu Stress, schlechten Angewohnheiten oder Verletzungen führt.
Sie müssen eine bestimmte Minimalanzahl üben, ungefähr einhundert Wiederholungen, damit die PPI eintritt.
Da wir aber über ein paar Takte reden, die mit hoher Geschwindigkeit gespielt werden, sollte das Üben von dutzenden oder hunderten Malen nur ein paar Minuten benötigen.
Seien Sie deshalb unbesorgt, wenn Sie hart üben aber keine große sofortige Verbesserung sehen.
Das könnte für diese bestimmte Passage normal sein.
Wenn Sie nichts finden, das Sie falsch machen, ist es Zeit aufzuhören und die Sache der PPI zu überlassen, nachdem Sie eine für die PPI genügende Anzahl von Wiederholungen ausgeführt haben.
Achten Sie auch darauf, dass Sie entspannt üben, weil Sie keine PPI einer stressbeladenen Bewegung möchten.

In Abhängigkeit davon, was Sie aufhält, gibt es verschiedene Typen von PPI.
Eine der Arten, in denen sich diese Typen offenbaren, ist die Zeitspanne, während der die PPI wirkt.
Sie variiert von einem Tag bis zu vielen Monaten.
Die kürzesten Zeiten können mit der Konditionierung verbunden sein, wie dem Gebrauch von Bewegungen oder Muskeln, die Sie vorher nicht benutzt haben oder Gedächtnisfragen.
Mittlere Zeiten von mehreren Wochen können mit dem Bilden von Nervenverbindungen, wie für das beidhändige Spielen, verbunden sein.
Längere Zeiten können mit dem tatsächlichen Wachstum von Hirn-, Nerven- oder Muskelzellen verbunden sein, sowie der Umwandlung von langsamen in schnelle Muskelzelltypen (siehe \hyperref[c1iii7aMuskeln]{Abschnitt III.7a}).

Sie müssen alles richtig machen, um die PPI zu maximieren.
Viele Schüler kennen die Regeln nicht und können die PPI \textit{umkehren}, mit dem Ergebnis, dass sie am nächsten Tag \textit{schlechter} spielen.
Die meisten dieser Fehler haben ihren Ursprung im falschen Gebrauch des schnellen und langsamen Übens; deshalb werden wir die Regeln für die richtige Wahl der Übungsgeschwindigkeiten in den folgenden Abschnitten behandeln.
Jeder Stress oder unnötige Bewegung während des Übens wird ebenfalls der PPI unterzogen und kann sich in eine schlechte Angewohnheit verwandeln.
Der am weitesten verbreitete Fehler, den Schüler begehen, wenn sie die PPI umkehren, ist, unmittelbar bevor sie mit dem Üben aufhören schnell zu spielen.
Das Letzte, was Sie vor dem Aufhören tun, sollte das korrekteste und beste Beispiel dessen sein, was Sie erreichen wollen; dies geht mit einer moderaten bis langsamen Geschwindigkeit am besten.
\textbf{Der jeweils letzte Durchlauf hat anscheinend einen außerordentlich starken PPI-Effekt.}
Die Methoden dieses Buchs sind ideal für die PPI, hauptsächlich weil sie es betonen, nur die Abschnitte zu üben, die man nicht spielen kann.
Wenn man langsam beidhändig spielt und die Geschwindigkeit für einen großen Abschnitt langsam steigert, wird die PPI ungenügend konditioniert, weil man nicht genügend Zeit hat, die erforderliche Zahl der Wiederholungen auszuführen.
Außerdem gerät der PPI-Prozess durcheinander, weil man eine große Menge an leichtem Material mit dem kleinen Anteil an schwierigem Material vermischt und die Geschwindigkeit, Bewegungen usw. ebenfalls nicht korrekt sind.

PPI ist nichts Neues; sehen wir uns drei bekannte Beispiele an: den Bodybuilder, den Marathonläufer und den Golfer.
Während der Bodybuilder Gewichte stemmt, wachsen seine Muskeln nicht; er verliert sogar Gewicht.
Aber während der folgenden Wochen reagiert der Körper auf die Stimulanz und baut die Muskeln auf.
Das ganze Muskelwachstum erfolgt \textit{nach} dem Üben.
So misst der Bodybuilder nach dem Üben nicht, wie viel Muskeln er gewonnen hat oder wie viel Gewicht er mehr heben kann, sondern er konzentriert sich darauf, dass die Übung die erforderliche Konditionierung hervorruft.
Der Unterschied ist hier, dass wir für das Klavierspielen Koordination und Ausdauer anstelle von starken und großen Muskeln entwickeln.
Der Bodybuilder möchte die langsamen Muskeln wachsen lassen, während der Klavierspieler die langsamen Muskeln in schnelle umwandeln möchte.
Ein weiteres Beispiel ist der Marathonläufer.
Wenn man noch nie im Leben einen Kilometer gelaufen ist und es das erste Mal versucht, ist man vielleicht in der Lage, einen halben Kilometer zu laufen, bevor man langsamer werden muss, um eine Pause zu machen.
Wenn man nach einer Pause versucht, wieder weiterzulaufen, wird man immer noch nach einem halben Kilometer oder weniger müde.
So ergibt der erste Lauf keine erkennbare Verbesserung.
Wenn man jedoch einen Tag wartet und es wieder versucht, wird man vielleicht in der Lage sein, einen Kilometer zu laufen, bevor man ermüdet - man hat gerade die PPI kennengelernt.
Auf diese Art konditionieren sich Marathonläufer, sodass sie schließlich 42 Kilometer laufen können.
Golfer sind mit dem Phänomen vertraut, dass sie den Ball an einem Tag gut treffen aber schlecht am nächsten, weil sie eine schlechte Angewohnheit angenommen haben.
So führt zu oft mit dem Driver (dem schwierigsten Schläger) zu schlagen dazu, dass man das Schwingen ruiniert, während das Üben mit dem \#5-Holz (ein viel einfacherer Schläger) es wieder herstellen kann; deshalb ist es wichtig, mit einem einfacheren Schläger zu üben, bevor man aufhört.
Die Analogie beim Klavierspielen ist, dass es oft die PPI zunichte macht, wenn man schnell mit voller Wucht spielt, während das Üben einfacher Abschnitte (kurzer Abschnitte einhändig) dazu führt, sie zu verbessern.

\textbf{Die PPI geschieht hauptsächlich während des Schlafs.}
Sie können Ihr Auto nicht reparieren, solange Sie auf der Autobahn fahren; genauso kann der größte Teil des Wachstums und der Reparatur des Körpers nicht während der wachen Zeit geschehen.
Der Schlaf ist nicht nur zum Ausruhen da, sondern auch für das Wachstums und die Pflege des Körpers.
Dieser Schlaf muss der normale Nachtschlaf einschließlich aller Hauptbestandteile sein, insbesondere des REM-Schlafs\footnote{REM = Rapid Eye Movement; schnelle Augenbewegungen}.
Babys brauchen so viel Schlaf, weil sie schnell wachsen.
Sie werden vielleicht keine gute PPI erreichen, wenn Sie nachts nicht gut schlafen.
Am besten wird sein, wenn Sie abends für die Konditionierung üben und morgens die PPI überprüfen.
Die PPI wird durch den Zelltod angestoßen; hartes Üben ruft einen vorzeitigen Zelltod hervor, und der Körper überkompensiert es, wenn übermäßig viele Zellen sterben.
Man könnte meinen, dass 100 Wiederholungen keinen Zelltod verursachen können, aber es werden ständig Zellen  erneuert, und jede zusätzliche Arbeit beschleunigt diese Erneuerungsrate.


\subsection{Gefahren des langsamen Spielens - Fallstricke der \enquote{Intuitiven Methode}}
\label{c1ii16}

\textbf{Warum ist wiederholtes langsames Spielen (intuitive Methode) schädlich, wenn man ein neues Stück beginnt?}
Wenn man beginnt, gibt es keine Möglichkeit zu wissen, ob die Bewegung, die man für das langsame Spielen benutzt, richtig oder falsch ist.
Die Wahrscheinlichkeit falsch zu spielen liegt nahe 100\%, weil es fast unendlich viele Möglichkeiten gibt, falsch zu spielen, aber nur eine beste Art.
Wenn diese falsche Bewegung beschleunigt wird, dann wird der Schüler auf eine Geschwindigkeitsbarriere treffen.
Angenommen, dieser Schüler hat die Geschwindigkeitsbarriere erfolgreich überwunden, indem er neue Arten zu spielen gefunden hat, so musste er jeweils die alte Art zu spielen vergessen, die neue Art erneut lernen und diese Zyklen für jede einzelne Geschwindigkeitssteigerung wiederholen, bis er die endgültige Geschwindigkeit erreicht hat.
So kann die Methode, die Geschwindigkeit langsam zu steigern, viel Zeit verschwenden.

Sehen wir uns ein Beispiel dafür an, wie unterschiedliche Geschwindigkeiten verschiedene Bewegungen erfordern.
Denken Sie an die Gangarten des Pferdes.
Wenn die Geschwindigkeit gesteigert wird, geht die Gangart vom Gehen über den Trott und Kanter (leichter Galopp) zum Galopp.
Jede dieser vier Gangarten hat normalerweise mindestens eine langsame und eine schnelle Art.
Auch unterscheidet sich eine Linksdrehung von einer Rechtsdrehung (der führende Huf ist unterschiedlich).
Das macht ein Minimum von 16 Bewegungen.
Das sind die sogenannten natürlichen Gangarten; die meisten Pferde haben sie automatisch; man kann ihnen drei weitere Gangarten beibringen: Schritt, Foxtrott und Rack, bei denen es ebenfalls langsam, schnell, links und rechts gibt.
All das mit nur vier Beinen von relativ einfacher Struktur und einem vergleichsweise eingeschränkten Gehirn.
Wir haben zehn komplexe Finger, vielseitigere Schultern, Arme und Hände und ein viel fähigeres Gehirn!
Unsere Hände sind deshalb fähig, viel mehr \enquote{Gangarten} auszuführen als ein Pferd.
Die meisten Schüler haben eine geringe Vorstellung davon, wie viele Bewegungen möglich sind, wenn der Lehrer sie nicht auf diese hinweist.
Zwei Schüler, die sich selbst überlassen werden und die man bittet, dasselbe Stück zu spielen, werden garantiert bei verschiedenen Handbewegungen landen.
Das ist ein weiterer Grund, warum es so wichtig ist, Stunden bei einem guten Lehrer zu nehmen, wenn man mit dem Klavier anfängt; solch ein Lehrer kann schnell die schlechten Bewegungen aussieben.

Ein langsames Klavierspielen schrittweise zu steigern ist so, als ob man ein Pferd dazu zwingen wollte, so schnell wie im Galopp zu rennen, indem man bloß das Gehen beschleunigt - es geht einfach nicht, denn wenn die Geschwindigkeit steigt, dann ändert sich der Impuls der Beine, des Körpers usw., was die verschiedenen Gangarten notwendig macht.
Deshalb müsste der Schüler, wenn er die Geschwindigkeit schrittweise steigert und die Musik einen \enquote{Galopp} erfordert, all die dazwischenliegenden \enquote{Gangarten} lernen.
Ein Pferd dazu zu bringen, so schnell wie im Galopp zu gehen, würde Geschwindigkeitsbarrieren aufbauen, Stress erzeugen und Verletzungen verursachen.

Ein verbreiteter Fehler beim langsamen Spielen ist die Angewohnheit, die Hand zu stützen oder zu heben.
Beim langsamen Spielen kann die Hand während der Zeit zwischen den Noten, wenn das Abwärtsdrücken nicht notwendig ist, angehoben werden.
Wenn man schneller wird, fällt diese Angewohnheit des \enquote{Hebens} mit dem nächsten Anschlag zusammen; diese Handlungen heben sich auf und resultieren in einer verpassten Note.
Ein anderer häufiger Fehler ist das Wedeln mit den freien Fingern - während er mit Finger 1 und 2 spielt, wedelt der Schüler eventuell mit den Fingern 4 und 5 mehrere Male durch die Luft.
Das stellt keine Schwierigkeit dar, bis die Bewegung so beschleunigt wird, dass keine Zeit bleibt, mit den Fingern zu wedeln.
In dieser Situation hören die freien Finger bei höheren Geschwindigkeiten nicht automatisch mit dem Wedeln auf, weil die Bewegung durch hunderte oder tausende Wiederholungen eingefahren wurde.
Stattdessen müssen die Finger mehrere Male mit Geschwindigkeiten wedeln, die sie nicht erreichen können - das erzeugt die Geschwindigkeitsbarriere.
Die Schwierigkeit ist hier, dass die meisten Schüler, die langsames Üben benutzen, sich dieser schlechten Angewohnheiten nicht bewusst sind.
\textbf{Wenn Sie wissen, wie man schnell spielen muss, ist es sicher, langsam zu spielen, aber wenn Sie nicht wissen, wie man schnell spielen muss, müssen Sie aufpassen, dass Sie nicht die falschen Angewohnheiten für langsames Spielen lernen oder enorm viel Zeit verschwenden.}
Langsames Spielen kann große Zeiträume verschwenden, weil jeder Durchgang so lange dauert.
Wenn Sie schneller werden, müssen Sie den abwärts gerichteten Druck erhöhen, weil Sie innerhalb derselben Zeitspanne mehr Tasten drücken.
\textbf{So funktioniert das \enquote{Fühlen der Schwerkraft} die meiste Zeit nicht, weil Sie beim Spielen mit unterschiedlich starker Kraft nach unten drücken müssen.}

Ein weiteres Problem im Zusammenhang mit dem intuitiven langsamen Üben sind unnötige Körperbewegungen.
Diese Bewegungen erzeugen bei höheren Geschwindigkeiten weitere Schwierigkeiten.
\textbf{Wenn sie nicht ihr Spielen \hyperref[c1iii13]{auf Video aufnehmen} und sorgfältig nach merkwürdigen Körperbewegungen Ausschau halten, sind den meisten Klavierspielern nicht alle Bewegungen, die sie machen, bewusst. 
Das kann unvorhersehbare Fehler zu nicht vorhersehbaren Zeiten verursachen, was zu psychologischen Problemen mit Unsicherheit und \hyperref[c1iii15]{Nervosität} führt.}
Ein Bewusstsein für die Körperbewegungen zu entwickeln, kann dieses Problem eliminieren.
Wir sehen, dass die Intuition zu einer Vielzahl von Schwierigkeiten führen kann; statt der Intuition benötigen wir ein auf Wissen basierendes System.


\label{c1ii17}

% zuletzt geändert 04.10.2009

\subsection{Die Wichtigkeit des langsamen Spielens}

Nachdem wir die Gefahren des langsamen Spielens herausgestellt haben, besprechen wir nun, warum langsames Spielen \textit{unentbehrlich} ist.
\textbf{Beenden Sie eine Übungssitzung immer damit, dass Sie mindestens einmal langsam spielen.
Das ist die wichtigste Regel für eine gute \hyperref[c1ii15]{PPI}}.
Sie sollten sich auch angewöhnen, das zu tun, wenn Sie beim \hyperref[c1ii7]{Üben mit getrennten Händen} die Hände wechseln; spielen Sie vor dem Wechseln mindestens einmal langsam.
\textbf{Das ist vielleicht eine der wichtigsten Regeln dieses Kapitels, weil sie einen solch ungeheuer großen Effekt auf die Verbesserung der Technik hat.}
Sie ist sowohl für die sofortige Verbesserung als auch für die PPI von Nutzen.
Ein Grund, warum es funktioniert, ist eventuell, dass man vollständig entspannen kann (siehe \hyperref[c1ii14]{Abschnitt II.14}).
Ein weiterer Grund kann sein, dass man dazu neigt, sich beim schnellen Spielen mehr schlechte Angewohnheiten anzueignen als man merkt, und man kann diese Angewohnheiten mit langsamem Spiel \enquote{löschen}.
Entgegen der Intuition ist langsames Spielen ohne Fehler schwierig (bis man die Passage komplett gemeistert hat).
So ist das langsame Spielen eine gute Möglichkeit, zu überprüfen, ob Sie dieses Musikstück wirklich gelernt haben.

Der Effekt des langsamen Spielens am Ende auf die PPI ist so dramatisch, dass Sie ihn sich leicht selbst demonstrieren können.
Versuchen Sie, in einer Übungssitzung nur schnell zu spielen, und schauen Sie, was am nächsten Tag geschieht.
Spielen Sie in der nächsten Sitzung langsam, bevor Sie aufhören, und schauen Sie wieder, was am nächsten Tag geschieht.
Oder Sie üben eine Passage nur schnell und eine andere Passage (derselben Schwierigkeit) am Ende langsam und vergleichen sie am nächsten Tag miteinander.
Dieser Effekt ist kumulativ, sodass Sie, wenn Sie dieses Experiment mit diesen beiden Passsagen längere Zeit wiederholen würden, schließlich einen riesigen Unterschied darin feststellen würden, wie gut Sie diese Passagen spielen können.

Wie langsam ist langsam?
Das ist eine Ermessensfrage, die von Ihrer Fertigkeitsstufe abhängt.
Unterhalb einer bestimmten Geschwindigkeit geht der nützliche Effekt des langsamen Spielens verloren.
Es ist wichtig, beim langsamen Spielen dieselbe Bewegung wie beim schnellen Spielen beizubehalten.
Wenn Sie zu langsam spielen, kann das unmöglich sein.
Auch braucht zu langsames Spielen zu viel Zeit und verschwendet somit Zeit.
Die beste Geschwindigkeit, die Sie zuerst ausprobieren sollten, ist eine, in der Sie so genau spielen können wie Sie möchten, ungefähr 1/2 bis 3/4 der endgültigen Geschwindigkeit.
Langsames Spielen wird auch für das Auswendiglernen benötigt (siehe \hyperref[c1iii6h]{Abschnitt III.6h}).
Die optimale langsame Geschwindigkeit für das Auswendiglernen, ungefähr unterhalb 1/2 der endgültigen Geschwindigkeit, ist niedriger als die für die Konditionierung der PPI benötigte.
Wenn die Technik besser wird, kann diese langsame Geschwindigkeit schneller werden.
Einige berühmte Pianisten üben \textit{sehr langsam}!
Einige Quellen sprechen vom Üben mit einer Note pro Sekunde, was fast irrational klingt aber eventuell für das Gedächtnis und die Musikalität nützlich sein mag.

\textbf{Eine wichtige Fertigkeit, die beim langsamen Spielen geübt werden muss, ist, den Noten voraus zu denken.}
Wenn man ein neues Stück schnell übt, gibt es eine Tendenz, gedanklich hinter die Noten zurückzufallen, und das kann zur Gewohnheit werden.
Das ist schlecht, weil man so die Kontrolle verliert.
Denken Sie voraus, wenn Sie langsam spielen, und versuchen Sie dann, diesen Vorsprung zu bewahren, wenn Sie zur höheren Geschwindigkeit zurückkehren.
Durch das Vorausdenken können Sie gewöhnlich Spielfehler oder Schwierigkeiten vorher kommen sehen und haben die Zeit, entsprechende Maßnahmen zu ergreifen.



<!-- c1ii18.html -->

\subsection{Fingersatz}
\label{c1ii18}

Außer in Büchern für Anfänger werden die grundlegenden Fingersätze in den Notationen nicht angegeben.
Diese grundlegenden Fingersätze finden Sie in den Abschnitten über die Tonleitern (\hyperref[c1iii5d]{III.5d} und \hyperref[c1iii5h]{III.5h}) und Arpeggios (\hyperref[c1iii5e]{III.5e}).
\textbf{Beachten Sie, dass die Tonleitern die Fingersätze für praktisch alle Läufe bestimmen.
Deshalb ist es wichtig, sich die \hyperref[table]{Fingersätze aller Tonleitern} zu merken}.
Das ist nicht schwierig, weil die meisten Tonleitern einem Standardfingersatz und die Ausnahmen einfachen Regeln folgen, wie einen Daumen auf den schwarzen Tasten zu vermeiden.
Eine schwarze Taste mit dem Daumen zu spielen, bringt die Hand zu nah an die Klappe heran und macht es schwierig, die weißen Tasten mit den anderen Fingern zu spielen.
Die meisten Notenblätter geben die Fingersätze für ungewöhnliche Situationen an, in denen besondere Fingersätze notwendig sind.
Befolgen Sie diese Fingersätze, solange Sie keinen besseren haben; wenn Sie dem angegebenen Fingersatz nicht folgen, werden Sie sich vermutlich Ärger einhandeln.
Ein angegebener Fingersatz mag Ihnen zunächst unhandlich erscheinen, aber er steht dort aus guten Gründen.
Diese Gründe werden oft erst offensichtlich, wenn man zur endgültigen Geschwindigkeit kommt oder \hyperref[c1ii25]{beidhändig spielt}.
\textbf{Es ist sehr wichtig, sich einen festen Fingersatz zu suchen und ihn nicht zu ändern, solange es keinen guten Grund dafür gibt.}
Keinen festen Fingersatz zu haben, wird den Lernprozess verlangsamen und Ihnen später Ärger machen, besonders während des \hyperref[c1iii14]{Vorspielens}, wenn eine Unschlüssigkeit beim Fingersatz zu einem Fehler führen kann.
Wenn Sie den Fingersatz ändern, dann bleiben Sie immer bei dem neuen.
Vermerken Sie die Änderung auf dem Notenblatt, sodass Sie den Fingersatz während des Übens nicht versehentlich ändern; auch kann es sehr ärgerlich sein, Monate später zu dieser Musik zurückzukommen und sich nicht mehr an diesen tollen Fingersatz erinnern zu können, den man sich vorher herausgearbeitet hat.

Nicht alle in der Notation vorgeschlagenen Fingersätze sind für jeden angemessen.
Sie haben vielleicht große oder kleine Hände.
Sie haben sich vielleicht aufgrund der Art, wie sie gelernt haben, einen anderen Fingersatz angewöhnt.
Sie könnten einen anderen Satz an Fertigkeiten haben; zum Beispiel könnten Sie \hyperref[c1iii3]{Triller} besser mit 1,3 als mit 2,3 spielen.
Noten von verschiedenen Herausgebern können unterschiedliche Fingersätze haben.
Für fortgeschrittene Spieler kann der Fingersatz einen profunden Einfluss auf den zu erzielenden musikalischen Effekt haben.	
Glücklicherweise sind die in diesem Buch beschriebenen Methoden gut geeignet, um den Fingersatz schnell zu ändern.
Wenn Sie erst einmal mit den Methoden dieses Buchs vertraut sind, werden Sie in der Lage sein, den Fingersatz sehr schnell zu ändern.
Führen Sie alle diese Änderungen durch, bevor Sie mit dem beidhändigen Üben anfangen, weil die Fingersätze sehr schwer zu ändern sind, wenn sie erst einmal in das beidhändige Spiel aufgenommen sind.
Auf der anderen Seite sind einige Fingersätze zwar \hyperref[c1ii7]{einhändig} leicht, werden aber beidhändig schwierig, sodass es sich auszahlt, sie beidhändig zu überprüfen, bevor man irgendeine Änderung dauerhaft akzeptiert.

Zusammengefasst ist der Fingersatz von entscheidender Bedeutung. 
\textbf{Anfänger sollten nicht mit dem Üben beginnen, wenn sie nicht die richtigen Fingersätze kennen.}
Wenn Sie sich über den Fingersatz unsicher sind, versuchen Sie, Noten mit zahlreichen Fingersatzangaben zu finden, oder gehen Sie in ein Internetforum, und bitten Sie um Hilfe.
Wenn Sie sich die Fingersätze für Tonleitern und Arpeggios ansehen, werden Sie ein paar dem \enquote{gesunden Menschenverstand} entsprechende Regeln finden; diese sollten für den Anfang genügen.


\subsection{Akkurates Tempo und das Metronom}
\label{c1ii19}

\textbf{Beginnen Sie alle Stücke mit sorgfältigem Zählen; das gilt insbesondere für Anfänger und Jugendliche.}
Kindern sollte beigebracht werden, laut zu zählen, weil das der einzige Weg ist, herauszufinden, was \textit{ihre}  Vorstellung des Zählens ist.
Sie kann völlig von der beabsichtigten abweichen!
Man sollte die Taktbezeichnung am Anfang jeder Komposition verstehen.
Diese sieht wie ein Bruch, bestehend aus Zähler und Nenner, aus.
Der Zähler gibt die Anzahl der Schläge je Takt an und der Nenner die Note je Schlag.
Zum Beispiel bedeutet 3/4, dass jeder Takt drei Schläge hat, und dass jeder Schlag eine Viertelnote ist.
Die Taktbezeichnung zu kennen ist beim Begleiten entscheidend, weil der Moment, in dem der Begleiter beginnt, durch den Anfangsschlag bestimmt ist, den der Dirigent mit dem Taktstock anzeigt.

Ein Vorteil des \hyperref[c1ii7]{Übens mit getrennten Händen} ist, dass man dazu neigt, genauer zu zählen als beim \hyperref[c1ii25]{beidhändigen Üben}.
Schüler, die beidhändig anfangen, können am Ende unerkannte Fehler beim Zählen haben.
Interessanterweise machen es diese Fehler im Allgemeinen unmöglich, die Musik auf Geschwindigkeit zu bringen.
Es gibt etwas beim falschen Zählen, das seine eigene Geschwindigkeitsbarriere erzeugt.
Es bringt wahrscheinlich den \hyperref[c1iii1b]{Rhythmus} durcheinander.
Deshalb sollten Sie das Zählen überprüfen, wenn Sie Probleme beim Erreichen der Geschwindigkeit bekommen.
Ein Metronom ist dafür sehr nützlich.

\textbf{Benutzen Sie das Metronom, um Ihre Geschwindigkeits- und Schlaggenauigkeit zu überprüfen.}
Ich wurde wiederholt von Fehlern überrascht, die ich auf diese Art beim Prüfen entdeckt habe.
Zum Beispiel neige ich dazu, bei schwierigen Abschnitten langsamer zu werden und schneller bei leichteren, obwohl es mir so vorkommt, als wäre es genau umgekehrt, wenn ich ohne das Metronom spiele.
Die meisten Lehrer prüfen das Tempo ihrer Schüler damit.
Wenn der Schüler das richtige Timing hat, sollte das Metronom abgeschaltet werden.
Das Metronom ist einer Ihrer verlässlichsten Lehrer - wenn Sie erst einmal angefangen haben, es zu benutzen, werden Sie froh sein, es getan zu haben.
Entwickeln Sie die Angewohnheit, das Metronom zu benutzen, und Ihr Spiel wird sich ohne Zweifel verbessern.
Alle ernsthaften Schüler müssen ein Metronom haben.

Metronome sollten nicht übermäßig benutzt werden.
\textbf{Lange Übungssitzungen, bei denen das Metronom Sie begleitet, sind schädlich für das Erwerben der Technik und führen zu einer unmusikalischen Spielweise.}
Wenn das Metronom kontinuierlich länger als ungefähr zehn Minuten benutzt wird, wird Ihr Gehirn anfangen, Ihnen mentale Streiche zu spielen, sodass Sie eventuell die Genauigkeit des Timings verlieren.
Wenn das Metronom Klicks abgibt, erzeugt das Gehirn zum Beispiel nach einiger Zeit Antiklicks in Ihrem Kopf, die den Metronomklick aufheben können, sodass Sie entweder das Metronom nicht mehr hören oder es zur falschen Zeit hören.
Deshalb haben die meisten modernen elektronischen Metronome einen Modus mit pulsierender Leuchtanzeige.
Das visuelle Zeichen ist für mentale Tricks weniger anfällig und stört die Musik nicht akustisch.
Der häufigste Missbrauch des Metronoms ist, es zum Steigern der Geschwindigkeit zu benutzen; das missbraucht das Metronom, den Schüler, die Musik und die Technik.
Wenn Sie die Geschwindigkeit schrittweise steigern müssen, benutzen Sie das Metronom, um das Tempo festzulegen.
Schalten Sie es dann aus, wenn Sie mit dem Üben fortfahren.
Benutzen Sie es dann wieder kurz, wenn Sie die Geschwindigkeit erhöhen.
\textbf{Das Metronom ist dazu da, das Tempo festzulegen und Ihre Genauigkeit zu prüfen.
Es ist kein Ersatz für Ihr eigenes internes Timing.}

Der Vorgang des Schnellerwerdens ist ein Prozess des Herausfindens der geeigneten neuen Bewegungen.
Wenn Sie die richtige neue Bewegung finden, können Sie einen Quantensprung zu einer höheren Geschwindigkeit machen, bei der die Hand komfortabel spielt; in Wahrheit ist bei mittlerer Geschwindigkeit weder die langsame noch die schnelle Bewegung anwendbar, und es ist oft schwieriger zu spielen als mit der schnellen Geschwindigkeit.
Wenn Sie das Metronom zufällig auf diese mittlere Geschwindigkeit gesetzt haben, müssen Sie eventuell längere Zeit damit kämpfen und bauen eine Geschwindigkeitsbarriere auf.
Einer der Gründe, warum die neue Bewegung funktioniert, ist, dass die menschliche Hand ein mechanisches Gerät ist und Resonanzen hat, bei denen bestimmte Kombinationen von Bewegungen auf natürliche Art gut funktionieren.
Es besteht kaum Zweifel darüber, dass manche Musik für bestimmte Geschwindigkeiten komponiert wurde, weil der Komponist diese Resonanzgeschwindigkeit herausgefunden hat.
Auf der anderen Seite hat jeder Einzelne eine andere Hand mit anderen Resonanzgeschwindigkeiten, und das erklärt teilweise, warum verschiedene Pianisten verschiedene Geschwindigkeiten wählen.
Ohne das Metronom können Sie von einer Resonanz zur nächsten wechseln, weil die Hände sich bei diesen Geschwindigkeiten wohl fühlen, während die Chancen sehr gering sind, dass Sie das Metronom genau auf diese Geschwindigkeiten setzen.
Deshalb übt man mit dem Metronom fast immer mit der falschen Geschwindigkeit, solange man die Resonanzen nicht kennt (niemand kennt sie) und das Metronom entsprechend einstellt.

\textbf{Elektronische Metronome sind mechanischen in jeder Hinsicht überlegen}, es sei denn, Sie bevorzugen das Aussehen der alten Modelle.
Elektronische sind genauer, können verschiedene Töne oder Blinkzeichen erzeugen, haben eine variable Lautstärke, sind billiger, sind weniger unhandlich, haben Speicherfunktionen usw., während die mechanischen anscheinend immer im falschen Moment aufgezogen werden müssen.


\subsection{Die schwache linke Hand - Eine Hand unterrichtet die andere}
\label{c1ii20}

\textbf{Schüler, die nicht \hyperref[c1ii7]{mit getrennten Händen üben}, werden immer eine stärkere rechte als linke Hand haben}.\footnote{Das scheint gemäß der eigenen Erfahrung auch für Linkshänder zu gelten.}
Das geschieht, weil die Passagen der rechten Hand im Allgemeinen technisch schwieriger sind.
Die Passagen der linken Hand erfordern meistens mehr Kraft, die linke Hand bleibt aber hinsichtlich Geschwindigkeit und Technik oft zurück.
Deshalb bedeutet \enquote{schwächer} hier technisch schwächer, nicht in Bezug auf die Kraft.
\textbf{Die Methode mit getrennten Händen balanciert die Hände aus, weil man automatisch der schwächeren Hand mehr zu arbeiten gibt.}
Für Passagen, die eine Hand besser spielen kann als die andere, ist die bessere Hand oftmals Ihr bester Lehrer.
Um eine Hand die andere unterrichten zu lassen, wählen Sie einen kurzen Ausschnitt, und spielen Sie ihn schnell mit der besseren Hand.
Wiederholen Sie ihn dann sofort mit der schwächeren Hand und zwar um eine Oktave versetzt, um Kollisionen zu vermeiden.
Sie werden entdecken, dass die schwächere Hand oftmals \enquote{aufschließen} kann oder \enquote{eine Vorstellung davon bekommt}, wie es die bessere Hand macht.
Der Fingersatz sollte ähnlich sein, muss aber nicht identisch sein.
Wenn die schwächere Hand erst einmal \enquote{eine Vorstellung bekommt}, dann machen Sie sie schrittweise stärker, indem Sie mit der schwächeren Hand zweimal und der stärkeren Hand einmal spielen, dann dreimal gegen einmal, usw.

Diese Fähigkeit, mit einer Hand die andere zu unterrichten, ist wichtiger als vielen bewusst ist.
Das obige Beispiel, in dem ein bestimmtes technisches Problem gelöst wurde, ist nur ein Beispiel - wichtiger ist, dass dieses Konzept auf praktisch jede Übungssitzung anwendbar ist.
Der Hauptgrund für diese breite Anwendbarkeit ist, dass eine Hand \textit{immer} etwas besser spielt als die andere, zum Beispiel hinsichtlich der Entspannung, der Geschwindigkeit, den \hyperref[ruhig]{ruhigen Händen} und der unzähligen Finger- bzw. Handbewegungen (\hyperref[c1iii5a]{Daumenübersatz}, \hyperref[c1iii4b]{flache Finger} usw. - siehe folgende Abschnitte), also allem Neuen, das Sie versuchen zu lernen.
Wenn Sie das Prinzip, eine Hand zu benutzen um die andere zu unterrichten, erst einmal gelernt haben, werden Sie es deshalb immer verwenden.



<!-- c1ii21.html -->

\subsection{Ausdauer aufbauen, Atmung}
\label{c1ii21}

\enquote{Ausdauer} ist ein umstrittener Begriff beim Klavierüben.
Diese Auseinandersetzung ist in der Tatsache begründet, dass \textbf{Klavierspielen Kontrolle und nicht Muskelkraft erfordert}, und viele Schüler haben den falschen Eindruck, dass sie keine Technik erwerben werden, bevor sie genug Muskeln entwickelt haben.
Auf der anderen Seite ist ein gewisses Maß an Ausdauer notwendig.
Dieser offensichtliche Widerspruch kann beseitigt werden, wenn man genau versteht, was benötigt wird und wie man es bekommt.
Offensichtlich kann man laute, grandiose Passagen nicht ohne Energieaufwand spielen.
Große, starke Pianisten, die ansonsten dieselben Fertigkeiten haben, können sicherlich mehr Klang erzeugen als kleine, schwache Pianisten.
Und die stärkeren Pianisten können leichter \enquote{anstrengende} Stücke spielen.
Jeder Pianist hat genug körperliche Ausdauer, um Stücke seiner Stufe zu spielen, einfach wegen der Menge an Übung, die erforderlich war, um auf diese Stufe zu kommen.
Doch wissen wir, dass Ausdauer ein Problem \textit{ist}.
Die Antwort liegt in der \hyperref[c1ii14]{Entspannung}.
Wenn die Ausdauer ein Thema wird, wird es fast immer durch übermäßige Spannung verursacht.

Ein Beispiel dafür ist das Oktavtremolo der linken Hand im ersten Satz von Beethovens Pathétique.
Das \textit{einzige}, was über 90\% der Schüler tun müssen, ist, den Stress zu eliminieren; doch viele Schüler üben es für Monate mit geringem Fortschritt.
Ihr erster Fehler ist, dass sie es zu laut üben.
Das fügt gerade dann zusätzlichen Stress und Ermüdung hinzu, wenn man es sich am wenigsten leisten kann.
Üben Sie es leise, und konzentrieren Sie sich nur darauf, den Stress zu eliminieren, wie in \hyperref[c1iii3b]{Abschnitt III.3b} beschrieben.
In einer Woche oder zwei werden Sie so viele Tremolos so schnell spielen wie Sie möchten.
Fangen Sie nun an, Lautstärke und Ausdruck hinzuzufügen. Fertig!
An diesem Punkt unterscheidet sich Ihre körperliche Stärke und Ausdauer nicht von der, die Sie hatten, als Sie vor wenigen Wochen angefangen hatten - Sie haben sich hauptsächlich damit beschäftigt, die beste Art zu finden, den Stress zu eliminieren.

Anspruchsvolle Stücke zu spielen, erfordert ungefähr so viel Energie wie ein langsames Joggen mit ungefähr vier Meilen pro Stunde, wobei das Gehirn fast die Hälfte der gesamten Energie benötigt.
Viele Jugendliche können nicht mehr als eine Meile ununterbrochen joggen.
Deshalb würde es die Ausdauer überbeanspruchen, wenn man einen jungen Menschen bitten würde, schwierige Passagen 20 Minuten lang ununterbrochen zu üben, weil es ungefähr dem Joggen von einer Meile entspräche.
Lehrer und Eltern müssen aufpassen, wenn Jugendliche ihre Klavierstunden beginnen, dass die Übungszeiten am Anfang auf weniger als 15 Minuten begrenzt sind, bis der Schüler genügend Ausdauer erlangt hat.
Marathonläufer haben Ausdauer, aber sie sind nicht muskulös.
Man muss den Körper für die für das Klavierspielen notwendige Ausdauer konditionieren, aber man braucht keine zusätzlichen Muskeln.

Nun \textit{gibt es} einen Unterschied zwischen dem Klavierspielen und dem Marathonlaufen wegen der Notwendigkeit, zusätzlich zur Muskelkonditionierung das Gehirn für die Ausdauer zu konditionieren.
Deshalb kann man mit stupidem Üben keine Ausdauer erreichen.
Die effizienteste Art, Ausdauer zu erlangen, ist, entweder fertig gelernte Stücke zu spielen und Musik zu machen oder schwierige Abschnitte kontinuierlich \hyperref[c1ii7]{mit getrennten Händen zu üben}.
Benutzen wir wieder den Vergleich mit dem Joggen.
Es wäre für die meisten Schüler sehr schwer, schwieriges Material ununterbrochen länger als einige Stunden zu üben, weil zwei Stunden zu üben, sechs Meilen zu joggen entsprechen würde, was ein \enquote{Wahnsinnstraining} ist.
Deshalb werden Sie zwischen den schweren Übungsteilen ein paar leichte Stücke spielen müssen.
Konzentrierte Übungssitzungen von mehr als ein paar Stunden sind nicht so hilfreich, bevor Sie nicht auf einer fortgeschrittenen Stufe sind.
Es ist wahrscheinlich besser, zu unterbrechen und nach einer Pause erneut mit dem Üben zu beginnen.
\textbf{Klar, hartes Üben ist anstrengende Arbeit, und ernsthaftes Üben kann den Schüler in eine gute körperliche Verfassung bringen.}
Mit getrennten Händen zu üben ist in dieser Hinsicht am wertvollsten, weil es einer Hand gestattet sich zu erholen, während die andere hart arbeitet, was dem Klavierspieler erlaubt, 100\% der Zeit ohne Verletzung oder Ermüdung so hart zu arbeiten wie er möchte.
Natürlich ist es von der Ausdauer her gesehen nicht schwierig (wenn man die Zeit hat), sechs oder acht Stunden an täglicher Übungszeit aufzuwenden, indem man jede Menge stupider Fingerübungen einschließt.
Das ist ein Prozess der Selbsttäuschung, in welchem der Schüler glaubt, dass der bloße Zeitaufwand ihn ans Ziel bringt - wird er aber nicht.
Wenn überhaupt, ist es wichtiger, das Gehirn zu konditionieren als die Muskeln, denn bei den meisten Schülern ist es das Gehirn, das mehr konditioniert werden muss.
Die Konditionierung des Gehirns ist für das \hyperref[c1iii14]{Vorspielen} besonders wichtig.
Eine anstrengende Konditionierung der Muskeln wird dazu führen, dass der Körper schnelle Muskeln in langsame umwandelt (diese sind ausdauernder) - genau das Gegenteil von dem, was man möchte.
Entgegen der verbreiteten Meinung benötigen Klavierspieler deshalb nicht mehr Muskeln; sie benötigen eine größere Nervenkontrolle und die Umwandlung langsamer Muskeln in schnelle - siehe \hyperref[c1iii7aMuskeln]{Abschnitt III.7a}.

Was ist Ausdauer? Sie ist etwas, das uns befähigt, weiter zu spielen ohne müde zu werden.
Bei langen Übungssitzungen über mehrere Stunden bekommen Klavierspieler genauso wie Athleten (insbesondere Marathonläufer) ihren neuen Energieschub.
Wenn Sie sich generell müde fühlen, warten Sie deshalb darauf, dass Sie den toten Punkt überwinden - dieses bewusste Wissen um den neuen Energieschub kann bedeuten, dass er zuverlässiger einsetzt, insbesondere, wenn Sie es schon mehrmals erlebt haben und wissen, wie es sich anfühlt.
Gewöhnen Sie sich deshalb nicht an, jedes Mal auszuruhen, wenn Sie müde werden, wenn die Möglichkeit besteht, dass Sie den neuen Energieschub bekommen.

Können wir irgendwelche biologischen Faktoren bestimmen, die die Ausdauer kontrollieren?
Die biologische Basis zu kennen ist der beste Weg, Ausdauer zu verstehen.
Da keine spezifischen biophysikalischen Studien für Klavierspieler vorliegen, können wir nur spekulieren.
\textbf{Klar ist, dass wir eine genügende Sauerstoffaufnahme und einen adäquaten Blutfluss zu den Muskeln, bestimmten Organen und dem Gehirn brauchen.}
Der größte Faktor, der die Sauerstoffaufnahme beeinflusst, ist die Leistungsfähigkeit der Lunge, und wichtige Komponenten davon sind Atmung und Körperhaltung.
Das ist wahrscheinlich ein Grund, warum Meditation mit einer Betonung der richtigen Atmung unter Benutzung des Zwerchfells so hilfreich ist.
\textbf{Wenn nur die Rippenmuskulatur zum Atmen benutzt wird, dann wird der Atmungsapparat zu viel und das Zwerchfell zu wenig benutzt.}
Das daraus resultierende schnelle Pumpen des Brustkorbs oder die übertriebene Ausdehnung der Brust kann mit dem Klavierspielen in Konflikt geraten.
Der Gebrauch des Zwerchfells gerät mit den Spielbewegungen weniger in Konflikt.
Wenn beim Spielen Stress aufgebaut wird, werden außerdem diejenigen, die das Zwerchfell nicht bewusst benutzen, eventuell das Zwerchfell anspannen und es nicht einmal merken.
Indem sowohl die Rippen als auch das Zwerchfell benutzt werden und eine gute Haltung aufrechterhalten wird, können die Lungen mit geringstem Aufwand bis zu ihrem maximalen Volumen ausgedehnt werden und somit die maximale Menge an Sauerstoff aufnehmen.

Die folgende Atemübung kann sehr hilfreich sein, nicht nur für das Klavierspielen, sondern auch für das allgemeine Wohlbefinden.
Atmen Sie tief ein, und dehnen Sie dabei Ihren Brustkorb, schieben Sie Ihr Zwerchfell nach unten (Ihr Unterbauch wölbt sich nach außen), heben Sie Ihre Schultern an, und ziehen Sie sie nach hinten; atmen Sie dann vollständig aus, und kehren Sie dabei alle vorigen Bewegungen um.
Wenn Sie tief einatmen, ist ein vollständiges Ausatmen wichtiger als ein vollständiges Einatmen.
Atmen Sie durch die Nase (Sie können dabei den Mund offen lassen oder schließen). Achten Sie aber darauf, dass die Muskeln in der Nase entspannt sind und die Nasenflügel nicht eingezogen werden, und dass der Rachenraum nicht verengt wird, weil das leicht geschehen kann, wenn man angestrengt durch die Nase einatmet. Am besten geht es wahrscheinlich, wenn Sie sich auf das Einatmen durch den Rachen - in der Nähe der Stimmbänder - konzentrieren und die Luft einfach durch die Nase einströmen lassen. Das entspannt die Nasenmuskeln, und der Luftstrom durch die Nase wird größer.
Wenn Sie lange Zeit nicht tief eingeatmet haben, werden Sie wahrscheinlich nach einer oder zwei solcher Übungen hyperventilieren, und es wird Ihnen schwindlig. \textbf{Hören Sie sofort auf, falls Sie hyperventilieren!}
Wiederholen Sie diese Übung dann zu einem späteren Zeitpunkt; Sie sollten dann mehr Atemzüge nehmen können ohne zu hyperventilieren.
Wiederholen Sie diese Übung, bis Sie mindestens fünf Atemzüge hintereinander nehmen können, ohne zu hyperventilieren.
Wenn Sie dann zu Ihrem Arzt gehen und er Sie beim Abhören mit dem Stethoskop bittet, tief einzuatmen, können Sie das tun, ohne dass es Ihnen schwindlig wird!
Normal zu atmen während man etwas Schwieriges spielt, ist ein wichtiges Element der \hyperref[c1ii14]{Entspannung}.
Führen Sie diese Übung mindestens einmal alle paar Monate durch, und bauen Sie sie in Ihre normalen Atemgewohnheiten sowohl am Klavier als auch sonst ein.

\label{c1ii21uebung}\footnote{Achten Sie auch im Alltag hin und wieder auf Ihre Atmung. Atmen Sie dann ein paarmal \enquote{mit dem Bauch} ein und aus - möglichst durch die Nase, nicht extra langsam (Sie sollen ja schließlich keine Atemnot bekommen) aber auch nicht zu schnell, das heißt Sie sollten die Luft nicht mit Kraft durch die Nase strömen, sondern eher wie von selbst ein- und ausfließen lassen. Den Brustkorb, das heißt die Brustmuskulatur, sollten Sie nicht mehr als notwendig einbeziehen. Es kommt nicht darauf an, die Lungen so weit wie möglich zu füllen, sondern darauf, das normale Atemvolumen so entspannt wie möglich aufzunehmen. Diese Übung eignet sich auch hervorragend für die schnelle Entspannung zwischendurch, wenn es mal wieder \enquote{hoch hergeht}. Und wenn Sie schon dabei sind, können Sie auch gleich nachprüfen, ob Ihre Muskulatur angespannt ist. Gute Kandidaten sind zum Beispiel übereinandergeschlagene oder unter dem Bürostuhl versteckte Beine, die Schulter- bzw. Nackenmuskulatur und die Kiefermuskeln.}

Klavierspielenüben kann gesund oder ungesund sein, je nachdem wie man übt.
Viele Schüler vergessen zu atmen, während sie schwieriges Material üben; diese schlechte Angewohnheit ist ungesund.
Sie reduziert den Sauerstoffzufluss zum Gehirn, wenn es ihn am meisten benötigt, was zu Sauerstoffmangel und zu Symptomen führt, die einer Schlafapnoe ähneln (Organschäden, hoher Blutdruck usw.)
Der Sauerstoffmangel erschwert das musikalische und das \hyperref[c1ii12mental]{mentale Spielen} und macht es unmöglich, eine mentale Ausdauer zu entwickeln.

Weitere Methoden zum Erhöhen der Ausdauer sind die Steigerung des Blutflusses und die Vergrößerung der Blutmenge im Körper.
Beim Klavierspielen wird zusätzliches Blut sowohl im Gehirn als auch im Spielmechanismus benötigt; \textbf{deshalb sollten Sie sowohl das Gehirn als auch die Muskeln während des Übens völlig und gleichzeitig trainieren.
Das veranlasst den Körper, als Reaktion auf den erhöhten Blutbedarf, mehr Blut herzustellen.}
Stupide Wiederholungen von Übungen usw. sind in dieser Beziehung nicht hilfreich, weil sie das Gehirn ausschließen und so die Notwendigkeit für mehr Blut reduzieren können.
Nach einer großen Mahlzeit zu üben, erhöht ebenfalls die Blutversorgung, und umgekehrt wird es die Ausdauer reduzieren, wenn man sich nach jeder Mahlzeit ausruht - ein bekanntes japanisches Sprichwort sagt, dass man sich in eine Kuh verwandelt, wenn man nach einem Essen schläft.
Da die meisten Menschen nicht genügend Blut haben, um anstrengende Tätigkeiten mit einem vollen Magen auszuführen, wird der Körper zunächst rebellieren, und Sie werden sich schlecht fühlen, aber das ist eine zu erwartende Reaktion.
Solche Aktivitäten müssen innerhalb sicherer medizinischer Grenzen durchgeführt werden; so können Sie zum Beispiel vorübergehend Probleme mit der Verdauung bekommen, oder Sie sind ein wenig benommen (was wahrscheinlich der Grund für die falsche Auffassung ist, dass man nach einer großen Mahlzeit nicht üben soll).
Wenn der Körper erst das notwendige zusätzliche Blut erzeugt, werden diese Probleme verschwinden.
Sie sollten deshalb nach einer Mahlzeit so aktiv wie möglich bleiben, um einem Blutmangel vorzubeugen.
Das Üben direkt nach einer Mahlzeit führt dazu, dass Blut für die Verdauung, für die Spielmuskeln und für das Gehirn benötigt wird und so die größten Anforderungen an die Blutversorgung gestellt werden.
Klar ist, dass die Teilnahme am Sport, eine gute Gesundheit und ein körperliches Training hilfreich sind, um Ausdauer beim Klavierspielen zu bekommen.

Zusammengefasst: Anfänger, die noch nie zuvor ein Klavier angerührt haben, müssen ihre Ausdauer schrittweise entwickeln, weil Klavierüben eine anstrengende Arbeit \textit{ist}.
Eltern müssen auf die Übungsdauer von sehr jungen Anfängern achten und ihnen erlauben, aufzuhören oder eine Pause einzulegen, wenn sie müde werden (nach ungefähr 10 bis 15 Minuten).
\textbf{Erlauben Sie niemals einem kranken Kind, Klavier zu üben, selbst einfache Stücke, wegen des Risikos die Krankheit zu verschlimmern und von Hirnschädigungen}\footnote{falls durch das Klavierspielen das Fieber stark ansteigt}.
Auf jeder Fertigkeitsstufe haben wir alle mehr Muskeln als wir brauchen, um die Klavierstücke unserer Stufe zu spielen.
Sogar professionelle Pianisten, die jeden Tag mehr als sechs Stunden üben, sehen am Ende nicht aus wie Popeye.
Franz Liszt war dünn, nicht muskulös.
So ist das Aneignen von Technik und Ausdauer keine Frage des Muskelaufbaus, sondern des Lernens wie man entspannt und seine Energie sinnvoll einsetzt.
 

% zuletzt geändert 31.10.2009

\subsection{Schlechte Angewohnheiten: Der größte Feind des Pianisten}
\label{c1ii22}

\textbf{Schlechte Angewohnheiten sind die schlimmsten Zeitverschwender beim Klavierüben.
Die meisten schlechten Angewohnheiten werden durch Stress beim beidhändigen Üben von Stücken, die zu schwierig sind, verursacht.}
Viele der aus dem beidhändigen Üben resultierenden schlechten Angewohnheiten sind schwierig zu diagnostizieren, was sie um einiges schlimmer macht.
Das beste Mittel gegen schlechte Angewohnheiten ist sicherlich das \hyperref[c1ii7]{Üben mit getrennten Händen}.
Unmusikalisches Spielen ist eine der schlechten Angewohnheiten; vergessen Sie deshalb nicht, dass \hyperref[c1iii14d]{musikalisches Üben} mit dem Üben mit getrennten Händen beginnt.

\textbf{Eine weitere schlechte Angewohnheit ist der übermäßige Gebrauch des Halte- oder Dämpferpedals}, wie weiter unten besprochen\footnote{Anmerkungen zu den Bezeichnungen der Pedale finden Sie \hyperref[Pedale]{hier}}.
Das ist das sicherste Zeichen eines Amateurschülers, der Unterricht bei einem unqualifizierten Lehrer nimmt.
Zu häufiger Gebrauch dieser Pedale kann nur einem Schüler mit ernsthaften technischen Defiziten \enquote{helfen}.

\textbf{Eine weitere schlechte Angewohnheit ist, ohne Rücksicht auf die Musikalität auf das Klavier einzuhämmern.}
Der Schüler setzt laut mit aufregend gleich.
Dazu kommt es oft, wenn der Schüler so ins Üben vertieft ist, dass er vergisst, auf die Töne zu hören, die aus dem Klavier kommen.
Das kann vermieden werden, indem man die Angewohnheit entwickelt, sich stets selbst beim Spielen zuzuhören.
Sich selbst zuhören ist viel schwerer als vielen Menschen bewusst ist, weil viele Schüler (besonders diejenigen, die mit Stress spielen) ihre ganze Mühe für das Spielen aufwenden und nichts für das Zuhören übrig bleibt.
Eine Möglichkeit, dieses Problem zu verringern, ist, das \hyperref[c1iii13]{eigene Spielen aufzunehmen}, sodass man es sich mit einem gewissen geistigen Abstand anhören kann.
Aufregende Passagen sind oft laut, aber sie sind dann am aufregendsten, wenn der Rest der Musik leise ist.
Zu viel lautes Üben kann die technische Entwicklung, und dass man auf Geschwindigkeit kommt, verhindern und den Sinn für die Musik ruinieren.
Diejenigen, die laut spielen, haben am Ende oft einen schrillen Klang.

\textbf{Dann sind da noch diejenigen mit schwachen Fingern.}
Dieses Problem ist unter Anfängern weit verbreitet und kann einfacher korrigiert werden als das zu laute Draufhämmern.
Schwache Finger werden dadurch verursacht, dass man die Arme nicht entspannt und der Schwerkraft nicht die Führung überlässt.
Der Schüler hebt unbewusst die Arme, und diese Angewohnheit ist eine Form von Stress.
Diesen Schülern muss man den vollen Dynamikumfang des Klaviers zeigen und wie man ihn benutzt.

Ebenfalls eine schlechte Angewohnheit ist, mit der falschen Geschwindigkeit zu spielen, also entweder zu langsam oder zu schnell - besonders wenn Sie während eines \hyperref[c1iii14]{Auftritts} zu aufgeregt sind und das Gefühl für das Tempo verlieren.
Die richtige Geschwindigkeit wird von mehreren Faktoren bestimmt, einschließlich der Schwierigkeit des Stückes in Bezug auf Ihre technischen Fähigkeiten, was das Publikum erwartet, der Zustand des Klaviers, welches Stück vorausging oder welches diesem folgt usw.
Einige Schüler könnten dazu neigen, Stücke gemäß ihrer Fertigkeitsstufe zu schnell vorzuführen und viele Fehler zu machen, während andere schüchtern sind, zu langsam spielen und so nicht den vollen Gehalt der Musik hervorbringen.
Langsam zu spielen kann schwieriger sein, als mit der richtigen Geschwindigkeit zu spielen, was die Probleme eines schüchternen Spielers verschlimmert.
Diejenigen, die zu schnell vorspielen, können entmutigt werden, weil sie zu viele Fehler begehen, und zu der Überzeugung kommen, dass sie schlechte Klavierspieler sind.
Diese Probleme treffen nicht nur auf Vorführungen zu, sondern auch auf das Üben;
diejenigen, die zu schnell üben, glauben eventuell am Ende, dass sie schlechte Klavierspieler sind, weil sie so viele Fehler machen.
Nur etwas langsamer zu spielen kann dazu führen, dass sie genau und schön spielen und auf lange Sicht die Technik für das schnelle Spielen beherrschen.

Eine schlechte Klangqualität ist ein weiteres verbreitetes Problem.
Während der meisten Zeit hört beim Üben niemand zu, sodass der Klang keine Rolle zu spielen scheint.
Wenn der Klang ein wenig schlechter wird, stört es den Schüler nicht, mit dem Ergebnis, dass der Klang ignoriert wird.
Schüler müssen sich immer um den Klang bemühen, weil er der wichtigste Teil der Musik ist.
Auf einem schlechten oder nicht gut eingestellten Klavier kann man keinen guten Klang erzeugen;
das ist der Hauptgrund, warum man einen guten Flügel statt eines qualitativ schlechten Klaviers möchte und warum das \hyperref[c2_1]{Stimmen}, das Einstellen und das \hyperref[c2_7_hamm]{Intonieren der Hämmer} wichtiger sind als den meisten Schülern bewusst ist.
Gute Aufnahmen anzuhören ist der beste Weg, in dem Schüler das Bewusstsein für die Existenz des guten Klangs zu erwecken.
Wenn sie nur ihr eigenes Spiel anhören, haben sie eventuell keine Ahnung, was guter Klang bedeutet.
Achtet man jedoch erst einmal auf den Klang und fängt an, Resultate zu erzielen, verstärkt sich das selbst, und man kann ohne weiteres die Kunst lernen, Klänge zu produzieren, die ein Publikum anziehen.
Was noch wichtiger ist: Ohne einen guten Klang ist eine fortgeschrittene technische Verbesserung nicht möglich, weil ein guter Klang Kontrolle erfordert und die technische Entwicklung von der Kontrolle abhängt.

\textbf{Stottern} wird durch Üben im \enquote{Stop and Go} verursacht, wenn der Schüler bei jedem Fehler anhält und den Abschnitt noch einmal spielt.
\textbf{Wenn Sie einen Fehler machen, spielen Sie immer durch den Fehler hindurch; halten Sie nicht an, um ihn zu korrigieren.}
Machen Sie im Geiste einen Vermerk an der fehlerhaften Stelle, und spielen Sie den Abschnitt später noch einmal, um zu sehen, ob sich der Fehler wiederholt.
Wenn ja, fischen Sie ein kurzes Stück heraus, das den Fehler enthält, und arbeiten Sie damit.
Haben Sie erst einmal die Angewohnheit entwickelt, durch Fehler hindurchzuspielen, können Sie zur nächsten Stufe aufsteigen, in der Sie Fehler vorhersehen (ihr Kommen fühlen können, bevor sie auftreten) und Ausweichmanöver durchführen können, wie langsamer werden, den Abschnitt vereinfachen oder bloß den \hyperref[c1iii1b]{Rhythmus} beibehalten.
Meistens macht dem Publikum ein Fehler nichts aus, solange der Rhythmus nicht unterbrochen wird, oder es hört den Fehler nicht einmal.

\textbf{Das Schlimmste an den schlechten Angewohnheiten ist, dass es so lange dauert, sie zu eliminieren, besonders wenn sie das beidhändige Spielen betreffen.}
Deshalb beschleunigt nichts Ihre Lernrate mehr als die Kenntnis aller schlechten Angewohnheiten und deren Vermeidung, bevor sie verfestigt sind.
Zum Beispiel \textbf{ist die richtige Zeit, das Stottern zu verhindern, wenn der Schüler das erste Mal mit dem Unterricht beginnt.
Am Anfang stottern die meisten Schüler nicht;
man muss ihnen aber sofort beibringen, durch Fehler hindurchzuspielen.}
Wenn das Hindurchspielen durch Fehler in diesem Stadium gelehrt wird, wird es zur zweiten Natur und ist einfach;
es ist kein zusätzlicher Aufwand nötig, um diesen \enquote{Trick} zu lernen.
Einem Stotterer beizubringen, durch Fehler hindurchzuspielen, ist eine sehr schwierige Aufgabe.

Die Zahl der möglichen schlechten Angewohnheiten ist so groß, dass sie hier nicht alle angesprochen werden können.
Nur so viel sei gesagt: Eine rigorose Einstellung zu schlechten Angewohnheiten ist eine Voraussetzung für rasche Verbesserung.



<!-- c1ii23.html -->

\subsection{Haltepedal}
\label{c1ii23}

\textbf{Üben Sie jedes neue Stück ohne Pedal - erst \hyperref[c1ii7]{mit getrennten Händen}, dann beidhändig - bis sie beidhändig bei der endgültigen Geschwindigkeit gut zurechtkommen.
Das ist eine entscheidende Übungsmethode, die alle guten Lehrer bei all ihren Schülern benutzen.}
Es mag zunächst schwierig erscheinen, dort, wo das Pedal benötigt wird, ohne das Pedal \hyperref[c1iii14d]{musikalisch zu spielen};
das ist jedoch der beste Weg, die präzise Kontrolle zu lernen, sodass man musikalischer spielen kann, wenn man das Pedal schließlich hinzufügt.
Schüler, die von Anfang an mit dem Pedal üben, werden zu nachlässigen Spielern, entwickeln zahlreiche \hyperref[c1ii22]{schlechte Angewohnheiten} und werden nicht einmal das Konzept der präzisen Kontrolle oder der wahren Bedeutung der Musikalität lernen.

Reine Amateure benutzen das Haltepedal häufig zu oft.
\textbf{Die offensichtliche Regel ist: Wenn die Noten kein Pedal anzeigen, dann benutzen Sie es nicht.}
Bei einigen Stücken mag es so erscheinen, als wären sie mit Pedal leichter zu spielen (besonders dann, wenn man langsam und beidhändig anfängt), aber das ist eine der schlimmsten Fallen, in die ein Anfänger tappen kann, und wird die Entwicklung behindern.
Die Mechanik fühlt sich mit getretenem Haltepedal leichter an, weil der Fuß anstelle der Finger die Dämpfer von den Saiten weghält.
Deshalb fühlt sich die Mechanik schwerer an, wenn das Pedal angehoben ist, besonders bei schnellen Abschnitten.
Einige Schüler merken nicht, dass es an den Stellen, an denen kein Pedal angezeigt wird, gewöhnlich unmöglich ist, die Musik mit der vorgegebenen Geschwindigkeit korrekt zu spielen, wenn man das Pedal benutzt.

Benutzen Sie bei \enquote{Für Elise} das Pedal nur für die großen gebrochenen Akkord-Begleitungen der linken Hand (Takt 3 und ähnliche), die Takte 83-96 und die Arpeggio-Passage der rechten Hand (Takte 100-104).
Praktisch die ganze erste schwierige Unterbrechung sollte ohne das Pedal gespielt werden.
Natürlich sollte alles zunächst ohne das Pedal geübt werden, bis Sie mit dem Stück im Grunde fertig sind.
Das wird die gute Angewohnheit fördern, die Finger nahe bei den Tasten zu halten und die schlechte Angewohnheit unterbinden, mit zu häufigem Springen und Heben der Hände zu spielen und nicht fest in die Tasten zu drücken.
Ein wichtiger Grund, das Pedal am Anfang nicht zu benutzen, ist, dass die Technik sich ohne das Pedal am schnellsten verbessert, weil man ohne die Störung durch zuvor gespielte Noten genau hören kann, was man spielt.
Sie sollten den Klang aktiv kontrollieren.

Das Pedal und die Hände richtig zu koordinieren, ist keine leichte Aufgabe.
Deshalb enden Schüler, die ein Stück von Anfang an beidhändig mit dem Pedal lernen, ausnahmslos mit inkonsistenten und schlechten Pedalangewohnheiten.
Die korrekte Prozedur ist, erst einhändig ohne Pedal zu üben, dann einhändig mit Pedal, danach beidhändig ohne Pedal und zum Schluss beidhändig mit Pedal.
Auf diese Weise können Sie sich auf jede einzelne neue Fertigkeit konzentrieren, während Sie diese in Ihr Spiel einführen.

Unaufmerksamkeit dem Pedal gegenüber kann die technische Entwicklung viel mehr verzögern als vielen Schülern bewusst ist; umgekehrt kann Aufmerksamkeit dem Pedal gegenüber hilfreich für die technische Entwicklung sein, indem sie die Genauigkeit erhöht und der Musikalität eine weitere Dimension hinzufügt.
Wenn Sie eine Sache falsch machen, wird es schwierig, alle anderen Dinge richtig zu tun.
Wenn man mit dem Pedal etwas falsch macht, kann man noch nicht einmal die korrekte Fingertechnik üben, weil die Musik auch dann falsch klingt, wenn die Fingertechnik korrekt ist.

Das Pedal existierte vor Mozarts Zeit praktisch nicht;
so wird zum Beispiel im gesamten Werk von Johann Sebastian Bach kein Pedal benutzt.
Mozart hat nie ein Pedal angegeben, aber heutzutage wird in einigen seiner Kompositionen ein wenig Pedal als optional angesehen, und viele Herausgeber haben seinen Noten Pedalzeichen hinzugefügt.
Das Pedal war zu Beethovens Zeit zwar im Grunde voll entwickelt aber als ernsthaftes musikalisches Werkzeug noch nicht völlig akzeptiert.
Beethoven benutzte es mit großem Erfolg als besonderen Effekt (dritter Satz der Waldstein-Sonate);
deshalb benutzte er es oft in hohem Maß (gesamter erster Satz der Mondschein-Sonate) oder überhaupt nicht (gesamte Pathétique, erster und zweiter Satz der Waldstein-Sonate). Chopin benutzte das Pedal ausgiebig, um seiner Musik eine zusätzliche Logikebene hinzuzufügen und nutzte die verschiedenen Arten des Pedalgebrauchs vollständig aus.
Deshalb kann man Chopin (und viele spätere Komponisten) ohne ein entsprechendes Pedaltraining nicht korrekt spielen.

Schauen Sie in den \hyperref[reference]{Quellen} nach all den unterschiedlichen Arten, die Pedale zu benutzen, wann sie benutzt werden, und wie man die Bewegungen übt (\hyperref[Gieseking]{Gieseking und Leimer}, \hyperref[Fink]{Fink}, \hyperref[Sandor]{Sandor}, \textit{Pedaling the Modern Pianoforte} von Bowen und \textit{The Pianist's Guide to Pedaling} von Banowetz).
Versuchen Sie, alle diese Bewegungen zu beherrschen, bevor Sie das Pedal mit einem tatsächlichen Musikstück benutzen.
In den Quellen gibt es einige sehr hilfreiche Übungen für den richtigen Pedalgebrauch.
Wenn Sie das Pedal benutzen, müssen Sie genau wissen, welche Bewegung Sie benutzen und warum.
Wenn Sie zum Beispiel möchten, dass so viele resonante Saiten wie möglich mitschwingen, treten Sie das Pedal bevor Sie die Note spielen.
Wenn Sie jedoch nur eine klare Note aushalten möchten, treten Sie das Pedal nachdem Sie die Note spielen; je länger Sie das Pedal verzögern, desto weniger resonante Schwingungen werden Sie bekommen.
Im Allgemeinen sollten Sie sich angewöhnen, das Pedal einen Sekundenbruchteil nach dem Spielen der Note zu treten.
Sie können einen Legato-Effekt ohne zu viel Verschwimmen erzielen, indem Sie jedes Mal, wenn sich der Akkord ändert, das Pedal schnell anheben und wieder treten.
Wie bei den Tasten ist es genauso wichtig zu wissen, wann das Pedal angehoben werden muss, wie wann es getreten werden muss.
\textbf{Das Pedal muss genauso sorgfältig \enquote{gespielt} werden, wie man die Tasten spielt.}


\subsection{Dämpferpedal, Timbre und Eigenschwingungen vibrierender Saiten}
\label{c1ii24}

\textbf{Das Dämpferpedal wird bei einem Flügel benutzt, um die Stimmung des Klangs von einem perkussiven hin zu einem mehr gelassenen und sanften (bei getretenem Dämpferpedal) zu ändern.}
Es sollte nicht einzig zum Reduzieren der Lautstärke benutzt werden, weil es auch das Timbre ändert.
Um pianissimo zu spielen, muss man nur lernen, wie man leiser spielt.
Man kann bei getretenem Dämpferpedal sehr laute Töne erzeugen.
Eine Schwierigkeit mit dem Gebrauch des Dämpferpedals ist, dass es oft nicht angezeigt wird (una corda, oder richtiger due corda für den modernen Flügel), sodass die Entscheidung es zu benutzen oft dem Klavierspieler überlassen wird.
\textbf{Bei aufrecht stehenden Klavieren macht es den Klang hauptsächlich leiser.}
Das Dämpferpedal hat bei den meisten Klavieren nur einen unbedeutend kleinen Einfluss auf das Timbre.
Anders als der Flügel kann das Klavier mit getretenem Dämpferpedal keine lauten Klänge erzeugen.

\textbf{Viele Klavierspieler verstehen nicht, wie wichtig das richtige \hyperref[c2_7_hamm]{Intonieren der Hämmer} für das Funktionieren des Dämpferpedals ist.}
Wenn Sie dazu neigen, das Dämpferpedal zum leisen Spielen zu benötigen, oder wenn es deutlich leichter ist, pianissimo zu spielen, wenn der Deckel des Flügels geschlossen ist, dann ist es fast sicher, dass die Hämmer intoniert werden müssen.
Sehen Sie dazu den Abschnitt über das Intonieren in Abschnitt 7 von Kapitel 2.
Mit richtig intonierten Hämmern sollten Sie in der Lage sein, das leise Spielen ohne das Dämpferpedal in jedem gewünschten Maß zu kontrollieren.
Mit abgenutzten, verdichteten Hämmern ist leises Spielen unmöglich, und das Dämpferpedal hat eine geringere Auswirkung auf die Veränderung des Tons.
In den meisten Fällen können die ursprünglichen Eigenschaften des Hammers durch das Intonieren (Form erneuern, Nadeln usw.) wieder hergestellt werden.
Die Mechanik muss ebenfalls gut eingestellt sein, mit einem richtig minimierten Abgang, um \textit{ppp} zu ermöglichen.

\textbf{Der Gebrauch des Dämpferpedals ist umstritten, weil zu viele Klavierspieler nicht wissen, wie es funktioniert.}
Viele benutzen es zum Beispiel, um pianissimo zu spielen, was falsch ist.
Wie in Abschnitt 7 von Kapitel 2 gezeigt, ist der Energietransfer vom Hammer zur Saite beim Auftreffen am effizientesten, bevor die Saite anfängt, sich zu bewegen.
Ein verdichteter Hammer überträgt seine Energie innerhalb einer extrem kurzen Zeitspanne beim Auftreffen, und der Hammer springt sofort wieder von den Saiten zurück.
Diese hohe Effizienz der Energieübertragung erweckt den Eindruck, dass die Mechanik sehr leicht ist.
Deshalb gibt es alte Flügel, die sich federleicht anfühlen.
Weiche Hämmer auf demselben Klavier (ohne dass etwas anderes geändert wird) würden dazu führen, dass sich die Mechanik schwerer anfühlt.
Das deshalb, weil der Hammer wegen des weicheren Aufprallpunkts länger auf der Saite bleibt und die Saite aus ihrer ursprünglichen Position gehoben wird, bevor die ganze Energie des Hammers auf die Saite übertragen wurde.
In dieser Position ist der Energietransfer ineffizienter (siehe Abschnitt 7 in Kapitel 2), und der Spieler muss kräftiger drücken, um einen Ton mit derselben Lautstärke zu erzeugen.
\textbf{So kann das Intonieren das scheinbare Tastengewicht wirkungsvoller ändern als Bleigewichte.}
Klar wird das \textit{effektive} Tastengewicht nur teilweise von der zum Niederdrücken der Taste erforderlichen Kraft kontrolliert, da es auch von der Kraft abhängt, die notwendig ist, um eine bestimmte Tonstärke zu erzeugen.
Der Klavierspieler weiß nicht, welcher Faktor (Bleigewichte oder weicher Hammer) das effektive Tastengewicht beeinflusst.
Der Klaviertechniker muss einen Kompromiss eingehen.
Einerseits muss der Hammer genügend weich intoniert sein, um einen gefälligen Ton zu erzeugen, andererseits muss er genügend hart sein, um einen angemessenen Klang zu erzeugen.
Bei allen Flügeln und Klavieren, außer denen höchster Qualität, muss der Hammer eher hart sein, um einen genügend lauten Ton zu erzeugen und damit sich die Mechanik leicht beweglich anfühlt, was es erschwert, solche Klaviere leise zu spielen.
Das kann wiederum \enquote{gerechtfertigen}, das Dämpferpedal zu benutzen, wo es nicht benutzt werden sollte.
Klavierbesitzer, die das Intonieren vernachlässigen, können die Arbeit des Klavierstimmers erschweren, denn nachdem die Hämmer richtig intoniert sind, wird sich der Besitzer beschweren, dass die Mechanik nun zum Spielen zu schwer sei.
In Wahrheit hat sich der Besitzer daran gewöhnt, mit einer federleichten Mechanik zu spielen, und nie gelernt, wie man mit wahrer Kraft spielt, um diesen großartigen Klavierklang zu erzeugen.

Bei den meisten Klavieren bewirkt das Dämpferpedal, dass alle Hämmer näher zu den Saiten hin bewegt werden und so die Hammerbewegung begrenzt und die Lautstärke verringert wird.
Anders als bei Flügeln, können bei Klavieren keine lauten Töne erzeugt werden, wenn das Dämpferpedal getreten ist.
Ein Vorteil der Klaviere ist, dass ein teilweise getretenes Dämpferpedal die entsprechende Wirkung hat;
bei Flügeln ist das teilweise getretene Dämpferpedal ein komplexes Thema, das im folgenden behandelt wird.
Es gibt ein paar hochwertige Klaviere, bei denen das Dämpferpedal ähnlich funktioniert wie das der Flügel.

\textbf{Bei modernen Flügeln bewirkt das Dämpferpedal eine Verschiebung der gesamten Mechanik (einschließlich der Hämmer) nach rechts, sodass die Hämmer im dreisaitigen Abschnitt eine Saite auslassen.}
Dadurch treffen die Hämmer jeweils nur auf zwei Saiten, was eine herrliche Transformation im Klangcharakter verursacht, wie im Folgenden beschrieben.
Die Verschiebung beträgt genau den halben Abstand zwischen benachbarten Saiten im dreisaitigen Abschnitt);
dadurch treffen die beiden aktiven Saiten die weniger benutzten Bereiche des Hammers zwischen den Saitennuten, was einen noch weicheren Klang erzeugt.
Die horizontale Bewegung darf nicht einen ganzen Saitenabstand betragen, weil sonst die Saiten in die Hammernuten der benachbarten Saiten fallen würden.
Da die Saitenabstände und die Verschiebung nicht hinreichend genau kontrolliert werden können, würde dies dazu führen, dass einige Saiten genau in die Nuten fallen, während andere sie verpassen, was einen unausgewogenen Klang erzeugen würde.

Warum ändert sich das Timbre, wenn zwei statt drei Saiten angeschlagen werden?
Das Timbre wird hierbei von mindestens vier Faktoren bestimmt:

\begin{enumerate} 
 \item der Existenz der nicht angeschlagenen Saite,
 \item dem Verhältnis von Anschlagston und Nachklang,
 \item dem harmonischen Gehalt und
 \item der Polarisation der Schwingung der Saiten.
\end{enumerate}
Die nicht angeschlagene Saite dient als Reservoir, in das die beiden anderen Saiten ihre Energie abladen können, und erzeugt viele neue Effekte.
Da die Schwingung der dritten Saite in Gegenphase zu den angeschlagenen Saiten ist (eine angeregte Saite ist immer in Gegenphase zu dem \enquote{Anreger}), nimmt sie dem anfänglichen Anschlagsklang die Spitze (siehe unten), und zur gleichen Zeit erregt sie Schwingungen, die sich von denen unterscheiden, die sich ergeben, wenn alle drei vereint angeschlagen werden.
Deshalb funktioniert das Dämpferpedal in Klavieren nicht so gut\footnote{wie in Flügeln} - auch beim Treten des Dämpferpedals werden alle Saiten angeschlagen, und das Timbre kann sich nicht ändern.

Das Klavier erzeugt zunächst einen Anschlagston und dann einen anhaltenden Nachklang;
in den im \hyperref[reference]{Quellenverzeichnis} aufgeführten Artikeln \hyperref[Lectures]{Five Lectures on the Acoustics of the Piano} und aus dem \hyperref[American]{Scientific American} finden Sie mehr Details zu den in diesem Abschnitt besprochenen Themen.
Im Gegensatz zu dem vereinfachten Bild von Grund- und Obertönen, das wir beim \hyperref[c2_1]{Klavierstimmen} benutzen, bestehen die realen Saitenschwingungen aus einer komplexen zeitabhängigen Folge von Ereignissen, die immer noch nicht vollständig verstanden werden.
In solchen Situationen sind die tatsächlichen Daten existierender Klaviere von größerem praktischen Wert, aber diese Daten sind gut gehütete Betriebsgeheimnisse der Klavierhersteller.
Deshalb fasse ich hier das auf der Physik des Klavierklangs basierende allgemeine Wissen zusammen.
Die Schwingungen der Saiten können entweder parallel zum Resonanzboden oder senkrecht dazu polarisiert sein.
Wenn die Saiten angeschlagen werden, werden vertikal polarisierte wandernde Wellen erzeugt, die sich vom Hammer in beide Richtungen bewegen: zu den Agraffen (Capotasto) und zum Steg.
Diese Wellen wandern so schnell, dass sie mehrere hundert Mal von beiden Enden der Saiten reflektiert werden und den Hammer passieren, bevor der Hammer von den Saiten zurückspringt;
es sind in Wahrheit diese Wellen, die den Hammer zurückwerfen.
Durch die vertikalen Wellen werden horizontal polarisierte Wellen erzeugt, weil das Klavier nicht symmetrisch ist.
Diese wandernden Wellen werden zu stehenden Wellen reduziert, die aus dem Grundton und harmonischen Obertönen bestehen, weil die stehenden Wellen \enquote{Eigenschwingungen} (siehe ein Lehrbuch der Mechanik) sind, die langsam Energie zum Resonanzboden übertragen und deshalb lange anhalten.
Das Konzept von Grundtönen und Obertönen bleibt jedoch von Anfang an gültig, weil die Fourier-Koeffizienten (siehe ein Lehrbuch der Mathematik oder Physik) der Grund- und Obertonfrequenzen immer groß sind, sogar für die wandernden Wellen.
Das ist leicht zu verstehen, weil die Enden der Saiten sich nicht bewegen, besonders bei gut konstruierten, großen, schweren Klavieren.
Mit anderen Worten: Wenn die Enden fest sind, werden hauptsächlich Wellenlängen mit Knoten (Punkte ohne Bewegung) an beiden Enden erzeugt.
Das erklärt, warum Stimmer trotz der wandernden Wellen genau stimmen können, indem sie nur die Frequenzen der Grund- und Obertöne benutzen.
Die vertikal polarisierten Wellen übertragen die Energie effizienter auf den Resonanzboden als die horizontal polarisierten Wellen, erzeugen deshalb einen lauteren Ton, fallen aber schneller ab und erzeugen den Anschlagston.
Die horizontal polarisierten stehenden Wellen erzeugen den Nachklang, der dem Klavier sein langes Sustain verleiht.
Wenn das Dämpferpedal getreten wird, können nur zwei Saiten den Anschlagston erzeugen, aber irgendwann tragen alle drei Saiten zum Nachklang bei.
Deshalb ist das Verhältnis von Anschlagston und Nachklang kleiner als für drei Saiten, und der Klang ist weniger perkussiv.

Der harmonische Gehalt ist ebenfalls unterschiedlich, weil die Energie des Hammers nur auf zwei statt auf drei Saiten übertragen wird.
Das wirkt so, als ob die Saite mit einem schwereren Hammer angeschlagen wird, und es ist bekannt, dass schwerere Hämmer stärkere Grundtöne erzeugen.
Die Polarisation der Saiten ändert sich durch das Dämpferpedal ebenfalls, weil die dritte Saite mehr horizontal polarisiert wird, was zu dem sanfteren Klang beiträgt.

Dieses Wissen hilft uns dabei, das Dämpferpedal richtig zu benutzen.
Wenn das Pedal getreten wird, \textit{bevor} eine Note gespielt wird, werden die anfänglichen zeitabhängigen wandernden Wellen alle Saiten anregen, was ein sanftes Dröhnen im Hintergrund erzeugt.
Das heißt, dass beim Anschlagston die nicht harmonischen Fourier-Koeffizienten ungleich Null sind.
Wenn man seinen Finger auf irgendeine Saite legt, kann man fühlen, wie diese vibriert.
Oktav- und harmonische Saiten werden jedoch mit größeren Amplituden schwingen als die dissonanten Saiten.
Das ist eine Folge der höheren Fourier-Koeffizienten der Obertöne.
Dadurch schließt das Klavier nicht nur selektiv die Obertöne ein, sondern erzeugt sie auch selektiv.
Wenn nun das Pedal getreten wird, \textit{nachdem} die Note angeschlagen wird, werden die Oktav- und die harmonischen Saiten resonant mitschwingen aber alle anderen Saiten fast ganz still sein, weil die stehenden Wellen nur reine Obertöne enthalten.
Das erzeugt eine klare, ausgehaltene Note.
Die Lektion ist hier, dass das Pedal im Allgemeinen unverzüglich nach dem Anschlagen der Note getreten werden sollte, nicht vorher, um Dissonanzen zu vermeiden.
Das ist eine gute Angewohnheit, die es zu entwickeln gilt.

Ein teilweise getretenes Dämpferpedal funktioniert bei einem Klavier; aber kann man ein halbes Dämpferpedal bei einem Flügel benutzen?
Das sollte nicht umstritten sein, ist es aber, weil sogar einige fortgeschrittene Klavierspieler glauben, dass wenn ein vollständig getretenes Dämpferpedal einen bestimmten Effekt erzeugt, dann erzeugt ein teilweise getretenes Dämpferpedal einen teilweisen Effekt, was aber nicht stimmt.
Wenn man das Dämpferpedal teilweise tritt, wird man natürlich einen neuen Klang erhalten.
Es gibt keinen Grund, warum einem Klavierspieler nicht erlaubt sein sollte, dieses zu tun, und wenn es einen interessanten neuen Effekt erzeugt, der dem Klavierspieler gefällt, ist daran nichts falsch.
Diese Art zu spielen wurde jedoch nicht mit Absicht in das Klavier konstruiert, und ich weiß von keinem Komponisten, der etwas für teilweise getretenes Dämpferpedal auf dem Flügel komponiert hätte, insbesondere da es nicht auf mehreren Klavieren und bei mehreren Noten innerhalb eines Klaviers in gleicher Weise reproduziert werden kann.
Der übermäßige Gebrauch des teilweisen Dämpferpedals auf dem Flügel führt dazu, dass einige Saiten eine Seite der Hämmer abrasieren, was das System aus der Einstellung bringt.
Auch ist es für den Klaviertechniker unmöglich, alle Hämmer und Saiten so genau auszurichten, dass jeweils die dritte Saite bei allen dreisaitigen Noten den Hammer bei derselben Pedalstellung verpasst.
Dadurch wird der Effekt des teilweisen Dämpferpedals ungleichmäßig und von Klavier zu Klavier unterschiedlich.
Deshalb ist das halbe Treten des Dämpferpedals auf einem Flügel nicht empfehlenswert, es sei denn, Sie haben damit experimentiert und versuchen, damit einen fremdartigen und nicht reproduzierbaren neuen Effekt zu erzeugen.
Nichtsdestoweniger zeigen anekdotenhafte Berichte, dass es den Gebrauch des teilweisen Dämpferpedals auf einem Flügel gibt; fast immer wegen der Unwissenheit des Klavierspielers über die Funktionsweise dieses Teils.
Die einzige Möglichkeit, ein teilweises Dämpferpedal mit reproduzierbaren Ergebnissen zu benutzen, ist, es nur ganz wenig zu treten; in diesem Fall werden alle Saiten auf die Seiten der Nuten in den Hämmern treffen.
Aber sogar dieses Verfahren funktioniert nicht wirklich, weil es nur den dreisaitigen Abschnitt beeinflusst und zu einem misstönenden Übergang vom zweisaitigen zum dreisaitigen Abschnitt führt.

Im ein- und zweisaitigen Abschnitt haben die Saiten einen viel größeren Durchmesser, sodass die Saiten die Seitenwände der Nuten treffen, wenn sich die Mechanik seitwärts bewegt, was ihnen eine horizontale Bewegung verleiht und die Nachklangkomponente vergrößert, indem die horizontal polarisierten Schwingungen der Saiten verstärkt werden.
Dadurch ist die Veränderung des Timbres der im dreisaitigen Abschnitt ähnlich.
Dieser Mechanismus ist geradezu genial!

Zusammenfassend ist der Name Dämpferpedal beim Flügel eine unzutreffende Bezeichnung.
Seine hauptsächliche Wirkung ist die Veränderung des Timbres des Klangs.
Wenn Sie einen lauten Ton mit getretenem Dämpferpedal spielen, wird er fast so laut sein wie ohne Dämpferpedal.
Das kommt daher, dass Sie ungefähr die gleiche Energiemenge in die Erzeugung des Tons gesteckt haben.
Auf der anderen Seite ist es auf den meisten Flügeln leichter leise zu spielen, wenn man das Dämpferpedal benutzt, weil die Saiten die weniger benutzten, weichen Teile der Hämmer treffen.
Vorausgesetzt, das Klavier ist gut eingestellt und die Hämmer sind richtig \hyperref[c2_7_hamm]{intoniert}, sollten Sie ohne Dämpferpedal in der Lage sein, genauso leise zu spielen.
\textbf{Ein teilweise getretenes Dämpferpedal wird unvorhersehbare, ungleichmäßige Wirkungen erzielen und sollte auf einem akustischen Flügel nicht benutzt werden.}
Ein teilweises Dämpferpedal funktioniert auf den meisten Klavieren und allen Digitalpianos.


% zuletzt geändert 05.12.2009


<!-- c1ii25.html -->

\subsection{Mit beiden Händen zusammen (\hyperref[HsHt]{HT}) üben und mental spielen}
\label{c1ii25}

\subsubsection{Einführung}
\label{c1ii25a}

\textbf{Wir können endlich damit anfangen, die Hände zusammenzuführen!
Dabei bekommen einige Schüler die meisten Probleme, besonders in den ersten paar Jahren des Klavierunterrichts.}
Obwohl die hier vorgestellten Methoden Ihnen sofort helfen sollten, sich die Technik schneller anzueignen, wird es ungefähr zwei Jahre dauern, bis Sie in der Lage sind, wirklich einen Vorteil aus allem zu ziehen, das die Methoden dieses Buchs zu bieten haben.

Beidhändig zu spielen ist fast wie zu versuchen, an zwei unterschiedliche Dinge gleichzeitig zu denken.
Es gibt keine bekannte, vorprogrammierte Koordination der zwei Hände, wie wir sie zwischen unseren beiden Augen (für das Abschätzen von Entfernungen), unseren Ohren (für die Bestimmung der Richtung eines ankommenden Geräuschs) oder unseren Armen und Beinen (zum Gehen) haben.
Deshalb wird es ein wenig Arbeit erfordern, das genaue Koordinieren der Finger der beiden Hände zu lernen.
Die vorangegangene Beschäftigung mit dem \hyperref[c1ii7]{einhändigen Spielen} macht das Lernen dieser Koordination viel leichter, weil wir uns nur auf das Koordinieren konzentrieren müssen und nicht gleichzeitig auf das Koordinieren \textit{und} auf das Entwickeln von Finger- und Handtechnik.

Die gute Nachricht ist, dass es nur ein grundlegendes \enquote{Geheimnis} dafür gibt, das beidhändige Spielen schnell zu lernen.
Dieses \enquote{Geheimnis} ist ein angemessenes \hyperref[c1ii7]{Arbeiten mit getrennten Händen}, Sie kennen es also bereits!
\textbf{Die ganze Technik muss einhändig erworben werden;
versuchen Sie nicht, Technik beidhändig zu erwerben, die Sie einhändig erwerben können.}
Mittlerweile sollten die Gründe offensichtlich sein.
Wenn Sie versuchen, Technik beidhändig zu erlangen, die Sie sich einhändig aneignen können, werden Sie Probleme bekommen wie:

\begin{enumerate}[label={\arabic*.}] 
 \item Stress entwickeln,
 \item Ungleichgewicht der Hände (die rechte Hand tendiert dazu stärker zu werden),
 \item \hyperref[c1ii22]{schlechte Angewohnheiten} aneignen,
 \item Geschwindigkeitsbarrieren erzeugen, usw.
 \end{enumerate}

Beachten Sie, dass alle Geschwindigkeitsbarrieren \textit{erzeugt} werden; sie resultieren aus unkorrektem Spielen oder Stress.
Vorzeitiges beidhändiges Üben kann eine beliebige Zahl von Geschwindigkeitsbarrieren erzeugen.
Falsche Bewegungen sind ein weiteres Hauptproblem; einige Bewegungen scheinen keine Probleme zu bereiten, wenn man langsam beidhändig spielt, werden aber unmöglich, wenn man die Geschwindigkeit steigert.
Das beste Beispiel ist das Spielen mit \hyperref[c1iii5a]{untersetztem Daumen}.

Als erstes benötigen Sie ein Kriterium für die Entscheidung, wann Sie ausreichend einhändig geübt haben.
Ein gutes Kriterium ist die einhändige Geschwindigkeit.
Typischerweise ist die maximale beidhändige Geschwindigkeit, mit der Sie spielen können, 50 bis 90\% der langsameren einhändigen Geschwindigkeit, entweder die der rechten oder der linken Hand.
Angenommen, Sie können mit der rechten Hand mit der Geschwindigkeit 10 und mit der linken Hand mit der Geschwindigkeit 9 spielen.
Dann mag Ihre maximale beidhändige Geschwindigkeit 7 sein.
Die schnellste Möglichkeit, die beidhändige Geschwindigkeit auf 9 zu erhöhen, wäre, die Geschwindigkeit der rechten Hand auf 12 zu erhöhen und die der linken Hand auf 11.
Als eine allgemeine Regel gilt: Bringen Sie die einhändige Geschwindigkeit um einiges über die endgültige Geschwindigkeit.
Deshalb ist das Kriterium, nach dem wir gesucht haben: Wenn Sie einhändig mit 110 bis 150\% der endgültigen Geschwindigkeit entspannt und kontrolliert spielen können, dann sind Sie bereit zum beidhändigen Üben.


\label{notenweise}

Falls Sie immer noch Probleme haben, benutzen Sie das \hyperref[c1iii8]{Konturieren}.
Angenommen, Sie können zufriedenstellend einhändig spielen.
Vereinfachen Sie nun eine oder beide Hände, sodass Sie leicht beidhändig spielen können, und fügen Sie dann die gelöschten Noten schrittweise hinzu.
Dafür gibt es viele Möglichkeiten, und Sie können in Abhängigkeit Ihrer Kenntnisse der Musiktheorie wirklich mächtige Methoden entwickeln, weshalb das Konturieren in Abschnitt III.8 detaillierter besprochen wird.
Sie benötigen jedoch keine Theorie, um das Konturieren zu benutzen; ein Beispiel ist die Methode, Noten hinzuzufügen: Nehmen Sie einen kurzen Ausschnitt des schwierigen Abschnitts.
Spielen Sie den Ausschnitt wiederholt einhändig mit der schwierigeren Hand (siehe \enquote{\hyperref[c1iii2]{Zyklisch spielen}} in Abschnitt III.2).
Fangen Sie nun an, die leichtere Hand Note für Note hinzuzufügen.
Fügen Sie zuerst nur eine Note hinzu, und üben Sie, bis Sie es zufriedenstellend spielen können.
Fügen Sie dann eine weitere hinzu usw., bis der Abschnitt vollständig ist.
Achten Sie darauf, dass Sie beim Hinzufügen der Noten denselben Fingersatz wie beim einhändigen Üben benutzen.
Sehr oft ist der Grund, warum Sie nicht beidhändig spielen können, obwohl Sie einhändig spielen können, dass irgendwo noch ein Fehler ist.
Dieser Fehler liegt häufig im \hyperref[c1iii1b]{Rhythmus}.
Versuchen Sie deshalb beim Hinzufügen der Noten herauszufinden, ob bei einer Hand ein Fehler vorliegt;
sehen Sie dazu am besten in den Noten nach.

Es gibt jede Menge Unterschiede in der Art und Weise, wie das Gehirn Aufgaben mit einer Hand bewältigt und wie Aufgaben, die eine Koordination der beiden Hände erfordern, weshalb es sich auszahlt, beides getrennt voneinander zu lernen.
Einhändiges Üben neigt nicht dazu, Angewohnheiten zu formen, die nicht direkt vom Gehirn kontrolliert werden, weil das Gehirn jede Hand direkt kontrolliert.
Auf der anderen Seite können beidhändige Bewegungen nur durch Wiederholungen entwickelt werden; diese erzeugen eine reflexartige Angewohnheit und beziehen eventuell Nervenzellen außerhalb des Gehirns mit ein.
Ein Anzeichen dafür ist die Tatsache, dass es länger dauert, beidhändige Bewegungen zu lernen.
\textbf{Deshalb sind schlechte beidhändige Angewohnheiten die schlimmsten, weil es sehr lange dauert, sie zu eliminieren.
Um die Technik schnell zu erwerben, müssen Sie diese Kategorie schlechter Angewohnheiten vermeiden.}

Das \hyperref[c1ii12mental]{mentale Spielen} ist sowohl für das beidhändige als auch das einhändige Spielen notwendig, aber das beidhändige mentale Spielen ist natürlich für Anfänger schwieriger.
Wenn Sie das mentale Spielen erst einmal beherrschen, ist es einhändig und beidhändig gleich einfach.
Da Sie das mentale Spielen bereits einhändig kennen (siehe Abschnitt 12), ist die verbleibende Hauptaufgabe, es beidhändig zu lernen.
Beim einhändigen Auswendiglernen für das mentale Spielen werden Sie in jeder Komposition auf Stellen gestoßen sein, die Sie am Klavier prüfen mussten - Sie können sie auf dem Klavier spielen aber nicht in Gedanken -, diese Stellen haben Sie sich noch nicht vollständig gemerkt.
Das sind die Stellen, an denen Sie während einer Aufführung Gedächtnisblockaden haben könnten.
Um zu testen, ob Ihr mentales Spielen solide ist, prüfen Sie, ob Sie drei Dinge in Gedanken können:

\begin{enumerate}[label={\arabic*.}] 
 \item Können Sie an einer beliebigen Stelle im Stück beginnen und beidhändig spielen?
 \item Können Sie bei einem beliebigen Abschnitt, den Sie mit der einen Hand spielen, den Abschnitt der anderen Hand hinzufügen?
 \item Können Sie mit beiden Händen gleichzeitig spielen?
 \end{enumerate}
Sie sollten feststellen, dass Sie, wenn Sie das in Gedanken können, es auch leicht auf dem Klavier können.

Zeigen wir nun das beidhändige Üben an praktischen Beispielen.
Ich habe drei Beispiele gewählt, um die beidhändigen Methoden zu veranschaulichen, angefangen mit dem einfachsten, dem \hyperref[c1ii25b]{ersten Satz von Beethovens Mondschein-Sonate}, dann \hyperref[c1ii25c]{Mozarts Rondo Alla Turca}, und zum Schluss die anspruchsvolle \hyperref[c1ii25d]{Fantaisie-Impromptu (FI) von Chopin}.
Sie sollten sich das Stück aussuchen, das Ihrer Fertigkeitsstufe am nächsten kommt.
Sie könnten auch Bachs Inventionen versuchen, die in den Abschnitten \hyperref[c1iii6l]{III.6l} und \hyperref[c1iii19]{III.19} detailliert behandelt werden.
Ich überlasse es Ihnen, es mit dem oben besprochenen \enquote{Für Elise} selbst zu versuchen, da es ziemlich kurz und relativ geradlinig ist.
Für viele Klavierspieler ist \enquote{Für Elise} zu \enquote{bekannt} und oft schwierig zu spielen; spielen Sie es in diesem Fall in gedämpfter Weise, konzentrieren Sie sich auf die Genauigkeit statt auf die Emotionen (kein Rubato), und lassen Sie die Musik für sich selbst sprechen.
Mit dem richtigen Publikum kann das sehr wirkungsvoll sein.
Dieses \enquote{unbeteiligte} Spielen kann bei populären, bekannten Stücken nützlich sein.

Die hier ausgewählten drei Kompositionen stellen ihre besonderen Anforderungen.
Die Mondschein-Sonate erfordert Legato, \textit{pp} und die Musik von Beethoven.
Das Alla Turca muss nach Mozart klingen, ist ziemlich schnell und erfordert eine genaue, unabhängige Kontrolle der Hände sowie ein solides Oktavspiel.
Die FI erfordert die Fähigkeit, beidhändig 4 gegen 3 und 2 gegen 3 zu spielen, extrem schnelle Fingersätze für die rechte Hand, die Romantik von Chopin und genaues Pedalieren.
Alle drei können relativ leicht beidhändig mental gespielt werden, weil die linke Hand hauptsächlich eine Begleitung der rechten Hand ist; in Bachs Inventionen spielen beide Hände die Hauptrollen, und es ist schwieriger, beidhändig mental zu spielen.
Das zeigt, dass Bach wahrscheinlich das mentale Spielen gelehrt und absichtlich anspruchsvolle Stücke für seine Schüler komponiert hat.
Diese erhöhte Schwierigkeit erklärt auch, warum einige Schüler die Inventionen - ohne eine entsprechende Anleitung (wie dieses Buch) - extrem schwierig auswendig zu lernen und mit der richtigen Geschwindigkeit zu spielen finden.


\subsubsection{Beethovens Mondschein-Sonate, 1. Satz, Op. 27, No. 2}
\label{c1ii25b}

Die bedeutendste Auseinandersetzung über diesen Satz dreht sich um den Gebrauch des Pedals.
Beethovens Anweisung \enquote{senza sordini} - übersetzt \enquote{ohne Dämpfer} - bedeutet, dass das Haltepedal vom Anfang bis zum Ende getreten sein sollte.
Die meisten Klavierspieler folgen dieser Anweisung \textit{nicht}, weil das Sustain moderner Konzertflügel so lang ist (viel länger als bei Beethovens Klavier), dass das Vermischen all dieser Noten ein Hintergrundgetöse erzeugt, das in der konventionellen Klavierpädagogik als roh angesehen wird.
Sicherlich wird kein Klavierlehrer einem Schüler das erlauben!
Beethoven war jedoch nicht nur Extremist, sondern liebte es auch, die Regeln zu brechen.
Die Mondschein-Sonate basiert auf dem Kontrast.
Der erste Satz ist langsam, legato, mit Pedal und leise.
Der dritte Satz ist das genaue Gegenteil; er ist einfach eine Variation des ersten Satzes, die sehr schnell und agitato gespielt wird - das wird durch die Beobachtung bestätigt, dass die oberste Doppeloktave in Takt 2 des dritten Satzes eine verkürzte Form des dreinotigen Hauptthemas des ersten Satzes ist, das unten besprochen wird (in \hyperref[Arpeggios]{Abschnitt III.5e} finden Sie weitere Informationen zum dritten Satz).
Es besteht auch ein starker Kontrast zwischen den Dissonanzen und den klaren Harmonien, die diesem ersten Satz seine berühmte Qualität verleihen.
Die Hintergrunddissonanz wird durch das Pedal erzeugt, sowie durch die Nonen usw.
Die Dissonanzen haben den Zweck, die Harmonien wie einen funkelnden Diamanten auf einem dunklen Samtuntergrund hervorzuheben.
Als Extremist wählte Beethoven das harmonischste mögliche Thema: eine Note, die dreimal wiederholt wird (Takt 5)!
Deshalb ist meine Interpretation, dass das Pedal während des ganzen Satzes unten sein sollte, so wie Beethoven es angab.
Bei den meisten Klavieren sollte das keine Probleme verursachen;
bei Konzertflügeln wird das jedoch schwierig, weil der Hintergrundlärm während des Spielens lauter wird und man immer noch \textit{pp} (\enquote{sempre pianissimo}) spielen muss; in diesem Fall könnten Sie den Hintergrund ein wenig reduzieren, bringen Sie ihn aber nie völlig zum Verstummen, denn er ist ein Teil der Musik.
Auf diese Art werden Sie es nicht auf Aufnahmen hören, bei denen der Schwerpunkt üblicherweise auf den klaren Harmonien liegt und der Hintergrund eliminiert wird - die \enquote{Standardkonvention} für den \enquote{korrekten} Pedalgebrauch.
Beethoven mag sich hierbei jedoch dafür entschieden haben, diese Regel zu brechen.
Deshalb setzte er im ganzen Satz keine Pedalzeichen - weil man es nie anheben soll.
\textbf{Da wir uns entschieden haben, das Haltepedal die ganze Zeit getreten zu lassen, ist die erste in diesem Stück zu lernende Regel, dass man das Pedal nicht benutzt, bis man zufriedenstellend beidhändig spielen kann.}
Das wird Sie in die Lage versetzen, zu lernen, wie man legato spielt, was man nur ohne das Pedal üben kann.
Obwohl es sehr leise gespielt wird, besteht keine Notwendigkeit, in diesem Stück das Dämpferpedal zu benutzen; außerdem ist bei den meisten Übungsklavieren die Mechanik mit getretenem Dämpferpedal nicht genügend leichtgängig, um bei \textit{pp} die gewünschte Kontrolle zu ermöglichen.

Fangen Sie damit an, dass Sie, sagen wir die Takte 1-5, einhändig auswendig lernen und anschließend sofort mental spielen.
Achten Sie auf alle Ausdruckszeichen.
Das Stück ist im Halbetakt, aber die ersten beiden Takte sind wie eine Einleitung und haben jeweils nur eine Oktavnote in der linken Hand; der Rest wird strenger im Halbetakt gespielt.
Beethoven sagt uns sofort, in Takt 2, dass die Dissonanz eine wichtige Komponente dieses Satzes sein wird, indem er die H-Oktave in der linken Hand einfügt und das Publikum mit einer Dissonanz schockt.
Fahren Sie mit dem Auswendiglernen abschnittsweise bis zum Ende fort.

Die Oktaven der linken Hand müssen \textit{gehalten} werden.
Spielen Sie zum  Beispiel die C\#-Oktave der linken Hand von Takt 1 mit den Fingern 51; ersetzen Sie die 5, indem Sie sofort Finger 4 und dann Finger 3 auf das untere C\# gleiten lassen und dabei das untere C\# unten halten.
Sie werden die Oktave mit 31 halten, bevor Sie Takt 2 erreichen.
Halten Sie nun die 3, während Sie die H-Oktave von Takt 2 mit 51 spielen.
Auf diese Weise behalten sie beim \textit{Abwärtsgehen} völlig das Legato in der linken Hand bei.
Mit diesem Verfahren können Sie das Legato mit Finger 1 nicht ganz aufrechterhalten, halten Sie diesen aber so lange wie Sie können.
Beim Übergang von Takt 3 zu 4 muss die Oktave der linken Hand \textit{aufwärts gehen}.
Spielen Sie in diesem Fall das F\# von Takt 3 mit 51, halten Sie dann die 5 und spielen Sie die nächste G\#-Oktave mit 41.
Spielen Sie ähnlich von Takt 4 zu 5 die zweite G\#-Oktave von Takt 4 mit 51, und ersetzen Sie dann Finger 1 mit 2, während Sie ihn unten halten (Sie müssen eventuell Finger 5 anheben), sodass Sie den folgenden Akkord von Takt 5 mit den Fingern 521 spielen und das Legato aufrechterhalten können.
Die allgemeine Idee ist, so viele Noten wie möglich zu halten, besonders die untere Note der linken und die obere Note der rechten Hand.
Es gibt gewöhnlich verschiedene Möglichkeiten, dieses \enquote{Halten} auszuführen, Sie sollten deshalb mit Ihnen experimentieren, um zu sehen, welche in einer bestimmten Situation am besten passt.
Die Wahl des jeweiligen Halteverfahrens hängt hauptsächlich von der Größe Ihrer Hand ab.
Die Oktave der linken Hand von Takt 1 könnte zum Beispiel mit 41 oder 31 gespielt werden, sodass Sie keine Finger ersetzen müssen; das hat den Vorteil der Einfachheit, hat aber den Nachteil, dass man sich daran erinnern muss, wenn man das Stück beginnt.
Benutzen Sie das \enquote{Ersetzen der Finger} während des Stücks, um so viel Legato wie möglich durchzuhalten.
\textbf{Sie müssen sich für eine bestimmte Methode des Ersetzens entscheiden, wenn Sie das Stück zum ersten Mal auswendig lernen, und dann immer dieselbe benutzen.}

Warum soll man die Noten legato halten, wenn man schließlich doch alle Noten mit dem Pedal aushält?
Erstens hängt wie Sie die Taste herunterdrücken davon ab, wie Sie sie unten halten; deshalb können Sie durch das Halten ein konsistenteres und zuverlässigeres Legato spielen.
Zweitens lässt der Fänger, wenn Sie die Taste anheben und die Note mit dem Pedal halten, den Hammer frei, was diesem gestattet umherzuspringen, und diese \enquote{Lockerheit} der Mechanik ist hörbar - die Natur des Klangs ändert sich.
Als Herr des Klaviers möchten Sie immer, dass der Fänger den Hammer festhält, sodass Sie die völlige Kontrolle über die gesamte Klaviermechanik haben.
Dieser Grad der Kontrolle ist extrem wichtig, wenn man \textit{pp} spielt - man kann das \textit{pp} nicht kontrollieren, wenn der Hammer umherspringt.
Ein weiterer Grund für das Halten ist, dass es zu einer absoluten Genauigkeit führt, weil Ihre Hand nie die Tastatur verlässt und die gehaltene Note als Referenz zum Finden der nachfolgenden Noten dient.

Musik - wie macht man Musik?
Takt 1 ist nicht nur eine Folge von vier Triolen.
Sie müssen logisch \textit{verbunden} werden; achten Sie deshalb auf die Verbindung zwischen der obersten Note jeder Triole und der untersten Note der nächsten Triole.
Diese Verbindung ist besonders wichtig, wenn man von einem Takt zum nächsten übergeht, und die unterste Note hat oft eine melodische Bedeutung, wie in den Takten 4-5, 9-10 usw.
Die rechte Hand von Takt 5 beginnt mit der tiefsten Note, E, und die Musik steigt kontinuierlich bis zum G\# des dreinotigen Themas.
Dieses Thema sollte nicht \enquote{alleine} gespielt werden, sondern ist der Höhepunkt des arpeggioartigen Anstiegs der vorangegangenen Triole.
Wenn Sie in Takt 8 Schwierigkeiten haben, die None mit der rechten Hand zu greifen, spielen Sie die untere Note mit der linken Hand - ähnlich in Takt 16.
In diesen Fällen können Sie das Legato in der linken Hand nicht völlig beibehalten, aber das Legato in der rechten Hand ist wichtiger, und das Anheben der linken Hand kann weniger hörbar gemacht werden, wenn Sie später das Pedal benutzen.
Wenn Sie die None jedoch leicht greifen können, sollten Sie versuchen, sie nur mit der rechten Hand zu spielen, weil Ihnen das erlaubt, mit der linken Hand mehr Noten zu halten.
Obwohl die erste Note des dreinotigen Themas eine G\#-Oktave ist, sollte die obere Note von der unteren Note unterschieden und lauter als diese gespielt werden.
Die Takte 32-35 sind eine Folge aufsteigender Triolen mit ansteigender Spannung.
Die Takte 36-37 sollten verbunden werden, weil sie ein weiches Lösen dieser Spannung sind.

Der Anfang ist \textit{pp} bis Takt 25, dort folgt ein Crescendo, dann ein Decrescendo zum \textit{p} in Takt 28 und schließlich die Rückkehr zum \textit{pp} in Takt 42.
Bei den meisten \textit{cresc.} und \textit{decresc.} sollte der größte Teil des Anstiegs oder Abfalls dem Ende zu erfolgen, nicht dem Anfang zu.
In Takt 48 gibt es ein unerwartetes Crescendo und bei der ersten Note von Takt 49 einen abrupten Sprung zu \textit{p}.
Das ist das deutlichste Zeichen, dass Beethoven eine klare Harmonie wollte, die einen vom Pedal erzeugten dissonanten Hintergrund überlagert.

Der \enquote{Schluss} beginnt bei Takt 55.
Beachten Sie den Halbetakt; betonen Sie besonders den ersten und dritten Schlag von Takt 57.
Was als ein normaler Schluss erscheint, wird durch die (falschen) Akzentzeichen auf dem vierten Schlag von Takt 58 und dem dritten Schlag von Takt 59 angezeigt.
Der erste Akkord von Takt 60 ist ein falscher Schluss.
Die meisten Komponisten würden das Stück hier beendet haben; es ist derselbe Akkord wie der erste Akkord dieses Satzes - ein Merkmal eines Standardschlusses.
Beethoven benutzte jedoch oft den doppelten Schluss, was den richtigen Schluss \enquote{endgültiger} macht.
Er nimmt den Schlag sofort wieder auf und führt Sie zu dem wahren Ende, indem er eine mit der linken Hand und \textit{pp} gespielte nostalgische Reprise des dreinotigen Themas benutzt.
Die letzten beiden Akkorde sollten die leisesten Noten des ganzen Satzes sein; das ist schwierig, weil sie so viele Noten beinhalten.

Beim beidhändigen Spielen bereitet dieser Satz keine Probleme.
Das einzige neue Element ist das Halten der Noten für das Legato, das eine zusätzliche gleichzeitige Kontrolle der Hände erfordert.

Wenn Sie den ganzen Satz auswendig gelernt haben und ihn zufriedenstellend beidhändig spielen können, fügen Sie das Pedal hinzu.
Wenn Sie sich dafür entscheiden, das Pedal die ganze Zeit getreten zu halten, kann die Melodie der oberen Noten in den Takten 5-9 als ätherische Erscheinung gespielt werden, die eine durch die Akkordprogression erzeugte Hintergrunddissonanz überlagert.
Beethoven wählte diese Konstruktion wahrscheinlich, um den vollen Klang der neuen Klaviere seiner Zeit zu demonstrieren und ihre Fähigkeiten zu erkunden.
Diese Beobachtung stützt die These, dass der dissonante Hintergrund nicht völlig durch das wohlüberlegte Anheben des Pedals eliminiert werden sollte.


\subsubsection{Mozarts Rondo Alla Turca, aus Sonate K300 (KV331)}
\label{c1ii25c}

Ich gehe davon aus, dass Sie \hyperref[c1ii7]{einhändig} bereits die Hausaufgaben gemacht haben, und beginne mit dem beidhändigen Teil, insbesondere weil das einhändige Spielen bei den meisten Stücken von Mozart relativ einfach ist.
Ich werde hauptsächlich die technischen Schwierigkeiten und \enquote{wie man es wie Mozart klingen lässt} behandeln.
Bevor wir mit den Einzelheiten anfangen, besprechen wir die Form der ganzen Sonate, denn wenn Sie den letzten Abschnitt lernen, entscheiden Sie sich vielleicht dafür, das ganze Stück zu lernen - es gibt keine einzige Seite dieser Sonate, die nicht faszinierend ist.

Der Begriff Sonate wurde für so viele Arten von Musik verwandt, dass er keine bestimmte Definition hat; er hat sich im Laufe der Zeit weiterentwickelt und verändert.
In frühester Zeit bedeutete Sonate einfach Musikstück oder Lied.
\textbf{Vor und während Mozarts Zeit bedeutete \enquote{Sonate} eine Instrumentalmusik mit einem bis vier Teilen, bestehend aus Sonatenhauptsatz, Menuett, Trio, Rondo usw.
Eine Sonatine ist eine kleine Sonate.
Der Sonatenhauptsatz wurde zunächst als erster Teil einer Sonate, einer Symphonie, eines Konzerts usw. entwickelt; er bestand im Allgemeinen aus einer Einführung\footnote{eventuell wiederholt}, einer Entwicklung und einer Reprise\footnote{ABA- oder AABA-Format}.}
Der Sonatenhauptsatz ist historisch wichtig, weil diese Grundstruktur schrittweise in die meisten Kompositionen einfloss. 
Seltsamerweise ist kein Teil dieser Mozart-Sonate (No. 16, K300) in der Sonatenhauptsatzform (Hinson, Seite 552).
Sie beginnt mit einem Thema und sechs Variationen.
Variation V ist Adagio und sollte nicht zu schnell gespielt werden.
Danach kommt eine Unterbrechung, die dem mittleren oder langsamen Satz einer Beethoven-Sonate entspricht.
Diese Unterbrechung hat die Form eines Menuett-Trios, einer Tanzform.
Das Menuett war ursprünglich ein französischer Hoftanz mit drei Schlägen und war der Vorläufer des Walzers.
Das Walzerformat schließt auch die Mazurkas ein; diese waren ursprünglich polnische Tänze, weshalb Chopin so viele Mazurkas komponierte.
Sie unterscheiden sich von den (Wiener) Walzern, die den Akzent auf dem ersten Schlag haben, dadurch, dass ihr Akzent auf dem zweiten oder dritten Schlag liegen kann.
Der Walzer begann unabhängig davon in Deutschland als langsamerer Tanz mit drei betonten Schlägen; er entwickelte sich dann zu dem populären Tanz, den wir nun als \enquote{Wiener} bezeichnen.
Die Trios traten schrittweise in den Hintergrund, als die Quartette an Popularität gewannen.
Sowohl das Menuett als auch das Trio in unserer Sonate haben die Taktart 3/4.
Deshalb trägt jede erste Note den betonten Schlag; zu wissen, dass es im Tanzformat (Walzer) ist, vereinfacht es, das Menuett-Trio richtig zu spielen.
Das Trio sollte einen vom Menuett völlig verschiedenen Eindruck vermitteln (eine Konvention in Mozarts Zeit); diese Veränderung des Eindrucks verleiht dem Übergang ein erfrischendes Gefühl.
\enquote{Trio} bezieht sich im Allgemeinen auf ein Stück, das mit drei Instrumenten gespielt wird; deshalb gibt es in dem Trio drei Stimmen, die man einer Violine, einer Viola und einem Cello zuordnen kann.
Vergessen Sie nicht das \enquote{Menuetto D. C.} (Da Capo, das heißt zum Anfang zurück gehen) am Ende des Trios; Sie müssen deshalb Menuett-Trio-Menuett spielen.
Der letzte Abschnitt ist das Rondo.
Rondos haben den allgemeinen Aufbau ABACADA usw., was einen Ohrwurm, A, ausgiebig benutzt.


Unser Rondo hat den Aufbau (BB')A(CC')A(BB')A'-Coda, eine sehr symmetrische Struktur.
Die Taktart ist ein lebhafter Halbetakt; können Sie die Tonart des BB'-Teils herausfinden?
Der Rest dieses Rondos steht in A-Dur, der angegebenen Tonart dieser Sonate.
Die ganze Sonate wird manchmal als Variation eines einzigen Themas bezeichnet, was wahrscheinlich falsch ist, obwohl das Rondo 
der Variation III und das Trio der Variation IV gleicht.
Das Rondo beginnt mit der B-Struktur, die aus einer kurzen Einheit von nur fünf Noten besteht, die in den Takten 1-3 zweimal, mit einer Pause dazwischen, wiederholt werden; sie wird in Takt 4 mit der doppelten Geschwindigkeit wiederholt; Mozart benutzte am Ende von Takt 3 dieselbe Einheit geschickt als Verbindung zwischen diesen Wiederholungen.
Sie wird mit halber Geschwindigkeit in den folgenden Takten 7 und 8 noch einmal wiederholt, und die letzten zwei Takte bilden das Ende.
Takt 9 ist der gleiche wie Takt 8, außer dass die letzte Note abwärts statt aufwärts geht; diese abrupte Änderung des Wiederholungsmusters ist eine einfache Möglichkeit, ein Ende anzuzeigen.
Die Einheiten mit halber Geschwindigkeit werden durch das Hinzufügen von zwei Vorschlagsnoten am Anfang verschleiert, sodass wir, wenn der gesamte Abschnitt B mit der richtigen Geschwindigkeit gespielt wird, nur die Melodie hören, ohne die einzelnen Wiederholungseinheiten zu erkennen.
Die Effizienz seines Kompositionsprozesses ist erstaunlich - er wiederholte dieselbe Einheit siebenmal in neun Takten mit drei Geschwindigkeiten, um eine seiner berühmten Melodien zu komponieren.
Die ganze Sonate besteht sogar aus diesen wiederholten Abschnitten, die acht bis zehn Takte lang sind.
Es gibt mehrere Abschnitte, die 16 oder 32 Takte lang sind, aber diese sind nur Vervielfältigungen der achttaktigen Basisabschnitte.
Weitere Beispiele zur Analyse der Mikrostruktur von Mozart und Beethoven finden Sie in \hyperref[c1iv4]{Abschnitt IV.4}.
Diese Art der Analyse kann für das \hyperref[c1iii6]{Auswendiglernen} und das \hyperref[c1ii12mental]{mentale Spielen} hilfreich sein - schließlich hat er die Sonate mental komponiert!

Die technisch anspruchsvollen Teile sind:

\begin{enumerate}[label={\arabic*.}] 
 \item der schnelle Triller der rechten Hand in Takt 25,
 \item die schnellen Läufe der rechten Hand in den Takten 36-60 - achten Sie auf einen guten \hyperref[c1ii18]{Fingersatz},
 \item die schnellen gebrochenen Oktaven der rechten Hand in den Takten 97-104 und 
 \item die schnelle Alberti-Begleitung der linken Hand in den Takten 119-125.
 \end{enumerate}

Ermitteln Sie, welcher dieser Teile Ihnen am meisten Schwierigkeiten bereitet, und beginnen Sie das Üben mit diesem Teil.
Die Reihe der gebrochenen Oktaven der Takte 97-104 ist nicht einfach nur eine Folge gebrochener Oktaven, sondern es sind zwei Melodien, die eine Oktave und einen Halbtonschritt voneinander entfernt sind und sich gegenseitig jagen.
Üben Sie alles einhändig, ohne Pedal, bis es zufriedenstellend ist, bevor Sie mit dem beidhändigen Üben beginnen.
Die \hyperref[c1iii7b]{Übungen für parallele Sets} sind der Schlüssel zur Entwicklung der zum Spielen dieser Elemente notwendigen Technik, und \hyperref[c1iii7b1]{Übung \#1} (Wiederholung von Quadrupeln, III.7b) ist die wichtigste, besonders zum Lernen der \hyperref[c1ii14]{Entspannung}.
Informationen zu \hyperref[c1iii3]{schnellen Trillern} finden Sie in Abschnitt III.3a.
Die gebrochenen Akkorde in der linken Hand (Takt 28 usw. und in der Coda) sollten sehr schnell gespielt werden, fast wie eine einzelne Note, und sich mit den Noten der rechten Hand decken.
Das beidhändige Spielen sollte zunächst ohne Pedal geübt werden, bis es zufriedenstellend ist.

Wie spielt man, damit die Musik wie Mozart klingt?
Es gibt kein Geheimnis - die Anweisungen waren die ganze Zeit da!
Es sind die Ausdrucksbezeichnungen auf dem Notenblatt; für Mozart hatte jedes Zeichen eine präzise Bedeutung, und wenn Sie \textit{jedes} einzelne davon befolgen, einschließlich der Taktart usw., wird die Musik zu einer vertraulichen, komplizierten Konversation.
Das \enquote{einzige}, das Sie tun müssen, ist, den Drang zu unterdrücken, einen eigenen Ausdruck hinzuzufügen.
Es gibt kein besseres Beispiel dafür, als die letzen drei Akkorde am Ende.
Es ist so einfach, dass es fast unglaublich ist (ein Kennzeichen von Mozart): Der erste Akkord ist ein Staccato, und die restlichen zwei sind legato.
Dieses einfache Mittel erzeugt einen überzeugenden Schluss; wenn man es anders spielt, wird das Ende zum Flop.
Deshalb sollten diese drei letzten Akkorde ohne Pedal gespielt werden, obwohl einige Ausgaben (Schirmer) Pedalzeichen haben.
Bessere Klavierspieler spielen meistens das gesamte Rondo ohne Pedal.

Lassen Sie uns die ersten acht Takte dieses Rondos genauer untersuchen.<br>
\textbf{Rechte Hand}: Das erste viernotige Thema (Takt 1) wird legato gespielt; darauf folgt eine Achtelnote und eine exakte Achtelpause.
Die Note und die Pause werden benötigt, um dem Publikum die Einführung dieser Einheit \enquote{darzureichen}.
Dieses Konstrukt wird wiederholt, dann wird das viernotige Thema mit doppelter Geschwindigkeit (zwei je Takt) in Takt 4 wiederholt und gipfelt im fest gespielten und mit den zwei folgenden Staccato-Noten verbindenden C6.
Diese Verdopplung der Geschwindigkeit ist ein Mittel, das von Komponisten zu jeder Zeit benutzt wurde.
In den Takten 5-7 spielt die rechte Hand staccato und hält so die Spannung aufrecht.
Die Folge der fallenden Noten in den Takten 8-9 bringt diesen Abschnitt zum Abschluss, wie jemand, der auf die Bremse eines Autos tritt.<br>
\textbf{Linke Hand: Die einfache Begleitung mit der linken Hand bietet ein festes Gerüst; ohne dieses würden die ganzen neun Takte wie eine durchgeweichte Nudel umherwackeln.}
Die geschickte Anordnung der Bögen (zwischen der ersten und zweiten Note von Takt 2 usw.) betont nicht nur den Halbetakt, sondern stellt auch die rhythmische Idee der Exposition heraus; \textbf{es klingt wie ein Foxtrott-Tanzschritt - langsam, langsam, schnell-schnell-langsam in den Takten 2-5, wiederholt in den Takten 6-9}.
Da in den Takten 6-8 jede Note staccato sein muss, ist die einzige Möglichkeit, den \hyperref[c1iii1b]{Rhythmus} zu betonen, den Akzent jeweils auf die erste Note jedes Takts zu legen.

Beide Noten von Takt 9 (beide Hände) sind legato und etwas leiser, um ein Ende darzustellen, und beide Hände heben sich im selben Moment.
Es ist klar, dass wir nicht nur wissen müssen, was die Ausdrucksbezeichnungen sind, sondern auch \textit{warum} sie dort sind.
Natürlich ist keine Zeit, über diese komplizierten Erklärungen nachzudenken; die Musik sollte sich darum kümmern - der Künstler \textit{fühlt} einfach die Wirkung dieser Zeichen.
Das strategische Setzen von Legato, Staccato, Bögen und Akzenten ist der Schlüssel zum Spielen dieses Stücks, während man den Rhythmus genau beibehält.
Ich hoffe, Sie sind nun in der Lage, die Analyse für den Rest dieses Stücks fortzuführen und Musik zu erzeugen, die eindeutig Mozart ist.

Das beidhändige Spielen ist etwas schwieriger als bei der \hyperref[c1ii25b]{Mondschein-Sonate}, weil dieses Stück schneller ist und eine höhere Genauigkeit erfordert.
Der schwierigste Teil ist vielleicht das Koordinieren des \hyperref[c1iii3]{Trillers} in der rechten Hand mit der linken Hand in Takt 25.
Versuchen Sie nicht, das zu lernen, indem Sie es langsamer spielen.
Stellen Sie nur sicher, dass die einhändige Arbeit komplett erledigt ist, indem Sie die Takte 25 und 26 als einen einzigen Übungsausschnitt benutzen und dann die beiden Hände bei der endgültigen Geschwindigkeit zusammenführen.
\textbf{Versuchen Sie beim beidhändigen Spielen immer erst, alles mit (oder nahe an) der endgültigen Geschwindigkeit zu kombinieren, 
und benutzen Sie die langsameren Geschwindigkeiten nur als letzten Ausweg, denn wenn Sie es schaffen, sparen Sie eine Menge Zeit und vermeiden es, sich \hyperref[c1ii22]{schlechte Angewohnheiten} anzueignen.}
Fortgeschrittene Klavierspieler müssen beim Zusammenführen der Hände fast nie langsamer werden.

Fügen Sie das Pedal hinzu, nachdem Sie zufriedenstellend beidhändig ohne das Pedal spielen können.
Im mit Takt 27 beginnenden Abschnitt erzeugt die Kombination der gebrochenen Akkorde der linken Hand, der Oktaven der rechten Hand und des Pedals ein Gefühl der Erhabenheit, die charakteristisch dafür ist, wie Mozart mit relativ einfachen Mitteln Erhabenheit erzeugen konnte.
Halten Sie die letzte Note dieses Abschnitts etwas länger als aufgrund des Rhythmus erforderlich (Tenuto, Takt 35), besonders nach der Wiederholung, bevor Sie mit dem nächsten Abschnitt beginnen.
Wie bereits gesagt, hat Mozart keine Pedalzeichen angegeben; deshalb sollten Sie, nachdem Sie beidhändig ohne Pedal geübt haben, das Pedal \textit{nur} dort hinzufügen, wo Sie glauben, dass es die Musik verbessert.
Besonders bei schwierigem Material, wie das von Rachmaninoff, wird weniger Pedal von der Klavierspielergemeinde als Anzeichen überlegener Technik angesehen.


\label{FI}
\subsubsection{Chopins Fantaisie-Impromptu, Op. 66}
\label{c1ii25d}

Dieses Beispiel wurde ausgewählt, weil:

\begin{enumerate}[label={\arabic*.}] 
 \item jeder diese Komposition mag,
 \item sie ohne gute Lernmethoden als unmöglich zu lernen erscheinen kann,
 \item das Hochgefühl, wenn man plötzlich in der Lage ist, sie zu spielen, unvergleichlich ist,
 \item die Herausforderungen des Stücks ideal zur Veranschaulichung sind und
 \item das die Art von Stück ist, an dem Sie Ihr ganzes Leben arbeiten werden, um \enquote{unglaubliche Dinge} damit zu machen, sodass Sie genauso gut \textit{jetzt} damit anfangen können!
 \end{enumerate}
Die meisten Schüler, die Schwierigkeiten damit haben, haben sie, weil Sie den Einstieg nicht finden, und die anfängliche Hürde erzeugt eine mentale Blockade, die sie ihre Fähigkeit das Stück zu spielen anzweifeln lässt.
Es gibt keine bessere Demonstration der Wirksamkeit der Methoden dieses Buchs, als zu zeigen, wie man diese Komposition lernen kann.
Da dieses Stück jedoch ziemlich schwierig ist, sollten Sie \hyperref[c1iii1]{Abschnitt III} lesen, bevor Sie es lernen.

Sie werden ungefähr 2 Jahre Klavierunterricht benötigen, bevor Sie mit diesem Stück beginnen können.
Etwas einfachere Stücke sind die oben besprochenen - \hyperref[c1ii25b]{Mondschein-Sonate} und das \hyperref[c1ii25c]{Rondo} - sowie \hyperref[c1iii6l2]{Bachs Inventionen} in Abschnitt III.6l.
Finden Sie stets die Tonart heraus, bevor Sie anfangen.
Tipp: Nach der \enquote{Ankündigung} G\# beginnt die Komposition in Takt 3 mit einem C\# und endet mit einem C\#, und das Largo beginnt mit einem Db (\footnote{enharmonisch verwechselt} dieselbe Note wie C\#!); aber sind die beiden in einer Dur- oder Molltonart?
\textbf{Wegen der großen Zahl Kreuze und Be's, wie in dieser Komposition, sind Anfänger häufig besorgt;
die schwarzen Tasten sind jedoch einfacher zu spielen als die weißen, wenn Sie erst einmal die \hyperref[c1iii4b]{flachen Fingerhaltungen} (siehe III.4b) und den \hyperref[c1iii5b]{Daumenübersatz} (siehe III.5) kennen.
Chopin mag diese \enquote{weit entfernten} Tonarten aus diesem Grund gewählt haben, weil die Tonleiter bei der \hyperref[et1]{Gleichschwebenden Temperatur}, die er wahrscheinlich benutzte, nichts ausmacht (siehe \hyperref[c2_2c]{Kapitel 2, Abschnitt 2c}).}

Wir fangen an, indem wir die vorbereitenden Hausaufgaben mit dem \hyperref[c1ii7]{einhändigen Üben} und dem \hyperref[c1ii12mental]{mentalen Spielen} noch einmal durchgehen.
Deshalb sollten Sie mit dem Ziel beidhändig üben, die beiden Hände sehr genau zu synchronisieren.
Obwohl die letzte Seite die schwierigste sein mag, werden wir die Regel über das Beginnen mit dem Ende durchbrechen und am Anfang beginnen, weil es schwierig ist, dieses Stück richtig anzufangen.
Wenn man aber erst einmal angefangen hat, geht es wie von selbst.
Sie brauchen einen starken, zuversichtlichen Anfang.
Wir werden deshalb mit den ersten beiden Seiten anfangen, bis zum langsamen Cantabile-Teil.
Das Strecken der linken Hand und das ständige Training machen die \hyperref[c1ii21]{Ausdauer} (das heißt die \hyperref[c1ii14]{Entspannung}) zu einem Hauptthema.
Diejenigen ohne genügende Erfahrung und besonders diejenigen mit kleineren Händen brauchen vielleicht für Wochen zusätzliche Arbeit an der linken Hand, bevor sie zufriedenstellend ist.
Glücklicherweise ist die linke Hand nicht so schnell, sodass die Geschwindigkeit kein einschränkender Faktor ist und die meisten Schüler in der Lage sein sollten, die linke Hand in weniger als zwei Wochen einhändig schneller als mit der endgültigen Geschwindigkeit, völlig entspannt und ohne Ermüdung zu spielen.

Für Takt 5, in dem die rechte Hand zum ersten Mal einsetzt, ist der vorgeschlagene Fingersatz für die linke Hand 532124542123.
Sie können damit anfangen, dass Sie Takt 5 mit der linken Hand üben und ihn fortlaufend \hyperref[c1iii2]{zyklisch spielen}, bis Sie ihn gut spielen können.
Sie sollten während des Spielens die \textit{Handfläche} und nicht die \textit{Finger} strecken, sonst kann es zu Stress und Verletzungen kommen.
Sehen Sie dazu in \hyperref[c1iii7e]{Abschnitt III.7e}, wie man seine Handflächen dehnt.

\textbf{Üben Sie ohne das Pedal.}
Üben Sie in kleinen Abschnitten.
Die vorgeschlagenen Abschnitte sind: Takte 1-4, 5-6, erste Hälfte von 7, zweite Hälfte von 7, 8, 10 (überspringen Sie 9, der der gleiche ist wie 5), 11, 12, 13-14, 15-16, 19-20, 21-22, 30-32, 33-34, dann zwei Akkorde in 35.
Wenn Sie den zweiten Akkord nicht greifen können, spielen Sie ihn als einen schnellen aufsteigenden, gebrochenen Akkord mit der Betonung auf der obersten Note.
Nachdem jeder Abschnitt auswendig gelernt und zufriedenstellend ist, verbinden Sie sie paarweise.
Spielen Sie dann die ganze linke Hand aus dem Gedächtnis, wobei sie mit dem Anfang beginnen und die Abschnitte hinzufügen.
Bringen Sie sie bis zur endgültigen Geschwindigkeit, und prüfen Sie Ihr mentales Spielen.

Wenn Sie diesen ganzen Abschnitt (nur mit der linken Hand) zweimal hintereinander entspannt und ohne sich müde zu fühlen spielen können, haben Sie die notwendige Ausdauer.
An diesem Punkt macht es viel Spaß, schneller zu werden als die endgültige Geschwindigkeit.
Gehen Sie zur Vorbereitung der beidhändigen Arbeit ungefähr bis zur 1,5-fachen endgültigen Geschwindigkeit.
Heben Sie das Handgelenk leicht, wenn Sie mit dem kleinen Finger spielen, und senken Sie es, wenn Sie zum Daumen kommen.
Beim Heben des Handgelenks werden Sie feststellen, dass Sie mehr Kraft in den kleinen Finger legen können, und durch das Senken des Handgelenks vermeiden Sie das Verpassen der Daumennote.
\textbf{In Chopins Musik sind die Noten des kleinen Fingers und des Daumens (aber besonders die des kleinen Fingers) die wichtigsten.}
Üben Sie deshalb, diese beiden Finger mit Autorität zu spielen.
Die in Abschnitt III.5 beschriebene \hyperref[c1iii5wagen]{Wagenradbewegung} kann hierbei hilfreich sein.

Wenn Sie damit zufrieden sind, fügen Sie das Pedal hinzu; grundsätzlich sollte das Pedal mit jedem Akkordwechsel angehoben werden, der im Allgemeinen ein- oder zweimal je Takt vorkommt.
Das Pedal ist eine schnelle Auf-und-Ab-Bewegung (die den Ton \enquote{abschneidet}) auf dem ersten Schlag, aber Sie können das Pedal für besondere Effekte früher heben.
Für die weite Streckung der linken Hand in der zweiten Hälfte von Takt 14 (beginnend mit E2) ist der Fingersatz 532124, wenn Sie ihn bequem ausführen können.
Wenn nicht, benutzen Sie 521214.

Gleichzeitig sollten Sie die rechte Hand geübt haben, wobei Sie die Hände wechseln, sobald die arbeitende Hand sich ein wenig müde anfühlt.
\textbf{Die Vorgehensweisen sind fast identisch mit denen für die linke Hand, einschließlich des Übens ohne das Pedal.}
Teilen Sie zunächst Takt 5 in zwei Hälften, lernen Sie jede Hälfte bis zur endgültigen Geschwindigkeit, und fügen Sie sie dann zusammen.
Benutzen Sie für das ansteigende Arpeggio in Takt 7 den \hyperref[c1iii5a]{Daumenübersatz}, weil es zu schnell ist, um es mit Untersatz zu spielen.
Der Fingersatz sollte so sein, dass beide Hände den kleinen Finger oder den Daumen nach Möglichkeit gleichzeitig spielen; das vereinfacht das beidhändige Spielen.
Deshalb ist es keine gute Idee, mit den Fingersätzen der linken Hand herumzualbern - benutzen Sie die Fingersätze, die in den Noten stehen.

Üben Sie nun beidhändig.
Sie können entweder mit der ersten oder der zweiten Hälfte von Takt 5 anfangen, bei dem die rechte Hand das erste Mal dazukommt.
Die zweite Hälfte ist wahrscheinlich einfacher - wegen der geringeren Streckung der linken Hand und weil es (im Gegensatz zur ersten Hälfte) kein Timing-Problem mit der ersten fehlenden Note in der rechten Hand gibt.
Lassen Sie uns deshalb mit der zweiten Hälfte anfangen.
\textbf{Der einfachste Weg das 3,4-Timing zu lernen, ist, es von Anfang an mit der endgültigen Geschwindigkeit zu tun.
Versuchen Sie nicht, langsamer zu werden und herauszufinden, wo jede Note hingehört, weil zu viel davon eine Ungleichmäßigkeit in Ihr Spielen einführen wird, die später eventuell nicht mehr korrigiert werden kann.}
Hier benutzen wir das \enquote{Zirkulieren} - sehen Sie dazu \enquote{\hyperref[c1iii2]{Zyklisch spielen}} in Abschnitt III.2.
Zirkulieren Sie zunächst fortlaufend ohne zu stoppen die sechs Noten der linken Hand.
Wechseln Sie dann die Hände und machen Sie das gleiche mit den acht Noten der rechten Hand mit dem gleichen (endgültigen) Tempo, wie Sie es mit der linken Hand getan haben.
Zirkulieren Sie als nächstes mehrere Male nur mit der linken Hand, und lassen Sie dann die rechte Hand einstimmen.
Am Anfang müssen Sie nur die ersten Noten genau zur Deckung bringen; machen Sie sich nichts daraus, wenn die anderen nicht so ganz stimmen.
Nach ein paar Versuchen sollten Sie in der Lage sein, es ziemlich gut beidhändig zu spielen.
Wenn nicht, hören Sie auf, und fangen Sie noch einmal von vorne mit dem einhändigen Zirkulieren an.
Da fast die ganze Komposition aus Stücken wie dem Abschnitt, den Sie gerade geübt haben, besteht, zahlt es sich aus, diesen gut zu üben, bis Sie sehr zufrieden sind.
Ändern Sie die Geschwindigkeit, um dieses zu erreichen.
Werden Sie sehr schnell, dann sehr langsam.
Während Sie langsamer werden können Sie sehen, wo all die Noten in Bezug zueinander hingehören.
Sie werden feststellen, dass schnell nicht notwendigerweise schwierig ist und langsamer nicht immer einfacher.
\textbf{Das 3,4-Timing ist ein mathematisches Mittel, das Chopin benutzt hat, um die Illusion von Hyper-Geschwindigkeit in diesem Stück zu erzeugen.}
Die mathematischen Erklärungen und zusätzlichen herausragenden Punkte dieser Komposition werden ausführlicher unter \enquote{\hyperref[c1iii2]{Zyklisch spielen}} in Abschnitt III.2 besprochen.
Sie werden wahrscheinlich diese Komposition jahrelang mit getrennten Händen üben, nachdem Sie zum ersten Mal mit dem Stück fertig sind, weil es so viel Spaß macht, mit dieser faszinierenden Komposition zu experimentieren.
Fügen Sie nun das Pedal hinzu.
Das ist der Punkt, an dem Sie die Angewohnheit entwickeln sollten, das Pedal exakt zu \enquote{pumpen}.

Wenn Sie mit der zweiten Hälfte von Takt 5 zufrieden sind, wiederholen Sie die gleiche Prozedur für die erste Hälfte.
Fügen Sie dann die beiden Hälften zusammen.
Ein Nachteil des Ansatzes, erst einhändig und dann beidhändig zu lernen ist, dass praktisch der ganze Technikerwerb einhändig erreicht wird, was möglicherweise zu schwach synchronisiertem beidhändigen Spielen führt.
Sie haben nun die meisten Werkzeuge, um den Rest dieser Komposition selbst zu lernen!

Der Cantabile-Abschnitt ist einfach viermal die gleiche Sache mit zunehmender Komplexität.
Lernen und merken Sie sich deshalb zunächst den ersten Teil, weil er der einfachste ist.
Lernen Sie dann den vierten Teil, weil er der schwerste ist.
Normalerweise sollte man den schwierigsten Teil zuerst lernen, aber in diesem Fall mag es für einige Schüler zu lange dauern, mit dem vierten Teil zu beginnen, und die einfachste Variante zuerst zu lernen kann es sehr vereinfachen, den vierten Teil zu lernen, weil sie einander ähnlich sind.
Wie  bei vielen Stücken von Chopin, ist, die linke Hand auswendig zu lernen, der schnellste Weg, um ein festes Fundament für das Auswendiglernen zu bilden, weil die linke Hand üblicherweise eine einfachere Struktur hat, die leichter zu analysieren, auswendig zu lernen und zu spielen ist.
Zudem erzeugte Chopin oft mehrere Versionen für die rechte Hand, während er im Grunde die gleichen Noten in der linken Hand wiederholte, wie er es in diesem Fall tat (dieselben Akkordprogressionen);
deshalb kennen Sie, nachdem Sie den ersten Teil gelernt haben, bereits das meiste für die linke Hand im vierten Teil, sodass Sie diese letzte Wiederholung schnell lernen können.

Der Triller im ersten Takt des letzten Teils, kombiniert mit dem 2,3-Timing, macht die zweite Hälfte dieses Takts schwierig.
Da es vier Teile gibt, könnten Sie den zweiten halben Takt im ersten Teil ohne Triller, im zweiten Teil den Triller als invertierten Mordent, einen kurzen Triller im dritten und im letzten Teil einen längeren Triller spielen.

Der dritte Abschnitt (Presto!) ist dem ersten Abschnitt ähnlich.
Wenn Sie es also geschafft haben, den ersten Abschnitt zu lernen, sind Sie fast schon fein raus.
Dieses Mal ist er jedoch schneller als beim ersten Mal (Allegro) - 
Chopin möchte offensichtlich, dass Sie das mit unterschiedlichen Geschwindigkeiten spielen.
Möglicherweise, weil er sah, dass die Abschnitte sehr unterschiedlich klingen können, wenn man die Geschwindigkeit ändert;
warum sollte es unterschiedlich klingen, und auf welche Weise?
Die Physik und die Psychologie dieser Geschwindigkeitsänderung werden in \hyperref[c1iii2]{Abschnitt III.2} besprochen.
Beachten Sie, dass ungefähr in den letzten 20 Takten der kleine Finger und der Daumen der rechten Hand die ganze Strecke über bis zum Ende Noten von bedeutendem thematischen Wert tragen.
Dieser Abschnitt kann eine Menge einhändiges Üben mit der rechten Hand erfordern.


\label{fpd}

\textbf{Wenn Sie irgendeine Komposition zu oft mit voller Geschwindigkeit (oder schneller) spielen, kann es sein, dass Sie erleiden, was ich \enquote{Schnellspiel-Abbau} (FPD = Fast Play Degradation) nenne.}
Am nächsten Tag werden Sie vielleicht feststellen, dass Sie sie nicht mehr so gut spielen oder beim Üben keinen Fortschritt machen können.
Das geschieht meistens beim beidhändigen Spielen.
Das Spielen mit getrennten Händen ist eher gegen FPD immun und kann sogar benutzt werden, um ihn zu korrigieren.
FPD tritt wahrscheinlich auf, weil der menschliche Spielmechanismus (Hände, Gehirn usw.) bei solchen Geschwindigkeiten durcheinander gerät, und tritt deshalb nur bei so komplexen Handlungen wie dem beidhändigen Spielen von vorstellungsmäßig oder technisch schwierigen Stücken auf.
Leichte Stücke leiden nicht unter FPD.
FPD kann enorme Probleme bei komplexer Musik, wie Bachs oder Mozarts Kompositionen, erzeugen.
Schüler, die versuchen, sie beidhändig auf Geschwindigkeit zu bringen, können auf alle Arten von Problemen stoßen, und die Standardlösung war, einfach immer langsam zu üben.
Es gibt jedoch eine tolle Lösung für dieses Problem: Üben Sie mit getrennten Händen!
Und denken Sie daran, dass Sie, wann immer Sie schnell spielen, im Allgemeinen unter FPD leiden werden, wenn Sie nicht mindestens einmal vor dem Aufhören langsam spielen.
FPD kann auch ein Zeichen dafür sein, dass Ihr mentales Spielen nicht solide ist oder noch nicht die richtige Geschwindigkeit erreicht hat.


\subsection{Zusammenfassung}
\label{c1ii26}

Damit kommen wir zum Schluss des grundlegenden Abschnitts.
Sie haben alles Notwendige für das Ausarbeiten von Abläufen, mit denen Sie praktisch jedes neue Stück lernen können.
Dieses ist der minimale Satz von Anweisungen, die Sie zum Anfangen benötigen.
In Abschnitt III werden wir noch mehr Anwendungen dieser grundlegenden Schritte erforschen, ebenso werden wir noch mehr Ideen dafür vorstellen, wie man einige weitverbreitete Probleme löst.




<!-- c1iii1.html -->

\section{Ausgewählte Themen des Klavierübens}
\label{c1iii1}

% zuletzt geändert 14.02.2010

\subsection{Klang, Rhythmus, Legato, Staccato}

\subsubsection{Was ist ein \enquote{Guter Klang}?}
\label{c1iii1a}

\paragraph{Der Basisanschlag}
\label{c1iii1a1}

Der Basisanschlag muss von jedem Klavierspieler gelernt werden.
Ohne ihn macht alles andere keinen bedeutenden Unterschied - man kann aus Lehmziegeln und Stroh kein indisches Grabmal bauen.
\textbf{Der Anschlag besteht aus drei Hauptkomponenten: dem Abschlag, dem Halten und dem Anheben.}
Das mag als trivial einfach zu lernen erscheinen, ist es aber nicht, und für die meisten Klavierlehrer ist es ein zähes Ringen, ihren Schülern den korrekten Anschlag beizubringen.
Die Schwierigkeiten entstehen hauptsächlich daraus, dass die Mechanismen des Anschlags noch nicht angemessen erklärt wurden; deshalb werden diese Erläuterungen das Hauptthema dieser Abschnitte sein.

Zunächst wird der Klavierklang vom Abschlag erzeugt; bei der korrekten Bewegung muss dieser so schnell wie möglich, die Lautstärke aber kontrolliert sein.
Diese Kontrolle ist nicht so einfach, da wir im Abschnitt über den \hyperref[c1ii10]{Freien Fall} herausgefunden haben, dass ein schnellerer Abschlag im Allgemeinen zu einem lauteren Ton führt.
Die Schnelligkeit verleiht der Note ihr präzises Timing; ohne diese Schnelligkeit fängt das Timing der Note an, unordentlich zu werden.
Deshalb muss der Abschlag - unabhängig davon, ob die Musik langsam oder schnell ist - im Grunde schnell sein.
\textbf{Die Erfordernis eines schnellen Abschlags, der Kontrolle der Lautstärke und vieler anderer Faktoren, die wir in Kürze kennenlernen werden, bringen uns zu einem sehr wichtigen Prinzip des Klavierspielenlernens: der Sensibilität der Finger.
Die Finger müssen in der Lage sein, viele Voraussetzungen zu erfühlen und zu erfüllen, bevor man den Basisanschlag meistern kann.}
Um die Lautstärke zu kontrollieren, sollte der Abschlag aus zwei Teilen bestehen: einer anfänglichen starken Komponente, um die Reibung bzw. Trägheit der Taste zu überwinden und die Bewegung zu beginnen, und einer zweiten Komponente mit einer der gewünschten Lautstärke entsprechenden Kraft.
Die Empfehlung, \enquote{tief in die Tasten zu spielen}, ist in dem Sinne gut, dass der Abschlag nicht langsamer werden darf; er muss zum unteren Punkt hin beschleunigt werden, sodass man nie die Kontrolle über den Hammer verliert.

\textbf{Diese zweiteilige Bewegung ist besonders wichtig, wenn man pianissimo spielt.}
Bei einem gut eingestellten Konzertflügel ist die Reibung fast null und die Trägheit des Systems gering.
Bei allen anderen Klavieren (was 99 Prozent aller Klaviere umfasst) gibt es eine zu überwindende Reibung, besonders wenn man mit dem Abschlag beginnt (die Reibung ist am höchsten, wenn die Bewegung null ist), und es gibt zahlreiche Ungleichgewichte im System, die Trägheit hervorrufen.
Vorausgesetzt, das Klavier ist richtig \hyperref[c2_7_hamm]{intoniert}, können Sie sehr leise pianissimo spielen, indem Sie zunächst die Reibung bzw. die Trägheit überwinden und dann den sanften Abschlag machen.
Diese beiden Komponenten müssen fließend ineinander übergehen, sodass es für einen Beobachter wie eine einzige Bewegung aussieht, bei der das Fleisch der Finger wie ein Stoßdämpfer wirkt.
Der erforderliche schnelle Abschlag bedeutet, dass die Fingermuskeln zu einem großen Teil aus schnellen Muskelfasern bestehen müssen (siehe unten \hyperref[c1iii7a]{Abschnitt 7a}).
Das wird durch das Üben schneller Bewegungen über einen längeren Zeitraum (ungefähr ein Jahr) und das Vermeiden von Kraftübungen erreicht; deshalb ist die Behauptung, dass Klaviertechnik Fingerkraft erfordert, absolut falsch.
Man muss die Geschwindigkeit und Empfindlichkeit der Finger entwickeln.

Die Haltekomponente des Anschlags ist notwendig, um den Hammer mit Hilfe des Fängers zu halten und die Notendauer genau zu kontrollieren.
Ohne das Halten kann der Hammer umherspringen, was zusätzliche Töne erzeugt, Probleme mit Trillern und wiederholten Noten verursacht usw.
Anfänger werden Schwierigkeiten mit dem Übergang vom Abschlag zum Halten haben.
\textbf{Drücken Sie die Taste während des Haltens nicht nach unten, um zu versuchen, \enquote{tief in das Klavier zu drücken}; die Schwerkraft reicht aus, um die Taste unten zu halten.}
Die Länge des Haltens kontrolliert die Farbe und den Ausdruck; deshalb ist es ein wichtiger Teil der Musik.

Das Anheben lässt den Dämpfer auf die Saiten fallen und beendet den Ton.
Zusammen mit dem Halten bestimmt es die Notendauer.
Wie der Abschlag muss auch das Anheben schnell geschehen, um die Notendauer genau zu kontrollieren.
Deshalb muss sich der Klavierspieler bewusst bemühen, so wie für den Abschlag in den Beugemuskeln, auch in den Streckmuskeln schnelle Muskelfasern zu bilden.
Besonders beim schnellen Spielen werden viele Schüler das Anheben völlig vergessen, was zu unsauberem Spielen führt.
Ein Lauf könnte somit aus Staccato, Legato und sich überlappenden Noten bestehen.
Schnelle parallele Sets könnten so klingen, als ob sie mit Pedal gespielt würden.

\textbf{Indem Sie alle drei Komponenten des Basisanschlags genau kontrollieren, behalten Sie die völlige Kontrolle über das Klavier; insbesondere über den Hammer und den Dämpfer, und diese Kontrolle ist für ein selbstsicheres Spielen notwendig}.
Diese Komponenten bestimmen die Natur jeder Note.
Sie können nun sehen, warum ein schneller Abschlag und ein genauso schnelles Anheben so wichtig sind - besonders während des langsamen Spielens.
Beim normalen Spielen fällt das Anheben der Note mit dem Abschlag der nächsten Note zusammen.
Beim \hyperref[c1iii1c]{Staccato} und Legato (siehe Abschnitt c) und schnellen Spielen (7i) müssen wir alle diese Komponenten verändern, und wir werden das gesondert behandeln.
Wenn Sie diese Komponenten nie zuvor geübt haben, beginnen Sie das Üben mit allen fünf Fingern, C bis G, wie beim Spielen einer Tonleiter, und wenden Sie alle Komponenten auf jeden einzelnen Finger an (\hyperref[c1ii7]{mit getrennten Händen}).
Wenn Sie die Streckmuskeln trainieren möchten, können Sie die schnellen Bewegungen zum Anheben übertreiben.
Versuchen Sie, alle nicht spielenden Finger leicht auf den Tasten liegen zu lassen.
Wenn Sie versuchen, die Ab- und Aufwärtsbewegungen zu beschleunigen, und ungefähr eine Note pro Sekunde spielen, werden Sie eventuell Stress aufbauen.
Üben Sie in diesem Fall so lange, bis Sie den Stress eliminieren können.
Denken Sie bei der Haltekomponente immer daran, dass Sie sofort nach dem schnellen Abschlag während des Haltens entspannen müssen.
Mit anderen Worten: \textbf{Sie müssen sowohl die Bewegungsgeschwindigkeit als auch die Entspannungsgeschwindigkeit trainieren.}
Steigern Sie dann schrittweise die Spielgeschwindigkeit; es ist jetzt noch nicht notwendig, schnell zu spielen.
Kommen Sie nur zu einer bequem handhabbaren Geschwindigkeit.
Machen Sie nun dasselbe mit getrennten Händen mit einem langsamen Musikstück, das Sie spielen können, wie dem \hyperref[c1ii25b]{ersten Satz von Beethovens Mondschein-Sonate}.
Wenn Sie es vorher noch nie getan haben, wird das \hyperref[c1ii25]{beidhändig} zunächst sehr merkwürdig sein, weil Sie so viele Komponenten mit beiden Händen koordinieren müssen.
Mit zunehmender Übung wird die Musik jedoch besser werden, Sie gewinnen eine größere Kontrolle über den Ausdruck und sollten das Gefühl bekommen, dass Sie nun musikalischer spielen können.
Es sollte keine fehlenden oder falschen Noten geben, alle Noten sollten gleichmäßiger sein, und Sie können alle Ausdruckszeichen präziser ausführen.
Der Vortrag wird von einem Tag zum anderen beständig sein, und die Technik wird sich schneller weiterentwickeln.
Ohne einen guten Basisanschlag können Sie in Schwierigkeiten geraten, wenn Sie auf verschiedenen Klavieren oder auf nicht gut eingestellten Klavieren spielen, und die Musik kann nach häufigerem Üben oft schlechter sein, da Sie sich \hyperref[c1ii22]{schlechte Angewohnheiten} aneignen können, zum Beispiel ein ungenaues Timing.
Natürlich mag es Wochen oder sogar Monate dauern, bis der ganze in diesem einen Absatz beschriebene Prozess abgeschlossen ist.


\paragraph{Klang: Einzelne gegenüber mehreren Noten, Pianissimo, Fortissimo}
\label{c1iii1a2}

\textbf{Klang ist die Qualität des Tons; sie ist ein Urteil darüber, ob die Summe aller Eigenschaften des Klangs der Musik angemessen ist}.
Es wird kontrovers diskutiert, ob ein Klavierspieler den \enquote{Klang} einer einzelnen Note auf dem Klavier steuern kann.
Wenn man sich an ein Klavier setzen und eine Note spielen sollte, scheint es fast unmöglich zu sein, den Klang - außer hinsichtlich solcher Eigenschaften wie staccato, legato, laut, leise usw. - zu ändern.
Auf der anderen Seite steht außer Frage, dass verschiedene Klavierspieler unterschiedliche Klänge hervorbringen.
Zwei Klavierspieler können dasselbe Stück auf demselben Klavier spielen und Musik von sehr unterschiedlicher Klangqualität erzeugen.
Das meiste dieses scheinbaren Widerspruchs kann aufgelöst werden, indem man sorgfältig definiert, was \enquote{Klang} bedeutet.
Ein großer Teil der Klangunterschiede zwischen Klavierspielern kann zum Beispiel auf das jeweilige Klavier zurückgeführt werden, das sie benutzen, und darauf, wie diese Klaviere eingestellt oder gestimmt sind.
Den Klang einer einzelnen Note zu steuern, ist wahrscheinlich nur ein Aspekt eines facettenreichen, komplexen Themas.
Deshalb ist die wichtigste Unterscheidung, die wir zunächst treffen müssen, ob wir über eine einzelne Note oder eine Gruppe von Noten sprechen.
Wenn wir verschiedene Töne hören, dann hören wir uns meistens eine Gruppe von Noten an.
In diesem Fall sind Klangunterschiede leichter zu erklären.
Der Klang wird größtenteils durch die Kontrolle der Noten relativ zueinander erzeugt.
Das bedeutet fast immer durch Präzision, Kontrolle und musikalischen Inhalt.
\textbf{Deshalb ist Klang hauptsächlich eine Eigenschaft einer Gruppe von Noten und hängt vom musikalischen Gespür des Spielers ab.}

\textbf{Es ist jedoch auch klar, dass wir den Klang einer einzelnen Note auf mehrere Arten steuern können.}
Wir können ihn durch den Gebrauch des Halte- und des Dämpferpedals steuern\footnote{Anmerkungen zu den Bezeichnungen der Pedale finden Sie \hyperref[Pedale]{hier}}.
Wir können auch den harmonischen Gehalt (die Zahl der Obertöne) ändern, indem wir lauter oder leiser spielen.
Das \hyperref[c1ii24]{Dämpferpedal ändert den Klang, oder das Timbre}, durch die Reduzierung des Anschlagklangs relativ zum Nachklang.
Wenn eine Saite mit einer stärkeren Kraft angeschlagen wird, werden mehr harmonische Schwingungen erzeugt.
Somit erzeugen wir, wenn wir leiser spielen, oft einen Klang mit stärkeren Grundtönen.
Unterhalb einer bestimmten Lautstärke kann die Energie jedoch zu gering sein, um den Grundton zu erzeugen, und es kann sein, dass nur einige wandernde Wellen mit höheren Frequenzen erregt werden - ähnlich dem Flautando bei der Geige (die Trägheit der Saiten wirkt wie die Finger beim Flautando).
Deshalb gibt es irgendwo zwischen \textit{pp} und \textit{ff} eine optimale Anschlagskraft, die den Grundton maximiert.
Das Haltepedal ändert ebenfalls das Timbre, indem es Schwingungen von den nicht angeschlagenen Saiten hinzufügt.

\textbf{Der Klang oder das Timbre können vom Klavierstimmer durch das \hyperref[c2_7_hamm]{Intonieren der Hämmer} oder durch eine andere Stimmung gesteuert werden.}
Ein härterer Hammer erzeugt einen brillanteren Klang (größerer harmonischer Gehalt), und ein Hammer mit einer flachen  Aufschlagsfläche erzeugt einen schrilleren Klang (mehr hochfrequente Obertöne).
Der Klavierstimmer kann die \hyperref[c2_5_stre]{Streckung} ändern oder den Grad der Verstimmung in den Unisoni steuern.
Bis zu einem bestimmten Punkt führt eine größere Streckung zu einem helleren Klang, und ungenügende Streckung kann ein Klavier mit einem wenig aufregenden Klang erzeugen.
Wenn alle Saiten einer Note innerhalb des \hyperref[c2_5_mits]{Mitschwingungsbereichs} verstimmt werden, sind sie in perfekter Stimmung (schwingen mit der gleichen Frequenz), reagieren aber unterschiedlich miteinander.
So kann zum Beispiel die Note zum \enquote{Singen} gebracht werden, das heißt die Lautstärke des Nachklangs schwankt.
Es gibt keine zwei Saiten, die wirklich identisch sind, sodass es einfach keine Möglichkeit gibt, identisch zu stimmen.

Zum Schluss kommen wir zu einer schwierigen Frage: \textbf{Kann man den Klang einer einzelnen Note durch die Steuerung des Abschlags variieren?}
Die meisten Argumente für die Klangsteuerung konzentrieren sich auf den Freien Fall des Hammers, bevor er die Saiten anschlägt.
Gegner (der Klangkontrolle einer einzelnen Note) argumentieren, dass, weil der Hammer im Freien Fall ist, nur seine Geschwindigkeit eine Rolle spielt und deshalb der Klang einer Note, die mit einer bestimmten Lautstärke gespielt wird, nicht steuerbar sei.
Aber die Annahme des Freien Falls wurde nie bewiesen, wie wir nun sehen werden.
\textbf{Ein Faktor, der den Klang beeinflusst, ist die Biegung des Hammerstiels.}
Bei einer lauten Note kann der Stiel deutlich gebogen werden, wenn der Hammer in den Freien Fall übergeht.
In diesem Fall kann der Hammer eine größere effektive Masse als seine wirkliche Masse haben, wenn er die Saiten trifft.
Das kommt daher, dass die Kraft (F), mit der der Hammer auf die Saiten wirkt, durch F = M*\textbf{a} gegeben ist, wobei M die Masse des Hammers und \textbf{a} seine Verzögerung beim Auftreffen auf der Saite ist.
Positive Biegung fügt eine zusätzliche Kraft hinzu, weil diese, wenn die Biegung nach dem Lösen der Stoßzunge aufgehoben wird, den Hammer vorwärts schiebt; wenn F zunimmt, ist es egal, ob M oder \textbf{a} zunimmt, der Effekt ist derselbe.
\textbf{a} ist jedoch schwieriger zu messen als M (zum Beispiel kann man leicht ein größeres M simulieren, indem man einen schwereren Hammer benutzt), weshalb wir in diesem Fall üblicherweise sagen, dass die \enquote{effektive Masse} zugenommen hat, um es leichter zu machen, sich den Effekt der größeren F darauf vorzustellen, wie die Saiten reagieren.
In Wirklichkeit erhöht die positive Biegung jedoch \textbf{a}.
Für eine staccato gespielte Note kann die Biegung negativ sein, wenn der Hammer die Saiten anschlägt, sodass der Klangunterschied zwischen \enquote{tiefem} Spielen und staccato erheblich sein kann.
Diese Veränderungen der effektiven Masse werden sicherlich die Verteilung der Obertöne verändern und den Ton, den wir hören, beeinflussen.
\textbf{Da der Stiel niemals hundertprozentig starr ist, wissen wir, dass es immer eine begrenzte Biegung gibt.
Die einzige Frage ist, ob sie ausreichend ist, den Klang, so wie wir ihn hören, zu beeinflussen.}
Sie ist es fast mit Sicherheit, da der Hammerstiel ein relativ biegsames Stück Holz ist.
Wenn das wahr ist, dann sollte der Klang der tieferen Noten mit den schwereren Hämmern kontrollierbarer sein, weil die schwereren Hämmer eine stärkere Biegung verursachen.
Obwohl man erwarten könnte, dass die Biegung vernachlässigbar ist, weil der Hammer so leicht ist, ist die Hammernuss sehr nah am Drehpunkt des Hammerstiels und erzeugt eine enorme Hebelwirkung.
Das Argument, dass der Hammer zu leicht sei, um eine Biegung zu erzeugen, zieht nicht, weil der Hammer genügend massiv ist, um die ganze kinetische Energie aufzunehmen, die erforderlich ist, sogar die lautesten Töne zu erzeugen.
Das ist eine Menge Energie!

Beachten Sie, dass der Hammerabgang nur ein paar Millimeter beträgt, und dass dieser Abstand extrem entscheidend für den Klang ist.
Solch ein kleiner Abgang suggeriert, dass der Hammer so gestaltet wurde, dass er in Beschleunigung ist, wenn er die Saite trifft.
Der Hammer ist nachdem die Stoßzunge auslöst nicht im Freien Fall, weil der Hammer auf den ersten wenigen Millimetern durch die Rückbildung der Stielbiegung beschleunigt wird.
Der Abgang ist die kleinste kontrollierbare Entfernung, welche die Beschleunigung aufrechterhalten kann, ohne dass der Hammer an den Saiten festhängen kann, weil die Stoßzunge nicht auslösen konnte.
Diese Biegung erklärt vier, ansonsten mysteriöse, Tatsachen:

\begin{enumerate}[label={\roman*.}] 
\item die gewaltige Energie, die solch ein leichter Hammer auf die Saiten übertragen kann,
\item die Abnahme der Klangqualität (oder -kontrolle), wenn der Abgang zu groß ist,
\item die entscheidende Abhängigkeit der Tonstärke und Klangsteuerung vom Hammergewicht und der Hammergröße
\item und den klickenden Ton, den das Klavier von sich gibt, wenn die Buchse des Hammerstiels ausleiert (ein klassisches Beispiel ist die klickende Teflonbuchse).
\end{enumerate}
Das Klicken ist der Ton der Buchse, die zurückspringt, wenn die Stoßzunge loslässt und die Stielbiegung übernimmt - ohne eine zurückgehende Biegung existiert keine Kraft für das Zurückschnappen der Buchse; deshalb gibt es ohne die Biegung kein Klicken.
Da das Klicken sogar bei einigermaßen leisen Tönen zu hören ist, ist der Stiel außer bei den leisesten Tönen bei allen gebogen.

Dieses Szenario hat auch wichtige Auswirkungen für den Klavierspieler (nicht nur für den Klavierstimmer).
Es bedeutet, dass der Klang einer einzelnen Note kontrolliert werden kann.
Es sagt uns auch, wie er kontrolliert werden kann.
Erstens ist bei \textit{ppp}-Tönen die Biegung vernachlässigbar, und wir kümmern uns um den unterschiedlichen Klang der lauteren Töne.
Pianisten wissen, dass man zum \textit{pp}-Spielen\footnote{die Tasten} mit einer konstanten Geschwindigkeit niederdrücken muss - beachten Sie, dass das die Biegung minimiert, weil es keine Beschleunigung beim Auslösen gibt.
Wenn man pianissimo spielt, möchte man die Biegung minimieren, um die effektive Masse des Hammers zu verringern.
Zweitens sollte der Abschlag für eine maximale Biegung am Ende am schnellsten sein.
Das macht Sinn: Ein \enquote{tiefer Ton} wird durch das Hineinlehnen in das Klavier und festes Niederdrücken erzeugt, auch bei leisen Tönen.
Genau so maximieren Sie die Biegung, es kommt dem Gebrauch eines größeren Hammers gleich.
Diese Information ist auch für den Klaviertechniker entscheidend.
Sie bedeutet, dass die optimale Hammergröße genügend klein ist, sodass die Biegung irgendwo um \textit{pp} null ist, aber groß genug ist, dass die Biegung um \textit{mf} deutlich anfängt.
Das ist eine sehr clevere mechanische Anordnung, die das Benutzen von relativ kleinen Hämmern erlaubt, die schnelle Wiederholungen gestatten und trotzdem eine maximale Energiemenge auf die Saiten übertragen können.
Es bedeutet, dass es ein Fehler ist, größere Hämmer zu benutzen, um mehr Klang zu erzeugen, weil die Repetiergeschwindigkeit und die Klangkontrolle verlorengehen.
Die Existenz der Biegung des Hammerstiels ist nun wohlbekannt (\hyperref[Lectures]{Five Lectures on the Acoustics of the Piano}).

Kann man den Unterschied im Klang einer einzelnen Note auf einem Klavier hören, indem man nur eine Note spielt?
Normalerweise nicht; die meisten Menschen sind nicht empfindlich genug, um diesen Unterschied bei den meisten Klavieren zu hören.
Sie werden ein Steinway B oder ein besseres Klavier benötigen, und Sie werden vielleicht anfangen, diesen Unterschied bei den tieferen Noten zu hören (wenn Sie das mit mehreren Klavieren mit stetig höherer Qualität testen).
Wenn jedoch wirklich Musik gespielt wird, ist das menschliche Ohr erstaunlich empfindlich dafür, wie der Hammer auf die Saiten trifft, und dieser Klangunterschied kann leicht gehört werden.
Das ist dem Stimmen ähnlich: Die meisten Menschen (einschließlich der meisten Klavierspieler) werden große Schwierigkeiten haben, den Unterschied zwischen einer hervorragenden Stimmung und einer gewöhnlichen Stimmung zu hören, indem sie einzelne Noten oder Intervalle probieren.
Praktisch jeder Klavierspieler kann jedoch den Unterschied in der Qualität der Stimmung hören, indem er eines seiner Lieblingsstücke spielt.
Sie können das selber demonstrieren.
Spielen Sie ein leichtes Stück zweimal auf die gleiche Weise, außer hinsichtlich des Anschlags.
Spielen Sie zunächst mit Armgewicht und \enquote{pressen Sie tief} in das Klavier, und stellen Sie sicher, dass der Tastenfall den ganzen Weg nach unten beschleunigt wird (korrekter \hyperref[c1iii1a1]{Basisanschlag}).
Vergleichen Sie das mit der Musik, die entsteht, wenn Sie nur leicht drücken, sodass die Taste zwar ganz nach unten geht, es aber keine Beschleunigung am Ende gibt.
Sie müssen wahrscheinlich ein wenig üben, um sicherzustellen, dass es beim ersten Mal nicht lauter ist als beim zweiten Mal.
Sie sollten bei der zweiten Spielweise eine mindere Klangqualität hören.
In den Händen eines großen Pianisten kann dieser Unterschied ziemlich groß sein.
Natürlich haben wir oben besprochen, dass der Klang am stärksten dadurch kontrolliert wird, wie man aufeinanderfolgende Noten spielt, sodass Musik zu spielen nicht der beste Weg ist, um den Effekt einzelner Noten zu testen.
Es ist jedoch der empfindlichste Test.

\textbf{Pianissimo}:
Wir haben gesehen, dass man für \textit{ppp} einen genauen \hyperref[c1iii1a1]{Basisanschlag} und eine schnelle \hyperref[c1ii14]{Entspannung} benötigt.
Die Tasten mit den Fingerpolstern zu erfühlen ist wichtig.
Im Allgemeinen sollten Sie immer mit einem leichten Anschlag üben, bis Sie die Passage gemeistert haben.
Fügen Sie dann das \textit{mf}, \textit{ff} oder was notwendig ist hinzu, weil mit einem leichten Anschlag zu spielen die am schwersten zu entwickelnde Fertigkeit ist.
Es gibt keine Beschleunigung des Abschlags und keine Biegung des Hammerstiels, aber der Fänger muss kontrolliert (die Taste unten gehalten) werden.
\textbf{Der wichtigste Faktor für \textit{ppp} ist das richtige Einstellen des Klaviers (besonders ein minimaler Abgang, das \hyperref[c2_7_hamm]{Intonieren der Hämmer} und das richtige Hammergewicht).
Zu versuchen, die Technik des \textit{ppp} ohne die richtige Wartung des Klaviers zu erreichen und zu erhalten ist sinnlos.}
Im Notfall (während einer \hyperref[c1iii14]{Aufführung} mit einem ungenügenden Instrument), können Sie es bei einem Klavier mit dem \hyperref[c1ii24]{Dämpferpedal} und bei einem Flügel mit teilweise getretenem Dämpferpedal versuchen.
\textit{ppp} ist auf den meisten Digitalpianos schwierig, weil die Mechanik der Tastatur qualitativ schlechter ist und zunehmend verschleißt, wenn das Klavier ungefähr fünf Jahre benutzt wurde.
Aber ein akustisches Klavier, das nicht gewartet wird, kann viel schlechter sein.

Das \textbf{Fortissimo} ist eine Frage der Übertragung von Gewicht auf das Klavier.
Das bedeutet, dass man sich nach vorne lehnt, damit der Schwerpunkt näher an der Tastatur liegt und man \textbf{aus den Schultern spielt}.
Benutzen Sie nicht nur die Hände oder Arme für das \textit{ff}.
Die \hyperref[c1ii14]{Entspannung} ist wieder wichtig, sodass Sie keine Energie verschwenden, für eine maximale Geschwindigkeit des Abschlags sorgen und die richtige Kraft nur dorthin richten, wo Sie benötigt wird.
\textbf{Üben Sie bei einer Passage, die \textit{ff} gespielt werden soll, ohne das \textit{ff}, bis die Passage gemeistert ist, und fügen Sie dann das \textit{ff} hinzu.}

Zusammengefasst ist Klang in erster Linie ein Ergebnis der Einheitlichkeit und der Kontrolle des Spielens und hängt von dem musikalischen Gefühl des Spielers ab.
\textbf{Klangkontrolle ist ein komplexes Thema, das jeden Faktor einbezieht, der die Natur des Tons verändert, und wir haben gesehen, dass es viele Wege gibt, den Klang des Klaviers zu ändern.}
Alles fängt damit an, wie das Klavier eingestellt ist.
Jeder Klavierspieler kann den Klang mit zahlreichen Mitteln steuern, wie laut oder leise zu spielen oder durch das Variieren der Geschwindigkeit.
Indem wir zum Beispiel lauter und schneller spielen, können wir Musik erzeugen, die hauptsächlich aus dem Anschlagston besteht;
ein langsameres und leiseres Spielen wird einen schwächeren Effekt erzeugen und benutzt mehr den Nachklang.
Und es gibt zahllose Arten, das Pedal in Ihr Spiel einzubeziehen.
Wir haben gesehen, dass der Klang einer einzelnen Note gesteuert werden kann, weil der Hammerstiel biegsam ist.
Die große Zahl der Variablen sorgt dafür, dass jeder Klavierspieler einen anderen Klang erzeugt.
 

\subsubsection{Was ist Rhythmus? (Beethovens Sturm-Sonate und Appassionata)}
\label{c1iii1b}

\textbf{Rhythmus ist der (sich wiederholende) zeitliche Rahmen der Musik}.
Wenn man etwas über Rhythmus liest (siehe \hyperref[Whiteside]{Whiteside}), erscheint er oft wie ein mysteriöser Aspekt der Musik, den man nur mit \enquote{angeborenem Talent} zum Ausdruck bringen kann.
Oder vielleicht muss man ihn das ganze Leben lang üben, wie Schlagzeuger.
\textbf{Meistens ist der korrekte Rhythmus jedoch einfach eine Frage des genauen Zählens und des korrekten Lesens des Notats, insbesondere der Taktart.
Das ist nicht so einfach wie es klingt; Schwierigkeiten treten oft auf, weil die meisten Rhythmuszeichen nicht überall ausdrücklich auf dem Notenblatt angegeben sind, da sie Teil von Merkmalen wie der Taktart sind, die nur einmal am Anfang angegeben wird} (es gibt zu viele solcher \enquote{Merkmale}, um sie hier aufzulisten, wie zum Beispiel den Unterschied zwischen einem Walzer und einer Mazurka.
Ein weiteres Beispiel: Ohne auf die Noten zu sehen wird mancher denken, dass bei dem Lied \enquote{Happy Birthday} der Schlag auf \enquote{Happy} liegt, er ist aber auf \enquote{Birth-}; dieses Lied ist ein Walzer).
In vielen Fällen wird die Musik hauptsächlich durch eine Manipulation dieser rhythmischen Variationen erzeugt, sodass der Rhythmus eines der wichtigsten Elemente der Musik ist.
Kurz gesagt: Die meisten Schwierigkeiten mit dem Rhythmus resultieren daraus, dass man die Noten nicht richtig liest.
Das geschieht oft, wenn man versucht, die Noten für beide Hände gleichzeitig zu lesen; das Gehirn hat einfach zu viele Informationen zu verarbeiten und kann sich nicht um den Rhythmus kümmern, besonders wenn die Musik neue technische Fertigkeiten einschließt.
Dieser anfängliche Fehler beim Notenlesen wird dann beim wiederholten Üben in die entstehende Musik eingebaut.

\textbf{Die Definition des Rhythmus:}
Der Rhythmus besteht aus zwei Teilen - der zeitlichen Abfolge und der Betonung -, die in zwei Formen auftreten: formal und logisch.
Das Geheimnisvolle am Rhythmus und die Schwierigkeiten bei seiner Definition resultieren aus dem \enquote{logischen} Teil, der gleichzeitig das Schlüsselelement und das am schwersten zu fassende Element ist.
Fangen wir also zunächst mit den einfacheren formalen Rhythmen an.
Nur weil sie einfacher sind, bedeutet das nicht, dass sie nicht wichtig sind; zu viele Schüler machen mit diesen Elementen Fehler, was dazu führen kann, dass die Musik nicht mehr wiederzuerkennen ist.

\textbf{Formale zeitliche Abfolge:
Der formale zeitliche Rhythmus ist durch die Taktart bestimmt}; sie wird einmal am Anfang der ersten Zeile des ersten Notenblatts angegeben\footnote{sowie bei einem Wechsel der Taktart an der entsprechenden Stelle}.
Die wichtigsten Taktarten sind Dreivierteltakt - zum Beispiel Walzer - (3/4), Viervierteltakt (4/4), Zweihalbetakt (2/2 oder alla breve) und Zweivierteltakt (2/4).
Der Walzer hat 3 Schläge je Takt, usw.; die Zahl der Schläge je Takt wird durch den Zähler des Bruchs angezeigt. 
4/4 ist der verbreitetste und wird oft nicht angegeben, obwohl er durch ein \enquote{C} am Anfang angezeigt werden sollte.
Der Zweihalbetakt wird durch das gleiche \enquote{C} angezeigt, das durch eine vertikale Linie in der Mitte in zwei Hälften geteilt wird.
Die Bezugsnote wird durch den Nenner des Bruchs angegeben, sodass der 3/4-Walzer 3 Viertelnoten je Takt umfasst und 2/4 im Prinzip doppelt so schnell ist wie 2/2.
Fast jede Taktart wird aus Vielfachen von 2 oder 3 gebildet, obwohl es Ausnahmen gibt, um besondere Effekte zu erzielen (zum Beispiel 5 oder 7 Schläge).

\textbf{Formale Betonung:}
Jede Taktart hat ihre eigene formale Betonung (lautere und leisere Schläge).
Wenn wir festlegen, dass 3 am lautesten ist, 2 leiser usw., dann hat der (Wiener) Walzer die formale Betonung 311 - das berühmte \enquote{um-ta-ta}; die Betonung liegt auf dem ersten Schlag.
Die Mazurka kann 131 oder 113 haben. 
Der Viervierteltakt hat die formale Betonung 3121, bei 2/2 und 2/4 ist die Betonung 21.
Eine Synkopierung ist ein Rhythmus, bei dem die Betonung an einer anderen Stelle als der formale Akzent liegt; ein synkopierter 4/4 könnte zum Beispiel 2131 oder 2113 sein.
Beachten Sie, dass der 2113-Rhythmus die ganze Komposition hindurch fest ist, die 3 aber an einer unkonventionellen Stelle liegt.

\textbf{Logische Abfolge und Betonung:}
Hier bringt der Komponist seine musikalischen Ideen ein.
Es ist eine Abweichung in der Abfolge und Lautstärke vom formalen Rhythmus.
Obwohl die rhythmische Logik nicht notwendig ist, so ist sie doch fast immer vorhanden.
Häufige Beispiele der zeitlichen rhythmischen Logik sind \enquote{accel.} (um die Dinge ein wenig aufregender zu gestalten), \enquote{decel.} (um vielleicht ein Ende anzuzeigen) oder \enquote{rubato}.
Beispiele der dynamischen rhythmischen Logik sind das Ansteigen oder Abfallen der Lautstärke, \enquote{forte}, \enquote{pp} usw.

\textbf{Beethovens Sonate \enquote{Der Sturm}} (Op. 31, \#2) verdeutlicht die formalen und logischen Rhythmen.
So sind zum Beispiel die ersten drei Takte des dritten Satzes drei Wiederholungen derselben Struktur, und sie folgen einfach dem formalen Rhythmus.
In den Takten 43-46 gibt es jedoch sechs Wiederholungen derselben Struktur in der rechten Hand, aber sie müssen in vier formale rhythmische Takte gepresst werden!
Wenn Sie in der rechten Hand sechs identische Wiederholungen spielen, ist das falsch!
Zusätzlich ist in Takt 47 ein unerwartetes \textit{sf}, das nichts mit dem formalen Rhythmus zu tun hat aber ein sehr wichtiges Element des logischen Rhythmus ist.

Wenn der Rhythmus so wichtig ist, welche Richtlinie kann man dann benutzen, um ihn zu entwickeln?
Offensichtlich \textbf{muss man Rhythmus als ein separates Thema des Übens behandeln, für das man einen besonderen Plan benötigt}.
Reservieren Sie deshalb während des anfänglichen Lernens eines neuen Stückes ein wenig Zeit, um am Rhythmus zu arbeiten.
Ein Metronom, besonders eines mit fortgeschrittenen Funktionen, kann hier hilfreich sein.
Zunächst müssen Sie noch einmal prüfen, ob Ihr Rhythmus mit der Taktart übereinstimmt.
Das kann man nicht in Gedanken tun, auch wenn man das Stück bereits spielen kann - man muss sich die Notenblätter noch einmal ansehen und jede Note überprüfen.
Zu viele Schüler spielen ein Stück einfach in einer bestimmten Weise, \enquote{weil es sich richtig anhört}; das darf man nicht tun.
Sie müssen anhand der Notenblätter überprüfen, ob die richtigen Noten die richtige Betonung gemäß der Taktart tragen.
Nur dann können Sie entscheiden, welche rhythmische Interpretation die beste Art zum Spielen ist und wo der Komponist Verstöße gegen die Grundregeln eingefügt hat (kommt sehr selten vor); viel öfter ist der von der Taktart vorgegebene Rhythmus sehr wohl richtig, klingt aber kontraintuitiv.
Ein Beispiel dafür ist das mysteriöse \enquote{Arpeggio} am Anfang von Beethovens Appassionata (Op. 57).
Ein normales Arpeggio (wie CEG) würde mit der ersten Note (C) beginnen, welche die Betonung (Abschlag) tragen sollte.
Beethoven beginnt jedoch jeden Takt bei der dritten Note des Arpeggios (der erste Takt ist unvollständig und trägt die ersten beiden Noten des \enquote{Arpeggios}); das zwingt Sie dazu, die dritte Note (G) zu betonen, nicht die erste, wenn Sie der Taktart korrekt folgen möchten.
Man findet den Grund für dieses ungewöhnliche \enquote{Arpeggio}, wenn das Hauptthema in Takt 35 eingeführt wird.
Beachten Sie, dass dieses \enquote{Arpeggio} einfach eine invertierte, schematisierte (vereinfachte) Form des Hauptthemas ist.
Beethoven hat uns psychologisch auf das Hauptthema vorbereitet, indem er uns nur den Rhythmus gegeben hat!
Deshalb wiederholt er es, nachdem er es um ein seltsames Intervall erhöht hat - er wollte bloß sichergehen, dass wir den ungewöhnlichen Rhythmus erkannt haben (er benutzte am Anfang seiner Fünften Symphonie dasselbe Mittel, indem er das viernotige Motiv mit einer niedrigeren Tonhöhe wiederholte).
Ein weiteres Beispiel ist Chopins Fantaisie-Impromptu.
Die erste Note der rechten Hand (Takt 5) muss leiser sein als die zweite.
Können Sie mindestens einen Grund dafür finden?
Obwohl das Stück im 2/2-Takt steht, kann es lehrreich sein, die rechte Hand im 4/4-Takt zu üben, um sicherzustellen, dass nicht die falschen Noten betont werden.

\textbf{Prüfen Sie den Rhythmus sorgfältig, wenn Sie mit \hyperref[c1ii7]{getrennten Händen} beginnen.
Prüfen Sie ihn noch einmal, wenn Sie mit dem \hyperref[c1ii25]{beidhändigen Üben} anfangen.
Wenn der Rhythmus falsch ist, wird es üblicherweise unmöglich, die Musik  mit der vorgegebenen Geschwindigkeit zu spielen.
Deshalb ist es eine gute Idee, den Rhythmus zu überprüfen, wenn man Schwierigkeiten damit hat, auf Geschwindigkeit zu kommen.
Tatsächlich ist eine falsche rhythmische Interpretation eine der häufigsten Ursachen für Geschwindigkeitsbarrieren und Probleme beim beidhändigen Spielen.
Wenn Sie einen rhythmischen Fehler begehen, wird kein Aufwand an Übung Sie in die Lage versetzen, auf Geschwindigkeit zu kommen!}
Das ist einer der Gründe, warum das \hyperref[c1iii8]{Konturieren} funktioniert: Es kann das korrekte Lesen des Rhythmus vereinfachen.
Konzentrieren Sie sich deshalb beim Konturieren auf den Rhythmus.
Auch werden Sie, wenn Sie das erste Mal mit dem beidhändigen Spielen beginnen, mehr Erfolg haben, wenn Sie den Rhythmus betonen.
Der Rhythmus ist ein weiterer Grund, warum Sie keine Stücke versuchen sollten, die zu schwierig für Sie sind.
Wenn Sie nicht genügend Technik haben, werden Sie nicht in der Lage sein, den Rhythmus zu kontrollieren.
Es kann passieren, dass der Mangel an Technik Ihrem Spielen einen falschen Rhythmus aufzwingt und so eine Geschwindigkeitsbarriere erzeugt.

Suchen Sie als nächstes nach besonderen Rhythmuszeichen, zum Beispiel \textit{\textbf{sf}} oder Akzentzeichen.
Schließlich gibt es auch Situationen, in denen keine Zeichen auf dem Notenblatt stehen und man einfach wissen muss, was zu tun ist, oder sich eine Aufnahme anhören muss, um besondere rhythmische Variationen zu erkennen.
Deshalb sollten Sie als Teil des Übungsplans mit dem Rhythmus experimentieren, unerwartete Noten betonen usw., um zu sehen, was passieren könnte.

Rhythmus ist auch eng mit der Geschwindigkeit verbunden.
Deshalb muss man die meisten Kompositionen von Beethoven oberhalb bestimmter Geschwindigkeiten spielen; ansonsten können die Gefühle, die mit dem Rhythmus und sogar mit der Melodieführung verbunden sind, verloren gehen.
Beethoven war ein Meister des Rhythmus; deshalb kann man Beethoven nicht mit Erfolg spielen, ohne dem Rhythmus besondere Aufmerksamkeit zu schenken.
Er gibt Ihnen üblicherweise mindestens zwei Dinge gleichzeitig:

\begin{enumerate}[label={\roman*.}] 
\item eine leicht zu verfolgende Melodie, die das Publikum hört,
\item und ein rhythmisches und harmonisches Mittel, das kontrolliert, was das Publikum \textit{fühlt.}
\end{enumerate}
Deshalb kontrolliert das erregende Tremolo der linken Hand im ersten Satz seiner Pathétique (Op. 13) die Gefühle, während das Publikum damit beschäftigt ist, der merkwürdigen rechten Hand zuzuhören.
Deshalb ist eine bloße technische Fähigkeit, das Tremolo der linken Hand zu bewältigen, ungenügend - man muss in der Lage sein, den emotionalen Gehalt durch dieses Tremolo zu kontrollieren.
Wenn Sie dieses rhythmische Konzept verstehen und ausführen können, wird es viel leichter, den musikalischen Gehalt des ganzen Satzes herauszubringen, und der starke Kontrast mit dem \textit{Grave}-Abschnitt wird offensichtlich.

Es gibt eine Klasse rhythmischer Schwierigkeiten, die mit einem einfachen Trick überwunden werden können: die Klasse der komplexen Rhythmen mit fehlenden Noten.
Ein gutes Beispiel dafür kann man im zweiten Satz von Beethovens Pathétique finden.
Der 2/4-Takt ist in den Takten 17 bis 21 wegen der wiederholten Akkorde der linken Hand, die den Rhythmus beibehalten, leicht zu spielen.
In Takt 22 fehlen jedoch die wichtigsten betonten Noten, was es schwierig macht, das etwas komplexe Spielen in der rechten Hand aufzunehmen.
Die Lösung für dieses Problem ist, einfach die fehlenden Noten der linken Hand aufzufüllen!
Auf diese Art können Sie mit der rechten Hand leicht den richtigen Rhythmus üben.

Zusammengefasst ist das \enquote{Geheimnis} eines großartigen Rhythmus kein Geheimnis - er muss mit dem richtigen Zählen beginnen (was, ich muss es noch einmal betonen, nicht einfach ist).
Für fortgeschrittene Klavierspieler ist er natürlich viel mehr; er ist Magie.
Er ist das, was das Große vom Gewöhnlichen unterscheidet.
Er ist nicht nur das Zählen der Betonungen in jedem Takt, sondern die Art und Weise wie die Takte zusammengefügt sind, um die sich entwickelnde musikalische Idee zu erzeugen - die logische Komponente des Rhythmus.
So ist zum Beispiel bei Beethovens Mondschein-Sonate (Op. 27) der Anfang des dritten Satzes im Grunde der erste Satz, der mit einer verrückten Geschwindigkeit gespielt wird.
Dieses Wissen sagt uns, wie man den ersten Satz spielt, weil es bedeutet, dass die Reihe der Triolen im ersten Satz so verbunden werden muss, dass sie zu einer Kulmination mit den drei wiederholten Noten führt.
Würde man die wiederholten Noten einfach unabhängig von den vorangegangenen Triolen spielen, so würden alle diese Noten ihre Bedeutung und Wirkung verlieren.
Rhythmus ist auch der seltsame oder unerwartete Akzent, den unser Gehirn irgendwie als besonders erkennt.
Klar ist der Rhythmus ein entscheidendes Element der Musik, dem man besondere Aufmerksamkeit schenken muss.
 

\subsubsection{Legato, Staccato}
\label{c1iii1c}

Legato bedeutet nahtloses Spielen.
Das wird durch das Verbinden von aufeinander folgenden Noten erreicht - heben Sie nicht die Taste der ersten Note, bis die zweite gespielt wird.
Fraser empfiehlt ein weitgehendes Überlappen der beiden Noten.
Die ersten Momente einer Note enthalten viel \enquote{Rauschen}, sodass überlappende Noten nicht so sehr auffallen.
Das Legato ist eine Gewohnheit, die Sie in Ihr Spielen aufnehmen müssen.
Experimentieren Sie deshalb mit verschiedenen Graden des Überlappens, um zu sehen, welches Ausmaß \textit{bei Ihnen} das beste Legato erzeugt.
Üben Sie dieses dann solange, bis es zur Gewohnheit wird, sodass Sie stets denselben Effekt reproduzieren können.
Chopin hielt das Legato für die wichtigste Fertigkeit, die ein Anfänger entwickeln muss.
Chopins Musik erfordert spezielle Arten des Legatos und Staccatos (Ballade Op. 23), weshalb es wichtig ist, auf diese Elemente zu achten, wenn man seine Musik spielt.
\textbf{Wenn Sie das Legato-Spielen üben möchten, spielen Sie etwas von Chopin.}
Der \hyperref[c1iii1a1]{Basisanschlag} ist eine Voraussetzung für das Legato.

\textbf{Beim Staccato prallt der Finger von den Tasten zurück, um so einen kurzen Ton ohne Nachklang zu erzeugen.}
Es ist irgendwie erstaunlich, dass die meisten Bücher über das Klavierlernen das Staccato behandeln aber nie definieren was es ist!
Der Fänger hakt beim Staccato nicht ein, und der Dämpfer unterbricht den Ton sofort nachdem die Note gespielt wird.
Deshalb ist die Halte-Komponente des \hyperref[c1iii1a1]{Basisanschlags} nicht vorhanden.
Es gibt zwei Notationen für das Staccato, die normale (Punkt) und das Staccatissimo (gefülltes Dreieck).
Bei beiden wird die Stoßzunge nicht freigegeben; beim Staccatissimo bewegt sich der Finger viel schneller ab- und aufwärts.
Deshalb kann der Tastenweg beim normalen Staccato ungefähr die Hälfte nach unten sein, aber beim Staccatissimo kann er weniger als die Hälfte sein.
Auf diese Art wird der Dämpfer schneller zu den Tasten zurückgeführt, was zu einer kürzeren Notendauer führt.
Weil der Fänger nicht eingehakt ist, kann der Hammer \enquote{herumspringen}, was Wiederholungen bei bestimmten Geschwindigkeiten trickreich werden lässt.
Geben Sie sich deshalb nicht sofort selbst die Schuld, wenn Sie Probleme mit schnell wiederholten Staccatos haben - es kann die falsche Frequenz sein, bei der der Hammer in die falsche Richtung springt.
Indem Sie die Geschwindigkeit, den Tastenweg usw. ändern, können Sie das Problem eventuell eliminieren.
Beim normalen Staccato kehrt der Dämpfer wegen der Schwerkraft schnell auf die Saiten zurück.
Beim Staccatissimo springt der Dämpfer sogar von der oberen Dämpferstange zurück, sodass er noch schneller zurückkehrt.
Die Biegung des Hammerstiels kann beim Staccato negativ sein, was die effektive Masse des Hammers verringert;
deshalb gibt es eine große Vielfalt an Tönen, die man mit dem Staccato erzeugen kann.
Darum ändern sich die Bewegungen des Fängers, der Stoßzunge und des Dämpfers beim Staccato.
\textbf{Ganz klar: Um Staccatos gut zu spielen ist es hilfreich, die Funktionsweise des Klaviers zu verstehen.}

Staccato wird, abhängig davon wie es gespielt wird, generell in drei Gruppen eingeteilt:

\begin{enumerate}[label={\roman*.}] 
\item Fingerstaccato,\item Handgelenksstaccato,\item Armstaccato, was sowohl die Auf- und Abbewegung als auch die Drehung des Arms einschließt.
\end{enumerate}
(i) wird hauptsächlich mit den Fingern gespielt, wobei die Hand und der Arm stillgehalten werden, (ii) wird hauptsächlich mit Bewegung des Handgelenks gespielt, und (iii) wird üblicherweise mit \hyperref[c1iii4SchubZug]{Schub} (siehe III.4a) gespielt, wobei die Spielbewegung aus dem Oberarm kommt.
Wenn man von (i) nach (iii) geht, steht mehr Masse hinter den Fingern; deshalb erzeugt (i) das leichteste und schnellste Staccato und ist für einzelne, leise Noten nützlich, und (iii) erzeugt das stärkste Gefühl, ist für laute Passagen und Akkorde mit vielen Noten nützlich, ist aber auch das langsamste.
(ii) liegt dazwischen.
In der Praxis kombinieren die meisten von uns wahrscheinlich alle drei; da das Handgelenk und der Arm langsamer sind, müssen ihre Amplituden entsprechend reduziert werden, um ein schnelles Staccato zu spielen.
Manche Lehrer rümpfen über das Handgelenksstaccato die Nase und bevorzugen hauptsächlich das Armstaccato; es ist jedoch wahrscheinlich besser, eine Wahl zwischen allen dreien zu haben (oder sie zu kombinieren).
So könnten Sie zum Beispiel in der Lage sein, die Ermüdung zu reduzieren, indem Sie vom einen zum anderen wechseln, obwohl die Standardmethode zum Reduzieren der Ermüdung das Wechseln der Finger ist.
Wenn Sie das Staccato üben, dann üben Sie zunächst alle drei Formen (Finger, Hand, Arm), bevor Sie entscheiden, welches Sie benutzen oder wie Sie sie kombinieren.

Da man das Armgewicht nicht für das Staccato benutzen kann, ist Ihr ruhiger Körper der beste Bezugspunkt.
Deshalb spielt der Körper beim Staccato-Spielen eine Hauptrolle.
Die Geschwindigkeit der Staccato-Wiederholung wird durch das Maß der Auf- und Abwärtsbewegung kontrolliert: je kleiner die Bewegung, desto größer die Wiederholrate, genau wie beim Dribbeln eines Basketballs.
  

% zuletzt geändert 21.03.2010

<!-- c1iii2.html -->

\subsection{Zyklisch spielen (Chopins Fantaisie Impromptu, Op. 66)}
\label{c1iii2}

\textbf{Zirkulieren ist die beste Technik aufbauende Prozedur für neue oder schnelle Passagen, die Sie nicht beherrschen.
Zirkulieren (auch \enquote{schleifen} genannt) bedeutet, einen Abschnitt zu nehmen und diesen wiederholt, üblicherweise fortlaufend und ohne Pausen, zu spielen.}
Wenn die Verbindung, die für das fortlaufende Zirkulieren notwendig ist, die gleiche ist wie die erste Note des Abschnitts, dann zirkuliert dieser Abschnitt \enquote{natürlich}; er wird ein selbst-zirkulierender Abschnitt genannt.
Ein Beispiel ist das CGEG-Quadrupel.
Wenn die Verbindung abweicht, müssen Sie eine erfinden, die zur ersten Note hinführt, sodass Sie ohne Pausen zirkulieren können.

\textbf{Zirkulieren ist im Grunde reine Wiederholung, aber es ist wichtig, es fast als eine Anti-Wiederholungs-Prozedur zu benutzen, als einen Weg, stupides Wiederholen zu vermeiden.
Die Idee hinter dem Zirkulieren ist, dass man die Technik so schnell erwirbt, dass es unnötiges stupides Wiederholen ausschließt.}
Ändern Sie die Geschwindigkeit und experimentieren Sie mit verschiedenen Hand-, Arm- bzw. Fingerpositionen für ein optimales Spielen, um zu vermeiden, dass Sie schlechte Angewohnheiten annehmen, und achten Sie immer auf die \hyperref[c1ii14]{Entspannung}; versuchen Sie, das exakt gleiche nicht zu oft zu wiederholen.
Spielen Sie leise (auch laute Abschnitte), bis Sie die Technik erlangt haben, gehen Sie bis zu Geschwindigkeiten von mindestens 20\% über der vorgegebenen Geschwindigkeit und wenn möglich bis zur doppelten Geschwindigkeit.
Mehr als 90\% der Zirkulierzeit sollten Sie mit Geschwindigkeiten spielen, die Sie bequem und genau handhaben können.
Zirkulieren Sie dann schrittweise langsamer bis zu sehr langsamen Geschwindigkeiten.
Sie sind fertig, wenn Sie bei jeder Geschwindigkeit, für beliebig lange Zeit, ohne auf die Hand zu sehen, völlig entspannt und mit voller Kontrolle spielen können.
Es könnte sein, dass Ihnen bestimmte Geschwindigkeiten Schwierigkeiten bereiten.
Üben Sie diese Geschwindigkeiten, weil diese eventuell gebraucht werden, wenn Sie mit dem \hyperref[c1ii25]{beidhändigen Spielen} anfangen.
Üben Sie ohne das Pedal (teilweise um die schlechte Angewohnheit zu vermeiden, die Taste während des Anschlags nicht ganz herunterzudrücken), bis die Technik erworben ist.
\textbf{\hyperref[c1ii7]{Wechseln Sie oft die Hände}, um Verletzungen zu vermeiden.}

Wenn eine Technik 10.000 Wiederholungen erfordert (eine typische Erfordernis für wirklich schwieriges Material), erlaubt Ihnen das Zirkulieren, diese in der kürzest möglichen Zeit auszuführen.
Typische Zykluszeiten liegen bei einer Sekunde, sodass man für 10.000 Zyklen weniger als vier Stunden benötigt.
Wenn Sie diesen Abschnitt täglich zehn Minuten, an fünf Tagen die Woche, zirkulieren, werden 10.000 Zyklen fast einen Monat dauern.
Natürlich dauert es Monate, sehr schwieriges Material zu lernen, wenn man die besten Methoden benutzt, und \textit{viel} länger, wenn man weniger effiziente Methoden benutzt.

\textbf{Zirkulieren ist potenziell die verletzungsgefährdendste Prozedur beim Klavierüben}; seien Sie deshalb bitte vorsichtig.
Übertreiben Sie es nicht am ersten Tag, und schauen Sie, was am nächsten Tag geschieht.
Wenn Ihnen am nächsten Tag nichts weh tut, können Sie mit dem Zirkuliertraining weitermachen bzw. es steigern.
Arbeiten Sie beim Zirkulieren vor allem immer an zwei Sachen gleichzeitig, einer für die rechte Hand und einer anderen für die linke Hand, sodass Sie die Hände oft abwechseln können.
Bei jungen Menschen kann zu viel zu zirkulieren zu Schmerzen führen; hören Sie in diesem Fall mit dem Zirkulieren auf, und die Hand sollte sich innerhalb weniger Tage erholen.
Bei älteren Menschen kann zu viel zu zirkulieren Ausbrüche von Arthrose verursachen, bei denen es Monate dauern kann, bis sie abklingen.
 

\label{c1iii2fi}

Lassen Sie uns das Zirkulieren auf Chopins Fantaisie Impromptu anwenden: das Arpeggio in der linken Hand, Takt 5.
Die ersten sechs Noten zirkulieren in sich selbst, sie können es also mit diesen versuchen.
Als ich es das erste Mal versucht habe, war die Streckung für meine kleinen Hände zu groß, sodass ich zu schnell müde wurde.
Ich zirkulierte deshalb die ersten 12 Noten.
Die leichteren zweiten sechs Noten erlaubten es meinen Händen, sich ein wenig zu erholen, und ich konnte so den Abschnitt aus 12 Noten länger und mit höherer Geschwindigkeit spielen.
Wenn Sie natürlich die Geschwindigkeit wirklich steigern möchten (für die linke Hand nicht notwendig, könnte aber in diesem Stück für die rechte Hand nützlich sein), zirkulieren Sie nur das erste parallele Set (die ersten drei oder vier Noten für die linke Hand).

Dass man den ersten Abschnitt spielen kann, bedeutet nicht, dass man nun all die anderen Arpeggios spielen kann.
Sie werden sogar für die gleichen Noten eine Oktave tiefer praktisch bei Null anfangen müssen.
Natürlich wird das zweite Arpeggio einfacher sein, wenn man das erste gemeistert hat, aber Sie werden überrascht darüber sein, wie viel Arbeit es bei den Wiederholungen erfordert, wenn sich nur ein klein wenig in dem Abschnitt ändert.
Das geschieht, weil es so viele Muskeln im Körper gibt, dass das Gehirn verschiedene Gruppen auswählen kann, um Bewegungen zu erzeugen, die nur ganz leicht anders sind (und es macht es üblicherweise).
Anders als ein Roboter haben Sie wenig Einfluss darauf, welche Muskeln sich Ihr Gehirn aussucht.
Nur wenn Sie eine sehr große Zahl von solchen Arpeggios gespielt haben, fällt Ihnen das nächste leicht.
Deshalb sollten Sie davon ausgehen, dass Sie einige Arpeggios zirkulieren müssen.

Damit man versteht, wie dieses Stück von Chopin zu spielen ist, ist es hilfreich, die mathematische Grundlage des Teils der Komposition mit dem \enquote{3 gegen 4}-Timing zu analysieren.
Die rechte Hand spielt sehr schnell, sagen wir (ungefähr) vier Noten je halber Sekunde.
Gleichzeitig spielt die linke Hand mit einer langsameren Geschwindigkeit, drei Noten je halber Sekunde.
Wenn alle Noten sehr genau gespielt werden, hört das Publikum eine Notenfrequenz von zwölf Noten je halber Sekunde, weil diese Frequenz dem kleinsten Zeitintervall zwischen Noten entspricht.
\textbf{Das heißt, wenn Ihre rechte Hand so schnell spielt wie sie kann, dann hat Chopin es erreicht, dieses Stück durch das Hinzufügen des \textit{langsameren} Spielens mit der linken Hand auf Ihre dreifache Maximalgeschwindigkeit zu bringen!}

Aber warten Sie, nicht alle der zwölf Noten sind vorhanden; es sind in Wirklichkeit nur sieben, fünf Noten fehlen also.
Diese fehlenden Noten erzeugen was man ein Moiré-Muster nennt, welches ein drittes Muster ist, das auftaucht, wenn zwei nicht vergleichbare Muster überlagert werden.
Dieses Muster erzeugt einen wellenartigen Effekt innerhalb jedes Takts und Chopin verstärkte diesen Effekt, indem er in der linken Hand ein Arpeggio benutzte, das synchron mit dem Moiré-Muster wie eine Welle aufsteigt und fällt.
Die Beschleunigung um einen Faktor von drei und das Moiré-Muster sind rätselhafte Effekte, die sich beim Publikum einschleichen, weil dieses keine Ahnung hat, was sie erzeugt hat oder dass sie überhaupt existieren.
Mechanismen, die das Publikum ohne sein Wissen beeinflussen, erzeugen oft dramatischere Effekte als jene, die offensichtlich sind (wie laut, legato oder rubato).
Die großen Komponisten haben eine unglaubliche Anzahl dieser versteckten Mechanismen erfunden, und eine mathematische Analyse ist oftmals der leichteste Weg, sie hervorzukitzeln.
Chopin dachte wahrscheinlich nie in Begriffen wie nicht vergleichbaren Gruppen und Moiré-Mustern; er hat diese Konzepte allein auf Grund seiner Genialität intuitiv verstanden.

Es ist aufschlussreich, über den Grund für die fehlende erste Note des Taktes (5) für die rechte Hand zu spekulieren, denn wenn wir den Grund ermitteln können, werden wir genau wissen, wie man ihn spielen muss.
Beachten Sie, dass dies direkt am Anfang der Melodie der rechten Hand auftritt.
Am Anfang einer Melodie oder musikalischen Phrase stoßen Komponisten immer auf zwei gegensätzliche Erfordernisse: Eines ist, dass die Phrase (im Allgemeinen) leise anfangen sollte, und das zweite ist, dass die erste Note des Takts ein Abschlag ist und betont sein sollte.
Der Komponist kann geschickt beiden Erfordernissen genügen, indem er die erste Note eliminiert und so den \hyperref[c1iii1b]{Rhythmus} bewahrt und doch leise anfängt (in diesem Fall kein Ton)!
Sie werden keine Schwierigkeiten haben, zahlreiche Beispiele dieses Mittels zu finden - sehen Sie dazu \hyperref[c1iii20]{Bachs Inventionen}.
Ein weiteres Mittel ist, die Phrase am Ende eines unvollständigen Takts beginnen zu lassen, sodass der erste Abschlag des ersten vollständigen Takts kommt, nachdem ein paar Noten gespielt sind (ein klassisches Beispiel dafür ist der Anfang des ersten Satzes von Beethovens Appassionata).
Das bedeutet, dass die erste Note der rechten Hand in diesem Takt von Chopins Fantaisie-Impromptu leise sein muss und die zweite Note lauter als die erste, um den Rhythmus streng aufrechtzuerhalten (ein weiteres Beispiel der Wichtigkeit des Rhythmus!).
Wir sind nicht gewohnt, auf diese Art zu spielen; normalerweise spielen wir so, dass wir mit der ersten Note als Abschlag beginnen.
Es ist in diesem Fall wegen der Geschwindigkeit besonders schwierig; deshalb benötigt dieser Anfang eventuell zusätzliches Üben.

Diese Komposition beginnt damit, dass sie das Publikum schrittweise wie eine unwiderstehliche Einladung mit der lauten Oktave im ersten Takt, gefolgt von dem rhythmischen Arpeggio im unteren Notensystem, in ihren Rhythmus zieht.
Die fehlende Note im fünften Takt wird nach einigen Wiederholungen wiederhergestellt und somit die Moiré-Wiederholungsfrequenz und der effektive Rhythmus verdoppelt.
Im zweiten Thema (Takt 13) wird die fließende Melodie der rechten Hand durch zwei gebrochene Akkorde ersetzt und somit der Eindruck einer Vervierfachung des Rhythmus erzeugt.
Diese \enquote{rhythmische Beschleunigung} gipfelt in dem Forte-Höhepunkt der Takte 19 und 20.
Das Publikum kann dann wegen der \enquote{Besänftigung} des Rhythmus durch die verzögerte melodische Note (des kleinen Fingers) der rechten Hand und durch das schrittweise Leiserwerden der rechten Hand, das durch das \textit{diminuendo} bis zum \textit{pp} verwirklicht wird, Atem holen.
Der ganze Zyklus wird dann wiederholt, dieses Mal mit zusätzlichen Elementen, die den Höhepunkt verstärken, bis er in den absteigenden donnernden gebrochenen Akkorden endet.
Um diesen Teil zu üben, kann jeder gebrochene Akkord einzeln zirkuliert werden.
Diesen Akkorden fehlt das \enquote{3,4}-Konstrukt. Sie bringen Sie aus der rätselhaften \enquote{3,4}-Unterwelt zurück und bereiten Sie auf den langsamen Abschnitt vor.

Wie bei den meisten Stücken von Chopin, gibt es für dieses Stück kein \enquote{korrektes} Tempo.
Wenn man jedoch schneller als ungefähr zwei Sekunden je Takt spielt, neigt der \enquote{3x4}-Multiplikationseffekt dazu, zu verschwinden, und man hat üblicherweise nur noch hauptsächlich das Moiré und andere Effekte.
Das ist teilweise wegen der abnehmenden Genauigkeit mit zunehmender Geschwindigkeit so, aber wichtiger noch, weil die zwölffache Geschwindigkeit zu schnell für das Ohr wird, um ihr zu folgen.
Oberhalb von ungefähr 20 Hz beginnen Wiederholungen für das menschliche Ohr eher die Eigenschaften von Klang anzunehmen.
Deshalb funktioniert das Multiplikationsmittel nur bis ungefähr 20 Hz; oberhalb davon bekommt man einen neuen Effekt, der sogar noch mehr als eine unglaubliche Geschwindigkeit etwas besonderes sein kann - die \enquote{schnellen Noten} verwandeln sich in einen \enquote{niederfrequenten Klang}.
Somit ist bei 20 Hz eine Klanggrenze.
Deshalb ist die tiefste Note des Klaviers ein A mit ungefähr 27 Hz.
Hier ist die große Überraschung: Es gibt Hinweise, dass Chopin diesen Effekt gehört hat!
Beachten Sie, dass der schnelle Teil am Anfang die Bezeichnung \enquote{Allegro agitato} hat; das bedeutet, dass jede Note deutlich hörbar sein muss.
Das Allegro auf dem Metronom entspricht einer zwölffachen Geschwindigkeit bei 10 bis 20 Hz, der richtigen Frequenz, um die Vervielfachung zu hören - direkt unterhalb der Klanggrenze.
Das \enquote{Agitato} stellt sicher, dass diese Frequenz hörbar wird.
Wenn dieser schnelle Abschnitt nach dem Moderato-Abschnitt noch einmal kommt, ist er mit Presto bezeichnet, was 20 bis 40 Hz entspricht - Chopin wollte, dass wir ihn unterhalb und oberhalb der Klanggrenze spielen!
Es gibt also mathematische Indizien dafür, dass Chopin diese Klanggrenze kannte.

Der langsame mittlere Abschnitt wurde kurz in \hyperref[c1ii25]{Abschnitt II.25} beschrieben.
Der schnellste Weg ihn zu lernen, ist, wie bei vielen Stücken von Chopin, mit dem Auswendiglernen der linken Hand anzufangen.
Das deshalb, weil der Verlauf der Akkorde oftmals der gleiche bleibt, selbst wenn Chopin die rechte Hand durch eine neue Melodie ersetzt, da die linke Hand hauptsächlich die Begleitakkorde beisteuert.
Beachten Sie, dass das \enquote{4,3}-Timing nun durch ein \enquote{2,3}-Timing ersetzt wird, das viel langsamer gespielt wird.
Es wird für einen anderen Effekt benutzt, um die Musik sanfter zu machen und ein freieres Rubato zuzulassen.

Der dritte Teil ist dem ersten ähnlich, außer dass er schneller gespielt wird, was zu einem ganz anderen Effekt führt, und der Schluss ist anders.
Dieser Schluss ist für kleine Hände schwierig und erfordert eventuell zusätzliche Zirkulierarbeit mit der rechten Hand.
In diesem Abschnitt trägt der kleine Finger der rechten Hand die Melodie, aber die antwortende Oktavnote des Daumens bereichert die melodische Linie.
Das Stück endet mit einer nostalgischen Wiederaufnahme des langsamen Satzthemas in der linken Hand.
Unterscheiden Sie die oberste Note dieser Melodie der linken Hand (Gis - im siebten Takt von hinten) deutlich von der gleichen Note, die von der rechten Hand gespielt wird, indem Sie sie ein wenig länger halten und sie dann mit dem Pedal aushalten.

Das Gis ist die wichtigste Note in diesem Stück.
So ist der \textit{\textbf{sf}}-Anfang mit der Gis-Oktave nicht nur eine Fanfare, die das Stück einleitet, sondern eine geschickte Art, wie Chopin das Gis in den Kopf der Zuhörer einpflanzt.
Deshalb sollten Sie diese Note nicht zu eilig spielen; nehmen Sie sich Zeit, und lassen Sie sie einwirken.
Wenn Sie das Stück durchsehen, werden Sie feststellen, dass das Gis alle wichtigen Positionen besetzt.
Im langsamen Abschnitt ist das Gis ein As, was\footnote{bei \hyperref[et1]{gleichmäßig temperierter Stimmung}} dieselbe Note ist.
Dieses Gis ist ein weiteres dieser Mittel, mit denen ein großer Komponist dem Publikum wiederholt \enquote{eins überziehen} kann, ohne dass das Publikum merkt, was ihm geschieht.
Dem Klavierspieler hilft das Wissen um das Gis beim Interpretieren und Auswendiglernen des Stücks.
So kommt der konzeptionelle Höhepunkt des Stücks am Ende (wie er sollte), wenn beide Hände dasselbe Gis spielen müssen (8. und 7. Takt vom Ende her); deshalb muss das beidhändige Gis mit äußerster Sorgfalt ausgeführt werden, während man die kontinuierlich ausklingende Gis-Oktave der rechten Hand beibehält.

Unsere Analyse führt uns zum Brennpunkt, das heißt zur Frage, wie schnell man dieses Stück spielt.
Eine hohe Genauigkeit ist erforderlich, um den Zwölf-Noten-Effekt zum Vorschein zu bringen, sowie ein unmenschlich genaues Spielen oberhalb der Klanggrenze.
Wenn man dieses Stück zum ersten Mal lernt, wird die Frequenz von zwölf Noten wegen des Mangels an Genauigkeit zunächst nicht zu hören sein.
Wenn man es am Ende \enquote{packt}, hört sich die Musik urplötzlich sehr \enquote{rege} an.
Wenn man zu schnell spielt und die Genauigkeit verliert, dann kann man die Verdreifachung verlieren -  es verwascht, und das Publikum hört nur die vier Noten.
Anfänger können erreichen, dass sich das Stück schneller anhört, indem sie langsamer werden und die Genauigkeit erhöhen.
Obwohl die rechte Hand die Melodie trägt, muss die linke deutlich zu hören sein, da sonst sowohl der Zwölf-Noten-Effekt als auch das Moiré-Muster verschwinden.
Da dies ein Stück von Chopin ist, ist es nicht erforderlich, dass der Zwölf-Noten-Effekt hörbar ist; diese Komposition ist einer unendlichen Zahl von Interpretationen zugänglich, und manche von Ihnen möchten vielleicht die linke Hand außen vor lassen und sich nur auf die rechte konzentrieren und können trotzdem etwas magisches erzeugen.

Ein Vorteil des Zirkulierens ist, dass die Hand fortlaufend spielt, was das fortlaufende Spielen besser simuliert, als wenn man isolierte Abschnitte übt.
Es erlaubt Ihnen auch, mit kleinen Änderungen in den Fingerpositionen usw. zu experimentieren, um die optimalen Bedingungen für das Spielen herauszufinden.
Der Nachteil ist, dass die Handbewegungen beim Zirkulieren von denen abweichen können, die man beim Spielen des Stücks braucht.
Die Arme sind während des Zirkulierens meistens unbeweglich, während im richtigen Stück die Hände üblicherweise bewegt werden müssen.
Deshalb müssen Sie eventuell in den Fällen, in denen der Abschnitt nicht natürlich zirkuliert, das abschnittsweise Üben benutzen, ohne zu zirkulieren.
Ohne das Zirkulieren haben sie den Vorteil, dass Sie nun die Verbindung einschließen können.



<!-- c1iii3.html -->

\subsection{Triller und Tremolos (Beethovens Pathétique, 1. Satz)}
\label{c1iii3}

\subsubsection{Triller}

\textbf{Es gibt nichts besseres, um die Wirksamkeit der \hyperref[c1iii7b]{Übungen für parallele Sets} (siehe III.7b) zu demonstrieren, als sie zu benutzen, um den Triller zu lernen.
Es gibt hauptsächlich zwei Probleme zu lösen, um zu trillern: Geschwindigkeit (mit Kontrolle) und so lange weiter zu machen, wie man möchte.}
Die Übungen für parallele Sets wurden entwickelt, um genau diese Art von Problemen zu lösen, und funktionieren deshalb beim Üben von Trillern sehr gut.
\hyperref[Whiteside]{Whiteside} beschreibt eine Methode, um den Triller zu üben, die eine Art von \hyperref[c1ii9]{Akkord-Anschlag} ist.
Somit ist es nichts Neues, den Akkord-Anschlag für das Üben des Trillers zu benutzen.
Da wir nun jedoch den Lernmechanismus detaillierter verstehen, können wir die direkteste und effektivste Vorgehensweise entwickeln, indem wir parallele Sets benutzen.

Das erste zu lösende Problem sind die ersten beiden Noten.
\textbf{Wenn man die ersten beiden Noten nicht richtig anfängt, wird das Lernen des Trillers zu einer schwierigen Aufgabe.
Die Wichtigkeit der ersten beiden Noten betrifft auch Läufe, Arpeggios usw.}
Aber die Lösung ist fast trivial - wenden Sie die \hyperref[c1iii7b2]{Übung für 2-notige parallele Sets} an.
Nehmen Sie deshalb für einen mit 2323 usw. gespielten Triller die erste 3 als die Verbindung und üben Sie 23.
Üben Sie danach die 32, dann 232 usw.
So einfach ist es! Versuchen Sie es! Es funktioniert zauberhaft!
Sie sollten eventuell die Abschnitte \hyperref[c1iii7b]{III.7b} und \hyperref[c1iii7c]{III.7c} lesen, bevor Sie die parallelen Sets auf den Triller anwenden.

\textbf{Der Triller besteht aus zwei Bewegungen: Einer Bewegung der Finger und aus der Drehung des Unterarms.
Üben Sie deshalb die beiden Fertigkeiten getrennt.}
Benutzen Sie zunächst nur die Finger für das Trillern, halten Sie die Hand und den Arm völlig ruhig.
Halten Sie dann die Finger still, und trillern Sie nur mit einer Drehung des Arms.
So finden Sie heraus, ob Sie von den Fingern oder der Armdrehung gebremst werden.
Viele Schüler haben nie die schnelle Armdrehung (Wiegen des Arms) geübt, und das wird oft die langsamere Bewegung sein.
Bei schnellen Trillern ist dieses Vor- und Zurückdrehen verschwindend gering aber notwendig.
Wenden Sie die \hyperref[c1iii7b]{Übungen für parallele Sets} sowohl auf die Fingerbewegungen als auch die Armdrehungen an.
Übertreiben Sie die Bewegungen bei langsamen Trillern, und steigern Sie die Geschwindigkeit, indem Sie das Ausmaß der Bewegung reduzieren.
Das endgültige Maß der beiden Bewegungen muss nicht identisch sein, da Sie für die langsamere Bewegung (Armdrehung) ein kleineres Ausmaß benutzen werden, um ihre Langsamkeit zu kompensieren.
Wenn Sie diese Bewegungen üben, experimentieren Sie mit verschiedenen Fingerhaltungen.
Sehen Sie dazu auch den Abschnitt über \hyperref[c1iii3b]{Tremolos}, bei denen ähnliche Methoden angewandt werden - der Triller ist nur ein verkürztes Tremolo.

Wegen der Notwendigkeit des schnellen Impulsausgleichs \textbf{ist \hyperref[c1ii14]{Entspannung} für den Triller sogar noch wichtiger als fast jede andere Technik}, das heißt da die parallelen Sets nur aus zwei Noten bestehen, gibt es zu viele Verbindungen, als dass wir uns nur auf die Parallelität verlassen könnten, um auf Geschwindigkeit zu kommen.
Deshalb müssen wir in der Lage sein, den Impuls der Finger schnell zu ändern.
Bei Trillern muss die Armdrehung dem Impuls der Finger entgegenwirken.
Stress bindet die Finger an die größeren Glieder wie Handfläche und Hand und vergrößert somit die effektive Masse der Finger.
Größere Masse bedeutet langsamere Bewegung: Denken Sie an die Tatsache, dass der Kolibri schneller mit seinen Flügeln schlagen kann als der Kondor und kleinere Insekten sogar schneller als der Kolibri.
Das ist sogar dann wahr, wenn der Luftwiderstand ignoriert wird; tatsächlich ist die Luft für den Kolibri effektiv viskoser als für den Kondor, und für ein kleines Insekt ist die Luft fast so viskos, wie es Wasser für einen großen Fisch ist; trotzdem können Insekten ihre Flügel rasch bewegen, weil die Flügelmasse so klein ist.
Es ist deshalb wichtig, von Anfang an die völlige Entspannung in den Triller einzubeziehen und somit die Finger von der Hand zu befreien.
\textbf{Trillern ist eine Fertigkeit, die dauernde Pflege erfordert.
Wenn man ein guter \enquote{Trillerer} sein möchte, dann muss man das Trillerspielen täglich üben.}
Die \hyperref[c1iii7b2]{Übung \#2 für parallele Sets} (zwei Noten) ist die beste Prozedur, um den Triller in guter Verfassung zu halten, besonders wenn man ihn eine Weile nicht benutzt hat oder ihn weiter verbessern möchte.

Der Triller ist keine Reihe von Staccatos.
Die Fingerspitzen müssen so lange wie möglich am Ende des Anschlags sein, das heißt die Fänger müssen bei jeder Note eingegriffen haben.
Beachten Sie sorgfältig das minimale Anheben, das notwendig ist, damit die Repetierung funktioniert\footnote{das heißt damit die Mechanik wieder in die Ausgangsstellung geht}.
Wer üblicherweise auf einem Flügel übt, sollte sich darüber im Klaren sein, dass die Strecke für das Anheben bei einem Klavier fast doppelt so hoch sein kein.
Schnellere Triller erfordern ein geringeres Anheben; deshalb muss man auf einem Klavier den Triller eventuell verlangsamen.
Schnelle Triller sind auf elektronischen Klavieren schwierig, weil deren Mechanik schlechter ist.


\subsubsection{Tremolos (Beethovens Pathétique, 1. Satz)}
\label{c1iii3b}

Tremolos werden genau auf die gleiche Art geübt wie Triller.
Lassen Sie uns dies auf die manchmal gefürchteten, langen Oktavtremolos von Beethovens Pathétique-Sonate (Opus 13) anwenden.
Für einige Schüler scheinen diese Tremolos unmöglich zu sein, und viele haben sich beim Üben die Hände verletzt, manche davon durch zu viel Üben dauerhaft.
Andere haben wenig Schwierigkeiten.
Wenn man weiß, wie man sie üben muss, sind sie in Wahrheit ziemlich einfach.
\textbf{Das letzte, was man tun möchte, ist, stundenlang diese Tremolos in der Hoffnung zu üben, Ausdauer aufzubauen - das ist der sicherste Weg, \hyperref[c1ii22]{schlechte Angewohnheiten} zu erwerben und \hyperref[c1iii10hand]{Verletzungen} zu erleiden.}

Da Sie die Oktavtremolos bei beiden Händen benötigen, werden wir mit der linken und der rechten Hand abwechselnd üben; wenn die rechte es schneller begreift, können Sie sie \hyperref[c1ii20]{benutzen, um die linke zu unterrichten}.
Ich werde Ihnen eine Folge von Übungsmethoden vorschlagen; mit ein wenig Phantasie sollten Sie in der Lage sein, sich Ihre eigene Folge zu erstellen, die vielleicht besser für Sie ist - mein Vorschlag dient nur der Illustration.
Aus Gründen der Vollständigkeit ist er zu detailliert und zu lang.
Sie sollten die Übungsfolge je nach Ihren spezifischen Bedürfnissen und Schwächen kürzen können.

Um das \hyperref[Noten]{C2-C3}-Tremolo zu üben, üben Sie zunächst die C2-C3-Oktave (linke Hand).
Lassen Sie die Hand leicht \hyperref[c1ii9]{hoch- und runterspringen}, wiederholen Sie die Oktave mit Betonung der \hyperref[c1ii14]{Entspannung} - können Sie die Oktave ohne Ermüdung oder Stress weiterspielen, besonders wenn Sie schneller werden?
Wenn Sie müde werden, finden Sie Möglichkeiten, die Oktave zu wiederholen, ohne Ermüdung zu entwickeln, indem Sie Ihre Handposition, -bewegung usw. ändern.
Sie könnten zum Beispiel das Handgelenk schrittweise anheben und dann wieder senken - so können Sie für jedes Quadrupel vier verschiedene Handpositionen benutzen.
Wenn Sie immer noch müde werden, hören Sie auf, und wechseln Sie die Hand; üben Sie mit der rechten Hand die Ab4-Ab5-Oktave, die Sie später benötigen werden.
Wenn Sie die Oktave viermal je Schlag (einschließlich des korrekten \hyperref[c1iii1b]{Rhythmus}) ohne Ermüdung wiederholen können, versuchen Sie, sie zu beschleunigen.
Bei der maximalen Geschwindigkeit werden Sie wieder ermüden; werden Sie dann entweder langsamer oder versuchen Sie, andere Möglichkeiten zu finden, die Ermüdung zu reduzieren.
Wechseln Sie die Hände, sobald Sie sich müde fühlen.
\textbf{Spielen Sie nicht laut; ein Trick, die Ermüdung zu reduzieren, ist, leise zu spielen.
Sie können die Dynamikbezeichnungen später hinzufügen, wenn Sie die Technik erworben haben.}
Es ist extrem wichtig, leise zu üben, sodass Sie sich auf die Technik und die Entspannung konzentrieren können.
Am Anfang, wenn Sie sich anstrengen um schneller zu spielen, wird sich Ermüdung einstellen.
Wenn Sie aber die richtigen Bewegungen, Handstellungen usw. finden, werden Sie fühlen, wie die Müdigkeit die Hand verlässt, und Sie sollten die Hand ausruhen und sogar neu beleben können \textit{während Sie schnell spielen}.
Sie haben gelernt zu entspannen!

Wie der Triller besteht das Tremolo aus der Fingerbewegung und der Armdrehung.
Üben Sie zuerst das Fingertremolo, übertreiben Sie die Fingerbewegungen, spielen Sie ein sehr langsames Tremolo, heben Sie die Finger so hoch Sie können, und senken Sie die Finger mit Kraft in die Tasten.
Genauso mit der Armdrehung: Halten Sie die Finger still, und spielen Sie das Tremolo nur mit der Armdrehung und auf übertriebene Weise.
Alle Auf- und Abwärtsbewegungen müssen schnell sein; um langsam zu spielen, warten Sie einfach zwischen den Bewegungen, und üben Sie während des Wartens ein schnelles und völliges Entspannen.
Steigern Sie nun die Geschwindigkeit schrittweise, indem Sie die Bewegungen reduzieren.
Nachdem beide zufriedenstellend sind, kombinieren Sie sie; da beide Bewegungen ihren Beitrag zum Tremolo leisten, benötigen Sie von jeder einzelnen nur wenig, weshalb Sie sehr schnell spielen können.

Sie können die Geschwindigkeit sogar noch mehr steigern, wenn Sie die \hyperref[c1iii7b]{Übungen für parallele Sets} sowohl zu den Übungen der Fingerbewegung, der Armdrehung oder einer Kombination aus beiden hinzufügen.
Zuerst das 5,1-Set.
Beginnen Sie mit den wiederholten Oktaven, und ersetzen Sie dann schrittweise jede Oktave mit einem parallelen Set.
Wenn Sie zum Beispiel Gruppen von vier Oktaven spielen (4/4-Takt), fangen Sie damit an, die vierte Oktave durch ein paralleles Set zu ersetzen, dann die dritte und vierte, usw.
Bald sollten Sie alles als parallele Sets üben.
Wenn die parallelen Sets ungleichmäßig werden oder die Hand anfängt müde zu werden, gehen Sie zur Oktave zurück, um zu entspannen, oder wechseln Sie die Hand.
Arbeiten Sie an den parallelen Sets, bis Sie die zwei Noten des parallelen Sets fast \enquote{unendlich schnell} und reproduzierbar und schließlich mit guter Kontrolle und völliger Entspannung spielen können.
Bei den schnellsten Geschwindigkeiten der parallelen Sets sollten Sie Schwierigkeiten haben, zwischen parallelen Sets und Oktaven zu unterscheiden.
Verlangsamen Sie dann die parallelen Sets, sodass Sie bei allen Geschwindigkeiten mit Kontrolle spielen können.
Beachten Sie, dass in diesem Fall die 5-Note etwas lauter als die 1 sein sollte.
Sie sollten es jedoch auf beide Arten üben - mit dem Schlag auf der 5 und mit dem Schlag auf der 1 -, damit Sie eine ausgeglichene, kontrollierbare Technik entwickeln.
Wiederholen Sie das ganze Verfahren mit dem 1,5-Set.
Dieses parallele Set ist, obwohl es nicht zwingend erforderlich ist, um dieses Tremolo zu spielen (nur das vorhergehende ist notwendig), für die Entwicklung einer ausgeglichenen Kontrolle nützlich.
Sobald das 5,1- und das 1,5-Set zufriedenstellend sind, gehen Sie zu 5,1,5 oder 5,1,5,1 über (gespielt wie ein kurzer Oktavtriller).
Wenn Sie das 5,1,5,1 sofort können, besteht keine Notwendigkeit, das 5,1,5 zu üben.
Das Ziel ist hier sowohl Geschwindigkeit als auch Ausdauer, Sie sollten deshalb Geschwindigkeiten üben, die \textit{viel} schneller als die endgültige Tremolo-Geschwindigkeit sind, zumindest für diese kurzen Tremolos.
Arbeiten Sie dann an dem 1,5,1,5.

Sind die parallelen Sets erst einmal zufriedenstellend, beginnen Sie, Gruppen von zwei Tremolos zu spielen, eventuell mit einer kurzen Pause zwischen den Gruppen.
Steigern Sie dann zu Gruppen von drei und dann zu vier Tremolos.
Der beste Weg, die Tremolos zu beschleunigen, ist, zwischen Tremolos und Oktaven zu wechseln.
Beschleunigen Sie die Oktave und versuchen Sie, bei dieser schnelleren Geschwindigkeit zum Tremolo zu wechseln.
Alles, was Sie jetzt noch tun müssen, ist, die Hände abzuwechseln und Ausdauer aufzubauen.
Auch hier bedeutet Ausdauer aufzubauen nicht so sehr den Aufbau von Muskeln, sondern zu wissen, wie man entspannt und wie man die richtigen Bewegungen benutzt.
Entkoppeln Sie die Hände von Ihrem Körper; binden Sie nicht das Hand-Arm-Körper-System zu einem festen Knoten, sondern lassen Sie die Hände und Finger unabhängig vom Körper operieren.
Sie sollten frei atmen, unbeeinflusst von dem, was die Finger machen.
\textbf{Langsames Üben mit übertriebenen Bewegungen ist überraschend effektiv, kehren Sie deshalb dazu zurück, sobald Sie Probleme bekommen.}

Bei der rechten Hand (B-Oktave in Takt 149) sollte die 1 lauter als die 5 sein, aber bei beiden Händen sollten die leiseren Noten klar hörbar sein, und ihr offensichtlicher Zweck ist, die Geschwindigkeit verglichen mit der beim Spielen von Oktaven zu verdoppeln.
Erinnern Sie sich daran, leise zu üben; Sie können lauter spielen, wann immer Sie es später möchten, wenn Sie erst die Technik und Ausdauer erworben haben.
Es ist wichtig, in der Lage zu sein, bei den höchsten Geschwindigkeiten leise zu spielen und trotzdem jede Note hören zu können.
Üben Sie, bis Sie bei der endgültigen Geschwindigkeit die Tremolos länger spielen können als Sie es im Stück benötigen.
Der endgültige Effekt der linken Hand ist ein konstantes Getöse, das Sie in der Lautstärke auf und ab modulieren können.
Die untere Note trägt den Rhythmus bei, und die obere Note verdoppelt die Geschwindigkeit.
Üben Sie dann die aufsteigenden Tremolos wie sie in den Noten stehen.

Das Grave, das diesen ersten Satz beginnt, ist wegen seines ungewöhnlichen \hyperref[c1iii1b]{Rhythmus} und den schnellen Läufen in den Takten 4 und 10 nicht einfach, obwohl das Tempo langsam ist.
Der Rhythmus des ersten Takts ist nicht einfach, weil die erste Note des zweiten Schlags fehlt.
Um den korrekten Rhythmus zu lernen, benutzen Sie ein \hyperref[c1ii19]{Metronom} oder spielen Sie mit der linken Hand einzelne Rhythmusnoten, während Sie mit der rechten üben.
Obwohl der Rhythmus 4/4 ist, ist es einfacher, wenn Sie die Noten der linken Hand verdoppeln und wie bei einem 8/8-Rhythmus üben.
Der Lauf in Takt 4 ist sehr schnell; es gibt neun Noten in der letzten Gruppe von 1/128-Noten.
Sie müssen deshalb als Triolen gespielt werden und mit der doppelten Geschwindigkeit wie die vorangegangenen zehn Noten.
Das entspricht 32 Noten je Schlag, was für die meisten Klavierspieler unmöglich ist, Sie benötigen deshalb vielleicht ein wenig Rubato; \textbf{die richtige Geschwindigkeit wird, nach dem Originalmanuskript, ungefähr die Hälfte der angegebenen Geschwindigkeit sein}.
Der zehnte Takt enthält so viele Noten, dass er in der Dover-Edition zwei Zeilen umfasst!
Die letzte Gruppe von 16 1/128-Noten wird wieder mit der doppelten Geschwindigkeit der vorangegangenen Noten gespielt, für die meisten Klavierspieler unmöglich schnell.
Der viernotige chromatische Fingersatz (\hyperref[c1iii5h]{III.5h}) kann bei solchen Geschwindigkeiten nützlich sein.
Jeder Schüler, der dieses Grave zum ersten Mal lernt, muss die Noten und Schläge sorgfältig zählen, damit er eine klare Vorstellung davon bekommt, worum es geht.
Diese verrückten Geschwindigkeiten sind vielleicht ein Fehler eines Herausgebers.

Der erste (und dritte) Satz ist eine Variation des Themas im Grave-Abschnitt. 
Dieses berühmte \enquote{Dracula}-Thema wurde der linken Hand des ersten Takts entnommen; die linke Hand trägt den emotionalen Inhalt, obwohl die rechte die Melodie trägt.
Achten Sie auf das Staccatissimo und das \textit{sf} in den Takten 3 und 4.
In den Takten 7 und 8 müssen die letzten Noten der drei ansteigenden chromatischen Oktaven als 1/16-, 1/8- und 1/4-Noten gespielt werden, die zusammen mit der ansteigenden Tonhöhe und dem Crescendo den dramatischen Effekt erzeugen.
Das ist der wahre Beethoven mit maximalem Kontrast: leise - laut, langsam - schnell, einzelne Noten - komplexe Akkorde.
In Beethovens Manuskript gibt es keine Pedalzeichen.


% zuletzt geändert 24.10.2010

<!-- c1iii4.html -->

\subsection{Bewegungen der Hand, der Finger und des Körpers}
\label{c1iii4}

\subsubsection{Bewegungen der Hand}

Für das Erwerben der Technik sind bestimmte Handbewegungen erforderlich.
Wir haben zum Beispiel oben die \hyperref[c1ii11]{parallelen Sets} besprochen aber nicht aufgeführt, welche Arten von Handbewegungen notwendig sind, um sie zu spielen.
\textbf{Es ist wichtig, von Anfang an zu betonen, dass die erforderlichen Handbewegungen extrem klein sein können, fast nicht wahrnehmbar.}
Nachdem Sie ein Experte geworden sind, können Sie sie so weit übertreiben wie Sie möchten.
Deshalb sind während des Konzerts eines berühmten Künstlers die meisten Handbewegungen nicht zu erkennen (sie geschehen meistens zu schnell, sodass das Publikum sie nicht wahrnimmt), sodass die meisten sichtbaren Bewegungen Übertreibungen oder irrelevant sind.
Deshalb kann es sein, dass zwei Künstler, einer mit scheinbar ruhigen Händen und einer mit Flair und Aplomb, in Wirklichkeit die gleichen Handbewegungen des Typs benutzen, den wir hier besprechen.
\textbf{Die hauptsächlichen Handbewegungen sind Pronation und Supination, Schub und Zug, Krallen und Schnellen, Rollung und Bewegungen des Handgelenks.
Sie sind fast immer zu komplexeren Bewegungen kombiniert.}
Beachten Sie, dass sie immer paarweise auftreten (es gibt eine rechte und linke Rollung, und ähnlich für die Handgelenksbewegungen).
Sie sind auch die hauptsächlichen natürlichen Bewegungen der Hände und Finger.

Alle Fingerbewegungen müssen von den Hauptmuskeln der Arme, der Schulterblätter im Rücken und den Brustmuskeln vorne, die in der Mitte der Brust verankert sind, unterstützt werden.
Das kleinste Zucken des Fingers bezieht deshalb alle diese Muskeln mit ein.
\textbf{Es gibt es nicht, dass sich nur ein Finger bewegt - jede Fingerbewegung bezieht den ganzen Körper mit ein}.
Stressreduzierung ist wichtig für die \hyperref[c1ii14]{Entspannung} dieser Muskeln, sodass sie auf die Bewegung der Fingerspitzen reagieren und diese unterstützen können.
Die hauptsächlichen Handbewegungen werden hier nur kurz besprochen; mehr Details dazu finden Sie in den Quellen (\hyperref[Fink]{Fink} oder \hyperref[Sandor]{Sandor} und Mark für die Anatomie).


\paragraph{Pronation und Supination}
\label{c1iii4ProSup}

Die Hand kann um die Achse des Unterarms gedreht werden.
Die Einwärtsdrehung (Daumen nach unten) wird \textbf{Pronation} und die Auswärtsdrehung (Daumen nach oben) \textbf{Supination} genannt.
Diese Bewegungen kommen zum  Beispiel bei \hyperref[c1iii3b]{Oktavtremolos} ins Spiel.
Es gibt zwei Knochen in Ihrem Unterarm: der innere Knochen (Speiche, verbunden mit dem Daumen) und der äußere Knochen (Elle, verbunden mit dem kleinen Finger).
Die Drehung der Hand geschieht durch die Drehung des inneren Knochens gegen den äußeren (Handposition bezogen auf die des Klavierspielers, dessen Handflächen nach unten zeigen).
Der äußere Knochen wird vom Oberarm in Position gehalten.
Wenn die Hand gedreht wird, bewegt sich deshalb der Daumen viel mehr als der kleine Finger.
Eine schnelle Pronation ist eine gute Art, mit dem Daumen zu spielen.
Beim Spielen eines Oktavtremolos ist es leicht, den Daumen zu bewegen, aber der kleine Finger kann nur schnell bewegt werden, wenn man eine Kombination der Bewegungen benutzt.
\textbf{Somit reduziert sich das Problem, schnelle Oktavtremolos zu spielen, auf das Lösen des Problems, wie man den kleinen Finger\footnote{schnell} bewegt.}
Das Oktavtremolo wird gespielt, indem man den kleinen Finger zusammen mit dem Oberarm und den Daumen zusammen mit dem Unterarm bewegt (kombiniert mit den Fingerbewegungen).
 

\paragraph{Schub und Zug}
\label{c1iii4SchubZug}

Schub ist eine schiebende Bewegung in Richtung der Klappe\footnote{also vom Körper weg}, die üblicherweise von einem leichten Anheben des Handgelenks begleitet wird.
Mit \hyperref[c1ii2]{gebogenen Fingern} bewirkt die Schubbewegung, dass die vorwärts gerichtete Vektorkraft der Hand entlang der Knochen der Finger geführt wird.
Das fügt Kontrolle und Kraft hinzu.
Er ist deshalb für das Spielen von Akkorden nützlich.
Der Zug ist eine ähnliche Bewegung weg von der Klappe\footnote{also zum Körper hin}.
Bei diesen Bewegungen kann die gesamte Bewegung größer oder kleiner als die abwärts gerichtete Vektorkomponente (der Anschlag) sein, was eine größere Kontrolle erlaubt.
Schub ist einer der Hauptgründe, warum die Grundhaltung der Finger gekrümmt ist.
Versuchen Sie, einen großen Akkord mit vielen Noten zu spielen, zuerst indem Sie die Hand gerade herunter bewegen wie im Freien Fall und dann mit der Schubbewegung.
Beachten Sie die überlegenen Resultate mit dem Schub.
Zug ist für einige Legato- und leise Passagen nützlich.
Experimentieren Sie also immer mit dem Hinzufügen von ein wenig Schub oder Zug, wenn Sie Akkorde üben.


<h4><br>Krallen und Schnellen</h4>

Krallen ist das Bewegen der Finger zur Handfläche hin und Schnellen das Öffnen der Finger in ihre gestreckte Position.
Viele Schüler erkennen nicht, dass die Fingerspitzen zum Spielen zusätzlich zur Auf- und Abwärtsbewegung auch nach innen und außen bewegt werden können.
Das sind nützliche zusätzliche Bewegungen.
Sie fügen eine größere Kontrolle hinzu, besonders bei Legato- und leisen Passagen und ebenso beim Staccato-Spiel.
Genau wie bei \hyperref[c1iii4SchubZug]{Schub und Zug} erlauben diese Bewegungen eine größere Bewegung mit einem viel kleineren Tastenweg.
Versuchen Sie deshalb, anstatt die Finger für den Anschlag immer möglichst gerade nach unten zu führen, mit etwas Krallen oder Schnellen zu experimentieren, um zu sehen, ob es etwas bringt.
Beachten Sie, dass die Krallbewegung viel natürlicher und leichter auszuführen ist als eine Bewegung gerade nach unten.
Die gerade Abwärtsbewegung der Fingerspitze ist in Wirklichkeit eine komplexe Kombination eines Krallens und eines Schnellens.
Die Anschlagsbewegung kann manchmal vereinfacht werden, indem man die Finger flach herausstellt und nur mit kleinen Krallbewegungen spielt.
Das ist der Grund, warum man manchmal mit flachen Fingern besser als mit gekrümmten Fingern spielen kann.


\paragraph{Rollung}
\label{Rollung}

Die Rollung ist eine der nützlichsten Bewegungen.
Sie ist eine schnelle Drehung und Gegendrehung der Hand: eine schnelle Kombination von \hyperref[c1iii4ProSup]{Pronation und Supination} oder umgekehrt.
Wir haben gesehen, dass \hyperref[c1ii11]{parallele Sets} fast mit jeder Geschwindigkeit gespielt werden können.
\textbf{Beim Spielen schneller Passagen tritt das Problem der Geschwindigkeit auf, wenn wir parallele Sets verbinden müssen.}
Es gibt nicht nur eine Lösung für dieses Verbindungsproblem.
\textbf{Die Bewegung, die einer generellen Lösung am nächsten kommt, ist die Rollung, besonders wenn der Daumen beteiligt ist, wie bei \hyperref[c1iii5a]{Tonleitern} und \hyperref[Arpeggios]{Arpeggios}.}
Einmalige Rollungen können extrem schnell ohne Stress ausgeführt werden und somit dem Spielen Geschwindigkeit hinzufügen; mehrere schnelle Rollungen müssen jedoch \enquote{aufgeladen} werden; das heißt fortlaufendes schnelles Rollen ist schwierig.
Es ist aber für das Verbinden von parallelen Sets ziemlich praktisch, weil die Rollung benutzt werden kann, um die Verbindung zu spielen, und während des parallelen Sets wieder aufgeladen wird.
Um es noch einmal zu betonen, was am Anfang des Abschnitts herausgestellt wurde: Diese Rollungen und andere Bewegungen müssen nicht groß sein und sind im Allgemeinen kaum wahrnehmbar klein; somit kann die Rollung eher als Rollungsimpuls als eine tatsächliche Bewegung angesehen werden.


<h4><br>Bewegung des Handgelenks</h4>

Wir haben bereits gesehen, dass die Bewegung des Handgelenks nützlich ist, wenn mit dem Daumen oder kleinen Finger gespielt wird; die allgemeine Regel ist, das Handgelenk für den kleinen Finger anzuheben und für den Daumen zu senken.
Natürlich ist dies keine strenge Regel; es gibt viele Ausnahmen.
Die Bewegung des Handgelenks ist auch in Kombination mit anderen Bewegungen nützlich.
Durch das Kombinieren der Handgelenksbewegung mit der \hyperref[c1iii4ProSup]{Pronation und Supination} kann man Drehbewegungen für das Spielen von sich wiederholenden Passagen erzeugen, wie bei Begleitungen durch die linke Hand oder im ersten Satz von Beethovens Mondschein-Sonate.
Das Handgelenk kann sowohl auf- und abwärts als auch von einer Seite zur anderen bewegt werden.
Es sollte jede Anstrengung unternommen werden, damit der spielende Finger parallel zum Unterarm ist; das wird durch die seitliche Bewegung des Handgelenks erreicht.
Diese Anordnung bewirkt die geringste Menge von seitlicher Anspannung in den Sehnen beim Bewegen der Finger und vermindert die Wahrscheinlichkeit von Verletzungen wie dem Karpaltunnel-Syndrom.
Wenn Sie feststellen, dass sie die Angewohnheit haben, mit seitwärts abgewinkelten Handgelenken zu spielen (oder zu tippen), kann das ein Warnsignal dafür sein, dass Sie Probleme bekommen werden.
Ein lockeres Handgelenk ist auch eine Voraussetzung für eine völlige \hyperref[c1ii14]{Entspannung}.


<h4><br>Zusammenfassung</h4>

Die obigen Ausführungen sind eine kurze Übersicht der Handbewegungen.
Ein ganzes Buch kann über dieses Thema geschrieben werden.
Und wir haben noch nicht einmal die Themen über das Hinzufügen anderer Bewegungen des Ellbogens, Oberarms, der Schultern, Füße usw. berührt.
Der Schüler wird ermutigt, dieses Gebiet so weit wie möglich zu erforschen, da dies nur hilfreich sein kann.
Die gerade besprochenen Bewegungen werden selten alleine benutzt.
Parallele Sets können mit jeder Kombination der meisten oben angeführten Bewegungen gespielt werden, ohne dass man einen Finger bewegt (relativ zur Hand).
Das meinte ich in dem Abschnitt über das \hyperref[c1ii7]{Üben mit getrennten Händen} mit der Empfehlung, mit den Handbewegungen zu experimentieren und sie zu ökonomisieren.
Das Wissen um jede Art der Bewegung wird dem Schüler gestatten, jede einzelne auszuprobieren, um zu sehen, welche gebraucht wird.
Es ist in der Tat der Schlüssel zum Gipfel der Technik.



<!-- c1iii4b.html -->

\subsubsection{Mit flachen (gestreckten) Fingern spielen}
\label{c1iii4b}

Wir stellten in \hyperref[c1ii2]{Abschnitt II.2} fest, dass die anfängliche Fingerform für das Lernen des Klavierspielens die teilweise gebogene Haltung ist.
Viele Lehrer lehren die gebogene Haltung als \enquote{korrekte} Haltung für das Klavierspielen, und dass die flachen Fingerhaltungen irgendwie falsch seien.
V. Horowitz demonstrierte jedoch, dass die flache, oder gestreckte, Haltung der Finger sehr nützlich ist.
\textbf{Wir besprechen hier, warum die flache Haltung der Finger nicht nur nützlich ist, sondern auch ein entscheidender Teil der Technik ist und alle vollendeten Pianisten sie benutzen.}

Wir werden zunächst den Begriff \enquote{\textbf{flache Fingerhaltung}} als die Haltung definieren, bei der die Finger gerade von der Hand weg gestreckt sind, um die Diskussion zu vereinfachen.
Wir werden diese Definition später dahingehend verallgemeinern, dass sie besondere Arten von \enquote{nicht gebogenen} Haltungen bedeutet; diese Haltungen sind wichtig, weil sie ein Teil des Repertoires an Fingerhaltungen sind, das Sie benötigen, um ein richtiger Klavierspieler zu werden.

\textbf{Die wichtigsten Vorteile der flachen Haltung sind, dass sie die Bewegung der Finger vereinfacht und eine völlige \hyperref[c1ii14]{Entspannung} ermöglicht}, das heißt die Zahl der Muskeln, die gebraucht werden, um die Fingerbewegung zu kontrollieren, ist kleiner als bei der gebogenen Haltung, weil alles was man tun muss, das Drehen des gesamten Fingers um den Knöchel ist.
Bei der gebogenen Haltung muss man jeden Finger jedesmal wenn man eine Note anschlägt genau im richtigen Maß strecken, damit man mit dem Finger den korrekten Winkel zur Oberfläche der Taste aufrechterhält.
Bei der Bewegung mit der flachen Fingerhaltung werden nur die Hauptmuskeln benutzt, die nötig sind, um die Tasten herunterzudrücken.
\textbf{Mit flachen Fingern zu üben kann die Technik verbessern, weil man nur die wichtigsten Muskeln und Nerven trainiert.}
Versuchen Sie, um die Komplexität der gebogenen Haltung zu demonstrieren, das folgende Experiment.
Strecken Sie zunächst den Zeigefinger Ihrer rechten Hand gerade aus (flache Fingerhaltung) und wackeln Sie mit ihm schnell auf und ab, wie Sie es beim Klavierspielen tun würden.
Behalten Sie nun diese Auf- und Abwärtsbewegung bei, und krümmen Sie den Finger schrittweise so weit Sie können.
Sie werden feststellen, dass es, wenn Sie den Finger krümmen, schwieriger wird, die Fingerspitze auf und ab zu bewegen, bis es unmöglich wird, wenn der Finger komplett eingerollt ist.
Ich habe dieses Phänomen \textbf{\enquote{Krümmungslähmung}} genannt.
Wenn es Ihnen gelingt, die Fingerspitze zu bewegen, können Sie es, verglichen mit der gestreckten Haltung, nur sehr langsam tun, weil Sie eine ganz neue Muskelgruppe benutzen müssen.
Tatsächlich ist die einfachste Art, Ihre Fingerspitze in der gebogenen Haltung schnell auf und ab zu bewegen, die ganze Hand zu bewegen.

Deshalb brauchen Sie mit der gebogenen Fingerhaltung mehr Geschicklichkeit als mit der flachen Haltung, um mit derselben Geschwindigkeit zu spielen.
\textbf{Entgegen der Meinung vieler Klavierspieler kann man mit der flachen Haltung schneller spielen als mit der gebogenen, weil jegliche Krümmung ein bestimmtes Maß an Krümmungslähmung nach sich zieht.}
Das wird besonders wichtig, wenn die Geschwindigkeit und/oder ein Mangel an Technik während des Übens von etwas schwierigem Stress erzeugen.
Das Maß an Stress ist bei der gebogenen Haltung größer und dieser Unterschied kann ausreichend sein, um eine Geschwindigkeitsbarriere zu erzeugen.

In der Literatur (Jaynes, Kapitel 6) wird behauptet, dass die tiefen Hohlhandmuskeln (die Mm. lumbricales unter und die Mm. interossei zwischen den Mittelhandknochen) für das Klavierspielen wichtig seien.
Es gibt jedoch keine Untersuchungen, die diese Behauptungen stützen, und man weiß nicht, ob diese Muskeln bei der flachen Fingerhaltung eine Rolle spielen.
Im Allgemeinen glaubt man, dass diese Muskeln hauptsächlich dazu benutzt werden, die Krümmung der Finger zu kontrollieren, sodass bei der flachen Fingerhaltung nur die Muskeln in den Armen benutzt werden, um die Finger zu bewegen, und die Mm. lumbricales die Finger nur in Position halten (gebogen oder flach), was die Bewegung vereinfachen und bei der flachen Haltung größere Kontrolle und Geschwindigkeit erlauben soll.
Man ist also heute nicht sicher, ob die Mm. lumbricales eine höhere Geschwindigkeit erlauben oder Krümmungslähmung verursachen.

Obwohl die flache Haltung einfacher ist, \textbf{sollten alle Anfänger zuerst die gebogene Haltung lernen und die flache Haltung nicht lernen, bis sie benötigt wird}.
Wenn Anfänger mit der einfacheren flachen Haltung anfangen, werden sie die gebogene Haltung nie richtig gut lernen.
Anfänger, die versuchen, mit der flachen Haltung schnell zu spielen, werden wahrscheinlich mit \hyperref[c1ii11]{parallelen Sets} mit festen Phasen spielen anstatt mit unabhängigen Fingern.
Das führt zu Kontrollverlust und ungleichmäßigen Geschwindigkeiten.
Haben sich diese schlechten Angewohnheiten erst einmal gebildet, ist es schwierig, die Unabhängigkeit der Finger zu lernen.
Aus diesem Grund verbieten viele Lehrer ihren Schülern, mit flachen Fingern zu spielen, was ein schrecklicher Fehler ist.
\hyperref[Sandor]{Sandor} nennt die flachen Fingerhaltungen \enquote{falsche Haltungen}, aber \hyperref[Fink]{Fink} empfiehlt bestimmte Haltungen, die klar flache Fingerhaltungen sind (wir werden unten verschiedene flache Haltungen behandeln).
\textbf{\hyperref[c1iii3]{Triller} erfordern oft wegen ihrer komplexen Art die gebogene Haltung.}

Die meisten Klavierspieler, die für sich selbst lernen, benutzen meistens flache Fingerhaltungen.
Sehr junge Kinder (unter 4 Jahren) haben üblicherweise Schwierigkeiten, ihre Finger zu krümmen.
Aus diesem Grund benutzen Jazz-Pianisten die flachen Fingerhaltungen öfter als klassische Pianisten (weil viele sich das Klavierspielen zunächst selbst beigebracht haben), und klassische Lehrer weisen zu Recht darauf hin, dass die frühen Jazz-Pianisten eine unterlegene Technik hatten.
Tatsächlich wies der frühe Jazz viel weniger technische Schwierigkeiten als die klassische Musik auf.
Dieser Mangel an Technik resultierte jedoch aus einem Mangel an Unterricht, nicht daraus, dass sie flache Fingerhaltungen benutzten.
Somit sind die flachen Fingerhaltungen nichts Neues, ziemlich intuitiv (nicht alles Intuitive ist schlecht) und eine natürliche Art zu spielen.
Deshalb ist der Weg zu guter Technik eine sorgfältige Balance zwischen dem Üben mit gebogenen Fingern und dem Wissen, wann die flachen Haltungen zu benutzen sind.
\textbf{\textit{Neu in diesem Abschnitt ist das Konzept, dass die gebogene Haltung nicht von Natur aus überlegen ist, und dass die flachen Fingerhaltungen ein notwendiger Teil der fortgeschrittenen Technik sind.}}

Der vierte Finger ist für die meisten Menschen besonders problematisch.
Ein Teil dieser Schwierigkeiten erwächst aus der Tatsache, dass er der Finger ist, der am schwierigsten zu heben ist, was es schwierig macht, schnell zu spielen und zu vermeiden, versehentlich zusätzliche Noten zu treffen.
Diese Probleme sind wegen der Komplexität der Bewegung und der Krümmungslähmung eng mit der gebogenen Haltung verbunden.
In der vereinfachten Anordnung mit flachen Fingern sind diese Schwierigkeiten reduziert, sodass der vierte Finger unabhängiger wird und leichter anzuheben ist.
Wenn Sie Ihre Hand in der gebogenen Haltung auf eine glatte Fläche legen und den vierten Finger heben, wird er eine bestimmte Strecke aufwärts gehen; wenn Sie nun die gleiche Prozedur mit der flachen Fingerhaltung wiederholen, wird die Fingerspitze \textit{zweimal} so weit hochgehen.
Deshalb ist es einfacher, die Finger, und besonders den vierten Finger, in der flachen Haltung anzuheben.
Die Leichtigkeit des Anhebens reduziert den Stress beim schnellen Spielen.
Wenn man versucht, schwierige Passagen mit der gebogenen Haltung schnell zu spielen, werden sich einige Finger (besonders 4 und 5) manchmal zu viel krümmen, was noch mehr Stress erzeugt und die Notwendigkeit, diese Finger \enquote{von sich zu werfen}, um eine Note zu spielen.
Diese Probleme können vermieden werden, indem man die flachen Fingerhaltungen benutzt.

Ein weiterer Vorteil der flachen Fingerhaltung ist, dass sie Ihre Reichweite vergrößert, weil die Finger weiter ausgestreckt sind.
Aus diesem Grund verwenden die meisten Klavierspieler (besonders jene mit kleinen Händen) bereits die flache Haltung für das Spielen \hyperref[c1iii7e]{großer Akkorde} usw., oftmals ohne es zu merken.
Solche Menschen fühlen sich jedoch manchmal wegen des Mangels an Krümmung \enquote{schuldig} und versuchen, soviel Krümmung wie möglich einzubauen, was Stress erzeugt.

Noch ein Vorteil der flachen Fingerhaltung ist, dass die Finger die Tasten mit den Fingerpolstern statt mit den Fingerspitzen herunterdrücken.
Dieser fleischige Teil reagiert empfindlicher auf Druck, und die Fingernägel stören weniger.
Wenn jemand etwas anfasst, um es zu fühlen, benutzt er immer diesen Teil des Fingers, nicht die Fingerspitze.
\textbf{Dieses zusätzliche Polster und die zusätzliche Empfindlichkeit können mehr Gefühl und Kontrolle, sowie mehr Schutz vor Verletzungen bieten.}
Bei der gebogenen Haltung kommen die Finger fast senkrecht auf die Tastenoberfläche herunter, sodass man mit den Fingerspitzen spielt, wo es zwischen dem Knochen und der Tastenoberfläche das geringste Polster gibt.
Wenn man sich die Fingerspitzen durch zu hartes Üben mit der gebogenen Haltung verletzt hat, kann man ihnen eine Pause gönnen, indem man die flache Fingerhaltung benutzt.
Zwei Arten von Verletzungen können an der Fingerspitze auftreten, wenn man die gebogene Haltung benutzt, und beide können vermieden werden, wenn man flache Finger benutzt.
Die erste ist einfach eine Prellung von zu vielem Klopfen.
Die zweite ist das Lösen des Fleisches unter dem Fingernagel, was häufig daraus resultiert, dass man die Fingernägel zu kurz schneidet.
Diese zweite Art der Verletzung ist gefährlich, weil sie zu einer schmerzhaften Infektion führen kann.
Sogar wenn man ziemlich lange Fingernägel hat, kann man immer noch mit der flachen Fingerhaltung spielen.

Noch wichtiger ist, dass man mit flachen Fingern \textbf{die schwarzen Tasten spielen kann, indem man hauptsächlich die großen Bereiche an der Unterseite der Finger benutzt; diese große Fläche kann benutzt werden, um zu vermeiden, dass man die schwarzen Tasten verfehlt}, die man mit der gebogenen Haltung leicht verfehlen kann, weil sie so schmal sind.
\textbf{Spielen Sie bei schnellen Passagen und großen Akkorden die schwarzen Tasten mit flachen Fingern und die weißen Tasten mit gebogenen Fingern}; das kann Ihre Geschwindigkeit und Genauigkeit in hohem Maß steigern.

Wenn die Finger flach ausgestreckt sind, kann man weiter in Richtung der Klappe reichen.
Da man bei dieser Haltung näher an den Waagepunkten (am Waagbalkenstift) ist und somit eine kleinere Hebelwirkung erhält, erfordert es eine etwas größere Kraft, um die Tasten herunterzudrücken.
\textbf{Das resultierende (effektiv) höhere Tastengewicht gestattet es Ihnen, ein leiseres \textit{pp} zu spielen.
Somit führt die Fähigkeit, näher an die Waagepunkte heranzugehen, zur Fähigkeit, das effektive Tastengewicht zu vergrößern.}
Das höhere Tastengewicht gestattet eine größere Kontrolle und ein leiseres Pianissimo.
Obwohl die Veränderung des Tastengewichts gering ist, wird dieser Effekt bei höherer Geschwindigkeit in hohem Maß verstärkt.
\textbf{Andere argumentieren, dass die Enden der Tasten einen größeren Hebel bieten, sodass man eine größere Kontrolle für das \textit{pp} erlangt.}
Probieren Sie deshalb beide Methoden, und schauen Sie, welche für Sie am besten funktioniert. 

Mit flachen Fingern zu spielen gestattet auch ein lauteres Fortissimo, besonders bei den schwarzen Tasten.
Dafür gibt es zwei Gründe.
Erstens ist, wie oben beschrieben, die Fläche des Fingers, die für den Kontakt zur Verfügung steht, größer, und das Polster ist dicker.
Deshalb kann man eine größere Kraft mit einer geringeren Chance der Verletzung oder des Schmerzes übertragen.
Zweitens ist die gesteigerte Genauigkeit, die aus der größeren Kontaktfläche resultiert, beim Erzeugen eines zufriedenstellenden, respekteinflößenden und reproduzierbaren Fortissimos hilfreich.
Bei der gebogenen Haltung ist die Wahrscheinlichkeit, die schwarzen Tasten zu verfehlen oder von ihnen abzurutschen, manchmal für ein volles Fortissimo zu Angst einflößend.
Befürworter der gebogenen Haltung argumentieren, dass diese die einzige ist, die stark genug ist, um das lauteste Fortissimo zu spielen.
Das stimmt nicht; Athleten, die Fingerstände ausführen, tun dies mit flachen Fingerhaltungen, nicht mit den Fingerspitzen.
Tatsächlich erleiden Klavierspieler, die mit der gebogenen Haltung zu viel üben oft Verletzungen der Fingerspitzen.

Die Fähigkeit, leichter fortissimo zu spielen, legt den Schluss nah, dass die flache Fingerhaltung viel entspannter sein kann als die gebogene Haltung.
Das stellt sich als wahr heraus, aber es gibt einen zusätzlichen Mechanismus, der die Entspannung vergrößert.
Mit flachen Fingern kann man sich darauf verlassen, dass die Sehnen an der Unterseite die Finger gerade halten, wenn man auf die Tasten herunterdrückt.
Das heißt, dass man sich, anders als bei der gebogenen Haltung, kaum anstrengen muss, um die Finger gerade zu halten, wenn man die Tasten herunterdrückt, weil, außer wenn man sehr gelenkig ist, die Sehnen an der Unterseite verhindern, dass sich die Finger zurückbiegen.
Lernen Sie deshalb, wenn Sie das Spielen mit flachen Fingern üben, diese Sehnen dazu zu benutzen, Ihnen beim Entspannen zu helfen.
Seien Sie vorsichtig, wenn Sie das erste Mal damit beginnen, flache Finger für das Spielen eines Fortissimos zu benutzen.
Wenn Sie völlig entspannen, können Sie eine Verletzung dieser Sehnen durch Überdehnung riskieren, besonders bei den kleinen Fingern, weil deren Sehnen so klein sind.
Wenn Sie anfangen Schmerzen zu spüren, dann versteifen Sie entweder die Finger während des Anschlags oder hören Sie mit dem Spielen mit flachen Fingern auf und krümmen diesen Finger.
Wenn Sie mit gebogenen Fingern fortissimo spielen, müssen Sie sowohl die Streck- als auch die Beugemuskeln jedes Fingers kontrollieren, um die Finger in der gebogenen Haltung zu halten.
Bei der flachen Haltung können Sie die Streckmuskeln völlig entspannen und nur die Beugemuskeln benutzen, somit den Stress fast völlig eliminieren (der daraus resultiert, dass die beiden Muskelgruppen gegeneinander arbeiten) und den Vorgang für das Niederdrücken der Taste um mehr als 50\% vereinfachen.

Die beste Möglichkeit, mit dem Üben des Spielens mit flachen Fingern anzufangen, ist das Üben der H-Dur-Tonleiter.
Bei dieser Tonleiter spielen alle Finger außer dem Daumen und dem kleinen Finger die schwarzen Tasten.
Da diese beiden Finger im Allgemeinen in Läufen keine schwarzen Tasten spielen, ist das genau das, was Sie üben möchten.
Der Fingersatz für diese Tonleiter ist bei der rechten Hand der \hyperref[table]{Standard-Fingersatz}, aber die linke Hand muss mit dem vierten Finger auf dem H beginnen.
Sie möchten vielleicht zunächst den folgenden Abschnitt (III.5) über das \hyperref[c1iii5]{Spielen schneller Tonleitern} lesen, bevor Sie mit dieser Übung fortfahren, weil Sie wissen müssen, wie man mit Daumenübersatz spielt, wie man die Glissandobewegungen benutzt usw.
Durch das Fühlen der Tasten werden Sie keine einzige Note verfehlen, weil sie im Voraus wissen, wo die Tasten sind.
Wenn eine Hand schwächer als die andere ist, wird dieser Unterschied mit flachen Fingern dramatischer zu Tage treten. Die flache Fingerhaltung offenbart die technischen Fertigkeiten und Mängel deutlicher, weil der Hebel unterschiedlich ist (die Finger sind effektiv länger) und Ihre Finger empfindlicher sind.
Benutzen Sie in diesem Fall \hyperref[c1ii20]{die stärkere Hand, um die schwächere zu unterrichten}, wie man spielt.
Mit flachen Fingern zu üben, mag eine der schnellsten Arten sein, die schwächere Hand zu ermutigen, mit der stärkeren gleichzuziehen, weil man direkt mit den Hauptmuskeln arbeitet, die für die Technik relevant sind.

Wenn Sie beim Spielen mit der flachen Fingerhaltung auf irgendwelche Schwierigkeiten stoßen, versuchen Sie die \hyperref[c1iii7b]{Übungen für parallele Sets} mit den schwarzen Tasten.
Spielen Sie alle fünf schwarzen Tasten mit den fünf Fingern: die zweinotige Gruppe mit Daumen und Zeigefinger und die dreinotige Gruppe mit den verbleibenden drei Fingern.
Anders als bei der H-Dur-Tonleiter wird diese Übung auch den Daumen und den kleinen Finger entwickeln.
Bei dieser Übung (oder bei der H-Dur-Tonleiter) können Sie mit allen Arten von Handpositionen experimentieren.
Anders als bei der gebogenen Haltung \textbf{kann man spielen, während die Handfläche die Oberfläche der weißen Tasten berührt.
Man kann auch das Handgelenk heben, sodass sich die Finger in Wahrheit rückwärts biegen (entgegen der Richtung beim Krümmen), wie bei der \hyperref[c1iii5wagen]{Wagenradbewegung} (III.5e).
Es gibt auch eine Haltung der flachen Finger dazwischen, bei der die Finger gerade bleiben aber im Fingergelenk nach unten gebeugt sind.}
Ich nenne das die \enquote{\textbf{Pyramidenhaltung}}, weil die Hand und die Finger eine Pyramide mit den Knöcheln als Spitze bilden.
Diese Pyramidenhaltung kann für sehr schnelle Passagen sehr wirksam sein, weil sie die Vorteile der gebogenen Haltung und der gestreckten Haltung kombiniert.

Die Nützlichkeit dieser verschiedenen Fingerhaltungen macht es notwendig, dass wir die Definition des \enquote{Spielens mit flachen Fingern} erweitern.
Die gestreckte Haltung ist nur ein Extremfall, und es gibt eine beliebige Zahl von Haltungsvariationen zwischen der völlig flachen Haltung und der gebogenen Haltung.
Zusätzlich zur Pyramidenhaltung können Sie die Finger im ersten Gelenk nach den Knöcheln beugen.
Das nennen wir die \enquote{\textbf{Spinnenhaltung}}.
\textbf{\textit{Der kritische Punkt ist hierbei, dass das letzte Gelenk (vor den Fingernägeln) völlig entspannt sein und man es strecken können muss, wenn man die Taste herunterdrückt.
Deshalb ist die verallgemeinerte Definition der flachen Fingerhaltung, dass das dritte (beim Daumen das zweite) Fingerglied völlig entspannt und gestreckt ist.}}
Die Glieder sind von der Handfläche zur Fingerspitze mit 1-3 durchnummeriert (der Daumen hat nur 1 und 2).\footnote{Der Daumen hat scheinbar ebenfalls drei äußere Glieder; das \enquote{erste} gehört aber anatomisch zur Handfläche. Wenn man es mitzählen würde, dann hätten die anderen Finger analog dazu vier Glieder.}
Wir werden sowohl die Pyramiden- als auch die Spinnenhaltung \enquote{flache Fingerhaltungen} nennen, weil alle drei flachen Haltungen zwei wichtige Eigenschaften gemeinsam haben: Das dritte Glied des Fingers ist niemals gebogen und ist immer entspannt, und man spielt mit der empfindlichen Unterseite der Fingerspitze (Fotos dazu finden Sie bei \hyperref[Prokop]{Prokop} auf den Seiten 13 bis 15).
Ab jetzt benutzen wir diese weiter gefasste Definition der flachen Fingerhaltungen.
Obwohl die Finger bei vielen dieser Haltungen gebeugt sind, werden wir sie flache Haltungen nennen, um sie von der gebogenen Haltung zu unterscheiden.
Der größte Teil der Krümmungslähmung resultiert aus der Beugung des dritten Glieds.
Das kann demonstriert werden, indem man nur das dritte Glied beugt (wenn man es kann) und dann versucht, diesen Finger schnell zu bewegen.
Beachten Sie, dass die völlige Entspannung des dritten Glieds nun Teil der Definition der flachen Fingerhaltungen ist.
Die flache Haltung vereinfacht die Berechnungen im Gehirn, weil man die Beugemuskeln des dritten Glieds fast völlig ignoriert.
Das sind zehn weniger zu kontrollierende Beugemuskeln, und diese sind besonders unhandliche und langsame Muskeln; deshalb kann es die Geschwindigkeit der Finger steigern, wenn man sie ignoriert.
Wir sind bei der Erkenntnis angelangt, dass \textbf{\textit{die gebogene Haltung zum Spielen von fortgeschrittenem Material geradezu falsch ist.
Die verallgemeinerte flache Fingerhaltung ist zum Spielen mit Geschwindigkeiten, die von fortgeschrittenen Spielern gebraucht werden, genau das richtige!}}
Wie weiter unten besprochen, gibt es jedoch bestimmte Situationen, in denen man bestimmte einzelne Finger schnell krümmen muss, um eine weiße Taste zu erreichen und um zu vermeiden, dass man mit den Fingernägeln an die Klappe stößt.
Die Wichtigkeit der verallgemeinerten flachen Fingerhaltung kann nicht überbetont werden, weil sie eines der Schlüsselelemente der Entspannung ist, das oftmals völlig ignoriert wird.

Die flache Fingerhaltung bietet eine viel größere Kontrolle, weil das Polster auf der Unterseite der Fingerspitze der empfindlichste Teil des Fingers ist und das entspannte dritte Glied sich wie ein Stoßdämpfer verhält.
Das gestattet Ihnen; die Tasten zu erfühlen; bei einem Auto machen Stoßdämpfer nicht nur das Fahren bequemer, sondern halten das Rad auch für eine bessere Kontrolle auf der Straße.
Wenn Sie Schwierigkeiten damit haben, in einer Komposition die Farbe herauszubringen, wird es einfacher werden, wenn Sie die flachen Fingerhaltungen benutzen.
\textbf{\textit{In der gebogenen Haltung mit den Fingerspitzen zu spielen, ist so, als ob man ein Auto ohne Stoßdämpfer fahren oder ein Klavier mit abgenutzten Hämmern spielen würde.
Der Ton neigt dann dazu, schriller zu sein; man ist effektiv auf eine Tonfarbe beschränkt.}}
Indem Sie die flachen Fingerhaltungen benutzen, können Sie die Tasten besser fühlen und den Ton und die Farbe leichter kontrollieren.
Da Sie das dritte Fingerglied völlig entspannen und auch einige der Streckmuskeln ignorieren können, sind die flachen Fingerbewegungen einfacher, und Sie können schneller spielen, besonders bei schwierigem Material wie schnellen \hyperref[c1iii3]{Trillern}.
\textbf{\textit{Wir sind deshalb bei einem sehr wichtigen allgemeinen Konzept angekommen: Wir müssen uns selbst von der \enquote{Tyrannei} der einen festen gebogenen Haltung befreien.}}
Wir müssen lernen, alle verfügbaren Fingerhaltungen zu benutzen, weil jede ihre Vorteile hat.

Sie möchten vielleicht die Bank niedriger stellen, um mit dem flachen Teil der Finger spielen zu können.
Wenn die Bank niedriger gestellt wird, wird es üblicherweise notwendig, sie weiter weg vom Klavier zu stellen, damit man zwischen dem Körper und dem Klavier mehr Platz zum Bewegen der Arme und Ellbogen hat.
Mit anderen Worten: Viele Klavierspieler sitzen zu hoch und zu nahe am Klavier, was nicht wahrnehmbar ist, wenn man mit gebogenen Fingern spielt.
Deshalb bieten Ihnen die flachen Fingerhaltungen eine präzisere Möglichkeit, die Höhe und die Position der Sitzbank zu optimieren.
Bei diesen niedrigeren Höhen könnten manchmal die Handgelenke und sogar die Ellbogen beim Spielen unter die Höhe der Tastatur fallen; das ist durchaus zulässig.
Weiter weg vom Klavier zu sitzen, bietet Ihnen auch mehr Platz, um sich zum Fortissimo-Spielen vorzulehnen.

Sämtliche flachen Fingerhaltungen können auf einem Tisch geübt werden.
Legen Sie für die völlig flache Haltung einfach alle Finger und die Handfläche flach auf einen Tisch, und üben Sie, jeden Finger - besonders Finger 4 - unabhängig von den anderen anzuheben.
Üben Sie die Pyramiden- und die Spinnenhaltung, indem Sie nur die fleischigen Polster der Fingerspitzen auf dem Tisch halten und beim Herunterdrücken das dritte Glied völlig entspannen, so dass es sogar nach hinten gebogen wird.
Bei der Pyramidenhaltung wird das zu einer Art Streckübung für alle Beugesehnen, und die beiden letzten Glieder sind entspannt.
Sie werden auch feststellen, dass die flache Fingerhaltung beim Tippen auf einer Schreibmaschinen- oder Computertastatur gut funktioniert.

Der vierte Finger bereitet im Allgemeinen jedem Probleme, und es gibt eine Übung zum Verbessern seiner Unabhängigkeit, die man mit der Spinnenhaltung ausführen kann.
Setzen Sie auf dem Klavier die Finger 3 und 4 auf das C\# und das D\# und die restlichen Finger auf weiße Tasten.
Drücken Sie alle fünf Tasten herunter.
Die erste Übung ist, mit Finger 4 zu spielen und ihn dabei so weit wie möglich anzuheben.
Bei allen Übungen müssen Sie die nicht spielenden Finger unten behalten.
Die zweite Übung ist, die Finger 3 und 4 abwechselnd zu spielen (343434 usw.), wobei Finger 4 wieder so weit wie möglich angehoben wird, aber Finger 3 nur hoch genug, um die Note zu spielen und so, dass der Finger ständig im Kontakt mit der Tastenoberfläche bleibt (ziemlich schwierig, besonders wenn man versucht, es zu beschleunigen).
Die meisten können den vierten Finger in der Spinnenhaltung am höchsten anheben, was zeigt, dass das eventuell im Allgemeinen die beste Haltung zum Spielen ist.
Spielen Sie mit dem Finger 3 lauter als mit dem Finger 4 (Betonung auf die 3).
Wiederholen Sie es mit den Fingern 4 und 5, mit der Betonung auf der 5, und halten Sie die 5 so weit wie möglich auf den Tasten.
Spielen Sie in der letzten Übung parallele Sets - (3,4), (4,3), (5,4) und (4,5) -, wobei alle anderen Finger ihre Tasten vollständig herunterdrücken; wählen Sie beliebige Tasten, die für Sie bequem sind.
Diese Übungen mögen zunächst schwierig erscheinen, Sie werden aber überrascht sein, wie schnell Sie (innerhalb von ein paar Tagen) in der Lage sein werden, sie zu spielen.
Hören Sie aber nicht auf, nur weil Sie diese Übungen ausführen können.
Üben Sie weiter, bis Sie sie mit völliger Kontrolle und Entspannung sehr schnell ausführen können, weil Sie sonst keinen Nutzen davon haben.
Diese Übungen simulieren die schwierige Situation, in der Sie mit den Fingern 3 und 5 spielen, während Sie den Finger 4 über den Tasten halten.

Die zusätzliche Reichweite, die große Kontaktfläche und das zusätzliche Polster unter den Fingern machen das Legato-Spielen mit flachen Fingern einfacher und anders als das Legato mit der gebogenen Haltung.
Die flache Fingerhaltung vereinfacht es auch, zwei Noten mit einem Finger zu spielen, besonders weil man spielen kann, ohne dass die Finger parallel zu den Tasten sind und man eine sehr große Fläche unter den Fingern benutzen kann, um mehr als eine Taste unten zu halten.
Da Chopin für sein Legato bekannt war, gut mehrere Noten mit einem Finger spielen konnte und empfahl, die H-Dur-Tonleiter zu üben, benutzte er wahrscheinlich das Spielen mit flachen Fingern.
Yvonne Combe, die den ersten Anstoß zu diesem Buch gab, lehrte die flachen Fingerhaltungen und wies darauf hin, dass sie besonders nützlich sind, um Chopin zu spielen.
Ein Legato-Trick, den sie lehrte, war, mit der flachen Haltung anzufangen und dann den Finger zur gebogenen Haltung einzurollen, sodass man die Hand bewegen kann, ohne die Finger von den Tasten zu heben.
Man kann es auch umgekehrt machen, wenn man von den schwarzen Tasten zu den weißen heruntergeht\footnote{das heißt man lässt die Finger zunächst auf den schwarzen Tasten und streckt die Finger während man die Hand zu den weißen Tasten nach unten bewegt}.

Sie können die Nützlichkeit der flachen Fingerhaltung demonstrieren, indem Sie sie auf irgend etwas anwenden, das Ihnen Schwierigkeiten bereitet.
So hatte ich zum Beispiel ein paar Stress-Probleme beim Beschleunigen von \hyperref[c1iii20]{Bachs Inventionen}, weil sie die Unabhängigkeit der Finger erfordern, besonders der Finger 3, 4 und 5.
Während ich nur mit der gebogenen Haltung übte, fühlte ich, dass ich begann, bei ein paar Passagen, für die ich nicht genug Fingerunabhängigkeit hatte, eine Geschwindigkeitsbarriere aufzubauen.
Als ich die flache Fingerhaltung anwandte, wurde es viel leichter, sie zu spielen.
Das erlaubte mir schließlich, mit höheren Geschwindigkeiten und mit größerer Kontrolle zu spielen.
Die Bach-Inventionen sind gute Stücke zum Üben der flachen Fingerhaltungen, was nahelegt, dass Bach eventuell an die flachen Fingerhaltungen dachte, als er sie komponierte.

Eine Diskussion des Spielens mit flachen Fingern wäre ohne eine Diskussion darüber, warum man die gebogene Haltung benötigt, sowie einiger ihrer Nachteile unvollständig.
Diese Haltung ist nicht wirklich eine mit Absicht gebogene Haltung, sondern eine entspannte Haltung, bei der es bei den meisten Menschen eine natürliche Krümmung gibt.
Diejenigen, deren entspannte Haltung zu gestreckt ist, müssen eventuell eine leichte Krümmung hinzufügen, um die ideale gebogene Haltung zu erreichen.
Bei dieser Haltung berühren alle Finger die Tasten in einem Winkel zwischen 45 und 90 Grad (der Daumen mag einen etwas kleineren Winkel bilden).
Es gibt bestimmte für das Klavierspielen absolut notwendige Bewegungen, die die gebogene Haltung erfordern.
Einige davon sind: bestimmte weiße Tasten spielen (wenn die anderen Finger schwarze Tasten spielen), zwischen den schwarzen Tasten spielen und vermeiden, mit den Fingernägeln an die Klappe zu stoßen.
Besonders für Klavierspieler mit großen Händen ist es notwendig, die Finger 2, 3 und 4 zu krümmen, wenn die Finger 1 und 5 die schwarzen Tasten spielen, um zu verhindern, dass man mit den Fingern 2, 3 und 4 die Klappe trifft.
\textbf{\textit{Deshalb ist die Freiheit, mit einem willkürlichen Maß an Krümmung zu spielen, eine notwendige Freiheit.
Einer der größten Nachteile der gebogenen Haltung ist, dass die Streckmuskeln nicht genügend trainiert werden.
Das führt dazu, dass die Beugemuskeln ihnen kraftmäßig überlegen sind und somit Probleme bei der Kontrolle entstehen.
Bei den flachen Fingerhaltungen sind die ungenutzten Beugemuskeln entspannt; die zugehörigen Sehnen werden gestreckt, was die Finger flexibler macht.
Es gibt zahlreiche Berichte über die außerordentliche Flexibilität der Finger von Franz Liszt.}}

Die falsche Vorstellung, dass die flache Haltung schlecht für die Technik sei, kommt von der Tatsache, dass sie zu \hyperref[c1ii22]{schlechten Angewohnheiten} führen kann, die mit dem falschen Gebrauch der \hyperref[c1ii11]{parallelen Sets} zusammenhängen.
Das geschieht, weil es mit flachen Fingern eine einfache Sache ist, die Finger zu positionieren und sie alle als paralleles Set, das sich als schneller Lauf maskiert, auf das Klavier herunter prasseln zu lassen.
Das kann zu ungleichmäßigem Spiel führen, und Anfänger könnten es als eine Möglichkeit benutzen, schnell zu spielen, ohne Technik zu entwickeln.
Indem man zuerst die gebogene Haltung lernt, und lernt, wie man die parallelen Sets richtig benutzt, kann man dieses Problem vermeiden.
Bei meinen zahlreichen Kontakten mit Lehrern habe ich festgestellt, dass die besten Lehrer mit der Nützlichkeit der flachen Fingerhaltungen vertraut sind.
Das gilt besonders für die Gruppe der Lehrer, deren Unterrichtsmethode auf Liszt zurückgeht, weil Liszt diese Haltung benutzte.
Liszt war Czernys Schüler, folgte aber nicht immer Czernys Lehren und benutzte die flachen Haltungen, um den Klang zu verbessern (Boissier, Fay, Bertrand).
Es ist tatsächlich schwer vorstellbar, dass es fortgeschrittene Klavierspieler gibt, die nicht wissen, wie man die flachen Fingerhaltungen benutzt.
Wenn Sie das nächste Mal ein Konzert besuchen oder ein Video ansehen, schauen Sie einmal zum Beweis nach, ob Sie diese flachen Haltungen erkennen können - Sie werden sehen, dass jeder vollkommene Klavierspieler sie benutzt.
Da traditionell hauptsächlich die gebogene Haltung gelehrt wurde, werden Sie jedoch vielleicht feststellen, dass einige Klavierspieler die gebogene Haltung zu oft benutzen.
Es ist erfreulich, dass der berühmteste Pianist sich\footnote{in dieser Hinsicht} oft dazu entschloss, seinen eigenen Lehrer, Czerny, zu ignorieren.

Wenn man Ihnen die ganze Zeit nur die gebogene Haltung beigebracht hat, mag es zunächst merkwürdig erscheinen, die flachen Fingerhaltungen zu lernen, weil einige wichtige Sehnen sich verkürzt haben.
Einige Lehrer sehen die flachen Fingerhaltungen als eine Art Betrug und als Hinweis auf mangelnde Fertigkeiten mit gebogenen Fingern an, aber das stimmt nicht; die flachen Haltungen sind eine notwendige Fertigkeit.
Beginnen Sie das Üben der flachen Haltungen mit Vorsicht, weil manche Sehnen der Finger vielleicht erst gedehnt werden müssen.
Alle Sehnen müssen von Zeit zu Zeit gedehnt werden, aber die gebogene Haltung gestattet das nicht.

Was ist die Reihenfolge der Wichtigkeit all dieser Haltungen - was ist die flache \enquote{Standardhaltung}, die wir am meisten benutzen sollten?
Die Spinnenhaltung ist die wichtigste.
Das Insektenreich hat diese Haltung nicht ohne guten Grund übernommen; sie haben in hunderten Millionen von Jahren herausgefunden, dass sie am besten funktioniert.
Beachten Sie, dass die Unterscheidung zwischen der Spinnenhaltung und der gebogenen Haltung subtil sein kann, und viele Klavierspieler, die glauben, sie würden die gebogene Haltung benutzen, verwenden in Wahrheit etwas, das den flachen Fingerhaltungen näher kommt.
Die zweitwichtigste Haltung ist die flach ausgestreckte Haltung, weil sie zum Spielen \hyperref[c1iii7e]{weiter Akkorde} und Arpeggios gebraucht wird.
Die dritte Haltung ist die gebogene Haltung, die zum Spielen der weißen Tasten benötigt wird, und die Pyramidenhaltung kommt an vierter Stelle.
Bei der Pyramidenhaltung wird während des Anschlags nur ein Beugemuskel je Finger benutzt, bei der Spinnenhaltung zwei und bei der gebogenen Haltung alle drei sowie die Streckmuskeln.
Die endgütige Wahl der Fingerhaltung ist aber eine persönliche Angelegenheit und muss dem Klavierspieler überlassen bleiben.

Im Allgemeinen kann man folgende Regel anwenden, um zu entscheiden, welche Fingerhaltung man benutzt:
Spielen Sie die schwarzen Tasten mit der völlig flachen Haltung, und benutzen Sie die gebogene und die Pyramidenhaltung für die weißen Tasten.
Die Spinnenhaltung ist vielseitig, wenn Sie sie sich in jungen Jahren angeeignet haben, und man kann mit ihr sowohl schwarze als auch weiße Tasten spielen.
Beachten Sie, dass es im Allgemeinen vorteilhaft ist, zwei Arten von Fingerhaltungen zu benutzen, wenn man innerhalb einer Gruppe von Noten sowohl schwarze als auch weiße Tasten spielen muss.
Das mag zunächst als eine zusätzliche Schwierigkeit erscheinen, aber bei hohen Geschwindigkeiten könnte das die einzige Möglichkeit sein.
Es gibt natürlich zahlreiche Ausnahmen: So benötigen Sie zum Beispiel in Passagen mit dem vierten Finger eventuell mehr flache als gebogene Haltungen, auch wenn die meisten oder alle Tasten weiß sind, um das Heben des vierten Fingers zu vereinfachen.

Obwohl diese Ausführungen über das Spielen mit flachen Fingern umfangreich sind, so sind sie keineswegs vollständig.
In einer detaillierteren Abhandlung müssen wir besprechen, wie man das Spielen mit flachen Fingern auf spezielle Fertigkeiten anwendet, wie zum Beispiel Legato oder das Spielen von zwei Noten mit einem Finger, wobei man jede Note einzeln kontrolliert.
Chopins Legato ist als etwas ganz besonderes dokumentiert, genauso wie sein Staccato.
Ist sein Staccato mit der flachen Fingerhaltung verbunden?
Beachten Sie, dass wir in allen flachen Haltungen einen Vorteil aus dem \enquote{Federeffekt} des entspannten dritten Glieds ziehen können, was beim Staccato-Spielen nützlich sein kann.
Klar müssen wir mehr Nachforschungen anstellen, damit wir lernen, wie man die flachen Fingerhaltungen benutzt.
Es gibt insbesondere eine Kontroverse darüber, ob man hauptsächlich mit der gebogenen Haltung spielen und die flache Fingerhaltung hinzufügen sollte, wann immer es notwendig ist, so wie von den meisten Lehrern gelehrt wurde, oder anders herum, wie Horowitz es getan hat und es hier empfohlen wird.
Die flachen Fingerhaltungen sind auch mit der \hyperref[c1ii3]{Höhe der Sitzbank} verbunden.
Es ist leichter, mit flachen Fingern zu spielen, wenn die Bank niedriger ist.
Es gibt zahlreiche Berichte von Pianisten, die entdeckt haben, dass sie bei einer niedrigeren Bankposition viel besser spielen können (Horowitz und Glenn Gould sind Beispiele).
Sie behaupten, dass sie eine bessere Kontrolle erhalten, insbesondere für das Pianissimo und die Geschwindigkeit, aber niemand hat bisher eine Erklärung dafür gegeben, warum das so ist.
Meine Erklärung ist, dass die niedrigere Höhe der Bank es ihnen erlaubte, mehr flache Fingerhaltungen zu benutzen.
Es scheint aber keinen guten Grund dafür zu geben, übermäßig tief zu sitzen, wie Glenn Gould es getan hat, weil man immer das Handgelenk senken kann, um denselben Effekt zu erzielen.

Zusammengefasst hatte Horowitz gute Gründe, mit flachen Fingern zu spielen, und die obige Diskussion legt nahe, dass ein Teil seiner hohen technischen Stufe daraus resultierte, dass er die flachen Fingerhaltungen mehr als die anderen benutzte.
\textbf{Die wichtigste Botschaft dieses Abschnitts ist, dass wir lernen müssen, das dritte Glied des Fingers zu entspannen und mit dem berührungsempfindlichen Teil der Fingerspitze spielen sowie die Beweglichkeit der Finger trainieren müssen.}
Die Aversion gegen oder sogar das Verbot des Spielens mit flachen Fingern einiger Lehrer stellt sich als Fehler heraus; in Wahrheit wird jedes Einkrümmen der Finger zu einem gewissen Maß an Krümmungslähmung führen.
Anfänger müssen jedoch die gebogene Haltung als erste lernen, weil sie oft benötigt wird und schwieriger als die flachen Haltungen ist.
Wenn Schüler die leichtere Methode mit flachen Fingern als erstes lernen, werden sie die gebogene Haltung später eventuell niemals ausreichend gut lernen.
Das Spielen mit flachen Fingern ist für die Geschwindigkeit nützlich, das Vergrößern der Reichweite, Spielen mehrerer Noten mit einem Finger, Vermeiden von Verletzungen, \enquote{Fühlen der Tasten}, Legato, Entspannung, pianissimo oder fortissimo spielen und das Hinzufügen von Farbe.
Obwohl die gebogene Haltung notwendig ist, ist die Behauptung \enquote{Man braucht die gebogene Haltung, um technisch schwieriges Material zu spielen.} irreführend - Sie benötigen dafür bewegliche Finger.
Mit flachen Fingern zu spielen, gibt uns die Freiheit, viele notwendige und vielseitige Haltungen zu benutzen.
Wir wissen nun, wie man all diese schwarzen Tasten spielen kann und keine einzige Note verfehlt.
Ich danke Ihnen, Johann Sebastian Bach, Frederic Chopin, Vladimir Horowitz und Yvonne Combe.



<!-- c1iii4c.html -->

\subsubsection{Bewegungen des Körpers}
\label{c1iii4c}

Viele Lehrer unterstützen \enquote{den Gebrauch des ganzen Körpers für das Klavierspielen} (siehe \hyperref[Whiteside]{Whiteside}).
Was bedeutet das?
Gibt es besondere Körperbewegungen, die für die Technik erforderlich sind?
Nicht wirklich; die Technik liegt in den Händen und in der \hyperref[c1ii14]{Entspannung}.
Da jedoch die Hände mit dem Körper verbunden sind und durch ihn unterstützt werden, kann man nicht einfach in einer Position sitzen und hoffen zu spielen.
Wenn man in den höheren Lagen spielt, sollte der Körper den Händen folgen, und Sie könnten sogar ein Bein in die Gegenrichtung strecken, um den Körper auszubalancieren, wenn es nicht für die Pedale gebraucht wird.
Auch erfordert selbst die kleinste Bewegung eines Fingers die Aktivierung einer Reihe von Muskeln - mindestens bis zur Körpermitte (nahe des Brustbeins) hin, wenn nicht sogar bis zu den Beinen und anderen Körperteilen, die den Körper unterstützen.
Die Entspannung ist, wegen der schieren Größe der einbezogenen Muskeln, im Körper genauso wichtig wie in den Händen und den Fingern.
Obwohl die meisten der erforderlichen Körperbewegungen einfach mit dem gesunden Menschenverstand zu verstehen sind und nicht derart wichtig erscheinen, sind die Körperbewegungen nichtsdestoweniger für das Klavierspielen absolut notwendig.
Lassen Sie uns diese Bewegungen besprechen, von denen einige nicht völlig offensichtlich sind.

Der wichtigste Aspekt ist die Entspannung.
Es ist die gleiche Art Entspannung, die Sie in den Händen und den Armen brauchen - benutzen Sie nur die Muskeln, die für das Spielen erforderlich sind, und nur für die kurzen Momente, während denen sie gebraucht werden.
Entspannung bedeutet auch freies Atmen; wenn Ihre Kehle nach hartem Üben trocken ist, dann schlucken Sie nicht richtig, ein sicheres Zeichen von Anspannung.
\textbf{Entspannung ist eng mit der Unabhängigkeit eines jeden Teils des Körpers verbunden.
Als erstes müssen Sie, bevor Sie über nützliche Körperbewegungen nachdenken, sicherstellen, dass die Hände und Finger völlig vom Körper entkoppelt sind.
Wenn Sie nicht entkoppelt sind, dann wird der \hyperref[c1iii1b]{Rhythmus} unordentlich, und man kann alle Arten von unerwarteten Fehlern machen.
Wenn man außerdem nicht merkt, dass der Körper und die Hände gekoppelt sind, dann wird man sich fragen, warum man so viele merkwürdige Fehler macht, für die man keinen Grund findet.}
Dieses Entkoppeln ist beim \hyperref[c1ii25]{beidhändigen Spielen} besonders wichtig, weil das Koppeln die Unabhängigkeit der beiden Hände stört.
Koppeln ist eine der Ursachen der Fehler: Eine Bewegung in einer Hand erzeugt zum Beispiel durch den Körper eine unfreiwillige Bewegung in der anderen.
Das bedeutet nicht, dass man das Entkoppeln des Körpers während des \hyperref[c1ii7]{einhändigen Übens} ignorieren kann; im Gegenteil, das Entkoppeln sollte während der einhändigen Arbeit bewusst geübt werden.
Beachten Sie, dass das Entkoppeln ein einfaches Konzept und leicht auszuführen ist, wenn man es erst einmal gelernt hat, dass es aber körperlich ein komplexer Prozess ist.
Jede Bewegung in einer Hand erzeugt \textit{notwendigerweise} eine gleiche und entgegengesetzte Reaktion im Körper, die automatisch zur anderen Hand übertragen wird.
Deshalb erfordert das Entkoppeln einen aktiven Aufwand; es ist keine bloße passive Entspannung.
Glücklicherweise sind unsere Gehirne genügend entwickelt, sodass wir leicht das Konzept des Entkoppelns begreifen können.
Darum muss das Entkoppeln aktiv geübt werden.
Wenn Sie irgendeine neue Komposition lernen, wird immer ein wenig Kopplung vorhanden sein, bis Sie diese heraustrainieren.
Die schlimmste Art der Kopplung ist jene, die man während des Übens erwirbt, wenn man mit Stress übt oder versucht, etwas zu spielen, das zu schwierig ist.
Während der intensiven Bemühungen, die notwendig sind, um schwieriges Material zu spielen, kann ein Schüler jede Zahl von unnötigen Bewegungen verinnerlichen - besonders während des beidhändigen Übens -, was schließlich das Spielen stört, wenn die Geschwindigkeit gesteigert wird.
Indem Sie einhändig auf Geschwindigkeit kommen, können Sie die meisten dieser aus dem beidhändigen Koppeln resultierenden Fehler vermeiden.

Der Körper wird wie oben beschrieben benutzt, um mittels der Schultern fortissimo zu spielen.
Er wird auch zum leisen Spielen benutzt, denn um leise zu spielen, braucht man eine stabile, konstante Plattform, von der aus man diese kleinen, kontrollierten Kräfte erzeugen kann.
Die Hand und der Arm haben selbst zu viele mögliche Bewegungen, um als stabile Plattform zu dienen.
Wenn sie sicher mit einem stabilen Körper verbunden sind, hat man eine viel stabilere Bezugsplattform.
Deshalb sollte die Ruhe des Pianissimo vom Körper ausgehen, nicht von den Fingerspitzen.
Und um mechanischen \enquote{Lärm} aufgrund von zusätzlichen Fingerbewegungen zu reduzieren, sollten die Finger soviel wie möglich auf den Tasten sein.
Tatsächlich bietet das Erfühlen der Tasten einen weiteren stabilen Bezugspunkt, von dem aus man spielen kann.
Wenn der Finger die Taste verlässt, hat man diesen wertvollen Bezugspunkt verloren, und der Finger kann nun überallhin wandern, was es schwierig macht, die nächste Note genau zu kontrollieren.


% zuletzt geändert 21.08.2011


<!-- c1iii5.html -->

\subsection{Schnell spielen: Tonleitern, Arpeggios und chromatische Tonleitern (Chopins Fantaisie Impromptu und Beethovens Mondschein-Sonate, 3. Satz)}
\label{c1iii5}

\subsubsection{Tonleitern: Daumenuntersatz, Daumenübersatz}
\label{c1iii5a}

\textbf{Tonleitern und Arpeggios sind die grundlegendsten Klavierpassagen; trotzdem wird die wichtigste Methode sie zu spielen oft überhaupt nicht gelehrt!}
Arpeggios sind einfach erweiterte Tonleitern und können deshalb ähnlich wie Tonleitern behandelt werden; darum werden wir zunächst die Tonleitern besprechen und dann beschreiben, wie ähnliche Regeln auf Arpeggios angewendet werden können.
Es gibt einen fundamentalen Unterschied darin, wie man ein Arpeggio im Vergleich zu einer Tonleiter spielen muss (mit einem flexiblen Handgelenk); wenn man diesen Unterschied erst einmal gelernt hat, werden Arpeggios viel einfacher, sogar für kleine Hände.

\textbf{Es gibt zwei Arten, eine Tonleiter zu spielen: Die erste ist der wohlbekannte Daumenuntersatz und die zweite der Daumenübersatz.}
Beim Untersatz wird der Daumen unter die Hand gebracht, damit der dritte oder vierte Finger zum Spielen der Tonleiter vorbeigehen kann.
Dieser Vorgang wird durch zwei Eigenschaften des Daumens erleichtert: Er ist kürzer als die anderen Finger und befindet sich unter der Handfläche.
\textbf{Beim Übersatz wird der Daumen wie die anderen vier Finger behandelt, was die Bewegung in hohem Maß vereinfacht.
Beide Methoden sind erforderlich, um eine Tonleiter zu spielen, aber jede wird unter verschiedenen Umständen benötigt}; der Übersatz wird für schnelle, technisch schwierige Passagen benötigt, und der Untersatz ist nützlich für langsame Legatopassagen oder wenn einige Noten gehalten werden müssen, während andere gespielt werden.

Den Begriff Daumenübersatz habe ich gewählt, weil es keine bessere Bezeichnung für die Methode gibt.
Es ist offensichtlich eine unzutreffende Bezeichnung und erschwert Anfängern eventuell, zu verstehen, wie man ihn spielt.
Ich habe es mit anderen Namen versucht, aber keiner davon ist besser als Daumenübersatz.
Der einzige mögliche Vorteil ist, dass diese offensichtliche Fehlbezeichnung die Aufmerksamkeit auf die Existenz des Daumenübersatzes lenkt.

Vielen Klavierlehrern war der Daumenübersatz überhaupt nicht bekannt.
Das war kaum ein Problem, solange die Schüler nicht auf fortgeschrittene Stufen kamen.
Tatsächlich ist es mit genügend Anstrengung und Arbeit möglich, ziemlich schwierige Passagen unter Verwendung des Daumenuntersatzes zu spielen, und \textbf{es gibt vollendete Pianisten, die glauben, dass der Untersatz die einzige Methode ist, die sie benötigen.
In Wahrheit haben sie unbewusst (durch sehr hartes Arbeiten) gelernt, den Untersatz für ausreichend schnelle Passagen so zu verändern, dass er  dem Übersatz nahe kommt.}
Diese Änderung ist notwendig, weil es für solch schnelle Tonleitern körperlich unmöglich ist, sie mit dem Untersatz zu spielen.
Deshalb ist es für die Schüler wichtig, mit dem Lernen des Übersatzes anzufangen, sobald sie über das Anfängerstadium hinaus sind, bevor die Angewohnheit verfestigt ist, den Untersatz für Passagen zu benutzen, die mit Übersatz gespielt werden sollten.

\textbf{Viele Schüler benutzen die Methode, am Anfang langsam zu spielen und die Geschwindigkeit dann zu steigern.
Sie kommen bei niedriger Geschwindigkeit gut mit dem Daumenuntersatz zurecht, gewöhnen sich infolgedessen den Untersatz an, und stellen, wenn sie zur vorgegebenen Geschwindigkeit kommen, fest, dass sie zum Daumenübersatz wechseln müssen.}
Dieser Wechsel kann eine sehr schwierige, frustrierende und zeitraubende Aufgabe sein, nicht nur für Tonleitern sondern auch für jeden schnellen Lauf - ein weiterer Grund, warum die Methode der schrittweisen Steigerung der Geschwindigkeit in diesem Buch nicht empfohlen wird.
\textbf{Die Daumenuntersatzbewegung ist eine der am meisten verbreiteten Ursachen von Geschwindigkeitsbarrieren und Spielfehlern.
Wenn der Daumenübersatz erst einmal gelernt ist, sollte er deshalb immer benutzt werden, um Läufe zu spielen, außer wenn der Untersatz bessere Ergebnisse liefert.}

Die wichtigsten Muskeln des Daumens für das Klavierspielen sind, so wie die der anderen vier Finger, im Unterarm.
Der Daumen besitzt jedoch weitere Muskeln in der Hand, die benutzt werden, um den Daumen beim Untersatz seitwärts zu bewegen.
Das Einbeziehen dieser zusätzlichen Muskeln für die Untersatzbewegung macht diese zu einer komplexeren Operation und verlangsamt somit die maximal zu erreichende Geschwindigkeit.
Die zusätzliche Komplikation verursacht ebenfalls Fehler.
\textbf{Lehrer, die den Daumenübersatz lehren, behaupten über diejenigen, die ausschließlich den Daumenuntersatz benutzen, dass 90\% ihrer Fehler ihren Ursprung in der Untersatzbewegung haben.}

Man kann den Nachteil des Untersatzes demonstrieren, indem man den Verlust der Beweglichkeit des Daumens in seiner eingeschlagenen Position beobachtet.
Strecken Sie zunächst Ihre Finger so aus, dass alle Finger in derselben Ebene liegen.
Sie werden feststellen, dass alle Finger, einschließlich des Daumens, nach oben und unten beweglich sind (die Bewegung, die man zum Klavierspielen braucht).
Wackeln Sie nun mit dem Daumen schnell auf und ab - Sie werden sehen, dass sich der Daumen vertikal leicht 3 oder 4 cm und ziemlich schnell bewegen kann (ohne den Unterarm zu drehen).
Ziehen Sie dann den Daumen, während Sie mit derselben schnellen Frequenz weiterwackeln, schrittweise unter die Hand - Sie werden sehen, dass der Daumen, während er unter die Hand geht, seine vertikale Beweglichkeit verliert, bis er unbeweglich, fast gelähmt wird, wenn er unter dem Mittelfinger ist.

Hören Sie nun mit dem Wackeln auf, und stoßen Sie den Daumen nach unten (ohne das Handgelenk zu bewegen) - er bewegt sich nach unten!
Das kommt daher, dass Sie nun eine andere Muskelgruppe benutzen.
Versuchen Sie dann, unter Benutzung dieser neuen Muskeln, den Daumen so schnell Sie können auf und ab zu bewegen - Sie sollten finden, dass diese neuen Muskeln viel schwerfälliger sind und die Auf- und Abwärtsbewegung langsamer ist als die Wackelrate des Daumens, als er ausgestreckt war.
Damit Sie in der Lage sind, den Daumen in seiner eingeschlagenen Position zu bewegen, müssen Sie deshalb nicht nur diese neue Muskelgruppe benutzen, sondern diese Muskeln sind zusätzlich langsamer.
Es ist die Einführung dieser schwerfälligen Muskeln, die beim Daumenuntersatz Fehler verursacht und das Spielen verlangsamt.
Der Daumenübersatz eliminiert diese Probleme.

Tonleitern und Arpeggios gehören zu den in der Klavierpädagogik am meisten missbrauchten Übungen - Anfängern wird nur der Untersatz gelehrt, wodurch sie unfähig bleiben, sich die richtigen Techniken für Läufe und Arpeggios anzueignen.
Nicht nur das, sondern wenn die Tonleiter beschleunigt wird, beginnt mysteriöserweise der Stress sich aufzubauen.
Schlimmer noch: Der Schüler baut ein großes Repertoire mit falschen Angewohnheiten auf, die mühevoll korrigiert werden müssen.
\textbf{Der Übersatz ist leichter zu lernen als der Untersatz, weil er nicht die Seitwärtsdrehungen von Daumen, Hand, Arm und Ellbogen erfordert.}
Der Daumenübersatz sollte gelehrt werden, sobald schnellere Tonleitern benötigt werden - innerhalb der ersten beiden Unterrichtsjahre.
Anfängern sollte der Untersatz zuerst gelehrt werden, weil er für langsame Passagen notwendig ist und es länger dauert, ihn zu lernen.
Talentierten Schülern muss der Übersatz innerhalb der ersten Monate ihres Unterrichts gelehrt werden oder sobald sie den Untersatz beherrschen.

Da es zwei Möglichkeiten gibt, die Tonleitern zu spielen, gibt es zwei Schulen hinsichtlich der Art, zu lehren wie man sie spielt.
Die Daumenuntersatzschule (Czerny, Leschetizky) behauptet, dass der Untersatz die einzige Art ist, wie man Legato-Tonleitern spielen kann, und dass man mit genügend Übung Tonleitern bei jeder Geschwindigkeit mit Daumenuntersatz spielen kann.
Die Daumenübersatzschule (\hyperref[Whiteside]{Whiteside}, \hyperref[Sandor]{Sandor}) hat nach und nach die Oberhand gewonnen, und die hartnäckigeren Anhänger \textit{verbieten} die Benutzung des Untersatzes unter allen Umständen.
Sehen Sie dazu im \hyperref[reference]{Quellenverzeichnis} weitere Diskussionen über das Lehren von Daumenuntersatz gegenüber Daumenübersatz.
Beide extremen Schulen liegen falsch, weil man beide Fertigkeiten benötigt.

Die Daumenübersatzlehrer sind verständlicherweise über die Tatsache verärgert, dass fortgeschrittene Schüler, die von privaten Lehrern an sie weitergereicht werden, oft den Daumenübersatz nicht kennen und es sechs Monate oder länger dauert, allein die Stunden von Repertoire zu korrigieren, die sie auf die falsche Art gelernt haben.
Ein Nachteil davon, sowohl den Untersatz als auch den Übersatz zu lernen, ist, dass man beim \hyperref[c1iii11]{Spielen vom Blatt} mit dem Daumen durcheinanderkommen kann und nicht weiß, welchen Weg man nehmen soll.
Diese Verwirrung ist ein Grund, warum manche Lehrer der Übersatzschule tatsächlich den Gebrauch des Untersatzes verbieten.
Ich empfehle, dass Sie als Standard den Daumenübersatz benutzen und den Daumenuntersatz als Ausnahme der Regel.
Beachten Sie, dass Chopin beide Methoden lehrte (\hyperref[Eigeldinger]{Eigeldinger}, Seite 37).

Obwohl der Daumenübersatz durch Whiteside und andere wiederentdeckt wurde, geht der früheste Bericht über seinen Gebrauch mindestens auf Franz Liszt zurück (Fay).
Liszt ist dafür bekannt, dass er im Alter von ungefähr 20 Jahren für mehr als ein Jahr nicht mehr auftrat und seine Technik weiterentwickelte.
Er war mit seiner Technik unzufrieden (insbesondere wenn er Tonleitern spielte), wenn er sie mit den wunderbaren Darbietungen Paganinis auf der Geige verglich, und experimentierte mit dem Verbessern seiner Technik.
Am Ende dieser Periode war er mit seinen neuen Fertigkeiten zufrieden, konnte aber anderen nicht genau erklären, was er getan hatte, um sich zu verbessern - er konnte es nur am Klavier vorführen (das galt für die meisten von Liszts \enquote{Lehren}).
Amy Fay erkannte jedoch, dass er die Tonleitern jetzt anders spielte; anstatt den Daumenuntersatz zu benutzen, \enquote{rollte er die Hand über den passierten Finger}, sodass der Daumen auf die nächste Taste fiel.
Es dauerte offensichtlich mehrere Monate, bis Fay diese Methode imitieren konnte, aber ihrer Aussage nach \enquote{änderte es ihre Art zu spielen völlig}, und sie behauptete, dass es allgemein zu einer deutlichen Verbesserung ihrer Technik führte, nicht nur beim Spielen von Tonleitern, weil der Daumenübersatz bei jedem Lauf und auch bei Arpeggios anwendbar ist.


\subsubsection{Daumenübersatz: Bewegung, Erklärung und Video}
\label{c1iii5b}

Lassen Sie uns damit anfangen, dass wir den grundlegenden Fingersatz für Tonleitern analysieren.
Nehmen Sie die C-Dur-Tonleiter der rechten Hand.
Wir beginnen mit dem leichtesten Teil, das heißt mit der absteigenden Tonleiter der rechten Hand, die 5432132,1432132,1 usw. gespielt wird.
Da der Daumen (1) unter der Hand ist, rollen der Mittelfinger (3) oder der Ringfinger (4) leicht über den Daumen, faltet sich der Daumen ganz natürlich unter diese Finger, und dieser Fingersatz für die absteigende Tonleiter funktioniert bestens.
Diese Bewegung ist im Grunde die Daumenuntersatzbewegung; die Bewegung für den absteigenden Übersatz ist ähnlich, wir werden jedoch eine leichte aber entscheidende Änderung vornehmen müssen, damit es zu einem wahren Übersatz wird; diese Änderung ist allerdings subtil und wird später besprochen.

Nehmen Sie nun die aufsteigende C-Dur-Tonleiter der rechten Hand.
Diese wird 1231234 usw. gespielt.
\textbf{Beim Daumenübersatz wird der Daumen genau so wie die Finger 3 und 4 bei der absteigenden Tonleiter gespielt; das heißt er wird einfach gehoben und gesenkt, ohne die Seitwärtsbewegung unter die Handfläche wie beim Daumenuntersatz.}
Da der Daumen kürzer als die anderen Finger ist, kann er fast parallel zu (und direkt hinter) dem passierten Finger herunterbewegt werden, ohne mit diesem zu kollidieren.
Um mit dem Daumen die richtige Taste zu treffen, werden Sie die Hand bewegen und mit dem Handgelenk ganz leicht \enquote{zucken} müssen.
Bei Tonleitern wie C-Dur sind sowohl der Daumen als auch der passierte Finger auf weißen Tasten und sind sich zwangsläufig ein wenig im Weg.
\textbf{Um jede Möglichkeit einer Kollision zu vermeiden, sollte der Arm fast 45 Grad zur Tastatur stehen (wobei er nach links zeigt), und die Hand rollt über den passierten Finger, indem sie diesen als Drehpunkt benutzt.
Der Finger 3 oder 4 muss dann schnell wegbewegt werden, wenn der Daumen herunterkommt.}
Beim Übersatz ist es, anders als beim Untersatz, nicht möglich, den Finger 3 oder 4 unten zu halten bis der Daumen spielt.
Wenn man den Übersatz das erste Mal ausprobiert, wird die Tonleiter ungleichmäßig sein, und es mag eine \enquote{Lücke} geben, wenn man mit dem Daumen spielt.
Deshalb muss der Übergang sehr schnell sein, auch wenn die Tonleiter langsam gespielt wird.
Wenn Sie Fortschritte machen, werden Sie merken, dass eine schnelle Rollung/Drehung des Handgelenks/Arms hilfreich ist.\footnote{Eine Anleitung für das Erlernen des Daumenübersatzes finden sie (in Englisch) unter anderem auch auf \enquote{http://www.pianostreet.com/smf/index.php/topic,7226.msg72166.html\#msg72166}.}
Anfänger finden den Übersatz üblicherweise leichter als den Untersatz, aber diejenigen, die den Untersatz viele Jahre gelernt haben, werden den Übersatz zunächst schwierig und ungleichmäßig finden.
Drehen Sie auch den Unterarm ein wenig im Uhrzeigersinn (Chopin nannte es die \enquote{Glissando-Position}, siehe unten in \hyperref[c1iii5c]{5c}), was den Daumen automatisch vorwärts bringt.
Die aufsteigende Tonleiter der rechten Hand ist schwieriger als die absteigende.
Bei der absteigenden Tonleiter dreht man mit dem Daumen und rollt über ihn, was einfach ist.
Bei der aufsteigenden Tonleiter rollt man aber über Finger 3 oder 4, und es gibt Finger oberhalb dieses Fingers, die beim Rollen stören können.

Die Logik hinter dem Daumenübersatz ist die folgende.
\textbf{Der Daumen wird wie jeder andere Finger benutzt.}
Der Daumen bewegt sich nur auf und ab.
Das vereinfacht die Fingerbewegungen, und außerdem müssen die Hände, Arme und Ellbogen nicht zum Anpassen an die Untersatzbewegungen verdreht werden.
So bleiben die Hand und der Arm immer im optimalen Winkel zur Tastatur und gleiten einfach mit der Tonleiter auf und ab.
Ohne diese Vereinfachung können technisch schwierige Passagen unmöglich werden, besonders weil man immer noch neue Handbewegungen hinzufügen muss, um solche Geschwindigkeiten zu erreichen, und die meisten dieser Bewegungen sind mit dem Daumenuntersatz nicht kompatibel.
Am wichtigsten ist, dass \textbf{die Bewegung des Daumens in seine korrekte Position hauptsächlich von der Hand gesteuert wird}, während es beim Untersatz die kombinierte Bewegung des Daumens und der Hand ist, die die Position des Daumens bestimmt.
Weil die Handbewegung weich ist, wird der Daumen genauer positioniert als beim Untersatz, was fehlende und falsch angeschlagene Noten reduziert und dem Daumen gleichzeitig eine bessere Klangkontrolle verleiht.
Auch wird die aufsteigende Tonleiter der absteigenden ähnlich, weil man die Finger zum Vorbeigehen immer \textit{über}setzt.
Das macht es auch einfacher, \hyperref[c1ii25]{beidhändig} zu spielen, da alle Finger beider Hände immer übersetzen.
Ein weiterer Pluspunkt ist, dass der Daumen nun eine schwarze Taste spielen kann.
Es sind diese vielen Vereinfachungen, das Eliminieren des aus dem \enquote{gelähmten} Daumen resultierenden Stresses und noch mehr Vorteile, die weiter unten besprochen werden, welche die Möglichkeiten für Fehler reduzieren und ein schnelleres Spielen ermöglichen.
Es gibt natürlich Ausnahmen: Langsame Legatopassagen oder einige Tonleitern, die schwarze Tasten enthalten, usw. werden mit einer untersatzähnlichen Bewegung leichter ausgeführt.
\textbf{Die meisten Schüler, die nur den Daumenuntersatz benutzen, haben es am Anfang furchtbar schwer, wenn sie versuchen zu verstehen, wie jemand mit Daumenübersatz spielen kann}.
Das ist der deutlichste Hinweis auf den Schaden, der dadurch angerichtet werden kann, dass man den Übersatz nicht so früh wie möglich lernt; für diese Schüler ist der Daumen nicht \enquote{frei}.
Wir werden sehen, dass der freie Daumen ein vielseitiger Finger ist.
Aber verzweifeln Sie nicht, denn es stellt sich heraus, dass die meisten fortgeschrittenen Untersatzschüler bereits wissen, wie man den Übersatz spielt - sie wissen es nur nicht.

Bei der linken Hand ist es umgekehrt wie bei der rechten; der Übersatz wird für die absteigende Tonleiter benutzt, und die aufsteigende Tonleiter ist dem Untersatz ziemlich ähnlich.
Wenn Ihre rechte Hand weiter fortgeschritten ist als die linke, führen Sie die Ausflüge zu höheren Geschwindigkeiten mit der rechten Hand durch, bis Sie sich genau entscheiden, was sie tun.
Nehmen Sie diese Bewegung dann für die linke Hand.

Da Schüler ohne Lehrer Schwierigkeiten haben, sich den Daumenübersatz vorzustellen, untersuchen wir einen Videoclip, der den Übersatz mit dem Untersatz vergleicht.
Ich habe dieses Video in zwei Formaten hochgeladen, da nicht jede Software alle Formate abspielen kann.
Falls Sie diesen Text nur als Ausdruck vorliegen haben, müssen Sie die URLs von Hand eingeben.
Öffnen Sie dann zuerst Ihr Videoprogramm, und suchen Sie anschließend nach dem Menüpunkt, mit dem Sie einen Link manuell eingeben können - meistens unter \enquote{Datei}.
Nachfolgend finden Sie zwei URLs; eine davon sollte funktionieren.
\textit{[Es gibt keine spezifische Notation in HTML, mit der Dateien beim Anklicken automatisch heruntergeladen oder direkt in der auf dem PC zugewiesenen Anwendung gestartet werden.
(Fast) jedes Browser-Programm bietet aber anhand des Dateityps die eine oder andere Möglichkeit an.
Benutzen Sie ggf. zum Herunterladen die entsprechenden Funktionen Ihres Browsers, zum Beispiel einen \enquote{Rechtsklick} auf den Link machen und dann \enquote{Speichern unter} auswählen.
Falls Ihr Video-Programm die Möglichkeit bietet, schauen Sie sich das Video auch in Zeitlupe oder sogar bildweise an.  Die Unterschiede in den Bewegungen werden dann deutlicher.]}

\begin{itemize} 
 \item \hyperref[http://www.pianopractice.org/TOscale.mp4]{http://www.pianopractice.org/TOscale.mp4} (extern; ca. 3,1 MB)
 \item \hyperref[http://www.pianopractice.org/TOscale.wmv]{http://www.pianopractice.org/TOscale.wmv} (extern; ca. 1,6 MB)
 \end{itemize}
Das Video zeigt die rechte Hand beim Spielen von zwei Oktaven mit Daumenübersatz - zweimal auf- und abwärts.
Das wird dann mit dem Daumenuntersatz wiederholt.
Für diejenigen, die keine Klavierspieler sind, mag das im Grunde dasselbe sein, obwohl die Untersatzbewegung leicht übertrieben ist.
Das zeigt, warum Videos von Bewegungen beim Klavierspielen nicht so hilfreich sind, wie man denken könnte.
Die aufsteigenden Übersatzbewegungen sind im Grunde korrekt.
Die absteigende Übersatzbewegung hat einen Fehler - das Nagelglied des Daumens wird leicht gebeugt.
Bei diesen langsamen Geschwindigkeiten beeinflusst dieses leichte Beugen nicht das Spielen, aber beim strengen Übersatz sollte der Daumen sowohl beim aufsteigenden als auch beim absteigenden Spielen gerade bleiben.
Dieses Beispiel zeigt, wie wichtig es ist, den Übersatz so früh wie möglich zu lernen.
Meine Neigung, das Nagelglied zu beugen, ist das Ergebnis davon, dass ich über viele Jahrzehnte hinweg nur den Untersatz benutzt hatte, bevor ich den Übersatz lernte.
Eine wichtige Schlussfolgerung ist hier, \textbf{den Daumen beim Übersatz die ganze Zeit gerade zu halten}.



<!-- c1iii5a2.html -->

\label{c1iii5a2}
\subsubsection{Daumenübersatz üben: Geschwindigkeit, Glissandobewegung}
\label{c1iii5c}

\textbf{Wir besprechen nun Verfahren für das Üben schneller Tonleitern mit Daumenübersatz.}
Die aufsteigende C-Dur-Tonleiter der rechten Hand besteht aus den \hyperref[c1iv2]{parallelen Sets} 123 und 1234.
Benutzen Sie zunächst die \hyperref[c1iii7b]{Übungen für parallele Sets} (Abschnitt III.7b), um ein schnelles 123 zu erreichen, wobei die 1 auf dem \hyperref[Noten]{C4} ist.
Üben Sie dann 1231, wobei der Daumen hochgeht und dann hinter der 3 herunterkommt.
Bewegen Sie dabei die 3 schnell aus dem Weg, wenn der Daumen herunterkommt.
Der größte Teil der Seitwärtsbewegung des Daumens wird von der Bewegung der Hand beigetragen.
Die letzte 1 in 1231 ist die Verbindung, die aufgrund der \hyperref[c1ii8]{Kontinuitätsregel} (siehe Abschnitt II.8) erforderlich ist.
Wiederholen Sie es mit 1234, wobei die 1 auf F4 ist, und dann 12341, wobei die letzte 1 über die 4 rollt und auf C5 landet.
Spielen Sie mit den Fingern 2 bis 4 nahe an den schwarzen Tasten, um dem Daumen mehr Platz zum Landen zu geben.
Drehen Sie den Unterarm und das Handgelenk, sodass die Spitzen der Finger 2 bis 5 eine gerade Linie parallel zur Tastatur bilden; beim Spielen des mittleren C sollte der Unterarm dadurch einen Winkel von ungefähr 45 Grad zur Tastatur bilden.
Verbinden Sie dann die beiden parallelen Sets, um die Oktave zu vervollständigen.
Nachdem Sie eine Oktave können, spielen Sie zwei, usw.

\textbf{Wenn man schnelle Tonleitern spielt, sind die Hand- und Armbewegungen denen eines Glissandos ähnlich.}
Die glissandoartige Bewegung gestattet es, den Daumen sogar noch näher an die übergangenen Finger heranzubringen, weil die Finger 2 bis 5 leicht rückwärts zeigen.
Sie sollten auf diese Weise nach einigen Minuten Übung in der Lage sein (Machen Sie sich jetzt noch keine Gedanken über die Gleichmäßigkeit!), eine schnelle Oktave zu spielen (ungefähr 1 Oktave/Sekunde).
Üben Sie bis zu dem Punkt zu \hyperref[c1ii14]{entspannen}, an dem Sie das Gewicht Ihres Arms spüren können.
Wenn Sie den Daumenübersatz beherrschen, sollten Sie finden, dass lange Tonleitern nicht schwieriger sind als kurze, und dass \hyperref[c1ii25]{beidhändiges Spielen} mit dem Übersatz nicht so schwierig ist wie mit dem Untersatz.
Das geschieht, weil die Drehungen des Ellbogens usw. beim Untersatz schwierig werden, insbesondere am oberen und unteren Ende der Tonleitern (es gibt viele weitere Gründe).
An dieser Stelle muss betont werden, \textbf{dass es nicht notwendig ist, Tonleitern beidhändig zu üben, und dass sie beidhändig zu üben - bis man es ziemlich gut beherrscht - mehr Schaden anrichtet als es nützt}.
Es gibt soviel wichtiges Material, das \hyperref[c1ii7]{einhändig} geübt werden muss, so dass man durch das beidhändige Üben - außer bei kurzen Experimenten - wenig erreicht.
Die meisten fortgeschrittenen Lehrer (\hyperref[Gieseking]{Gieseking}) halten das beidhändige Üben von schnellen Tonleitern für eine Zeitverschwendung.

\textbf{Um die \hyperref[c1iv2a]{Phasenwinkel} (zeitliche Folge der einzelnen Finger) innerhalb des parallelen Sets exakt zu kontrollieren, heben Sie Ihr Handgelenk (ein ganz klein wenig), während Sie die parallelen Sets 123 oder 1234 spielen.
Machen Sie dann den Übergang zum nächsten parallelen Set, indem Sie das Handgelenk senken, um den Übersatz zu spielen.
Diese Bewegungen des Handgelenks sind extrem klein, für das untrainierte Auge fast nicht wahrnehmbar, und sie werden sogar kleiner, wenn Sie schneller werden.}
Sie können das gleiche auch erreichen, wenn Sie das Handgelenk im Uhrzeigersinn drehen, um die parallelen Sets zu spielen, und gegen den Uhrzeigersinn drehen, um den Daumen beim Zurückspielen zu senken.
Die Auf- und Abwärtsbewegung des Handgelenks ist jedoch gegenüber der Drehung zu bevorzugen, weil sie einfacher ist, und die Drehung kann für andere Zwecke reserviert werden (\hyperref[Sandor]{Sandor}).
Wenn Sie nun versuchen, mehrere Oktaven zu spielen, kann es zunächst wie ein Waschbrett klingen.

Der schnellste Weg, das Spielen von Tonleitern zu beschleunigen, ist, nur eine Oktave zu üben.
Wenn Sie die höheren Geschwindigkeiten erreicht haben, \hyperref[c1iii2]{zirkulieren} Sie zwei Oktaven auf und ab.
Bei hohen Geschwindigkeiten sind diese kürzeren Oktaven nützlicher, weil es schwieriger ist, die Richtung am Anfang und Ende umzukehren, wenn die Geschwindigkeit erhöht wird.
Die kurzen Oktaven geben Ihnen die Gelegenheit, die Richtungsänderungen öfter zu üben.
Bei längeren Läufen kommt man nicht so oft dazu, die Enden zu üben, und die zusätzliche Streckung des Arms, um die höheren und die tieferen Oktaven zu erreichen, ist eine unnötige Ablenkung von der Konzentration auf den Daumen.
\textbf{Der Weg zum Spielen schneller Umkehrungen am Anfang und am Ende ist, sie mit einem einzigen Abwärtsdrücken der Hand zu spielen.}
Um zum Beispiel am oberen Ende umzukehren, spielen Sie das letzte aufsteigende parallele Set, die Verbindung und das erste abwärts laufende parallele Set alle in einer Abwärtsbewegung.
In diesem Schema wird die Verbindung dadurch eliminiert, dass sie in eines der parallelen Sets eingebaut wird.
Das ist eine der effektivsten Arten, eine schnelle Verbindung zu spielen - indem man sie verschwinden lässt!

\textbf{Drehen Sie beim Glissando die Hände so ein- oder auswärts (\hyperref[c1iii4ProSup]{Pronation oder Supination}), dass die Finger von der Bewegungsrichtung der Hand weg zeigen.}
Nun sind die Anschlagsbewegungen der Finger nicht gerade nach unten gerichtet, sondern haben eine horizontale Rückwärtskomponente, die die Fingerspitzen in die Lage versetzt, ein wenig länger auf den Tasten zu verweilen, während die Hand an der Tastatur entlang bewegt wird.
Das ist besonders für das Legatospielen hilfreich.
Drehen Sie zum Beispiel bei der aufsteigenden Tonleiter der rechten Hand den Unterarm ein wenig im Uhrzeigersinn, sodass die Finger nach links zeigen.
Mit anderen Worten: Wenn die Finger (relativ zur Hand) gerade herunterkämen und die Hand sich bewegen würde, dann würden die Finger nicht gerade auf die Tasten herunterkommen.
Durch das leichte Drehen der Hand in die Glissandorichtung kann dieser Fehler kompensiert werden.
Somit gestattet die Glissandobewegung der Hand, sanft zu gleiten.
Sie können diese Bewegung durch das Auf- und Abwärtszirkulieren von einer Oktave üben; die Handbewegung sollte der eines Schlittschuhläufers gleichen, der mit den Füßen abwechselnd seitwärts tritt und dessen Körper sich abwechselnd nach beiden Seiten neigt, während er geradeaus gleitet.
Die Hand sollte mit jeder Richtungsänderung der Oktave ein- oder auswärts drehen.
So wie man sich beim Schlittschuhlaufen in die entgegengesetzte Richtung lehnen muss, bevor man die Bewegungsrichtung ändern kann, muss die Drehung der Hand (Umkehrung der Glissando-Handposition) dem Wechsel der Richtung der Tonleiter vorausgehen.
Diese Bewegung übt man am besten mit einer einzelnen Oktave.

Üben Sie für die absteigende Tonleiter der rechten Hand mit Daumenübersatz das parallele Set 54321 und die anderen relevanten Sets mit und ohne ihre Verbindungen.
\textbf{Sie müssen nur eine kleine Veränderung vornehmen, um zu vermeiden, dass der Daumen sich komplett unter die Hand falten kann, während das nächste parallele Set über den Daumen rollt.}
Heben Sie, während Sie die Tonleiter gleichmäßig halten, den Daumen so früh wie möglich, indem Sie das Handgelenk anheben und/oder drehen, um den Daumen hoch zu ziehen - fast das Gegenteil von dem, was Sie bei der aufsteigenden Tonleiter getan haben.
Wenn Sie den Daumen komplett unter die Hand falten, wird er gelähmt und ist schwer auf die nächste Position zu bewegen.
Das ist die \enquote{leichte Änderung}, die oben angesprochen wurde, und ist der Daumenbewegung für die aufsteigende Tonleiter ziemlich ähnlich.
Beim Spielen mit Untersatz darf sich der Daumen komplett unter die Handfläche falten.
\textbf{Weil diese Bewegung beim Daumenübersatz und -untersatz ziemlich ähnlich ist und sich nur graduell unterscheidet, kann sie leicht unkorrekt gespielt werden.}
Obwohl die Unterschiede in der Bewegung sichtbar gering sind, sollte der Unterschied im Gefühl für den Klavierspieler - besonders bei schnellen Passagen - wie Tag und Nacht sein.

Denken Sie bei superschnellen Tonleitern (mehr als eine Oktave je Sekunde) nicht in Begriffen von einzelnen Noten, sondern in Einheiten von parallelen Sets.
Benennen Sie bei der rechten Hand 123=A und 1234=B, und spielen Sie AB anstelle von 1231234, das heißt zwei Elemente anstelle von sieben.
Denken Sie bei noch schnellerem Spielen in Einheiten von Paaren paralleler Sets: AB, AB usw.
Wenn Sie in der Geschwindigkeit voranschreiten und anfangen, in größeren Einheiten zu denken, sollte die Kontinuitätsregel von A1 über AB1 zu ABA geändert werden (wobei das letzte A die Verbindung ist).
Es ist eine schlechte Idee, zu viel schnell zu üben, mit Geschwindigkeiten, die man nicht bequem handhaben kann.
\textbf{Die Ausflüge in sehr schnelles Spielen sind nur nützlich, um das genaue Üben mit einer geringeren Geschwindigkeit zu vereinfachen.
Üben Sie deshalb die meiste Zeit mit einer geringeren als der Maximalgeschwindigkeit; Sie werden auf diese Art schneller an Geschwindigkeit gewinnen.}

Versuchen Sie das folgende Experiment, um ein Gefühl für wirklich schnelle Tonleitern zu bekommen.
Zirkulieren Sie das fünffingrige parallele Set 54321 für die absteigende Tonleiter der rechten Hand nach dem Schema, wie es in den \hyperref[c1iii7b]{Übungen für parallele Sets} beschrieben wird (beginnen Sie mit \hyperref[c1iii7b1]{Übung \#1}).
Beachten Sie, dass Sie beim Steigern der Wiederholungsgeschwindigkeit die Hand ausrichten und ein gewisses Maß an Schub oder Drehung benutzen müssen, um das schnellste, flüssige und gleichmäßige parallele Spielen zu erreichen.
Sie müssen eventuell den Abschnitt über \hyperref[c1iii5SchubZug]{Schub und Zug bei Arpeggios} (Abschnitt f) weiter unten durchgehen, bevor Sie es korrekt ausführen können.
Ein Mittelstufenschüler sollte in der Lage sein, schneller als zwei Zyklen pro Sekunde zu werden.
Wenn Sie das erst einmal schnell, zufriedenstellend und entspannt können, spielen Sie einfach eine weitere Oktave mit derselben hohen Geschwindigkeit nach unten, und stellen Sie sicher, dass Sie alles mit Daumenübersatz spielen.
Sie haben gerade entdeckt, wie man einen sehr schnellen Lauf spielt!
Wie schnell Sie spielen können, hängt von Ihrer technischen Stufe ab, und wenn Sie besser werden, wird Ihnen diese Methode gestatten, sogar noch schnellere Tonleitern zu spielen.
Üben Sie diese schnellen Läufe nicht zu viel, wenn sie anfangen ungleichmäßig zu werden, weil Sie sonst am Ende eventuell die Angewohnheit haben, unmusikalisch zu spielen.
\textbf{Diese Experimente sind hauptsächlich für das Entdecken der Bewegungen wertvoll, die bei solchen Geschwindigkeiten benötigt werden, und das Gehirn zu trainieren, solche Geschwindigkeiten zu bewältigen.}
Gewöhnen Sie sich nicht an, schnell zu spielen und zuzuhören; stattdessen muss das Gehirn zuerst eine klare Vorstellung von dem haben, was erwartet wird, bevor Sie es spielen.

Am besten fängt man nicht an, Tonleitern beidhändig zu spielen, bis man einhändig sehr zufrieden ist.
Falls Sie der Meinung sind, dass Sie Tonleitern beidhändig üben müssen (manche benutzen sie zum Aufwärmen), beginnen Sie das beidhändige Üben mit einer Oktave oder einem Teil einer Oktave, zum Beispiel einem parallelen Set.
Die C-Dur-Tonleiter ist für das Üben mit parallelen Sets nicht ideal, weil die Daumen nicht synchron sind - benutzen Sie sie \hyperref[hdur]{H-Dur-Tonleiter}, bei der die Daumen der beiden Hände synchron sind (siehe nächster Absatz).
Pflegen Sie die Angewohnheit, mit einer hohen Geschwindigkeit zum beidhändigen Spielen überzugehen (obwohl es viel leichter erscheint, langsam zu starten und dann die Geschwindigkeit schrittweise zu steigern).
Spielen Sie dazu eine Oktave mehrere Male mit der linken Hand mit einer handhabbaren hohen Geschwindigkeit, wiederholen Sie mehrere Male mit der rechten Hand mit der gleichen Geschwindigkeit, und kombinieren Sie die Hände dann mit der gleichen Geschwindigkeit.
Machen Sie sich nichts daraus, wenn die Finger zunächst nicht perfekt zusammenpassen.
Bringen Sie zuerst die ersten Noten zur Deckung, danach die ersten und letzten Noten.
Zirkulieren Sie anschließend die Oktave fortlaufend.
Arbeiten Sie dann daran, jede Note zur Deckung zu bringen.
Üben Sie \textit{dann} entspannt mit langsamer Geschwindigkeit und behalten dieselben Bewegungen bei, bis die Tonleitern sehr genau und völlig kontrolliert sind.


\label{hdur}

\textbf{Bevor Sie mit der C-Dur-Tonleiter zu weit gehen, überlegen Sie sich, die H-Dur-Tonleiter zu üben.}
Sehen Sie dazu in der \hyperref[table]{Tabelle} weiter unten die Fingersätze der Tonleitern.
Bei dieser Tonleiter spielen nur der Daumen und der kleine Finger auf den weißen Tasten, außer beim tiefsten Ton der linken Hand (Finger 4).
Alle anderen Finger spielen auf den schwarzen Tasten.
Diese Tonleiter hat folgende Vorteile:

\begin{enumerate}[label={\arabic*.}] 
\item Sie ist zunächst einfacher zu spielen, besonders für jemanden mit größeren Händen oder langen Fingern.
Jede Taste kommt ganz natürlich unter die einzelnen Finger, und jeder Finger hat genug Platz.
Aus diesem Grund lehrte Chopin diese Tonleiter den Anfängern vor der C-Dur-Tonleiter.
\item Sie gestattet es Ihnen, das Spielen der schwarzen Tasten zu üben.
Die schwarzen Tasten sind schwieriger zu spielen (leichter zu verfehlen), weil sie schmaler sind, und erfordern eine größere Genauigkeit.
\item Sie erlaubt es, mit \hyperref[c1iii4b]{flacheren Fingern} (weniger gekrümmt) zu spielen, was zum Üben des Legatos und für die Klangkontrolle besser ist.
\item Das Spielen mit Daumenübersatz ist viel einfacher.
Das ist der Grund, warum ich die C-Dur-Tonleiter benutzte, um den Übersatz zu veranschaulichen.
Bei H-Dur ist es schwieriger, den Unterschied zwischen den Unter- und den Übersatzbewegungen zu sehen.
Um jedoch die richtigen Bewegungen zu üben, ist H-Dur eventuell überlegen, wenn Sie bereits den Unterschied zwischen Untersatz und Übersatz verstehen, weil es einfacher ist, zu den höheren Geschwindigkeiten zu kommen, ohne schlechte Angewohnheiten zu erwerben.
\item Die Daumen sind bei der H-Dur-Tonleiter synchron, was es ermöglicht, das beidhändige Spielen paralleles Set für paralleles Set zu üben.
Somit ist das beidhändige Spielen leichter als bei der C-Dur-Tonleiter.
Wenn Sie diese Tonleiter erst beidhändig beherrschen, wird das beidhändige Lernen der C-Dur-Tonleiter einfacher, was Ihnen Zeit spart.
Sie werden auch genau verstehen, warum die C-Dur-Tonleiter schwieriger ist.

 \end{enumerate}
\textbf{Dieser Abschnitt ist für diejenigen gedacht, die nur den Daumenuntersatz gelernt haben und nun den Übersatz lernen müssen.
Zunächst wird es Ihnen eventuell so vorkommen, als ob die Finger sich alle verknoten und es schwierig sei, eine klare Vorstellung davon zu bekommen, was Daumenübersatz ist.}
Der Hauptgrund für diese Schwierigkeit ist die Angewohnheit, die man beim Spielen mit Untersatz erworben hat und die nun verlernt werden muss.
Der Daumenübersatz ist eine neue Fertigkeit, die Sie lernen müssen, und er ist nicht schwieriger zu lernen als eine Bach-Invention.
\textbf{Die beste Nachricht von allen ist aber, dass Sie wahrscheinlich bereits wissen, wie man mit Daumenübersatz spielt!
Versuchen Sie, eine sehr \hyperref[c1iii5h]{schnelle chromatische Tonleiter} zu spielen.}
Beginnen Sie mit C, und spielen Sie 13131231313 usw.
Die \hyperref[c1iii4b]{flache Fingerhaltung} wird hierbei nützlich sein.
Wenn Sie eine sehr schnelle chromatische Tonleiter spielen können, dann ist die Daumenbewegung genau die, die Sie für den Übersatz benötigen, weil es unmöglich ist, eine sehr schnelle chromatische Tonleiter mit dem Untersatz zu spielen.
Verlangsamen Sie nun die Daumenbewegung der schnellen chromatischen Tonleiter und übertragen Sie diese auf die H-Dur-Tonleiter; betrachten Sie diese H-Dur-Tonleiter als eine chromatische Tonleiter, in der nur ein paar der weißen Tasten gespielt werden.
Wenn Sie die H-Dur-Tonleiter mit Übersatz spielen können, übertragen Sie diese Bewegung auf C-Dur.

Natürlich ist das Lernen von Tonleitern und \hyperref[Arpeggios]{Arpeggios} (siehe unten) mit Daumenübersatz nur der Anfang.
Dieselben Prinzipien sind auf jede Situation anwendbar, in die der Daumen einbezogen ist - in jedem Musikstück, an jeder Stelle, die ziemlich schnell ist.
Wenn die Tonleitern und Arpeggios erst einmal gemeistert sind, dann sollte in diesen anderen Situationen mit Daumenübersatz alles fast wie von selbst gehen.
Damit sich dies gewissermaßen automatisch entwickelt, müssen Sie gleichbleibende und optimierte Fingersätze für die Tonleitern benutzen; diese sind in den \hyperref[table]{Tabellen} weiter unten aufgelistet.

Diejenigen, für die der Daumenübersatz neu ist und die viele Stücke mit dem Daumenuntersatz gelernt haben, werden zurückgehen und alle alten Stücke überarbeiten müssen, die schnelle Läufe und gebrochene Akkorde enthalten.
Ideal wäre es, all die alten Stücke, die mit Untersatz gelernt wurden, zu wiederholen, um an den Stellen, an denen der Übersatz besser geeignet ist, die Angewohnheit des Untersatzes völlig loszuwerden.
Es ist eine schlechte Idee, einige Stücke mit Untersatz und andere mit Übersatz zu spielen, wenn sie ähnliche Fingersätze haben.
Eine Möglichkeit, die Umstellung zum Übersatz zu verwirklichen, ist, zunächst Tonleitern und Arpeggios zu üben, sodass Sie sich an den Übersatz gewöhnen.
Lernen Sie dann einige \textit{neue} Kompositionen, und benutzen Sie dabei den Übersatz.
Nach ungefähr sechs Monaten, wenn Sie sich an den Übersatz gewöhnt haben, können Sie damit beginnen, Ihre ganzen alten Stücke umzustellen.

Daumenübersatz und Daumenuntersatz sollten als die Extreme von zwei verschiedenen Arten, den Daumen zu benutzen, angesehen werden.
Das heißt, es gibt viele weitere Bewegungen dazwischen.
\textbf{Ein unerwarteter Nutzen des Lernens des Daumenübersatzes ist, dass man beim Untersatzspielen besser wird.
Das geschieht, weil Ihr Daumen technisch fähiger und geschickter wird: Er wird \underline{frei}.}
Und Sie gewinnen die Fähigkeit, all die Bewegungen zwischen Über- und Untersatz zu benutzen, die eventuell erforderlich sind, je nachdem welche weiteren Noten gespielt werden oder welche Art von Ausdruck Sie erzeugen möchten.
\textbf{Der Daumen hat nun die Freiheit, alle ihm zur Verfügung stehenden Bewegungen zu benutzen und für die Steuerung des Klangs.
Diese Freiheit sowie die Fähigkeit, nun technisch viel schwierigeres Material korrekt zu spielen, verwandeln den Daumen in einen sehr vielseitigen Finger.}


\subsubsection{Tonleitern: Herkunft, Namensgebung, Fingersätze}
\label{c1iii5d}

\textbf{Es wird in diesem Buch davon abgeraten, Tonleitern und Übungen stupide zu wiederholen.
Es ist jedoch von entscheidender Bedeutung, die Fertigkeit zu entwickeln, perfekte Tonleitern und Arpeggios zu spielen, um einige grundlegende Techniken und Standard-Fingersätze für das routinemäßige Spielen und das \hyperref[c1iii11]{Spielen vom Blatt} zu erwerben.}
Tonleitern und Arpeggios sollten in allen Dur- und Molltonarten geübt werden, bis Sie mit den Fingersätzen vertraut sind.
Sie sollten frisch und respekteinflößend klingen, nicht laut aber überzeugend; sie anzuhören sollte einem die Stimmung anheben.
Das wichtigste Ziel ist dabei, solange zu üben, bis die Fingersätze der einzelnen Tonleitern automatisiert sind. 

Lassen Sie uns, bevor wir mit den Fingersätzen fortfahren, einige grundlegende Eigenschaften von Tonleitern besprechen: die Namensgebung der Tonarten und die Frage, was eine Tonleiter ist.
\textbf{Es ist nichts magisches oder gar musikalisches an der C-Dur-Tonleiter; sie erwächst einfach aus dem Wunsch, so viele Intervalle wie möglich in eine Oktave zu fassen, die mit einer Hand gespielt werden kann.}
Das ist nur ein auf die Bequemlichkeit ausgerichtetes Designelement (so wie die modernsten Elemente in das Design eines jeden neuen Autos einfließen), das sowohl das Lernen des Klavierspielens als auch das Spielen vereinfacht.
Anhand der Größe der menschlichen Finger und Hände können wir annehmen, dass das größte Intervall acht Tasten umfassen sollte.
Wie viele Intervalle kann man darin unterbringen?
Wir benötigen die Oktave, sowie Terzen, Quarten, Quinten und Sexten.
Wenn wir mit C4 beginnen, haben wir nun E4, F4, G4, A4 und C5, also insgesamt sechs Noten, was nur noch Platz für zwei Noten mit einem Intervall von einem Ganztonschritt und einem Halbtonschritt lässt.
Beachten Sie, dass sogar die kleine Terz bereits als A4-C5 vorhanden ist.
Fügt man den Halbtonschritt nach C4 ein, benötigt man ein Vorzeichen (schwarze Taste) bei C4 und vier Vorzeichen bei C5, um die chromatische Tonleiter zu vervollständigen\footnote{das heißt nach den derzeitigen Notenbezeichnungen kämen auf die weißen Tasten die Noten C4-C\#4-E4-F4-G4-A4-C5 und auf die schwarzen Tasten die Noten D4-F\#4-G\#4-A\#4-H4}.
Es ist deshalb besser, den Halbtonschritt vor C5 einzufügen, sodass die Oktave mit zwei Vorzeichen bei C4 und drei bei C5 ausgewogener ist.
Das vervollständigt die Konstruktion der C-Dur-Tonleiter, einschließlich der Vorzeichen (Sabatella, Mathiew).

Für die Namensgebung ist es unglücklich, dass die C-Dur-Tonleiter auf der Tastatur mit dem C und nicht mit dem A beginnt.
Somit wechseln die Oktavnummern beim C, nicht beim A; deshalb tragen die Noten um C4 die Bezeichnungen ...A3-H3-C4-D4-E4...\footnote{die internationale Reihenfolge ist ...A3-B3-C4-D4-E4...}.
Bei jeder Tonleiter wird die erste Note als \textbf{Tonika} bezeichnet, das heißt C ist die Tonika der C-Dur-Tonleiter.
Die tiefste Note auf einer Tastatur mit 88 Tasten ist A0, und die höchste Note ist C8.


\label{table}

Die \textbf{Standard-Fingersätze für aufsteigende Durtonleitern} sind 12312345 (rechte Hand, eine Oktave) und 54321321 (linke Hand) für die C-G-D-A-E Dur-Tonleitern (mit jeweils 0-1-2-3-4 Kreuzen); diese Fingersätze werden im Folgenden mit S1 und S2 abgekürzt, wobei \textbf{S für \enquote{Standard} steht}.
Die Kreuze kommen in der Reihenfolge der Noten F-C-G-D-A hinzu (G-Dur hat F\#; D-Dur hat F\# und C\#; A-Dur hat F\#, C\# und G\#; usw.), und bei den F-B-Es-As-Des-Ges Dur-Tonleitern kommen die Be's in der Reihenfolge der Noten H-E-A-D-G-C hinzu; \textbf{jedes Intervall zwischen zwei aufeinander folgenden Buchstaben ist eine Quinte}.
Sie sind deshalb leicht zu merken, besonders wenn Sie ein Geigenspieler sind (die freien Saiten der Geige sind G-D-A-E).
Die Buchstaben erscheinen immer in der Reihenfolge G-D-A-E-H-F-C, das heißt im kompletten Quintenzirkel; diese Reihenfolge sollten Sie sich merken.
Schauen Sie sich die H-Dur- oder Ges-Dur-Tonleiter in einem Notenheft an, und Sie werden sehen, wie die fünf Kreuze oder sechs Be's in derselben Folge aufgereiht sind.
Somit stehen zwei Kreuze bei F-C, drei Kreuze stehen bei F-C-G, usw.
Die Be's nehmen in umgekehrter Reihenfolge wie die Kreuze zu.
Jede Tonleiter wird durch ihre \textbf{Tonartenvorzeichnung} identifiziert; so hat zum Beispiel die Vorzeichnung der G-Dur-Tonleiter ein Kreuz (F\#).
\textbf{Wenn Sie gelernt haben, eine Quinte zu erkennen, können Sie alle Tonleitern in der ansteigenden Reihenfolge der Kreuze erzeugen (indem Sie vom C aus in Quinten aufwärts gehen) oder in der absteigenden Reihenfolge der Be's (indem Sie in Quinten abwärts gehen)}; das ist nützlich, wenn Sie alle Tonleitern nacheinander üben möchten, ohne auf einen Ausdruck sehen zu müssen.
In der folgenden Tabelle sehen Sie die Fingersätze für die aufsteigenden Durtonleitern (kehren Sie die Fingersätze für die absteigenden Tonleitern um).

\label{enharmonisch}\footnote{In der Literatur sind manchmal auch die weiteren Tonarten mit mehr als fünf Kreuzen bzw. sechs Be's zu finden.
Diese können aber - zumindest auf dem Klavier - durch enharmonische Verwechslung aus den hier angegebenen erzeugt werden, zum Beispiel wird Cis-Dur mit 7 Kreuzen zu Des-Dur mit 12-7=5 Be's.}

<table border cellpadding=\enquote{7}>
 <tr>
  <td bgcolor=\enquote{\#E0E0E0}>Linke Hand</td>
  <td bgcolor=\enquote{\#E0E0E0}>Rechte Hand</td>
  <td bgcolor=\enquote{\#E0E0E0}>Tonarten</td>
  <td bgcolor=\enquote{\#E0E0E0}>Kreuze / Be's \\ 
 S2=54321321 & S1=12312341 & CGDAE & 0-4 Kreuze \\ 
 43214321321 & S1 & H & 5 Kreuze \\ 
 S2 & 12341231 & F & 1 Be \\ 
 32143213 & 41231234 & B & 2 Be's \\ 
 32143213 & 31234123 & Es & 3 Be's \\ 
 32143213 & 34123123 & As & 4 Be's \\ 
 32143213 & 23123412 & Des & 5 Be's \\ 
 43213214 & 23412312 & Ges & 6 Be's \\ 
</table>

\label{tablemoll}

Die Molltonleitern sind komplex, weil es drei davon gibt.
Es kann verwirrend sein, dass sie oftmals nur Molltonleitern genannt werden, ohne genau anzugeben, welche der drei jeweils gemeint ist.
Es werden auch verschiedene Bezeichnungen benutzt.
Die Molltonleitern wurden geschaffen, weil sie eine von den anderen Tonleitern abweichende Atmosphäre erzeugen.
Die einfachste Molltonleiter ist die \textbf{reine Molltonleiter} (auch \textbf{natürliche Molltonleiter} genannt); sie ist einfach, weil sie dieselbe Tonartenvorzeichnung wie die Durtonart hat aber die Tonika sechs Noten höher als die der Durtonleiter ist.
Ich finde es leichter, sich das als kleine Terz abwärts statt als eine Sexte aufwärts zu merken. 
Somit hat die reine Molltonleiter zu G-Dur ihre Tonika bei E, die Vorzeichnung ist F\#, und sie wird e-Moll genannt.\footnote{Gemeint ist die \textbf{Paralleltonart}, die aus denselben Noten wie die Durtonart besteht aber mit der sechsten Stufe der Durtonleiter beginnt. Wenn man die Noten der Durtonleiter und der parallelen reinen Molltonleiter durchnummeriert, haben die Noten mit derselben Nummer jeweils denselben Abstand voneinander: drei Halbtöne, das heißt eine kleine Terz. Davon zu unterscheiden ist die \textbf{Varianttonart}. Die Molltonleiter der Varianttonart beginnt mit derselben Note wie die Durtonleiter, das heißt die Varianttonart von G-Dur ist g-Moll mit den Noten G-A-B-C-D-Eb-F-G und der Vorzeichnung B-Eb.}
Die \textbf{harmonische  Molltonleiter} wird am häufigsten benutzt und entsteht, wenn man die siebte Note der reinen Molltonleiter um einen Halbton anhebt.
Die \textbf{melodische  Molltonleiter} entsteht, wenn man die sechste und die siebte Note der reinen Molltonleiter um einen Halbton anhebt.
Die sechste und die siebte Note werden meistens nur beim Aufsteigen angehoben und bleiben beim Absteigen unverändert.

\textbf{Die Fingersätze für die harmonischen Molltonleitern} (die letzte Spalte bezeichnet die geänderte Note\footnote{das heißt auf welchen Ton die 7. Stufe der reinen Molltonleiter jeweils um einen Halbtonschritt angehoben wird, um die harmonische Molltonleiter zu bilden.}; die harmonische a-Moll-Tonleiter ist A-H-C-D-E-F-G\#-A, und die parallele Dur-Tonart ist C):


<table border cellpadding=\enquote{7}>
 <tr>
  <td bgcolor=\enquote{\#E0E0E0}>LH</td>
  <td bgcolor=\enquote{\#E0E0E0}>RH</td>
  <td bgcolor=\enquote{\#E0E0E0}>Tonarten</td>
  <td bgcolor=\enquote{\#E0E0E0}>Kreuze / Be's</td>
  <td bgcolor=\enquote{\#E0E0E0}>Note \\ 
 S2 & S1 & A &   & Gis \\ 
 S2 & S1 & E & 1 Kreuz & Dis \\ 
 43214321 & S1 & H & 2 Kreuze & Ais \\ 
 43213214 & 34123123 & Fis & 3 Kreuze & Eis \\ 
 32143213 & 34123123 & Cis & 4 Kreuze & His \\ 
 32143213 & 34123123 & Gis & 5 Kreuze & Fisis \\ 
 S2 & S1 & D & 1 Be & Cis \\ 
 S2 & S1 & G & 2 Be's & Fis \\ 
 S2 & S1 & C & 3 Be's & H \\ 
 S2 & 12341231 & F & 4 Be's & E \\ 
 21321432 & 21231234 & B & 5 Be's & A \\ 
 21432132 & 31234123 & Es & 6 Be's & D \\ 
</table>
Wie bereits gesagt, ist an den Tonleitern nichts Magisches; sie sind einfach menschliche Erzeugnisse, die aus Bequemlichkeit erdacht wurden - nur ein Rahmen, in den wir die Musik einspannen.
Deshalb kann man eine beliebige Anzahl davon erzeugen, und die hier behandelten sind, obwohl sie häufig benutzt werden, nicht die einzigen.
\textit{[Informationen über weitere Tonleitern finden Sie unter anderem in Marc Sabatellas \enquote{A Jazz Improvisation Primer}: \hyperref[http://www.outsideshore.com/primer/primer/index.html]{das Original in Englisch} <font color=\enquote{blue} size=\enquote{-1}>(extern), \hyperref[http://msjipde.uteedgar-lins.de/index.html]{als deutsche Übersetzung} (extern).]}</font>

Man kann Tonleitern nicht zu gut spielen.
Wenn Sie Tonleitern üben, versuchen Sie immer, etwas Bestimmtes zu erreichen - weicher, leiser, deutlicher, schneller.
Bringen Sie die Hände zum Gleiten, die Tonleiter zum Singen; fügen Sie Farbe hinzu\footnote{gemeint ist der Gesamteindruck, der aus der Kombination von Dynamik, Rhythmus, Phrasierung usw. entsteht, zum Beispiel Gefühle oder die Beschreibung einer Landschaft}, Ausdrucksstärke oder das Gefühl von  Begeisterung.
Hören Sie auf, sobald Ihre Konzentration nachzulassen beginnt.
Es gibt keine maximale Geschwindigkeit beim parallelen Spielen.
Deshalb können Sie im Prinzip Ihr ganzes Leben lang die Geschwindigkeit und die Genauigkeit steigern - was ordentlich Spaß machen kann und sicherlich auch süchtig.
Wenn Sie Ihre Geschwindigkeit einem Publikum demonstrieren möchten, können Sie das wahrscheinlich mit Tonleitern und Arpeggios mindestens genauso gut wie mit jedem Musikstück.



<!-- c1iii5b.html -->

\label{c1iii5e}
\subsubsection{Arpeggios (Chopin, Wagenradbewegung, \enquote{gespreizte} Finger)}
\label{Arpeggios}

\textbf{Arpeggios korrekt zu spielen ist technisch komplex.
Deshalb eignen sich Arpeggios besonders gut für das Lernen einiger wichtiger \hyperref[c1iii4]{Handbewegungen}, wie Schub, Zug und die Wagenradbewegung.}
\enquote{Arpeggio}, so wie es hier benutzt wird, schließt gebrochene Akkorde und Kombinationen von kurzen arpeggioartigen Passagen ein.
Wir werden diese Konzepte hier verdeutlichen, indem wir den 3. Satz von Beethovens Mondschein-Sonate für den Schub und Zug und Chopins Fantaisie Impromptu (FI) für die Wagenradbewegung benutzen.
Erinnern Sie sich daran, dass die Geschmeidigkeit der Hände, insbesondere im Handgelenk, für das Spielen von Arpeggios entscheidend ist.
Die technische Komplexität der Arpeggios kommt von der Tatsache, dass in den meisten Fällen diese Geschmeidigkeit mit allem anderen kombiniert werden muss: Schub, Zug, Wagenradbewegung, Glissandobewegung (oder \enquote{gespreizte} Finger) und Daumenuntersatz oder Daumenübersatz.
Ein Warnhinweis: Die Mondschein-Sonate ist wegen der erforderlichen Geschwindigkeit schwierig.
Viele Kompositionen von Beethoven können nicht verlangsamt werden, weil sie so eng mit dem Rhythmus verbunden sind.
Außerdem erfordert dieser Satz, dass Sie mindestens eine None bequem greifen können.
Diejenigen mit kleineren Händen werden größere Schwierigkeiten haben, dieses Stück zu lernen, als diejenigen mit einer angemessenen Reichweite.

Lassen Sie uns zunächst besprechen, wie man Arpeggios mit Daumenübersatz spielt.
Arpeggios, die über mehrere Oktaven gehen, werden wie Tonleitern mit Übersatz gespielt.
Deshalb wissen Sie, wenn Sie Tonleitern mit Übersatz spielen können, im Prinzip, wie man Arpeggios mit Übersatz spielt.
Arpeggios mit Übersatz zu spielen, ist jedoch ein extremeres Beispiel für die Übersatzbewegung als Tonleitern und dient deshalb als das deutlichste Beispiel für diese Bewegung.
Wir haben oben festgestellt, dass die einfachste Übersatzbewegung jene ist, die beim Spielen von chromatischen Tonleitern benutzt wird (1313123131312 usw. für die rechte Hand).
Die chromatische Übersatzbewegung ist einfach, weil die horizontale Bewegung des Daumens gering ist.
Die nächste, etwas schwierigere Bewegung ist die zum Spielen der H-Dur-Tonleiter.
Diese Übersatzbewegung ist einfach, weil man die gesamte Tonleiter mit flachen Fingern spielen kann, sodass es kein Kollisionsproblem mit dem vorbeigehenden Daumen gibt.
Die nächstschwierigere ist die C-Dur-Tonleiter; sie ist schwieriger, weil alle Finger im engen Bereich der weißen Tasten zusammengedrängt sind.
Die schwierigste Bewegung ist schließlich das Arpeggio mit Übersatz, bei dem die Hand schnell und exakt bewegt werden muss.
Diese Bewegung erfordert eine leichte Beugung und eine kurze, schnelle Drehung des Handgelenks, die manchmal als \enquote{Wurfbewegung} bezeichnet wird.
Das Schöne am Aneignen der Übersatzbewegung für das Arpeggios ist, dass man, sobald man sie gelernt hat, einfach eine kleinere Version derselben Bewegung machen muss, um die leichteren Übersatzbewegung zu spielen.

Der Standard-Fingersatz für das aufsteigende Arpeggio CEGCEG...C mit der rechten Hand ist 123123...5, mit der linken Hand 5421421...1 und umgekehrt für die absteigenden Arpeggios.
In \enquote{Michael Aaron, Adult Piano Course, Book Two} finden Sie die Fingersätze aller Arpeggios und Tonleitern.

\textbf{Weil Arpeggios mehrere Noten übergehen, spreizen die meisten die Finger, um die Noten zu erreichen.
Bei schnellen Arpeggios ist das ein Fehler, weil das Spreizen der Finger ihre Bewegung verlangsamt.}
Der Schlüssel zu schnellen Arpeggios ist, die Hand zu bewegen, anstatt die Finger zu spreizen.
Wenn Sie die Hand und das Handgelenk entsprechend bewegen, werden Sie feststellen, dass es nicht notwendig ist, die Finger zu spreizen.
Diese Methode vereinfacht auch das \hyperref[c1ii14]{Entspannen}.


\paragraph{Die Wagenradbewegung (Chopins FI)}
\label{c1iii5wagen}

Um die Wagenradbewegung zu verstehen, legen Sie Ihre linke Handfläche flach auf die Tasten, und spreizen Sie die Finger wie die Speichen eines Rades.
Beachten Sie, dass die Fingerspitzen vom kleinen Finger bis zum Daumen ungefähr auf einen Halbkreis fallen.
Halten Sie nun den kleinen Finger über die C3-Taste und parallel dazu; Sie müssen die Hand drehen, sodass der Daumen näher zu Ihnen kommt.
Bewegen Sie dann die Hand zur Klappe hin, sodass der kleine Finger die Klappe berührt; achten Sie darauf, dass die Hand stets fest gespreizt ist.
Wenn der vierte Finger zu lang ist und die Klappe zuerst berührt, drehen Sie die Hand weit genug, sodass der kleine Finger die Klappe berührt, aber halten Sie den kleinen Finger so parallel wie möglich zur C3-Taste.
\textbf{Drehen Sie nun die Hand wie ein Rad gegen den Uhrzeigersinn (von oben gesehen), sodass jeder nachfolgende Finger die Klappe (ohne Gleiten) berührt, bis Sie den Daumen erreichen.
Das ist die Wagenradbewegung in der horizontalen Ebene.
Wenn Ihre normale Reichweite mit ausgestreckten Fingern eine Oktave beträgt, werden Sie feststellen, dass die Wagenradbewegung fast zwei Oktaven abdeckt!}
Sie erhalten eine zusätzliche Reichweite, weil diese Bewegung die Tatsache ausnutzt, dass die mittleren drei Finger länger sind als der kleine Finger oder der Daumen und der Umfang eines Halbkreises viel größer ist als der Durchmesser.
Wiederholen Sie nun die Bewegung mit vertikaler Hand (die Handfläche parallel zur Klappe), sodass die Finger abwärts zeigen.
Beginnen Sie mit dem senkrecht stehenden kleinen Finger und senken Sie die Hand, um C3 zu spielen.
Wenn Sie nun die Hand zum C4 aufwärts rollen (machen Sie sich keine Sorgen, wenn es sich sehr unbeholfen anfühlt), wird jeder Finger die Note \enquote{spielen}, die er berührt.
Wenn Sie den Daumen erreichen, werden Sie wieder feststellen, dass Sie eine Entfernung überdecken, die fast das Doppelte Ihrer normalen Reichweite beträgt.
\textbf{In diesem Absatz haben wir drei Dinge gelernt:}

\begin{enumerate} 
 \item \textbf{Wie man mit der Hand \enquote{ein Wagenrad schlägt}.}
 \item \textbf{Diese Bewegung erweitert Ihre effektive Reichweite, ohne dass Sie Sprünge ausführen.}
 \item \textbf{Die Bewegung kann benutzt werden, um die Tasten zu \enquote{spielen}, ohne die Finger relativ zur Hand zu bewegen.}
\end{enumerate}
Beim tatsächlichen Üben wird das Wagenrad so benutzt, dass die Hand irgendwo zwischen der vertikalen und der horizontalen Position ist und die Finger in \hyperref[c1iii4b]{Pyramidenhaltung} oder leicht gebogen sind.
Obwohl diese Wagenradbewegung einen Beitrag zur Tastenbewegung leistet, werden Sie zum Spielen auch die Finger bewegen müssen.

Es ist kaum zu glauben, \textbf{aber Sie können die Reichweite sogar noch weiter ausdehnen, indem Sie die Finger \enquote{spreizen} (Fraser), was eine Form der Glissandobewegung ist}.
Stellen Sie sich vor, dass Sie eine übertriebene Glissandobewegung auf das aufsteigende Arpeggio CEGCEG... mit der rechten Hand anwenden;
Sie können nun den Abstand zwischen den Fingern gegenüber dem Wagenrad vergrößern.
Um das zu zeigen, bilden Sie ein V mit den Fingern 2 und 3, und legen Sie das V so an die Kante einer ebenen Fläche, dass nur das V auf der Fläche liegt.
Spreizen Sie das V so weit Sie es leicht und bequem können.
Drehen Sie dann Ihren Arm und die Hand im Uhrzeigersinn um 90 Grad, sodass die Finger nun die Fläche mit den Seiten berühren.
Das ist eine übertriebene Glissandobewegung.
Sie können nun die Finger noch weiter spreizen.
Das funktioniert mit jedem Fingerpaar.

Indem Sie eine Kombination aus \hyperref[c1iii5a]{Daumenübersatz}, \hyperref[c1iii4b]{flachen Fingerhaltungen}, Wagenradbewegung und \enquote{gespreizten} Fingern benutzen, können Sie deshalb leicht schnelle Arpeggios mit geringer Belastung für die Streckmuskeln greifen und spielen.
Beachten Sie, dass diese komplexe Kombination von Bewegungen durch ein lockeres Handgelenk ermöglicht wird.
Wenn diese Kombination von Bewegungen erst einmal leicht für Sie ist, verfügen Sie über genügend Kontrolle, sodass Sie die Gewissheit erlangen, nie eine Note zu verfehlen.
Üben Sie das CEG-Arpeggio mit diesen Bewegungen.

Wir wenden diese Methode auf die gebrochenen Akkorde in der linken Hand von Chopins FI an.
In Abschnitt III.2 haben wir die \hyperref[c1iii2]{Anwendung des Zirkulierens} beim Üben der linken Hand besprochen.
Wir werden dem Zirkulieren nun die Wagenradbewegung usw. hinzufügen.
Zirkulieren Sie die ersten 6 (oder 12) Noten der linken Hand von Takt 5 (bei dem die rechte Hand zum ersten Mal einfällt).
Beginnen wir zunächst nur mit der Wagenradbewegung.
Wenn Sie die Hand fast waagrecht halten, dann muss praktisch der ganze Tastenweg durch die Fingerbewegung zurückgelegt werden.
Wenn Sie jedoch die Hand mehr und mehr zur Vertikalen anheben, trägt die Wagenradbewegung mehr zum Tastenweg bei, und Sie brauchen weniger Fingerbewegung zum Spielen.
\textbf{Die Wagenradbewegung ist besonders für diejenigen mit kleinen Händen nützlich, weil sie automatisch die Reichweite ausdehnt.
Ein Wagenrad zu schlagen vereinfacht es auch zu entspannen, weil es weniger notwendig ist, die Finger weit auseinander gespreizt zu halten.
Sie werden auch feststellen, dass Ihre Kontrolle gesteigert wird, weil die Bewegungen nun zum Teil von den großen Bewegungen der Hand gesteuert werden, was das Spielen weniger abhängig von der Bewegung der einzelnen Finger macht und zu einheitlicheren, gleichmäßigeren Ergebnissen führt.}
Benutzen Sie soviel flache Fingerhaltungen wie notwendig, und fügen Sie ein wenig Glissandobewegung hinzu.  


Die rechte Hand ist sogar eine noch größere Herausforderung.
Die meisten schnellen Läufe sollten mit dem \hyperref[c1iii1a1]{Basisanschlag} (langsam) und den \hyperref[c1ii11]{parallelen Sets} (schnell) geübt werden.
Der mit Takt 13 beginnende Teil sollte erst wie ein Tremolo geübt werden (\hyperref[c1iii3b]{Abschnitt 3b}) und dann mit parallelen Sets.
Das heißt, üben Sie zunächst (langsam), indem Sie nur die Finger benutzen und ohne Handbewegungen.
Benutzen Sie dann hauptsächlich eine Drehung des Arms und der Hand, um Takt 15 zu spielen.
Übertreiben Sie diese Bewegungen während Sie langsam üben; steigern Sie dann schrittweise die Geschwindigkeit, indem Sie jede dieser Bewegungen vermindern, und kombinieren Sie anschließend die Bewegungen, um die Geschwindigkeit noch weiter zu steigern.
Fügen Sie dann die parallelen Sets hinzu, wobei Sie alle vier Noten während einer Abwärtsbewegung der Hand spielen.
Spielen Sie die weißen Tasten mit \hyperref[c1ii2]{gebogenen Fingern} und die schwarzen Tasten mit \hyperref[c1iii4b]{flachen Fingerhaltungen}.
Benutzen Sie die Muskeln zum Spreizen der Handflächen (\hyperref[c1iii7e]{Abschnitt 7e}) statt denen zum Spreizen der Finger, und üben Sie das schnelle \hyperref[c1ii14]{Entspannen} nach dem Spielen jeder der Oktaven in Takt 15.

\label{c1iii5f}
\subsubsection{Schub und Zug, Beethovens Mondschein-Sonate, 3. Satz}
\label{c1iii5SchubZug}

Für diejenigen, die Beethovens Mondschein-Sonate das erste Mal lernen, ist das beidhändige Arpeggio-Ende des dritten Satzes (Takte 196-198; dieser Satz hat 200 Takte) der schwierigste Abschnitt.
Indem wir darstellen, wie man diese schwierige Passage übt, können wir zeigen, wie Arpeggios geübt werden sollten.
Lassen Sie uns die rechte Hand zuerst versuchen.
Um das Üben zu vereinfachen, überspringen wir die erste Note in Takt 196 und üben nur die vier folgenden aufsteigenden Noten (E, G\#, C\#, E), die wir zirkulieren.
\textbf{Machen Sie beim \hyperref[c1iii2]{Zirkulieren} mit der Hand eine elliptische Bewegung im Uhrzeigersinn (von oben gesehen).}
Wir teilen diese Ellipse in zwei Teile auf: Der obere Teil ist die Hälfte zum Klavier hin, und der untere Teil ist die Hälfte zu Ihrem Körper hin.
Wenn Sie die obere Hälfte spielen, \enquote{schieben} Sie Ihre Hand zum Klavier hin, und wenn Sie die untere Hälfte spielen, \enquote{ziehen} Sie die Hand vom Klavier weg.
Spielen Sie die vier Noten zuerst während der oberen Hälfte, und führen Sie die Hand mit der unteren Hälfte in ihre ursprüngliche Position zurück.
Das ist die Schubbewegung für das Spielen dieser vier Noten.
Ihre Finger neigen dazu, auf das Klavier zu zu gleiten, während Sie die einzelnen Noten spielen.
Machen Sie nun mit der Hand eine Bewegung gegen den Uhrzeigersinn, und spielen Sie dieselben vier aufsteigenden Noten während der unteren Hälfte der Ellipse.
Jeder Finger neigt dazu, vom Klavier weg zu gleiten, während Sie jede Note spielen.
Diejenigen, die nicht beide Bewegungen geübt haben, finden wahrscheinlich die eine viel unhandlicher als die andere.
Fortgeschrittene Spieler sollten beide Bewegungen gleich bequem finden.

Die obige Anleitung war für das aufsteigende Arpeggio der rechten Hand.
Lassen Sie uns für das absteigende Arpeggio der rechten Hand die ersten vier Noten des nächsten Takts benutzen (die gleichen Noten wie im vorangegangenen Absatz, nur eine Oktave höher und in umgekehrter Reihenfolge).
Die Zugbewegung wird wieder für die untere Hälfte der Bewegung im Uhrzeigersinn benutzt und der Schub für die obere Hälfte der Bewegung gegen den Uhrzeigersinn.
Üben Sie sowohl für aufsteigende als auch für absteigende Arpeggios sowohl den Schub als auch den Zug, bis Sie damit zufrieden sind.
Sehen Sie nun, ob Sie die entsprechenden Übungen für die linke Hand selbst herausfinden können.
\textbf{Beachten Sie, dass diese Zyklen alle \hyperref[c1ii11]{parallele Sets} sind und deshalb extrem schnell gespielt werden können.}

Nachdem Sie nun gelernt haben, was die Schub- und Zugbewegungen sind, mögen Sie zu Recht fragen: \enquote{Warum brauche ich sie?}
Zunächst sollte darauf hingewiesen werden, dass \textbf{für die Schub- und Zugbewegungen völlig verschiedene Muskelgruppen benutzt werden.
Deshalb muss bei einer bestimmten Anwendung eine Bewegung besser sein als die andere.}
Wir werden unten lernen, dass eine Bewegung stärker als die andere ist.
Schüler, die mit diesen Bewegungen nicht vertraut sind, werden, ohne die geringste Ahnung, was sie getan haben, zufällig eine davon auswählen oder zwischen den beiden wechseln.
Das kann zu unerwarteten Spielfehlern, unnötigem Stress oder Geschwindigkeitsbarrieren führen.
Die Existenz des Schubs und Zugs ist der Situation mit \hyperref[c1iii5b]{Daumenübersatz} und Daumenuntersatz analog.
Erinnern Sie sich daran, dass Sie erst durch das Lernen des Untersatzes \textit{und} des Übersatzes alle Fähigkeiten des Daumens völlig ausnutzen.
Besonders bei hohen Geschwindigkeiten wird der Daumen auf eine Art benutzt, die ungefähr in der Mitte zwischen Untersatz und Übersatz liegt; das wichtige, das man in Erinnerung behalten muss, ist jedoch, dass die Daumenbewegung auf der Übersatzseite des genauen Mittelpunkts sein muss.
Wenn Sie nur ein wenig auf der Untersatzseite sind, dann treffen Sie auf eine Geschwindigkeitsbarriere.

Die Analogie von Schub und Zug zu Untersatz und Übersatz geht sogar noch weiter, weil Schub und Zug ebenfalls eine neutrale Bewegung haben, so wie es eine Reihe von Bewegungen gibt, die zwischen Untersatz und Übersatz liegen.
\textbf{Man bekommt die neutrale Bewegung durch das Reduzieren der kleineren Achse der Ellipse zu Null}; das heißt man verschiebt einfach die Hand nach rechts und links ohne jegliche \textit{offensichtliche} elliptische Bewegung.
Aber hier macht es wieder einen großen Unterschied, ob man sich der neutralen Position von der Schubseite oder der Zugseite nähert, weil die scheinbar ähnlichen neutralen Bewegungen in Wahrheit mit unterschiedlichen Muskelgruppen gespielt werden müssen.
Lassen Sie mich das an einem mathematischen Beispiel verdeutlichen.
Mathematiker werden entsetzt sein, wenn man ihnen sagt, dass 0 = 0 ist, was auf den ersten Blick auf triviale Weise  richtig erscheint.
Die Realität schreibt jedoch vor, dass wir sehr vorsichtig sein müssen.
Das kommt daher, dass wir die wahre Bedeutung von Null kennen müssen, das heißt wir brauchen eine mathematische Definition von Null.
Sie ist definiert als die Zahl 1/N, wobei N gegen unendlich geht.
Man bekommt \enquote{dieselbe} Zahl Null, egal ob N positiv oder negativ ist!
Unglücklicherweise bekommt man, wenn man durch Null dividiert, 1/0, ein unterschiedliches Ergebnis, je nachdem ob N positiv oder negativ ist: \enquote{1/0 = +unendlich} wenn N positiv ist und \enquote{1/0 = -unendlich} wenn N negativ ist!
Wenn Sie angenommen haben, dass die beiden Nullen dasselbe sind, könnte Ihr Fehler nach der Division so groß wie \enquote{2 * unendlich} sein, je nachdem welche Null Sie benutzt haben!
Auf ähnliche Weise sind \enquote{dieselben} neutralen Positionen, die beim Beginnen aus dem Daumenuntersatz oder Daumenübersatz heraus erreicht werden, grundlegend verschieden, und ähnlich ist es bei Schub und Zug.
Unter bestimmten Bedingungen ist entweder eine von der Schubseite oder eine von der Zugseite erreichte neutrale Position besser.
Der Unterschied im Gefühl ist beim Spielen nicht zu verkennen.
Deshalb muss man beide lernen.

Dieser Punkt ist von solch allgemeiner Wichtigkeit, insbesondere für die Geschwindigkeit, dass ich ein weiteres Beispiel anführe.
Das Leben eines Samurais hängt von der Geschwindigkeit seines Schwerts ab.
Um diese Geschwindigkeit zu maximieren, muss das Schwert stets in Bewegung sein.
Wenn der Samurai das Schwert einfach hebt, stoppt und es senkt, ist die Bewegung zu langsam, und sein Leben ist in Gefahr.
Das Schwert muss kontinuierlich auf einer kreisförmigen, elliptischen oder gekrümmten Bahn bewegt werden, auch wenn es so aussieht, als ob es nur angehoben und gesenkt wird.
Das ist eine der ersten Lektionen des Schwertkampfs.
Die Anwendung der im Grunde bogenförmigen Bewegungen zur Steigerung der Geschwindigkeit hat allgemeine Gültigkeit (Aufschlag beim Tennis, Schmettern beim Badminton usw.) und somit auch für das Klavierspielen.

Nun gut, wir haben also festgestellt, dass sowohl Schub als auch Zug notwendig sind, aber wie wissen wir, wann wir was benutzen müssen?
Im Fall des Unter- und Übersatzes waren die Regeln klar; bei langsamen Passagen kann man beide benutzen, und in bestimmten Legato-Situationen braucht man den Untersatz; bei allen anderen sollte man den Übersatz benutzen.
Bei Arpeggios lautet die Regel, dass man die starken Bewegungen als erste Wahl benutzt und die schwachen Bewegungen als zweite Wahl.
Jeder Einzelne hat eine andere starke Bewegung, sodass Sie zunächst experimentieren sollten, um zu sehen, welche für Sie die stärkste ist.
Die Zugbewegungen sollten stärker sein, da die Armmuskeln für den Zug stärker als die für den Schub sind.
Bei den Zugbewegungen werden auch die fleischigen Teile der Finger benutzt, während bei den Schubbewegungen eher die Fingerspitzen benutzt werden, wobei man sich leicht die Fingerspitzen oder das Nagelbett verletzen kann.

Man könnte die Frage stellen: \enquote{Warum nicht immer neutral spielen - weder Schub noch Zug?}
Oder nur eine Bewegung lernen (nur Schub) und einfach sehr gut darin werden?
Hier werden wir wieder an die Tatsache erinnert, dass es zwei Möglichkeiten gibt, neutral zu spielen, je nachdem, ob man sich von der Schubseite oder der Zugseite nähert, und für eine bestimmte Anwendung ist die eine üblicherweise besser als die andere.
Beachten Sie bei der zweiten Frage, dass eine zweite Bewegung wegen der Ausdauer nützlich sein könnte, da eine andere Gruppe von Muskeln benutzt wird.
Nicht nur das, sondern um die starken Bewegungen gut zu spielen, muss man wissen, wie die schwachen Bewegungen gespielt werden.
Das heißt, Sie spielen am besten, wenn die Hand in dem Sinne ausgewogen ist, dass Sie beide Bewegungen spielen können.
Deshalb sollten Sie, egal ob Sie sich entscheiden, für eine bestimmte Passage Schub oder Zug zu benutzen, immer auch die andere Bewegung üben.
Nur so können Sie wissen, welche Bewegung für Sie die beste ist.
\textbf{Wenn Sie zum Beispiel den Schluss der Beethoven-Sonate üben, sollten Sie feststellen, dass Sie einen schnelleren technischen Fortschritt machen, wenn Sie jeden Zyklus sowohl mit Schub als auch mit Zug üben.}
Am Ende sollten die meisten Schüler sehr nah an der neutralen Bewegung spielen, obwohl sich ein paar dafür entscheiden werden, übertriebene Schub- und Zugbewegungen zu benutzen.

Es gibt viel mehr neues Material, das wir in diesem dritten Satz üben sollten, bevor wir beidhändig spielen, sodass Sie in diesem Stadium wahrscheinlich nichts HT üben müssen - außer als Experiment, um zu sehen, was Sie tun können und was nicht.
Insbesondere ist, beidhändig mit den höchsten Geschwindigkeiten zu versuchen, kontraproduktiv und nicht zu empfehlen.
Einen kurzen Ausschnitt beidhändig zu zirkulieren kann jedoch sehr nützlich sein; aber dieser sollte nicht zu viel geübt werden, wenn man ihn noch nicht zufriedenstellend einhändig spielen kann.
Die Hauptschwierigkeiten in diesem Satz sind in den Arpeggios und Alberti-Begleitungen (\enquote{do-so-mi-so}-Typ)  konzentriert; haben Sie diese gemeistert, haben Sie 90\% des Satzes bezwungen.
Diejenigen ohne genügende technische Fertigkeiten sollten zufrieden sein, wenn sie \textit{vivace}-Geschwindigkeit (120) erreichen.
Wenn Sie den ganzen Satz zufriedenstellend mit dieser Geschwindigkeit spielen können, dann können Sie die zusätzliche Anstrengung für den Versuch in Richtung \textit{presto} (über 160) auf sich nehmen.
Es ist wahrscheinlich kein Zufall, dass beim 4/4-Takt \textit{presto} mit der schnellen Herzschlagrate einer sehr aufgeregten Person übereinstimmt.
Beachten Sie, dass in Takt 1 die Begleitung in der linken Hand tatsächlich wie ein schlagendes Herz klingt.

Wir werden nun unseren \enquote{Schlachtplan} für das Lernen dieses Satzes skizzieren.
Wir begannen mit dem schwierigsten Teil, dem beidhändigen Arpeggio am Ende.
Die meisten Schüler werden mit der linken Hand mehr Schwierigkeiten als mit der rechten haben; fangen Sie deshalb, sobald die rechte Hand ziemlich zufriedenstellend ist, damit an, das rechtshändige Arpeggio der ersten beiden Takte dieses Satzes zu üben, während Sie weiterhin den linkshändigen Teil des Schlusses üben.
Eine wichtige Regel für das Spielen schneller Arpeggios ist, die Finger soviel wie möglich über den Tasten zu halten und diese fast zu berühren.
Heben Sie die Finger nicht weit von den Tasten.
Erinnern Sie sich daran: Benutzen Sie die flachen Haltungen für die schwarzen Tasten und die gebogene Haltung für die weißen Tasten.
Deshalb wird in den ersten beiden Takten dieses dritten Satzes nur die Note D mit gebogenen Fingern gespielt.
Diese Angewohnheit, für jedes ansteigende Arpeggio nur bestimmte Finger zu beugen, entwickelt man am besten durch das Zirkulieren paralleler Sets.
Natürlich ist die Fähigkeit, mit jedem Finger schnell und unabhängig von den anderen Fingern von einer flachen zur gebogenen Haltung zu wechseln, eine wichtige Fertigkeit, die Sie lernen müssen.

Das Pedal wird in diesem Stück nur in zwei Situationen benutzt:

\begin{enumerate} 
 \item beim doppelten Staccato-Akkord am Ende des zweiten Takts und in allen weiteren ähnlichen Situationen.
 \item in den Takten 165 und 166, in denen das Pedal eine entscheidende Rolle spielt.
\end{enumerate}
Der nächste zu übende Abschnitt ist der tremoloartige Abschnitt der rechten Hand, der in Takt 9 beginnt.
Arbeiten Sie sorgfältig am Fingersatz für die linke Hand - diejenigen mit kleineren Händen können eventuell den fünften Finger nicht über die gesamte Dauer der beiden Takte unten halten.
Wenn Sie Schwierigkeiten damit haben, den \hyperref[c1iii1b]{Rhythmus} dieses Abschnitts zu interpretieren, hören Sie sich verschiedene Aufnahmen an, um ein paar Anregungen zu erhalten.
Dann kommen die Alberti-Begleitung der linken Hand, die in Takt 21 beginnt, und ähnliche Teile der rechten Hand, die später auftreten.
Die Alberti-Begleitung kann, wie es ab \hyperref[c1ii8]{Abschnitt II.8} erklärt wird, mit \hyperref[c1iii7b]{parallelen Sets} geübt werden.
Der nächste schwierige Abschnitt ist der Triller der rechten Hand in Takt 30.
Dieser erste Triller wird am besten mit dem Fingersatz 3,5 ausgeführt, und der zweite erfordert 4,5.
Falls Sie kleine Hände haben, sind diese Triller genauso schwierig wie die Arpeggios am Schluss und sollten von Anfang an geübt werden, wenn Sie beginnen diesen Satz zu lernen.
Das sind die grundlegenden technischen Erfordernisse dieses Stücks.
Die Kadenz von Takt 186 ist eine interessante Kombination einer \enquote{Tonleiter} und eines Arpeggios; wenn Sie Schwierigkeiten damit haben, sie zu interpretieren, hören Sie sich wieder verschiedene Aufnahmen an, um ein paar Anregungen zu erhalten.
Beachten Sie, dass die Takte 187 und 188 \textit{adagio} sind.

Beginnen Sie das beidhändige Üben, nachdem alle diese technischen Probleme einhändig gelöst sind.
\textbf{Es besteht keine Notwendigkeit, den Gebrauch des Pedals zu üben, bis Sie mit dem beidhändigen Üben anfangen.}
Beachten Sie, dass die Takte 163 und 164 ohne Pedal gespielt werden.
Dann gibt die Anwendung des Pedals bei den Takten 165 und 166 diesen beiden letzten Takten eine Bedeutung.
Wegen des schnellen Tempos besteht die Neigung, zu laut zu üben.
Das ist nicht nur musikalisch unkorrekt, sondern auch technisch schädlich.
\textbf{Zu laut zu üben kann zu Ermüdung und Geschwindigkeitsbarrieren führen; der Schlüssel zur Geschwindigkeit ist \hyperref[c1ii14]{Entspannung}.}
Es sind die \textit{p}-Abschnitte, die den größten Teil der Spannung erzeugen.
So ist zum Beispiel das \textit{ff} in Takt 33 nur eine Vorbereitung für das nachfolgende \textit{p}, und es gibt tatsächlich im ganzen Satz sehr wenige \textit{ff}.
Der ganze Abschnitt von Takt 43 bis 48 wird \textit{p} gespielt und führt zu einem einzigen Takt, 50, der \textit{f} gespielt wird.

Schließlich sollten Sie, wenn Sie richtig geübt haben, bestimmte Geschwindigkeiten finden, bei denen es einfacher ist, schneller zu spielen als langsamer zu spielen.
Das ist am Anfang völlig natürlich und ist eines der besten Zeichen, dass Sie die Lektionen dieses Buchs gut gelernt haben.
Selbstverständlich sollten Sie, wenn Sie erst die Technik beherrschen, in der Lage sein, bei jeder Geschwindigkeit mit der gleichen Leichtigkeit zu spielen.
 

\label{c1iii5g}
\subsubsection{Der Daumen: Der vielseitigste Finger}
\label{Daumen}
\textbf{Der Daumen ist der vielseitigste Finger; er lässt uns Tonleitern, Arpeggios und breite Akkorde spielen (wenn Sie es nicht glauben, versuchen Sie, eine Tonleiter ohne den Daumen zu spielen!).}
Die meisten Schüler lernen nicht, wie man den Daumen richtig benutzt, bis sie Tonleitern üben.
Deshalb ist es wichtig, Tonleitern so früh wie möglich zu üben.
Die C-Dur-Tonleiter ständig zu wiederholen, ist, auch wenn man die H-Dur-Tonleiter einschließt, nicht die richtige Art Tonleitern zu üben.
Es ist wichtig, alle Dur- und Molltonleitern und -arpeggios zu üben; das Ziel ist, die richtigen Fingersätze aller Tonleitern sozusagen in den Fingern zu verinnerlichen. 

Spielen Sie mit der Spitze des Daumens, nicht mit dem ersten Gelenk.
Das macht den Daumen effektiv so lang wie möglich, was notwendig ist, weil er der kürzeste Finger ist.
Um eine gleichmäßige Tonleiter zu erzeugen, müssen alle Finger so ähnlich wie möglich sein.
Damit Sie mit der Daumenspitze spielen können, müssen Sie das Handgelenk vielleicht ein wenig anheben.
Die Daumenspitze zu benutzen, ist bei hohen Geschwindigkeiten, für eine bessere Kontrolle und wenn man Arpeggios und Akkorde spielt hilfreich.
Mit der Spitze zu spielen, erleichtert den \hyperref[c1iii5b]{Daumenübersatz} und die \enquote{Glissandobewegung}, bei der die Finger von der Bewegungsrichtung der Hand weg zeigen.
Übertreiben Sie die Glissandobewegung nicht, Sie brauchen nur ein wenig davon.

Es ist sehr wichtig, den Daumen zu befreien, indem man den Daumenübersatz und ein sehr flexibles Handgelenk übt.
Außer beim Daumenuntersatz ist der Daumen immer gerade, wird gespielt, indem man das am Handgelenk befindliche Glied dreht, und wird durch Bewegungen der Hand und des Handgelenks in Position gebracht.
Eine von Liszts signifikantesten technischen Verbesserungen geschah, als er lernte, den Daumen korrekt zu verwenden.

\subsubsection{Schnelle chromatische Tonleitern}
\label{c1iii5h}

\textbf{Die chromatische Tonleiter besteht aus Halbtonschritten.
Die wichtigste Überlegung gilt dem Fingersatz, weil es so viele Möglichkeiten dafür gibt.}
Die Standard-Fingersätze für eine aufsteigende Oktave sind - beginnend mit C - 1313123131345 für die rechte Hand und 1313132131321 für die linke Hand (der Fingersatz für die oberen Noten ist für eine Wendung) und jeweils das gleiche rückwärts für eine absteigende Oktave.
Es ist schwierig, diese Fingersätze schnell zu spielen, weil sie aus den kürzest möglichen parallelen Sets aufgebaut sind und deshalb eine große Anzahl Verbindungen enthalten; meistens begrenzen die Verbindungen die Geschwindigkeit. 
Ihr größter Vorteil ist ihre Einfachheit, die sie praktisch auf alle chromatischen Folgen anwendbar macht, egal mit welcher Note man beginnt, und man kann sie sich am leichtesten merken.
Eine Variation davon ist 1212123121234, was ein wenig mehr Geschwindigkeit und Legato ermöglicht und bei großen Händen bequemer ist.

In dem Bestreben, die chromatische Tonleiter zu beschleunigen, wurden verschiedene Folgen mit längeren parallelen Sets erdacht; alle \enquote{akzeptierten} Folgen vermeiden die Benutzung des Daumens bei einer schwarzen Taste.
Die am meisten verwendete ist - beginnend mit E - 123123412312 (Hauer, Czerny, Hanon).
Eine Schwierigkeit mit diesem Fingersatz ist, dass der Anfang der Folge in Abhängigkeit von der ersten Note angepasst werden sollte, um die Geschwindigkeit zu maximieren.
Auch unterscheiden sich die rechte und die linke Hand voneinander; diese Folge benutzt vier parallele Sets.
Man kann sie auf drei parallele Sets verkürzen, indem man - beginnend mit C - 123412312345 spielt.
Mit guter Daumenübersatztechnik mag diese Tonleiter spielbar sein, aber sogar mit Übersatz benutzen wir selten einen Übergang mit 51 oder 15, weil das schwierig ist.
Ohne Frage begrenzt die Einschränkung, den Daumen auf einer schwarzen Taste zu vermeiden, die Wahl des Fingersatzes und macht es komplizierter, weil der Fingersatz von der ersten Note abhängig wird.

\textbf{Wenn wir einen Daumen auf einer schwarzen Taste zulassen, ist eine gute Tonleiter - beginnend mit C -:}

\begin{itemize} 
 \item \textbf{1234,1234,1234; 1234,1234,12345 für 2 aufsteigende Oktaven mit der rechten Hand}
 \item \textbf{5432,1432,1432; 1432,1432,14321 für 2 aufsteigende Oktaven mit der linken Hand}
 \end{itemize}
\textbf{mit dem Daumen bei beiden Händen auf G\# und drei identischen parallelen Sets je Oktave - die einfachste und schnellste mögliche Konfiguration.}
Wieder jeweils das gleiche rückwärts für die absteigenden Oktaven.
Ich nenne das die \textbf{\enquote{vierfingrige chromatische Tonleiter}}; soweit ich weiß, wurde dieser Fingersatz wegen des Daumens auf einer schwarzen Taste gefolgt von einem Passieren des vierten Fingers in der Literatur nicht besprochen.
Zusätzlich zur Geschwindigkeit ist die Einfachheit der größte Vorteil; Sie benutzen denselben Fingersatz, egal wo Sie beginnen (benutzen Sie Finger 3, um mit der rechten Hand auf D zu beginnen), aufsteigend und absteigend, der Fingersatz ist für beide Hände der gleiche, nur umgekehrt, der Daumen und Finger 3 sind synchronisiert, und der Anfang und das Ende sind immer 1,5.  
Mit guter Daumenübersatztechnik ist diese Tonleiter unschlagbar; Sie müssen nur auf das 14 oder 41 achten, wenn die 1 auf dem G\# ist.
Versuchen Sie das beim letzten chromatischen Lauf im Grave von Beethovens Pathetique, und Sie sollten eine merkliche Abnahme der Fehler und schließlich eine deutliche Steigerung der Geschwindigeit feststellen.
Haben Sie das für diesen Lauf gelernt, wird es auch bei jedem anderen chromatischen Lauf funktionieren.
Um einen flüssigen Lauf zu lernen, üben Sie mit dem Schlag auf jeder Note, dann auf jeder zweiten Note, jeder dritten usw.

Obwohl die meisten Übungen nicht hilfreich sind, nimmt das Üben von Tonleitern, Arpeggios und der vierfingrigen chromatischen Tonleiter einen besonderen Platz beim Aneignen der Klaviertechnik ein.
Da man mit ihnen so viele grundlegende technische Fertigkeiten erlernen kann, müssen sie ein Teil des täglichen Lernprogramms eines Klavierspielers sein.


[Ab hier wird der Text noch überarbeitet.]



<!-- c1iii6.html -->

\subsection{Auswendiglernen}
\label{c1iii6} 

\subsubsection{Warum auswendig lernen?}
\label{c1iii6a}

Die Gründe für das Auswendiglernen sind so zwingend, daß es überraschend ist, wie vielen Menschen diese nicht bewußt waren.
\textbf{Fortgeschrittene Pianisten müssen wegen des hohen Grades an technischer Fertigkeit, der erwartet wird, aus dem Gedächtnis spielen.}
Von fast allen Schülern (einschließlich denen, die von sich selber glauben, daß sie schlechte Merkfähigkeiten haben) werden die meisten schwierigen Passagen aus dem Gedächtnis gespielt.
Wer nicht auswendig spielt, muß zwar eventuell zur psychologischen Unterstützung und für einen kleinen Wink ab und zu die Notenblätter vor sich haben, spielt aber in Wirklichkeit schwierige Passagen fast völlig aus dem \enquote{\hyperref[c1iii6d]{Hand-Gedächtnis}}.
\textbf{Wegen dieser Notwendigkeit, aus dem Gedächtnis zu spielen, hat sich das Auswendiglernen zu einem wissenschaftlichen Vorgang entwickelt, der untrennbar mit jedem stichhaltigen Prozeß des Klavierstudiums verbunden ist.}

\textbf{Auswendiglernen ist ein Weg, neue Stücke schnell zu lernen.}
Auf lange Sicht lernen Sie technisch bedeutsame Stücke viel schneller durch Auswendiglernen als durch das Benutzen der Noten.
Auswendiglernen gestattet dem Klavierspieler, irgendwo mitten in einem Stück mit dem Spielen zu beginnen, es ist eine Methode, mit der man über Gedächtnisblockaden und Spielfehler hinwegkommt oder sie sogar völlig vermeidet, und  mit seiner Hilfe entwickelt man ein besseres Verständnis der Musik.
Es erlaubt auszugsweises Spielen (kleine Auszüge aus einer Komposition spielen), eine sehr nützliche Fähigkeit für zwangloses Vorspielen, zum Unterrichten und zum Lernen wie man vorspielt.
Wenn Sie 10 Stunden Repertoire auswendig gelernt haben, was ohne weiteres erreichbar ist, erkennen Sie den Vorteil davon, daß Sie nicht Ihre ganzen Noten mit sich herumtragen müssen oder sie durchsuchen müssen, um ein Musikstück oder einen Auszug zu finden.
Wenn Sie von Auszug zu Auszug springen möchten, wäre die Suche danach in einem Stapel Noten unpraktisch.
Bei Flügeln stört der Notenständer den Ton, d.h. man kann sich selbst nicht spielen hören, wenn der Notenständer aufgestellt ist.
Dieser Effekt ist in einer Konzerthalle oder einem Zuschauerraum mit guter Akustik besonders dramatisch - der Flügel kann praktisch unhörbar werden.
Aber vor allem \textbf{können Sie sich durch das Auswendiglernen zu 100\% auf die Musik konzentrieren.}
Klavierspielen ist eine darstellende Kunst, und ein auswendig gespielter Vortrag ist für das Publikum lohnender, weil es die Fähigkeit zum Auswendigspielen als ein besonderes Talent ansieht.
Ja, wenn Sie auswendig lernen, werden Sie zu einem dieser genialen Künstler, die von \hyperref[memorizer]{Nichtauswendiglernenden} beneidet werden!

Der Gewinn aus diesem Buch vervielfacht sich, weil es ein Komplettpaket ist; d.h. das Ganze ist \textit{viel} größer als die Summe seiner Teile.
Auswendiglernen ist ein gutes Beispiel.
Um das zu verstehen, lassen Sie uns diejenigen Schüler betrachten, die nicht auswendig lernen.
Sobald ein neues Stück \enquote{gelernt} aber noch nicht perfektioniert ist, verlassen diese Schüler üblicherweise das Stück und gehen zum nächsten; teilweise, weil es so lange dauert, neue Stücke zu lernen und teilweise, weil die Noten zu lesen dem Aufführen schwieriger Stücke nicht förderlich ist.
In der Regel lernen Schüler, die nicht auswendig lernen, niemals ein Stück wirklich gut, und dieses Handicap begrenzt die technische Entwicklung.
\textbf{Wenn sie nun in der Lage wären, gleichzeitig schnell zu lernen und auswendig zu lernen, würden sie \textit{für den Rest ihres Lebens} mit den fertigen Stücken Musik machen!}
Wir sprechen nicht nur darüber, ein Stück auswendig zu lernen oder nicht auswendig zu lernen - wir sprechen über einen \textit{lebenslangen} Unterschied in Ihrer Entwicklung als Künstler und darüber, ob Sie wirklich eine Chance haben zu musizieren.
Es ist der Unterschied zwischen einem darstellenden Künstler und einem Schüler, der niemals ein vorführbares Stück hat.
Erst wenn Sie mit einem Stück technisch fertig sind, können Sie überhaupt daran denken, es wirklich musikalisch zu spielen.
Wie schade, daß Schüler, die nicht richtig informiert sind, sich den besten Teil davon entgehen lassen, was es bedeutet, ein Pianist zu sein, und sich die Gelegenheit entgehen lassen, sich als Künstler zu entwickeln.

\textbf{Auswendiglernen nützt der Entwicklung des Gehirns in der Jugend und verlangsamt seinen altersbedingten Verfall.} Das Auswendiglernen von Klaviermusik wird nicht nur Ihr Gedächtnis im täglichen Leben - außerhalb des Klavierspielens - verbessern, sondern auch den Gedächtnisverlust im Alter verlangsamen und sogar die Leistungsfähigkeit des Gehirns für das Auswendiglernen verbessern.
Sie werden Methoden lernen, mit denen man das Gedächtnis verbessern kann, und ein Verständnis der Funktion des menschlichen Gedächtnisses entwickeln.
Sie werden zu einem \enquote{Gedächtnisexperten}, was Ihnen Vertrauen in Ihr Erinnerungsvermögen verleiht; ein Mangel an Selbstvertrauen ist ein wichtiger Grund sowohl für ein schlechtes Gedächtnis als auch für viele andere Probleme, wie z.B. ein geringes Selbstwertgefühl.
Das Gedächtnis beeinflußt die Intelligenz in hohem Maße, und ein gutes Gedächtnis erhöht den effektiven IQ.

In meiner Jugend schien das Leben so kompliziert zu sein, daß ich mich, um es zu vereinfachen, intuitiv dem \enquote{Prinzip des geringsten Wissens} anschloß, welches besagt: \enquote{Je weniger unnötige Information man in sein Gehirn stopft, desto besser.}
Diese Theorie ist der für Plattenspeicher in einem Computer analog: \enquote{Je mehr Müll man löscht, desto mehr Speicher hat man zur Nutzung übrig.}
Ich weiß nun, daß dieser Ansatz Faulheit und einen Minderwertigkeitskomplex, daß man kein guter Auswendiglernender sei, erzeugt und schädlich für das Gehirn ist, weil es so ist, als ob man sagt, daß man um so stärker wird, je weniger Muskeln man benutzt, weil mehr Energie übrig bleibt.
Das Gehirn hat die Kapazität, viel mehr zu speichern als jemand in seinem ganzen Leben hineinstecken könnte.
Wenn man aber nicht lernt, es zu benutzen, wird man nie von seinem ganzen Potential profitieren.
Ich habe durch meinen früheren Fehler viel gelitten.
Ich fürchtete mich davor, zum Bowling zu gehen, weil ich meinen Punktestand nicht so wie die anderen im Kopf behalten konnte.
Seit ich meine Philosophie geändert habe, so daß ich nun versuche alles zu behalten, hat sich mein Leben dramatisch verbessert.
Ich versuche nun sogar, mir die Neigung und die Unebenheiten auf jedem Golfgrün, das ich spiele, zu merken.
Das kann einen großen Effekt auf den Score haben.
Natürlich war der entsprechende Nutzen für meine Laufbahn als Klavierspieler unbeschreiblich.


\label{assoziativ}

\textbf{Das Gedächtnis ist eine assoziative Funktion des Gehirns}.
Bei einer assoziativen Funktion wird ein Objekt mit einem anderen in eine bestimmte Beziehung gesetzt.
Praktisch alles, was wir erleben, wird in unserem Gehirn gespeichert, ob wir das wollen oder nicht.
Wenn das Gehirn diese Informationen vom Kurzzeitgedächtnis in das permanente Gedächtnis überträgt (ein automatischer Vorgang, der gewöhnlich 2 bis 5 Minuten dauert), verbleiben sie dort im Grunde ein Leben lang.
Wenn wir auswendig lernen, ist deshalb das Speichern der Informationen nicht das Problem - das Abrufen der Informationen ist das Problem, weil unser Gedächtnis nicht wie ein Computerspeicher arbeitet, bei dem alle Daten eine Adresse haben, sondern mit einem Verfahren, das wir noch nicht verstehen.
Der Vorgang, den man am besten versteht, ist der assoziative Prozeß: Um uns an \hyperref[johndoe]{Otto}s Telefonnummer zu erinnern, denken wir zuerst an Otto, dann erinnern wir uns daran, daß er verschiedene Telefone besitzt, und dann fällt uns ein, daß seine Mobiltelefonnummer 0xxx-1234567 ist.
Das heißt, die Nummer ist mit dem Mobiltelefon verknüpft, das mit Otto verknüpft ist.
Jede Ziffer der Telefonnummer hat eine umfangreiche Reihe von Verknüpfungen zu unseren Erfahrungen mit Zahlen, angefangen mit den ersten Zahlen, die wir als kleines Kind gelernt haben.
Ohne diese Assoziationen hätten wir keinerlei Vorstellung davon, was Zahlen sind, und wären deshalb nicht in der Lage, uns überhaupt an sie zu erinnern.
\enquote{Otto} hat ebenfalls viele Assoziationen (wie sein Haus, seine Familie usw.), und das Gehirn muß diese alle herausfiltern und nur der Assoziation \enquote{Telefon} folgen, um die Nummer zu finden.
Wegen der großen Leistungsfähigkeit des Gehirns bei der Informationsverarbeitung ist der Abrufprozeß effizienter, wenn mehr Assoziationen existieren, und die Anzahl der Assoziationen wächst schnell, wenn mehr Informationen gespeichert werden, weil sie untereinander vernetzt werden können.
Deshalb ist das menschliche Gedächtnis zum Computerspeicher fast diametral verschieden: Je mehr man auswendig lernt, desto leichter wird es, mehr auswendig zu lernen, weil man mehr Assoziationen erzeugen kann.
Leider nutzen die meisten von uns die Vorteile dieser unglaublich effizienten Methode des Gedächtnisses nicht vollständig aus; wir erzeugen nur eine kleine Zahl von Assoziationen und unser Gehirn filtert diese nicht immer erfolgreich, um bei einer bestimmten gespeicherten Information anzukommen.
Die Kapazität unseres Gedächtnisses ist so groß, daß sie im Grunde unendlich ist.
Sogar gute Auswendiglernende \enquote{sättigen} ihr Gedächtnis nie, bis die Zeichen des Alters anfangen, Ihren Tribut zu fordern.
Wenn mehr Material in das Gedächtnis gestellt wird, erhöht sich die Anzahl der Assoziationen geometrisch.
Dieser geometrische Zuwachs erklärt teilweise den enormen Unterschied in der Speicherkapazität von guten und schlechten Auswendiglernenden.
Somit sagt uns alles, was wir über das Gedächtnis wissen, daß uns Auswendiglernen nur nützlich sein kann.


\subsubsection{Wer kann auswendig lernen, was und wann?}
\label{c1iii6b}

\textbf{Jeder kann lernen auswendig zu lernen, wenn man ihm die richtigen Methoden dafür beibringt.}
Wir zeigen hier, daß man die für das Auswendiglernen erforderliche Zeit auf ein vernachlässigbares Maß reduzieren kann, wenn man das Auswendiglernen mit dem anfänglichen Lernen des Musikstücks kombiniert.
Tatsächlich kann \textbf{die richtige Integration der Verfahren für das Lernen und das Auswendiglernen die für das Lernen erforderliche Zeit reduzieren und dem Auswendiglernen praktisch einen negativen Zeitanteil zuweisen.
Es stellt sich heraus, daß fast alle für das Auswendiglernen erforderlichen Elemente auch erforderliche Elemente für das Lernen sind.
Wenn man diese Prozesse voneinander trennt, muß man am Ende dieselbe Prozedur zweimal durchlaufen.
Niemand würde eine solche Tortur auf sich nehmen (oder zumindest wenige); das erklärt, warum diejenigen, die nicht bereits während des ersten Lernens auswendig lernen, niemals gut auswendig lernen.}

Da Auswendiglernen der schnellste Weg zu lernen ist, sollten Sie jedes lohnende Stück, das sie spielen, auswendig lernen.
Das Auswendiglernen ist ein kostenloses Nebenprodukt des Prozesses, ein neues Musikstück zu lernen.
\textbf{Deshalb sind die Anweisungen für das Auswendiglernen im Prinzip trivial: Befolgen Sie einfach die in diesem Buch angegebenen Lernregeln - mit der zusätzlichen Anforderung, daß Sie während dieser Lernvorgänge alles aus dem Gedächtnis heraus durchführen.}
Lernen Sie z.B., während Sie eine LH-Begleitung Takt für Takt lernen, diese Takte auswendig.
Da ein Takt üblicherweise aus 6 bis 12 Noten besteht, ist es einfach, diesen auswendig zu lernen.
Sie werden dann in Abhängigkeit von der Schwierigkeit diese Abschnitte 10, 100, oder mehr als 1000mal wiederholen müssen, bevor Sie das Stück spielen können - das sind viel mehr Wiederholungen als zum Auswendiglernen benötigt werden.
Sie können gar nicht anders als es auswendig zu lernen!
Warum also eine solch unbezahlbare und einmalige Gelegenheit versäumen?

Wir haben in den Abschnitten I und II gesehen, daß es der Schlüssel zum schnellen Lernen der Technik ist, die Musik auf völlig einfache Teilmengen zu reduzieren; dieselben Prozeduren vereinfachen auch das Auswendiglernen dieser Teilmengen.
Auswendiglernen kann eine enorme Menge an Übungszeit einsparen.
Sie müssen nicht jedesmal auf die Noten sehen, so daß Sie einen RH-Ausschnitt einer Beethoven-Sonate und einen LH-Ausschnitt eines Chopin-Scherzos mit HS üben und beliebig von Ausschnitt zu Ausschnitt springen können.
Sie können sich auf das Lernen der Technik konzentrieren, ohne sich jedesmal durch das Nachsehen der Noten ablenken lassen zu müssen.
Das beste von allem ist, daß die Vielzahl der Wiederholungen, die Sie benötigen, um das Stück zu üben, das Stück ohne zusätzlichen Zeitaufwand und in einer Art und Weise an das Gedächtnis übergibt, die mit keiner anderen Prozedur erreicht wird.
Das sind einige der Gründe, warum das Auswendiglernen vor dem Lernen der einzige Weg ist.

\textbf{Schließlich führt Auswendiglernen, und das ist der \enquote{entscheidende Faktor}, zu \hyperref[c1iii6tastatur]{mentalem Spielen} (s. Abschnitt 1.III.6j), das der Schlüssel zu einem \hyperref[c1iii12]{absoluten Gehör}, einem höheren effektiven IQ, reduzierter \hyperref[c1iii15]{Nervosität} bzw. Streß, zum Komponieren und zur Fähigkeit, mit Leichtigkeit und ohne Fehler vorzuspielen, ist.}
Beim mentalen Spielen können Sie das ganze Stück in Gedanken, ohne Klavier, spielen.
Um auf die Stufe eines Konzertpianisten zu gelangen, müssen Sie das mentale Spielen lernen.
Alle großen Pianisten und Komponisten wurden auf diese Art zu dem, was sie waren.
Praktisch jeder vollendete Klavierspieler komponiert am Ende Musik; Auswendiglernen, absolutes Gehör und mentales Spielen sind entscheidende Elemente für ein erfolgreiches Komponieren.


\subsubsection{Auswendiglernen und Pflege des Gelernten}
\label{c1iii6c}

\textbf{Ein auswendig gelerntes Repertoire erfordert zwei Zeitinvestitionen: die erste für das anfängliche Auswendiglernen des Stücks und eine zweite \enquote{Pflegekomponente}, um das Gedächtnis dauerhafter zu verankern und um etwaige vergessene Abschnitte zu reparieren.}
Während der Lebensspanne eines Pianisten ist die zweite Komponente die bei weitem größere, weil die anfängliche Investition null oder sogar negativ ist.
Die Pflege ist ein Grund, warum einige das Auswendiglernen aufgeben: \enquote{Warum auswendig lernen, wenn ich es sowieso wieder vergesse?}
Die Pflege kann die Größe des Repertoires begrenzen, denn wenn man z.B. fünf bis zehn Stunden Musik auswendig gelernt hat, dann schließen die Erfordernisse der Pflege in Abhängigkeit von der Person eventuell das Auswendiglernen zusätzlicher Stücke aus.
Es gibt mehrere Wege, Ihr Repertoire über diese pflegebedingte Grenze hinaus zu erweitern.
Ein offensichtlicher Weg ist, die auswendig gelernten Stück wegzulegen und sie später erneut auswendig zu lernen, wenn es notwendig ist.
\textbf{Stücke, die hinreichend gut auswendig gelernt wurden, können sehr schnell wieder aufpoliert werden, selbst wenn man sie jahrelang nicht gespielt hat.}
Es ist fast wie Fahrradfahren; hat man erst einmal gelernt, wie man Fahrrad fährt, muß man nie wieder alles erneut lernen.
Wir werden im folgenden verschiedene Pflegeverfahren besprechen, die Ihr auswendig gelerntes Repertoire bedeutend vergrößern können. 

\textbf{Lernen Sie so viele Stücke wie möglich auswendig, bevor Sie 20 Jahre alt sind.
Stücke, die man in diesen frühen Jahren lernt, werden praktisch nie vergessen, und selbst wenn sie vergessen werden, kann man sie sich am leichtesten wieder in Erinnerung rufen.}
Deshalb sollten junge Menschen ermutigt werden, alle Stücke ihres Repertoires auswendig zu lernen.
Stücke, die später als mit 40 Jahren gelernt werden, erfordern einen höheren Aufwand für das Auswendiglernen und die Pflege, obwohl viele Menschen über 60 keine Probleme damit haben, neue Stücke auswendig zu lernen (wenn auch langsamer als vorher).
Beachten Sie das Wort \enquote{lernen} in den vorangegangenen Sätzen; die Stücke müssen nicht unbedingt\footnote{in jüngeren Jahren} auswendig gelernt worden sein, und Sie können sie trotzdem später im Vergleich zu Stücken, die sie im späteren Alter gelernt oder auswendig gelernt haben, mit besserer Merkfähigkeit auswendig lernen.

Es gibt Gelegenheiten, bei denen Sie nicht auswendig lernen müssen, wie z.B. wenn Sie eine große Zahl leichter Stücke, besonders Begleitungen, lernen möchten, bei denen es zu lange dauern würde, sie auswendig zu lernen und zu pflegen.
Eine weitere Gruppe von Musikstücken, die Sie nicht auswendig lernen sollten, sind jene, die Sie zum Üben des Blattspiels benutzen.
\hyperref[c1iii11]{Vom Blatt zu spielen} ist eine gesonderte Fertigkeit, die in Abschnitt III.11 behandelt wird.
\textbf{Jeder sollte ein auswendig gelerntes Repertoire haben, sowie ein vom Blatt zu spielendes Repertoire}, um die Fertigkeit im Blattspiel zu verbessern.

\textbf{Wenn Sie ein Stück gut spielen können, es aber nicht auswendig gelernt haben, kann es sehr frustrierend sein, zu versuchen, das Stück auswendig zu lernen.}
Zu viele Schüler sind aufgrund dieser Schwierigkeiten davon überzeugt, daß Sie schlecht auswendig lernen können.
Das geschieht, weil der Teil der Motivation zum Auswendiglernen, der aus der Zeitersparnis während des ersten Lernens des Stücks resultiert, entfällt, wenn man das Stück bereits mit der vorgegebenen Geschwindigkeit spielen kann.
Die einzige übrigbleibende Motivation ist die Annehmlichkeit, aus dem Gedächtnis vorzuspielen.
Mein Vorschlag an diejenigen, die glauben, sie seien schlechte Auswendiglernende: Lernen Sie ein völlig neues Stück, das Sie nie zuvor studiert haben, indem Sie es von Anfang an mit den Methoden dieses Buchs auswendig lernen.
Sie werden angenehm überrascht sein, wie gut Sie beim Auswendiglernen sind.
Die meisten Fälle von \enquote{schlechtem Gedächtnis} resultieren aus der Lernmethode, nicht aus der Speicherfähigkeit des Gehirns.
Wegen der Wichtigkeit des Themas \enquote{\hyperref[c1iii6l]{Blattspieler und Auswendiglernende}} wird es später noch einmal behandelt.
 

\subsubsection{Hand-Gedächtnis}
\label{c1iii6d}

\textbf{Eine große Komponente Ihres anfänglichen Gedächtnisses wird das Hand-Gedächtnis sein, das vom wiederholten Üben kommt.
Die Hand spielt einfach weiter, ohne daß Sie sich wirklich an jede einzelne Note  erinnern.}
Obwohl wir weiter unten alle bekannten Arten des Gedächtnisses besprechen werden, fangen wir zunächst mit der Analyse des Hand-Gedächtnisses an, weil es früher häufig als die einzige und beste Gedächtnismethode angesehen wurde, obwohl es in Wirklichkeit die unwichtigste ist.
Das Hand-Gedächtnis besteht aus mindestens zwei Komponenten: einer reflexartigen Handbewegung, die aus der Berührung der Tasten resultiert, und einem Reflex im Gehirn auf den Klang des Klaviers.
Beide dienen als Stichwort für Ihre Hand, sich in einer bestimmten vorprogrammierten Weise zu bewegen.
Der Einfachheit halber werden wir sie zusammenfassen und als Hand-Gedächtnis bezeichnen.
Das Hand-Gedächtnis ist nützlich, weil es Ihnen hilft, während des Übens gleichzeitig auswendig zu lernen.
Tatsächlich muß jeder allgemeine Konstrukte - wie Tonleitern, Arpeggios, Alberti-Begleitungen usw. - aus dem Hand-Gedächtnis heraus üben, so daß Ihre Hände sie automatisch spielen können, ohne daß Sie an jede einzelne Note denken müssen.
Deshalb gibt es, wenn Sie anfangen ein neues Stück auswendig zu lernen, keinen Grund, das Hand-Gedächtnis bewußt zu vermeiden.
Einmal erworben, wird man das Hand-Gedächtnis niemals verlieren, und wir zeigen unten, wie man es benutzen kann, um nach einem Hänger den Faden wiederzufinden.

Der biologische Mechanismus, durch den die Hände das Hand-Gedächtnis erwerben, wird nicht so gut verstanden, aber meine Hypothese ist, daß er Nervenzellen außerhalb des bewußten Teils des Gehirns, wie z.B. Nervenzellen im Rückenmark, zusätzlich zum Gehirn einbezieht.
Die Zahl der Nervenzellen außerhalb des Gehirns ist wahrscheinlich der Zahl derer im Gehirn vergleichbar.
Obwohl die Befehle für das Klavierspielen aus dem Gehirn stammen müssen, ist es sehr wahrscheinlich, daß die schnellen Spielreflexe nicht den ganzen Weg zum bewußten Gehirn hinauf zurücklegen.
\textbf{Deshalb muß das Hand-Gedächtnis eine Art Reflex sein, der viele Nervenzelltypen einbezieht.
Als Antwort auf das Spielen der ersten Note spielt der Reflex die zweite Note, was die dritte Note anregt, usw.}
Das erklärt, warum Ihnen das Hand-Gedächtnis nicht dabei hilft, neu zu starten, wenn Sie hängengeblieben sind, solange Sie nicht bis zur ersten Note zurückgehen.
Tatsächlich ist das Neustarten eines Stücks an einer willkürlichen Stelle ein hervorragender Test, ob Sie aus dem Hand-Gedächtnis spielen oder ob Sie für den Notfall noch eine andere Gedächtnismethode haben.
Da es nur eine konditionierte Antwort ist, ist das Hand-Gedächtnis kein wirkliches Gedächtnis und hat zahlreiche schwerwiegende Nachteile.

Wenn wir über das Hand-Gedächtnis sprechen, meinen wir im allgemeinen ein HT-Gedächtnis.
\textbf{Da das Hand-Gedächtnis nur nach vielen Wiederholungen erworben wird, ist es eines der am schwersten zu löschenden oder zu ändernden Gedächtnisse.}
Das ist einer der Hauptgründe für das HS-Üben - zu vermeiden, sich inkorrekte HT-Angewohnheiten anzueignen, die man nur sehr schwer ändern kann.
Das HS-Gedächtnis ist vom HT-Gedächtnis grundlegend verschieden.
Das HS-Spielen ist einfacher und kann direkt vom Gehirn gesteuert werden.
Beim HT-Gedächtnis braucht man eine Art Rückkopplung, um die Hände (und wahrscheinlich die beiden Gehirnhälften) bis zu der Genauigkeit zu koordinieren, die für die Musik erforderlich ist.
Deshalb ist das HS-Üben die effektivste Methode zur Vermeidung der Abhängigkeit vom Hand-Gedächtnis.

Es ist nicht möglich, eine klare Trennlinie zwischen Technik und Gedächtnis zu ziehen.
Ein Klavierspieler mit mehr Technik kann schneller auswendig lernen.
Ein Grund, warum man das Gedächtnis nicht von der Technik trennen kann, ist, daß beides zum Spielen notwendig ist, und solange man nicht spielen kann, kann man weder Technik noch Gedächtnis demonstrieren.
Es gibt deshalb (neben der bloßen Annehmlichkeit Zeit zu sparen usw.) eine tiefere biologische Basis, die der Methode dieses Buchs zugrunde liegt, durch die Gedächtnis und Technik gleichzeitig erworben werden.
 


<!-- c1iii6e.html -->

\subsubsection{Wie fängt man an?}
\label{c1iii6e}

\textbf{Es steht außer Frage, daß es der einzig wirklich effektive Weg zum Auswendiglernen ist, die Musiktheorie zu kennen und beim Auswendiglernen eine detaillierte musikalische Analyse und ein tiefes Verständnis der Musik zu benutzen.}
Mit dieser Art von Gedächtnis werden Sie in der Lage sein, den ganzen Notensatz aus dem Gedächtnis aufzuschreiben, d.h. Sie sollten das Stück in Gedanken - ohne Klavier - spielen können.
Die meisten Schüler haben jedoch keine derart fortgeschrittene Ausbildung.
Deshalb beschreiben wir hier einige allgemeine Verfahren für das Auswendiglernen, die nicht von einer umfangreichen Ausbildung in Musiktheorie abhängig sind, und mit denen wir trotzdem das \enquote{mentale Spielen} lernen können.

Lassen Sie mich die Wichtigkeit verdeutlichen, das zu verstehen, was man auswendig lernen möchte.
Wenn wir 50 in einer ausgewählten Reihenfolge angeordnete Buchstaben des Alphabets auswendig lernen sollten, würden die meisten von uns nach einer Weile aufgeben.
Wenn es uns gelingen würde, dann könnten sich die meisten nach 20 Jahren nicht mehr daran erinnern.
Aber überraschenderweise tun wir das alle unser ganzes Leben lang!
Die meisten von uns kennen das erste der Zehn Gebote (oder längere Sätze) - wir haben tatsächlich 60 Buchstaben in der richtigen Reihenfolge gelernt (die Satzzeichen und Leerstellen nicht mitgerechnet)\footnote{wobei ich von der Version \enquote{Ich bin der Herr, Dein Gott. Du sollst keine anderen Götter haben neben mir.} ausgegangen bin}.
Und wir werden sie wahrscheinlich für den Rest unseres Lebens nicht mehr vergessen.
Sie mögen nun sagen: \enquote{Ja, aber die Buchstaben in den Zehn Geboten sind nicht in einer zufälligen Reihenfolge angeordnet!}
Das sind die Noten in einem Musikstück aber auch nicht.
Deshalb ist es für jeden von uns einfach, große Mengen Material auswendig zu lernen, wenn wir lernen können, dieses Material mit Dingen zu assoziieren, die wir verstehen oder mögen, wie z.B. Musik.
Das Publikum ist oft völlig erstaunt darüber, welche unglaublichen Mengen an Musik die Musiker auswendig lernen können, da ihnen die Leistungsfähigkeit des assoziativen Gedächtnisprozesses nicht bewußt ist.
\textbf{Somit kann das Wissen über die Musik und die Musiktheorie einen großen Unterschied darin ausmachen, wie schnell und wie gut jemand auswendig lernen kann.}
Die Musiktheorie ist nützlich aber nicht notwendig, da die Musik uns auf ihre eigene Weise ansprechen kann - jeder, der Musik mag, \enquote{spricht bereits die Sprache der Musik}.

Beginnen Sie mit dem Auswendiglernen, indem Sie einfach die Anweisungen in den \hyperref[c1i1]{Abschnitten I} und \hyperref[c1ii1]{II} befolgen und dabei jeden Abschnitt auswendig lernen, bevor Sie anfangen ihn zu üben.
\textbf{Der beste Test Ihres Gedächtnisses ist, diesen Abschnitt in Gedanken - ohne das Klavier - zu spielen.}
Wir werden dieses entscheidend wichtige Konzept das ganze Buch hindurch wiederholt aufgreifen.
Vergewissern Sie sich deshalb nach jedem Schritt des Auswendiglernens, daß sie ihn in Gedanken spielen können.
Ein gutes Gedächtnis können Sie nur erreichen, wenn Sie von Anfang an auswendig lernen; ebenso werden Sie nicht in der Lage sein, das Stück in Gedanken zu spielen, wenn Sie nicht beim ersten Auswendiglernen damit beginnen.

Wie gut Sie ein Stück verstehen und sich daran erinnern, hängt von der Geschwindigkeit ab.
Wenn man schneller spielt, tendiert man dazu, sich auf höheren Abstraktionsstufen an die Musik zu erinnern.
Beim sehr langsamen Spielen muß man sich Note für Note daran erinnern; in unserem Beispiel mit den Zehn Geboten muß man sich bei \enquote{niedriger Geschwindigkeit} Buchstaben für Buchstaben daran erinnern.
Bei höheren Geschwindigkeiten werden Sie in musikalischen Phrasen denken (Worten bei den Zehn Geboten).
Bei noch höheren Geschwindigkeiten denken Sie vielleicht in Beziehungen zwischen Phrasen oder ganzen musikalischen Konzepten (allmächtiger Gott und falsche Götter).
Diese Konzepte einer höheren Ebene kann man sich stets leichter merken.
Deshalb werden Sie, während Sie die Geschwindigkeit ändern, sehr unterschiedliche Zustände des Gedächtnisses durchlaufen und völlig neue Assoziationen erzeugen.

Während des HS-Übens kann man zu höheren Geschwindigkeiten gelangen als mit HT, was den Verstand dazu zwingt, die Musik in einem anderen Licht zu betrachten.
Die Musik aus vielen Blickwinkeln auswendig zu lernen ist notwendig, um gut auswendig zu lernen; deshalb hilft das Üben mit verschiedenen Geschwindigkeiten dem Gedächtnis enorm.
Es ist im allgemeinen einfacher, schnell auswendig zu lernen als langsam, da auf höheren Abstraktionsstufen weniger Konzepte bzw. Assoziationen notwendig sind.
Deshalb sollten Sie, wenn Sie ein neues Stück anfangen und es nur langsam spielen können, nicht befürchten, daß Sie Schwierigkeiten damit haben werden, es auswendig zu lernen.
Wenn Sie schneller werden, wird es einfacher, es auswendig zu lernen.
Das erklärt, warum das schnelle Hochschrauben der Geschwindigkeit mittels HS-Üben der schnellste Weg zum Auswendiglernen ist.
Viele Schüler werden instinktiv langsamer, wenn sie beim Auswendiglernen auf Schwierigkeiten stoßen; in Wahrheit ist es einfacher, auswendig zu lernen, wenn man schneller wird.
Sie müssen sich nur daran erinnern, daß Sie auch die Stufe des assoziativen Gedächtnisses ändern müssen, wenn Sie schneller werden.
Schüler, die niemals die Stufe des assoziativen Gedächtnisses bewußt geändert haben, werden es zunächst vielleicht schwierig finden, aber das ist etwas, das alle Klavierspieler lernen müssen.

\textbf{Sogar wenn Sie einen bestimmten Abschnitt leicht HT spielen können, sollten Sie ihn HS auswendig lernen}, da wir dieses später brauchen.
Das ist einer der wenigen Fälle, in denen die Prozeduren für das Auswendiglernen und das Lernen voneinander abweichen.
Wenn Sie einen Abschnitt leicht HT spielen können, müssen Sie ihn für die Technik nicht HS üben.
Wenn Sie das Stück aufführen möchten, müssen Sie es sich jedoch HS einprägen, weil Sie das für das Weiterspielen nach einem Hänger, für die Pflege usw. brauchen werden.
Diese Regel ist z.B. auf viele Stücke von Bach und Mozart anwendbar, welche oft technisch einfach aber schwer auswendig zu lernen sind.
Kompositionen dieser Komponisten sind gelegentlich schwieriger mit HS auswendig zu lernen, weil die Noten häufig keinen Sinn ergeben, wenn die Hände voneinander getrennt sind.
Genau deshalb ist das HS-Gedächtnis notwendig - es zeigt, wie tückisch die Musik sein kann, wenn man das Stück nicht vorher mit HS durchgearbeitet hat.
Wenn Sie das Gedächtnis prüfen (z.B. indem Sie versuchen, irgendwo in der Mitte mit dem Spielen anzufangen), werden Sie oft feststellen, daß Sie es nicht können, solange Sie das Stück nicht HS auswendig gelernt haben.
Wir beschreiben weiter unten, wie man die Musik in Gedanken, ohne Klavier, als Teil des Prozesses auswendig zu lernen \enquote{spielt}; das ist ebenfalls mit HS viel einfacher als mit HT, weil der Geist sich nicht auf zwei Dinge gleichzeitig konzentrieren kann.

\textbf{Das Gedächtnis ist ein assoziativer Vorgang; deshalb gibt es nichts hilfreicheres als Ihren eigenen Einfallsreichtum beim Erzeugen von Assoziationen.}
Bis hierhin haben wir gesehen, daß HS, HT und das Spielen mit unterschiedlichen Geschwindigkeiten Elemente sind, die Sie in diesem assoziativen Vorgang kombinieren können.
Jedes Musikstück, das Sie auswendig lernen, wird Ihnen zukünftig dabei helfen, Musikstücke auswendig zu lernen.
Die Funktion des Gedächtnisses ist außerordentlich komplex; seine komplexe Natur ist der Grund, warum intelligente Leute oft auch gute Auswendiglernende sind.
Ihnen fallen schnell nützliche Assoziationen ein.
\textbf{Durch das Auswendiglernen mit HS fügen Sie zwei weitere assoziative Verfahren (RH und LH) mit einem viel einfacheren Aufbau als HT hinzu.}
Haben Sie erst einmal eine Seite oder mehr auswendig gelernt, teilen Sie diese in logische kleinere musikalische Phrasen von ungefähr 10 Takten auf und beginnen Sie, diese Phrasen in zufälliger Reihenfolge zu spielen; d.h. üben Sie die Kunst, mit dem Spielen an einer beliebigen Stelle im Stück anzufangen.
Wenn Sie die Methoden dieses Buchs benutzt haben, um dieses Stück zu lernen, dann sollte es leicht sein, irgendwo anzufangen, weil Sie es in kleinen Abschnitten gelernt haben.
\textbf{Es ist wirklich ein tolles Gefühl, in der Lage zu sein, ein Stück ab einer beliebigen Stelle zu spielen, und diese Fertigkeit hört nie auf, das Publikum zu verblüffen.}
Ein weiterer nützlicher Trick beim Auswendiglernen ist, mit einer Hand zu spielen und sich gleichzeitig die andere Hand in Gedanken vorzustellen.
Wenn Sie das können, dann haben Sie das Stück sehr gut auswendig gelernt!
Es gibt aber noch mehr.
Diese Übungsmethode gestattet Ihnen nur, mit dem Anfang der Abschnitte, die Sie geübt haben, zu beginnen - wir werden im folgenden das mentale Spielen benutzen, um an einer beliebigen Stelle einer Phrase mit dem Spielen beginnen zu können.

Wenn man etwas auswendig lernt, wird es zunächst im temporären oder Kurzzeitgedächtnis gespeichert.
Es dauert ungefähr 2 bis 5 Minuten, bis diese Erinnerungen in das Langzeitgedächtnis übertragen werden (wenn sie überhaupt übertragen werden).
Das wurde unzählige Male durch Tests mit Traumaopfern bestätigt: sie können sich nur an das erinnern, was mindestens 2 bis 5 Minuten vor dem traumatischen Ereignis geschah.
Nach der Übertragung in das Langzeitgedächtnis verringert sich die Fähigkeit, die Erinnerung abzurufen, schrittweise, es sei denn, es erfolgt eine Wiederauffrischung.
Wenn man eine Passage viele Male innerhalb einer Minute wiederholt, erwirbt man Hand-Gedächtnis und Technik, aber das ganze Gedächtnis wird nicht proportional zur Anzahl der Wiederholungen aufgefrischt.
\textbf{Für das Auswendiglernen ist es besser, 2 bis 5 Minuten zu warten und dann erneut auswendig zu lernen.
Das ist ein Grund, warum man während einer Übungseinheit mehrere Dinge auf einmal auswendig lernen sollte.}
Konzentrieren Sie sich deshalb nicht nur lange Zeit auf eine Sache, in dem Glauben, daß mehr Wiederholungen zu einem besseren Gedächtnis führen.
 
Lernen Sie Phrasen oder Gruppen von Noten auswendig; versuchen Sie nie, sich jede Note zu merken.
Je schneller Sie spielen, desto leichter ist das Auswendiglernen, weil Sie die Phrasen und die Struktur bei höherer Geschwindigkeit leichter sehen können.
Deshalb ist das Auswendiglernen mit HS so effektiv.
Viele schlechte Auswendiglernende werden instinktiv langsamer und versuchen am Ende einzelne Noten auswendig zu lernen, wenn sie auf Schwierigkeiten stoßen.
Das ist genau das Falsche.
Schlechte Auswendiglernende können nicht deshalb nicht auswendig lernen, weil ihr Gedächtnis nicht gut wäre, sondern weil sie nicht wissen, wie man auswendig lernt.
\textbf{\textit{Ein Grund für schlechtes Auswendiglernen ist, durcheinander zu geraten.}}
Deshalb ist auswendig lernen mit HT keine gute Idee; man kann nicht so schnell spielen wie mit HS, und es gibt mehr Material, das Verwirrung stiften kann.
Gute Auswendiglernende verfügen über Methoden, ihr Material zu organisieren, so daß es nicht verwirrend ist.
Merken Sie sich die musikalischen Themen und wie diese sich entwickeln oder die Grundstruktur, die ausgebaut wird, um die endgültige Musik zu erzeugen.
Langsames Üben ist gut für das Auswendiglernen - nicht weil das Auswendiglernen beim langsamen Spielen einfacher wäre, sondern weil es ein schwieriger Test dafür ist, wie gut man auswendig gelernt hat.


\subsubsection{Auffrischung des Gedächtnisses}
\label{c1iii6f}

Eines der nützlichsten Mittel für das Gedächtnis ist das Wiederauffrischen.
\textbf{Eine vergessene Erinnerung, die wiedererlangt wird, wird stets besser erinnert.}
Viele Menschen sind darüber beunruhigt, daß sie vergessen.
Der Trick ist, aus der Not eine Tugend zu machen, d.h. wenden Sie auf sich selbst \hyperref[reversepsychology]{umgekehrte Psychologie} an!
Die meisten Menschen müssen etwas vergessen und es drei- oder viermal erneut auswendig lernen, bevor es dauerhaft erinnert wird.
Um die Frustrationen durch das Vergessen und Wiederauffrischen des Gedächtnisses zu eliminieren, versuchen Sie mit Absicht zu vergessen, z.B. indem Sie ein Stück für eine Woche oder länger nicht spielen und es dann erneut lernen.
Oder hören Sie auf, bevor Sie es komplett auswendig gelernt haben, so daß Sie das nächste Mal wieder von vorne anfangen müssen.
Oder anstatt kurze Abschnitte zu wiederholen (die Methode, die Sie anfänglich benutzt haben, um das Stück auswendig zu lernen), spielen Sie das ganze Stück, aber nur einmal am Tag oder mehrere Male am Tag aber mit mehreren Stunden dazwischen.
Finden Sie Wege um zu vergessen (wie mehrere Dinge gleichzeitig auswendig zu lernen); versuchen Sie, künstliche Hänger zu erzeugen - halten Sie mitten in einer Phrase an und versuchen Sie weiterzumachen.

\textbf{Neues Material auswendig zu lernen führt oft dazu, daß Sie vergessen, was Sie sich vorher eingeprägt haben.}
Deshalb ist es nicht effizient, viel Zeit auf das Auswendiglernen eines kleinen Abschnitts zu verwenden.
Wenn man die richtige Anzahl Dinge zum Auswendiglernen wählt, kann man das eine benutzen, um das \enquote{Vergessen} des anderen damit zu steuern, so daß man es für ein besseres Behalten erneut auswendig lernen kann.
Das ist ein Beispiel dafür, wie erfahrene Auswendiglernende ihre Abläufe zum Auswendiglernen feinabstimmen können.

Die Frustration und die Furcht zu vergessen können wie die Furcht vor dem Ertrinken behandelt werden.
Menschen, die nicht schwimmen können, fürchten sich davor, zu sinken und zu ertrinken.
Man kann diese Furcht oftmals mittels Psychologie kurieren.
Sagen Sie ihnen zunächst, sie sollen einen tiefen Atemzug machen und die Luft anhalten.
Halten Sie sie dann waagerecht mit dem Gesicht nach unten aufs Wasser, mit ihrem Gesicht und den Füßen im Wasser.
Bleiben Sie nahe bei ihnen, und halten Sie sie fest, so daß sie sich sicher fühlen (einen Schnorchel zu benutzen ist hilfreich, weil sie dann nicht den Atem anhalten müssen).
Sagen Sie ihnen dann, sie sollen untertauchen, und lassen Sie los. Sie werden erkennen, daß sie nicht tauchen können, weil der Körper dazu neigt zu treiben!
Das funktioniert in Salzwasser am besten, weil das Tauchen in einem Süßwasserbecken leichter ist.
Das Wissen, daß sie nicht sinken können, wird sie auf den langen Weg zur Abschwächung ihrer Angst zu ertrinken führen.
Genauso werden Sie beim Versuch zu vergessen entdecken, daß es gar nicht so einfach ist zu vergessen, und eigentlich glücklich sein, wenn Sie wirklich vergessen, so daß Sie den Prozeß des erneuten Lernens öfter durchlaufen können, um das Gedächtnis wieder aufzufrischen.
\textbf{Die Frustration, die durch den natürlichen Prozeß des Vergessens verursacht wird, zu eliminieren, kann Sie beruhigen und dem Auswendiglernen förderlich sein.}
Wir beschreiben nun weitere Methoden für das Wiederauffrischen und des Einspeicherns in das Gedächtnis.
 

\subsubsection{Kaltstart}
\label{c1iii6g}

\textbf{Üben Sie, auswendig gelernte Stücke \enquote{kalt} (ohne Ihre Hände aufzuwärmen) zu spielen;} das ist offensichtlich schwieriger als mit aufgewärmten Händen, aber unter ungünstigen Bedingungen zu üben ist eine Möglichkeit, Ihre Fähigkeit zum Vorspielen vor Publikum zu stärken.
Diese Fähigkeit, sich einfach hinzusetzen und kalt zu spielen, mit einem ungewohnten Klavier oder in einer ungewohnten Umgebung oder nur mehrmals am Tag, wenn Sie ein paar Minuten übrig haben, ist einer der nützlichsten Vorteile davon, Stücke auswendig zu lernen.
Und Sie können dies überall tun, außerhalb Ihres Zuhauses, wenn Ihre Noten nicht zur Verfügung stehen.
Kalt zu spielen bereitet Sie darauf vor, in einer Gruppe zu spielen usw., ohne 15 Minuten lang Hanon spielen zu müssen, bevor Sie auftreten können.
Kalt zu spielen ist eine Fähigkeit, die erstaunlich leicht zu entwickeln ist, obwohl das zunächst fast unmöglich erscheinen mag.
Das ist ein guter Zeitpunkt, die Passagen zu finden, die zu schwierig sind, um sie mit kalten Händen zu spielen, und zu üben, wie man schwierige Abschnitte verlangsamt oder vereinfacht.
Wenn Sie einen Fehler machen oder hängen bleiben, hören Sie nicht auf und gehen wieder zurück, sondern versuchen Sie, zumindest den Rhythmus oder die Melodie durchzuhalten, und spielen Sie geradewegs durch den Fehler hindurch.

\textbf{Die ersten paar Takte, sogar der einfachsten Stücke, sind oftmals kalt schwer anzufangen} und werden wahrscheinlich zusätzliche Übung erfordern, sogar wenn sie gut auswendig gelernt wurden.
Oftmals ist es bei technisch schwierigen Anfängen leichter zu beginnen.
Lassen Sie sich also von scheinbar leichter Musik nicht aufs Glatteis führen.
Es ist eindeutig wichtig, die Anfänge aller Stücke kalt zu üben.
Natürlich sollten Sie nicht immer mit dem Anfang beginnen; ein weiterer Vorteil des Auswendiglernens ist, daß Sie kleine Auszüge, die irgendwo aus dem Stück stammen, spielen können, und Sie sollten immer üben, Auszüge zu spielen (s. Abschnitt III.14 zur \enquote{\hyperref[c1iii14]{Vorbereitung auf Auftritte und Konzerte}}).
Es ist natürlich ratsam, den Anfang gut auswendig zu lernen.
Welche Tonart und welche Taktart hat das Stück?
Welches ist die erste Note, und welche absolute Tonhöhe hat sie?
 

\subsubsection{Langsam spielen}
\label{c1iii6h}

\textbf{Der allerwichtigste Weg zum Wiederauffrischen des Gedächtnisses ist langsames Spielen, \textit{sehr} langsames Spielen, mit weniger als der halben Geschwindigkeit.}
Die langsame Geschwindigkeit wird auch benutzt, um die Abhängigkeit vom \hyperref[c1iii6d]{Hand-Gedächtnis} zu reduzieren und es durch ein \enquote{\hyperref[c1iii6tastatur]{echtes Gedächtnis}} (s.u.) zu ersetzen, weil der Reiz für den Abruf des Hand-Gedächtnisses verändert und reduziert wird, wenn man langsam spielt.
Die Stimulation durch den Klavierklang ist ebenfalls wesentlich verändert.
Der größte Nachteil des langsamen Spielens ist, daß es so viel Ihrer Übungszeit einnimmt; wenn Sie doppelt so schnell spielen können, üben Sie das Stück in der gleichen Zeit zweimal so oft.
Warum also langsam spielen?
Außerdem kann es schrecklich langweilig werden.
Warum etwas üben, das man nicht braucht, wenn man mit voller Geschwindigkeit spielt?
Man muß wirklich gute Gründe haben, um das sehr langsame Üben zu rechtfertigen.
Damit sich das langsame Spielen auszahlt, versuchen Sie so viele Dinge wie möglich mit Ihrem langsamen Spielen zu kombinieren, so daß es keine Zeit verschwendet.
Einfach langsam zu spielen, ohne wohldefinierte Ziele, \textit{ist} Zeitverschwendung; Sie müssen mehrere Vorzüge gleichzeitig suchen, und dazu wissen, welche es sind. Lassen Sie uns deshalb einige davon auflisten:

\begin{enumerate}[label={\arabic*.}] 
\item Langsames Spielen ist überraschend nützlich für gute Technik, besonders für das Üben der Entspannung.
\item 
Langsames Spielen frischt Ihr Gedächtnis wieder auf, weil Zeit dafür vorhanden ist, daß die Spielsignale mehrere Male von Ihren Fingern zum Gehirn und zurück wandern, bevor nachfolgende Noten gespielt werden.
Wenn Sie nur mit der vorgegebenen Geschwindigkeit üben würden, könnten Sie das Hand-Gedächtnis wieder auffrischen und das wahre Gedächtnis verlieren.
\item Langsames Spielen gestattet es Ihnen, zu üben, der Musik, die Sie gerade spielen, in Gedanken vorauszugehen (nächster Abschnitt).
Das verleiht Ihnen mehr Kontrolle über das Stück und kann Ihnen sogar gestatten, drohende Spielfehler vorauszusehen.
Das ist der Zeitpunkt, an Ihren \hyperref[c1iii7f]{Sprüngen} und \hyperref[c1iii7e]{Akkorden} zu arbeiten (Abschnitte III.7e, f).
Seien Sie immer mindestens einen Sekundenbruchteil voraus, und üben Sie, die Noten vor dem Spielen zu fühlen, um eine hundertprozentige Genauigkeit zu garantieren.
Als generelle Regel gilt: Seien Sie ungefähr einen Takt voraus - mehr dazu später.
\item Langsames Spielen ist einer der besten Wege, um Ihre Hand von schlechten Angewohnheiten zu befreien, besonders von jenen, die Sie unbewußt während des schnellen Übens angenommen haben (\hyperref[fpd]{FPD}).
FPD ist größtenteils ein reflexartiges Hand-Gedächtnis, das das Gehirn umgeht; deshalb ist man sich der schlechten Angewohnheiten im allgemeinen nicht bewußt.
\item Sie haben nun während des Spielens Zeit, die Details der Struktur des Stückes zu analysieren und Ihre Aufmerksamkeit auf alle Ausdrucksbezeichnungen zu richten.
Konzentrieren Sie sich vor allem auf das Erzeugen der Musik.
\item Eine der Hauptursachen von Gedächtnisblockaden und Spielfehlern während des Vorspielens ist, daß das Gehirn viel schneller als gewöhnlich arbeitet, und man kann während einer Aufführung in der gleichen Zeit zwischen den Noten mehr \enquote{denken} als während des Übens.
Dieses zusätzliche Denken führt neue Variablen ein, die das Gehirn durcheinander bringen, was Sie in unbekanntes Gebiet führt und Ihren Rhythmus unterbrechen kann - das ist während einer Aufführung besonders lästig.
Üben Sie deshalb während des langsamen Übens, zwischen den Noten zusätzliche Gedanken einzufügen.
Was sind die vorangegangenen und folgenden Noten?
Sind diese genau richtig oder kann ich sie verbessern?
Was mache ich an dieser Stelle, wenn ich einen Fehler mache?
Usw., usw.
Denken Sie sich Gedanken aus, die während einer Aufführung typisch sind.
Sie können die Fähigkeit entwickeln, sich geistig von den einzelnen Noten zu lösen, die Sie gerade spielen, und in Gedanken an einer anderen Stelle durch die Musik zu wandern, während Sie einen bestimmten Abschnitt spielen.
 \end{enumerate}
Wenn Sie alle obigen Ziele kombinieren, lohnt sich die Zeit wirklich, die mit dem langsamen Spielen verbracht wird, und alle diese Ziele gleichzeitig zu verwirklichen wird eine Herausforderung sein, die keinen Raum für Langeweile läßt.
 

\subsubsection{Vorausschauend spielen}
\label{c1iii6i}

\textbf{Wenn man aus dem Gedächtnis spielt, muß man in Gedanken dem was man spielt stets voraus sein, so daß man vorausplanen, die völlige Kontrolle haben, Schwierigkeiten vorausahnen und sich veränderten Bedingungen anpassen kann.}
Man kann z.B. einen Spielfehler oft kommen sehen und einen der Tricks benutzen, die in diesem Buch besprochen werden (sehen Sie dazu in Abschnitt III.9 \hyperref[c1iii9]{wie man ein Stück auf Hochglanz bringt}), um darüber hinweg zu kommen.
Sie werden diesen Spielfehler nicht kommen sehen, sofern Sie nicht vorausdenken.
Ein Weg, das Vorausdenken zu üben, ist, schnell zu spielen und dann langsamer zu werden.
Durch das schnelle Spielen zwingen Sie Ihr Gehirn, schneller zu denken, so daß Sie, wenn Sie langsamer werden, automatisch der Musik voraus sind.
Sie können nicht vorausdenken, solange die Musik nicht gut auswendig gelernt ist; somit testet und verbessert das Vorausdenken wirklich das Gedächtnis.

Sie können auf mehreren Ebenen der Komplexität vorausdenken.
Sie können eine Note vorausdenken, wenn Sie sehr langsam spielen.
Bei höheren Geschwindigkeiten müssen Sie eventuell in Takten oder Phrasen denken.
Sie können auch in Themen, musikalischen Ideen, verschiedenen Stimmen oder Akkordübergängen denken.
Das sind verschiedene Assoziationen, die Ihnen beim Auswendiglernen helfen werden.

Der beste Weg, sehr schnell zu spielen, ist natürlich HS.
Das ist ein weiteres wertvolles Nebenprodukt des HS-Übens; Sie werden zunächst überrascht sein, welche Auswirkungen wirklich schnelles Spielen auf Ihr Gehirn hat.
Es ist eine völlig neue Erfahrung.
Da man richtig schnell werden muß, um gegen das Gehirn zu gewinnen, sind solche Geschwindigkeiten mit HT nicht leicht zu erreichen.
Ein solch schnelles Spielen ist eine gute Möglichkeit, das Gehirn so zu beschleunigen, daß es vorausdenken kann.



<!-- c1iii6j.html -->

\subsubsection{Langzeitgedächtnis aufbauen}
\label{c1iii6j} 

\textbf{Es gibt mindestens fünf grundlegende Arten von Gedächtnis:}

\begin{enumerate}[label={\arabic*.}] 
\item \textbf{Hand-Gedächtnis (hören, fühlen)}
\item \textbf{Musik-Gedächtnis (hören)}
\item \textbf{fotografisches Gedächtnis (sehen)}
\item \textbf{Tastatur-Gedächtnis und mentales Spielen (sehen, fühlen, Gehirn)}
\item \textbf{theoretisches Gedächtnis (Gehirn)}
 \end{enumerate}
Praktisch jeder benutzt eine Kombination davon.
Die meisten verlassen sich hauptsächlich auf eine dieser Arten und benutzen die anderen zur Ergänzung.
\label{c1iii6hand}
Wir haben \hyperref[c1iii6d]{oben} bereits das \textbf{Hand-Gedächtnis} besprochen.
Es wird durch bloßes Wiederholen \enquote{bis die Musik in der Hand ist} erworben.
Bei der intuitiven Lehrmethode wurde das als bester Weg zum Auswendiglernen angesehen, da man bessere Methoden nicht kannte.
Wir wollen es nun durch die anderen Gedächtnisverfahren ersetzen, um ein dauerhafteres und verläßlicheres Gedächtnis aufzubauen.


\label{c1iii6musik}

Das \textbf{Musik-Gedächtnis} basiert auf der Musik, d.h. der Melodie, dem Rhythmus, dem Ausdruck, den Emotionen usw.
Dieser Ansatz funktioniert bei emotionalen und musikalischen Menschen, die mit der Musik starke Gefühle assoziieren, am besten.
Diejenigen mit \hyperref[c1iii12]{absolutem Gehör} werden damit ebenfalls Erfolge erzielen, weil sie die Noten auf dem Klavier einfach nach ihrer Erinnerung der Musik finden können.
Menschen, die gerne komponieren, neigen ebenfalls dazu, dieses Gedächtnisverfahren zu benutzen.
Musiker haben meistens nicht automatisch ein gutes musikalisches Gedächtnis.
Es hängt von der Art ihres Gehirns ab, und man kann diese Fähigkeit trainieren, wie es unten in Abschnitt \hyperref[c1iii6m]{III.6m} besprochen wird.
Zum Beispiel können sich Menschen mit gutem Musik-Gedächtnis auch an andere Dinge erinnern, wie den Namen des Komponisten und der Komposition.
Bei den meisten Kompositionen, die sie ein paar mal gehört haben, können sie die Melodien gut abrufen, so daß sie das Lied summen können, wenn man ihnen den Titel nennt.


\label{c1iii6foto}

Beim \textbf{fotografischen Gedächtnis} merken Sie sich die ganzen Notenblätter, stellen sie sich bildlich vor und lesen sie in Gedanken.
Selbst diejenigen, die glauben, sie hätten kein fotografisches Gedächtnis, können es sich aneignen, wenn sie das fotografische Gedächtnis \textit{von Anfang an} routinemäßig trainieren, wenn sie das Stück üben.
Wenn sie das Verfahren ab dem ersten Tag (wenn sie mit dem Stück beginnen) gewissenhaft anwenden, werden viele Menschen feststellen, daß es im Durchschnitt je Seite nur ein paar Takte gibt, die noch nicht fotografisch abgespeichert sind, wenn sie das Stück zufriedenstellend spielen können.
Ein Weg, fotografisch auswendig zu lernen, ist, die hier umrissenen Methoden für Technik und Gedächtnis genau zu befolgen, aber auch gleichzeitig das Notenblatt Hand für Hand, Takt für Takt und Abschnitt für Abschnitt fotografisch auswendig zu lernen.

Eine andere Möglichkeit, zu einem fotografischen Gedächtnis zu gelangen, ist, sich zunächst den allgemeinen Aufbau einzuprägen, z.B. wie viele Zeilen es auf der Seite gibt und wie viele Takte je Zeile, danach die Noten jedes Takts, dann die Ausdrucksbezeichnungen usw.
Fangen Sie mit den groben Zügen an, und ergänzen Sie schrittweise die Details.
Beginnen Sie das fotografische Gedächtnis mit dem Einprägen jeweils einer Hand.
Sie müssen wirklich ein genaues Foto der Seite anfertigen, komplett mit ihren Fehlern und zusätzlichen Markierungen.
Wenn Sie Schwierigkeiten haben, sich bestimmte Takte zu merken, zeichnen Sie etwas ungewöhnliches dort hin, wie ein lachendes Gesicht (\enquote{Smiley}) oder Ihre eigenen Markierungen, die Ihr Gedächtnis wachrütteln.
Wenn Sie sich das nächste Mal an diesen Abschnitt erinnern möchten, denken Sie zuerst an das lachende Gesicht.

Ein Vorteil des fotografischen Gedächtnisses ist, daß man ohne das Klavier an dem Stück arbeiten kann, jederzeit, überall.
Tatsächlich muß man sich das Stück, wenn man es sich erst einmal angeeignet hat, abseits des Klaviers in Gedanken so oft wie möglich vorstellen, bis es dauerhaft gespeichert ist.
Ein weiterer Vorteil ist, daß man, wenn man beim Spielen eines Stücks in der Mitte stecken bleibt, leicht wieder anfangen kann, indem man diesen Abschnitt in Gedanken liest.
Das fotografische Gedächtnis gestattet es Ihnen auch, vorauszulesen während Sie spielen, was Ihnen dabei hilft vorauszudenken.
Ein weiterer Vorteil ist, daß es Ihnen beim Spielen vom Blatt helfen wird.

Der Hauptnachteil ist, daß die meisten Menschen die fotografische Erinnerung nicht für lange Zeiträume aufrecht halten können, weil die Pflege dieser Art von Gedächtnis üblicherweise mehr Arbeit erfordert als andere Methoden.
Im Gegensatz zu den meisten anderen Methoden erhält sich das fotografische Gedächtnis ohne einen zusätzlichen Aufwand nicht selbst.
Ein weiterer Nachteil ist, daß es ein vergleichsweise langsamer geistiger Prozeß ist, sich die Noten in Gedanken vorzustellen und zu lesen, der mit dem Spielen in Konflikt geraten kann.
\textbf{Deshalb ist das fotografische Gedächtnis für die meisten Menschen nicht das praktikabelste Gedächtnisverfahren.}
Es eignet sich nur für diejenigen, die bereits ein gutes fotografisches Gedächtnis haben und denen es Spaß macht, es weiterzuentwickeln.

Ich arbeite nicht bewußt für das fotografische Gedächtnis, außer bei den ersten paar Takten, um mir beim Anfangen zu helfen.
Trotzdem habe ich am Anfang, wenn ich ein neues Stück lerne, wegen der Notwendigkeit oft auf die Noten zurückzugreifen, eine erhebliche fotografische Erinnerung.
Selbst für diejenigen, die nicht planen, ein fotografisches Gedächtnis zu erwerben, ist es eine gute Idee, jede fotografische Erinnerung, die man in diesem Stadium bekommen kann, zu behalten; d.h. unterstützen Sie es, weisen Sie es nicht von sich.
Sie werden überrascht sein, wie lange und wie gut es Ihnen erhalten bleibt, besonders wenn Sie es weiter pflegen.
Ich zwinge mich selbst nicht dazu, fotografisch auswendig zu lernen, weil ich weiß, daß ich am Ende meistens ein Tastatur-Gedächtnis wie unten beschrieben und ein Musik-Gedächtnis habe.
Es ist erstaunlich, daß man oftmals etwas viel besser tun kann, wenn kein Druck dahinter ist, und ich eigne mir wie von selbst eine ganze Menge fotografischer Erinnerungen an, die ich ein Leben lang behalte.
Ich wünschte mir sicherlich, daß ich das fotografische Gedächtnis früher mehr praktiziert hätte, da ich vermute, daß ich darin viel besser geworden wäre als ich es jetzt bin.

Diejenigen, die glauben, sie hätten kein fotografisches Gedächtnis, können es mit dem folgenden Trick versuchen.
Lernen Sie zunächst ein kurzes Musikstück mit so viel fotografischem Gedächtnis auswendig, wie Sie ohne weiteres zuwege bringen, und machen Sie sich keine Sorgen, wenn es nur teilweise klappt.
Wenn ein Abschnitt auswendig gelernt ist, bilden Sie ihn jeweils auf die Noten ab, von denen Sie das Stück gelernt haben, d.h. versuchen Sie sich für jede Note, die Sie spielen, die entsprechende Note auf dem Blatt vorzustellen.
Da Sie jeden Teil HS kennen, sollte dieses Abbilden von der Tastatur auf die Notenblätter einfach sein.
Beim Abbilden werden Sie auf das Notenblatt sehen müssen, um sich zu vergewissern, daß jede Note in der korrekten Position auf der richtigen Seite ist.
Sogar die Ausdrucksbezeichnungen sollten abgebildet werden.
Spielen Sie so lange abwechselnd aus dem fotografischen Gedächtnis und bilden die Tastatur auf die Noten ab, bis die Fotografie vollständig ist.
Dann können Sie Ihre Freunde verblüffen, indem Sie die Noten für das ganze Stück aufschreiben und das ab einer beliebigen Stelle!
Beachten Sie, daß Sie in der Lage sind, alle Noten zu schreiben, sowohl vorwärts als auch rückwärts, oder von irgendwo in der Mitte oder sogar jede Hand einzeln.
Und sie dachten, nur Wolfgang\footnote{A. Mozart} könnte das!


\label{c1iii6tastatur}

\textbf{Tastatur-Gedächtnis und mentales Spielen}: Beim Tastatur-Gedächtnis erinnern Sie sich während des Spielens zusammen mit der Musik an die Reihenfolge der Tasten und die Handbewegungen.
Es ist, als ob Sie ein Klavier im Kopf hätten und es spielen könnten.
Beginnen Sie mit dem Tastatur-Gedächtnis, indem Sie HS auswendig lernen, dann HT.
Spielen Sie dann, wenn Sie nicht am Klavier sind, das Stück in Ihrem Kopf, zunächst wieder HS.
\textbf{In Gedanken zu spielen (mentales Spielen), ohne Klavier, ist unser endgültiges Ziel; wir benutzen das Tastatur-Gedächtnis als Zwischenstufe.}
Es ist zunächst nicht notwendig, in Gedanken HT zu spielen, insbesondere wenn Sie es zu schwierig finden, obwohl Sie schließlich mit Leichtigkeit HT spielen werden.
Merken Sie sich, wenn Sie in Gedanken spielen, welche Abschnitte sie vergessen haben.
Nehmen Sie danach die Noten oder gehen Sie ans Klavier, und frischen Sie Ihr Gedächtnis auf.
Sie könnten auch das fotografische Gedächtnis für Teile ausprobieren, die Sie beim Benutzen des Tastatur-Gedächtnisses vergessen, da Sie sich ohnehin die Noten anschauen müssen, um sie erneut auswendig zu lernen.
Das mentale Spielen ist nicht nur deshalb schwierig, weil Sie das Stück auswendig gelernt haben müssen, sondern auch, weil Sie das Hand-Gedächtnis oder den Klavierklang nicht als Hilfe haben; aber genau deshalb ist es so mächtig.

Das Tastatur-Gedächtnis besitzt die meisten Vorteile des fotografischen Gedächtnisses, hat aber zusätzlich den Vorteil, daß die auswendig gelernten Noten Klaviertasten anstelle von dicken ovalen Punkten auf einem Blatt Papier sind; deshalb müssen Sie nicht von den ovalen Punkten zu den Tasten übersetzen.
Das erlaubt Ihnen, im Vergleich zum fotografischen Gedächtnis mit weniger Aufwand zu spielen, da der zusätzliche Prozeß, das Notenbild umzusetzen, entfällt.
Die Ausdrucksbezeichnungen sind keine Markierungen auf dem Papier, sondern gedankliche Vorstellungen der Musik (Musik-Gedächtnis).
Jedesmal, wenn Sie üben, pflegt sich das Tastatur-Gedächtnis - einschließlich der Handbewegungen - im Gegensatz zum fotografischen Gedächtnis von selbst.
Sie können das mentale Spielen ohne ein Klavier üben und so die zum Üben verfügbare Zeit mehr als verdoppeln, und Sie können vorausspielen, wie beim fotografischen Gedächtnis.

Als ich begann, das Tastatur-Gedächtnis zu benutzen, war meine seltsamste Beobachtung, daß ich dazu neigte, an den gleichen Stellen die gleichen Fehler zu machen und stecken zu bleiben, wie wenn ich tatsächlich am Klavier saß!
Wenn man darüber nachdenkt, macht das Sinn, weil alle Fehler ihren Ursprung im Gehirn haben, ob man am Klavier sitzt oder nicht.
Das Klavier macht niemals den Fehler, ich mache ihn.
Das läßt darauf schließen, daß wir vielleicht dazu in der Lage sind, bestimmte Aspekte des Klavierspielens zu üben und zu verbessern, indem wir in Gedanken üben, ohne ein Klavier.
Das wäre ein wahrhaft einzigartiger Vorteil des mentalen Spielens!
Die meisten Vorschläge für das Auswendiglernen, die in diesem Buch gemacht wurden, sind am besten auf das Tastatur-Gedächtnis anwendbar.
Das ist ein weiterer seiner Vorteile.
Das mentale Spielen ist der beste Test für das wahre Gedächtnis; wenn Sie das mentale Spielen ausführen, wird Ihnen bewußt werden, wie stark Sie noch vom Hand-Gedächtnis abhängig sind - auch nachdem Sie dachten, Sie hätten das Tastatur-Gedächtnis erworben.
Erst nachdem Sie sich genügend mentales Spielen angeeignet haben, können Sie im Grunde vom Hand-Gedächtnis befreit sein.
Das Hand-Gedächtnis ist jedoch immer eine gute Reserve - auch wenn Sie Ihr mentales Gedächtnis verloren haben, können Sie es gewöhnlich wiederherstellen, ohne auf die Notenblätter zu sehen, indem Sie das Stück einfach aus dem Hand-Gedächtnis auf dem Klavier spielen.

\textbf{Bei denjenigen, die das Singen vom Blatt lernen und ein absolutes Gehör erwerben möchten (\hyperref[c1iii12]{s. Abschnitt 12}), entwickeln sich diese Fertigkeiten durch das mentale Spielen automatisch.}
Bei der Tastatur-Methode stellt man sich die Tastatur vor, was dabei hilft, die richtigen Tasten für die absoluten Tonhöhen zu finden - eine Fertigkeit, die Sie für das Komponieren oder das Improvisieren am Klavier benötigen.
Deshalb sollten diejenigen, die daran interessiert sind, das Tastatur-Gedächtnis zu erlernen, auch das Blattsingen und absolute Tonhöhen üben, weil sie schon teilweise auf dem Wege dorthin sind.
Das ist ein erstklassiges Beispiel dafür, wie Ihnen das Lernen einer Fertigkeit (Auswendiglernen) dabei hilft, viele andere zu lernen.
Zweifellos ist das mentale Spielen eine der Arten, wie die musikalischen Genies das geworden sind, was sie sind oder waren.
So können viele dieser \enquote{genialen Fähigkeiten} praktisch von uns allen erworben werden, wenn wir wissen, wie man sie übt.
Wir sind nun bei einer erstaunlichen Schlußfolgerung angelangt: Gedächtnis führt zu mentalem Spielen, was wiederum zu relativem bzw. absolutem Gehör führt!
Mit anderen Worten: Das sind wesentliche Bausteine der Technik - wenn Sie alle drei erwerben, wird Ihre Fähigkeit auswendig zu lernen und vorzuspielen einen Quantensprung machen.

Wie bei jedem Gedächtnisverfahren muß das mentale Spielen von Anfang an geübt werden, ansonsten wird es Ihnen \textbf{niemals} gelingen.
Spielen Sie deshalb einen Abschnitt, sobald sie ihn auswendig gelernt haben, sofort in Gedanken, und behalten Sie dieses wie die anderen Gedächtnisverfahren bei.
Sie sollten schließlich in der Lage sein, die ganze Komposition in Gedanken zu spielen.
Sie werden erstaunt zurückblicken und sagen: \enquote{Nanu, das war leichter als ich dachte!}, weil dieses Buch alle für das mentale Spielen notwendigen Voraussetzungen bietet.

Können Sie die ganze Komposition erst in Gedanken spielen, werden Sie feststellen, daß Sie nun mit Leichtigkeit überall im Stück mit dem Spielen beginnen können, sogar mitten in einem Abschnitt oder einer Phrase.
Auch wenn sie abschnittsweise üben, ist das Beginnen in der Mitte eines Abschnitts gewöhnlich ziemlich schwierig; das mentale Spielen wird Sie in die Lage versetzen, irgendwo in einem Abschnitt zu beginnen - was mit jeder anderen Methode schwer zu erlernen ist.
Sie können auch eine viel deutlichere Vorstellung der Struktur der Komposition und der Folge der Melodien bekommen, weil Sie nun diese ganzen Konstrukte in Ihrem Kopf analysieren können.
Sie können sogar mit Geschwindigkeiten \enquote{üben}, die Ihre Finger nicht bewältigen können.
Die Finger können niemals Geschwindigkeiten erreichen, die das Gehirn nicht erreichen kann; man kann es sicherlich mit teilweisem Erfolg versuchen, aber es wird zu vielen Fehlern führen.
Das mentale Spielen mit hohen Geschwindigkeiten wird das schnelle Spielen der Finger fördern.
Das Spielen in Gedanken braucht nicht unbedingt viel Zeit, da es sehr schnell geht, wenn Sie es beherrschen.
Sie können es auch abkürzen, indem Sie einfache Abschnitte übergehen und sich nur auf Stellen konzentrieren, an denen Sie gewöhnlich auf Schwierigkeiten stoßen.

Das Spielen in Gedanken hat einen weiteren Vorteil: Je mehr Stücke Sie auswendig gelernt haben, desto leichter wird es, mehr auswendig zu lernen!
Das geschieht, weil Sie die Zahl der Assoziationen erhöhen.
Auch wird sich Ihre Fähigkeit zum mentalen Spielen rasch steigern, während Sie es üben und seine zahlreichen Vorteile entdecken.
Beim \hyperref[c1iii6d]{Hand-Gedächtnis} ist es im Gegenteil dazu so, daß es schwieriger wird, mehr auswendig zu lernen, wenn Ihr Repertoire größer wird, weil die Wahrscheinlichkeit der Konfusion steigt.
Praktisch alle Konzertpianisten wenden das mentale Spielen in einem gewissen Maß an.
Auf der Stufe eines Konzertpianisten wird von Ihnen erwartet, daß Sie es kennen, da es absolut notwendig ist.
Es wird aber nicht immer offiziell gelehrt.
Einigen glücklichen Schülern wurde das mentale Spielen gelehrt; für die anderen ist es ein zähes Ringen, diese \enquote{neue} Fertigkeit zu erlernen, die man von ihnen erwartet, wenn sie eine bestimmte Fertigkeitsstufe erreichen.
Zum Glück ist es eine Fertigkeit, die für den ernsthaften Schüler nicht schwierig zu meistern ist, weil der Nutzen so 
unmittelbar und weitreichend ist, daß die Motivation kein Problem darstellt.
Auf der fortgeschrittenen Stufe ist es einfach zu lernen, da solche Schüler einiges an Theorie gelernt haben.
Ein guter Solfège-Kurs sollte diese Fertigkeit lehren, aber Solfège-Lehrer lehren nicht immer Fertigkeiten zum Auswendiglernen oder das mentale Spielen.

Zusammengefaßt: Das Tastatur-Gedächtnis sollte Ihre hauptsächliche Gedächtnismethode sein.
Sie müssen gleichzeitig\footnote{in Gedanken} die Musik hören, so daß das \hyperref[c1iii6musik]{Musik-Gedächtnis} ein Teil dieses Prozesses ist.
Benutzen Sie das \hyperref[c1iii6foto]{fotografische Gedächtnis}, wann immer es einfach ist, und fügen Sie soviel \hyperref[c1iii6theorie]{theoretisches Gedächtnis} (s.u.) wie möglich hinzu.
\textbf{Sie haben das Stück erst dann wirklich auswendig gelernt, wenn Sie es in Gedanken spielen können} - das ist die einzige Möglichkeit, die Zuversicht zu bekommen, daß Sie ohne hörbare Fehler vorspielen können (alle Konzertpianisten können das).
Sie können damit die \hyperref[c1iii15]{Nervosität} reduzieren, und es ist der schnellste und leichteste Weg, sich ein \hyperref[c1iii12]{relatives und ein absolutes Gehör} anzueignen.
Das mentale Spielen ist in der Tat eine mächtige eigenständige Methode, die praktisch jede Ihrer musikalischen Aktivitäten beeinflußt, egal ob am Klavier oder nicht.
Das überrascht nicht, da alles, was man tut, seinen Ursprung im Gehirn hat.
Es verfestigt nicht nur das \hyperref[c1iii6tastatur]{Tastatur-Gedächtnis}, sondern unterstützt auch das \hyperref[c1iii6musik]{Musik-Gedächtnis}, \hyperref[c1iii6foto]{fotografische Gedächtnis}, \hyperref[c1iii14]{Vorspielen}, die \hyperref[c1iii12]{Genauigkeit der Tonhöhe}, \hyperref[c1iii6g]{kalt spielen} usw., und sollte der erste Schritt beim \hyperref[c1iii14d]{musikalischen Spielen} sein.
Seien Sie nicht passiv, warten Sie nicht, bis die Musik aus dem Klavier kommt, sondern sehen Sie die Musik, die Sie erzeugen möchten, aktiv voraus - das ist die einzige Möglichkeit, eine überzeugende Vorstellung zu geben.
Mit Hilfe des mentalen Spielens haben die großen Genies viel von dem erreicht, was sie vollbrachten, aber wenige Lehrer haben bisher diese Methode gelehrt: Es verwundert wenig, daß so viele Schüler diese Leistungen der großen Pianisten als unerreichbar ansehen.
Wir haben hier gezeigt, daß das mentale Spielen nicht nur erreichbar ist, sondern ein integraler Bestandteil des Klavierspielenlernens sein muß.

Wenn Sie beim Lernen der anderen Methoden dieses Buchs eine Art Erleuchtung hatten, warten Sie, bis Sie das mentale Spielen beherrschen.
Sie werden sich fragen, wie Sie es jemals wagen konnten, etwas öffentlich aufzuführen, ohne es in Gedanken spielen zu können.
Sie haben eine völlig neue Welt betreten und haben Fähigkeiten erworben, die Ihr Publikum in hohem Maß bewundern wird.


\label{c1iii6theorie}

Wir müssen alle danach streben, soviel \textbf{theoretisches Gedächtnis} wie möglich zu benutzen.
Das schließt Elemente wie Tonart, Taktart, Rhythmus, Akkordstruktur, Ausweichungen in andere Tonarten, Harmonien, melodische Struktur usw. ein.
Ein wahres Gedächtnis kann nicht ohne ein Verständnis der theoretischen Basis des jeweiligen Stücks aufgebaut werden. 
Leider erhalten die meisten Klavierschüler nicht genügend theoretische Ausbildung, um eine solche Analyse durchzuführen.
Beginnen Sie mit dem Lernen der chromatischen Tonleiter und des Quintenzirkels (\hyperref[c2_2]{Kapitel 2, Abschnitt 2}).
Jeder kann eine einfache Strukturanalyse durchführen: \hyperref[c1iv4]{Mozarts Wiederholungen}, \hyperref[c1iii20]{einfache parallele Sets in Bachs technischen Stücken}, Konzepte ähnlich der \hyperref[c1iv4Gruppe]{Gruppentheorie in Beethovens Musik}, wiederholtes Benutzen derselben Akkordprogressionen in Chopins Musik usw.
Die älteren Kompositionen liefern einfachere Beispiele, wie die Theorie angewandt oder dagegen verstoßen wird, um Musik zu erzeugen.
Obwohl man darüber streiten kann, ob sich die Qualität der Musik im Laufe der Zeit verbessert oder verschlechtert hat, steht es außer Frage, daß die Theorie sich weiterentwickelt hat.
Beim Spielen zeitgenössischer Musik und nach \enquote{Fake Books} sowie beim Üben in der Improvisation stehen Sie der Theorie von Angesicht zu Angesicht gegenüber und sind gezwungen, die praktischen Grundlagen zu lernen.
Deshalb sollte zu lernen, moderne Musik zu spielen, stets ein Teil des Prozesses das Klavierspielen zu lernen sein und wird eine gute Grundlage und einen Klangvorrat für das Gedächtnis bieten.


\subsubsection{Pflege}
\label{c1iii6k}

\textbf{Es gibt keine effektivere Pflegeprozedur als das \hyperref[c1iii6tastatur]{Tastatur-Gedächtnis und das mentale Spielen}.}
Gewöhnen Sie sich an, bei jeder sich bietenden Gelegenheit in Gedanken zu spielen.
Der Unterschied zwischen einem guten und einem schlechten Auswendiglernenden ist nicht so sehr die \enquote{Auswendiglernstärke}, sondern die geistige Haltung - was machen Sie mit Ihrem Gehirn während Sie wach sind und während Sie schlafen?
Gute Auswendiglernende haben die Angewohnheit entwickelt, ihr Gedächtnis ständig zu \hyperref[c1iii2]{zirkulieren}.
Wenn Sie das Auswendiglernen üben, müssen Sie deshalb auch Ihren Geist dazu trainieren, immer wieder mit dem Auswendiggelernten zu arbeiten.
Bei schlechten Auswendiglernenden wird das zunächst einen hohen Aufwand erfordern, aber wenn es über einen größeren Zeitraum (Jahre) geübt wird, ist es nicht so schwierig.
Sobald Sie das mentale Spielen gelernt haben, wird diese Aufgabe viel leichter.
Manche Behinderte haben ein Problem mit sich wiederholenden Bewegungen: Ihr Gehirn zirkuliert immer wieder dasselbe.
Das kann eine Erklärung dafür sein, daß sie viele normale Funktionen nicht ausführen können aber über ein unglaubliches Gedächtnis und erstaunliche musikalische Fähigkeiten verfügen, besonders wenn man diese Behinderten im Licht unserer obigen Diskussion über das Gedächtnis und das Spielen in Gedanken betrachtet.

Während der Pflege sollten Sie sich noch einmal die Noten ansehen und die Genauigkeit prüfen, sowohl für die einzelnen Noten als auch für die Ausdrucksbezeichnungen.
Da Sie beim Lernen des Stücks dieselben Notenblätter benutzt haben, ist die Wahrscheinlichkeit hoch, daß Sie, falls Sie beim ersten Lesen der Noten etwas falsch gemacht haben, diesen Fehler später wiederholen und den Fehler niemals bemerken werden. 
Eine Möglichkeit, dieses Problem zu umgehen, ist, sich Aufnahmen anzuhören.
Jeder größere Unterschied zwischen Ihrem Spielen und der Aufnahme wird sich deutlich abheben und im allgemeinen leicht zu erkennen sein.

\textbf{Eine weitere Aufgabe der Pflege ist es, sicherzustellen, daß Sie das Stück noch HS erinnern.}
Das kann bei größeren Stücken eine wahre Last werden, aber das ist es wert, weil Sie nicht während eines Konzerts herausfinden möchten, daß Sie es brauchen.
Beachten Sie, daß diese HS-Pflege-Sitzungen nicht nur dem Gedächtnis dienen.
Das ist die richtige Zeit, neue Dinge zu versuchen, viel schneller als die endgültige Geschwindigkeit zu spielen und Ihre Technik allgemein zu versäubern.
Ausgedehntes HT-Spielen bringt oft Timing- und andere unerwartete Fehler in das Spielen ein, und das ist die Zeit sie zu korrigieren.
\textbf{Deshalb ist HS-Spielen sowohl für die Gedächtnis- als auch für die Technik-Verbesserung eine lohnende Anstrengung.}
Das ist einer der besten Zeitpunkte, ein Metronom dazu zu benutzen, die Genauigkeit des Rhythmus und des Timings sowohl für das Spielen mit HS als auch mit HT zu überprüfen.
Die beste Vorbereitung darauf, nach einem Fehler während des Vorspielens den Anschluß wiederzufinden, ist das HS-Üben und das Spielen in Gedanken.
Dann haben Sie nach dem Fehler oder nach einer Gedächtnisblockade viele Möglichkeiten, den Faden wiederzufinden, wie z.B. mit einer Hand weiterzuspielen oder zunächst das Spielen nur mit einer Hand wieder aufzunehmen und dann die andere hinzuzufügen.
Diese Methode, sich wieder zu fangen, funktioniert deshalb, weil Fehler und Gedächtnisblockaden selten bei beiden Händen gleichzeitig auftreten - sie treten gewöhnlich nur in einer auf, in der anderen aber nicht, besonders wenn Sie HS geübt hatten.

Zusammengefaßt besteht die Pflege aus folgenden Komponenten:

\begin{enumerate}[label={\arabic*.}] 
\item Prüfen Sie mit dem Notenblatt oder durch Anhören von Aufnahmen die Genauigkeit jeder Note und Ausdrucksbezeichnung.
\item Stellen Sie sicher, daß Sie das ganze Stück HS spielen können.
Sie könnten sehr schnell HS üben, um die Technik aufzupolieren.
\item Üben Sie, an willkürlichen Stellen im Stück anzufangen.
Das ist eine exzellente Art, das Gedächtnis und Ihr Verständnis der Struktur der Komposition zu testen.
Wenn Ihr \hyperref[c1iii6tastatur]{mentales Spielen} gut ist, sollten Sie mit jeder Note beginnen können, nicht nur mit dem Anfang einer Phrase.
\item Stellen Sie fest, ob sie ohne Fehler und Gedächtnisblockaden sehr \hyperref[c1iii6h]{langsam spielen} können.
\item Spielen Sie \enquote{\hyperref[c1iii6g]{kalt}}. Es wird Ihre Fähigkeit zum \hyperref[c1iii14]{Vorspielen} sehr steigern.
\item \hyperref[c1iii6tastatur]{Spielen Sie \enquote{in Gedanken}}, zumindest HS.
Wenn Sie dies von Anfang an tun, wenn Sie das Stück zuerst lernen, und es beibehalten, ist es überraschend einfach.
 \end{enumerate} 


<!-- c1iii6l.html -->

\subsubsection{Blattspieler und Auswendiglernende (Bachs Inventionen)}
\label{c1iii6l}

\textbf{Viele gute Blattspieler sind schlechte Auswendiglernende, und viele gute Auswendiglernende sind schlechte Blattspieler.}
Dieses Problem tritt auf, weil gute Blattspieler es am Anfang kaum als notwendig erachten, auswendig zu lernen und Ihnen das Blattspiel Spaß macht, so daß Sie am Ende das Blattspiel auf Kosten des Auswendiglernens üben.
Je mehr sie vom Blatt spielen, desto weniger Gedächtnis brauchen sie, und je weniger sie auswendig lernen, desto schlechtere Auswendiglernende werden sie, mit dem Ergebnis, daß sie eines Tages wach werden und zu dem Schluß kommen, sie seien unfähig, auswendig zu lernen.
Selbstverständlich gibt es von Natur aus begabte Blattspieler, die echte Gedächtnisprobleme haben, aber diese bilden eine vernachlässigbar kleine Minderheit.
Deshalb tritt die Schwierigkeit auswendig zu lernen prinzipiell aufgrund einer psychologischen Denkblockade auf, die sich über einen langen Zeitraum aufgebaut hat.
Gute Auswendiglernende können das umgekehrte Problem erfahren: Sie können nicht vom Blatt spielen, weil sie automatisch alles auswendig lernen und selten die Chance haben, das Blattspiel zu üben.
Das ist jedoch kein symmetrisches Problem, weil praktisch alle fortgeschrittenen Klavierspieler wissen, wie man auswendig lernt.
\textbf{Schlechte Auswendiglernende hatten zusätzlich das Pech, daß sie nie eine fortgeschrittene Technik erworben haben, d.h. die technische Stufe von schlechten Auswendiglernenden ist im allgemeinen niedriger als jene von guten Auswendiglernenden.}

\enquote{Blattspiel} wird in diesem Abschnitt in einem weiteren Sinne gebraucht, d.h. es meint sowohl das wahre Vom-Blatt-Spielen (\hyperref[c030530]{Prima-Vista-Spiel}) als auch das Üben der Musik mit der Hilfe der Noten.
Die Unterscheidung zwischen dem Vom-Blatt-Spielen eines Stücks, das man vorher noch nie gesehen hat, und eines Stücks, das zuvor schon gespielt wurde, ist hier nicht wichtig.
Im Interesse der Kürze wird diese Unterscheidung dem Zusammenhang des Satzes überlassen.

\textbf{Es ist wichtiger, auswendig lernen zu können, als vom Blatt spielen zu können, weil man als Pianist ohne die Fähigkeit, gut vom Blatt zu spielen, existieren kann, aber man kann ohne die Fähigkeit, auswendig zu lernen, kein guter Pianist werden.}
Auswendiglernen ist für den durchschnittlichen Klavierspieler, dessen Gedächtnis nicht geschult wurde, nicht einfach.
Blattspieler, die nicht auswendig lernen können, sehen sich einem noch schwierigeren Problem gegenüber.
\textbf{Deshalb müssen schlechte Auswendiglernende, die ein auswendig gelerntes Repertoire erwerben möchten, es mit der Einstellung anfangen, daß dies ein Langzeitprojekt mit zahlreichen zu überwindenden Hindernissen sein wird.}
Wie oben gezeigt, ist die Lösung im Prinzip einfach: Machen Sie es sich zur Gewohnheit, alles auswendig zu lernen, \textit{bevor} Sie das Stück lernen.
In der Praxis ist die Versuchung, durch das Vom-Blatt-Spielen schnell zu lernen, oftmals zu unwiderstehlich.
Sie müssen die Art, wie sie neue Stücke üben, grundlegend ändern.

Das schwierigste Problem, auf das Blattspieler stoßen, ist das psychologische Problem der Motivation.
\textbf{Für diese guten Blattspieler erscheint Auswendiglernen als eine Zeitverschwendung, weil sie schnell lernen können, viele Stücke hinreichend gut vom Blatt zu spielen.}
Sie könnten sogar in der Lage sein, schwierige Stücke unter Benutzung des \hyperref[c1iii6d]{Hand-Gedächtnisses} zu spielen, und wenn sie hängenbleiben, können sie immer in den Noten vor sich nachsehen.
Deshalb kommen sie ohne Auswendiglernen zurecht.
Wenn man jahrelang auf diese Art Klavier geübt hat, wird es sehr schwierig, zu lernen wie man auswendig lernt, weil das Gehirn von den Noten abhängig geworden ist.
Schwierige Stücke sind bei diesem System unmöglich und werden deshalb zugunsten einer großen Zahl leichterer Stücke gemieden.
Lassen Sie uns im Bewußtsein dieser potentiellen Schwierigkeiten versuchen, ein typisches Programm zum Lernen wie man auswendig lernt durchzuarbeiten.

\textbf{Die beste Art anzufangen ist, ein paar kurze, neue Stücke auswendig zu lernen.}
Haben Sie erst einmal ein paar Stücke ohne allzuviel Aufwand erfolgreich auswendig gelernt, können Sie anfangen, Ihr Selbstvertrauen aufzubauen und Ihre Fertigkeiten zum Auswendiglernen zu verbessern.
Wenn diese Fertigkeiten ausreichend entwickelt sind, können Sie auch daran denken, alte Stücke auswendig zu lernen, die Sie durch Vom-Blatt-Spielen gelernt haben.

Meine Klaviersitzungen sind entweder Sitzungen zum Auswendiglernen oder technische Übungssitzungen, denn wenn ich zwischen den Sitzungen zum Auswendiglernen anderes Material spiele, vergesse ich, was auch immer ich zuletzt auswendig gelernt habe.
Während der technischen Übungssitzungen brauche ich fast nie die Noten.
Sogar während der Sitzungen zum Auswendiglernen benutze ich die Noten nur zu Anfang und lege sie weg, sobald ich kann.

Lassen Sie uns als ein Beispiel zum Auswendiglernen kurzer Stücke drei von Bachs zweistimmigen Inventionen lernen: \#1, \#8 und \#13.
Ich werde mit Ihnen \#8 durchgehen.
Nachdem Sie \#8 gelernt haben, versuchen Sie \#1 selbst, und fangen Sie dann mit \#13 an.
Die Idee ist, alle drei gleichzeitig zu lernen, aber wenn sich das als zu schwierig erweist, versuchen Sie nur zwei (\#8 und \#1) oder sogar nur \#8.
Es ist wichtig, daß Sie nur das versuchen, von dem Sie glauben, daß Sie gut damit zurechtkommen, weil hier gezeigt werden soll, wie einfach es ist.
Der unten angegebene Plan ist dafür gedacht, alle drei auf einmal zu lernen.
Wir nehmen an, daß Sie das Material der Abschnitte I bis III gelernt haben, und daß Sie technisch soweit sind, die Bach-Inventionen in Angriff zu nehmen.
Das Pedal wird bei keiner der Bach-Inventionen benutzt.



<!-- c1iii6l2.html -->

\label{c1iii6l2}

\textbf{Bachs Invention \#8, erster Tag.}
Die Taktart ist 3/4, d.h. es kommt jeweils ein Schlag auf eine Viertelnote, und jeder Takt hat drei Schläge.
Die Tonartenvorzeichnung hat ein Be, was die Tonart im Quintenzirkel einen Schritt gegen den Uhrzeigersinn von C-Dur zu F-Dur verschiebt.
Beginnen Sie, indem Sie die Takte 2 bis 4 der LH einschließlich der ersten beiden Noten von Takt 5 (Verbindung) auswendig lernen.
Es sollte weniger als eine Minute dauern, sich das zu merken; fangen Sie dann an, es mit der richtigen Geschwindigkeit zu spielen.
Heben Sie Ihre Hände vom Klavier, schließen Sie die Augen, und spielen Sie diesen Abschnitt in Gedanken, wobei Sie sich jede Note und jede Taste, die Sie spielen, bildlich vorstellen.
Machen Sie dann das gleiche mit der RH, Takte 1 bis 4, einschließlich der ersten vier Noten von Takt 5.
Kehren Sie nun zur LH zurück, und sehen Sie, ob Sie es ohne die Noten spielen können, und genauso mit der RH.
Wenn Sie es können, sollten Sie diesen Teil der Noten nie wieder nachsehen müssen, es sei denn, Sie hängen fest, was hin und wieder passieren kann.
Wechseln Sie zwischen der LH und der RH hin und her, bis Sie zufrieden sind.
Das sollte nur einige weitere Minuten dauern.
Sagen wir, die ganze Prozedur dauert fünf Minuten, bei einem schnellen Lerner sind es weniger.

Lernen Sie nun die Takte 5 bis 7 einschließlich der ersten beiden Noten der LH und der ersten vier Noten der RH in Takt 8.
Das sollte in ungefähr vier Minuten abgeschlossen sein.
Das sind alles HS-Übungen; wir werden nicht mit HT anfangen, bis wir das ganze Stück HS auswendig gelernt haben.
Es steht Ihnen jedoch frei, es jederzeit mit HT zu versuchen; verschwenden Sie aber keine Zeit mit dem HT-Üben, wenn Sie nicht sofort einen schnellen Fortschritt machen, da wir einem Ablaufplan folgen müssen!
Wenn Sie mit den Takten 5 bis 7 beginnen, machen Sie sich keine Sorgen darüber, daß Sie die zuvor auswendig gelernten Takte vergessen - Sie sollten nicht mehr an diese denken.
Das wird nicht nur die geistige Anspannung und Verwirrung verringern (da Sie nicht verschiedene auswendig gelernte Abschnitte vermischen), sondern auch dafür sorgen, daß Sie den zuvor auswendig gelernten Abschnitt teilweise vergessen, gezwungen sind, ihn erneut auswendig zu lernen, und ihn somit besser behalten.
Wenn Sie mit den Takten 5-7 zufrieden sind, verbinden Sie die Takte 1-7 einschließlich der Verbindung in Takt 8.
Das wird weitere 3 Minuten für beide Hände (HS) benötigen.
Wenn Sie die Takte 2-4 vergessen haben, während Sie 5-7 lernten, wiederholen Sie den Lernprozeß - es wird Ihnen schnell wieder einfallen, und die Erinnerung wird dauerhafter sein.
Vergessen Sie nicht, jeden Abschnitt in Gedanken zu spielen.

Lernen Sie als nächstes die Takte 8-11 auswendig, und fügen Sie sie den vorhergehenden Abschnitten hinzu.
Lassen Sie uns für diesen Teil 8 Minuten veranschlagen, was insgesamt 20 Minuten macht, um die Takte 1-11 auswendig zu lernen und HS auf die vorgegebene Geschwindigkeit zu bringen.
Wenn Sie mit einigen Teilen technische Schwierigkeiten haben, machen Sie sich keine Sorgen, wir werden später daran arbeiten.
Es wird nicht erwartet, daß Sie jetzt schon etwas perfekt spielen können.

Als nächstes verlassen wir die Takte 1-11 und arbeiten nur an den Takten 12-23 (machen Sie sich keine Gedanken darüber, ob Sie sich an die Takte 1-11 erinnern werden - es ist wichtig, daß Sie alle Befürchtungen zerstreuen und das Gehirn frei halten, um sich auf das Auswendiglernen zu konzentrieren).
Teilen Sie diesen Abschnitt in folgende Teilstücke (die Verbindungen sollten offensichtlich sein): 12-15, 16-19 und 19-23.
Takt 19 wird zweimal geübt, weil das zusätzliche Zeit dafür bedeutet, den schwierigen vierten Finger der LH zu trainieren.
Arbeiten Sie nur an den Takten 12-23, bis Sie sie alle HS hintereinander spielen können.
Das sollte ungefähr weitere 20 Minuten dauern.

Nehmen Sie dann den Rest bis zum Ende (24-34).
Diese Takte können in folgenden Teilstücken gelernt werden: 24-25, 26-29 und 30-34.
Das wird weitere 20 Minuten erfordern, also insgesamt eine Stunde, um das Ganze auswendig zu lernen.
Sie können nun entweder aufhören und morgen weitermachen oder sich jeden der drei Abschnitte noch einmal vornehmen.
Wichtig ist hier, daß Sie sich keine Gedanken darüber machen, ob Sie sich morgen an das alles erinnern können (wahrscheinlich nicht), aber um Spaß zu haben, können Sie sogar versuchen, die drei Abschnitte zu verbinden, oder Sie spielen den Anfangsteil HT, um zu sehen, wie weit Sie kommen.
Arbeiten Sie an den Teilen, die Ihnen technische Probleme bereiten, wenn Sie versuchen die Geschwindigkeit zu steigern.
Üben Sie diese technischen Trainingseinheiten in so kleinen Teilstücken wie Sie können; das bedeutet häufig zweinotige parallele Sets.
D.h. üben Sie nur die Noten, die Sie nicht zufriedenstellend spielen können.
Springen Sie von Abschnitt zu Abschnitt.
Die gesamte am ersten Tag für das Auswendiglernen aufgewandte Zeit ist eine Stunde.
Sie können auch das zweite Stück, Invention \#1, beginnen.
Üben Sie zwischen dem ersten und zweiten Tag, in Gedanken zu spielen, wann immer Sie Zeit dazu haben.

\textbf{Zweiter Tag:} Gehen Sie jeden der drei Abschnitte noch einmal durch, und verbinden Sie sie dann.
Spielen Sie jeden Abschnitt zuerst in Gedanken, bevor Sie etwas auf dem Klavier spielen.
Es kann sein, daß Sie an manchen Stellen die Notenblätter benötigen.
Legen Sie dann die Notenblätter beiseite - Sie werden sie, außer in Notfällen und um die Genauigkeit während der Pflege noch einmal zu prüfen, selten nochmals nötig haben.
Die einzige Anforderung am zweiten Tag ist, in der Lage zu sein, das ganze Stück - sowohl am Klavier als auch in Gedanken - von Anfang bis Ende HS zu spielen.
Konzentrieren Sie sich darauf, die Geschwindigkeit zu steigern, und werden Sie so schnell Sie können ohne Fehler zu machen.
Üben Sie zu \hyperref[c1ii14]{entspannen}.
Wenn Sie anfangen Fehler zu machen, werden Sie langsamer, und zirkulieren Sie die Geschwindigkeit auf und ab.
Beachten Sie, daß es eventuell leichter ist auswendig zu lernen, wenn Sie schnell spielen, und daß Sie vielleicht auf Gedächtnislücken stoßen, wenn Sie langsam spielen.
Üben Sie deshalb mit verschiedenen Geschwindigkeiten.
Fürchten Sie sich nicht vor dem schnellen Spielen, aber achten Sie darauf, daß Sie es mit genügend mittlerer Geschwindigkeit und langsamem Spielen ausgleichen, damit Sie jeglichen \hyperref[fpd]{FPD} eliminieren.
Anfänger haben bei Akkordwechseln, die oft am Anfang eines Takts auftreten, die größten Schwierigkeiten.
Akkordwechsel erzeugen Schwierigkeiten, da man nach dem Wechsel eine neue Gruppe ungewohnter Noten spielen muß.

Wenn sie am zweiten Tag mit HS völlig zufrieden sind, könnten Sie mit HT beginnen und dazu die gleichen kleinen Teilstücke verwenden, mit denen Sie HS gelernt haben.
Die erste Note von Takt 3 ist eine Kollision der beiden Hände. Benutzen Sie nur die LH für diese Note, ebenso in Takt 18.
Spielen Sie leise, auch wenn \textit{f} angegeben ist, so daß Sie die Schlagnoten betonen können, um die beiden Hände zu synchronisieren und das \hyperref[c1ii14]{Entspannen} zu üben.
Sie werden am Anfang wahrscheinlich etwas angespannt sein, aber konzentrieren Sie sich so früh wie möglich auf das Entspannen.

\textbf{Mäßige Geschwindigkeit ist oft die Geschwindigkeit, mit der man am einfachsten aus dem Gedächtnis spielen kann, weil man den Rhythmus benutzen kann, um weiter zu machen, und man die Musik in Phrasen statt in einzelnen Noten erinnern kann.}
Richten Sie Ihre Aufmerksamkeit deshalb von Anfang an auf den \hyperref[c1iii1b]{Rhythmus}.
Werden Sie nun langsamer, und arbeiten Sie an der Genauigkeit.
Um zu vermeiden, daß das langsame Spielen schneller wird, achten Sie auf jede einzelne Note.
Wiederholen Sie diesen \enquote{Schnell-Langsam-Zyklus}, und Sie sollten mit jedem Zyklus merklich besser werden.
Die Hauptziele sind, das Stück komplett HS auswendig zu lernen und das HS-Spielen soviel wie möglich zu beschleunigen.
Wo immer Sie technische Probleme haben, benutzen Sie die \hyperref[c1iii7b]{Übungen für parallele Sets}, um die Technik schnell zu entwickeln.
Sie sollten nicht mehr als 1 Stunde benötigen.

\textbf{Dritter Tag:} Lernen Sie HT mit den drei Hauptabschnitten, wie Sie es mit HS getan haben.
Sobald Sie feststellen, daß Sie mit HT durcheinander kommen, gehen Sie zu HS zurück, um die Dinge aufzuklären.
Das ist auch ein guter Moment, um die Geschwindigkeit mit HS weiter zu steigern, schneller als die endgültige Geschwindigkeit (später mehr darüber, wie man das macht).
Natürlich werden diejenigen mit geringeren technischen Fertigkeiten langsamer spielen müssen.
Erinnern Sie sich daran, daß das Entspannen wichtiger ist als die Geschwindigkeit.
Sie werden mit HS schneller spielen als mit HT, und alle Versuche, die Geschwindigkeit zu steigern, sollten HS durchgeführt werden.
Da die Hände noch nicht richtig koordiniert sind, werden Sie wahrscheinlich ein paar Gedächtnislücken haben, und es mag schwierig sein, HT ohne Fehler zu spielen, außer wenn Sie langsam spielen.
Ab hier werden Sie sich auf die langsamere \hyperref[c1ii15]{PPI} verlassen müssen, um eine größere Verbesserung zu erzielen.
Sie haben jedoch in 3 Stunden innerhalb von 3 Tagen das Stück im Grunde auswendig gelernt und können es, vielleicht schleppend, HT spielen.
Sie können auch das ganze Stück in Gedanken spielen.

Fangen Sie nun mit dem zweiten Stück (Invention \#1) an, während Sie das erste Stück auf Hochglanz polieren.
Üben Sie die beiden Stücke abwechselnd.
Arbeiten Sie an \#1, bis Sie anfangen \#8 zu vergessen, frischen Sie dann \#8 wieder auf, und arbeiten Sie daran, bis Sie anfangen \#1 zu vergessen.
Erinnern Sie sich daran, daß Sie ein wenig vergessen \textit{wollen}, damit Sie erneut lernen können; das wird gebraucht, um ein Langzeitgedächtnis aufzubauen.
Es hat psychologische Vorteile, diese Art Programme zu benutzen, bei denen man nur gewinnen kann: Wenn man vergißt, ist es genau das, was man erreichen wollte; wenn man nicht vergessen kann, um so besser!
Dieses Programm gibt Ihnen auch ein Maß dafür, wieviel Sie in einer vorgegebenen Zeitspanne auswendig lernen können, bzw. wieviel nicht.
Jüngere Menschen sollten finden, daß die Menge, die man auf einmal auswendig lernen kann, rapide ansteigt, wenn man Erfahrung bekommt und mehr Tricks zum Auswendiglernen kennt.
Je schneller Sie auswendig lernen, desto schneller können Sie spielen, und je schneller sie spielen, desto leichter wird es auswendig zu lernen.
Gesteigertes Selbstvertrauen spielt auch eine größere Rolle.
Letzten Endes wird der hauptsächliche begrenzende Faktor Ihre technische Fertigkeitsstufe sein, nicht die Fähigkeit zum Auswendiglernen.
Wenn Sie genügend Technik haben, werden Sie innerhalb weniger Tage mit der richtigen Geschwindigkeit spielen.
Wenn Sie es nicht können, bedeutet das nur, daß Sie mehr Technik brauchen.
Es bedeutet nicht, daß Sie ein schlechter Auswendiglernender sind.

\textbf{Vierter Tag:} Es gibt nicht viel, was Sie nach zwei bis drei Tagen tun können, um das erste Stück technisch zu beschleunigen.
Beginnen Sie das Üben von \#8 einige Tage lang zunächst mit HS, und wechseln Sie dann zu HT, beides jeweils mit verschiedenen Geschwindigkeiten gemäß Ihrer augenblicklichen Laune.
Sobald Sie sich dazu bereit fühlen, üben Sie HT, aber kehren Sie zu HS zurück, wenn Sie anfangen Fehler zu machen, Gedächtnislücken mit HT bekommen oder wenn Sie technische Probleme damit haben, auf die endgültige Geschwindigkeit zu kommen.
Üben Sie, das Stück in Abschnitten HT zu spielen, springen Sie zufällig von Abschnitt zu Abschnitt durch das Stück.
Versuchen Sie, mit dem letzten kleinen Abschnitt zu beginnen und sich zum Anfang zurückzuarbeiten.

Isolieren Sie die Problemstellen, und üben Sie diese gesondert.
Die meisten Menschen haben eine schwächere LH, so daß es problematisch sein kann, die LH schneller als die endgültige Geschwindigkeit werden zu lassen.
Es mag z.B. schwer sein, die letzten vier Noten der LH in Takt 3 der Invention \#8, 4234(5), wobei (5) die Verbindung ist, schnell zu spielen.
Teilen Sie sie in diesem Fall in drei parallele Sets auf - 42, 423 und 2345 -, und üben Sie diese unter Benutzung der \hyperref[c1iii7b]{Übungen für parallele Sets}.
Verbinden Sie sie dann zu 423 und 2345.
423 ist kein paralleles Set (4 und 3 spielen dieselbe Note), so daß man es nicht so schnell wie parallele Sets spielen kann.
Bringen Sie sie zuerst auf annähernd unendliche Geschwindigkeit (fast wie ein Akkord), und lernen Sie dann, bei diesen Geschwindigkeiten zu entspannen, indem Sie in schnellen Quadrupeln spielen (s. \hyperref[c1iii7b1]{Abschnitt III.7b}).
Werden Sie dann schrittweise langsamer, um die Unabhängigkeit der Finger zu entwickeln.
Verbinden Sie die parallelen Sets paarweise, und verbinden Sie sie zum Schluß alle miteinander.
Das ist eine wirkliche Verbesserung der Technik und wird deshalb nicht über Nacht geschehen.
Sie werden während des Übens nur eine geringe Verbesserung wahrnehmen, aber Sie sollten am nächsten Tag eine eindeutige Verbesserung spüren und eine große Verbesserung nach einigen Wochen \hyperref[c1ii15]{(PPI)}.

Wenn Sie das Stück HT spielen können, beginnen Sie damit, es in Gedanken HT zu spielen.
Dieses HT-Üben sollte einen oder zwei Tage benötigen.
Wenn sie die Aufgabe, in Gedanken zu spielen, jetzt nicht erledigen, werden Sie es, wie die meisten Menschen, niemals tun.
Wenn Sie aber erfolgreich sind, wird es zu dem mächtigsten Werkzeug Ihres Gedächtnisses.

\textbf{Ungefähr am 5. oder 6. Tag} sollten Sie in der Lage sein, die Invention \#13 zu beginnen, so daß Sie im folgenden alle drei Stücke täglich üben können.
Eine weitere Vorgehensweise ist, zunächst nur die Invention \#8 gut zu lernen, und dann, nachdem man mit der ganzen Prozedur vertraut ist, mit \#1 und \#13 anzufangen.
Der Hauptgrund dafür, mehrere Stücke auf einmal zu lernen, ist, daß diese Stücke so kurz sind, daß man zu viele Wiederholungen an einem Tag spielt, wenn man nur eins übt.
Erinnern Sie sich daran, daß Sie ab dem ersten Tag mit der richtigen Geschwindigkeit spielen (HS), und ab dem zweiten Tag sollten Sie zumindest einige Abschnitte schneller als die endgültige Geschwindigkeit spielen.
Auch dauert es länger, diese drei Stücke nacheinander zu lernen, als sie zusammen zu lernen.

Wie schnell Sie Fortschritte machen, hängt nach dem zweiten oder dritten Tag mehr von Ihrer Fertigkeitsstufe als von den Fähigkeiten Ihres Gedächtnisses ab.
Wenn Sie erst das ganze Stück HS nach Belieben spielen können, sollten Sie das Stück als auswendig gelernt ansehen.
Wenn Sie über die mittlere Stufe hinaus sind, werden Sie sehr schnell in der Lage sein, es HT zu spielen.
Wenn Sie aber nicht derart fortgeschritten sind, werden die technischen Schwierigkeiten jeder Hand den Fortschritt verlangsamen.
Das Gedächtnis wird nicht der begrenzende Faktor sein.
Für die Arbeit am HT werden Sie offensichtlich an der Koordination der beiden Hände arbeiten müssen.
Bach hat die Inventionen so konstruiert, daß man gleichzeitig das Koordinieren der Hände und mit beiden Händen unabhängig voneinander zu spielen lernt.
Das ist der Grund, daß es zwei Stimmen gibt und sie sich überlagern; auch spielt in \#8 eine Hand staccato während die andere legato spielt.\footnote{Hier unterscheiden sich die verschiedenen Editionen.
Im Original von Invention \#8 ist kein Staccato angegeben.
Bach weist bereits in seiner Einleitung zur Reinschrift von 1723 darauf hin, daß die Stücke u.a. für die \enquote{Lehrbegierigen} geschrieben wurden.
Die zweistimmigen Inventionen und dreistimmigen Sinfonien sind demnach Lehrstücke für den Klavierspieler, und es liegt im Ermessen des Lehrers und Spielers, sie dem augenblicklichen Zweck entsprechend anzupassen; so wurden z.B. bei Invention \#1 die Figuren mit vier Sechzehntelnoten in Bachs Reinschrift nachträglich teilweise mit Durchgangsnoten ergänzt - wahrscheinlich um einem Schüler zu zeigen, daß man an diesen Stellen statt dessen auch zwei Sechzehnteltriolen spielen kann.
Gleichzeitig sind die Inventionen und Sinfonien Lehrstücke für das Komponieren (und deshalb auch gut komponierte Musik) - Zitat aus der Einleitung: \enquote{... einen starcken Vorschmack von der Composition zu überkommen})}

Alle drei oben besprochenen Stücke sollten in eins bis zwei Wochen vollständig auswendig gelernt sein, und Sie sollten anfangen, zumindest mit dem ersten Stück gut zurechtkommen.
Nehmen wir an, Sie haben sich in den zwei Wochen nur auf das Auswendiglernen dieser drei Stücke konzentriert.
Wenn Sie nun zu alten Stücken zurückkehren, die Sie zuvor auswendig gelernt haben, werden Sie feststellen, daß Sie sich weniger gut daran erinnern können.
Das ist ein guter Zeitpunkt, um sie wieder aufzupolieren und diese \hyperref[c1iii6f]{Pflegeaufgabe} mit dem weiteren Verbessern Ihrer neuen Bach-Stücke abzuwechseln.
Sie sind im Grunde fertig.
Herzlichen Glückwunsch!

Wie gut Sie aus dem Gedächtnis spielen können, hängt sowohl von Ihrer Technik als auch davon ab, wie gut Sie etwas auswendig gelernt haben.
\textbf{Es ist wichtig, den Mangel an Technik nicht mit der Unfähigkeit zum Auswendiglernen zu verwechseln.}
Die meisten Menschen mit Schwierigkeiten beim Auswendiglernen haben ein ausreichendes Gedächtnis aber eine unzulängliche Technik.
Deshalb benötigen Sie Methoden für das Testen Ihrer Technik und Ihres Gedächtnisses.
Wenn Ihre Technik ausreichend ist, sollten Sie in der Lage sein, HS entspannt ungefähr mit dem 1,5-fachen der endgültigen Geschwindigkeit zu spielen.
Bei \#8 ist die Geschwindigkeit ungefähr MM = 100 auf dem Metronom, so daß Sie in der Lage sein sollten, mit beiden Händen ungefähr mit 150 HS zu spielen.
Bei 150 haben Sie Glenn Gould geschlagen (wenn auch HS - er spielte ungefähr 140)!
Wenn Sie oberhalb von 100 nicht gut HS spielen können, dann müssen Sie Ihre Technik verbessern, bevor Sie erwarten können, irgend etwas HT nahe an 100 zu spielen.
Der beste Test für das Gedächtnis ist, festzustellen, ob Sie das Stück in Gedanken spielen können.
Wenn Sie diese Tests durchführen, können Sie herausfinden, ob Sie an der Technik oder am Gedächtnis arbeiten müssen.

Die meisten Menschen haben eine schwächere LH; bringen Sie die LH-Technik so nah an die RH-Stufe wie möglich.
Benutzen Sie, wie oben für Takt 3 der LH veranschaulicht, die Übungen für parallele Sets, um an Ihrer Technik zu arbeiten.
Bach ist besonders nützlich, um die Techniken der LH und RH auszubalancieren, weil beide Hände ähnliche Passagen spielen.
Deshalb wissen Sie sofort, daß die LH schwächer ist, wenn sie nicht auf die gleiche Geschwindigkeit kommen kann wie die RH.
Bei anderen Komponisten, wie z.B. Chopin, ist die LH gewöhnlich viel einfacher und bietet keinen guten Test für die LH.
Schüler mit unzulänglicher Technik müssen eventuell wochenlang HS üben, bevor Sie darauf hoffen können, diese Inventionen HT mit der vorgegebenen Geschwindigkeit zu spielen.
Spielen Sie in diesem Fall HT nur mit zufriedenstellenden langsamen Geschwindigkeiten und warten Sie darauf, daß sich Ihre HS-Technik entwickelt, bevor Sie die HT-Geschwindigkeit steigern.

Bachs Musik ist dafür berüchtigt, schwer schnell zu spielen und hochanfällig für \hyperref[fpd]{FPD} (\enquote{Schnellspiel-Abbau}, s. Abschnitt II.25) zu sein.
Die intuitive Lösung für dieses Problem war, geduldig langsamer zu üben.
Man muß bei vielen Kompositionen von Bach nicht sehr schnell spielen, um unter FPD zu leiden.
Wenn Ihre maximale Geschwindigkeit MM = 20 ist, während die vorgeschlagene Geschwindigkeit 100 ist, dann ist für Sie 20 schnell, und der FPD kann  bereits bei dieser Geschwindigkeit sein schreckliches Haupt erheben.
Deshalb erzeugt langsames HT-Spielen und der Versuch, es zu beschleunigen, nur mehr Verwirrung und FPD.
Wir kennen nun den Grund für den Ruf von Bachs Musik: Die Schwierigkeit resultiert aus zu vielen Wiederholungen mit langsamem HT-Spielen, die lediglich die Verwirrung vergrößern, ohne Ihr Gedächtnis oder Ihre Technik zu unterstützen.
Die bessere Lösung ist das abschnittsweise HS-Üben.
Für diejenigen, die das nie zuvor getan haben: Sie werden bald mit Geschwindigkeiten spielen, die Sie nie für möglich gehalten haben.


\label{ruhig}

\textbf{Ruhige Hände.}
Viele Lehrer bezeichnen zu Recht \enquote{ruhige Hände} als ein wünschenswertes Ziel.
\textbf{Dabei spielen hauptsächlich die Finger, und die Hände bewegen sich so wenig wie möglich.
Ruhige Hände sind der Lackmustest für den Erwerb der Technik.}
Das Eliminieren unnötiger Bewegungen gestattet nicht nur ein schnelleres Spielen, sondern erhöht auch die Kontrolle.
Viele von Bachs Stücken wurden für das Üben ruhiger Hände entwickelt.
Einige der unerwarteten Fingersätze, die auf dem Notenblatt verzeichnet sind, wurden gewählt, um für das Spielen mit ruhigen Händen passend zu sein oder es zu erleichtern.
Einige Lehrer drängen alle Schüler - auch Anfänger - dazu, jederzeit mit ruhigen Händen zu spielen, eine solche Vorgehensweise ist jedoch kontraproduktiv, da man \enquote{ruhige Hände} nicht langsam spielen kann; es gibt also keine Möglichkeit, ruhige Hände bei niedriger Geschwindigkeit zu lehren.
Der Schüler fühlt nichts und fragt sich, warum das jetzt gut sein soll.
Wenn man langsam spielt oder wenn der Schüler nicht über genügend Technik verfügt, ist eine gewisse zusätzliche Bewegung unvermeidlich und angemessen.
Die Hände unter diesen Bedingungen zur Bewegungslosigkeit zu zwingen, würde nur das Spielen erschweren und Streß erzeugen.
Diejenigen, die bereits die Technik der ruhigen Hände besitzen, können beim langsamen oder schnellen Spielen ohne Schaden viel Bewegung hinzufügen.
Manche Lehrer versuchen, ruhige Hände zu lehren, indem sie eine Münze auf die Hand legen, um zu sehen, ob die Hand ruhig genug ist, so daß die Münze nicht herunterfällt.
Diese Methode zeigt sicherlich, daß der Lehrer die Bedeutung der ruhigen Hände erkannt hat, bringt aber dem Schüler nichts.
Wenn man Bach mit der vollen Geschwindigkeit mit ruhigen Händen spielt, dann wird eine auf die Hand gelegte Münze sofort davonfliegen.
Nur wenn man jenseits einer bestimmten Geschwindigkeit spielt, werden ruhige Hände für den Klavierspieler offensichtlich und notwendig.
Wenn Sie das erste Mal ruhige Hände bekommen, ist es absolut nicht zu verkennen, machen Sie sich also keine Sorgen, daß Sie es nicht mitbekommen könnten.
Die beste Zeit, den Schüler zu lehren was ruhige Hände bedeuten, ist, wenn er genügend schnell spielt, so daß er die ruhigen Hände fühlen kann.
Wenn Sie die Technik erworben haben, dann können Sie sie auf das langsame Spielen anwenden; Sie sollten nun das Gefühl haben, daß Sie über viel mehr Kontrolle verfügen und zwischen den Noten mehr freie Zeit haben.
Somit sind ruhige Hände keine besondere Bewegung der Hand, sondern ein Gefühl der Kontrolle und der nahen völligen Abwesenheit von Geschwindigkeitsbarrieren.

Im Fall der hier besprochenen Stücke von Bach werden die ruhigen Hände bei Geschwindigkeiten nahe der endgültigen Geschwindigkeit zur Notwendigkeit; offensichtlich wurden die Geschwindigkeiten im Hinblick auf die ruhigen Hände ausgewählt.
Ohne ruhige Hände werden Sie bei den empfohlenen Geschwindigkeiten auf Geschwindigkeitsbarrieren treffen.
HS-Üben ist für ruhige Hände wichtig, weil sie viel leichter zu erwerben und zu fühlen sind, wenn man HS spielt, und weil HS-Spielen es erlaubt, viel schneller als mit HT zu den Geschwindigkeiten für ruhige Hände zu gelangen.
Tatsächlich ist es am besten, erst mit HT anzufangen, wenn man mit jeder der beiden Hände ruhig spielen kann, weil das die Wahrscheinlichkeit verringert, daß man schlechte Angewohnheiten verfestigt.
Das bedeutet, HT mit oder ohne ruhige Hände macht einen Unterschied, so daß man sich nicht angewöhnen sollte, HT ohne ruhige Hände zu spielen.
Diejenigen mit ungenügender Technik benötigen eventuell zu lange, um ruhige Hände zu erreichen, so daß solche Schüler eventuell HT ohne ruhige Hände anfangen müssen; sie können die ruhigen Hände später schrittweise erwerben, indem sie mehr HS üben.
Das erklärt, warum diejenigen mit genügender Technik diese Inventionen so viel schneller lernen können als diejenigen ohne.
Solche Schwierigkeiten sind einige der Gründe dafür, nicht zu versuchen, Stücke zu lernen, die zu schwierig für Sie sind, und bieten nützliche Tests dafür, ob die Komposition zu schwierig oder für Ihre Fertigkeitsstufe angemessen ist.
Diejenigen mit ungenügender Technik werden mit Sicherheit riskieren, Geschwindigkeitsbarrieren aufzubauen.
Obwohl manche behaupten, daß die Bach-Inventionen \enquote{mit jeder Geschwindigkeit} gespielt werden können, stimmt das nur für den musikalischen Gehalt; diese Kompositionen müssen mit ihrer empfohlenen Geschwindigkeit gespielt werden, um den vollen Nutzen aus der Klavierlektion zu ziehen, die Bach im Sinn hatte.
In diesem Abschnitt wird die Geschwindigkeit wegen der Notwendigkeit, ruhige Hände zu erklären und zu erwerben, überbetont;
üben Sie jedoch nicht die Geschwindigkeit um der Geschwindigkeit willen, da dies wegen des Stresses und den schlechten Angewohnheiten nicht funktionieren wird; musikalisches Spielen ist immer noch der beste Weg, die Geschwindigkeit zu steigern - siehe Abschnitt III.7i.

Bei denjenigen mit einer stärkeren RH, wird die RH auch als erste ruhig; wenn Sie das Gefühl erst einmal kennen, können Sie es schneller auf die LH übertragen.
Wenn es einsetzt, werden Sie plötzlich merken, daß es leichter wird, schnell zu spielen.
Deshalb funktioniert das HT-Üben beim Lernen von neuen Bach-Stücken nicht - es gibt keine Möglichkeit, schnell zu ruhigen Händen zu kommen.

Bach hat diese Inventionen für die technische Entwicklung geschrieben.
Deshalb hat er beiden Händen gleich schwieriges Material gegeben; das stellt für die LH eine größere Herausforderung dar, weil die Baß-Hämmer und -Saiten schwerer sind.
Bach hätte sich geärgert, wenn er Übungen wie die \hyperref[c1iii7h]{Hanon-Serie} gesehen hätte, weil er wußte, daß Übungen ohne Musik eine Zeitverschwendung sind. Dies wird durch den Aufwand deutlich, den er in diese Kompositionen steckte, um die Musik einzubeziehen.
Die Menge des technischen Materials, das er in diese Kompositionen packte, ist unglaublich: Unabhängigkeit der Finger (ruhige Hände, Kontrolle, Geschwindigkeit), sowohl Koordination als auch Unabhängigkeit der beiden Hände (mehrere Stimmen, staccato gegen legato, kollidierende Hände, Verzierungen), Harmonien, Musik erzeugen, sowohl die LH als auch die schwachen Finger (4 und 5) stärken, alle hauptsächlichen parallelen Sets, Benutzungsarten des Daumens, Standard-Fingersätze usw.
Beachten Sie, daß die Verzierungen \hyperref[c1iii7b]{Übungen für parallele Sets} und nicht nur musikalische Verzierungen, sondern ein integraler Bestandteil der technischen Entwicklung sind.
Durch die Verzierungen verlangt Bach von Ihnen, daß Sie mit der einen Hand parallele Sets üben, während Sie gleichzeitig mit der anderen Hand einen anderen Teil spielen und mit dieser Kombination Musik erzeugen!

Achten Sie darauf, Bach nicht zu laut zu spielen, auch wenn \textit{f} angezeigt ist.
Die Instrumente seiner Zeit erzeugten viel weniger Klang als moderne Klaviere, so daß Bach Musik schreiben mußte, die mit Klang gefüllt ist und wenige Pausen hat.
Ein Zweck der zahlreichen Verzierungen und Triller war zu Bachs Zeiten, den Klang auszufüllen.
Deshalb neigt seine Musik dazu, zu viel Klang zu haben, wenn sie auf modernen Klavieren laut gespielt wird.
Besonders bei den Inventionen und Sinfonien, in denen der Schüler versucht, alle die konkurrierenden Melodien herauszubringen, besteht die Neigung, jede folgende Melodie lauter zu spielen, was zu lauter Musik führt.
Die verschiedenen Melodien müssen auf der Basis des musikalischen Konzepts miteinander konkurrieren, nicht durch die Lautstärke.
Leiser zu spielen trägt auch zum Erreichen einer völligen Entspannung und der wahren Unabhängigkeit der Finger bei.

Wenn Sie eine der Sinfonien (dreistimmigen Inventionen) lernen möchten, könnten Sie \#15 probieren, die leichter als die meisten anderen ist.
Sie ist sehr interessant und hat einen Abschnitt in der Mitte, in dem die beiden Hände kollidieren und oft dieselben Noten spielen.
Wie alle Bach-Kompositionen enthält diese viel mehr als man auf den ersten Blick sieht.
Gehen Sie deshalb vorsichtig zu Werke.
Vor allem ist es allegro vivace!
Die Taktbezeichnung ist ein fremdartiges 9/16, was bedeutet, daß die Gruppen von sechs 32tel-Noten im dritten Takt als 3 Schläge gespielt werden müssen und nicht als 2 (drei Notenpaare anstelle von zwei Triolen).
Diese Taktbezeichnung führt zu den drei wiederholten Noten (zwei im dritten Takt), die thematischen Wert haben und in einer für Bach charakteristischen Weise über die Tastatur wandern.
Wenn die beiden Hände in Takt 28 kollidieren, heben Sie die RH und lassen Sie die LH daruntergleiten.
Spielen Sie alle Noten mit beiden Händen.
Falls die Kollision der Daumen problematisch ist, können Sie den RH-Daumen weglassen und nur mit dem LH-Daumen spielen.
Achten Sie darauf, daß Sie in Takt 36 den richtigen Fingersatz für die RH benutzen: (5),(2,3),(1,4),(3,5),(1,4),(2,3).

\textbf{Lassen Sie uns zum Schluß den letzen notwendigen Schritt zum Auswendiglernen besprechen: das Analysieren der Struktur bzw. der \enquote{Geschichte} hinter der Musik.}
Der Merkprozeß ist so lange unvollständig, bis Sie die Geschichte hinter der Musik verstehen.
Wir werden Invention \#8 benutzen.
Die ersten 11 Takte bilden die \enquote{Einführung}.
Hierbei spielen die RH und die LH im Grunde dasselbe, die LH jeweils einen Takt verzögert, und das Hauptthema wird eingeführt.
Der \enquote{Hauptteil} besteht aus den Takten 12 bis 28, in denen die Rollen der beiden Hände zunächst vertauscht sind, die LH führt die RH, gefolgt von einigen faszinierenden Entwicklungen.
Das \enquote{Ende} beginnt mit Takt 29 und bringt das Stück zu einem ordnungsgemäßen Abschluß, bei dem die ursprüngliche Rolle der RH erneut bekräftigt wird.
Beachten Sie, daß das Ende das gleiche ist wie das Ende der Einführung - das Stück endet effektiv zweimal, was das Ende überzeugender macht.
Beethoven hat dieses Mittel, ein Stück mehrfach zu beenden, weiterentwickelt und in unglaubliche Höhen geführt.

Wir präsentieren nun einige Erklärungen dafür, warum die Entwicklung einer solchen \enquote{Geschichte} - alle großen Musiker haben Ihre Musik auf eine bestimmte Weise aufgebaut - die beste und vielleicht einzig zuverlässige Art ist, eine Komposition dauerhaft auswendig zu lernen.



<!-- c1iii6m.html -->

\subsubsection{Funktion des menschlichen Gedächtnisses}
\label{c1iii6m}

Die Gedächtnisfunktion des Gehirns wird bisher nur unvollständig verstanden.
\textbf{Es gibt keinen Beweis für die Existenz eines \enquote{fotografischen Gedächtnisses} im engeren Sinne des Wortes}, obwohl ich diesen Ausdruck in diesem Buch benutzt habe.
\textbf{Das ganze Gedächtnis ist assoziativ.}
Somit assoziieren wir, wenn wir uns ein Gemälde von Monet visuell einprägen, in Wirklichkeit die Motive des Gemäldes mit etwas tief im Inneren unseres Gedächtnisses und merken uns nicht bloß ein zweidimensionales Bild, das aus so vielen Bildpunkten besteht.
Deshalb ist es leichter, sich an bedeutende Gemälde oder ungewöhnliche Fotografien zu erinnern als an ähnliche Bilder mit weniger charakteristischen Merkmalen, obwohl beide vielleicht dieselbe Bandbreite (Anzahl Bildpunkte) haben.
Ein weiteres Beispiel: Wenn Sie einen Kreis auf einem Blatt Papier fotografieren, wird das Foto genau sein; der Durchmesser und die Lage des Kreises wird übereinstimmen.
Wenn Sie in Gedanken ein \enquote{fotografisches Engramm} desselben Kreises anfertigen und dann versuchen, ihn auf einem anderen Blatt Papier zu zeichnen, wird der Durchmesser und die Lage abweichen.
Das bedeutet, daß Sie ihn sich begrifflich gemerkt (und mit dem bereits vorhandenen Wissen über Kreise, ungefähre Größen und Orte assoziiert) haben.
Wie ist es nun mit dem fotografischen Gedächtnis bei Notenblättern?
Ich kann sie tatsächlich in Gedanken sehen!
Ist das nicht fotografisch?
Man kann leicht beweisen, daß das ebenfalls assoziativ ist - in diesem Fall assoziiert mit Musik.
Wenn Sie einen Musiker mit \enquote{fotografischem} Gedächtnis bitten, sich eine ganze Seite zufälliger Noten einzuprägen, wird er große Schwierigkeiten haben, obwohl er vielleicht keine Probleme haben wird, sich innerhalb kurzer Zeit eine ganze Sonate zu merken\footnote{dasselbe gilt analog für Schachspieler für zufällige und sinnvolle Stellungen von Figuren auf einem Schachbrett}.
Deshalb gibt es keinen besseren Weg, sich Musik einzuprägen (fotografisch oder auf eine andere Weise), als vom Standpunkt der Musiktheorie aus.
Sie müssen nur die Musik mit der Theorie assoziieren, und Sie haben sie sich eingeprägt.
Mit anderen Worten: Wenn Menschen sich etwas einprägen, speichern sie nicht die Datenbits im Gehirn wie ein Computer, sondern sie assoziieren die Daten mit einem Grundgerüst oder einem \enquote{Algorithmus}, bestehend aus vertrauten Dingen im Gehirn.
In diesem Beispiel ist die Musiktheorie das Grundgerüst.
Natürlich kann ein sehr guter Auswendiglernender (der kein Musiker sein muß) Methoden dafür entwickeln, sich sogar eine zufällige Reihenfolge von Noten zu merken, indem er einen angemessenen Algorithmus ausarbeitet, wie wir im folgenden erklären.

Der beste Beweis für die assoziative Natur des menschlichen Gedächtnisses stammt aus Tests mit guten Auswendiglernenden, die unglaubliche Meisterleistungen ausführen können, wie Hunderte von Telefonnummern aus einem Telefonbuch auswendig lernen usw.
Es gibt zahlreiche Gedächtniswettbewerbe, in denen gute Auswendiglernende miteinander wetteifern.
Diese guten Auswendiglernenden wurden intensiv befragt, und es stellt sich heraus, daß keiner von Ihnen fotografisch auswendig lernt, obwohl das Endresultat fast nicht von einem fotografischen Gedächtnis zu unterscheiden ist.
Wenn sie gefragt werden, wie sie sich etwas einprägen, stellt sich heraus, daß sie alle assoziative Algorithmen benutzen.
Der Algorithmus ist bei jedem einzelnen verschieden (auch bei der gleichen Aufgabe), aber alle Algorithmen sind Mittel dafür, die Objekte mit etwas zu assoziieren, das ein Muster hat, an das man sich erinnern kann.
Zum Erinnern von hunderten von Zahlen ist z.B. ein Algorithmus, jede Zahl mit einem Klang zu assoziieren.
Die Klänge werden so gewählt, daß sie \enquote{Worte} bilden, wenn man sie aneinanderreiht - nicht in Deutsch, sondern in einer anderen Sprache, die für diesen Zweck geeignet ist.
Japanisch ist eine Sprache mit einer solchen Eigenschaft.
Ein Beispiel: Die Quadratwurzel von 2 ist 1,41421356, was man als Satz lesen kann, der übersetzt ungefähr \enquote{Gute Menschen, gute Menschen sind das Ansehen wert.} lautet, und die Japaner benutzen ständig solche Algorithmen, um sich an Nummern, wie z.B. Telefonnummern, zu erinnern.
Auf 7 Stellen ist die Quadratwurzel von 3 \enquote{Behandele die ganze Welt!}, und die Wurzel von 5 ist \enquote{An der sechsten Station des Fudschijama schreit eine Eule.}
Das Erstaunliche ist die Geschwindigkeit, mit der gute Auswendiglernende das auswendig zu lernende Objekt auf ihren Algorithmus abbilden können.
Es stellt sich auch heraus, daß diese guten Auswendiglernenden nicht so geboren wurden - obwohl sie vielleicht mit mentalen Fähigkeiten geboren wurden, die zu einem guten Gedächtnis führen können.
\textbf{Auswendiglernende entwickeln sich nach harter Arbeit zur Perfektionierung ihrer Algorithmen und täglichem Üben, genau wie Pianisten.}
Diese \enquote{harte Arbeit} leisten sie aber ohne Anstrengung, weil sie es genießen.

Ein einfacher, aber weniger effizienter Algorithmus ist, die Nummern in eine Geschichte zu verpacken.
Angenommen, Sie möchten sich an die 14 Ziffern 53031791389634 erinnern.
Sie könnten z.B. folgende Geschichte verwenden: \enquote{Ich wachte morgens um 5:30 Uhr mit meinen 3 Brüdern und 1 Großmutter auf; das Alter meiner Brüder ist 7, 9 und 13, und meine Großmutter ist 89 Jahre alt, und wir sind abends um 6:34 Uhr zu Bett gegangen.}
Das ist ein Algorithmus, der auf alltäglichen Erfahrungen basiert, was die Zufallszahlen \enquote{bedeutungsvoll} macht.
Das Faszinierende daran ist, daß der Algorithmus 43 Worte enthält und trotzdem viel einfacher zu behalten ist als die 14 Ziffern\footnote{wobei ich \enquote{5:30 Uhr} und \enquote{6:34 Uhr} als \enquote{5 Uhr 30} bzw. \enquote{6 Uhr 34} mit jeweils 3 Worten gerechnet habe}.
Tatsächlich haben Sie sich 203 Zeichen und Ziffern\footnote{inkl. Leerstellen und Satzzeichen} leichter gemerkt als die 14 Ziffern!
Sie können das leicht selbst testen.
Merken Sie sich zunächst sowohl die 14 Ziffern (wenn Sie können - für mich ist es nicht einfach) als auch den obigen Algorithmus.
Versuchen Sie 24 Stunden später, die Ziffern aus dem Gedächtnis und anhand des Algorithmus aufzuschreiben; Sie werden den Algorithmus viel leichter und genauer finden.
Alle guten Auswendiglernenden haben unglaublich effektive Algorithmen entwickelt und die Kunst gepflegt, jede Gedächtnisaufgabe sofort in ihre Algorithmen zu übersetzen.

Können Klavierspieler einen Vorteil aus diesem Gebrauch der effizienten Algorithmen ziehen?
Natürlich können wir das!
Wie hätte Liszt sonst innerhalb kurzer Zeit mehr als 80 Kompositionen auswendig lernen und aufführen können?
Es gibt keinen guten Grund, anzunehmen, daß Liszt in bezug auf das Gedächtnis spezielle Fähigkeiten hatte, er muß also einen Algorithmus benutzt haben.
Dieser Algorithmus ist überall - er heißt Musik!
Musik ist einer der effizientesten Algorithmen für das Auswendiglernen großer Datenmengen.
Praktisch alle Pianisten können leicht mehrere Beethoven-Sonaten auswendig lernen.
Von der Datenmenge her entspricht jede Sonate mindestens den Telefonnummern von vier Seiten eines Telefonbuchs.
Wir können also das Äquivalent von mehr als 20 Seiten Telefonnummern auswendig lernen - das würde als Wunder angesehen, wenn es tatsächlich Telefonnummern wären.
Und wir könnten uns wahrscheinlich noch mehr merken, wenn wir nicht soviel Zeit für das Üben der Technik und der Musikalität aufwenden müßten.
Deshalb unterscheidet sich, was Klavierspieler routinemäßig erreichen, nicht so sehr davon, wofür die \enquote{Gedächtniskünstler} berühmt sind.
Musik ist ein besonders effizienter Algorithmus, weil sie einigen strengen Regeln folgt.
Komponisten wie Liszt sind mit diesen Regeln vertraut und können schneller auswendig lernen (siehe dazu \hyperref[c1iv4]{Mozarts Formel} in Kapitel IV.4).
Zudem ist uns allen die musikalische Logik angeboren; diesen Teil des Musik-Algorithmus müssen wir nicht lernen.
Deshalb haben Musiker hinsichtlich des Auswendiglernens praktisch gegenüber jedem anderen Beruf einen Vorteil, und die meisten von uns sollten eine Stufe des Gedächtnisses erreichen, die der von guten Auswendiglernenden eines Gedächtniswettbewerbs nahekommt, da wir eine Menge darüber wissen, wie man es macht.

Es ist nun möglich, zu verstehen, wie Auswendiglernende viele Seiten von Telefonnummern auswendig lernen können.
Sie haben am Ende einfach eine \enquote{Geschichte} anstelle einer Reihe von Zahlen.
Beachten Sie, daß ein 90-jähriger Mann sich eventuell nicht mehr an Ihren Namen erinnern kann, aber er kann sich hinsetzen und Ihnen stunden- oder sogar tagelang Geschichten aus dem Gedächtnis erzählen.
Und er muß keine Art von Gedächtnisspezialist sein, um das zu tun.
Wenn man weiß, wie man sein Gehirn benutzen muß, dann kann man Dinge, die zunächst vollkommen unmöglich erschienen.

\textbf{Was an den Assoziationen befähigt uns also tatsächlich, etwas zu tun, was wir ansonsten nicht können?
Die vielleicht einfachste Art, dieses zu beschreiben, ist zu sagen, daß Assoziationen uns befähigen, das einzuprägende Objekt zu \textit{verstehen}.}
Das ist eine sehr nützliche Definition, weil sie jedem dabei helfen kann, in der Schule oder bei jedem Bemühen, etwas zu lernen, besser zu sein.
Wenn man Physik, Mathematik oder Chemie wirklich versteht, dann muß man sie nicht auswendig lernen, weil man sie nicht vergessen kann.
Das mag sinnlos erscheinen, weil wir bloß die Frage \enquote{Was ist Gedächtnis?} zunächst in \enquote{Was ist Assoziation?} und dann in \enquote{Was ist Verstehen?} umgewandelt haben.
Es ist nicht sinnlos, wenn wir \textit{Verstehen} definieren können: Es ist der geistige Prozeß, ein neues Objekt mit anderen Objekten (je mehr, desto besser!) zu assoziieren, die Ihnen bereits vertraut sind.
D.h., das neue Objekt wird nun \enquote{bedeutungsvoll}.

Was bedeutet \enquote{verstehen} und was \enquote{bedeutungsvoll}?
Das menschliche Gedächtnis besteht aus zahlreichen Komponenten, wie der visuellen, auditiven, taktilen, emotionalen, bewußten, automatischen, dem Kurz- und Langzeitgedächtnis usw.
Deshalb kann jede Eingabe in das Gehirn zu einer fast unendlichen Zahl von Assoziationen führen.
Die meisten Menschen stellen jedoch nur einige wenige her.
Das Gehirn von guten Auswendiglernenden erzeugt - beinahe automatisch und regelmäßig - aus jeder Eingabe zahlreiche Assoziationen.
Die große Zahl der Assoziationen gewährleistet, daß auch wenn man einige davon vergißt, eine genügende Anzahl übrig bleibt, um die Erinnerung aufrecht zu erhalten.
Das reicht jedoch nicht.
Wir haben gesehen, daß wir zum Auswendiglernen etwas verstehen müssen, was bedeutet, daß diese Assoziationen verknüpft und auf eine logische Weise geordnet sein müssen.
Ein guter Auswendiglernender kann also diese Assoziationen auch gut organisieren, so daß er, wenn er eine Eingabe erhält (z.B. den Namen einer Person), sofort das gewünschte (z.B. die Telefonnummer) finden kann, indem er diese Zusammenhänge verfolgt.
Wenn die Assoziationen nicht gut geordnet und untereinander verbunden sind, dann ist man eventuell nicht in der Lage, sich an die Nummer zu erinnern, obwohl sie irgendwo im Gedächtnis gespeichert ist.
Gute Auswendiglernende erzeugen ständig eine große Anzahl Assoziationen, sie verstärken sie laufend und sind in der Lage, diese Assoziationen in logischen Strukturen zu ordnen, so daß sie abgerufen werden können.
Die Gehirne guter Auswendiglernender suchen ständig nach \enquote{interessanten}, \enquote{erstaunlichen}, \enquote{unerklärlichen}, \enquote{außergewöhnlichen} usw. Assoziationen, die das Abrufen vereinfachen.
Dieselben Prinzipien gelten für das Auswendiglernen von Musik.

Die assoziative Natur des Gedächtnisses erklärt, warum das \hyperref[c1iii6tastatur]{Tastatur-Gedächtnis} funktioniert: Sie assoziieren die Musik mit den einzelnen Bewegungen und den Tasten, die zum Erzeugen der Musik gespielt werden müssen.
Das sagt uns auch, wie man das Tastatur-Gedächtnis optimiert.
Es ist sicher ein Fehler, zu versuchen, sich an jeden Tastendruck zu erinnern; wir sollten in Begriffen wie \enquote{RH-Arpeggio, das mit C anfängt und mit der LH eine Oktave tiefer wiederholt wird, Staccato, mit fröhlichem Gefühl} usw. denken und diese Bewegungen mit der daraus resultierenden Musik und ihrer Struktur assoziieren; merken Sie sich Notengruppen, Notenfamilien und abstrakte Konzepte.
Sie sollten so viele Assoziationen wie möglich herstellen: Bachs Musik kann bestimmte Eigenschaften haben, wie spezielle Verzierungen, kollidierende Hände und \hyperref[c1ii11]{parallele Sets}.
Sie machen damit die Aktion des Spielens \enquote{bedeutungsvoll}, und zwar in Begriffen, wie die Musik erzeugt wird und wie die Musik in Ihr geistiges Universum paßt.
Deshalb ist das Üben von Tonleitern und Arpeggios so wichtig.
Wenn Sie auf einen Lauf aus 30 Noten treffen, können Sie ihn sich einfach als Teil einer Tonleiterfolge statt als 30 Noten merken.
Sich ein \hyperref[c1iii12]{absolutes Gehör oder zumindest ein relatives Gehör} anzutrainieren, ist für das Gedächtnis ebenfalls hilfreich, weil das weitere Assoziationen zu den einzelnen Noten ermöglicht.
Musiker erzeugen am häufigsten Assoziationen mit den von der Musik hervorgerufenen Emotionen.
Einige benutzen Farben oder Landschaften.
\enquote{Geborene Auswendiglernende} ist ein Begriff ohne Definition, da jeder gute Auswendiglernende ein System hat, und alle Systeme scheinen einigen sehr ähnlichen Grundprinzipien zu folgen.


\subsubsection{Ein guter Auswendiglernender werden}
\label{c1iii6n}

Niemand wird ohne Übung ein guter Auswendiglernender, so wie niemand ohne Übung ein guter Klavierspieler wird.
Die gute Nachricht ist, daß mit dem richtigen Training praktisch jeder ein guter Auswendiglernender werden kann, so wie jeder mit den richtigen Übungsmethoden ein guter Klavierspieler werden kann.
Bei den meisten Schülern ist der Wunsch auswendig zu lernen stark genug, und sie sind deshalb bereit zu üben; trotzdem scheitern viele.
Wissen wir, warum sie scheitern, und gibt es eine einfache Lösung für das Problem?
Die Antwort ist: \enquote{Ja!}

\textbf{Schlechte Auswendiglernende versagen beim Auswendiglernen, weil sie aufhören, bevor sie angefangen haben.}
Man hat sie nie in effektive Gedächtnismethoden eingeführt, und sie haben genügend Fehlschläge erlebt, um zu dem Schluß zu kommen, daß es nutzlos sei, das Auswendiglernen zu versuchen.
Um ein guter Auswendiglernender zu werden, ist die Erkenntnis sehr hilfreich, daß unser Gehirn unabhängig davon, ob wir das möchten oder nicht, alles aufzeichnet.
Unser einziges Problem beim Auswendiglernen ist, daß wir diese Daten nicht so leicht abrufen können.

Wir haben gesehen, daß das endgültige Ziel aller besprochenen Gedächtnisprozeduren das gute und solide \hyperref[c1ii12mental]{mentale Spielen} war.
\textbf{Bevor ich das mentale Spielen untersuchte, dachte ich, daß es nur von wirklich begabten Künstlern durchgeführt werden könnte.
Das hat sich als völlig falsch erwiesen.}
Wir alle führen das mentale Spielen in unserem täglichen Leben aus!
Das mentale Spielen ist nur ein Prozeß, bei dem wir Informationen aus unserem Gedächtnis abrufen und sie für das Planen unserer Aktionen, Lösen unserer Probleme usw. ordnen oder benutzen.
Wir tun das praktisch in jedem wachen Moment und wahrscheinlich sogar während des Schlafens.
Wenn eine Mutter mit drei Kindern am Morgen aufsteht, die täglichen Aktivitäten für ihre Familie plant, überlegt, was es zu essen geben soll, wie jedes Gericht für das Frühstück, Mittagessen und Abendessen zubereitet wird, usw., führt sie eine mentale Prozedur aus, die genauso komplex ist wie bei Mozart, wenn dieser eine Bach-Invention in Gedanken spielte.
Wir halten diese Mutter nur deshalb nicht für ein Genie vom Range Mozarts, weil wir mit diesen mentalen Prozessen, die wir jeden Tag ohne Anstrengung ausführen, so vertraut sind.
Obwohl Mozarts Fähigkeit, Musik zu komponieren, in der Tat außerordentlich war, ist mentales Spielen nichts ungewöhnliches - mit ein wenig Übung können wir es alle.
\textbf{In der heutigen Lehr- und Übungspraxis wurde das mentale Durchgehen des Ablaufs in den meisten Disziplinen, die den Einsatz des Gedächtnisses erfordern, zum Standard (z.B. beim Golf, Eiskunstlauf, Tanzen, Abfahrtsskilauf usw.).
Es sollte auch Klavierschülern von Anfang an gelehrt werden.}

Eine weitere Möglichkeit, das Gedächtnis zu verbessern, ist die Anwendung der \enquote{Vergiß-es-dreimal-Regel}: Wenn man dasselbe dreimal vergessen und erneut auswendig lernen kann, wird man sich gewöhnlich ewig daran erinnern.
Diese Regel funktioniert, weil sie die Frustration über das Vergessen eliminiert und Ihnen dreimal die Gelegenheit bietet, verschiedene Methoden zum Auswendiglernen und Abrufen zu üben.
Die Frustration über das Vergessen und die Furcht vor dem Vergessen sind die größten Feinde schlechter Auswendiglernender.
Sie müssen etwas nicht wirklich völlig vergessen, aber lassen Sie sich genügend Zeit (ein paar Tage oder mehr), so daß die Wahrscheinlichkeit groß ist, daß sie etwas vergessen, und lernen Sie es dann erneut auswendig.

Nachdem Sie angefangen haben, das Auswendiglernen und dessen \hyperref[c1iii6k]{Pflege} zu üben, können Sie nach und nach die oben besprochenen Konzepte hinzufügen (Assoziationen, Verstehen, Informationen ordnen usw.).
Ein junger Mensch, der am Anfang des Lebens diese Techniken wie selbstverständlich anwendet, wird auf fast allen Gebieten ein guter Auswendiglernender werden.
Mit anderen Worten: \textbf{Das Gehirn beschäftigt sich ständig mit dem Auswendiglernen, es wird zur mühelosen, automatischen Routine.
Das Gehirn sucht automatisch nach interessanten Assoziationen und pflegt das Gedächtnis fortlaufend ohne bewußten Aufwand.}
Bei älteren Menschen ist dieser \enquote{Automatismus} viel schwieriger und wird länger dauern.
Wenn es Ihnen gelingt, sich die ersten Informationen einzuprägen (z.B. ein Repertoire von Klavierstücken), dann werden Sie gleichzeitig anfangen, dieselben Prinzipien auf alles andere anzuwenden, und Ihr allgemeines Gedächtnis wird sich verbessern.
Deshalb müssen Sie, um ein guter Auswendiglernender zu werden, zusätzlich zur Anwendung der hier besprochenen Gedächtnismethoden, die Art und Weise, wie Sie Ihr Gehirn benutzen, ändern.
Das Gehirn muß darauf trainiert werden, ständig Assoziationen zu suchen, besonders nach stimulierenden (lustigen, fremdartigen, furchteinflößenden usw.), die Ihnen dabei helfen, das Auswendiggelernte abzurufen.
Das ist der schwierigste Teil: zu ändern, wie das Gehirn arbeitet.


\subsubsection{Zusammenfassung}
\label{c1iii6o} 

\textbf{Benutzen Sie zum Auswendiglernen von Klaviermusik einfach die Regeln für das Lernen mit dem zusätzlichen Vorbehalt, daß Sie alles auswendig lernen, \textit{bevor} Sie anfangen, das Stück zu üben.}
Es ist diese Wiederholung während des Übens aus dem Gedächtnis, die das Gedächtnis automatisch mit wenig zusätzlichem Aufwand implantiert, d.h. zusätzlich zu dem Aufwand, der für das Lernen des Stücks notwendig ist.
Der wichtigste erste Schritt ist das Auswendiglernen mit HS.
Wenn man etwas über einen bestimmten Punkt hinaus auswendig lernt, wird man es fast niemals vergessen.
HS-Spielen ist auch das Hauptelement der \hyperref[c1iii6k]{Gedächtnispflege}.
Für das Auswendiglernen können Sie das \hyperref[c1iii6hand]{Hand-Gedächtnis}, \hyperref[c1iii6foto]{fotografische Gedächtnis}, \hyperref[c1iii6tastatur]{Tastatur-Gedächtnis und mentale Spielen}, \hyperref[c1iii6musik]{Musik-Gedächtnis} und die \hyperref[c1iii6theorie]{Musiktheorie} benutzen.
Das menschliche Gedächtnis ist assoziativ, und ein guter Auswendiglernender kann gut Assoziationen finden und sie so ordnen, daß sie zu einem \enquote{Verständnis} des Themas führen.
Erinnern Sie sich daran, daß die Musik einer der effizientesten Algorithmen für das Gedächtnis ist; ein \hyperref[c1iii12]{absolutes Gehör} ist ebenfalls hilfreich.
Alle diese Gedächtnismethoden sollten in das mentale Spielen münden - Sie können die Musik in Gedanken spielen und hören, so als ob Sie ein Klavier im Kopf hätten.
Das mentale Spielen ist praktisch für alles, was man am Klavier tut, unentbehrlich; es versetzt Sie z.B. in die Lage, das Auswendiglernen und Abrufen jederzeit zu üben.
Wir haben gesehen, daß gute Auswendiglernende deshalb gut sind, weil ihr Gehirn sich stets automatisch etwas einprägt; man kann sein Gehirn nur dazu trainieren, wenn man mental spielen kann.
Sie sollten zwei Repertoires haben: ein auswendig gelerntes und ein weiteres für das Spielen vom Blatt.
Auswendiglernen ist notwendig, um ein Stück schnell und gut zu lernen, \hyperref[c1iii14d]{musikalisch zu spielen}, sich schwierige Technik anzueignen usw.
Das mentale Spielen bringt eine ganz neue wunderbare Welt musikalischer Fähigkeiten mit sich, wie z.B. ein  Stück ab einer Stelle irgendwo in der Mitte zu spielen, das absolute Gehör zu erlernen, Komponieren, fehlerfrei vorzuspielen usw.
Viele dieser unglaublichen Meisterleistungen, die den musikalischen Genies nachgesagt werden, sind für uns alle tatsächlich in Reichweite!



<!-- c1iii7.html -->

\subsection{Übungen}
\label{c1iii7}

\subsubsection{Einführung}
\label{c1iii7a}

Aufgrund einer überwältigenden Zahl von Nachteilen (s. \hyperref[c1iii7h]{Abschnitt 7h}) \textbf{sind die meisten Fingerübungen nicht nützlich}.
Ein Einwand ist, daß sie viel Zeit verschwenden.
Wenn man so übt, daß man schwierige Stücke spielen kann, warum sollte man dann die Zeit mit Fingerübungen verbringen, anstelle die schwierigen Stücke zu üben?
Ein weiterer Einwand ist, daß die meisten Übungen sich zu sehr wiederholen und keinen musikalischen Aufwand erfordern, so daß man sie mit abgeschaltetem musikalischen Teil des Gehirns spielen kann, was gemäß eines jeden sachkundigen Klavierlehrers die schlechteste Art ist, das Klavierspielen zu üben.
\textbf{Stupides Üben ist schädlich}.
Übungen sind dazu da, die Ausdauer zu steigern - die meisten von uns besitzen jedoch eine Menge physischer Ausdauer zum Spielen aber eine ungenügende Ausdauer des Gehirns, weshalb stupide wiederholende Übungen unsere gesamte musikalische Ausdauer verringern können.
Wenn die Schüler nicht sorgfältig angeleitet werden, üben sie diese Wiederholungen mechanisch und lassen das Klavierüben für jeden, der unglücklich genug ist zuhören zu müssen, als eine Strafe erscheinen.
Das ist ein Weg, \enquote{Stille-Kämmerlein-Pianisten} zu erzeugen, die nur üben können, wenn niemand zuhört, da sie nie geübt haben, Musik zu machen.
Einige vollendete Klavierspieler benutzen Übungen routinemäßig zum Aufwärmen, aber diese Angewohnheit resultiert aus ihrer früheren Ausbildung, und Konzertpianisten benötigen sie für ihre Übungssitzungen nicht.

Statt dieser schädlichen Übungen bespreche ich hier eine völlig andere Klasse von Übungen, die Ihnen dabei helfen, Ihre technischen Defizite zu diagnostizieren, die für das Beseitigen dieser Defizite erforderliche Technik zu erwerben und musikalisch zu spielen.
In \hyperref[c1iii7b]{Abschnitt 7b} bespreche ich die Übungen für das Erwerben der Technik, insbesondere der Geschwindigkeit.
\hyperref[c1iii7c]{Abschnitt 7c} zeigt, wann und wie man sie benutzt.
In den \hyperref[c1iii7d]{Abschnitten 7d} bis \hyperref[c1iii7g]{7g} bespreche ich andere nützliche Übungen.
Ich habe die meisten Einwände gegen Übungen vom Hanon-Typ in \hyperref[c1iii7h]{Abschnitt 7h} zusammengetragen.
In der Vergangenheit wurden diese Übungen vom Hanon-Typ wegen mehrerer falscher Vorstellungen weithin akzeptiert:

\begin{enumerate}[label={\roman*.}] 
 \item Man kann Technik dadurch erwerben, daß man eine begrenzte Zahl einfacher Übungen lernt.
 \item Musik und Technik können getrennt gelernt werden.
 \item Technik erfordert hauptsächlich eine Entwicklung der Muskeln ohne eine Entwicklung des Gehirns.
 \item Technik erfordert eine Stärke der Finger.
\end{enumerate}
Solche Übungen wurden bei vielen Lehrern populär, denn wenn sie funktionierten, konnten die Schüler von den Lehrern mit wenig Aufwand in der Technik unterwiesen werden!
Das ist nicht der Fehler dieser Lehrer, weil diese falschen Vorstellungen über Generationen hinweg weitergereicht wurden, auch von berühmten Lehrern, wie Czerny, Hanon und vielen anderen.
Die Wahrheit ist, daß Klavierpädagogik ein herausfordernder, zeitintensiver und wissensbasierter Beruf ist.

%: bisher 7i
\textbf{Wenn wir Technik einfach als die Fähigkeit zu spielen definieren, dann besteht sie mindestens aus drei Komponenten.}
Sie hat eine innere Technikkomponente, die einfach Ihre Fertigkeitsstufe ist.
Die Fertigkeit zu haben bedeutet jedoch nicht, daß man spielen kann.
Wenn Sie z.B. für einige Tage nicht gespielt haben und die Finger eiskalt sind, werden Sie wahrscheinlich nicht in der Lage sein, irgend etwas zufriedenstellend zu spielen.
Somit wäre die zweite Komponente das Maß, in dem die Finger \enquote{aufgewärmt} sind.
Es gibt noch eine dritte Komponente, die wir hier \enquote{Konditionierung} nennen wollen.
Wenn Sie z.B. eine Woche lang große Bäume gefällt haben oder nichts anderes getan haben als tagelang Pullover zu stricken, werden die Hände in keiner guten Verfassung zum Klavierspielen sein.
Die Hände haben sich körperlich an eine andere Aufgabe angepaßt.
Wenn Sie auf der anderen Seite monatelang jeden Tag mindestens drei Stunden Klavier üben, werden Ihre Hände Dinge tun, die sogar Sie erstaunen.
Das Konditionieren bezieht größtenteils den ganzen Körper und wahrscheinlich das Gehirn mit ein und sollte deshalb nicht \enquote{Handkonditionierung} genannt werden.

Übungen können etwas zu allen drei Komponenten der Technik (innere, Aufwärmen und Konditionierung) beitragen, und Schüler verwechseln häufig Übungen zum Aufwärmen oder andere ineffiziente Übungen mit dem Erwerben innerer Technik.
Diese Verwechslung tritt auf, weil praktisch jede Übung zum Aufwärmen und zur Konditionierung beitragen kann, der Schüler dies aber leicht als innere Verbesserung mißverstehen kann, wenn er sich der drei Komponenten nicht bewußt ist.
Dieses Mißverständnis kann von Nachteil sein, falls der Schüler zuviel Aufwand in Übungen steckt und deshalb nicht all die anderen, wichtigeren Arten der Entwicklung innerer Technik lernt.
Dieses Wissen über die Komponenten der Technik ist auch bei der \hyperref[c1iii14]{Vorbereitung auf Konzerte} wichtig, weil man in diesem Fall fragen muß: \enquote{Was ist die beste Art, die Hände aufzuwärmen und zu konditionieren?}

Die innere Fertigkeitsstufe und das Aufwärmen der Hände sind leicht zu verstehen, aber das Konditionieren ist sehr komplex.
Die wichtigsten Faktoren, die das Konditionieren kontrollieren, sind die Dauer und die Häufigkeit des Übens und der Zustand des aus Gehirn, Nerven und Muskeln bestehenden Systems.
\textbf{Um die Hände in ihrer besten Kondition für das Spielen zu halten, müssen die meisten Menschen jeden Tag üben.}
Lassen Sie das Üben ein paar Tage ausfallen, wird die Konditionierung merklich nachlassen.
\textbf{Obwohl an anderer Stelle bemerkt wurde, daß man mit mindestens drei Tagen Üben in der Woche einen deutlichen Fortschritt erzielen kann, wird das deshalb sicherlich nicht zur besten Konditionierung führen.}
Konditionierung ist ein weitaus größerer Effekt als einigen Menschen klar ist.
Fortgeschrittene Klavierspieler achten stets genau auf die Konditionierung, da sie ihre Fähigkeit zum musikalischen Spielen beeinflußt.
Sie ist wahrscheinlich mit physiologischen Veränderungen verbunden, wie z.B. einer Erweiterung der Blutgefäße und der Ansammlung bestimmter Stoffe an spezifischen Stellen des Nerven- und Muskelsystems.
Dieser Faktor der Konditionierung wird wichtiger, wenn Ihre Fertigkeitsstufe steigt und wenn Sie anfangen, sich routinemäßig mit den höheren musikalischen Konzepten zu befassen, wie z.B. der Farbe oder das Charakteristische verschiedener Komponisten herauszubringen.
Unnötig zu sagen, daß sie entscheidend wird, wenn man technisch anspruchsvolles Material spielt.
Deshalb muß sich jeder Klavierspieler der Konditionierung bewußt sein, um zu wissen, was zu einer bestimmten Zeit gespielt oder geübt werden kann.

Ein schwerer zu bestimmender Faktor, der die Konditionierung beeinflußt, ist der Zustand des Gehirns bzw. des Nervensystems.
\textbf{Sie können deshalb ohne offensichtlichen Grund \enquote{gute} Tage und \enquote{schlechte} Tage haben.}
Das ist wahrscheinlich den \enquote{Löchern} analog, in die Athleten fallen.
Tatsächlich kann man für ausgedehnte Perioden \enquote{schlechte Tage} haben.
Indem man sich dieses Phänomens bewußt ist und durch Experimentieren kann dieser Faktor in einem gewissen Ausmaß kontrolliert werden.
Das bloße Bewußtsein, daß solch ein Faktor existiert, kann einem Schüler helfen, besser mit diesen \enquote{schlechten} Tagen zurechtzukommen.
Professionelle Athleten, wie z.B. Golfer, diejenigen, die Meditation praktizieren, usw. wußten schon lange von der Wichtigkeit mentaler Konditionierung.
Die allgemeinen Ursachen solcher schlechten Tage zu kennen, wäre sogar noch hilfreicher.
Die häufigste Ursache ist \hyperref[fpd]{FPD (Schnellspiel-Abbau)}, der am Ende von Abschnitt II.25 besprochen wurde.
Eine weitere häufige Ursache ist ein Abweichen von den Grundlagen: Genauigkeit, Timing, Rhythmus, korrekte Ausführung der Ausdrucksbezeichnungen usw.
Zu schnell zu spielen oder mit zuviel Ausdruck kann der Konditionierung abträglich sein.
Mögliche Abhilfen sind, sich eine gute Aufnahme anzuhören, ein Metronom zu Hilfe zu nehmen oder sich die Notenblätter noch einmal anzusehen.
\textbf{Eine Komposition einmal langsam zu spielen bevor man aufhört, ist eine der effektivsten präventiven Maßnahmen gegen zukünftiges unerklärliches \enquote{schlechtes Spielen} dieser Komposition.}
Deshalb hängt die Konditionierung nicht nur davon ab, wie oft Sie üben, sondern auch davon, was und wie Sie üben.
%: bisher 7i Ende
Solides \hyperref[c1ii12]{mentales Spielen} kann Löcher vermeiden; zumindest können Sie mit ihm erkennen, daß Sie sich in einem Loch befinden, \textit{bevor} sie spielen.
Noch besser ist, daß Sie es dazu benutzen können, aus dem Loch herauszukommen.
Wir benutzen alle ein gewisses Maß an mentalem Spielen, ob wir es wissen oder nicht.
Wenn Sie das mentale Spielen nicht bewußt benutzen, dann kommen und gehen die Löcher, scheinbar ohne Grund, in Abhängigkeit vom Zustand Ihres mentalen Spielens.
Deshalb ist das mentale Spielen so wichtig, wenn Sie \hyperref[c1iii14]{vorspielen}.


\paragraph{Schnelle und langsame Muskeln}
\label{c1iii7aMuskeln}

Für die Technik ist es wichtig, den Unterschied zwischen Kontrolle und Geschwindigkeit einerseits sowie Fingerstärke andererseits zu verstehen.
\textbf{Alle Muskelstränge bestehen hauptsächlich aus schnellen und langsamen Muskeln.}
Die langsamen Muskeln dienen der Kraft und Ausdauer.
Die schnellen Muskeln sind für die Technik notwendig.
Je nachdem wie sie üben, wächst die eine Gruppe zu Lasten der anderen.
Offensichtlich möchte man, wenn man für die Technik übt, daß die schnellen Muskeln wachsen und die langsamen abnehmen.
\textbf{Deshalb sollte man isometrische oder Kraftübungen vermeiden.
Man möchte alle Bewegungen schnell ausführen, und die einzelnen Finger entspannen, sobald sie ihre Arbeit verrichtet haben.}
Deshalb kann jeder Klavierspieler einem Sumoringer auf der Tastatur \enquote{davonlaufen}, obwohl der Ringer mehr Muskeln hat.
Merken Sie sich dieses Konzept der schnellen Muskeln, da es dieser grundlegende schnelle Fingerschlag (auf- oder abwärts) ist, den Sie bei jeder hier besprochenen Übung trainieren müssen; sehen Sie dazu den \enquote{\hyperref[c1iii7i]{schnellen Anschlag}} in Abschnitt (i).

Natürlich brauchen wir eine gewisse Balance der schnellen und langsamen Muskeln, damit die Finger, Hände usw. richtig funktionieren, aber die Forschung auf diesem Gebiet ist für das Klavierspielen beklagenswert unzureichend.
Da diejenigen, die in der Vergangenheit Übungen entwickelten, kaum eine Vorstellung davon hatten oder über Forschungsergebnisse darüber verfügten, was Übungen bewirken müssen, waren die meisten dieser Übungen nur geringfügig hilfreich, und wie hilfreich sie waren hing mehr davon ab, wie man sie benutzte, als von ihrem ursprünglichen Aufbau.
Der Grundgedanke hinter den meisten Übungen war z.B., daß man Fingerstärke benötigt; wir wissen nun, daß das völlig falsch ist.
Ein weiteres Konzept war, daß man um so mehr Technik lernt, je schwieriger die Übung ist.
Das ist offensichtlich falsch; richtig ist nur, daß man, wenn man fortgeschritten ist, schwieriges Material spielen kann.
Einige der leichtesten Übungen können Ihnen die fortgeschrittensten Techniken lehren, und das ist die Art von Übungen, die am nützlichsten sind.



<!-- c1iii7b.html -->

\subsubsection{Parallele Sets}
\label{c1iii7b}

\textbf{Das Hauptziel von Übungen ist das Aneignen der Technik, was sich für alle Absichten und Zwecke auf Geschwindigkeit, Kontrolle und Klang reduzieren läßt.}
Damit Übungen nützlich sind, müssen sie dafür geeignet sein, Ihre Schwächen zu identifizieren und diese Fähigkeiten zu stärken.
Um das zu erreichen, \textbf{müssen wir einen vollständigen Satz Übungen haben, und sie müssen in einer logischen Reihenfolge angeordnet sein, so daß man leicht eine Übung ausfindig machen kann, die für einen bestimmten Zweck notwendig ist.}
Solch eine Übung muß deshalb auf einem grundlegenden Prinzip des Klavierspielens basieren, das alle Aspekte abdeckt.
Und wie \textit{identifizieren} wir unsere einzelnen Schwächen?
Die Tatsache, daß man etwas nicht spielen kann, sagt einem noch nicht, warum das so ist oder wie man das Problem lösen kann.

\textbf{Das Konzept der parallelen Sets stellt meines Erachtens den Rahmen für einen universellen Satz von Übungen zur technischen Entwicklung zur Verfügung.}
Das kommt daher, daß jede beliebige musikalische Passage aus Kombinationen paralleler Sets konstruiert werden kann (d.h. aus Gruppen von Noten, die unendlich schnell gespielt werden können).
Ich beschreibe im folgenden eine vollständige Gruppe von Übungen für parallele Sets, die alle diese Anforderungen erfüllt.
In \hyperref[c1ii11]{Abschnitt II.11} finden Sie eine Einführung zu den parallelen Sets.
Natürlich stellen parallele Sets alleine keinen vollständigen Satz von Übungen dar; Verbindungen, Wiederholungen, Sprünge, Dehnungen usw. werden auch benötigt.
Diese Themen werden hier ebenfalls behandelt.
Offenbar lehrte Louis Plaidy im späten 19. Jahrhundert Übungen, die den Übungen für parallele Sets ähnlich waren.

Alle Übungen für parallele Sets sind HS-Übungen, wechseln Sie deshalb häufiger die Hände.
Sie können sie jedoch jederzeit HT üben und in jeder miteinander vereinbaren Kombination, sogar 2 Noten gegen 3, usw.
Tatsächlich sind diese Übungen vielleicht der beste Weg, um solche ungleichen RH-LH-Kombinationen zu üben.
Probieren Sie zunächst jede der Übungen ein wenig aus, und lesen Sie dann in \hyperref[c1iii7c]{Abschnitt (c)}, wie man sie benutzt.
Wenn man sie erweitert, gibt es eine unendliche Zahl (was sie sein sollten, wenn sie vollständig sind), so daß Sie sie niemals alle üben werden.
Sie werden sowieso niemals alle benötigen, und wahrscheinlich sind mehr als die Hälfte davon redundant.
Benutzen Sie diese Übungen nur, wenn Sie sie brauchen (Sie werden sie \textit{ständig} brauchen!).
Zu diesem Zeitpunkt ist es nur erforderlich, daß Sie genügend vertraut mit ihnen werden, so daß Sie sofort eine bestimmte Übung abrufen können, wenn es notwendig wird.
So verschwenden Sie niemals Zeit mit unnötigen Übungen.

Ich wiederhole es noch einmal: \textbf{Übungen für parallele Sets sind nicht dafür gedacht, daß man sie wie \hyperref[c1iii7h]{Hanons Übungen} jeden Tag übt; sie sollen dazu benutzt werden, Ihre Schwächen zu diagnostizieren und sie zu korrigieren.}
Wenn die Schwierigkeiten beseitigt sind, benötigen Sie diese Übungen für parallele Sets nicht mehr.

\textbf{Die Übungen für parallele Sets werden stets Ihre Technik auf die Probe stellen.
Wenn Sie ein Anfänger ohne Technik sind, werden sie wahrscheinlich alle nicht ausführen können}.
Es wird für alle Übungen unmöglich sein, sie mit den erforderlichen Geschwindigkeiten zu spielen.
Die meisten Schüler werden zunächst keine Ahnung haben, wie man sie korrekt spielt.
Wenn Sie die Übungen noch nie durchgeführt haben, wäre es sehr hilfreich, wenn Sie sich ein paar von jemandem zeigen lassen könnten.
Mittelstufenschüler mit 2 bis 5 Jahren Unterricht sollten in der Lage sein, mehr als die Hälfte davon zufriedenstellend zu spielen.
Somit liefern diese Übungen ein Mittel zum Messen Ihres Fortschritts.
Das ist die völlige Entwicklung der Technik und schließt deshalb die Klangkontrolle und das musikalische Spielen ein, wie in Kürze erklärt wird.
Fortgeschrittene Schüler werden die Übungen immer noch benötigen, aber - anders als noch in der Entwicklung befindliche Schüler - nur kurz, oftmals nur für ein paar Sekunden während des Übens.


\paragraph{Übung \#1}
\label{c1iii7b1}

Diese Übung baut die grundlegende Bewegung auf, die für alle folgenden Übungen benötigt wird.
\textbf{Spielen Sie nur eine Note}, z.B. Finger 1 (Daumen der RH oder LH), als vier Wiederholungen: 1111.
In dieser Übung lernen wir nur, wie man eine \enquote{Sache} rasch wiederholt; dann ersetzen wir die \enquote{Sache} durch ein paralleles Set, so daß wir Zeit sparen können, indem wir so viele parallele Sets wie möglich innerhalb einer kurzen Zeitspanne spielen.
Denken Sie daran, daß ein wichtiger Grund für das Durchführen von Übungen das Zeitsparen ist.

Sie können das 1111 als Quadrupel gleicher Stärke spielen oder als Einheiten eines 4/4- oder 2/4-Takts.
\textbf{Die Idee ist, sie so schnell zu spielen wie Sie können, bis zu Geschwindigkeiten von mehr als einem Quadrupel je Sekunde.}
Wenn Sie ein Quadrupel zu Ihrer Zufriedenheit spielen können, versuchen Sie zwei: 1111,1111.
Das Komma repräsentiert eine Pause von willkürlicher Länge, die verkürzt werden sollte, wenn Sie Fortschritte machen.
Wenn Sie zwei schaffen, hängen Sie vier in schneller Folge aneinander: 1111,1111,1111,1111.
Sie \enquote{meistern} diese Übung, wenn sie vier Quadrupel in Folge mit ungefähr einem Quadrupel je Sekunde und ohne Pause zwischen den Quadrupeln spielen können.
Spielen Sie sie leise, entspannt und nicht staccato, wie weiter unten detaillierter erklärt wird.
Die Übung gilt als bewältigt, wenn Sie die Quadrupel so lange und so schnell spielen können wie Sie möchten und zwar mit völliger Kontrolle und ohne Ermüdung.
Diese scheinbar triviale Bewegung ist viel wichtiger als sie auf den ersten Blick erscheint, weil sie die Grundlage für alle geschwindigkeitsbezogenen Bewegungen ist.
Das wird offensichtlich, wenn wir zu parallelen Sets mit mehreren Fingern kommen, wie z.B. solchen in schnellen Alberti-Begleitungen oder Tremolos.
Deshalb widmen wir im folgenden dieser Übung so viele Absätze.

Wenn sich beim Aneinanderhängen der Quadrupel Streß aufbaut, arbeiten Sie weiter daran, bis Sie vier Quadrupel schnell und streßfrei spielen können.
Beachten Sie, daß jeder Teil Ihres Körpers einbezogen sein muß: Finger, Hand, Arm, Schulter usw., nicht nur die Finger.
Das bedeutet nicht, daß jeder Teil Ihres Körpers sich in einem sichtbaren Maß bewegen muß - sie mögen stillstehend erscheinen, aber sie müssen teilhaben.
Ein großer Teil des \enquote{Einbeziehens} wird das bewußte \hyperref[c1ii14]{Entspannen} sein, weil das Gehirn dazu tendiert, sogar für die kleinste Aufgabe zu viele Muskeln zu benutzen.
Versuchen Sie, nur die notwendigen Muskeln für die Bewegung zu isolieren, und entspannen Sie alle anderen.
Die endgültige Bewegung mag den Anschein erwecken, daß sich nur der Finger bewegt.
Aus ein paar Metern Entfernung werden wenige Menschen eine Bewegung von 1 mm erkennen; wenn jeder Teil Ihres Körpers sich weniger als 1 mm bewegt, kann sich das leicht zu den mehreren mm aufaddieren, die für den Anschlag notwendig sind, auch wenn dabei kein Finger bewegt wird.
Experimentieren Sie deshalb  für ein optimales Spielen mit unterschiedlichen Positionen Ihrer Hand, des Handgelenks usw.

Wenn Ihre Geschwindigkeit steigt, werden die Finger, Hände und Arme automatisch ideale Positionen einnehmen; ansonsten werden Sie nicht in der Lage sein, mit diesen Geschwindigkeiten zu spielen.
Diese Positionen gleichen denen berühmter Pianisten während des Spielens bei einem Konzert - letzten Endes ist das der Grund, daß sie es spielen können.
Deshalb ist es wichtig, beim Besuch eines Konzerts sein Opernglas mitzubringen und sich die Details der Bewegungen professioneller Pianisten anzusehen.
Die Positionen und Bewegungen der Hände und des Körpers sind für die unten vorgestellten fortgeschrittenen Übungen besonders wichtig.
Beginnen Sie deshalb \textit{jetzt} damit, das Erkennen dieser Verbesserungen zu trainieren.

Anfänger werden im ersten Jahr ihres Klavierunterrichts nicht in der Lage sein, ein Quadrupel je Sekunde zu spielen, und sollten mit geringeren Geschwindigkeiten zufrieden sein.
Zwingen Sie sich nicht, mit Geschwindigkeiten zu üben, die Sie nicht handhaben können.
Periodische kurze Exkursionen zu Ihrer schnellsten Geschwindigkeit sind jedoch für Erkundungszwecke notwendig.
Sogar Schüler mit mehr als fünf Jahren Unterricht werden Teile dieser Übungen schwierig finden.
Wenn Sie das hier zum ersten Mal lesen, üben Sie \hyperref[c1iii7b1]{Übung \#1} für eine Weile und dann \hyperref[c1iii7b2]{Übung \#2} (s.u.).
Wenn Sie mit \#2 Probleme bekommen, dann können Sie diese lösen, indem Sie wieder \#1 üben (Versuchen Sie es, es funktioniert!).
Sie können auch einen kurzen Blick auf die anderen Übungen werfen.
Es ist aber nicht notwendig, sie jetzt schon alle durchzuführen.
Wenn Sie mit dem Üben schwieriger Kompositionen beginnen, wird es oft genug notwendig werden, die Übungen durchzuführen, so daß Sie genügend Gelegenheit dazu haben werden.

\textbf{Üben Sie, bis jeglicher Streß verschwindet und Sie fühlen können, wie die Schwerkraft Ihren Arm nach unten zieht.}
Sobald sich Streß aufbaut, werden Sie den Zug durch die Schwerkraft nicht mehr fühlen.
Versuchen Sie nicht zu viele Quadrupel auf einmal, wenn Sie nicht das Gefühl haben, daß Sie die völlige Kontrolle haben.
Zwingen Sie sich nicht, unter Streß weiter zu üben, weil Üben unter Streß zur Gewohnheit werden kann, bevor man es merkt.
Wenn Sie unter Streß üben, werden Sie sogar anfangen langsamer zu werden.
Das ist ein deutliches Zeichen dafür, daß Sie entweder langsamer werden oder die Hände wechseln müssen.
Sie verbessern sich am schnellsten bei den höchsten Geschwindigkeiten, mit denen Sie zurechtkommen.
Spielen Sie zuerst nur ein Quadrupel sehr gut, bevor Sie ein weiteres hinzufügen - auf diese Weise werden Sie schneller vorankommen, als wenn Sie viele Quadrupel auf einmal herunterspielen.
Bei vier Quadrupeln kann man aufhören, denn wenn man vier spielen kann, dann kann man üblicherweise eine beliebige Anzahl hintereinander spielen.
Wie viele man \textit{genau} benötigt, bevor man eine unendliche Anzahl in Folge spielen kann, hängt jedoch von jedem einzelnen ab.
Wenn Sie bereits nach zwei Quadrupeln eine unendliche Anzahl mit jeder gewünschten Geschwindigkeit spielen können, dann haben Sie den Test für Übung \#1 ebenfalls bestanden, und Sie müssen sie \textit{niemals} mehr durchführen!
Diese Übungen unterscheiden sich völlig von den \hyperref[c1iii7h]{Hanon-Übungen}, die man jeden Tag wiederholen muß; sobald Sie den Test bestanden haben, machen Sie mit etwas schwierigerem weiter - das ist Fortschritt.
An den ersten paar Übungstagen sollten Sie während des Übens einige Verbesserungen bemerken, da Sie schnell neue Bewegungen lernen und falsche eliminieren.
Um weitere Fortschritte zu machen, müssen Sie wegen des notwendigen Wachstums von Muskeln und Nerven die \hyperref[c1ii15]{PPI} benutzen.
Anstatt \textit{während} des Übens auf die Geschwindigkeit zu drängen, warten Sie, bis die Hand die Schnelligkeit von selbst entwickelt, so daß Sie schneller spielen, wenn Sie das \textit{nächste Mal} üben; das kann geschehen, wenn Sie die Hände wechseln oder wenn Sie am folgenden Tag erneut üben.
\textbf{Nach der ersten oder zweiten Woche wird sich die Schnelligkeit überwiegend zwischen den Übungen und nicht während des Übens entwickeln}; lassen Sie sich deshalb nicht entmutigen, wenn Sie sich scheinbar auch nach einem harten Training nicht viel verbessern - das ist normal.
Der größte Teil des Wachstums der Muskeln und Nerven scheint im Schlaf zu geschehen, wenn die Ressourcen des Körpers nicht für die Tagaktivitäten benötigt werden.
Deshalb sollten Sie nach einem guten Nachtschlaf die beste PPI bemerken.
Beim Versuch, während des Übens eine sichtbare Verbesserung zu erzielen, zuviel zu üben, ist eine der Hauptursachen für \hyperref[c1iv2b]{Geschwindigkeitsbarrieren}, \hyperref[c1iii10hand]{Verletzungen} und Streß.
\textbf{Ihre Aufgabe während des Übens ist die Konditionierung der Hand für eine maximale PPI.}
Die Konditionierung erfordert nur ungefähr hundert Wiederholungen; darüber hinaus beginnt der Gewinn je Wiederholung zu sinken, und die Konditionierung für die PPI wird nur unbedeutend gesteigert.

Es geht hier um das Aneignen von Technik, nicht um Muskelaufbau.
\textbf{Technik bedeutet Musik machen, und diese Übungen sind für die Entwicklung eines musikalischen Spielens wertvoll.}
Hämmern Sie nicht drauflos wie ein Wilder.
Wenn Sie den Klang einer Note nicht kontrollieren können, wie wollen Sie ihn dann bei mehreren Noten kontrollieren?
Ein wesentlicher Trick, um den Klang zu kontrollieren, ist, leise zu üben.
Indem Sie leise spielen, bringen Sie sich selbst aus dem Übungszustand, in dem Sie die Natur des Klangs völlig ignorieren, draufloshämmern und nur versuchen, die Wiederholungen zu erreichen.
Drücken Sie die Taste völlig nieder, und halten Sie sie einen Moment unten (ganz kurz - nur den Bruchteil einer Sekunde).
Das stellt sicher, daß der Fänger den Hammer greift und die Schwingungen stoppt, die dieser aufnimmt, wenn er von den Saiten abspringt.
Wenn diese Schwingungen nicht beseitigt werden, kann man den nächsten Anschlag nicht kontrollieren.
\textbf{Lesen Sie Abschnitt III.4b über das \hyperref[c1iii4b]{Spielen mit flachen Fingern}; dieser Abschnitt ist obligatorischer Stoff, bevor Sie eine der Übungen für parallele Sets ernsthaft ausführen.}

\textbf{Halten Sie den spielenden Finger zu jeder Zeit so nah wie möglich über der Taste}, um die Geschwindigkeit und Genauigkeit zu steigern und den Klang zu kontrollieren.
Wenn der Finger die Taste nicht hin und wieder berührt, verlieren Sie die Kontrolle.
Lassen Sie die Finger nicht die ganze Zeit auf den Tasten ruhen, sondern berühren Sie die Tasten so leicht Sie können, so daß Sie wissen, wo sie sind.
Das wird Ihnen ein zusätzliches Gefühl dafür geben, wo all die anderen Tasten sind, und wenn die Zeit kommt sie zu spielen, werden Ihre Finger nicht die falschen Tasten anschlagen, weil Sie wissen, wo die richtigen Tasten sind.
\textbf{Ermitteln Sie den minimalen Tastenhub, der für die Wiederholung notwendig ist, und üben Sie, mit dem geringstmöglichen Tastenhub zu spielen.}
Der Tastenhub ist für Klaviere größer als für Flügel.
Mit kleineren Tastenhüben können Sie schnellere Geschwindigkeiten erzielen.

\textbf{Es ist wichtig, das Handgelenk in die Bewegung für die Wiederholung einzubeziehen.}
Das Handgelenk bestimmt alle drei Ziele, die wir verfolgen: Geschwindigkeit, Kontrolle und Klang.
Erinnern Sie sich daran, daß \enquote{das Handgelenk einbeziehen} keine übertriebene Bewegung des Handgelenks bedeutet; seine Bewegung ist eventuell nicht wahrnehmbar, da Sie den Impuls und nicht die Bewegung des Handgelenks benötigen.

\textbf{Arbeiten Sie an der Kontrolle und dem Klang, anstatt immer an der Geschwindigkeit zu arbeiten.
Wiederholungen, die für die Kontrolle und den Klang geübt werden, zählen ebenfalls für die Konditionierung der Geschwindigkeit.}
Sowohl Kontrolle als auch Geschwindigkeit erfordern dieselbe Fertigkeit: Genauigkeit.
Schnelles Üben unter Streß konditioniert lediglich für ein gestreßtes Spielen, was die Bewegung sogar verlangsamt.

Sie müssen ständig nachforschen und experimentieren.
Ist der Klang unterschiedlich, wenn Sie Ihre Fingerspitze an einem Punkt halten oder wenn Sie sie leicht über die Taste gleiten lassen?
Üben Sie, Ihre Finger vorwärts (zum Klavier hin) und rückwärts (zu Ihrem Körper hin) gleiten zu lassen.
Der Daumen ist eventuell der am leichtesten gleitende Finger.
\textbf{Spielen Sie mit der Spitze des Daumens, nicht mit dem Gelenk}; das wird Sie in die Lage versetzen, den Daumen gleiten zu lassen und Ihre Hand oder Ihr Handgelenk anzuheben, und somit die Chancen verringern, daß die anderen Finger aus Versehen einige Tasten anschlagen.
Mit der Spitze zu spielen erhöht auch die effektive Reichweite und Geschwindigkeit der Daumenbewegung, d.h. bei unveränderter Bewegung des Daumens bewegt sich die Spitze weiter und schneller als das Gelenk.
Wenn Sie wissen, wie man die Finger gleiten läßt, können Sie auch dann mit Selbstvertrauen spielen, wenn die Tasten schlüpfrig oder vom Schwitzen feucht sind.
\textbf{Lassen Sie Ihre Fähigkeit zu spielen nicht von der Griffigkeit der Tastenoberfläche abhängig werden, da diese nicht immer gegeben ist.}
Mit angehobenem Handgelenk zu spielen führt dazu, daß die Finger zu Ihnen hin gleiten.
Wenn Sie das Handgelenk absenken, tendieren die Finger dazu, von Ihnen weg zu gleiten, besonders die Finger 2 bis 5.
Üben Sie jede dieser gleitenden Bewegungen: Üben Sie für eine Weile alle fünf Finger mit angehobenem Handgelenk; wiederholen Sie es dann mit gesenktem Handgelenk.
Bei einer mittleren Höhe des Handgelenks werden die Finger nicht gleiten, auch nicht wenn die Tasten schlüpfrig sind.

Experimentieren Sie mit der Kontrolle des Klangs, indem Sie absichtlich etwas gleiten.
Gleiten steigert die Kontrolle, weil man mit einer größeren Bewegung einen geringen Tastenweg erzeugt.
Das Ergebnis ist, daß jegliche Fehler in der Bewegung um das Verhältnis von Tastenweg zu totaler Bewegung vermindert werden, was immer kleiner als 1 ist.
Deshalb können Sie mittels Gleiten gleichmäßigere Quadrupel spielen, als wenn Sie gerade herunterkommen.
Sie können auch leiser spielen.
Das Gleiten vereinfacht auch die Fingerbewegung, weil der Finger nicht gerade herunter kommen muß - jede Bewegung mit einer abwärts gerichteten Komponente wird ausreichen, was Ihre Optionen vermehrt.

Wiederholen Sie es mit allen anderen Fingern.
Schüler, die diese Übung das erste Mal ausführen, sollten finden, daß einige Finger (typischerweise 4 und 5) schwieriger als die anderen sind.
Das ist ein Beispiel, wie man diese Übungen als Diagnosewerkzeug benutzt, um schwache Finger aufzuspüren.

Das \hyperref[c2_7_hamm]{Intonieren des Klaviers (der Hämmer)} ist für die richtige Ausführung dieser Übungen entscheidend.
Das gilt sowohl für den Erwerb neuer Fertigkeiten als auch für das Vermeiden des unmusikalischen Spielens.
Mit abgenutzten Hämmern, die intoniert werden müssen, ist es unmöglich, leise (oder kraftvolle oder tiefe) musikalische Töne zu erzeugen.


\paragraph{Übung \#2}
\label{c1iii7b2}

\textbf{Übungen für parallele Sets mit 2 Fingern:} spielen Sie 12 (RH: Daumen auf C, gefolgt vom Zeigefinger auf D) so schnell Sie können, wie eine Vorschlagsnote.
Die Idee ist, es schnell aber völlig kontrolliert zu spielen.
Offensichtlich werden hier die Methoden von Abschnitt I und II gebraucht.
Wenn z.B. die RH eine Übung leicht ausführen kann aber die entsprechende Übung für die LH schwierig ist, \hyperref[c1ii20]{benutzen Sie die RH um die LH zu unterrichten}.
Üben Sie sowohl mit dem Schlag auf der 1 als auch mit dem Schlag auf der 2.
Wenn das zufriedenstellend ist, spielen Sie ein Quadrupel wie in Übung \#1: 12,12,12,12.
Wenn Sie Schwierigkeiten damit haben, ein Quadrupel aus parallelen Sets mit 12 zu beschleunigen, spielen Sie die beiden Noten gleichzeitig als \enquote{Akkord} und üben Sie dieses Akkord-Quadrupel so wie Sie das aus einer Note bestehende Quadrupel in \hyperref[c1iii7b1]{Übung \#1} geübt haben.
Bringen Sie das Quadrupel wieder auf Geschwindigkeit, d.h. ungefähr ein Quadrupel pro Sekunde.
Verbinden Sie dann vier Quadrupel in Folge.
Wiederholen Sie die ganze Übung jeweils mit 23, 34 und 45.
Dann nach unten: 54, 43 usw.
\textbf{Alle Anmerkungen darüber, wie man für Übung \#1 übt, gelten hier ebenfalls.}

Bei dieser und den folgenden Übungen sind die Anmerkungen der vorangegangenen Übungen fast immer auf die nachfolgenden Übungen anwendbar und werden im allgemeinen nicht wiederholt.
Auch werde ich nur repräsentative Mitglieder einer Familie von Übungen auflisten und es dem Leser überlassen, alle anderen Familienmitglieder herauszufinden.
Die gesamte Anzahl der Übungen ist viel größer als man am Anfang denkt.
Wenn man z.B. versuchen würde, verschiedene Übungen für parallele Sets HT zu kombinieren, würde die Zahl der Möglichkeiten schnell die Vorstellungskraft übersteigen.
Für Anfänger, die Schwierigkeiten mit dem HT-Spielen haben, könnten diese Übungen die beste Art darstellen, das HT-Spielen zu üben.

Alle Noten eines parallelen Sets müssen so schnell wie möglich gespielt werden, weil parallele Sets hauptsächlich für die Entwicklung der Geschwindigkeit benutzt werden.
Ein Zweck paralleler Sets ist, das Gehirn das Konzept der extremen - fast bis zur unendlichen - Geschwindigkeit zu lehren.
Es stellt sich heraus, daß sobald das Gehirn sich an eine bestimmte Maximalgeschwindigkeit gewöhnt hat, alle langsameren Geschwindigkeiten einfacher auszuführen sind.

Führen Sie die Übungen zunächst nur unter Benutzung der weißen Tasten aus.
Haben Sie alle Übungen mit den weißen Tasten durchgeführt, arbeiten Sie mit ähnlichen Übungen, bei denen Sie die schwarzen Tasten einschließen.

Am Anfang werden Sie vielleicht in der Lage sein, die zwei Noten hintereinander sehr schnell aber ohne viel unabhängige Kontrolle zu spielen.
Sie können am Anfang \enquote{schummeln} und die Geschwindigkeit erhöhen, indem Sie die beiden Finger \enquote{phasenkoppeln}, z.B. die Finger in einer festen Position halten und einfach die Hand senken, um die zwei Noten zu spielen.
Erinnern Sie sich daran, daß der \hyperref[c1iv2a]{Phasenwinkel} der Abstand zwischen aufeinanderfolgenden Fingern beim parallelen Spielen ist.
\textbf{Aber irgendwann müssen Sie mit Unabhängigkeit der Finger spielen.
Die anfängliche Phasenkoppelung wird nur benutzt, um schnell auf Geschwindigkeit zu kommen.
Das ist ein Grund, warum einige Lehrer das parallele Spielen nicht lehren, weil sie der Meinung sind, daß paralleles Spielen mit Phasenkopplung gleichzusetzen ist, was die Musik zerstören würde.}
Der Grund für dieses Problem ist, daß beide Finger nach dem Spielen mit Phasenkopplung auf ihren Tasten liegenbleiben und die beiden Noten sich überlappen.
Es ist genauso wichtig, den Finger zu einem bestimmten Zeitpunkt zu heben, wie ihn zu einem bestimmten Zeitpunkt zu senken.
Zum Spielen mit unabhängigen Fingern muß der Finger genau dann gehoben werden, wenn der zweite Finger spielt, so daß aufeinanderfolgende Noten eindeutig getrennt werden.
Den Finger zu senken und ihn unten zu behalten führt dazu, daß der Fänger den Hammer greift und ihn so unter Kontrolle hält.
Wenn man dann den Finger hebt, kommt der Dämpfer herunter und stoppt den Ton, was die Dauer des Tons kontrolliert.
Deshalb werden die parallelen Sets in Quadrupeln gespielt; am Ende des Quadrupels muß man den letzten Finger heben und kontrolliert somit das Ende des Quadrupels präzise.

Können Sie erst einmal entspannte, schnelle parallele Sets spielen, müssen Sie langsamer werden und daran arbeiten, jede einzelne Note korrekt zu spielen.
Anfänger werden Schwierigkeiten damit haben, die Finger zum richtigen Zeitpunkt anzuheben, um die Notendauer zu kontrollieren.
In diesem Fall können Sie entweder abwarten, bis sich Ihre Technik weiterentwickelt oder die Übungen zum Anheben der Finger unten in \hyperref[c1iii7d]{Abschnitt d)} durchführen.


\paragraph{Übung \#3}
\label{c1iii7b3}

\textbf{Größere parallele Sets:} z.B. 123 und seine Familie, 234 usw.
Wiederholen Sie alle Prozeduren wie in \hyperref[c1iii7b2]{Übung \#2}.
Arbeiten Sie dann mit der 1234-Gruppe und schließlich mit den 12345-Sets.
Bei diesen großen Sets müssen Sie vielleicht die Wiederholungsgeschwindigkeit der Quadrupel ein wenig reduzieren.
\textbf{Die Zahl der möglichen Übungen ist für diese größeren Sets sehr groß.}
Der Schlag kann auf jeder Note sein, und Sie können mit jeder Note beginnen.
123 kann z.B. als 231 und 312 geübt werden.
Wenn man abwärts spielt, kann die 321 als 213 oder 132 gespielt werden; alle sechs sind unterschiedlich, weil Sie feststellen werden, daß einige leicht und andere schwierig sind.
Wenn man die Schlagvarianten einschließt, gibt es bereits für drei Finger auf den weißen Tasten 18 Übungen.


\paragraph{Übung \#4}
\label{c1iii7b4}

\textbf{Erweiterte parallele Sets:} beginnen Sie mit den zweinotigen Sets 13, 24 usw. (die Terz-Gruppe).
Diese Sets schließen die Gruppen vom Typ 14 (Quarte) und 15 (Quinte und Oktave) ein.
Dann gibt es die dreinotigen erweiterten parallelen Sets: 125-, 135-, 145-Gruppen (Quinte und Oktave).
Hier haben Sie eventuell mehrere Möglichkeiten für die mittlere der drei Noten.
Außerdem gibt es erweiterte Sets, die mit 12 gespielt werden: Terzen, Quarten, Quinten usw.; diese können auch mit 13 usw. gespielt werden.


\paragraph{Übung \#5}
\label{c1iii7b5}

\textbf{Die zusammengesetzten parallelen Sets:} 1.3,2.4, wobei 1.3 und 2.4 jeweils zwei gleichzeitig gespielte Noten, z.B. CE, darstellen.
Üben Sie dann die 1.4,2.5 Gruppe.
Ich habe oft Sets gefunden, die leicht aufwärts aber schwer abwärts zu spielen sind oder umgekehrt.
1.3,2.4 ist für mich z.B. einfacher als 2.4,1.3.
\textbf{Diese zusammengesetzten Sets erfordern einiges an Geschicklichkeit.}
Solange Sie nicht mindestens einige Jahre Unterricht hatten, sollten Sie nicht erwarten, sie mit einer gewissen Fertigkeit spielen zu können.

Das ist das Ende der wiederholenden Quadrupel-Übungen, die auf \hyperref[c1iii7b1]{Übung \#1} basieren.
Im Prinzip sind die Übungen \#1 bis \#5 die einzigen Übungen, die Sie benötigen, weil man sie benutzen kann, um die im folgenden besprochenen parallelen Sets zu konstruieren.
Die Übungen \#6 und \#7 sind zu komplex, um sie in schnellen Quadrupeln zu wiederholen.


\paragraph{Übung \#6}
\label{c1iii7b6}

\textbf{Komplexe parallele Sets:} Diese werden am besten einzeln geübt anstatt als schnelle Quadrupel.
In den meisten Fällen sollten sie in einfachere parallele Sets aufgeteilt werden, die als Quadrupel geübt werden können - zumindest am Anfang.
\enquote{Abwechselnde Sets} sind vom Typ 1324, und \enquote{gemischte Sets} sind vom Typ 1342, 13452 usw., d.h. Mischungen aus abwechselnden und normalen Sets.
Es gibt natürlich eine große Zahl davon.
\textbf{Die meisten der komplexen parallelen Sets, die technisch wichtig sind, können in Bachs Unterrichtsstücken gefunden werden, besonders in den \hyperref[c1iii20]{zweistimmigen Inventionen}} (s. Abschnitt III.20).
Deshalb sind Bachs Unterrichtsstücke - im Gegensatz zu \hyperref[c1iii7h]{Hanon} - einige der besten Übungsstücke für das Erwerben der Technik.


\paragraph{Übung \#7}
\label{c1iii7b7}

Üben Sie nun \textbf{verbundene parallele Sets}, z.B. 1212, die eine oder mehrere Verbindungen enthalten.
Das kann entweder ein Triller (CDCD) oder ein Lauf (CDEF, bei dem Sie den \hyperref[c1iii5a]{Daumen übersetzen} müssen) sein.
Der 1212 Triller unterscheidet sich von \hyperref[c1iii7b2]{Übung \#2}, weil das 12 Intervall in jener Übung so schnell wie möglich gespielt werden muß, während das nachfolgende 21 Intervall langsamer sein kann.
Hier muß das Intervall zwischen den Noten immer das gleiche sein.
Nun können diese Sets nicht unendlich schnell gespielt werden, weil die Geschwindigkeit durch Ihre Fähigkeit, die Sets zu verbinden, begrenzt ist.
Das Ziel ist hier immer noch Geschwindigkeit - wie schnell Sie sie akkurat und entspannt spielen können und wie viele Sie zusammenbinden können.
Das ist eine Übung, um zu lernen, wie man Verbindungen spielt.
Spielen Sie während einer Bewegung der Hand so viele Noten wie möglich.
Üben Sie z.B., 1212 mit einer Abwärtsbewegung der Hand zu spielen.
Üben Sie dann, zwei davon mit einer Abwärtsbewegung zu spielen usw., bis sie vier aufeinanderfolgend mit einer Bewegung schaffen.

\textbf{Für ein schnelles Spielen sind die ersten beiden Noten die wichtigsten}; sie müssen mit der richtigen Geschwindigkeit begonnen werden.
Es mag hilfreich sein, nur die ersten beiden Noten phasengekoppelt zu spielen, um sicherzustellen, daß diese korrekt anfangen.
Wenn die ersten beiden Noten mit hoher Geschwindigkeit begonnen werden, dann folgt der Rest meistens leichter.

\textbf{Benutzen Sie bei Sets, die den Daumen beinhalten, den \hyperref[c1iii5a]{Daumenübersatz}, um sie zu verbinden, außer in besonderen Situationen, in denen der Daumenuntersatz benötigt wird (sehr wenige).}
Erforschen Sie verschiedene Verbindungsbewegungen, um zu sehen, welche am besten funktionieren.
Eine kleine \hyperref[Rollung]{Rollung} des Handgelenks ist eine nützliche Bewegung.
Um Sets zu verbinden, die nicht den Daumen einbeziehen, kreuzen Sie fast immer darüber, nicht darunter.
Viele dieser daumenlosen Überkreuzbewegungen sind jedoch von fragwürdigem Wert, weil man sie selten benötigt.

\textbf{Verbundene parallele Sets sind das Hauptübungselement in Bachs zweistimmigen Inventionen.}
Halten Sie deshalb in diesen Inventionen nach einigen der einfallsreichsten und technisch wichtigsten verbundenen parallelen Sets Ausschau.
Wie in \hyperref[c1iii19c]{Abschnitt III.19c} erklärt, ist es für viele Schüler oft extrem schwierig, bestimmte Kompositionen von Bach auswendig zu lernen und sie jenseits einer bestimmten Geschwindigkeit zu spielen.
Deshalb werden Stücke von Bach oft nicht so gerne gespielt, und diese wertvolle Quelle für das Erwerben der Technik wird nur begrenzt benutzt.
Wenn man sie jedoch im Hinblick auf die parallelen Sets analysiert und gemäß der Methoden dieses Buchs lernt, sind solche Kompositionen im allgemeinen leicht zu lernen.
Deshalb sollte dieses Buch die Beliebtheit der Bachstücke in hohem Maß steigern.
Sehen Sie dazu in \hyperref[c1iii19c]{Abschnitt III.19c} weitere Erklärungen darüber, wie man Bach übt.

Die nahezu unendliche Zahl an notwendigen Übungen für parallele Sets zeigt, wie beklagenswert unzulänglich die älteren Übungen sind (z.B. \hyperref[c1iii7h]{Hanon} - ich werde Hanon als einen gattungsmäßigen Vertreter davon benutzen, was hier als die \enquote{falsche} Art von Übung angesehen wird).
Es gibt jedoch einen Vorteil der Übungen vom Hanon-Typ: Sie beginnen mit den am meisten anzutreffenden Fingersätzen und den leichtesten Übungen, d.h. sie sind gut geordnet.
Die Chancen sind jedoch fast 100\%, daß sie wenig hilfreich sind, wenn man in einem beliebigen Musikstück auf einen schwierigen Abschnitt trifft.
Das Konzept der parallelen Sets erlaubt uns, die einfachste mögliche Serie von Übungen zu identifizieren, die einen vollständigeren Satz bilden, der auf praktisch alles anwendbar ist, dem man begegnen kann.
Sobald diese Übungen jedoch ein wenig komplex werden, wird ihre Anzahl unüberschaubar groß.
Wenn Sie zur Komplexität selbst der einfachsten Hanon-Übung kommen, wird die Zahl der möglichen Übungen für parallele Sets widerspenstig groß.
Sogar Hanon erkannte diese Unzulänglichkeit und schlug Variationen vor, wie z.B. die Übungen in allen möglichen Transpositionen zu üben.
Das ist sicher hilfreich, aber es fehlen immer noch ganze Kategorien, wie z.B. \hyperref[c1iii7b1]{Übung \#1} und \hyperref[c1iii7b2]{Übung \#2} (die grundlegendsten und nützlichsten) oder die unglaublichen Geschwindigkeiten, die wir ohne weiteres mit den Übungen für parallele Sets erreichen können.
\textbf{Beachten Sie, daß die Übungen \#1 bis \#4 einen kompletten Satz von rein parallelen Übungen (ohne Verbindungen) bilden.}
Intervalle, die größer sind als das, was man als Akkord erreichen kann, fehlen auf der Liste der hier beschriebenen parallelen Sets, weil sie nicht unendlich schnell gespielt werden können und den Sprüngen zugeordnet werden müssen.
Methoden für das \hyperref[c1iii7f]{Üben von Sprüngen} werden weiter unten in Abschnitt (f) besprochen.

Es ist leicht, Hanon zu erstaunlichen Geschwindigkeiten zu bringen, indem man die Methoden dieses Buchs benutzt.
Sie könnten es zum Spaß versuchen, werden sich aber schnell fragen, wozu es gut sein soll.
Sogar diese hohen Geschwindigkeiten können nicht an das heranreichen, was man ohne weiteres mit parallelen Sets erreichen kann, weil jede Hanon-Übung mindestens eine Verbindung enthält und deshalb nicht unendlich schnell gespielt werden kann.
\textbf{Das ist klarerweise der größte Vorteil der Übungen für parallele Sets: Es gibt keine Geschwindigkeitsbegrenzung, weder in der Theorie noch in der Praxis, und gestattet Ihnen deshalb, die Geschwindigkeiten in ihrem gesamten Umfang ohne Beschränkungen und ohne Streß zu erforschen.}
Wie bereits gesagt, ist die Hanon-Reihe aufsteigend nach Schwierigkeit geordnet, und diese Steigerung wird hauptsächlich durch das Einschließen von mehr Verbindungen und schwierigeren parallelen Sets erzeugt.
Bei den Übungen für parallele Sets werden diese einzelnen \enquote{Elemente der Schwierigkeit} ausdrücklich auseinandergehalten, so daß man sie getrennt üben kann.

Stellen Sie sich zur Verdeutlichung der Nützlichkeit dieser Übungen vor, daß Sie einen zusammengesetzten vierfingrigen Triller basierend auf \hyperref[c1iii7b5]{Übung \#5} üben möchten (z.B. C.E,D.F,C.E,D.F, . . .).
Indem Sie die Übungen in der Reihenfolge von \#1 bis \#7 befolgen, haben Sie nun ein schrittweises Rezept für das Diagnostizieren Ihrer Schwierigkeiten und um sich diese Fertigkeit anzueignen.
Stellen Sie zunächst sicher, daß Ihre zweinotigen Akkorde gleichmäßig sind, indem Sie die Übungen \#1 und \#2 anwenden.
Versuchen Sie dann 1.3,2 und dann 1.3,4.
Wenn diese zufriedenstellend sind, dann versuchen Sie 1.3,2.4.
Arbeiten Sie dann an den umgekehrten Richtungen: 2.4,1 und 2.4,3. Und schließlich 2.4,1.3.
Der Rest sollte offensichtlich sein, wenn Sie bis hierhin gelesen haben.
Das kann ein hartes Training sein; denken Sie deshalb daran, oft die Hände zu wechseln, bevor eine Ermüdung einsetzt.

\textbf{Es soll hier noch einmal betont werden, daß in den Methoden dieses Buchs kein Platz für stupide wiederholende Übungen ist.}
Solche Übungen haben einen weiteren heimtückischen Nachteil.
Viele Klavierspieler benutzen sie, um \enquote{sich aufzuwärmen} und für das Spielen in gute Form zu kommen.
Das kann jemandem den falschen Eindruck vermitteln, daß diese wunderbare Form eine Konsequenz des stupiden Übens sei.
Sie ist es nicht; die \enquote{aufgewärmte Form} ist dieselbe, unabhängig davon, wie man zu ihr gelangt ist.
Deshalb kann man die Fallen der stupiden Übungen vermeiden, indem man nützlichere Möglichkeiten benutzt, um die Hände aufzuwärmen.
\hyperref[c1iii5a]{Tonleitern} sind für das Lockern der Finger hilfreich und \hyperref[c1iii5e]{Arpeggios} für das Lockern der Handgelenke.
Und sie dienen dem Lernen einiger grundlegender Fertigkeiten, wie wir oben in \hyperref[c1iii5]{Abschnitt 5} gesehen haben.


\subsubsection{Wie verwendet man die Übungen für parallele Sets (Appassionata, 3. Satz)?}
\label{c1iii7c}

\textbf{Die Übungen für parallele Sets sind nicht dafür gedacht, \hyperref[c1iii7h]{Hanon}, Czerny usw. oder irgendeine Art von Übung zu ersetzen.}
Die Philosophie dieses Buchs ist, daß die Zeit besser damit verbracht werden kann, \enquote{wahre} Musik zu üben als \enquote{Übungsmusik}.
Die Übungen für parallele Sets wurden eingeführt, weil es keinen bekannten schnelleren Weg gibt, um Technik zu erwerben.
Deshalb sind technische Stücke, wie Liszt- oder Chopin-Etüden oder die Bach-Inventionen, keine \enquote{Übungsmusik} in diesem Sinne.
\textbf{Die Übungen für parallele Sets sind folgendermaßen zu benutzen:}

\begin{enumerate}[label={\roman*.}] 
\item \textbf{Zu Diagnosezwecken:} Indem Sie diese Übungen systematisch durchgehen, können Sie viele Ihrer Stärken und Schwächen erkennen.
Wenn Sie zu einer Passage kommen, die Sie nicht spielen können, haben Sie mit den Übungen eine Methode, um genau zu ermitteln, warum Sie die Passage nicht spielen können.
Im nachhinein erscheint es offensichtlich, daß man ein gutes Diagnosewerkzeug braucht, wenn man ein technisches Detail verbessern möchte.
Ansonsten ist es so, als ob man für eine Operation in eine Klinik gehen würde, ohne zu wissen welche Krankheit man hat.
Gemäß dieser medizinischen Analogie ist Hanon zu üben so, als wenn man in ein Krankenhaus gehen und täglich die einfachsten Routineuntersuchungen durchführen lassen würde.
Die Fähigkeit der parallelen Sets zur Diagnose ist am nützlichsten, wenn man eine schwierige Passage übt.
Es hilft Ihnen, genau zu bestimmen welche Finger schwach, langsam, unkoordiniert usw. sind.


\item \textbf{Für das Erwerben von Technik:} Die in (i) gefundenen Schwächen können nun durch die Benutzung genau derselben Übungen korrigiert werden, die zu ihrer Diagnose benutzt wurden.
Sie arbeiten einfach an den Übungen, die die Probleme offenbart haben.
Im Prinzip hören diese Übungen niemals auf, weil die Obergrenze der Geschwindigkeit offen ist.
In der Praxis enden sie jedoch bei Geschwindigkeiten um ein Quadrupel je Sekunde, weil wenige Stücke (wenn überhaupt eines) höhere Geschwindigkeiten erfordern.
In den meisten Fällen können Sie diese hohen Geschwindigkeiten nicht benutzen, sobald Sie nur eine Verbindung hinzufügen.
Das zeigt das Gute dieser Übungen, indem sie Ihnen erlauben, mit Geschwindigkeiten zu üben, die schneller sind als das, was Sie benötigen werden, und Ihnen so diesen zusätzlichen Spielraum an Sicherheit und Kontrolle verleihen.
Sie sollten diese Übungen während des HS-Übens am meisten benutzen, wenn Sie die Geschwindigkeit höher als die endgültige Geschwindigkeit bringen.


\end{enumerate}
\textbf{Die Prozeduren (i) und (ii) sind alles, was Sie brauchen, um die meisten Probleme beim Spielen von schwierigem Material zu lösen.}
Haben Sie sie erst erfolgreich auf mehrere zuvor \enquote{unmögliche} Situationen angewandt, werden Sie die Gewißheit erlangen, daß - innerhalb eines vernünftigen Rahmens - nichts unbezwingbar ist.

Nehmen Sie als Beispiel eine der schwierigsten Passagen des dritten Satzes von Beethovens Appassionata: die LH-Begleitung zu dem höhepunkthaften Lauf der RH in Takt 63 und ähnliche nachfolgende Passagen.
Wenn Sie sich Aufnahmen davon sorgfältig anhören, werden Sie feststellen, daß sogar die berühmtesten Pianisten Schwierigkeiten mit dieser LH haben und dazu neigen, sie langsam zu beginnen und dann zu beschleunigen oder sogar die Noten vereinfachen.
Diese Begleitung besteht aus den zusammengesetzten parallelen Sets 2.3,1.5 und 1.5,2.3, wobei 1.5 eine Oktave ist.
Die erforderliche Technik zu erwerben reduziert sich einfach dazu, diese parallelen Sets zu perfektionieren und sie dann zu verbinden.
Für viele wird eines der beiden parallelen Sets schwierig sein, und dieses müssen Sie meistern.
Zu versuchen es zu lernen, indem man es HT erst langsam spielt und dann beschleunigt, würde viel länger dauern.
Tatsächlich garantiert bloßes Wiederholen nicht, daß Sie jemals erfolgreich sein werden, weil es zu einem Rennen zwischen dem Erfolg und dem Aufbau einer Geschwindigkeitsbarriere wird, wenn Sie versuchen schneller zu werden.
Sie müssen HS üben und häufig die Hände abwechseln, um Streß und Ermüdung zu vermeiden.
Sie müssen am Anfang auch leise üben, damit Sie lernen zu \hyperref[c1ii14]{entspannen}.
Ohne parallele Sets besteht eine hohe Wahrscheinlichkeit, daß Sie streßbeladene Angewohnheiten entwickeln und eine Geschwindigkeitsbarriere erzeugen.
Ist diese Geschwindigkeitsbarriere erst einmal aufgebaut, können Sie Ihr ganzes Leben üben, ohne eine Verbesserung zu erzielen.
Wenn Sie ohne parallele Sets üben, können Sie zudem nicht herausfinden, welcher Teil dieses Abschnitts Sie davon abhält, Fortschritte zu machen.

Ein weiterer Grund für die Geschwindigkeitsbarriere ist das \enquote{Spielen unter Zwang}.
Der Rest des dritten Satzes läßt sich größtenteils einfach auf Geschwindigkeit bringen, weshalb man dazu neigt, ihn mit der endgültigen Geschwindigkeit zu spielen, und dann zu versuchen, sich ohne die erforderliche Technik durch diese schwierige Passage \enquote{hindurchzuzwängen}.
Das daraus resultierende streßbeladene Spielen erzeugt eine Geschwindigkeitsbarriere.
Dieses Beispiel zeigt, wie wichtig es ist, die schwierigen Teile als erste zu üben.
In dieser Situation wird jemand, der weiß, wie man üben muß, langsamer werden und somit nicht jenseits seiner Fähigkeiten spielen müssen.

\textbf{Um es zusammenzufassen: Die Übungen für parallele Sets bilden einen der Hauptpfeiler der Methoden dieses Buchs.}
Sie sind einer der Gründe für die Behauptung, daß nichts zu schwierig zum Spielen ist, wenn man weiß, wie man üben muß.
Sie dienen sowohl als Diagnosewerkzeug als auch als ein Werkzeug zur Entwicklung der Technik.
Praktisch die ganze Technik sollte durch den Gebrauch von parallelen Sets während des HS-Übens erworben werden, um die Geschwindigkeit nach oben zu bringen, das Entspannen zu lernen und Kontrolle zu erlangen.
Sie bilden einen vollständigen Satz, so daß Sie wissen, daß Sie alle notwendigen Werkzeuge besitzen.
Anders als Hanon usw. können sie sofort zur Hilfe herbeigeholt werden, wenn Sie auf \textit{irgendeine} schwierige Passage treffen, und sie erlauben das Üben mit jeder Geschwindigkeit, einschließlich von Geschwindigkeiten, die weit höher sind als Sie sie jemals benötigen werden.
Sie sind ideal dafür, das Spielen ohne Streß und mit Klangkontrolle zu üben.
Insbesondere ist es wichtig, sich anzugewöhnen, mit den Fingern über die Tasten zu gleiten und die Tasten zu erfühlen, bevor man sie spielt.
Das Gleiten der Finger (über die Tasten zu streichen) verleiht Klangkontrolle und das Erfühlen der Tasten verbessert die Genauigkeit.
Ohne eine schwierige Passage in einfache parallele Sets aufzuteilen ist es unmöglich, diese zusätzlichen Feinheiten in Ihr Spielen aufzunehmen.
Wir kommen nun zu mehreren anderen nützlichen Übungen.



<!-- c1iii7d.html -->

\subsubsection{Tonleitern, Arpeggios, Unabhängigkeit der Finger und Anheben der Finger}
\label{c1iii7d}

\textbf{\hyperref[c1iii5a]{Tonleitern} und \hyperref[Arpeggios]{Arpeggios} müssen gewissenhaft geübt werden.}
Sie gehören wegen der zahlreichen notwendigen Techniken, die man unter ihrer Verwendung am schnellsten lernen kann (wie z.B. \hyperref[c1iii5a]{Daumenübersatz}, \hyperref[c1iii4b]{flache Fingerhaltungen}, Tasten fühlen, Geschwindigkeit, parallele Sets, Glissandobewegung, Klang und Farbe, wie man Umkehrungen spielt, geschmeidiges Handgelenk usw.), nicht zur Klasse der stupide wiederholenden Übungen.
Tonleitern und Arpeggios müssen HS geübt werden; sie stets HT zu üben stellt sie in die gleiche Kategorie wie \hyperref[c1iii7h]{Hanon}.
Es gibt zwei Ausnahmen von dieser \enquote{Kein-HT-Regel}:

\begin{enumerate} 
 \item Wenn Sie sie (z.B. vor Konzerten usw.) zum Aufwärmen benutzen.
 \item Wenn Sie üben, um sicherzustellen, daß die Hände exakt synchronisiert werden können.
\end{enumerate}
Zu lernen, sie gut zu spielen, ist sehr schwierig, und Sie werden dafür sicherlich parallele Sets benötigen; weitere Einzelheiten finden Sie in den Abschnitten \hyperref[c1iii4b]{III.4b} und \hyperref[c1iii5a]{III.5}.


\label{c1iii7finger}

\textbf{Die Übungen zum \hyperref[c1iii7anheben]{Anheben} (s.u.) und für die Unabhängigkeit der Finger werden ausgeführt, indem man zunächst alle fünf Finger herunterdrückt, z.B. von C bis G mit der RH.}
Spielen Sie dann mit jedem Finger drei- bis fünfmal, z.B. CCCCDDDDEEEEFFFFGGGG.
Während ein Finger spielt, müssen die anderen vier unten gehalten werden.
Drücken Sie nicht fest herunter, da dies eine Form von Streß ist und sehr schnell Ermüdung verursachen wird.
Auch möchten Sie die langsamen Muskeln nicht mehr als notwendig wachsen lassen.
Alle heruntergedrückten Tasten müssen völlig unten sein, aber die Finger ruhen nur mit gerade soviel abwärts gerichteter Kraft auf ihnen, wie notwendig ist, um die Tasten unten zu halten.
Das durch die Schwerkraft verursachte Gewicht Ihrer Hand sollte genügen.
Anfänger werden diese Übung zu Beginn schwierig finden, weil die nicht spielenden Finger dazu neigen, aus ihrer optimalen Position einzuknicken oder sich ungewollt anzuheben, besonders wenn sie anfangen müde zu werden.
Wenn sie dazu neigen einzuknicken, versuchen Sie es ein paarmal, und wechseln Sie die Hände, oder hören Sie auf.
Versuchen Sie es nach einer Pause erneut.
Eine Variation dieser Übung ist, die Noten auf eine Oktave auszudehnen.
Diese Art der Übungen wurde bereits in der Zeit von F. Liszt benutzt (Moscheles).
Sie sollten sowohl mit der \hyperref[c1ii2]{gebogenen} als auch mit allen \hyperref[c1iii4b]{flachen Fingerhaltungen} ausgeführt werden.

Versuchen Sie bei der \textbf{Übung für die Unabhängigkeit der Finger}, die Geschwindigkeit zu steigern.
Beachten Sie die Ähnlichkeit zu \hyperref[c1iii7b1]{Übung \#1} für parallele Sets in Abschnitt (b); für die allgemeine technische Entwicklung ist Übung \#1 dieser jedoch überlegen.
Das Hauptziel von Übung \#1 war die Geschwindigkeit; die Betonung liegt hier auf etwas anderem - auf der Unabhängigkeit der Finger.
Einige Klavierlehrer empfehlen, diese Übung einmal während jeder Übungssitzung durchzuführen, wenn Sie sie zufriedenstellend spielen können.
Bis Sie sie zufriedenstellend spielen können, könnten Sie sie während jeder Übungssitzung mehrmals ausführen.
Sie während einer Sitzung viele Male zu üben und in den folgenden Sitzungen wegzulassen funktioniert nicht so gut.

Alle Übungsmethoden und Übungen, die in diesem Buch besprochen werden, behandeln hauptsächlich die Muskeln, die benutzt werden, um die Taste nach unten zu drücken (Flexoren = Beugemuskeln).
Es ist für diese Muskeln möglich, weitaus stärker entwickelt zu werden als jene, die benutzt werden, um die Finger anzuheben (Extensoren = Streckmuskeln).
Das gilt besonders, wenn man stets laut übt und nie die Kunst schnell zu spielen entwickelt und somit Probleme bei der Kontrolle verursacht, ganz besonders, wenn es dazu führt, daß man sehr viele langsame Muskeln entwickelt.
Wenn man älter wird, können die Beugemuskeln die Streckmuskeln schließlich an Kraft übertreffen.
Deshalb ist es eine gute Idee, die relevanten Streckmuskeln durch Übungen zum Anheben zu trainieren.
Die \hyperref[c1iii4b]{flachen Fingerhaltungen} sind beim Trainieren der Streckmuskeln für das Anheben der Finger und das gleichzeitige Entspannen der Streckmuskeln der Fingerspitzen wertvoll.
Diese beiden sind verschiedene Streckmuskeln.


\label{c1iii7anheben}

Wiederholen Sie zum \textbf{Üben des Anhebens} die \hyperref[c1iii7finger]{obige Übung}, aber heben Sie jeden Finger schnell so hoch Sie können und senken ihn sofort wieder.
Die Bewegung sollte so schnell wie möglich sein, aber langsam genug, daß Sie die völlige Kontrolle haben; das ist kein Geschwindigkeitswettbewerb, Sie müssen nur vermeiden, daß die langsamen Muskeln wachsen.
Behalten Sie wieder alle anderen Finger mit minimalem Druck unten.
Wie üblich ist es wichtig, die Anspannung in den Fingern zu reduzieren, die nicht angehoben werden.

Jeder hat Probleme damit, den 4. Finger anzuheben.
Viele glauben fälschlicherweise, man müßte den 4. Finger so hoch wie alle anderen Finger anheben können, und sie wenden deshalb ungeheuer viel Energie bei dem Versuch auf, das zu erreichen.
Solch ein Aufwand ist erwiesenermaßen vergeblich und kontraproduktiv.
Das kommt daher, daß die Anatomie des 4. Fingers es nicht erlaubt, ihn über einen bestimmten Punkt hinaus anzuheben.
Der 4. Finger darf nur nicht versehentlich eine Taste niederdrücken; dies erfordert ein viel geringeres Anheben.
Deshalb können Sie zu jeder Zeit mit dem 4. Finger knapp über den Tasten spielen oder ihn sogar darauf ruhen lassen.
Schwierige Passagen mit übertriebenem Aufwand zum höheren Heben dieses Fingers zu üben, kann Streß im 3. und 5. Finger verursachen.
Es ist produktiver, zu lernen mit weniger Streß zu spielen, solange der 4. Finger nicht in irgendeiner Weise stört.
\label{c1iii7finger4}
Die \textbf{Übung für das unabhängige Anheben des 4. Fingers} wird folgendermaßen ausgeführt:

\begin{itemize} 
 \item Drücken Sie alle Finger auf CDEFG nach unten wie zuvor.
 \item Spielen Sie 1,4,1,4,1,4 usw., mit Betonung auf der 1, und heben Sie den Finger 4 so schnell und so hoch Sie können.
 Finger 1 sollte so nah an der Taste bleiben wie möglich.
 \item Wiederholen Sie es mit 2,4,2,4,2,4 usw.
 \item Nun mit Finger 3 und 4.
 \item Und zum Schluß mit 5 und 4.
 \end{itemize}
Sie können diese Übung auch mit der 4 auf einer schwarzen Taste durchführen.

Sowohl die Übung für die Unabhängigkeit der Finger als auch die zum Anheben können ohne ein Klavier, auf jeder glatten Oberfläche, durchgeführt werden.
Das ist der beste Zeitpunkt, um das Entspannen der Streckmuskeln der beiden letzten Fingerglieder (Nagelglied und mittleres Glied) der Finger 2 bis 5 zu üben; Details s. \hyperref[c1iii4b]{Abschnitt III.4b}.
Während der gesamten Übung sollten diese beiden Glieder bei allen Fingern völlig entspannt sein, sogar bei dem angehobenen Finger.
 

\subsubsection{(Große) Akkorde spielen, Dehnung der Handflächen}
\label{c1iii7e}

Wir behandeln zunächst das Problem, genaue Akkorde zu spielen, in denen alle Noten so simultan wie möglich gespielt werden müssen.
Dann gehen wir das Problem an, große Akkorde zu spielen.
Wenn Sie kleine Hände haben, müssen Sie sowohl die Handflächen so weit wie möglich dehnen als auch die Finger seitwärts ausstrecken.

Wir haben bereits gesehen, daß der \hyperref[c1ii10]{Freie Fall} (Abschnitt II.10) benutzt werden kann, um die Genauigkeit der Akkorde zu verbessern.
Wenn jedoch nach dem Benutzen des Freien Falls immer noch eine Ungleichmäßigkeit vorhanden ist, dann gibt es ein fundamentales Problem, das mit den \hyperref[c1iii7b]{Übungen für parallele Sets} diagnostiziert und behandelt werden muß.
Akkorde werden ungleichmäßig, wenn die Kontrolle über einzelne Finger ungleichmäßig ist.
Welche Finger schwach sind oder langsam usw. kann mit den Übungen für parallele Sets diagnostiziert und korrigiert werden.
Lassen Sie uns ein Beispiel betrachten.
Angenommen, Sie spielen einen C.E-Akkord gegen ein G (alles mit der LH) in der dritten Oktave.
Dann werden C3.E3 und G3 mit den Fingern 5.3 und 1 gespielt.
Sie spielen eine Serie von 5.3,1,5.3,1,5.3,1 usw., wie ein Tremolo.
Lassen Sie uns weiter annehmen, daß es ein Akkord-Problem mit dem 5.3 gibt.
Diese beiden Finger landen nicht simultan und ruinieren so das Tremolo.
Versuchen Sie zur Diagnose dieses Problems das parallele Set 5,3, um zu sehen, ob Sie es spielen können.
Testen Sie nun das umgekehrte Set 3,5.
Wenn Sie ein Problem mit dem Akkord haben, ist es möglich, daß Sie eher ein Problem mit einem der beiden parallelen Sets haben als mit dem anderen, oder daß Sie Probleme mit beiden parallelen Sets haben.
Meistens ist 3,5 schwieriger als 5,3.
Arbeiten Sie an dem/n problematischen parallelen Set/s.
Wenn Sie beide parallele Sets gut spielen können, sollte der Akkord viel besser klingen.
Es besteht auch die - geringere - Möglichkeit, daß Ihr Problem am parallelen Set 5,1 oder 3,1 liegt.
Wenn also die Arbeit an 5,3 nichts bringt, versuchen Sie es mit diesen.

\textbf{In der Hand gibt es zwei Muskelgruppen, mit denen man die Finger und die Handfläche spreizen kann, um große Akkorde zu erreichen.
Eine Gruppe öffnet hauptsächlich die Handfläche, und die andere spreizt hauptsächlich die Finger auseinander.}
Benutzen Sie hauptsächlich die Muskelgruppe, die die Handfläche öffnet, wenn Sie die Hand strecken, um große Akkorde zu spielen.
Das Gefühl ist das gleiche wie beim Strecken der Handfläche aber mit freien Fingern; d.h. spreizen Sie die Knöchel statt der Fingerspitzen.
Die zweite Muskelgruppe spreizt einfach die Finger auseinander.
Dieses Spreizen hilft, die Handfläche zu verbreitern, beeinträchtigt aber die Bewegung der Finger, weil es dazu führt, daß die Finger an die Handfläche gefesselt werden.
Gewöhnen Sie sich an, die Muskeln der Handflächen getrennt von den Fingermuskeln zu benutzen.
Das wird sowohl den Streß als auch die Ermüdung reduzieren, wenn Sie Akkorde spielen, und mehr zur Kontrolle beitragen.
Natürlich ist es am einfachsten, beide Muskelgruppen gleichzeitig zu benutzen, aber es ist nützlich zu wissen, daß es zwei Muskelgruppen gibt, wenn man seine Übungen plant und wenn man entscheidet, wie man Akkorde spielt.

\textbf{Spreizen der Finger: Um zu testen, ob die Finger völlig gespreizt sind, öffnen Sie Ihre Handfläche bis zum Maximum und spreizen Sie die Finger für eine maximale Reichweite - wenn der kleine Finger und der Daumen fast eine gerade Linie bilden, dann werden Sie nicht in der Lage sein, sie mehr zu spreizen.}
Wenn sie ein \enquote{V} bilden, dann sind Sie eventuell in der Lage weiter zu reichen, indem Sie Dehnungsübungen durchführen.
Eine andere Möglichkeit, diese Ausrichtung zu testen, ist, Ihre Hand so auf eine Tischplatte zu legen, daß die Finger 2 bis 4 und die Handfläche so weit wie möglich auf der Tischplatte liegen und Sie mit dem Daumen und dem kleinen Finger horizontal gegen die Tischkante drücken.
Wenn der Daumen und der kleine Finger ein Dreieck mit der Tischkante bilden, sind Sie eventuell in der Lage, sie mehr zu strecken.
Sie können eine Dehnungsübung durchführen, indem Sie die Hand zur Tischkante hin schieben, um so den Daumen und den kleinen Finger weiter auseinander zu spreizen.
Sie können viel Zeit sparen, wenn Sie eine Hand an der oberen Kante des Klaviers dehnen, während Sie mit der anderen Hand HS üben.

\textbf{Spreizen der Handfläche:}
Es ist wichtiger, aber schwieriger, die Handfläche anstatt der Finger zu dehnen.
Eine Möglichkeit ist, die Innenseiten der Handflächen so vor der Brust übereinander zu legen, daß jeweils der Daumen der einen Hand den kleinen Finger der anderen Hand berührt und die Ellbogen nach außen zeigen.
Verschränken Sie die Daumen mit den kleinen Fingern, so daß die Finger 2 bis 4 auf der Innenseite der Handflächen sind und die Finger 1 und 5 an der Rückseite der Handflächen hervorstehen.
Schieben Sie dann die Hände aufeinander zu, so daß die Daumen und die kleinen Finger sich gegenseitig zurückdrücken und somit die Handflächen dehnen.
Sehen Sie dazu ein \hyperref[http://www.pianopractice.org/palmstretch.jpg]{Foto} (extern).
Die zum Spreizen notwendige Kraft wird erzeugt, indem Sie die Hände aufeinander zu bewegen (RH nach links und die LH nach rechts).
Sie können auch die Muskeln für das Spreizen der Finger und Handflächen gleichzeitig trainieren, während Sie die Hände aufeinander zu bewegen.
Das ist keine isometrische Übung, d.h. die Spreizbewegungen sollten schnell und kurz sein.
Diese Fähigkeit, schnell zu spreizen und sofort zu entspannen ist für das Entspannen wichtig.
Regelmäßiges Dehnen in jungen Jahren kann einen beträchtlichen Unterschied bei der Reichweite ausmachen wenn man älter wird, und regelmäßiges Üben verhindert, daß die Reichweite mit zunehmendem Alter nachläßt.
Die Haut zwischen den Fingern können Sie auch dehnen, indem Sie jeweils einen Fingerzwischenraum der einen Hand gegen den gleichen Fingerzwischenraum der anderen drücken, wobei die Hände um 90 Grad zueinander gedreht sind.
Benutzen Sie für eine maximale Wirkung bei jeder Drückbewegung sowohl die Hand- als auch die Fingermuskeln, um die Handfläche zu spreizen.
Führen Sie auch das nicht wie eine isometrische Übung sondern mit schnellen Bewegungen aus.
Die meisten Menschen haben eine etwas größere linke als rechte Hand, und einige können mit 1.4 weiter reichen als mit 1.5.

Wenn Sie große Akkorde spielen, sollte der Daumen leicht einwärts gekrümmt sein, nicht völlig ausgestreckt.
Es ist kontraintuitiv, daß man mit eingezogenem Daumen weiter reichen kann; das geschieht wegen der besonderen Krümmung der Fingerspitze des Daumens.
Beim Spielen von Akkorden müssen Sie im allgemeinen die Hand bewegen, und diese Bewegung muß sehr genau sein, wenn die Akkorde richtig erklingen sollen.
Das ist die \hyperref[c1iii7f]{Sprungbewegung}, die unten besprochen wird.
Für das Spielen von Akkorden ist es notwendig, daß Sie sowohl saubere Sprungbewegungen entwickeln als auch die Angewohnheit, die Tasten zu fühlen.
Sie können nicht bloß Ihre Hand hoch über die Tasten heben, alle Ihre Finger an der richtigen Stelle positionieren, sie herunterdonnern lassen und erwarten, daß Sie alle richtigen Noten genau zum gleichen Zeitpunkt treffen.
Bei großen Pianisten erscheint es oft so, als ob sie dies tun würden, aber wie wir unten sehen werden, tun sie es nicht.
Bis Sie die Sprungbewegung perfektioniert haben und in der Lage sind, die Tasten zu erfühlen, werden deshalb irgendwelche Probleme mit dem Spielen von Akkorden (fehlende oder falsche Noten) eventuell nicht durch einen Mangel an Reichweite oder Kontrolle der Finger verursacht.
Wenn Sie Schwierigkeiten haben, die Akkorde zu treffen \textit{und} keine Sicherheit bei Ihren Sprüngen haben, ist das ein sicheres Zeichen, daß Sie die Sprünge lernen müssen, bevor Sie daran denken können, Akkorde zu treffen.
 


<!-- c1iii7f.html -->

\subsubsection{Sprünge}
\label{c1iii7f}

Viele Schüler beobachten, wie berühmte Pianisten diese schnellen, weiten Sprünge machen, und fragen sich, warum sie selbst es nicht können, egal wie viele Male sie es üben.
Es erscheint so, als ob diese großen Pianisten ohne Anstrengung von Position zu Position springen und Noten oder Akkorde flüssig spielen könnten, egal wo sie sich gerade befinden.
In Wirklichkeit machen die Pianisten mehrere Bewegungen, die zu schnell und zu fein sind, um vom Auge wahrgenommen zu werden, solange man nicht weiß, wonach man sehen muß.
\textbf{Sprünge bestehen im wesentlichen aus zwei Bewegungen:}

\begin{enumerate}[label={\arabic*.}] 
 \item die horizontale Verschiebung der Hand zur richtigen Position
 \item die wirkliche Abwärtsbewegung zum Spielen
 \end{enumerate}
Es gibt noch zwei optionale Bewegungen: das Fühlen der Tasten und die Bewegung des Abhebens; diese werden unten erklärt.
Die kombinierte Bewegung sollte mehr wie ein umgekehrtes \enquote{U} als ein umgekehrtes \enquote{V} aussehen.

Schüler ohne Ausbildung für die Sprünge neigen dazu, die Hand in einer umgekehrten V-Bewegung zu führen.
Mit dieser Art von Bewegung (keine horizontale Beschleunigung) ist es schrecklich schwierig, eine Note genau zu treffen, weil man in einem beliebigen Winkel herunterkommt.
Dieser Winkel ist niemals der gleiche (sogar wenn derselbe Abschnitt ein anderes Mal erneut gespielt wird), weil er von der Entfernung des Sprungs, dem Tempo, wie hoch man die Hand anhebt usw. abhängt.
Zu üben, gerade herunterzukommen, ist schwer genug; kein Wunder, daß manche Schüler Sprünge als unmöglich ansehen, wenn sie alle diese Winkel üben müssen.
Deshalb ist es wichtig, am Ende des Sprungs gerade herunterzukommen (oder die Tasten unmittelbar bevor man sie spielt zu fühlen).

Schüler ohne Ausbildung für die Sprünge erkennen im allgemeinen auch nicht, daß die horizontale Bewegung in hohem Maß beschleunigt werden kann; \textbf{deshalb ist die erste Fertigkeit, die trainiert werden muß, die horizontale Bewegung so schnell wie möglich auszuführen, damit genug Zeit für das genaue Lokalisieren der Tasten übrigbleibt, wenn man am Ziel angekommen ist.}
Lokalisieren Sie die Tasten, indem Sie sie erfühlen, bevor Sie sie tatsächlich spielen.
\textbf{Die Tasten zu erfühlen ist die 3. Komponente eines Sprungs.}
Diese 3. Komponente ist optional, weil sie nicht immer notwendig ist und manchmal nicht genug Zeit dafür bleibt.
Wenn diese Kombination der Bewegungen perfekt ist, sieht es so aus, als würde sie in einer Bewegung ausgeführt.
Das kommt daher, daß Sie nur den Bruchteil einer Sekunde bevor Sie die Note spielen dort sein müssen.
Wenn man nicht übt, die horizontale Bewegung zu beschleunigen, neigt man dazu, den Bruchteil einer Sekunde später anzukommen als man müßte.
Dieser kaum wahrnehmbare Unterschied macht den ganzen Unterschied zwischen 100\% Genauigkeit und schlechter Genauigkeit aus.
Stellen Sie sicher, daß Sie auch bei langsamen Sprüngen die schnellen horizontalen Bewegungen üben.

Obwohl das Erfühlen der Tasten vor dem Spielen optional ist, werden Sie überrascht sein, wie schnell man es tun kann.
In den meisten Fällen hat man die Zeit dazu.
Deshalb ist es ein guter Grundsatz, \textit{immer} die Tasten zu erfühlen, wenn man Sprünge langsam übt.
Wenn Sie alle Fertigkeiten lernen, die hier aufgeführt sind, werden Sie jede Menge Zeit haben, um die Tasten zu erfühlen, sogar bei der endgültigen Geschwindigkeit.
Es gibt ein paar Fälle, in denen keine Zeit bleibt, die Tasten zu erfühlen, und in diesen Fällen können Sie genau spielen, wenn Sie die meisten der anderen Sprünge durch das Erfühlen genau lokalisiert haben.

\textbf{Die vierte Komponente des Sprungs ist das Abheben.}
Gewöhnen Sie sich an, immer schnell abzuheben, unabhängig von der Geschwindigkeit des Sprungs.
Es ist nichts falsch daran, weit vor der Zeit anzukommen.
Sie sollten das schnelle Abheben auch bei langsamen Sprüngen üben, damit Sie diese Fertigkeit bereits haben, wenn Sie schneller werden.
Beginnen Sie das Abheben mit einem kleinen, abwärts und seitwärts gerichteten Ausschlag des Handgelenks.
Obwohl es für das Spielen der Noten notwendig ist, daß Sie gerade herunterkommen, gibt es keine Notwendigkeit dafür, beim Abheben gerade nach oben zu gehen.
Offensichtlich ist der ganze Sprungvorgang so gestaltet, daß die Hand schnell, genau und reproduzierbar am Ziel ankommt, so daß viel Zeit übrig bleibt, um gerade nach unten zu spielen und die Tasten zu erfühlen.

\textbf{Das wichtigste Element, das Sie üben müssen, wenn Sie die Komponenten eines Sprungs erst einmal kennen, ist das Beschleunigen der horizontalen Bewegung.}
Sie werden überrascht sein, wie schnell man die Hand horizontal bewegen kann, wenn man sich nur auf diese Bewegung konzentriert.
Sie werden auch darüber verwundert sein, wieviel schneller Sie sie nach ein paar Tagen Übung bewegen können - etwas, das einige Schüler im ganzen Leben nicht erreichen, weil man ihnen nie beigebracht hat es zu üben.
Diese Geschwindigkeit ist das, was die notwendige zusätzliche Zeit dazu beiträgt, eine 100\%-ige Genauigkeit zu sichern und alle anderen Komponenten des Sprungs ohne Anstrengung in sich aufzunehmen - besonders das Erfühlen der Tasten.
Üben Sie das Erfühlen der Tasten wann immer es möglich ist, so daß es zu einer zweiten Natur wird und Sie nicht auf Ihre Hände sehen müssen.
Haben Sie es erst einmal einwandfrei in Ihr Spielen aufgenommen, werden die meisten Menschen, die Ihnen beim Spielen zusehen, es nicht einmal merken, daß Sie die Tasten erfühlen, weil Sie es im Bruchteil einer Sekunde tun können.
Wie ein vollendeter Magier werden Sie Ihre Hände schneller bewegen als das Auge sehen kann.
Die \hyperref[c1iii4b]{flachen Fingerhaltungen} sind dafür wichtig, weil Sie den empfindlichsten Teil der Finger für das Erfühlen der Tasten benutzen können und diese Haltungen die Genauigkeit beim Treffen der Tasten, insbesondere der schwarzen, erhöhen.

Da Sie nun die Komponenten eines Sprungs kennen, können Sie danach Ausschau halten, wenn Sie Konzertpianisten beim Auftritt zusehen.
Sie sollten nun in der Lage sein, jede Komponente zu erkennen, und Sie werden verblüfft sein, wie oft die Pianisten die Tasten erfühlen, bevor sie sie anschlagen und wie sie diese Komponenten in Windeseile ausführen.
Diese Fertigkeiten versetzen auch Sie in die Lage, ohne auf die Hände zu sehen zu spielen und weite Sprünge auszuführen.

Die beste Art, schnelle horizontale Bewegungen zu üben, ist, es ohne Klavier zu tun.
Setzen Sie sich so, daß der Ellbogen gerade nach unten und der Unterarm nach vorne zeigt.
\textbf{Bewegen Sie die Hand schnell seitwärts, indem Sie den Unterarm um den Ellbogen schwingen, wobei der Ellbogen am Ort bleibt.
Denken Sie wieder daran, am Ende der Bewegung völlig zu entspannen.
Lassen Sie nun den Unterarm gerade nach vorne zeigen und bewegen Sie ihn horizontal seitwärts (nicht in einem Bogen aufwärts), indem Sie den Oberarm um das Schultergelenk drehen und gleichzeitig die Schulter nach unten bewegen.}\footnote{Die Beschreibung der Bewegungen ist optisch zu verstehen.
Anatomisch geschieht folgendes:
Beim \enquote{Drehen des Unterarms um den Ellbogen} dreht sich nur der Oberarm im Schultergelenk und bleibt seitlich am Körper.
Der Unterarm wird nicht aktiv bewegt, d.h. Elle und Speiche behalten ihre relative Position zum Ellbogen.
Beim \enquote{horizontalen Seitwärtsbewegen des Unterarms} soll der Unterarm immer auf derselben Höhe und gerade nach vorne gerichtet bleiben.
Er wird entlang der gedachten Klavierkante seitwärts bewegt.
Dazu wird der Oberarm im Schultergelenk seitlich vom Körper weg rotiert.
Zum Ausgleich der daraus resultierenden Aufwärtsbewegung des Ellbogens muß die Schulter gesenkt bzw. der Oberkörper zur Seite geneigt werden.
Zusätzlich rotiert der Unterarm, d.h. Elle und Speiche, im Ellbogengelenk gegen die Drehrichtung des Oberarms, damit die Handfläche waagerecht bleibt.}
Beim wirklichen Sprung werden diese Bewegungen auf eine komplexe Art kombiniert.
Üben Sie die Bewegungen nach rechts und nach links so schnell Sie können, mit jeder der beiden Bewegungsarten einzeln und mit der Kombination der beiden.
\textit{Versuchen Sie nicht}, diese Bewegungen innerhalb eines Tages zu lernen.
Es ist möglich, sich dabei selbst zu verletzen, und bedeutende Verbesserungen lassen sich nur mit der \hyperref[c1ii15]{Automatischen Verbesserung nach dem Üben (PPI)} erzielen.

Wenn Sie gelernt haben, die horizontale Bewegung zu beschleunigen, werden die Sprünge sofort einfacher.
\textbf{Um den Streß zu reduzieren, entspannen Sie alle Muskeln, sobald die horizontale Bewegung vorbei ist.
Dasselbe ist auf die nachfolgende Abwärtsbewegung anwendbar - entspannen Sie alle Muskeln, sobald die Noten gespielt sind}, und lassen Sie das Gewicht der Hand auf dem Klavier ruhen (heben Sie nicht die Hand bzw. die Finger von den Tasten).
Ein gutes Stück für das Üben dieser Sprünge mit der LH ist die 4. Variation in Mozarts berühmter A-Dur Sonate \#11 (KV331 bzw. K300i).
Diese Variation hat große Sprünge, in denen die LH über die RH kreuzt.
Ein beliebtes Stück, das Sie benutzen können, um Sprünge mit der RH zu üben, ist der 1. Satz von Beethovens Pathétique (Opus 13), direkt nach den Oktavtremolos der LH, in denen die RH Sprünge macht, die die LH kreuzen.

Üben Sie, die horizontale Bewegung zu beschleunigen, indem Sie mit langsamem Tempo spielen aber sich so schnell Sie können horizontal bewegen und dann über der richtigen Position anhalten und warten bevor Sie spielen.
Sie werden nun die Zeit haben, das Erfühlen der Noten vor dem Spielen zu üben, um 100\% Genauigkeit zu garantieren.
Die Idee ist hier, sich anzugewöhnen, immer vorzeitig an der Position anzukommen.
Steigern Sie das Tempo erst, wenn Sie überzeugt sind, eine schnelle horizontale Bewegung zu haben.
Alles was Sie tun müssen, um schneller zu werden, ist, einfach die Wartezeit vor dem Spielen der Noten zu reduzieren, wenn das Tempo steigt.
Wenn Sie fortgeschrittener werden, werden Sie immer \enquote{mindestens den Bruchteil einer Sekunde vorher ankommen}.
Kombinieren Sie dann schrittweise alle vier Sprungbewegungen zu einer gleichmäßigen Bewegung.
Nun sieht Ihre Bewegung genauso aus, wie diese der großen Pianisten, die Sie beneidet haben!
Besser sogar, Sprünge sind schließlich gar nicht so schwer, und man muß keine Angst vor ihnen haben.
 

\subsubsection{Weitere Übungen}
\label{c1iii7g}

Die meisten Dehnungsübungen für die großen Muskeln des Körpers sind hilfreich (s. \hyperref[Bruser]{Bruser}).
Eine Dehnungsübung für die Beugemuskeln (der Finger) kann folgendermaßen ausgeführt werden.
Drücken Sie mit der Handfläche der einen Hand die Finger der anderen Hand rückwärts zur Oberseite des Unterarms.
Menschen haben eine sehr unterschiedliche Gelenkigkeit, und einige werden in der Lage sein, die Finger ganz zurück zu drücken, so daß die Fingernägel den Arm berühren (180 Grad zur gestreckten Position!), während andere vielleicht in der Lage sind, nur ungefähr 90 Grad zurück zu drücken (die Finger zeigen bei horizontalem Arm aufwärts).
Die Fähigkeit, die Beugemuskeln zu strecken, nimmt mit zunehmendem Alter ab; deshalb ist es eine gute Idee, sie im Laufe des Lebens oft zu dehnen, um ihre Flexibilität zu erhalten.
Um die Streckmuskeln zu dehnen, drücken Sie die Finger zur Unterseite des Unterarms herunter.
Sie könnten diese Dehnungsübungen unmittelbar vor dem \enquote{\hyperref[c1iii6g]{kalt Spielen}} ausführen.

 Es gibt zahlreiche Übungen bei \hyperref[Sandor]{Sandor} und \hyperref[Fink]{Fink} (s. Quellenverzeichnis).
Diese sind interessant, weil jede Übung ausgewählt wurde, um eine bestimmte Handbewegung zu demonstrieren.
Zusätzlich werden die Bewegungen oft mit Passagen aus klassischen Kompositionen berühmter Komponisten veranschaulicht.



<!-- c1iii7h.html -->

\subsubsection{Probleme mit Hanons Übungen}
\label{c1iii7h}

Ungefähr seit 1900 wurden die Übungen von Charles Louis Hanon (1820-1900) von zahlreichen Klavierspielern in der Hoffnung benutzt, ihre Technik zu verbessern.
Es gibt nun zwei Lehrmeinungen: einmal die, daß Hanons Übungen hilfreich sind und die, daß sie es nicht sind.
Viele Lehrer empfehlen Hanon, während andere meinen, daß die Übungen kontraproduktiv sind.
Es gibt einen \enquote{Grund}, den viele Menschen dafür angeben, daß sie Hanon benutzen: um die Hände vom einen auf den anderen Tag in guter Verfassung zum Spielen zu halten.
Dieser Grund wird am meisten von Personen zitiert, die ihre Finger mit abgeschaltetem Gehirn aufwärmen möchten.
Ich habe den Verdacht, daß diese Angewohnheit daraus resultiert, daß die Person Hanon in der frühen Klavierkarriere gelernt hat, und daß dieselbe Person Hanon nicht benutzen würde, wenn sie nicht so daran gewöhnt wäre.

Ich habe Hanons Übungen während meiner Jugend ausgiebig benutzt, gehöre aber nun fest zur Anti-Hanon-Schule.
Im folgenden liste ich einige Gründe dafür auf.
Czerny, Cramer-Bülow und verwandte Übungsstücke teilen viele dieser Nachteile.
\textbf{Hanon ist möglicherweise das beste Beispiel dafür, wie intuitive Methoden ganze Scharen von Klavierspielern  dazu verleiten können, Methoden zu benutzen, die im Grunde nutzlos oder sogar schädlich sind.}


\begin{enumerate}[label={\roman*.}]  

\item \label{c1iii7h1}\textbf{Hanon stellt in seiner Einführung einige überraschende Behauptungen, ohne eine rationale Erklärung oder einen experimentellen Nachweis, auf.
Das wird in seinem Titel deutlich: \enquote{Der Klaviervirtuose, in 60 Übungen}.}
Bei sorgfältigem Lesen seiner Einführung stellt man fest, daß er einfach fühlte, daß diese nützliche Übungen seien und er sie deshalb niedergeschrieben hat.
Es ist ein weiteres sehr gutes Beispiel für den \enquote{intuitiven Ansatz}.
\textbf{Die meisten fortgeschrittenen Lehrer, die diese Einführung lesen, würden zu dem Schluß kommen, daß dieser Ansatz für das Aneignen der Technik amateurhaft ist und nicht funktionieren wird.}
Hanon unterstellt, daß die Fähigkeit, diese Übungen zu spielen, sicherstellt, daß man alles spielen kann - das ist nicht nur völlig falsch, sondern enthüllt auch einen überraschenden Mangel an Verständnis dafür, was Technik ist.
\textbf{Technik kann nur durch das Lernen von vielen Kompositionen vieler Komponisten erworben werden.}

Es steht außer Frage, daß es viele vollendete Pianisten gibt, die die Hanon-Übungen benutzen.
\textbf{Die fortgeschrittenen Pianisten stimmen jedoch alle darin überein, daß Hanon nicht für das Erwerben der Technik ist}, aber dafür nützlich sein könnte, sich aufzuwärmen oder die Hände in guter Verfassung zum Spielen zu halten.
Ich glaube, es gibt viele bessere Stücke zum Aufwärmen als Hanon, wie z.B. Etüden, zahlreiche Kompositionen von Bach und andere leichte Stücke.
Die Fertigkeiten, die für das Spielen jedes bedeutenden Musikstücks notwendig sind, sind unglaublich vielfältig - fast unendlich in der Zahl.
\textbf{Zu denken, daß Technik auf 60 Übungen reduziert werden könnte, offenbart die Naivität Hanons, und jeder Schüler, der das glaubt, wurde in die Irre geführt.}



\item \label{c1iii7h2}\textbf{Alle 60 sind fast nur beidhändige Übungen}, bei denen die beiden Hände die gleichen Noten um eine Oktave versetzt spielen, zuzüglich ein paar Übungen mit Gegenbewegungen, bei denen die Hände in entgegengesetzte Richtungen bewegt werden.
\textbf{Diese gekoppelte HT-Bewegung ist eine der größten Einschränkungen dieser Übungen, weil die bessere Hand keine fortgeschritteneren Fertigkeiten üben kann als die schwächere Hand.}
Bei langsamer Geschwindigkeit wird keine der Hände stark trainiert.
Bei maximaler Geschwindigkeit wird die langsamere Hand gestreßt, während die bessere entspannt spielt.
\textbf{Weil Technik hauptsächlich dann erworben wird, wenn man entspannt spielt, entwickelt die schwächere Hand schlechte Angewohnheiten, und die stärkere Hand wird stärker.}
Der beste Weg, die schwächere Hand zu stärken, ist, nur mit dieser Hand zu üben, \textit{nicht} HT zu spielen.
Tatsächlich ist die beste Art mit Hanon zu lernen, die Hände, wie hier in diesem Buch empfohlen, zu trennen, aber Hanon hat anscheinend noch nicht einmal daran gedacht.
Zu glauben, daß durch das HT-Spielen die schwächere Hand die stärkere Hand einholt, offenbart eine für jemanden mit soviel Lehrerfahrung erstaunliche Unwissenheit.
Das ist ein Teil dessen, was ich oben mit \enquote{amateurhaft} meinte; weitere Beispiele folgen.

Die Hände zu koppeln hilft dabei zu lernen, wie man die Hände koordiniert, tut aber nichts dafür, die unabhängige Kontrolle jeder Hand zu lehren.
Praktisch in der ganzen Musik spielen die beiden Hände unterschiedliche Teile.
Hanon gibt uns keine Gelegenheit, das zu üben.
Bachs Inventionen sind viel besser und stärken (wenn Sie HS üben) wirklich die schwächere Hand.
\textbf{Der Punkt ist hier, daß Hanon sehr begrenzt ist; er lehrt nur einen kleinen Bruchteil der gesamten Technik, die man benötigt.}



\item \label{c1iii7h3}\textbf{Es gibt keine Vorkehrung für das Ausruhen einer ermüdeten Hand.
Das führt im allgemeinen zu Streß und Verletzungen.}
Ein eifriger Schüler, der die Schmerzen und Ermüdung bei dem Bemühen, den Anweisungen Hanons zu folgen, bekämpft, wird fast mit Sicherheit Streß aufbauen, schlechte Angewohnheiten annehmen und Verletzungen riskieren.
\textbf{Das Konzept der Entspannung wird noch nicht einmal erwähnt.}
Klavierspielen ist eine Kunst zur Erzeugung von Schönheit und Eleganz; es ist keine Demonstration von Machos, wieviel Bestrafung ihre Hände, Ohren und Gehirne aushalten können.

\textbf{Hingebungsvolle Schüler benutzen Hanon am Ende oft als eine Möglichkeit, intensive Übungen auszuführen, in dem falschen Glauben, daß Klavierspielen wie Gewichtheben ist, und daß \enquote{ohne Schmerzen kein Erfolg} auch für das Klavier gilt.}
Solche Übungen sollen angeblich bis zur Grenze der menschlichen Belastbarkeit ausgeführt werden können, sogar bis einige Schmerzen spürbar sind.
Das offenbart einen Mangel an richtiger Ausbildung darüber, was für den Erwerb der Technik notwendig ist.
Die verschwendeten Ressourcen aufgrund solcher falschen Vorstellungen können den Unterschied zwischen Erfolg und Versagen für eine große Zahl von Schülern ausmachen, sogar wenn sie keine Verletzungen erleiden.
Natürlich sind viele Schüler erfolgreich, die routinemäßig Hanon üben; in diesem Fall arbeiten sie so hart, daß sie \textit{trotz} Hanon Erfolg haben.

 

\item \label{c1iii7h4}\textbf{Die Hanon-Übungen sind frei von Musik, so daß Schüler am Ende möglicherweise wie Roboter üben.}
Es erfordert keine musikalische Genialität, um eine Serie von Übungen der Hanon-Art zusammenzustellen.
Die Freude am Klavier kommt von der direkten Auseinandersetzung mit den größten Genies, die jemals gelebt haben, wenn Sie deren Kompositionen spielen.
Für zu viele Jahre hat Hanon die falsche Botschaft verbreitet, daß Technik und Musik getrennt gelernt werden können.
Bach ist in dieser Hinsicht überlegen; seine Musik trainiert sowohl die Hände als auch das Gehirn.
\textbf{Hanon hat wahrscheinlich das meiste seines Material aus Bachs berühmter Toccata und Fuge entnommen und so geändert, daß jede Einheit selbstzyklisch ist.
Der Rest wurde wahrscheinlich ebenfalls aus Bachs Werken entnommen, besonders aus den Inventionen und Sinfonien.}

Einer der größten Schäden, die Hanon anrichtet, ist, daß er soviel Zeit verschwendet.
Der Schüler hat am Ende nicht genügend Zeit, um sein Repertoire zu entwickeln oder wirkliche Technik zu erwerben.
\textbf{Hanon kann schädlich für die Technik und das Aufführen sein!}



\item \label{c1iii7h5}\textbf{Viele Klavierspieler benutzen Hanon routinemäßig als Übung zum Aufwärmen.
Das konditioniert die Hände so, daß Sie unfähig werden, sich einfach hinzusetzen und \enquote{\hyperref[c1iii6g]{kalt}} zu spielen; etwas, das jeder vollendete Pianist innerhalb vernünftiger Grenzen können sollte.}
Da die Hände für mindestens 10 bis 20 Minuten kalt sind, raubt das \enquote{Aufwärmen} dem Schüler dieses kostbare kleine Fenster der Gelegenheit, das kalte Spielen zu üben.
Diese Angewohnheit, Hanon zum Aufwärmen zu benutzen, ist heimtückischer als viele erkennen.
Diejenigen, die Hanon zum Aufwärmen benutzen, können zu dem Glauben verleitet werden, daß Hanon ihre Finger zum Fliegen bringt, während die Finger in Wahrheit nach jeder guten Übungseinheit fliegen, egal ob mit oder ohne Hanon.
Es ist tückisch, weil die hauptsächliche Konsequenz aus diesem Mißverständnis ist, daß die Person weniger in der Lage ist vorzuspielen, unabhängig davon, ob die Finger aufgewärmt sind oder nicht.
Es ist wirklich unglücklich, daß die Hanon-Art zu denken einen großen Bestand an Schülern hervorgebracht hat, die glauben, daß man ein Mozart sein muß, um in der Lage zu sein, sich einfach hinzusetzen und zu spielen, und daß man von gewöhnlichen Sterblichen nicht erwarten kann, solche zauberhaften Meisterleistungen zu vollbringen.
Wenn Sie in der Lage sein möchten, \enquote{auf Kommando zu spielen}, fangen Sie am besten damit an, daß Sie aufhören Hanon zu üben.

 

\item \label{c1iii7h6}Es gibt kaum einen Zweifel daran, daß ein gewisses Maß an Technik erforderlich ist, um diese Übungen zu spielen, besonders ungefähr die letzten 10.\textbf{Das Problem ist, daß Hanon keine Anleitung dafür liefert, wie man diese Technik erwirbt.}
Es ist exakt analog dazu, einem armen Menschen zu sagen, er solle etwas Geld verdienen, wenn er reich werden möchte.
Es hilft ihm nicht.
Wenn ein Schüler die mit Hanon verbrachte Zeit dazu benutzt hätte, eine Beethoven-Sonate zu üben, hätte er viel mehr Technik erworben.
\textbf{Wer würde nicht lieber Mozart, Bach, Chopin usw. statt Hanon-Übungen mit besseren Ergebnissen spielen und am Ende ein aufführbares Repertoire haben?}

Sogar wenn Sie alle Hanon-Übungen gut spielen können, wird Ihnen Hanon nicht helfen, wenn Sie bei einer schwierigen Passage einer anderen Komposition festhängen.
Hanon stellt keine Diagnosen zur Verfügung, die Ihnen sagen, warum Sie eine bestimmte Passage nicht spielen können.
Die \hyperref[c1iii7b]{Übungen für parallele Sets} bieten Ihnen sowohl die Diagnosewerkzeuge als auch die Lösungen für praktisch jede Situation einschließlich der Verzierungen usw., die Hanon noch nicht einmal berücksichtigt.



\item \label{c1iii7h7}\textbf{Die wenigen Ratschläge, die er erteilt, erweisen sich \textit{alle} als falsch!}
Schauen wir sie uns an:
(a) Er empfiehlt, \enquote{die Finger weit anzuheben}, was für schnelles Spielen offensichtlich nicht in Frage kommt, da es die größte Quelle von Streß ist.
Ich habe nie einen berühmten Pianisten gesehen, der im Konzert die Finger weit anhebt, um einen schnellen Lauf zu spielen; tatsächlich habe ich dies nie \textit{jemanden} tun sehen!
Dieser Rat von Hanon hat eine enorme Zahl von Schülern zu dem Glauben verleitet, daß das Klavier gespielt werden sollte, indem man den Finger anhebt und auf die Taste herunterknallt.
Es ist eine der unmusikalischsten und technisch unkorrektesten Arten zu spielen.
Es ist wahr, daß die Streckmuskeln oft vernachlässigt werden, aber es gibt \hyperref[c1iii7finger]{Übungen}, um dieses Problem direkt zu behandeln.

(b) Er empfiehlt fortlaufendes Üben beider Hände, als ob Klaviertechnik eine Art Training für das Gewichtheben wäre.
Schüler dürfen niemals mit ermüdeten Händen üben.
Deshalb funktioniert die HS-Methode dieses Buchs so gut - sie erlaubt Ihnen, 100\% der Zeit ohne Ermüdung hart zu trainieren, weil eine Hand ruht, während die andere arbeitet.
Ausdauer wird nicht durch Üben mit Ermüdung und Streß erzielt, sondern durch die richtige Konditionierung.
Außerdem brauchen die meisten von uns geistige Ausdauer und keine Ausdauer der Finger.
Und seine Empfehlung ignoriert völlig die \hyperref[c1ii14]{Entspannung}.

(c) Er empfiehlt, unabhängig von Ihrer Fertigkeitsstufe, Ihr ganzes Leben lang jeden Tag zu spielen.
Aber wenn man erst einmal eine Fertigkeit erworben hat, muß man sie nicht immer und immer wieder neu erwerben; man muß nur an der Technik arbeiten, die man noch nicht hat.
Somit gibt es, wenn man alle 60 Stücke gut spielen kann, keine Notwendigkeit, sie weiter zu spielen - was wird man dabei gewinnen?

(d) Er kennt offenbar nur den Daumenuntersatz, während der \hyperref[c1iii5a]{Daumenübersatz} wichtiger ist.

(e) Bei den meisten Übungen empfiehlt er ein festes Handgelenk, was nur teilweise korrekt ist.
Seine Empfehlung offenbart einen Mangel an Verständnis, was \enquote{\hyperref[ruhig]{ruhige Hände}} sind.
(f) Es gibt keine Möglichkeit, einen Großteil der wichtigen Handbewegungen zu üben, obwohl es ein paar Handgelenksübungen für Wiederholungen gibt.



\item \label{c1iii7h8}\textbf{Die Hanon-Übungen erlauben es nicht, mit den Geschwindigkeiten zu üben, die mit den oben beschriebenen \hyperref[c1iii7b]{Übungen für parallele Sets} möglich sind.}
Ohne solche Geschwindigkeiten zu benutzen, können bestimmte hohe Geschwindigkeiten nicht geübt werden, können Sie nicht die \enquote{Über-Technik} trainieren (d.h. mehr Technik als für das Spielen dieser Passage notwendig ist - eine notwendige Sicherheitsreserve für \hyperref[c1iii14]{Auftritte}), und die Hanon-Übungen bieten keine Möglichkeit, ein bestimmtes technisches Problem zu lösen.



\item \label{c1iii7h9}Die ganze Übung ist ein Üben von Verschwendung.
Alle Ausgaben, die ich gesehen habe, drucken die ganzen Läufe, während alles was man braucht, höchstens die ersten 2 aufsteigenden Takte, die ersten 2 absteigenden Takte und der Schlußtakt sind.
Obwohl die Zahl der Bäume, die gefällt wurden, um Hanon zu drucken, bei näherem Hinsehen vernachlässigbar ist, offenbart das die Mentalität hinter diesen Übungen, einfach das intuitiv \enquote{offensichtliche} zu wiederholen, ohne wirklich zu verstehen was man tut oder die wichtigen Elemente jeder Übung aufzuzeigen.
\textbf{\enquote{Wiederholung ist wichtiger als die zugrunde liegenden technischen Konzepte} - das ist wahrscheinlich die schlechteste Einstellung, die die Schüler in der Geschichte des Klaviers am meisten behindert hat.}
Eine Person, die 2 Stunden täglich übt und dabei wie empfohlen eine Stunde Hanon spielt, verschwendet die Hälfte ihrer Klavierzeit!
Eine Person, die 8 Stunden zum Üben zur Verfügung hat, \textit{braucht} keinen Hanon.



\item \label{c1iii7h10}Ich habe festgestellt, daß sich Lehrer ebenfalls in Abhängigkeit davon, ob sie Hanon lehren oder nicht, in zwei Schulen aufteilen.
Diejenigen, die nicht Hanon lehren, neigen dazu sachkundiger zu sein, weil sie die wahren Methoden für den Erwerb der Technik kennen und damit beschäftigt sind, diese zu lehren - für Hanon bleibt dann keine Zeit.
Wenn sie nach einem Klavierlehrer Ausschau halten, erhöhen Sie deshalb die Chancen einen überlegenen zu finden, wenn Sie ihn nur aus denjenigen auswählen, die nicht Hanon lehren.

 
\end{enumerate} 


<!-- c1iii7i.html -->

\subsubsection{Die Geschwindigkeit steigern}
\label{c1iii7i}

Beim Klavierspielen dreht sich alles um eine ausgezeichnete Fingerkontrolle.
Wenn wir die Geschwindigkeit steigern, wird eine solche Kontrolle zunehmend schwierig, da die menschlichen Hände ursprünglich nicht für solche Geschwindigkeiten gemacht sind.
Die Hände sind jedoch komplex und anpassungsfähig, und wir wissen aus der Geschichte, daß solch schnelles Spielen möglich ist.
Deshalb werden wir - so wie im Rest dieses Buchs - versuchen, die richtigen, oder besten, Methoden für das Erreichen  unseres Geschwindigkeitsziels zu finden.


\paragraph{Schneller Anschlag, Entspannung}
\label{c1iii7iAnschlag}

\textbf{Es scheint offensichtlich, daß eine schnelle Anschlagsbewegung der Schlüssel zum schnellen Spielen ist, obwohl sie nicht immer gelehrt wird.
Der wichtigste Punkt für die Geschwindigkeit ist die Fingerbewegung im Knöchelgelenk.}
Jeder Finger besteht aus drei Knochen.
Das Knöchelgelenk ist das Gelenk zwischen Finger und Handfläche.
Beim Daumen ist das Knöchelgelenk sehr nah am Handgelenk.
Denken Sie sich beim schnellen Spielen jeden Finger als eine Einheit, und bewegen Sie ihn einfach am Knöchelgelenk.
Diese Bewegung hat zahllose Vorteile.
Sie benutzt für das Bewegen der Finger nur die eine Muskelgruppe, die auch die schnellste ist.
Den Finger am Knöchel zu bewegen, ist für den Daumen besonders wichtig.
Sie können nichts schnell spielen, wenn der Daumen nicht mit den anderen Fingern Schritt halten kann.
Andere Muskeln zum Beugen der Finger einzubeziehen würde die Bewegung stark verkomplizieren, was zu Verzögerungen der Nervenimpulse auf dem Weg vom Gehirn führen würde.
Das ist die Erklärung, warum der Daumenuntersatz beim schnellen Spielen nicht funktioniert - beim Daumenuntersatz muß man die beiden anderen Gelenke des Daumens beugen, was eine langsamere Bewegung ist.
Das erklärt auch, warum \hyperref[c1iii4b]{flache Fingerhaltungen} schneller sind als die gebogenen.
Konzentrieren Sie sich deshalb, wenn Sie schnell spielen, nicht auf die Fingerspitzen, sondern benutzen Sie das Gefühl, daß die Finger sich an den Knöcheln bewegen.
Die Bewegung an den Knöcheln fördert auch sehr die \hyperref[c1ii14]{Entspannung}.
Wie im folgenden besprochen, müssen Sie beim schnellen Spielen auch das schnelle Entspannen üben.

\textbf{Wir müssen nun jede der drei Komponenten des \hyperref[c1iii1a1]{Basisanschlags} (siehe 1a) beschleunigen.}
Beim schnellen Spielen muß der Abschlag so schnell wie möglich sein, aber in dem Sinn kontrolliert, daß die Noten gleichmäßig sind und die gewünschte Lautstärke haben.
Das Halten ist wichtig, weil Sie während des Haltens sofort entspannen müssen, den Finger aber nicht anheben dürfen, so daß der Fänger nicht vorzeitig gelöst wird.
Dann müssen Sie das Anheben genau zum richtigen Zeitpunkt beginnen; dieses Anheben muß genauso beschleunigt sein.
Oben in \hyperref[c1iii7a]{Abschnitt 7a} haben wir gesehen, daß alle Muskelbündel aus langsamen und schnellen Muskeln bestehen; deshalb müssen wir, wenn wir für die Geschwindigkeit üben, schnelle Muskeln und schnelle Nervenreaktionen entwickeln und die Zahl der langsamen Muskeln reduzieren.
Das bedeutet, daß einfach mit ganzer Kraft stundenlang zu üben kontraproduktiv sein wird.
Schneller zu spielen funktioniert auch nicht, weil es nur erschwert, eine dieser Anschlagkomponenten zu üben, und man am Ende hauptsächlich die falschen Bewegungen übt.
Es bedeutet auch, daß es einige Zeit dauert, die Geschwindigkeit zu entwickeln, weil dazu körperliche Veränderungen im Gehirn, den Muskeln und den Nerven notwendig sind.
Dann muß man alle für die Geschwindigkeit notwendigen Bewegungen lernen.
So möchten Sie sich z.B. nicht angewöhnen, sich in das Klavier zu lehnen, um die Tasten während des Haltens unten zu halten, da keine langsamen Muskeln wachsen sollen - Sie müssen den Fingerdruck sorgfältig kontrollieren, wenn Sie \enquote{für einen guten Klang tief in das Klavier spielen}.
Für die Geschwindigkeit müssen wir jede Komponente des Anschlags separat üben, und wenn sie alle richtig beschleunigt werden, kann man sie zusammenfügen.
Das bedeutet, jede Note langsam zu üben aber jede Komponente schnell auszuführen.
Wenn Sie viele Noten schnell spielen, werden Sie es \textit{niemals} richtig hinbekommen.

Die einfachste Möglichkeit, den schnellen Anschlag zu üben, ist, die fünf Noten von C bis G in Folge zu spielen und dabei jede Komponente des Anschlags sorgfältig zu üben.
Üben Sie die Abwärtsbewegung so schnell Sie können, bewahren Sie aber die Fähigkeit, die Lautstärke zu kontrollieren, den Druck für die Haltekomponente ständig aufrechtzuerhalten und sofort zu entspannen.
Das ist dem Basisanschlag ähnlich, außer daß nun alles beschleunigt sein muß.
Üben Sie während des Übergangs zum Halten das sofortige Entspannen, während Sie genügend Druck aufrechterhalten, um den Fänger in seiner Position zu halten.
Heben Sie für die Aufwärtskomponente den Finger danach schnell und gleichzeitig mit dem Ausführen des Abschlags durch den nächsten Finger.
Alle nicht spielenden Finger sollten nur die Tastenoberflächen berühren und nicht hoch über den Tasten schweben.
Es ist vermutlich einfacher, die Noten erst paarweise zu üben: 121212..., dann 232323... usw.
Spielen Sie zunächst nur eine oder zwei Noten je Sekunde, und werden Sie schrittweise schneller, wenn Sie besser werden.
Übertreiben Sie die Aufwärtsbewegung, da die Streckmuskeln bei den meisten zu schwach sind und zusätzliches Training benötigen.
Beziehen Sie den ganzen Körper mit ein, während Sie entspannt bleiben; das Gefühl ist, daß jede Note ihren Ursprung unten im Bauch hat.
Für diese Übungen sind schnelle Auf- und Abwärtsbewegungen das Ziel, nicht wie schnell Sie aufeinanderfolgende Noten spielen können.

Schnelles Spielen wird nicht durch das Lernen von nur einer Fertigkeit erreicht; es ist eine Kombination vieler Fertigkeiten, und das ist ein weiterer Grund, warum es lange dauert, es zu lernen.
Geschwindigkeit ist wie eine Kette, und die Maximalgeschwindigkeit wird durch das schwächste Glied der Kette begrenzt; deshalb müssen Sie die schnellen Muskeln wachsen lassen, indem Sie die schnellen Komponenten des Anschlags üben, so daß die Geschwindigkeit der Finger nicht zum begrenzenden Faktor wird.
Wenn die Geschwindigkeit steigt, wird es offensichtlich, daß man den Basisanschlag ändern muß, damit man schneller als mit einer bestimmten Geschwindigkeit spielen kann.
Die erste Änderung ist, das Halten wegzulassen, da es nur Zeit verschwendet.
Wir haben jedoch eine wichtige Lektion gelernt, die wir nicht vergessen dürfen - die Entspannung.
Zwischen dem Abschlag und dem Anheben muß ein Moment der Entspannung sein.
Mit anderen Worten: Man möchte nicht in irgend eine der unerwünschten Situationen geraten, die Streß erzeugen.
Einige Schüler werden z.B. diese Bewegung \enquote{vereinfachen}, indem sie alle Streckmuskeln anspannen (alle Finger anheben) und schnell spielen, indem einfach die Beugemuskeln stärker angespannt werden als die Streckmuskeln.
Das baut eindeutig Streß auf und erzeugt eine Geschwindigkeitsbarriere, weil die eigenen Muskeln gegeneinander arbeiten.
Die Lektion, die wir beim Basisanschlag gelernt haben, daß beim Abschlag nur der Beugemuskel und beim Anheben nur der Streckmuskel aktiviert wird, ist für die Geschwindigkeit und die Entspannung entscheidend.


\paragraph{Andere Geschwindigkeitsmethoden}
\label{c1iii7iAndere}

Fügen Sie nun alle anderen Bewegungen hinzu, die zur Geschwindigkeit führen.
Wir befassen uns hier mit \textit{allgemeinen} Geschwindigkeitstricks; es gibt noch mehr \textit{spezielle} Tricks für praktisch jede schwierige, schnelle Passage.
Deshalb sind Übungen wie \hyperref[c1iii7h]{Hanon} so schädlich - sie halten Sie vom Lernen dieser speziellen Tricks ab, indem sie Sie zu dem falschen Glauben verleiten, daß Hanon alle diese allgemeinen und speziellen Probleme lösen wird.
Ein Beispiel für einen speziellen Geschwindigkeitstrick ist der ungewöhnliche Fingersatz der RH ab Takt 20 des dritten Satzes von Beethovens Appassionata (eigentlich gibt es mehrere mögliche Fingersätze.
Es folgen ein paar allgemeine Tricks, die auf viele Arten von Fällen anwendbar sind.

Die \hyperref[c1ii11]{parallelen Sets} lehren Ihnen, alle Finger gleichzeitig zu bewegen, so daß aufeinander folgende Noten viel schneller als mit der Geschwindigkeit jedes einzelnen Fingers gespielt werden können.
Aber ohne zunächst einen soliden Basisanschlag aufzubauen, können die parallelen Sets zu zahlreichen schlechten Angewohnheiten und zu unordentlichem Spielen führen.
Die parallelen Sets alleine trainieren Sie nicht automatisch darauf, den Finger zur richtigen Zeit zu heben, um die Dauer der Noten exakt zu kontrollieren.
Und sie lehren Ihnen nicht notwendigerweise schnelle Anschläge, weil schnelle Anschläge am besten geübt werden, indem man den Basisanschlag langsam übt.
Parallele Sets müssen auch einem anderen Zweck dienen: dem Gehirn die Vorstellung von der Geschwindigkeit zu vermitteln.
Bis man wirklich schnell spielen kann, hat das Gehirn keine Vorstellung davon, was es bedeutet, körperlich schnell zu spielen - man ist also in einem Dilemma: Man kann nicht schnell spielen, aber man muß schnell spielen, um das Gehirn so zu trainieren, daß man schnell spielen kann.
An dieser Stelle kommen die phasengekoppelten parallelen Sets ins Spiel.
Bei den phasengekoppelten parallelen Sets ordnet man einfach die Finger so an, daß einer etwas höher als der andere ist, so daß wenn man die ganze Hand senkt, die einzelnen Finger automatisch nacheinander spielen.
Wenn man die vertikalen Abstände der Finger sehr gering hält, kann man die Noten so schnell spielen, wie man möchte, fast unendlich schnell.
Diese schnellen parallelen Sets sind das, was Sie benötigen, um dem Gehirn die Vorstellung von Geschwindigkeit zu lehren.
Aber die Bewegung der phasengekoppelten parallelen Sets verstößt völlig gegen die Regeln der Bewegungen des Basisanschlags.
Deshalb ist die Idee der parallelen Sets, von zwei extremen Positionen ausgehend zu beginnen - dem Basisanschlag (korrekte Bewegung aber langsam) und den phasengekoppelten parallelen Sets (schnell aber die falschen Bewegungen) - und so zu üben, daß man ungefähr die Mitte trifft, bei der man sowohl die Geschwindigkeit als auch die Genauigkeit hat sowie die Kontrolle, die Unabhängigkeit der Finger, \hyperref[ruhig]{ruhige Hände} und die Entspannung.
Deshalb haben Sie, wenn Sie sehr schnelle parallele Sets nur phasengekoppelt spielen können, nur einen kleinen Teil dessen gelernt, was Sie zur Vollendung benötigen.

Die \hyperref[c1iii4b]{flachen Fingerhaltungen} können schneller als die gebogenen sein, weil sie die Krümmungslähmung vermeiden, und die Fingerspitzen ausgestreckter Finger können sich schneller bewegen als die Spitzen gekrümmter Finger.
Durch das Entspannen der letzten beiden Glieder an der Fingerspitze vereinfachen Sie auch die Bewegung, so daß Sie die schnellen Muskeln in einer geringeren Zahl von Muskelsträngen entwickeln müssen.
Gewöhnen Sie sich an, die flachen und gebogenen Fingerhaltungen fingerweise zu kombinieren (jeder einzelne Finger kann je nach Bedarf gerade oder gebogen sein), weil Ihnen das einen schnelleren Zugriff auf die Tasten geben kann.
Die allgemeine Regel ist: Benutzen Sie die flachen Fingerhaltungen, wenn Sie nur schwarze oder nur weiße Tasten spielen; wenn beide Farben benötigt werden, benutzen Sie die flachen Haltungen für die schwarzen und die gebogenen für die weißen Tasten.
Diejenigen mit langen Fingern müssen eventuell den Daumen und den kleinen Finger flach halten und die Finger 2 bis 4 gebogen.

Geschwindigkeit ist, nach der Musikalität, die am schwersten zu erwerbende Fertigkeit.
Es ist ein weit verbreitetes intuitives Mißverständnis, daß man das schnelle Spielen üben muß, um sich die Geschwindigkeit anzueignen.
Erfahrene Lehrer wissen um die Zwecklosigkeit eines solchen vereinfachten Ansatzes und haben versucht, Methoden für das Aneignen der Geschwindigkeit zu entwickeln.
Ein verbreiteter Ansatz ist, den Schülern vom schnellen Spielen abzuraten - dieser Ansatz wird zumindest alle Arten potentieller, irreversibler Probleme verhindern: psychologische, körperliche, musikalische, technische usw., geht aber das Geschwindigkeitsproblem nicht direkt an und kann den Lernprozeß unnötig verlangsamen.

Die falsche Vorstellung, daß man \enquote{Klaviermuskeln} aufbauen muß, um schnell zu spielen, hat bei vielen dazu geführt, daß sie lauter spielen als es notwendig wäre.
Geschwindigkeit bedeutet Fertigkeit, nicht Kraft.
Deshalb muß man die Lautstärke von der Technik trennen; erwerben Sie zum Spielen lauter Passagen erst die Technik, und fügen Sie dann die Lautstärke hinzu.
Schwierige Passagen führen beim Üben häufig zu Streß und Ermüdung.
Leise zu spielen reduziert beides und beschleunigt dadurch den Technikerwerb.
Wenn man mehrere Jahre Klavier spielt, wird man jedes Jahr stärker, und es kann sein, daß man schließlich entsprechend lauter spielt, ohne es zu merken.
Das lautere Spielen erschwert auch das musikalische Spielen.
Unter Pianisten herrscht Einigkeit darüber, daß es schwierig ist, gleichzeitig leise und schnell zu spielen - und der Grund ist einfach: Man benötigt mehr Technik dazu.

Ein \hyperref[c1iii1a]{guter Klang} wird erzeugt, indem man \enquote{tief in das Klavier hineingeht}.
Man muß aber auch entspannen.
Es ist nicht notwendig, dauernd nach unten zu drücken.
Dieser konstante Abwärtsdruck verschwendet nicht nur Energie (was ermüdet), sondern verhindert auch, daß die Finger sich schnell bewegen.
Es besteht oft die Neigung, sich in das Klavier hineinzulehnen, um \enquote{tief zu spielen}, und nachdem man das mehrere Jahre getan hat, kann es sein, daß man schließlich ohne es zu merken mit einer enormen Kraft nach unten drückt (siehe \hyperref[testimonials06]{Punkt 6 der Leserkommentare}).
Sogar Schüler der Armgewichtsmethode, die den korrekten Armdruck lehrt, haben manchmal am Ende einen ungeeigneten Abwärtsdruck.
Deshalb prüfen Armgewichtslehrer immer den Abwärtsdruck, indem sie die Entspannung des Arms kontrollieren.
Trotz des Begriffs \enquote{Armgewicht} ist das Gewicht des Arms in der Regel nicht die richtige abwärts gerichtete Kraft.
Die Armgewichtsmethode verlangt, daß man genügend entspannt ist, so daß man das Gewicht des Arms spüren kann.
Der optimale Abwärtsdruck hängt von mehreren Faktoren ab (Geschwindigkeit, Lautstärke, staccato-legato usw.).
Eine Möglichkeit, den richtigen Abwärtsdruck zu prüfen, ist, ihn zu verringern, bis man anfängt, Noten auszulassen.
Fügen Sie dann soviel Abwärtsdruck hinzu, daß Sie keine mehr auslassen - das sollte der richtige Wert sein, bei dem Sie für die Technik und die Geschwindigkeit üben - Sie werden vielleicht feststellen, daß Ihr ursprünglicher Druck zu hoch war.
Haben Sie den richtigen Druck gefunden, sollten Sie schneller spielen können.
Die \hyperref[c1iii3]{Triller} und Verzierungen werden auch schneller und klarer.
Das Pianissimo wird sich auch verbessern.
Den Abwärtsdruck zu reduzieren bedeutet nicht, daß man wie staccato spielt oder die Finger über dem unteren Punkt des Anschlags \enquote{schweben}, ohne daß die Fänger greifen, da dieses gegen die Regeln des Basisanschlags verstößt.
Mit dem optimalen Abwärtsdruck wird der \enquote{schwache vierte Finger} - wegen der Reduktion des Stresses und weil der vierte Finger nicht ständig mit den stärkeren Fingern konkurrieren muß - weniger ein Problem sein.
Das richtige Anheben der Finger, das Staccato, Legato usw. werden alle beeinflußt, wenn man die Lautstärke und den Abwärtsdruck ändert.
Trennen Sie deshalb immer die Technik von der Lautstärke, und üben Sie leise aber bestimmt für die Technik.

\hyperref[c1iii1b]{Rhythmus} ist extrem wichtig.
Nicht nur der Rhythmus der Musik, wie sie von den Fingerspitzen gespielt wird, aber auch vom ganzen Körper, so daß sich nicht ein Teil gegen einen anderen bewegt.
Weitere Probleme sind unnötige Bewegungen und solche, die nicht zum Rhythmus passen.
Eine erforderliche Bewegung in einer Hand kann eine unbeabsichtigte Bewegung an einer anderen Stelle des Körpers verursachen.
\textbf{Viele dieser unerwünschten Bewegungen werden sichtbar, wenn man \hyperref[c1iii13]{sich selbst auf Video aufnimmt}.}
Selbstverständlich muß man zunächst einmal den richtigen Rhythmus im Kopf haben; lassen Sie sich den Rhythmus nicht vom Klavier diktieren, da der Rhythmus genauso ein Teil der Musik ist wie die Melodie; man muß ihn bewußt durch das \hyperref[c1ii12]{mentale Spielen} kontrollieren.
Der Rhythmus besteht nicht nur aus der zeitlichen Abfolge, sondern auch aus der Kontrolle des Klangs und der Lautstärke.

Balance ist ein weiterer wichtiger Faktor.
Nicht nur die Balance Ihres Körpers auf der Bank, sondern auch der Schwerpunkt jeder spielenden Hand und der gemeinsame Schwerpunkt beider Hände.
Achten Sie beim HS-Üben darauf, wo der Schwerpunkt der Hand liegt (von wo die Abwärtskraft ausgeht).
Versuchen Sie, diesen Punkt entlang einer Linie zu plazieren, die gerade durch den Arm verläuft.
Das ist nur wichtig, wenn man sehr schnell spielt, da während des langsamen Spielens alle Impulse vernachlässigbar sind und der Schwerpunkt genau in dem Finger liegt, der gerade den Abschlag ausführt, so daß man ihn nicht umherbewegen kann.
Wenn der Schwerpunkt nicht am richtigen Ort ist, muß man zum Ausgleichen zusätzliche Muskeln benutzen, was zu Streß und Ermüdung führt.

In Abhängigkeit von der Situation werden Sie die an anderer Stelle besprochenen Methoden benötigen, wie den \hyperref[c1iii5b]{Daumenübersatz}, die \hyperref[c1iii5wagen]{Wagenradbewegung}, \hyperref[c1iii4b]{mit flachen Fingern spielen}, \hyperref[c1iii8]{Konturieren} usw.
Um auf alle diese Faktoren zu achten, werden Sie häufig mit moderaten oder langsamen Geschwindigkeiten üben müssen.


\paragraph{Geschwindigkeit und Musik}
\label{c1iii7iMusik}

Ein Schlüssel zum Verständnis, wie man für die Geschwindigkeit üben muß, ist die Frage: \enquote{Warum ist die Geschwindigkeit ein ungeeignetes Kriterium, um den Erfolg zu messen?}
Die Antwort ist, daß Geschwindigkeit allein, ohne die richtige Technik, die Musik ruinieren wird.
Deshalb sollten wir die Musik als Kriterium für den Erwerb der Geschwindigkeit benutzen.
D.h. um die Geschwindigkeit zu erwerben, muß man musikalisch spielen.
Die Musikalität ist jedoch nur eine notwendige Bedingung; sie ist keine hinreichende Bedingung.
Musikalisch zu spielen garantiert nicht automatisch die Geschwindigkeit.
Aber zumindest sind wir halbwegs am Ziel!
Wir wissen nun, daß wir schnell spielen können, aber nur bis zu Geschwindigkeiten, bei denen wir die Musikalität aufrechterhalten können.
Eine Lösung ist, nur Kompositionen zu spielen, die so leicht sind, daß man sie musikalisch spielen kann.
Deshalb ist es so wichtig, daß Sie Ihre fertigen Stücke spielen - üben Sie nicht immer nur schwieriges Material und ignorieren Ihre fertigen Stücke.
Die Lösung, nur leichte Stücke zu spielen, ist nicht durchführbar, weil die besten Schüler Stücke spielen möchten, die sie herausfordern, und sie bereit sind, dafür zu arbeiten. 
Viel wichtiger ist vielleicht, daß herausfordernde Stücke Ihnen dabei helfen können, schneller Fortschritte zu machen.
In diesem Fall müssen Sie mehrfach wiederholen: Lernen Sie das Stück zunächst mit langsamerer Geschwindigkeit, so daß Sie es noch musikalisch spielen können; Benutzen Sie dann parallele Sets usw., um schnellere Geschwindigkeiten zu ermöglichen (hauptsächlich HS) und dann das musikalische Spielen mit diesen höheren Geschwindigkeiten zu üben.
Wiederholen Sie dann die gesamte Prozedur, d.h. üben Sie mit verschiedenen Geschwindigkeiten.
Außerdem müssen Sie wissen, wie man den größten Nutzen aus der \hyperref[c1ii15]{automatischen Verbesserung nach dem Üben (PPI)} zieht.

Musikalität kann man nicht in ein paar kurzen Sätzen definieren.
Das ist angesichts der Tatsache, daß auch der Begriff der Musik nicht einfach zu definieren ist, nicht verwunderlich.
Viele Schüler sind darüber verzweifelt, daß sie nicht musikalisch sind.
Auf der anderen Seite verfügen wir alle in dem Sinn über genügend Musikalität, daß wir musikalische Qualität auf sehr hohen Stufen beurteilen können - denken Sie an die häufigen Bemerkungen (auch von Nichtklavierspielern) über die Unzulänglichkeiten oder feinen Unterschiede von Klavierspielern oder berühmten Künstlern.
Wenn es aber darum geht, selbst Musik zu machen, wird es unerklärlicherweise etwas anderes.
Warum?
Die Antwort ist einfach: Es ist nicht so, daß wir nicht musikalisch wären; wir haben nur nicht die Fertigkeit erlernt, musikalisch zu spielen.
Das musikalische Spielen fällt nicht leicht, besonders wenn man keinen guten Lehrer hat, der einem erklären kann, was man falsch macht.
Eine der besten Methoden zur Entwicklung der Musikalität ist, das eigene Spielen \hyperref[c1iii13]{aufzunehmen oder zu filmen}, sich die Aufnahme kritisch anzuhören oder anzusehen und dabei jene hohe Stufe des musikalischen Urteilsvermögens zu benutzen, über die wir alle verfügen.
Das Aufnehmen sollte - unabhängig vom Alter - bereits im ersten Unterrichtsjahr beginnen.
Man muß sich auch professionelle Aufnahmen der Stücke anhören, die man spielt.
Anfänger werden Schwierigkeiten damit haben, Aufnahmen ihrer einfachen Übungsstücke zu finden; bitten Sie in diesem Fall den Lehrer, sie zu spielen, so daß Sie sie aufnehmen können.
Die meisten Klavierspieler hören sich eine hinreichende Menge Musik an, aber der entscheidende Punkt ist hier, daß Sie sich Aufführungen der Stücke, die Sie spielen, anhören müssen.
Der grundlegendste Teil der Musikalität ist Genauigkeit (Taktart usw.) und das Befolgen der Ausdrucksbezeichnungen in den Noten.
\textbf{Fehler beim Notenlesen, besonders beim \hyperref[c1iii1b]{Rhythmus}, können es unmöglich machen, ein Stück auf die endgültige Geschwindigkeit zu bringen.}

Die meisten nehmen einfach an, daß sie, wenn sie üben, auch üben \hyperref[c1iii14]{vorzuspielen}.
Für die meisten Menschen ist das absolut falsch.
Der mentale Zustand für das Üben und das Vorspielen sind üblicherweise zwei völlig voneinander verschiedene Zustände.
Beim Üben möchte der Geist die Technik erwerben und das Stück lernen.
Beim Vorspielen ist dessen einzige Aufgabe, Musik zu erzeugen.
Für einige ist es unmöglich, sich während des Übens in den Zustand des Vorspielens zu versetzen, weil das Gehirn weiß, daß kein Publikum anwesend ist.
Deshalb ist die Videoaufzeichnung oder das Aufnehmen so wichtig; außerdem kann man seine eigenen Stärken und Schwächen sehen und hören.
Machen Sie nicht nur Aufnahmen zu Übungszwecken, sondern auch zur dauerhaften Archivierung des Erreichten - ein Album all Ihrer fertigen Stücke.
Eine gute Möglichkeit dafür ist das Veröffentlichen im Internet.\footnote{Auf deutschen Servern Copyright usw. beachten!} 
Wenn die Aufnahmen nicht für das dauerhafte Archivieren gedacht sind, dann werden sie einfach zu einer weiteren Übungsaufnahme, die sich nicht sehr von den routinemäßigen Übungsaufnahmen unterscheidet.
Alle guten Klavierlehrer veranstalten Konzerte ihrer Schüler; diese Konzerte lehren ihnen den Zustand des Vorspielens.
Sie werden überrascht sein, wie schnell Sie Fortschritte machen, wenn Sie eine Aufnahme bis zu einem bestimmten Termin fertig haben oder auf einem Konzert vorspielen müssen.
Die meisten schreiben diesen Fortschritt dem Druck zu, das Lernen eines Stücks bis zu einem bestimmten Datum abzuschließen, was nur teilweise stimmt.
Eine große Komponente des Fortschritts kann der Psychologie des musikalischen Übens zugeschrieben werden.
Das beweist, daß wir alle wissen, was \enquote{musikalisch spielen} bedeutet.
Aber es mangelt uns an der mentalen Disziplin, musikalisch zu üben.

Es gibt zwei entgegengesetzte Arten, sich der Musikalität zu nähern.
Eine ist die \enquote{künstlerische} Herangehensweise, bei der im Geist ein musikalischer Ausdruck erzeugt wird und die Hände \enquote{einfach alles ausführen}, um den gewünschten Effekt zu erzielen.
Leider können das die meisten Menschen mit normaler Intelligenz nicht - es erfordert wirkliches \enquote{Talent}.
Die andere Art ist der analytische Ansatz, bei dem die Person jede einzelne zur Erzeugung des endgültigen Effekts notwendige anatomische Bewegung lernt.
Leider haben wir keine vollständige Liste all dieser notwendigen Bewegungen.
Wir befinden uns alle irgendwo zwischen diesen beiden gegensätzlichen Arten.
Mit anderen Worten: Erfolgreiche Pianisten beider Extreme werden letzten Endes im Grunde dasselbe tun, so daß es keinen \enquote{korrekten} oder bevorzugten Ansatz gibt - jeder kann von beiden profitieren.

Die Schlußfolgerung ist, daß man die Geschwindigkeit nicht erwerben kann, indem man die Finger zwingt, schneller zu spielen als sie es gemäß ihrer technischen Stufe können, da man die Entspannung verliert, schlechte Angewohnheiten entwickelt und Geschwindigkeitsbarrieren aufbaut.
Der Basisanschlag muß auch bei hohen Geschwindigkeiten aufrechterhalten werden.
Die beste Möglichkeit, innerhalb Ihrer technischen Grenzen zu bleiben, ist, musikalisch zu spielen.
Sie können kurzzeitig parallele Sets, \hyperref[c1iii2]{Zirkulieren} usw. benutzen, um die Geschwindigkeit unter einer geringeren Beachtung der Musikalität schnell zu steigern, aber Sie sollten das die Ausnahme sein lassen, nicht die Regel.
Wenn Sie es notwendig finden, längere Zeit zu zirkulieren, sollte das musikalisch geübt werden.
Das ist ein weiterer Grund, warum das HS-Üben so effektiv ist; man kann HS mit höheren Geschwindigkeiten musikalisch spielen als HT.
Als nächstes müssen Sie alle analytischen Methoden für das Steigern der Geschwindigkeit verinnerlichen, wie die Entspannung, die verschiedenen Hand- und Fingerhaltungen, den \hyperref[c1iii5b]{Daumenübersatz}, den korrekten Abwärtsdruck usw.
Letzten Endes wird das Üben der Geschwindigkeit um der Geschwindigkeit willen kontraproduktiv; wenn Sie Klavier spielen, müssen Sie Musik machen.
Das befreit Sie von dem Geschwindigkeitsdämon und führt Sie in die sagenhafte Welt des wundervollen Klavierklangs.



<!-- c1iii8.html -->

\subsection{Konturieren (Beethovens Sonate \#1)}
\label{c1iii8} 

\textbf{Konturieren ist eine Methode, den Lernprozeß durch die Vereinfachung der Musik zu beschleunigen.}
Es gestattet Ihnen, den musikalischen Fluß oder Rhythmus beizubehalten, und das fast sofort mit der endgültigen Geschwindigkeit.
Das versetzt Sie in die Lage, den musikalischen Gehalt eines Abschnitts, lange bevor dieser befriedigend oder mit der richtigen Geschwindigkeit gespielt werden kann, zu üben.
\textbf{Es hilft Ihnen auch dabei, sich schwierige Techniken schnell anzueignen, da man zunächst den größeren Spielgliedern (Arme, Schultern) lehrt, wie sie sich richtig bewegen müssen; wenn das erreicht ist, finden die kleineren Glieder oftmals leichter ihren Platz.}
Es eliminiert auch viele Fallen für Timing- und für musikalische Interpretationsfehler.
Die Vereinfachungen werden durch verschiedene Mittel erreicht, wie z.B. \enquote{weniger wichtige Noten} zu löschen oder Notenfolgen zu einem Akkord zusammenzufassen.
Sie gehen dann schrittweise zur Originalmusik zurück, indem Sie nach und nach die vereinfachten Noten wieder herstellen.
\hyperref[Whiteside]{Whiteside} hat eine gute Beschreibung des Konturierens auf Seite 141 des ersten Buchs und den Seiten 54-61, 105-107 und 191-196 des zweiten Buchs, in dem verschiedene Beispiele analysiert werden.

Für einen bestimmten Abschnitt gibt es üblicherweise viele Möglichkeiten, den Notensatz zu vereinfachen, und wenn jemand das Konturieren das erste Mal benutzt, braucht es einige Übung, bis er den vollen Nutzen aus der Methode ziehen kann.
Es ist offensichtlich am leichtesten, das Konturieren unter der Anleitung eines Lehrers zu lernen.
Die Idee hinter dem Konturieren ist, daß man zunächst den Zugang zur Musik findet und dadurch die Technik schneller folgt, weil Musik und Technik untrennbar sind.
In der Praxis erfordert es einige Arbeit, bevor das Konturieren nützlich werden kann.
Anders als das HS-Üben usw. kann es nicht so leicht gelernt werden.
Benutzen Sie es nur, wenn es absolut notwendig ist (wenn andere Methoden versagt haben).
Es kann hilfreich sein, wenn Sie es, nachdem Sie Ihre Arbeit mit HS beendet haben, schwierig finden, HT zu spielen.
Das Konturieren kann auch benutzt werden, um die Genauigkeit zu erhöhen und das \hyperref[c1iii6]{Auswendiglernen} zu verbessern.

Ich werde das Konturieren anhand von zwei sehr einfachen Beispielen verdeutlichen.
Allgemeine Methoden zur Vereinfachung sind:
\begin{enumerate}[label={\arabic*.}] 
\item Noten löschen
\item Läufe usw. in Akkorde verwandeln
\item Komplexe Passagen in einfachere umwandeln
 \end{enumerate}
Eine wichtige Regel ist, daß Sie, obwohl die Musik vereinfacht ist, im allgemeinen denselben Fingersatz beibehalten sollten, der vor der Vereinfachung erforderlich war.

Chopin benutzte in seiner Musik oft ein Rubato und andere Mittel, die eine ausgezeichnete Kontrolle und Koordination der beiden Hände erfordern.
In seiner \hyperref[c1iii2fi]{Fantaisie Impromptu (Op. 66)} können die sechs Noten jedes LH-Arpeggios (z.B. \textit{\hyperref[Noten]{C\#3}, G\#3, C\#4, E4, C\#4, G\#3}) zu zwei Noten vereinfacht werden (\textit{C\#3.E4} gespielt mit 51).
Es sollte nicht notwendig sein, die RH zu vereinfachen.
Das ist eine gute Möglichkeit, um sicherzustellen, daß alle Noten der beiden Hände, die auf denselben Schlag fallen, genau zusammen gespielt werden.
Auch wird dies Schülern, die Schwierigkeiten mit dem 3-4-Timing haben, erlauben, mit jeder Geschwindigkeit ohne diese Schwierigkeiten zu spielen.
Wenn Sie die Geschwindigkeit zunächst auf diese Art steigern, wird es später einfacher sein, sich das 3-4-Timing anzueignen, besonders wenn Sie nur einen halben Takt \hyperref[c1iii2]{zirkulieren}.

Die zweite Anwendung ist Beethovens Sonate \#1 (Op. 2, No. 1).
Ich habe im Quellenverzeichnis angemerkt, daß \hyperref[Gieseking]{Gieseking} nachlässig war, als er den vierten Satz trotz des schwierigen und sehr schnellen LH-Arpeggios mit \enquote{bringt keine weiteren neuen Probleme}\footnote{S. 38} bewertete.
Lassen Sie uns versuchen, die wunderbare Arbeit zu vervollständigen, die Gieseking mit der Einführung in diese Sonate geleistet hat, indem wir sicherstellen, daß wir diesen aufregenden Schlußsatz spielen können.

Die ersten vier Triolen der LH kann man lernen, indem man die \hyperref[c1iii7b]{Übungen für parallele Sets} auf jede Triole anwendet und danach \hyperref[c1iii2]{zirkuliert}.
Die \hyperref[c1iii7b1]{Übung \#1 für parallele Sets} ist dabei nützlich (spielen Sie die Triolen als Akkorde), und üben Sie das \hyperref[c1ii14]{Entspannen}.
Die erste Triole im dritten Takt kann auf die gleiche Art geübt werden, mit dem Fingersatz 524524.
Hier habe ich eine falsche Verbindung eingefügt, um ein leichtes, fortlaufendes Zirkulieren zu ermöglichen, damit man in der Lage ist, am schwachen vierten Finger zu arbeiten.
Wenn der vierte Finger stark und unter Kontrolle ist, können Sie die richtige Verbindung 5241 hinzufügen.
Hierbei ist der \hyperref[c1iii5]{Daumenübersatz} erforderlich.
Danach können Sie das absteigende \hyperref[Arpeggios]{Arpeggio} 5241235 üben.
Üben Sie das darauffolgende aufsteigende Arpeggio mit den gleichen Methoden, aber seien Sie darauf bedacht, beim aufsteigenden Arpeggio nicht den Daumenuntersatz zu benutzen, da dies sehr leicht geschehen kann.
Erinnern Sie sich an die Notwendigkeit eines geschmeidigen Handgelenks bei allen Arpeggios.
Für die RH können Sie die Regeln für das Üben von \hyperref[c1iii7e]{Akkorden} und \hyperref[c1iii7f]{Sprüngen} benutzen (Abschnitte 7e und 7f weiter oben).
Bis jetzt ist alles ein Arbeiten mit HS.

Benutzen Sie für das HT-Spielen das Konturieren.
Vereinfachen Sie die LH, so daß Sie nur die Schlagnoten spielen (beginnend mit dem 2. Takt): \textit{F3, F3, F3, F3, F2, E2, F2, F3}, mit dem Fingersatz 55\textbf{5}155\textbf{5}1, der fortlaufend zirkuliert werden kann.
Das sind nur die ersten Noten jeder Triole.
Haben Sie das HS gemeistert, können Sie mit dem HT beginnen.
Wenn das mit HT zufriedenstellend gelingt, wird das Hinzufügen der Triolen einfacher, und es besteht die viel geringere Wahrscheinlichkeit, daß Sie dabei Fehler in sich aufnehmen.
Da diese Arpeggios die herausforderndsten Teile dieses Satzes sind, können Sie nun durch deren Konturieren den ganzen Satz mit jeder Geschwindigkeit üben.

Bei der RH sind die ersten drei Akkorde \textit{piano} und die zweiten drei \textit{forte}.
Üben Sie am Anfang hauptsächlich die Genauigkeit und die Geschwindigkeit, d.h. üben Sie alle sechs Akkorde \textit{piano}, bis dieser Abschnitt gemeistert ist.
Fügen Sie dann das \textit{forte} hinzu.
Um zu vermeiden, daß Sie die falschen Noten treffen, gewöhnen Sie sich an, die Tasten der Akkorde zu erfühlen, bevor Sie sie niederdrücken.
Stellen Sie bei der RH-Oktavmelodie der Takte 34-36 sicher, daß Sie ohne jegliches \textit{crescendo} spielen, besonders das letzte G.
Und die ganze Sonate wird natürlich ohne Pedal gespielt.
Um jede Möglichkeit eines katastrophalen Endes zu eliminieren, achten Sie darauf, daß Sie die letzten vier Noten dieses Satzes mit der LH spielen und diese immer ein gutes Stück früher als notwendig in Position bringen.

\label{pausen}\footnote{Anfänger (ich hoffe, nur diese) neigen oft dazu, an den Tasten zu \enquote{kleben}, die sie gerade zum Spielen benutzt haben, und die Hände erst im letzten Moment zu den als nächstes zu spielenden Tasten zu bewegen.
In vielen Stücken, genau wie in dieser Sonate, hat der Komponist an solchen Stellen freundlicherweise eine Pause, Staccato-Noten oder das Ende einer 
Phrase eingefügt, um Ihnen Zeit zu geben, die Position der Hand zu verändern.
\textbf{\enquote{Spielen} Sie die Bewegung der Hand während dieser Pausen ganz bewußt, und lernen Sie diese Bewegung so wie alle anderen Bewegungen auswendig.}}

\textbf{Für das Erwerben der Technik sind die anderen Methoden dieses Buchs üblicherweise effektiver als das Konturieren, das, auch wenn es funktioniert, zeitaufwendig sein kann.}
Wie bei dem obigen Beispiel der Sonate kann ein einfaches Konturieren Sie jedoch in die Lage versetzen, einen ganzen Satz mit der vorgegebenen Geschwindigkeit zu üben und die meisten musikalischen Gesichtspunkte einzubeziehen.
Währenddessen können Sie die anderen Methoden dieses Buchs benutzen, um sich die Technik anzueignen, die notwendig ist, um die Konturen zu füllen.



<!-- c1iii9.html -->

\subsection{Ein Stück auf Hochglanz bringen - Fehler beseitigen}
\label{c1iii9}

Beim Ausfeilen eines \enquote{fertigen} Stückes möchte man fünf Ziele erreichen: ein gutes \hyperref[c1iii6]{Gedächtnis} gewährleisten, Fehler beseitigen, \hyperref[c1iii14d]{Musik machen}, die Technik weiterentwickeln und sich \hyperref[c1iii14]{auf Auftritte vorbereiten}.
\textbf{Der erste Schritt ist das Gewährleisten des Gedächtnisses, und wir haben in Abschnitt III.6 gesehen, daß es dazu am besten ist, das ganze Stück in Gedanken - ohne das Klavier - zu spielen.}
\hyperref[c1ii12]{Mentales Spielen} garantiert, daß das Gedächtnis praktisch unfehlbar ist.
Wenn einige Teile etwas unsicher sind, können Sie jederzeit an ihnen arbeiten, auch wenn Sie nicht am Klavier sind.
Das mentale Spielen ist die sicherste Form des Gedächtnisses, weil es ein rein mentales Gedächtnis ist - es ist nicht von akustischen, taktilen oder visuellen Reizen abhängig.
Es beseitigt auch die meisten Fehler, weil diese ihren Ursprung im Gehirn haben.
Sehen wir uns ein paar verbreitete Ursachen von Fehlern an.
Erinnerungsblockaden treten wegen einer zu großen Abhängigkeit vom \hyperref[c1iii6hand]{Hand-Gedächtnis} auf.
\hyperref[c1ii22]{Stottern} ist die Angewohnheit, bei jedem Fehler anzuhalten, während man HT spielt, ohne vorher genügend HS geübt zu haben.
Man trifft falsche Noten, weil die Hände nicht stets die Tasten fühlen und man nicht mehr weiß, wo welche Tasten sind.
Fehlende Noten resultieren aus einem Mangel an \hyperref[c1ii14]{Entspannung} und dem ungewollten Heben der Hände - eine Angewohnheit, die man üblicherweise durch zuviel langsames HT-Üben erwirbt.
Wir haben Lösungen  zum Beseitigen all dieser Fehlerquellen besprochen.
Das musikalische Spielen und das Hervorbringen der \enquote{Farbe} einer Komposition ist die endgültige Aufgabe beim Ausfeilen.
Man kann nicht einfach nur die Noten exakt spielen und erwarten, daß die Musik und die Farbe auf magische Weise zum Vorschein kommen; man muß beides aktiv in Gedanken erzeugen, bevor man die Noten spielt.
Das mentale Spielen gestattet Ihnen das alles.
Wenn die Finger diese geistigen Vorstellungen nicht reproduzieren können, ist vielleicht das Stück zu schwierig.
Sie werden die Technik schneller entwickeln, indem Sie Stücke üben, die Sie bis zur Perfektion ausfeilen können.
Geben Sie aber auch nicht zu leicht auf, da die Ursache der Schwierigkeiten eventuell nicht bei Ihnen liegt, sondern ein anderer Faktor ist, wie z.B. die Qualität oder der Zustand des Klaviers.

Ein großer Teil des Ausfeilens ist die Aufmerksamkeit gegenüber den Details.
Die beste Art, den korrekten Ausdruck sicherzustellen, ist, zu den Noten zurückzugehen und jedes Ausdruckszeichen, jedes Staccato, jede Pause, Tasten, die unten gehalten werden, das Heben des Fingers und des Pedals usw. noch einmal durchzugehen.
Das wird Ihnen das exakteste Bild der logischen Struktur der Musik vermitteln, das notwendig ist, um den richtigen Ausdruck hervorzubringen.
Die Schwächen jedes Einzelnen sind verschieden und sind diesem im allgemeinen nicht bekannt.
Jemand, dessen Timing unsauber ist, kann üblicherweise das falsche Timing nicht hören.
\textbf{Das ist der Punkt, an dem ein Lehrer eine Schlüsselrolle beim Erkennen dieser Schwächen spielt.}\footnote{\enquote{Digital-Pianisten} können diese Probleme teilweise selbst verringern, indem sie ihr Spiel \hyperref[c1iii13MIDI]{mit einem Sequenzer-Programm aufnehmen} und sich die \hyperref[midi_check]{MIDI-Signale genauer ansehen}.}

Musik zu machen ist der wichtigste Teil des Ausfeilens eines Stücks.
Einige Lehrer betonen diesen Punkt, indem sie sagen, man solle 10\% seiner Zeit mit dem Erlernen der Technik und 90\% der Zeit mit dem Lernen, Musik zu machen, verbringen.
Die meisten Schüler ringen mehr als 90\% ihrer Zeit mit der Technik, in dem falschen Glauben, daß zu üben, was man nicht spielen kann, die Technik entwickeln wird.
Dieser Fehler erwächst aus der intuitiven Logik, daß man, wenn man etwas übt, das man nicht spielen kann, irgendwann in der Lage sein sollte, es zu spielen.
Das stimmt aber nur für Material, das im Rahmen Ihrer Fertigkeitsstufe ist.
Bei zu schwierigem Material weiß man nie, was geschieht, und häufig führt solch ein Versuch zu irreversiblen Problemen wie Streß und \hyperref[c1iv2b]{Geschwindigkeitsbarrieren}.
Wenn Sie z.B. die Geschwindigkeit steigern möchten, ist der schnellste Weg dazu, leichte Stücke zu spielen, die Sie bereits ausgefeilt haben, und dieses Spielen zu beschleunigen.
Wenn die Geschwindigkeit der Finger erst einmal steigt, sind Sie bereit, schwierigeres Material mit einer höheren Geschwindigkeit zu spielen.
\textbf{Somit ist die Zeit des Ausfeilens auch die beste Zeit für die technische Entwicklung und kann sehr viel Spaß machen.}

Ihre Fertigkeiten zum Vorspielen zu perfektionieren, ist Teil des Ausfeilens; das wird unten in \hyperref[c1iii14]{Abschnitt 14} besprochen.
Viele Klavierspieler begegnen dem folgenden merkwürdigen Phänomen.
Es gibt Zeiten, in denen sie nichts falsch machen können und sich ohne Fehler oder Probleme die Seele aus dem Leib spielen können.
Zu anderen Zeiten wird jedes Stück schwierig und sie machen Fehler an Stellen, die ihnen normalerweise keine Probleme bereiten.
Was verursacht diese Höhen und Tiefen?
Nicht zu wissen, welcher der beiden Zustände auf einen zukommt, kann ein schrecklicher Gedanke sein, der \hyperref[c1iii15]{Nervosität} hervorrufen kann.
Offensichtlich gibt es viele Faktoren, wie z.B. \hyperref[fpd]{FPD}, der besonnene Gebrauch des \hyperref[c1ii17]{langsamen Spielens} usw.
Der wichtigste Faktor ist jedoch das \hyperref[c1ii12]{mentale Spielen}.
Alle Klavierspieler benutzen, bewußt oder unbewußt, etwas mentales Spielen.
Das Vorspielen hängt oft von der Qualität dieses mentalen Spielens ab.
Solange man das mentale Spielen nicht bewußt benutzt, weiß man nie, in welchem Zustand es ist.
So stört z.B. das Üben eines neuen Stücks das mentale Spielen eines anderen Stücks.
Deshalb ist es so wichtig, zu wissen, was dieses mentale Spielen ist, ein gutes mentales Spielen aufzubauen und zu wissen, wann man es überprüfen und wieder auffrischen muß.
Wenn Ihr mentales Spielen sich aus irgendeinem Grund verschlechtert hat, wird es vor einem Konzert zu überprüfen Sie auf die drohende Gefahr aufmerksam machen und Ihnen die Gelegenheit geben, den Schaden zu reparieren.

Ein verbreitetes Problem ist, daß Schüler dauernd neue Stücke lernen und wenig Zeit für das Ausfeilen der Stücke haben.
Das passiert hauptsächlich Schülern, die die intuitiven Lernmethoden benutzen.
Es dauert so lange, jedes einzelne Stück zu lernen, daß keine Zeit bleibt, sie auszufeilen, bevor man mit einem anderen Stück anfangen muß.
Die Lösung sind natürlich bessere Lernmethoden.
 
Zusammengefaßt \textbf{ist ein solides \hyperref[c1ii12]{mentales Spielen} die wichtigste Voraussetzung dafür, ein Stück auszufeilen und es für einen Auftritt vorzubereiten}.
Fortgeschrittene Technik erlangt man nicht nur durch das Üben neuer Fertigkeiten, sondern auch durch das Spielen fertiger Stücke.
Das ständige Üben neuer Fertigkeiten ist sogar kontraproduktiv und führt zu Geschwindigkeitsbarrieren, Streß und unmusikalischem Spielen.
 


<!-- c1iii10.html -->

\subsection{Kalte Hände, rutschende Finger, Krankheiten, Handverletzungen, Gehörschäden}
\label{c1iii10}

\subsubsection{Kalte Hände}

Kalte, steife Hände an einem kalten Tag sind ein verbreitetes Leiden, das hauptsächlich durch die natürliche Reaktion des Körpers auf die Kälte verursacht wird.
Ein paar Menschen haben sicherlich pathologische Probleme, die eventuell medizinische Betreuung erfordern, aber die Mehrzahl der Fälle sind natürliche Reaktionen des Körpers auf Unterkühlung.
In diesem Fall zieht der Körper das Blut - hauptsächlich aus den Extremitäten - zum Zentrum des Körpers zurück, um den Wärmeverlust gering zu halten.
Die Finger sind für diese Abkühlung am anfälligsten, gefolgt von den Händen und den Füßen.

In solchen Fällen ist die Lösung im Prinzip einfach.
Man muß nur die Körpertemperatur anheben.
In der Praxis ist das oft nicht so leicht.
In einem kalten Raum wird das Problem auch dadurch, daß die Körpertemperatur (durch zusätzliche Bekleidung) so weit erhöht wird, daß einem zu warm ist, nicht immer eliminiert.
Sicherlich sollte jede Methode zur Vermeidung von Wärmeverlusten helfen.
Natürlich ist es am besten, wenn man die Raumtemperatur erhöhen kann.
Wenn nicht, sind allgemeine Hilfen:
\begin{enumerate}[label={\arabic*.}] 
\item die Hände bzw. Arme in warmes Wasser eintauchen
\item ein Heizgerät, wie z.B. ein tragbarer Heizstrahler (ca. 1 kW), den Sie direkt auf den Körper richten können
\item dicke Socken, Pullover oder thermale Unterwäsche
\item Handschuhe ohne Finger (damit Sie mit den Handschuhen Klavier spielen können)
\item \footnote{Wer autogenes Training o. ä. beherrscht, kann es auch mit den dafür erlernten Techniken zur Steuerung der Durchblutung versuchen.}
 \end{enumerate}
Wenn Sie die Hände nur vor dem Spielen warmhalten möchten, sind Fausthandschuhe wahrscheinlich besser als Fingerhandschuhe.
Die meisten Haartrockner haben nicht genügend Energie, sind nicht dafür entwickelt, länger als ungefähr 10 Minuten benutzt zu werden, ohne gefährlich zu überhitzen, und sind für den Zweck, warme Luft um einen Klavierspieler zu erzeugen, zu laut.

Es ist nicht klar, ob es besser ist, die ganze Zeit warm zu bleiben oder nur, wenn man Klavier übt.
Wenn man die ganze Zeit warm bleibt (wie z.B. durch das Tragen thermaler Wäsche), wird der Körper eine Abkühlung eventuell nicht erkennen und deshalb den gewünschten Blutfluß aufrechterhalten.
Auf der anderen Seite wird der Körper eventuell sensibler gegenüber Kälte und schließlich auch, wenn der Körper warm ist, mit kalten Händen darauf reagieren, daß der Raum kalt ist.
Wenn man z.B. immer die fingerlosen Handschuhe trägt, gewöhnen sich die Hände an diese Wärme und fühlen sich sehr kalt an, wenn man die Handschuhe entfernt.
Und der wärmende Effekt dieser Handschuhe geht eventuell nach und nach verloren, wenn sich die Hände daran gewöhnt haben.
Deshalb ist es wahrscheinlich am besten, sie nur beim Üben oder nur vor dem Üben zu tragen.
Das Gegenargument ist, daß sie immer zu tragen es Ihnen gestattet, zu jeder Zeit Klavier zu spielen, ohne Aufwärmen oder die Hände in warmes Wasser tauchen zu müssen.
Das ist sicherlich ein komplexes Problem, und nur Handschuhe zu tragen löst im allgemeinen das Problem nicht und kann es verschlimmern.

\textbf{Die Spielmuskeln befinden sich in den Armen, wenn Sie also die Klaviermuskeln erwärmen möchten, ist es wichtiger, die Unterarme und Ellbogen zu erwärmen als die Finger.}
Es ist sogar jeder Muskel von den Unterarmen bis zur Körpermitte in das Klavierspielen einbezogen.
Deshalb sollten Sie, wenn Sie warmes Wasser benutzen, um die Hände vor einem Auftritt zu erwärmen, versuchen, die Unterarme einzutauchen, besonders die obere Hälfte (in der Nähe der Ellbogen), wo die Beuge- und Streckmuskeln konzentriert sind.
Wenn das nicht möglich ist, dann müssen Sie Ihre Hände lange genug eintauchen, daß das warme Blut von den Händen in die Arme fließen kann.
Die tiefen Hohlhandmuskeln (die Mm. lumbricales unter und die Mm. interossei zwischen den Mittelhandknochen) befinden sich in den Händen, diese müssen deshalb auch erwärmt werden.

Kalte Finger dieser Art sind klar die Reaktion des Körpers auf niedrige Temperaturen.
\textbf{Die beste Lösung mag sein, die Hände mehrmals am Tag in sehr \textit{kaltes} Wasser zu tauchen, um sie an niedrige Temperaturen zu gewöhnen.
Dann reagieren sie vielleicht überhaupt nicht auf Kälte.
Das könnte eine dauerhafte Lösung bieten.}
Sie könnten die Hände z.B. direkt nach dem Üben auf diese Art kühlen, so daß es das Üben nicht stört.
Das Ziel des Kühlens ist, die Haut an kalte Temperaturen anzupassen.
Tauchen Sie die Hände nicht länger als 5 bis 10 Sekunden in kaltes Wasser; kühlen Sie nicht die ganze Hand bis auf die Knochen ab.
Sie könnten sogar zunächst die Hände in warmem Wasser aufwärmen und dann nur die Haut in eiskaltem Wasser kühlen.
Solch eine Behandlung sollte sich gut anfühlen, ohne jeglichen Kälteschock oder Schmerz.
Das ist genau das Prinzip hinter der nordischen Praxis, nach einer heißen Sauna in eine Öffnung in einem gefrorenen See zu springen.
Diese scheinbar masochistische Praxis ist in Wahrheit völlig schmerzlos und hat nützliche Konsequenzen, wie die Haut an kalte Temperaturen anzupassen und die Schweißbildung zu stoppen, die sonst dazu führen würde, daß die Kleidung durchnässen und in der extremen Kälte gefrieren würde.
Ohne in kaltes Wasser zu springen, könnte jemand mit nach der Sauna naßgeschwitzter Kleidung sogar erfrieren!
Die Poren der Haut können geschlossen werden, indem man die Hände nach dem Erwärmen in kaltes Wasser taucht.
So wird das Schwitzen verhindert, und die Wärme bleibt in den Händen gespeichert.
 

\subsubsection{Rutschende (trockene oder schwitzende) Finger}
\label{c1iii10rutschen}

Wenn die Finger übermäßig trocken oder feucht sind, können sie rutschig werden.
Zu häufiges Waschen mit starken Reinigungsmitteln kann die Hände trocken werden lassen.
Die Anwendung der meisten qualitativen Feuchtigkeitslotionen wie z.B. Xxxxxxx wird das Problem lösen.
Um zu vermeiden, daß Sie die Klaviertasten mit der überschüssigen Lotion beschmieren, tragen Sie jeweils nur eine geringe Menge der Lotion auf, und warten Sie, bis sie vollständig in die Haut eingezogen ist, bevor Sie wieder neue auftragen.
Mehrmals kleine Menge aufzutragen hält länger vor als einmal viel aufzutragen.
Wischen Sie vor dem Klavierspielen alle überschüssigen Reste ab.
Menschen, die zum Schwitzen während des Spielens neigen, müssen auch auf rutschige Finger achten.
Wenn Sie zunächst eine Lotion aufgetragen haben, weil Ihre Hände trocken waren, Sie aber während des Spielens zu schwitzen  anfangen, dann können Sie in große Schwierigkeiten geraten, wenn auf den Finger noch überschüssige Lotion ist.
Seien Sie deshalb, wenn Sie zum Schwitzen neigen, mit jeder Art von Lotion vorsichtig.
Sogar ohne Lotion können feuchte oder trockene Finger rutschen.
Üben Sie in diesem Fall, mit \hyperref[c1iii4SchubZug]{Schub- und Zugbewegungen} zu spielen, so daß Sie die Position der Finger genauer kontrollieren können.
Diese Bewegungen erfordern ein gewisses Gleiten der Finger über die Tasten und sind deshalb für rutschige Finger besser geeignet.


\subsubsection{Krankheiten}
\label{c1iii10krank}

\textbf{Einige Menschen könnten glauben, daß eine harmlose Krankheit, wie z.B. eine Erkältung, es ihnen immer noch erlaubt, Klavier zu üben.
Schließlich gibt es, während man wegen der Erkältung zuhause bleibt, nichts zu tun, und Klavierspielen wird nicht als anstrengende Tätigkeit angesehen.
Das ist eine schlechte Idee.}
Es ist besonders wichtig für Eltern, zu verstehen, daß Klavierspielen, insbesondere für das Gehirn, eine enorme Anstrengung darstellt, und daß das Klavierspielen im Falle einer Krankheit nicht als ein entspannender Zeitvertreib behandelt werden soll.
Deshalb sollten Kinder, auch bei leichten Erkältungen, nicht zum Klavierüben gezwungen werden, solange sie nicht spielen möchten.
Das Gehirn ist während des Klavierspielens aktiver als die meisten glauben.
Infektionen wirken sich nicht auf den ganzen Körper gleich aus; sie setzen sich gewöhnlich bevorzugt in gestreßten Organen fest.
Wenn man Fieber hat und Klavier spielt, besteht ein gewisses Risiko für eine Schädigung des Gehirns\footnote{sofern das Fieber durch die Anstrengung stark ansteigt}.
Zum Glück verlieren die meisten Menschen die Lust zum Klavierüben bereits, wenn sie nur leicht krank sind, und das ist ein deutliches Signal, daß man nicht üben sollte.

Ob jemand Klavier spielen kann, wenn er krank ist, ist eine persönliche Angelegenheit.
Ob man spielt oder nicht, ist für den Klavierspieler ziemlich klar; die meisten Menschen fühlen den Streß des Klavierspielens bereits, bevor die Symptome der Krankheit deutlich werden.
Deshalb ist es wahrscheinlich am sichersten, die Entscheidung, zu spielen oder nicht zu spielen, dem Klavierspieler zu überlassen.
\textbf{Es ist nützlich, zu wissen, daß wenn Sie sich plötzlich müde fühlen oder andere Symptome spüren, die das Spielen erschweren, es ein Anzeichen dafür sein kann, daß Sie krank werden.}
Das Problem damit, während einer Krankheit nicht zu spielen, ist, daß die Hände einen beträchtlichen Teil der Technik verlieren, wenn die Krankheit länger als eine Woche dauert.
Vielleicht sind Übungen, die das Gehirn nicht belasten, wie z.B. \hyperref[c1iii5]{Tonleitern}, \hyperref[Arpeggios]{Arpeggios} und Übungen der \hyperref[c1iii7h]{Hanon}-Art, in einer solchen Situation geeignet.
 

\subsubsection{Gesundes und ungesundes Üben}
\label{c1iii10ungesund}

Es ist wichtig, etwas über die Auswirkungen des Klavierübens auf die Gesundheit zu lernen, da jede Tätigkeit auf gesunde oder ungesunde Weise ausgeübt werden kann.
Ein streßfreies, psychologisch gesundes Herangehen an das Klavierüben kann die Gesundheit einer Person stärken, hingegen kann es ungesund sein, ohne Beachtung des Wohlbefindens zu üben.
Es ist wichtig, das richtige Atmen zu lernen, um einen Sauerstoffmangel zu vermeiden.
Aus der Unfähigkeit zum \hyperref[c1iii6]{Auswendiglernen} oder zum Erwerben bestimmter Fertigkeiten resultierende Frustrationen müssen durch das Lernen effizienter Übungsmethoden verhindert werden.
In diesem Buch werden Methoden zum \hyperref[c1ii14]{Vermeiden von Ermüdung} besprochen.
\hyperref[c1iii10hand]{Verletzungen der Hand} sind vermeidbar.
Übermäßige \hyperref[c1iii15]{Nervosität} ist nicht nur schlecht für das \hyperref[c1iii14]{Auftreten}, sondern auch für die Gesundheit.
Wir müssen die richtigen Beziehungen zwischen den Schülern, Lehrern, Eltern und dem Publikum bedenken oder durch die Erfahrungen lernen.
Indem man die gesundheitlichen Aspekte beachtet, kann das Klavierüben zu einer nützlichen Aktivität werden, die genauso wirksam ist, wie die richtige Ernährung und das richtige Training.


\subsubsection{Verletzungen der Hand}
\label{c1iii10hand}

Handverletzungen sind ungefähr bis zur Mittelstufe für die Schüler im allgemeinen kein großes Problem.
Für fortgeschrittene Klavierspieler sind sie ein wichtiges Thema, weil die menschliche Hand nicht dafür gedacht ist, solchen extremen Belastungen standzuhalten.
Verletzungsbedingte Probleme sind bei professionellen Pianisten denen von professionellen Sportlern z.B. im Tennis, Golf oder Fußball ähnlich.
\textbf{Deshalb sind nach der zum Üben zur Verfügung stehenden Zeit die Einschränkungen durch mögliche Verletzungen vielleicht die zweitwichtigsten.}
Es mag so erscheinen, als ob keine Verletzungen auftreten sollten, da die \hyperref[c1ii14]{Entspannung} eine wichtige Komponente der Klaviertechnik ist.
Leider sind die körperlichen Anforderungen des Spielens auf fortgeschrittenen Stufen so hoch, daß (wie im Sport) Verletzungen trotz der bekannten Vorsichtsmaßnahmen und anderer Mittel, die professionelle Spieler anwenden, sehr wohl auftreten können.
Verletzungen treten häufig beim Üben zum Erwerb schwieriger Techniken auf.
Schüler, die die Methoden dieses Buchs benutzen, müssen sich der Möglichkeit der Verletzung besonders bewußt sein, weil sie schnell damit anfangen werden, Material zu üben, das hohe technische Fertigkeiten verlangt.
Deshalb ist es wichtig, die verbreiteten Arten von Verletzungen zu kennen, und zu wissen, wie man sie vermeidet.

\textbf{Jede Verletzung hat eine Ursache.}
Obwohl es eine Vielzahl dokumentierter Berichte über Verletzungen und Erfolge bzw. Fehlschläge von Heilanwendungen gibt, sind definitive Informationen über Ursachen und Heilung schwer zu finden.
Die einzigen Heilmittel, die erwähnt werden, sind Ruhe und schrittweises Zurückkehren zum Spielen mit streßfreien Methoden.
Ich verletzte mir z.B. die Beugesehnen meiner linken Hand durch die Benutzung von Golfschlägern mit abgenutzten, harten Griffen, obwohl ich immer Golfhandschuhe trug.
Mein Orthopäde diagnostizierte sofort die Ursache meiner Schmerzen (eine Kerbe in meinen Sehnen), konnte mir aber nicht sagen, wie ich meine Hand verletzt hatte, so daß er mir nicht richtig sagen konnte, wie man es heilt.
Ich fand später heraus, daß der Druck des Golfgriffs die Kerben in meinen Sehnen erzeugt hatte, und diese Kerben bewegten sich in meiner Hand während des Klavierspielens auf und ab; die daraus resultierende Reibung erzeugte nach langen Übungseinheiten am Klavier Entzündungen und Schmerzen.
Der Arzt zeigte mir, wie man diese Kerben fühlen kann, wenn man auf die Sehne drückt und den Finger bewegt.
Nun ersetze ich die Griffe meiner Schläger häufiger und habe in meinen Golfhandschuh Polster eingesetzt (aus Xxxxxxxxs selbstklebenden Fußpolstern geschnitten\footnote{Ich will hier keine Schleichwerbung für eine bestimmte Marke machen und habe auch keine Werbeverträge. Wer es also genau wissen möchte, den verweise ich auf den Originaltext.}), und mein Problem ist gelöst.
Das jahrelange zu feste Greifen des Schlägers (ich wußte damals noch nichts über \hyperref[c1ii14]{Entspannung}) führte jedoch an meinen Händen zu einem dauerhaften Schaden, so daß meine Finger nicht so unabhängig sind, wie ich es gerne hätte.

Man kann sich versehentlich bestimmte Muskeln oder Sehnen zerren, besonders in den Schultern und im Rücken.
Das wird meistens durch ein schlechtes Ausrichten der Hände oder des Körpers und durch nicht ausbalanciertes Spielen verursacht.
Das beste Vorgehen ist hier Vorsicht - Klavierspieler müssen besonders vorsichtig sein und solche Verletzungen vermeiden, weil es Jahre dauern kann, bis sie geheilt sind.
Hören Sie mit dem Üben auf, sobald Sie einen Schmerz spüren.
Ein paar Tage Pause werden Ihrer Technik nicht schaden und ernste Verletzungen vermeiden.
Natürlich ist es am besten, zu einem Orthopäden zu gehen; viele Orthopäden sind jedoch mit Verletzungen durch Klavierspielen nicht vertraut.

Fingerspitzen können durch zu hartes (lautes) Spielen verletzt werden.
Dieser Zustand kann mit geeigneter Bandage etwas gemildert werden.
\textbf{Die \hyperref[c1ii2]{gebogene Fingerhaltung} kann Prellungen der Fingerspitzen verursachen, weil das Polster zwischen Knochen und Haut an der Spitze minimal ist.}
Bei der gebogenen Haltung kann es auch passieren, daß sich das Fleisch unter dem Fingernagel von diesem löst, wenn man die Fingernägel zu kurz schneidet.
Sie können beide Arten der Verletzung vermeiden, indem Sie die \hyperref[c1iii4b]{flachen Fingerhaltungen} benutzen (s. Abschnitt III.4b).

Die meisten Handverletzungen sind vom Typ der Verletzungen durch wiederholten Streß (RSI = repetitive stress injury).
Das Karpaltunnelsyndrom und Sehnenentzündungen sind verbreitete Leiden.
Erlebnisberichte legen nahe, daß chirurgische Eingriffe im allgemeinen das Problem des Karpaltunnelsyndroms nicht lösen und mehr schaden als nutzen können.
Hinzu kommt, daß chirurgische Eingriffe irreversibel sind.
Zum Glück haben Masseure vor kurzem das Problem gelöst, das Karpaltunnelsyndrom zu heilen.
Warum Masseure?
Weil sowohl Pianisten als auch Masseure Ihre Finger als ihr hauptsächliches berufliches Werkzeug benutzen.
Deshalb leiden sie unter den gleichen Verletzungen.
Masseure sind jedoch eher in der Lage, zu experimentieren und Heilmethoden zu entdecken, während Pianisten nicht medizinisch ausgebildet sind und nicht einmal wissen, wie sie ihre Gebrechen diagnostizieren sollen.
Es ist jedoch glücklicherweise so, daß man Schmerzen bereits lange bevor ein irreversibler Schaden auftreten kann spürt, so daß das Syndrom geheilt werden kann, wenn man es behandelt, sobald man Schmerzen fühlt.
Obwohl man die Schmerzen üblicherweise in der Nähe der Handgelenke spürt, liegt die Ursache der Schmerzen nicht in den Handgelenken, sondern hauptsächlich in den Armen und im Nacken, wo große Muskeln und Sehnen schädliche Kräfte auf die Sehnen ausüben können, die durch den querliegenden Sehnenring des Handgelenks verlaufen, der alle zu den Fingern führende Sehnen bündelt.
Deshalb beseitigt eine Behandlung des Handgelenks nicht die Schmerzen und verschlimmert eine Operation des Handgelenks nur das Problem.
Die Gruppe mit den fortgeschrittensten Methoden zur Behandlung des Karpaltunnelsyndroms sind die Spezialisten für SET-Massage (Structural Energetic Therapy	extregistered); Sie beginnen mit dem Schädel und gehen dann zu einer Behandlung der tieferen Gewebeschichten der entsprechenden Bereiche des Kopfes, der Arme und des Körpers über.
Das Einbeziehen des Schädels ist notwendig, weil es am schnellsten Erleichterung verschafft und die Behandlung des Gewebes alleine das Problem nicht beseitigt.
Bevor man eine Behandlung bekommt, ist es schwer vorstellbar, daß die Schädelknochen eine Auswirkung auf das Karpaltunnelsyndrom haben.
Mehr Informationen dazu finden Sie auf der Website von SET (www.structuralenergetictherapy.com).
Obwohl diese Website für Masseure gedacht ist, können Sie erfahren, was in die Behandlung des Karpaltunnelsyndroms einbezogen wird, bis zu welchem Ausmaß es heilbar ist und wie Sie den richtigen Therapeuten finden.
Bis jetzt sind wenige Therapeuten in diesem Verfahren ausgebildet, aber zumindest können Sie mit den Experten Kontakt aufnehmen und Ihr Problem diskutieren.
Es gibt einen einfachen Test für fortgeschrittene Fälle des Karpaltunnelsyndroms.
Stellen Sie sich vor einen Spiegel, und lassen Sie die Arme völlig entspannt in ihrer \enquote{natürlichen} Haltung gerade herunterhängen.
Wenn die Daumen dem Spiegel am nächsten sind, dann ist alles in Ordnung.
Wenn man mehr Knöchel sieht (die Arme sind einwärts gedreht), dann haben Sie ein fortgeschritteneres Karpaltunnelsyndrom.
Auch sollte der Körper aufrecht sein.
Praktisch niemand hat eine perfekt aufrechte Haltung, und es kann notwendig sein, eine eventuelle ungenügende Haltung zu korrigieren, um das Karpaltunnelsyndrom vollständig zu behandeln.
Athleten wie Golfer und Tennisspieler sind eine Ausnahme, weil ihre asymmetrischen Spielbewegungen zu asymmetrischen Veränderungen der Knochendichte und Knochenstruktur führen.
Rechtshändige Golfer haben in ihrer rechten Hüfte eine höhere Knochendichte.

Methoden zur Streßreduzierung beim Klavierüben, wie z.B. die Methoden  von Taubman, Alexander und Feldenkrais können sowohl für die  Vermeidung von Verletzungen als auch für die Erholung von Verletzungen wirksam sein.
Im allgemeinen ist es das beste, den spielenden Finger (außer den Daumen) soviel wie möglich in einer Linie mit dem Unterarm zu halten, um Verletzungen durch wiederholten Streß zu vermeiden.
Natürlich ist die beste Vorbeugungsmaßnahme, nicht zuviel mit  Streß zu üben.
Die HS-Methode ist besonders nützlich, weil der Streß minimiert wird  und jede Hand zur Ruhe kommt, bevor ein Schaden auftreten kann.
Der Ansatz \enquote{ohne Schmerzen kein Erfolg} ist extrem schädlich.
Klavierspielen kann eine enorme Anstrengung erfordern, aber es darf niemals  schmerzhaft sein.
Sehen Sie dazu im \hyperref[Websites]{Quellenverzeichnis}  einige informative Websites über Handverletzungen bei Klavierspielern.


\subsubsection{Gehörschäden}
\label{c1iii10gehoer}

\textbf{Gehörschäden treten im allgemeinen altersbedingt auf; der Gehörverlust kann bereits im Alter von 40 Jahren beginnen, und mit 70 haben die meisten Menschen etwas von ihrer Hörfähigkeit verloren.}
Gehörverlust kann entstehen, wenn man hohen Lautstärken zu häufig ausgesetzt ist oder durch Infektionen und andere pathologische Ursachen.
Man verliert das Gehör meistens zuerst im unteren oder im oberen Frequenzbereich.
Das wird oft von einem Tinnitus (Pfeifen oder Klingeln im Ohr) begleitet.
Diejenigen, die das Gehör bei den niedrigen Frequenzen verlieren, neigen zu einem tiefen, tosenden oder pochenden Tinnitus, und diejenigen, die das Gehör bei den hohen Frequenzen verlieren, neigen dazu, ein hochtönendes Pfeifen zu hören.
Tinnitus kann durch ein ungewolltes Feuern der Hörnerven im beschädigten Abschnitt des Ohres verursacht werden; es gibt jedoch viele weitere Ursachen.
Im \hyperref[Websites]{Quellenverzeichnis} finden Sie Internet-Adressen zum Thema Gehörschäden.

Obwohl schwere Fälle von Gehörverlust von einem HNO-Arzt oder Hörgeräteakustiker leicht diagnostiziert werden können, sind die Ursachen und die Möglichkeiten zur Verhütung von Schäden noch nicht völlig bekannt.
Leichte Fälle von Gehörverlust sind auch für Fachleute schwer zu diagnostizieren, weil das menschliche Gehirn versucht, diese Verluste auszugleichen, indem die internen Mechanismen zur Tonverstärkung heraufgesetzt werden.
Diejenigen mit leichtem Gehörverlust haben z.B. Schwierigkeiten, Unterhaltungen zu verstehen, sind aber gegenüber lauten Geräuschen sehr empfindlich - sogar etwas laute Geräusche, die andere Menschen nicht stören, können schmerzhaft laut sein - einfache Hörtests würden zeigen, daß diese Menschen ein empfindliches Gehör haben.
Es gibt keine Methode, einen Tinnitus zu diagnostizieren, außer anhand der Beschreibungen des Patienten.
Für die Tests und die Behandlung muß man sich an einen HNO-Spezialisten wenden.
In nicht durch Krankheit bedingten Fällen werden Schäden im  allgemeinen dadurch verursacht, daß jemand lauten  Geräuschen ausgesetzt ist.
Trotzdem leiden viele Menschen, die sehr lauten Geräuschen ausgesetzt sind, wie z.B. Pianisten, die täglich mehrere Stunden auf Konzertflügeln spielen, Klavierstimmer, die routinemäßig während des Stimmens auf das Klavier \enquote{einhämmern} oder Mitglieder von Rockbands, nicht unter Gehörverlust.
Auf der anderen Seite können einige, die weniger Geräuschen ausgesetzt sind, ihr Gehör verlieren, besonders im Alter.
Deshalb gibt es große Unterschiede in der Anfälligkeit für Gehörverlust.
Es besteht jedoch eine Tendenz, daß Menschen, die lauteren Geräuschen ausgesetzt sind, mehr unter Gehörverlusten leiden.
\textbf{Es ist ziemlich wahrscheinlich, daß Gehörverluste bei Pianisten und Klavierstimmern (sowie bei Mitgliedern von Rockbands usw. und Menschen, die ständig sehr laute Musik hören) viel verbreiteter sind als allgemein bekannt ist, weil die meisten Fälle nicht dokumentiert werden.}

Ein Tinnitus ist im Grunde bei allen Menschen die ganze Zeit vorhanden, ist aber bei den meisten Menschen so leise, daß sie ihn, außer in schalldichten Räumen, nicht hören können.
Er kann durch ein spontanes Feuern der Hörnerven bei Abwesenheit eines genügend großen Reizes ausgelöst werden, d.h. der menschliche Hörapparat \enquote{dreht automatisch die Verstärkung auf}, wenn es kein Geräusch gibt.
Vollständig zerstörte Regionen erzeugen keinen Laut, weil der Schaden so ernsthaft ist, daß sie nicht mehr funktionieren.
Teilweise zerstörte Regionen erzeugen offenbar einen Tinnitus, weil sie geschädigt genug sind, daß sie fast kein Umgebungsgeräusch mehr wahrnehmen; diese Stille führt dazu, daß das Gehirn die Verstärkung aufdreht und die Detektoren abfeuert, oder das System entwickelt eine Fehlerstelle in der Weiterleitung des Schallsignals.
Diese Detektoren sind entweder piezo-elektrisches Material an der Basis von Haaren in der Gehörschnecke (Cochlea) oder Ionenkanäle, die durch an diesen Haaren befindliche Moleküle geöffnet und geschlossen werden - zu diesem Thema gibt es in der Literatur widersprüchliche Angaben.
Es gibt selbstverständlich viele weitere Ursachen von Tinnitus, und einige könnten ihren Ursprung sogar im Gehirn haben.
Ein Tinnitus ist fast immer ein Zeichen eines einsetzenden Gehörverlusts.

Für diejenigen, die keinen hörbaren Tinnitus haben, besteht wahrscheinlich - innerhalb vernünftiger Grenzen - keine Notwendigkeit, laute Musik zu meiden.
Deshalb sollte das Klavierüben bis zu einem Alter von 25 Jahren mit jeder Lautstärke unschädlich sein.
Diejenigen, die bereits einen Tinnitus haben, sollten laute Klaviermusik meiden.
\textbf{Tinnitus \enquote{beschleicht} einen jedoch üblicherweise, so daß das Einsetzen des Tinnitus oftmals unentdeckt bleibt, bis es zu spät ist.
Deshalb sollte jeder über Tinnitus Bescheid wissen und ab 40 während des Klavierübens einen Gehörschutz tragen.
Gehörschutz ist für die meisten Klavierspieler eine abscheuliche Vorstellung, aber wenn man die Konsequenzen bedenkt (s.u.), lohnt es sich absolut.}
Bevor Sie einen Gehörschutz tragen, tun Sie alles, um die Lautstärke zu verringern, wie den Raum schallarm machen (Teppiche auf harten Böden auslegen usw.), den Deckel eines Flügels schließen und im allgemeinen leise üben (auch laute Passagen - was sowieso eine gute Idee ist, auch ohne die Möglichkeit eines Gehörschadens).

Einen Gehörschutz kann man leicht in Baumärkten kaufen, weil viele Arbeiter, die Baumaschinen oder Gartengeräte benutzen, einen Gehörschutz benötigen.
Für Klavierspieler reicht ein preisgünstiger Schutz, weil sie noch etwas von der Musik hören müssen.
Sie können auch die meisten größeren Audio-Kopfhörer benutzen.
Kommerzielle Schützer umschließen das Ohr völlig und isolieren den Schall besser.
Da die heute verfügbaren Schützer nicht für Klavierspieler entwickelt wurden, haben sie keinen gleichmäßigen Frequenzgang; d.h., der Klang des Klaviers wird verändert.
Das menschliche Ohr kann sich jedoch gut an verschiedene Arten von Klängen anpassen und wird sich sehr schnell an den neuen Klang gewöhnen.
Das Klavier wird auch ziemlich anders klingen, wenn Sie den Gehörschutz entfernen (was Sie hin und wieder tun müssen, damit Sie wissen, wie der wahre Klang ist).
Diese verschiedenen Klänge können uns lehren, wie das Gehirn Einfluß darauf nimmt, welche Klänge man hört und welche nicht und wie unterschiedlich verschiedene Menschen dieselben Klänge interpretieren.
\textbf{Es lohnt sich, einen Gehörschutz auszuprobieren, so daß man diese verschiedenen Klänge erfahren kann.
Sie werden z.B. feststellen, daß das Klavier viele fremdartige Klänge erzeugt, die Sie nie zuvor gehört haben!}
Die Unterschiede im Klang sind so erstaunlich und komplex, daß man sie nicht in Worte fassen kann.
Bei Klavieren geringerer Qualität führt das Benutzen eines Gehörschutzes dazu, daß der Klang eines höherwertigen Instruments simuliert wird, weil die unerwünschten hohen Obertöne und zusätzliche Geräusche herausgefiltert werden.

Das Gehirn verarbeitet automatisch alle eingehenden Informationen, ob Sie es wollen oder nicht.
Das ist natürlich ein Teil dessen, was Musik ist - sie ist die Interpretation der hereinkommenden Klänge durch das Gehirn, und der größte Teil unserer Reaktion auf die Musik geschieht automatisch.
Wenn man einen Gehörschutz trägt, verschwindet deshalb das meiste dieses Reizes, und ein großer Anteil der Verarbeitungskapazität des Gehirns wird für andere Aufgaben frei.
Insbesondere haben Sie nun mehr Mittel zur Verfügung, die Sie für das HS-Üben verwenden können.
Schließlich üben Sie deshalb HS und nicht HT - so daß Sie mehr Energie auf die schwierige Aufgabe, mit dieser einen Hand zu spielen, verwenden können.
Deshalb werden Sie eventuell feststellen, daß Sie HS schnellere Fortschritte machen, wenn Sie einen Gehörschutz tragen!
Aus dem gleichen Grund schließen viele Pianisten ihre Augen, wenn sie etwas mit einem hohen emotionalen Gehalt spielen möchten - sie brauchen alle verfügbaren Kräfte, um das hohe Maß an Emotion zu erzeugen.
Wenn man die Augen schließt, eliminiert man eine enorme Menge an Informationen, die in das Gehirn strömt, weil das Sehen eine zweidimensionale, vielfarbige, bewegte Quelle eines Datenstroms mit hoher Bandbreite ist, der sofort und automatisch auf viele komplexe Arten interpretiert werden muß.
Obwohl das Publikum meistens bewundert, daß ein Pianist mit geschlossenen Augen spielen kann, ist es in Wahrheit einfacher.
\textbf{Darum werden in naher Zukunft wahrscheinlich die meisten Klavierschüler einen Gehörschutz tragen, so wie heutzutage viele Athleten und Bauarbeiter einen Helm tragen.}
Es macht für niemanden von uns einen Sinn, die letzten 10, 30 oder mehr Jahre unseres Lebens ohne Gehör zu verbringen.

Wie schädigt der Klavierklang das Gehör?
Sicherlich sind laute Töne mit vielen Noten am schädlichsten.
Deshalb ist es wahrscheinlich kein Zufall, daß Beethoven vorzeitig taub wurde.
Das ermahnt uns auch, beim Üben seiner Musik an den Schutz des Gehörs zu denken.
Der Typ des Klaviers ist auch wichtig.
\textbf{Die meisten \enquote{\hyperref[upright]{Aufrechten}}, die keinen ausreichenden Klang erzeugen, sind wahrscheinlich am wenigsten schädlich.
Große Flügel, die die Energie effizient auf die Saiten übertragen und den Ton lange aushalten, verursachen wahrscheinlich nicht so große Schäden wie Klaviere mittlerer Qualität, bei denen im Moment des ersten Schlags beim Auftreffen des Hammers auf den Saiten eine große Energiemenge übertragen wird.}
Obwohl ein großer Teil dieser schädigenden Tonenergie wahrscheinlich nicht im hörbaren Bereich liegt, können wir sie als unangenehmen oder schrillen Klang erkennen.
Deshalb sind Flügel mittlerer Größe (6 bis 7 ft; ca. 1,80 bis 2,10 m) eventuell am schädlichsten.
In dieser Hinsicht ist der Zustand der Hämmer wichtig, da ein abgenutzter Hammer einen viel lauteren Anschlagsklang als ein neuer oder richtig intonierter Hammer erzeugen kann.
Deshalb verursachen abgenutzte Hämmer öfter einen Saitenbruch als neue bzw. gut intonierte Hämmer.
Mit alten, verhärteten Hämmern können wahrscheinlich die meisten Klaviere das Gehör schädigen.
Deshalb ist das richtige, regelmäßige \hyperref[c2_7_hamm]{Intonieren der Hämmer} für das musikalische Spielen, die technische Entwicklung und den Schutz des Gehörs viel wichtiger als vielen Menschen bewußt ist.
Wenn Sie den Deckel eines Flügels schließen müssen, um leise zu spielen oder den Ton auf ein erträgliches Maß zu reduzieren, dann müssen die Hämmer wahrscheinlich intoniert werden.

Einige der lautesten Geräusche werden von jenen kleinen Ohrhörern erzeugt, die man zum Musikhören benutzt.
Eltern sollten ihre Kinder davor warnen, die Lautstärke ständig weit aufzudrehen, besonders wenn sie Fans von Musik sind, die sehr laut gespielt wird.
Einige Kinder schlafen mit lärmenden Ohrhörern ein; das kann sehr schädlich sein, weil die Zeit, die man der Lautstärke ausgesetzt ist, ebenfalls wichtig ist.
\textbf{Es ist eine schlechte Idee, Kindern Geräte mit solchen Ohrhörern zu geben - zögern Sie es so lange wie möglich hinaus.}
Früher oder später bekommen sie jedoch eins; in diesem Fall sollten Sie sie warnen, \textit{bevor} sie Gehörschäden bekommen.

Außer in einigen besonderen Fällen von Tinnitus (besonders jene, in denen man den Klang durch Bewegen des Kiefers usw. verändern kann), gibt es bisher kein Heilverfahren.
Große Dosen von Aspirin können Tinnitus verursachen; in diesem Fall kann die Einnahme zu beenden den Prozeß manchmal umkehren.
Kleine Mengen von Aspirin, die wegen Problemen mit dem Herzen genommen werden (81mg), verursachen offensichtlich keinen Tinnitus, und es wird in der Literatur manchmal behauptet, daß diese kleinen Mengen vielleicht das Einsetzen des Tinnitus verzögern.
Ein lauter Tinnitus kann sehr anstrengend sein, weil er nicht verändert werden kann, ständig vorhanden ist und mit der Zeit immer schlimmer wird.
Viele, die darunter leiden, haben schon an Selbstmord gedacht.
Obwohl es keine Heilung gibt, ist eine Abhilfe möglich, und alles deutet darauf hin, daß man irgendwann in der Lage sein sollte, Möglichkeiten zur Heilung zu finden.
Es gibt Hörhilfen, die die Wahrnehmung des Tinnitus reduzieren, z.B. indem sie ein Geräusch abgeben, so daß der Tinnitus entweder maskiert oder die Verstärkung im geschädigten Bereich reduziert wird.
Deshalb kann für diejenigen, die unter Tinnitus leiden, absolute Stille schädlich sein.

Eine der ärgerlichsten Eigenschaften des Gehörverlusts ist nicht, daß das Gehör seine Empfindlichkeit verloren hätte (oft zeigen Tests der Empfindlichkeit nur sehr geringe Verluste), sondern die Unfähigkeit der Person, die Geräusche richtig zu verarbeiten, so daß man ein Gespräch verstehen kann.
Menschen mit normalem Gehör können Sprache verstehen, die mit vielen zusätzlichen Geräuschen vermischt ist.
\textbf{Sprache zu verstehen ist im allgemeinen die erste Fähigkeit, die man mit dem Einsetzen des Gehörverlusts verliert.}
Moderne Hörgeräte können sehr hilfreich sein, indem sie sowohl die Frequenzen verstärken, die notwendig sind, um Sprache zu verstehen, als auch Geräusche unterdrücken, die laut genug sind, um Schäden zu verursachen.
Mit anderen Worten: Wenn das Hörgerät lediglich alle Geräusche verstärkt, kann es sogar einen größeren Schaden verursachen.
Ein weiteres Vorgehen gegen den Tinnitus ist, das Gehirn darauf zu trainieren, den Tinnitus zu ignorieren.
Das Gehirn kann erstaunlich gut trainiert werden, und ein Teil des Grunds, warum Tinnitus Leid verursachen kann, ist eine unangemessene Reaktion des Gehirns.
\textbf{Das Gehirn hat die Fähigkeit, sich entweder auf das Geräusch zu konzentrieren und Sie verrückt zu machen, oder das Geräusch zu ignorieren, so daß Sie es nicht hören, solange Sie nicht daran erinnert werden.}
Das beste Beispiel für diesen Effekt ist das Metronom.
Die meisten Klavierspieler wissen nicht, daß ihnen das Gehirn, wenn sie zu lange mit dem Metronom üben, einen Streich spielt und man das Klicken entweder überhaupt nicht mehr hört oder zur falschen Zeit, besonders wenn das Klicken hoch und laut ist.
Das ist ein Grund, warum moderne Metronome blinkende Lichter haben; es versetzt Sie nicht nur in die Lage, ohne Ton den richtigen Rhythmus zu halten, sondern Sie können auch prüfen, ob das, was Sie hören, mit den blinkenden Lichtern übereinstimmt.
Deshalb beginnen moderne Behandlungen des Tinnitus damit, dem Patienten beizubringen, daß andere bereits gut und mit minimalen Beschwerden damit zurechtkommen.
Danach erhält der Patient ein Gehörtraining, so daß er in der Lage ist, den Tinnitus zu ignorieren.
Zum Glück kann das Gehirn besonders leicht lernen, ein konstantes Geräusch zu ignorieren, das immer vorhanden ist.

Wenn Sie genug Berichte über das Leiden an Tinnitus gelesen haben, werden Sie wahrscheinlich dem Rat folgen, ab 40 einen Gehörschutz zu tragen, zumindest wenn Sie längere Zeit laute Passagen üben.
Bei den ersten Anzeichen von Tinnitus ist es dringend erforderlich, daß Sie etwas für den Schutz Ihres Gehörs tun, denn wenn der Tinnitus eingesetzt hat, kann eine Verschlechterung des Gehörs, wenn es lauten Geräuschen ausgesetzt ist, schnell mit einer hohen jährlichen Rate fortschreiten.
Ein Digital-Piano zu benutzen und die Lautstärke herunterzudrehen ist eine sehr gute Lösung.
Gehen Sie sofort zu einem HNO-Arzt, der möglichst auf die Behandlung von Tinnitus spezialisiert ist.
Der Schutz des Gehörs ist auch für die anderen Mitglieder des Haushalts wichtig; wenn es möglich ist, isolieren Sie deshalb den Raum, in dem das Klavier steht, akustisch vom Rest des Hauses.
Die meisten Qualitätstüren (aus Glas) werden genügen.
Es gibt ein paar Kräuter und \enquote{natürliche} Medikamente, die eine Wirkung gegen den Tinnitus versprechen.
Die meisten davon wirken nicht, und jene, die scheinbar einigen Menschen nützen, haben gefährliche Nebenwirkungen.
Obwohl es wahr ist, daß es herzlich wenig Spezialisten gibt, die Tinnitus behandeln, verbessert sich die Situation rasch, und es gibt nun viele Seiten im Internet mit Informationen zum Tinnitus, wie z.B. von der American Tinnitus Association.



<!-- c1iii11.html -->

\subsection{Blattspiel}
\label{c1iii11}

\textbf{Es ist nützlich, das Spielen vom Blatt in drei Stufen zu unterteilen, damit wir wissen, worüber wir sprechen}, weil der Begriff \enquote{Blattspiel} für verschiedene Vorgänge benutzt wird.
Auf der Anfängerstufe bedeutet Blattspiel, Kompositionen zu spielen, die wir nicht auswendig gelernt haben, und die wir spielen, während wir auf das Notenblatt sehen.
Wir sind vielleicht schon mit den Melodien der Komposition vertraut und haben sie bereits gespielt.
In der Mittelstufe können wir Musik vom Blatt spielen, die wir noch nicht kennen und die wir noch nicht geübt haben.
Diese Stufe wird im allgemeinen als das richtige Blattspiel angesehen und ist das Thema dieses Abschnitts.\footnote{Siehe auch \hyperref[c030530]{Prima-Vista-Spiel} im Quellenverzeichnis.}
Auf der fortgeschrittenen Stufe schließt das Blattspiel die Anwendung der grundlegenden Musiktheorie ein, wie z.B. Akkordprogressionen und Harmonien, sowie die Interpretation der Musik.
Es folgen die Grundregeln des Blattspiels (siehe auch \hyperref[Richman]{Richman}):

\begin{enumerate}[label={\arabic*.}] 
\item \textbf{Blicken Sie stets auf die Noten; schauen Sie nicht auf die Tastatur oder Finger.}
Schauen Sie gelegentlich auf die Hände, wenn es für große \hyperref[c1iii7f]{Sprünge} notwendig ist.
Versuchen Sie, einen \enquote{Randblick} auf die Tastatur entwickeln, so daß Sie eine ungefähre Vorstellung davon haben, wo die Hände sind, während Sie immer noch auf die Noten schauen.
Mit dem Randblick können Sie beide Hände gleichzeitig im Auge behalten.
\textbf{Gewöhnen Sie sich an, die Tasten vor dem Spielen zu erfühlen.}
Obwohl diese Regel unabhängig davon anwendbar ist, ob man vom Blatt spielt oder nicht, wird sie beim Blattspiel entscheidend.
Die Tasten vor dem Spielen zu erfühlen hilft auch, bei Sprüngen \enquote{vorzeitig in der richtigen Position zu sein}, s. Abschnitte \hyperref[c1iii7e]{7e} und \hyperref[c1iii7f]{7f} oben; deshalb sollten Sie das Üben der Sprungbewegungen mit dem Üben des Blattspiels verbinden.


\item \textbf{Spielen Sie durch Fehler hindurch, und machen Sie sie so unhörbar wie möglich.}
Am besten lassen Sie die Musik so klingen, als ob Sie etwas geändert hätten - dann weiß das Publikum nicht, ob Sie einen Fehler gemacht oder es geändert haben.
Deshalb haben Schüler mit einer Grundausbildung in Musiktheorie solch einen Vorteil beim Blattspiel.
Drei Möglichkeiten, Fehler weniger hörbar zu machen, sind:

\begin{enumerate}[label={\roman*.}] 
<li>den Rhythmus intakt halten
\item eine fortlaufende Melodie beibehalten (falls Sie nicht alles lesen können, behalten Sie die Melodie bei, und lassen Sie die Begleitung weg)
\item üben, die Teile zu vereinfachen, die zum Ablesen zu schwierig sind
 \end{enumerate}
Als erstes müssen Sie die Angewohnheit loswerden, bei jedem Fehler anzuhalten und zurückzugehen (Stottern), falls Sie diese Angewohnheiten bereits haben.
\textbf{Die beste Zeit, die Fertigkeit zu entwickeln, nicht bei jedem Fehler anzuhalten, ist, wenn Sie Ihre ersten Unterrichtsstunden beginnen.}
Wenn sich die Angewohnheit zu stottern erst einmal verfestigt hat, erfordert es viel Arbeit, sie zu eliminieren.
Für diejenigen mit der Angewohnheit zu stottern ist es das beste, sich dafür zu entscheiden, nie wieder zurückzugehen (egal, ob es gelingt oder nicht) - sie wird langsam anfangen zu verschwinden.
Spielfehler voraussehen zu lernen ist eine große Hilfe und wird weiter unten besprochen.
Das wirksamste Werkzeug ist die Fähigkeit, die Musik zu vereinfachen.
Eliminieren Sie Verzierungen, fischen Sie die Melodie aus schnellen Läufen, usw.

</li>
\item \textbf{Lernen Sie alle allgemeinen musikalischen Konstrukte: Alberti-Begleitungen, Dur- und Moll-\hyperref[c1iii5]{Tonleitern} und ihre \hyperref[table]{Fingersätze} genauso wie die zugehörigen \hyperref[Arpeggios]{Arpeggios}, einfache Akkorde und Akkordumstellungen, Triller, Verzierungen usw.}
Beim Spielen vom Blatt sollten Sie die Konstrukte erkennen und nicht die einzelnen Noten lesen.
Lernen Sie die Positionen der sehr hohen und sehr tiefen Noten auf dem Notenblatt auswendig, so daß Sie sie sofort erkennen können.
Lernen Sie zunächst die Oktav-Cs, und fügen Sie dann die anderen Noten hinzu, beginnend mit den Noten, die den Cs am nächsten sind.


\item \textbf{Schauen Sie dem voraus, was Sie spielen; mindestens einen Takt voraus oder auch mehr, wenn Sie die Fertigkeit entwickeln, die Musikstruktur zu lesen.}
Versuchen Sie, eine Struktur voraus zu lesen.
Durch das Vorausschauen können Sie sich nicht nur vorzeitig vorbereiten, sondern auch Spielfehler voraussehen, bevor sie auftreten.
Sie können auch Probleme mit dem Fingersatz vorhersehen und können vermeiden, sich selbst in unmögliche Situationen zu bringen.
Obwohl Vorschläge zum Fingersatz in den Noten im allgemeinen hilfreich und vielleicht die besten Fingersätze sind, sind sie oft nutzlos, weil Sie sie eventuell nicht ohne Übung benutzen können.
Deshalb sollten Sie Ihren eigenen Vorrat an Fingersätzen entwickeln.


\item \enquote{Üben, üben, üben}.
\textbf{Obwohl das Blattspiel relativ leicht zu lernen ist, muß es jeden Tag geübt werden, um es zu verbessern.
Bei den meisten Schülern erfordert es ein bis zwei Jahre} eifriges Üben, um gut darin zu werden.
Da das Spielen vom Blatt so stark vom Erkennen von Strukturen abhängt, ist es nah mit dem \hyperref[c1iii6]{Auswendiglernen} verwandt.
Das bedeutet, daß Sie die Fähigkeit zum Blattspiel verlieren können, wenn Sie aufhören zu üben.
Wie beim Auswendiglernen bleibt Ihnen die Fähigkeit jedoch ein Leben lang erhalten, wenn Sie bereits in jungen Jahren ein guter Blattspieler geworden sind.
Versuchen Sie nach dem Üben des Blattspiels, einige Strukturen des Stücks, die häufig vorkommen, in Gedanken zu spielen (siehe \hyperref[c1iii6j]{Abschnitt III.6j}).


\end{enumerate}
Sie sollten stets weitere \enquote{Verkaufstricks} hinzufügen, wenn Sie besser werden.
Üben Sie die Kunst, eine Komposition zu überfliegen, bevor Sie sie vom Blatt spielen, um ein Gefühl dafür zu bekommen, wie schwierig sie ist.
Dann können Sie bereits vorzeitig herausfinden, wie Sie um die \enquote{unmöglichen} Abschnitte herumkommen.
Sie können sie sogar schnell üben, indem Sie eine verdichtete Version der Lerntricks benutzen (HS, schwierige Abschnitte verkürzen, parallele Sets benutzen usw.) und zwar gerade so viel, um es passabel klingen zu lassen.
Ich habe Blattspieler getroffen, die sich eine Weile mit mir über einige Abschnitte eines neuen Stücks unterhielten und es dann ohne Schwierigkeiten durchspielten.
Ich habe später herausgefunden, daß sie diese Abschnitte in den paar Sekunden geübt hatten, in denen sie mich mit ihrer \enquote{Besprechung} abgelenkt hatten.

Nehmen Sie ein paar Bücher mit leichten Stücken.
Da es am Anfang leichter ist, das Blattspiel mit bekannten Stücken zu üben, können Sie die gleiche Komposition mehrere Male benutzen, um das Blattspiel zu üben, z.B. erneut nach einer Woche oder mehr.
Bücher mit \enquote{Sonatinas}, Mozarts leichtere Sonaten und Bücher für Anfänger mit leichten beliebten Liedern sind für das Üben gut geeignet.
Als leichteste Stücke könnten Sie Beyer benutzen, die Bücher für Anfänger in Abschnitt III.18c oder die leichtesten Stücke von Bach. 
Obwohl Sie mit bereits bekannten Stücken eine ganze Menge Fertigkeiten zum Spielen vom Blatt erlernen können, sollten Sie jedoch auch mit Stücken üben, die Sie nie zuvor gesehen haben, um die wahre Fertigkeit des Blattspiels zu entwickeln.
Die nützlichste Fertigkeit, die eine Hilfe beim wahren Blattspiel sein kann, ist das Singen vom Blatt, mit dem wir uns nun beschäftigen.


\subsection{Absolutes Gehör und relatives Gehör (vom Blatt singen)}
\label{c1iii12}

\textbf{Relatives Gehör ist die Fähigkeit, eine Note in bezug auf eine Vergleichsnote zu erkennen.
Absolutes Gehör ist die Fähigkeit, eine Note ohne eine vorgegebene Vergleichsnote zu erkennen.}
Die Qualität Ihres absoluten Gehörs wird dadurch bestimmt, wie genau Sie eine Tonhöhe wiedergeben können, wie schnell Sie eine Note erkennen können und wie viele Noten Sie erkennen können, wenn sie gleichzeitig gespielt werden.
Menschen mit einem guten absoluten Gehör können sofort (innerhalb von 3 bis 5 Sekunden) 10 gleichzeitig gespielte Noten erkennen.
Beim Standardtest für das absolute Gehör werden zwei Klaviere benutzt: Der Tester sitzt an dem einem und der Schüler an dem anderen Klavier, und der Schüler versucht, die vom Tester gespielten Noten zu wiederholen.
Wenn nur ein Klavier benutzt wird, benennt der Schüler die vom Tester gespielte Note (do, re, mi usw. oder C, D, E usw.).
Benutzen Sie in den folgenden Übungen zunächst CDE, weil die meisten Lehrbücher diese Namensgebung verwenden.
Es ist jedoch in Ordnung, wenn Sie do-re-mi benutzen, weil das für Sie besser funktioniert.
\textbf{Niemand wird mit einem relativen oder absoluten Gehör geboren; beides sind erlernte Fertigkeiten}, weil die chromatische Tonleiter eine menschliche Erfindung ist - es gibt keine physische Beziehung zwischen den Tonhöhen der chromatischen Tonleiter und der Natur.
Die einzige physische Verbindung zwischen der chromatischen Tonleiter und dem Ohr ist, daß beide mit einer logarithmischen Skala arbeiten, um einen großen Frequenzbereich abzudecken.
Wir wissen, daß das Ohr mit einer logarithmischen Skala arbeitet, weil Harmonien eine besondere Bedeutung haben, Harmonien Verhältnisse sind und Verhältnisse am leichtesten auf einer logarithmischen Skala zu handhaben sind.
Deshalb ist uns das Erkennen von Harmonien angeboren, obwohl uns ein absolutes Gehör nicht angeboren ist.
Der Effekt des logarithmischen menschlichen Hörens ist, daß das Ohr einen großen Unterschied in der Tonhöhe zwischen 40 und 42,4 Hz hört (ein Halbton oder 100 Cent) aber fast keinen Unterschied zwischen 2000 Hz und 2002,4 Hz (ungefähr 2 Cent) hört, also den gleichen Unterschied von 2,4 Hz.
Das menschliche Ohr reagiert auf alle Frequenzen innerhalb seines Bereichs und ist nicht von Geburt an auf eine absolute Skala geeicht.
Hierin unterscheidet es sich vom Auge, das auf Farbe auf einer absoluten Skala reagiert (jeder sieht ab seiner Geburt ohne Übung Rot als Rot, und diese Wahrnehmung ändert sich mit dem Alter nicht), weil das Erkennen der Farben durch chemische Reaktionen erreicht wird, die auf spezifische Wellenlängen des Lichts reagieren.
Einige Menschen, die bestimmte Tonhöhen mit bestimmten Farben verbinden, können ein absolutes Gehör durch die Farben erwerben, die von den Tönen hervorgerufen werden.
Sie eichen das Ohr auf einen absoluten Bezugswert.
Die meisten Schüler lernen das absolute Gehör durch das assoziative Gedächtnis.

\textbf{Absolutes und relatives Gehör lernt man in frühester Jugend am besten.}
Babys, die kein einziges Wort verstehen können, reagieren entsprechend auf eine beruhigende Stimme, ein Schlaflied oder ein Angst einflößendes Geräusch, was ihre Bereitschaft zu einer musikalischen Ausbildung zeigt.
\textbf{Die beste Möglichkeit für Kleinkinder, ein absolutes Gehör zu erwerben, ist, von Geburt an fast täglich einem gut gestimmten Klavier zuzuhören.}
Deshalb sollten alle Eltern, die ein Klavier besitzen, es gestimmt halten und in Gegenwart des Babys darauf spielen.
Dann sollten sie ab und zu prüfen, ob das Kind ein absolutes Gehör hat.
Dieser Test kann ausgeführt werden, indem man eine Note spielt (wenn das Kind nicht hinsieht) und es dann bittet, die Note auf dem Klavier zu finden.
Natürlich müssen Sie dem Kind zuerst den Tonumfang des Klavier erklären: Fangen Sie mit der C-Dur-Tonleiter in der Mitte des Klaviers an, und erklären Sie danach die Tatsache, daß alle anderen Noten mit dieser Tonleiter im Oktavabstand verbunden sind.
Wenn das Kind die Note nach ein paar Versuchen finden kann, hat es ein relatives Gehör; wenn es die Note jedesmal sofort findet, hat es ein absolutes Gehör.
Die jeweilige Stimmung des Klaviers (\hyperref[c2_6_et]{gleichschwebend}, \hyperref[c2_2_wtk2]{wohltemperiert} usw.) ist nicht wichtig; in der Tat wissen die meisten Menschen mit absolutem Gehör nichts über Stimmungen, und wenn Noten auf Klavieren mit unterschiedlichen Stimmungen gespielt werden, haben sie keine Schwierigkeiten, die Noten zu erkennen, weil sich die meisten Frequenzen durch die verschiedenen Stimmungen um weniger als 5\% ändern und niemand ein absolutes Gehör mit einer solchen Genauigkeit hat.
Absolutes und relatives Gehör können später im Leben erworben werden, aber es wird in einem höheren Alter als 20 bis 30 schwieriger.
Tatsächlich \textbf{fangen diejenigen mit absolutem Gehör ungefähr im Alter von 20 Jahren an, es langsam wieder zu verlieren, wenn es nicht gepflegt wird.}
Viele Klavierschulen lehren allen Schülern routinemäßig ein absolutes Gehör mit einer Erfolgsquote von über 90\%.
Das Problem beim Unterrichten einer Gruppe älterer Schülern ist, daß es meistens einen gewissen Prozentsatz \enquote{tonhöhenbehinderte} Schüler gibt, die nie auf Tonhöhen trainiert wurden und deshalb sogar eventuell Schwierigkeiten haben, ein relatives Gehör zu erlernen.
Eine Anleitung, wie man sehr jungen Kindern ein absolutes Gehör lehrt, finden Sie in \hyperref[c1iii16c]{Abschnitt 16c}, weil es ziemlich einfach ist und ein integraler Bestandteil des Unterrichts für sehr junge Kinder; eine Anleitung für Erwachsene folgt weiter unten in diesem Abschnitt.

\textbf{Ein absolutes Gehör zu haben, ist sicherlich von Vorteil.}
Es ist eine große Hilfe für das \hyperref[c1iii6]{Auswendiglernen}, \hyperref[c1iii11]{Blattspiel}, beim Überwinden von Erinnerungsblockaden und für das Komponieren.
Sie können eine Stimmpfeife für Ihren Chor sein und leicht eine Geige oder ein Blasinstrument stimmen.
Es macht viel Spaß, weil Sie sagen können, wie schnell ein Auto fährt, wenn Sie das Singen der Reifen hören.
Sie können den Unterschied zwischen verschiedenen Autohupen und Pfeifen von Lokomotiven insbesondere dadurch erkennen, ob sie Terzen oder Quinten benutzen.
Sie können sich Telefonnummern leicht anhand ihrer Töne merken.
Es gibt jedoch Nachteile.
Musik, die außerhalb der Stimmung gespielt wird, kann unangenehm sein.
Da so viel Musik außerhalb der Stimmung gespielt wird, kann das ein ziemliches Problem darstellen.
Manchmal können Menschen heftig auf solch eine Musik reagieren; körperliche Reaktionen wie z.B. tränende Augen oder feuchte Haut können auftreten.
Transponierte Musik ist in Ordnung, weil jede Note immer noch korrekt ist.
Es wird schwierig, verstimmte Klaviere zu spielen.
\textbf{Ein absolutes Gehör ist ein zweischneidiges Schwert.}

\textbf{Es gibt eine Methode, die das Erlernen des absoluten und relativen Gehörs beschleunigt und vereinfacht!}
Diese Methode wird im allgemeinen nicht an Musikschulen oder in der Literatur gelehrt, obwohl sie von denjenigen mit einem absoluten Gehör seit den Anfängen der Musik benutzt wurde (meistens ohne zu wissen, wie sie es erworben haben).
Mit der hier beschriebenen Methode werden die Fähigkeiten zur Erkennung von Tonhöhen einfach zu Nebenprodukten des \hyperref[c1iii6]{Auswendiglernens}.
Es erfordert wenig zusätzlichen Aufwand, sich die Tonhöhenerkennung anzueignen, weil das Auswendiglernen ohnehin erforderlich ist, wie in Abschnitt III.6 erklärt wurde.
In diesem Abschnitt haben wir gesehen, daß das endgültige Ziel des Auswendiglernens die Fähigkeit ist, die Musik in Gedanken zu spielen (\hyperref[c1iii6tastatur]{mentales Spielen}).
Wenn man während des Übens des mentalen Spielens auf die relative und absolute Tonhöhen achtet, dann erwirbt man die Fertigkeiten zur Tonhöhenerkennung wie von selbst!
\textbf{Spielen Sie die Musik nicht nur in Gedanken, sondern spielen Sie sie auch mit der korrekten Tonhöhe.}
Das ist absolut sinnvoll, weil Sie, wenn Sie nicht mit der korrekten Tonhöhe spielen, viele der Vorteile des mentalen Spielens verschenken.
Umgekehrt wird das mentale Spielen nicht gut funktionieren, wenn es nicht mit der absoluten Tonhöhe ausgeführt wird, weil das mentale Spielen eine Funktion des Gedächtnisses, das Gedächtnis assoziativ und die absolute Tonhöhe eine der wichtigsten Assoziationen ist - die absolute Tonhöhe gibt der Musik ihre wahre Melodielinie, Farbe, Ausdruck usw.
Für die meisten genügt das Auswendiglernen von zwei bedeutsamen Kompositionen, um sich ein absolutes Gehör mit der Auflösung eines Halbtons anzueignen, was schneller ist als jede heute gelehrte Methode; bei den meisten sollte das ein paar Wochen oder ein paar Monate dauern.
Junge Kinder werden das ohne zusätzlichen Aufwand, fast automatisch, erreichen (s. \hyperref[c1iii16c]{Abschnitt 16c}); wenn man älter wird, wird man wegen all der verwirrenden Klänge, die bereits im Gedächtnis gespeichert sind, mehr Aufwand benötigen.

Zwei nützliche Kompositionen für das Üben des relativen und absoluten Gehörs sind Bachs Invention \#1 und der erste Satz von Beethovens Mondscheinsonate.
Durch Bach erhalten Sie das mittlere C (die erste Note des Stücks) und die C-Dur-Tonleiter; das sind die nützlichste Note und Tonleiter, die man mit den absoluten Tonhöhen lernen kann.
Die Mondscheinsonate hat bezaubernde Melodien, die das Auswendiglernen einfach und unterhaltsam machen.
Außerdem erzeugen die komplexen Modulationen der Tonarten eine Vielzahl von Noten und Intervallen, und die Komplexität verhindert, daß man die Noten erraten kann - es erfordert ein beträchtliches Maß an Übung und Wiederholungen, bevor man das Stück perfekt in Gedanken spielen kann.
Es ist auch für jeden technisch hinreichend einfach.
Beide Kompositionen sollten zunächst für das Üben der Tonhöhen HS geübt werden und erst später HT.

Wenn Sie sich die Noten vorstellen, versuchen Sie nicht, sie zu summen oder zu singen, weil der Tonumfang des Klaviers viel größer als der Ihrer Stimmbänder ist und Sie Ihr Gedächtnis darauf trainieren müssen, mit diesen höheren und tieferen Noten zurechtzukommen.
Auch muß das Abbild jeder Note im Gedächtnis zunächst alles einschließen - die Obertöne, das Timbre und andere Eigenschaften Ihres Klaviers.
Sie brauchen so viele Assoziationen wie möglich, um den Gedächtnisprozeß zu beschleunigen.
Benutzen Sie deshalb dasselbe Klavier, bis Sie das Gefühl haben, sie haben das absolute Gehör, und versuchen Sie sich jede Eigenschaft des Klangs Ihres Klaviers einzuprägen.
Wenn Sie kein elektronisches Klavier besitzen, sorgen Sie dafür, daß das Klavier gestimmt ist.
Wenn Sie ein sicheres absolutes Gehör erworben haben, wird es mit jeder Tonquelle funktionieren.
Solange Sie kein ausgebildeter Sänger sind, der mit der richtigen Tonhöhe singen kann (in diesem Fall müßten Sie keine Erkennung der absoluten Tonhöhen üben), werden Sie die Tonhöhen nicht genau singen können.
Der daraus resultierende falsche Klang wird das Gehirn verwirren und jegliche Fähigkeit zur Tonhöherkennung zerstören, die Sie vielleicht schon erworben haben.
So wie das Spielen in Gedanken den Klavierspieler von den Einschränkungen durch das Klavier befreit, befreit Sie das mentale Vorstellen der Tonhöhe (im Gegensatz zum Singen) von den Beschränkungen der Stimmbänder.


\paragraph{Verfahren zum Lernen der relativen und absoluten Tonhöhenerkennung}
\label{c1iii12tonhoehe}

Nachdem Sie das Stück von Bach vollständig auswendiggelernt haben und das ganze Stück vollständig in Gedanken spielen können, beginnen Sie damit, die relative Tonhöhenerkennung zu lernen.
Spielen Sie die erste Note (C4) auf dem Klavier, benutzen Sie sie als Referenz, um die ersten ein oder zwei Takte in Gedanken zu spielen, und prüfen Sie die letzte Note mit dem Klavier.
Die meisten Anfänger werden sich die Intervalle fast richtig vorstellen, weil das Gehirn automatisch versucht, Schritte in gesungenen Intervallen zu machen.
Aufsteigende Noten werden so zu tief und absteigende Noten zu hoch gesungen.
Beginnen Sie mit einem oder zwei Takten, korrigieren Sie alle Fehler, und wiederholen Sie es so lange, bis die Fehler verschwinden.
Fügen Sie dann mehr Takte hinzu, usw.
Wenn Sie das ganze Stück auf diese Art durchgearbeitet haben, sollte Ihre relative Tonhöhenerkennung ziemlich gut sein.
Fangen Sie dann mit der absoluten Tonhöhenerkennung an.
Spielen Sie die ersten paar Takte ohne Referenznote vom Klavier in Gedanken, und prüfen Sie anschließend, ob Ihre Vorstellung von C4 richtig ist.
Jeder hat seine höchste und tiefste Note, die er summen kann.
Summen Sie nun zunächst ohne das Klavier bis zur maximalen Note aufwärts und dann bis zur minimalen Note abwärts; prüfen Sie danach noch einmal Ihre Vorstellung von C4 mit dem Klavier.
Wiederholen Sie das, bis Ihr C4 maximal um einen Halbton abweicht.
Ab diesem Punkt hängt der weitere Fortschritt von der Übung ab; versuchen Sie jedes Mal, wenn Sie am Klavier vorbeikommen, sich das C4 vorzustellen (indem Sie die ersten Takte der Invention \#1 benutzen), und prüfen Sie es.
Sie können das C4 direkt finden, indem Sie sich darauf konzentrieren, wie es auf dem Klavier klingt, aber mit richtiger Musik es ist einfacher, weil Musik aus mehr Assoziationen besteht.
Wenn das C4 ziemlich korrekt ist, beginnen Sie damit, Noten einer beliebigen Stelle des Klaviers zu testen, und raten Sie, welche es sind (nur die weißen Tasten).
Am Anfang werden Sie vielleicht ziemlich daneben liegen.
Es gibt einfach zu viele Noten auf dem Klavier.
Um die Erfolgsquote zu erhöhen, raten Sie die Noten durch das Herstellen des Bezugs zur Oktave von C4 bis C5; so ist z.B. C2 wie C4, nur zwei Oktaven tiefer.
Auf diese Art reduzieren Sie die Aufgabe, 88 Noten auf der Tastatur auswendig zu lernen, auf das Lernen von lediglich 8 Noten und eines Intervalls (Oktave).
Diese Vereinfachung ist möglich, weil die \hyperref[c2_2]{chromatische Tonleiter} logarithmisch aufgebaut ist; eine weitere Vereinfachung der Noten innerhalb der Oktave wird durch die Intervalle erreicht (Halbton, Terz, Quarte, Quinte).
Machen Sie sich mit allen Noten des Klaviers vertraut, indem Sie sie oktavweise spielen und das Gehirn darauf trainieren, alle Noten im Oktavabstand zu erkennen - alle Cs, Ds usw.
Bis Sie sich ein rudimentäres absolutes Gehör antrainiert haben, üben Sie die absolute Tonhöhenerkennung am Klavier, so daß Sie sich sofort korrigieren können, wenn Sie von der richtigen Tonhöhe abweichen.
Üben Sie nicht längere Zeit in Gedanken mit der falschen Tonhöhe; Sie sollten immer das Klavier in der Nähe haben, um sich selbst zu korrigieren.
Beginnen Sie erst mit dem Üben ohne Klavier, wenn Ihre absolute Tonhöhenerkennung höchstens um zwei Halbtöne abweicht.

Lernen Sie dann den ganzen ersten Satz der Mondscheinsonate auswendig, und fangen Sie an, mit den schwarzen Tasten zu arbeiten.
Der Erfolg mit der absoluten Tonhöhenerkennung hängt davon ab, wie Sie sich überprüfen.
Denken Sie sich mehrere Möglichkeiten aus; ich zeige Ihnen ein paar Beispiele.
Benutzen wir zunächst die ersten drei Noten der RH der Mondscheinsonate.
Merken Sie sich den absoluten Klang dieser Noten, und prüfen Sie ihn mehrere Male am Tag.
Prüfen Sie, ob Sie jedesmal die erste Note (G\#3) richtig treffen, wenn Sie am Klavier sind.
Üben Sie die relative Tonhöhe, indem Sie die zweite Note prüfen - C\#4, eine Quarte von G\#3 -, gehen Sie dann in Gedanken einen Halbton nach unten zum C4, und prüfen Sie wieder.
Gehen Sie zur dritten Note, E4, prüfen Sie sie, dann in Gedanken abwärts zu C4, und prüfen Sie.
Von G\#3 in Gedanken einen Halbton abwärts, dann aufwärts zu C4.
Springen Sie nun zu einer beliebigen Stelle dieses Satzes, und wiederholen Sie den Vorgang in ähnlicher Weise.

Der Fortschritt mag Ihnen zunächst langsam vorkommen, aber Ihre Vorstellungen sollten mit zunehmender Übung dem richtigen Klang immer näher kommen.
Am Anfang braucht das Identifizieren der Noten seine Zeit, weil Sie Ihre Vorstellung durch Summen zu Ihrer höchsten oder tiefsten Note überprüfen müssen oder indem Sie sich an den Anfang der Invention oder der Mondscheinsonate erinnern.
Eines Tages sollten Sie dann plötzlich die wundervolle Erfahrung machen, daß Sie jede Note direkt, ohne Zwischenschritte,  identifizieren können.
Sie haben das wahre absolute Gehör erworben!
Dieses anfängliche absolute Gehör ist zerbrechlich, und Sie können es mehrmals verlieren und wieder zurückgewinnen.
Der nächste Schritt ist, Ihr absolutes Gehör zu stärken, indem Sie üben, die Noten so schnell wie möglich zu identifizieren.
Die Stärke Ihres absoluten Gehörs wird von der Geschwindigkeit bestimmt, mit der Sie Noten identifizieren können.
Beginnen Sie danach mit zwei gleichzeitig gespielten Noten, dann mit Akkorden aus drei Noten usw.
Wenn Sie ein starkes absolutes Gehör haben, üben Sie, die Noten mit der richtigen Tonhöhe zu summen und zu singen und das Singen vom Blatt mit der richtigen Tonhöhe.
Gratulation, Sie haben es geschafft!

Der biologische Mechanismus, der dem absoluten Gehör zugrunde liegt, wird noch nicht ganz verstanden.
Er scheint vollständig eine Funktion des Gedächtnisses zu sein.
Um das absolute Gehör wirklich zu erwerben, müssen Sie deshalb Ihre geistigen Gewohnheiten ändern, so wie Sie es tun müssen, um ein guter Auswendiglernender zu werden.
Beim Auswendiglernen war die notwendige Veränderung, die Angewohnheit zu entwickeln, ständig Assoziationen zu erfinden (je ungeheuerlicher oder schockierender, desto besser!) und sie automatisch im Gehirn zu wiederholen.
Bei guten Auswendiglernenden geschieht dieser Vorgang von selbst, oder ohne Aufwand, und deshalb sind sie gut.
Die Gehirne von schlechten Auswendiglernenden werden entweder untätig, wenn Sie nicht gebraucht werden, oder schweifen zu logischen oder anderen Interessen ab, statt für das Auswendiglernen zu arbeiten.
Menschen mit absolutem Gehör neigen dazu, ständig in Gedanken Musik zu machen; in ihren Köpfen spielt ständig Musik, egal ob es ihre eigene Kompositionen sind oder Musik, die sie gehört haben.
Deshalb beginnen die meisten Menschen mit absolutem Gehör automatisch damit, Musik zu komponieren.
Das Gehirn kehrt immer zur Musik zurück, wenn es mit nichts anderem beschäftigt ist.
Das ist wahrscheinlich eine Voraussetzung dafür, ein permanentes absolutes Gehör zu erwerben.
Beachten Sie, daß ein absolutes Gehör Sie nicht zu einem Komponisten macht, das mentale Spielen tut es.
Deshalb ist das mentale Spielen viel wichtiger als ein absolutes Gehör; diejenigen mit einem starken mentalen Spielen können leicht die absolute Tonhöhenerkennung lernen, sie aufrechterhalten und alle in diesem Abschnitt besprochenen Vorteile genießen.
Wie beim Auswendiglernen ist der schwerste Teil des Erwerbs eines permanenten absoluten Gehörs nicht das Üben, sondern die Änderung Ihrer geistigen Gewohnheiten.
Im Prinzip ist es einfach: Spielen Sie soviel sie können in Gedanken, und überprüfen Sie es für das absolute Gehör am Klavier.

Im Rahmen der Gedächtnispflege muß man sich regelmäßig mit dem absoluten Gehör und dem Auswendiglernen mit Hilfe des mentalen Spielens beschäftigen.
Diese Vorgehensweise pflegt automatisch die Tonhöhenerkennung; prüfen Sie einfach hin und wieder, ob Ihr mentales Spielen noch mit der richtigen Tonhöhe arbeitet.
Das sollte ebenfalls automatisch geschehen, weil Sie zumindest den Anfang jedes Stücks mental spielen sollten, bevor Sie es am Klavier spielen.
Indem Sie es zuerst in Gedanken spielen stellen Sie sicher, daß die Geschwindigkeit, der Rhythmus und der Ausdruck korrekt sind.
Musik klingt aufregender, wenn sie mental geführt wird, und weniger aufregend, wenn man sie spielt und darauf wartet, daß das Klavier die Musik erzeugt.
Die Kombination des absoluten Gehörs, mentalen Spielens und Tastatur-Gedächtnisses führt zu einem mächtigen Satz an Werkzeugen, die das Komponieren von Musik einfach werden lassen, sowohl das Komponieren in Gedanken als auch das Spielen auf dem Klavier.

Konventionelle Methode für das Lernen der absoluten Tonhöhenerkennung benötigen viel Zeit, oft mehr als sechs Monate und üblicherweise einiges mehr, und das resultierende absolute Gehör ist schwach.
Eine Möglichkeit anzufangen ist, sich eine Note zu merken.
Sie könnten z.B. das A mit 440 Hz nehmen, weil Sie es jedesmal hören, wenn Sie in ein Konzert gehen und sich vielleicht am einfachsten daran erinnern können.
Das A ist jedoch keine brauchbare Note, um zu den verschiedenen Intervallen der C-Dur-Tonleiter zu kommen, die die nützlichste Tonleiter ist, die man sich merken sollte.
Wählen Sie deshalb das C, E oder G, je nachdem, welches Sie sich am besten merken können; C ist wahrscheinlich am besten.
Die übliche Vorgehensweise für das Lernen der absoluten Tonhöhenerkennung ist in Musikschulen das Solfège (Gesangsübungen).
Solfège-Bücher können über den Buchhandel oder das Internet bezogen werden.
Es besteht aus zunehmend komplexen Folgen von Übungen mit verschiedenen Tonleitern, Intervallen, Taktarten, Rhythmen, Vorzeichen usw. für das Gesangstraining.
Es deckt auch die Tonhöhenerkennung und Diktate ab.
Solfège-Bücher verwendet man am besten in einer Klasse mit einem Lehrer.
Die absolute Tonhöhenerkennung wird als Nebenprodukt zu den Übungen gelehrt, indem man lernt, diese mit der richtigen Tonhöhe zu singen.
Deshalb gibt es für den Erwerb des absoluten Gehörs keine speziellen Methoden - man wiederholt alles so lange, bis die richtige Tonhöhe im Gedächtnis verankert ist.
Da die absolute Tonhöhenerkennung mit vielen anderen Dingen zusammen gelernt wird, ist der Fortschritt langsam.

Kurz gesagt, muß jeder Klavierspieler die absolute Tonhöhenerkennung lernen, weil es so einfach, nützlich und in vielen Situationen sogar notwendig ist.
Wir haben oben gezeigt, daß die absolute Tonhöhenerkennung mit Musik einfacher zu lernen ist als durch reines Auswendiglernen.
Die absolute Tonhöhenerkennung ist untrennbar mit dem mentalen Spielen verbunden, was Sie von den mechanischen Beschränkungen von Musikinstrumenten befreit.
Diese Fähigkeiten zum mentalen Spielen und zur absoluten Tonhöhenerkennung qualifizieren Sie gemäß der gängigen Vorstellungen automatisch als \enquote{talentiert} oder sogar als \enquote{Genie}, aber eine solche Beurteilung ist hauptsächlich für das Publikum wichtig; für Sie selbst ist es beruhigend zu wissen, daß Sie Fähigkeiten erworben haben, die notwendig sind, um ein vollendeter Musiker zu werden.


\paragraph{Vom Blatt singen und komponieren}
\label{c1iii12blatt}

Ein relatives und absolutes Gehör befähigen Sie nicht automatisch dazu, Musik, die Sie gerade gehört haben, sofort niederzuschreiben oder auf dem Klavier zu spielen.
Diese Fertigkeiten müssen genauso geübt werden, wie Sie für die Technik, das Blattspiel oder das Auswendiglernen üben, und braucht eine Weile, bis Sie es gelernt haben; ein relatives und absolutes Gehör zu entwickeln sind die erste Schritte zum Erreichen dieser Ziele.
Um in der Lage zu sein, ein Stück oder Ihre Komposition niederzuschreiben, ist es offensichtlich notwendig, Diktate zu lernen und zu üben.
Ein schneller Weg, Diktate zu üben, ist, das Singen vom Blatt zu üben.
Nehmen Sie ein beliebiges Stück, lesen Sie ein paar Takte, und singen Sie es oder spielen es in Gedanken (nur eine Stimme).
Prüfen Sie es anschließend am Klavier.
Wenn Sie das mit genügend Musik durchführen, die Sie nie zuvor gehört haben, dann werden Sie das Singen vom Blatt lernen und den größten Teil der Fertigkeiten entwickeln, die Sie für Diktate brauchen.
Um zu üben, jede Melodie auf dem Klavier zu spielen, üben Sie das Blattspiel.
Wenn Sie im Blattspiel ziemlich gut geworden sind (das wird mehr als sechs Monate benötigen), fangen Sie an, Ihre eigenen Melodien auf dem Klavier zu spielen.
Der Zweck für das Lernen des Blattspiels ist, daß Sie sich mit allgemeinen Läufen, Akkorden, Begleitungen usw. vertraut machen, so daß Sie sie schnell auf dem Klavier finden können.
Eine andere Möglichkeit ist, mit dem Spielen nach \enquote{Fake Books} anzufangen und das Improvisieren (Abschnitt 23) zu lernen.
Machen Sie sich beim Komponieren keine Sorgen, wenn Sie es zunächst schwierig finden, ein Stück anzufangen oder es zu beenden - das sind die schwierigsten Elemente des Komponierens.
Sammeln Sie nur einige Ideen, die Sie später zu einer Komposition zusammenfügen können.
Sorgen Sie sich nicht darum, daß Sie nie Unterricht im Komponieren hatten; es ist am besten, zuerst den eigenen Stil zu entwickeln und dann das Komponieren zu lernen, um den eigenen Stil weiterzuentwickeln.
Die Musik kommt nie \enquote{auf Befehl}, was frustrierend sein kann; deshalb müssen Sie, wenn die Ideen auftauchen, sofort an ihnen arbeiten.
Musik anzuhören, die Sie mögen, oder an einem guten Konzertflügel zu komponieren, kann inspirierend sein.
Obwohl Digitalpianos für das Komponieren von Popmusik und das Üben von Jazz-Improvisationen ausreichend sind, kann ein qualitativ guter Flügel sehr hilfreich sein, wenn man klassische Musik auf hoher Ebene komponiert.



<!-- c1iii13.html -->

\subsection{Filmen und Aufnehmen des eigenen Spielens,\footnote{MIDI, Digitalpianos, Keyboards usw.}}
\label{c1iii13} 

Eine der besten Möglichkeiten, Ihr musikalisches Spielen zu verbessern und für Auftritte zu üben, ist, sich selbst zu filmen bzw. aufzunehmen und es sich anzusehen oder anzuhören.
Sie werden überrascht sein, wie gut oder wie schlecht die verschiedenen Teile Ihres Spielens sind.
Oftmals unterscheidet es sich sehr von dem, was Sie zu tun glauben.
Haben Sie einen guten Anschlag?
Haben Sie Rhythmus?
Ist Ihr Tempo genau und konstant?
Welche Bewegungen unterbrechen den Rhythmus?
Stellen Sie die Melodielinien klar heraus?
Ist eine Hand zu laut oder zu leise?
Sind die Arme, Hände und Finger in der optimalen Position?
Benutzen Sie den ganzen Körper, d.h. sind die Körperbewegungen mit den Händen synchron oder arbeiten sie gegeneinander?
All das und noch viel mehr wird sofort offensichtlich.
Die gleiche Musik klingt sehr unterschiedlich, wenn Sie sie spielen oder sich Ihre Aufnahme anhören.
Man hört viel mehr, wenn man sich eine Aufnahme anhört, als wenn man spielt.
Eine Videoaufnahme ist die beste Möglichkeit, sich auf ein Konzert vorzubereiten und kann manchmal die \hyperref[c1iii15]{Nervosität} fast völlig eliminieren, da Sie eine genauere Vorstellung von Ihrem Auftritt haben.

Zuerst stellten die meisten Pianisten nur Audioaufnahmen her, weil sie dachten, daß das musikalische Ergebnis das wichtigste sei; hinzu kommt, daß die älteren Videokameras Musik nicht angemessen aufzeichnen konnten.
Audioaufnahmen haben den Nachteil, daß eine gute Aufzeichnung des Klavierklangs schwieriger ist als den meisten bewußt ist, und solche Versuche führen oft zu einem Fehlschlag und zur Aufgabe der Bemühungen.
Videokameras sind mittlerweile so erschwinglich und vielseitig, daß die Videoaufzeichnung nun zweifellos die bessere Methode ist.
Obwohl der resultierende Klang eventuell nicht der CD-Qualität entspricht (glauben Sie den Behauptungen der Hersteller von digitalen Videokameras nicht), brauchen sie keine solche Qualität, um alle nützlichen pädagogischen Ziele zu erreichen.
Wählen Sie eine Videokamera aus, bei der man die automatische Aussteuerung (AGC = automatic gain control) der Audioaufnahme abschalten kann; ansonsten werden die pianissimo gespielten Passagen verstärkt und verzerrt.
Viele Verkäufer kennen sich mit diesen Eigenschaften nicht aus, da sie meistens Optionen in den Einstellungen der Software sind.
Sie werden auch ein ziemlich stabiles Stativ benötigen; ein leichtes könnte wackeln, wenn Sie auf das Klavier \enquote{einhämmern}.
Nur Konzertpianisten benötigen höherwertige Audioaufnahmesysteme; suchen Sie sich, um die besten Resultate kosteneffizient zu erzielen, ein Aufnahmestudio. 
Hochwertige Audioaufnahmen benötigen Sie eventuell für mehrere Zwecke; die Aufnahmetechnik ändert sich so schnell, daß Sie am besten im Internet nach den aktuellen Geräten und Methoden suchen, ich werde deshalb hier nicht weiter darauf eingehen.

Fangen Sie damit an, daß Sie eine 1:1-Zuordnung zwischen dem, was Sie \textit{denken}, daß Sie spielen, und dem tatsächlichen Ergebnis (Video oder Audio) herstellen.
Auf diese Art können Sie Ihre allgemeinen Spielgewohnheiten so abändern, daß das Ergebnis richtig herauskommt.
Wenn Sie z.B. bei leichteren Abschnitten schneller spielen als Sie denken und langsamer bei schwereren Abschnitten, können Sie die richtigen Anpassungen vornehmen.
Sind Ihre Pausen lange genug?
Sind die Enden überzeugend?

Das Aufnehmen wird offenbaren, wie Sie bei einer richtigen Aufführung reagieren würden, z.B. wenn Sie einen Fehler machen oder hängenbleiben.
Reagieren Sie negativ auf Fehler und werden entmutigt, oder können Sie sich davon lösen und auf die Musik konzentrieren?
Während eines Konzerts neigt man dazu, Gedächtnisblockaden usw. an unerwarteten Stellen zu bekommen, an denen man im allgemeinen während des Übens keine Probleme hatte.
Das Aufnehmen kann die meisten dieser Problemstellen zutage fördern.
Ihre Stücke sind nicht \enquote{fertig}, solange Sie sie nicht zufriedenstellend aufnehmen können.
Videoaufnahmen sind eine sehr gute Simulation für das Spielen in einem Konzert.
Wenn Sie also während der Videoaufzeichnung zufriedenstellend spielen können, sollten Sie wenig Probleme haben, dieses Stück in einem Konzert zu spielen.
Wenn Sie erst einmal mit dem Aufnehmen begonnen haben, möchten Sie die Aufnahmen vielleicht sogar anderen zusenden!

Was sind die Nachteile?
Der Hauptnachteil ist, daß es viel Zeit beansprucht, da man sich die Aufnahmen ansehen und anhören muß.
Sie können vielleicht etwas Zeit sparen, indem Sie sich die Aufnahmen anhören, während Sie etwas anderes erledigen.
Die Aufnahme selbst braucht wenig zusätzliche Zeit, weil das als Teil der Übungszeit zählt.
Jedesmal wenn Sie einen Abschnitt korrigieren, müssen Sie jedoch erneut aufnehmen und wieder abhören.
Deshalb läßt sich die Tatsache nicht leugnen, daß sich selbst anzusehen oder anzuhören eine zeitaufwendige Aufgabe ist.
Es ist jedoch etwas, das jeder Klavierschüler tun muß.
Ein Problem mit Videokameras ist, daß ihr Motorgeräusch durch das eingebaute Mikrofon aufgenommen wird.
Wenn Sie das stört, finden Sie entweder ein Modell, das ein anschließbares Mikrofon von guter Qualität hat oder eines mit einem  Mikrofoneingang und kaufen Sie ein separates Mikrofon guter Qualität.


\label{c1iii13MIDI}

<h3><br>MIDI, Digitalpianos, Keyboards usw.</h3>

\footnote{Die folgenden Abschnitte sind im Original nicht enthalten.
Ich habe sie jedoch wegen der besseren Lesbarkeit in normaler Schrift belassen.
MIDI steht für Musical Instruments Digital Interface. Die Töne werden dabei als Code gespeichert, der in unserer Sprache z.B. \enquote{Ein C3 auf Kanal 2 mit der Lautstärke 90 und der Länge von 192 Ticks.} oder \enquote{Eine Viertelnote der Tonhöhe C3, die forte und legato gespielt wird.} bedeutet.}


\textbf{Besitzer eines PCs und eines MIDI-Keyboards haben eine gute Alternative zu Audioaufnahmen per MD, Tonband usw.}
Dazu muß entweder ein Soundchip auf dem \hyperref[Motherboard]{Motherboard} des PC integriert sein und der PC einen Joystick/MIDI-Anschluß haben oder Sie brauchen eine extra Soundkarte.
Die preisgünstigeren Karten haben einen kombinierten Joystick/MIDI-Anschluß, für den Sie dann ein zusätzliches Adapterkabel brauchen.
Die teureren Karten haben teilweise richtige 5-polige MIDI-Buchsen.
Leider gibt es ein paar Soundkarten, die zwar MIDI ausgeben aber nicht MIDI aufnehmen können; auf der Verpackung steht davon natürlich nichts.
Am besten lassen Sie sich vor dem Kauf vom Händler oder Vorbesitzer bestätigen, daß die Karte in beiden Richtungen MIDI-fähig ist und bringen die Karte wieder zurück, wenn es nicht stimmt.

Beim Verkabeln werden gerne die Anschlüsse vertauscht.
Haben Sie ein Joystick/MIDI-Adapterkabel, dann stecken Sie den Stecker, der mit \textit{MIDI-Out} beschriftet ist, in die \textit{MIDI-In}-Buchse des Keyboards und den \textit{MIDI-In}-Stecker in die \textit{MIDI-Out}-Buchse des Keyboards.
Bei mehreren Geräten gibt es natürlich noch weitere Möglichkeiten.
\textbf{Es muß aber immer der MIDI-Out-Anschluß eines Geräts mit dem MIDI-In- oder MIDI-Thru-Anschluß eines anderen Geräts verbunden werden.}

Zu guter Letzt brauchen Sie noch ein Programm, mit dem Sie MIDI aufnehmen, bearbeiten und wiedergeben können, d.h. ein Sequenzer-Programm.
Es gibt einige professionelle Programme, die natürlich auch einiges kosten.
Sie bieten jede Menge Funktionen, insbesondere für die nachträgliche Bearbeitung, teilweise auch Notensatz und virtuelle (d.h. vom Programm erzeugte) Geräte, wie z.B. Drum-Computer oder Synthesizer.
Wenn es nur darum geht, das eigene Klavierspielen zu kontrollieren, kann man jedoch erst einmal getrost auf diese ganzen Funktionen verzichten.
Es gibt auch ein paar Free- und Shareware-Programme, die für diesen Zweck völlig ausreichen.

Sie können die aufgenommenen MIDI-Signale direkt über das Keyboard wiedergeben, d.h. Sie erhalten original den gleichen Klang wie bei der Aufnahme.
Dazu müssen Sie im Sequenzer-Programm die Ausgabe auf die MIDI-Out-Schnittstelle anstatt auf den Klangerzeuger-Chip der Soundkarte einstellen.

\label{midi_check}

Wenn Sie weiter ins Detail gehen möchten, können Sie sich im Piano-Roll-Editor ansehen, ob der Einsatz der einzelnen Noten exakt auf die Taktschläge kommt, ob die Notenlängen stimmen, ob Sie wirklich legato spielen usw.
Der Piano-Roll-Editor ist das Fenster, in dem die Noten wie auf der Steuerrolle einer Kirmesorgel als mehr oder weniger lange Striche dargestellt sind.
Im Event-Editor, also in der Einzelanzeige der MIDI-Daten können Sie die Stärke Ihres Anschlags, die meistens als Velocity bezeichnet wird, sehen.
Natürlich dürfen Sie die Werte auch nicht zu ernst nehmen. Schließlich sind Sie kein Roboter und außerdem klingt es tot und langweilig, wenn alle Noten exakt die gleiche Lautstärke und auf die interne Auflösung genau die gleiche Länge haben.
\textbf{Auch dürfen Sie trotz der ganzen technischen Unterstützung das Training Ihres Gehörs nicht vernachlässigen.}
Hören Sie sich bei der Kontrolle per MIDI ebenfalls selber beim Spielen zu, und achten Sie beim Abhören auf die gleichen Dinge, die im \hyperref[c1iii13]{ersten Abschnitt} beschrieben wurden.

Wenn Ihr Keyboard keine \enquote{Split}-Funktion hat (Aufteilung der Tastatur bei einer bestimmten Note in 2 MIDI-Spuren) können Sie mit dem Programm die Noten der RH und LH voneinander trennen und jeweils separat anhören.
Bei mehrstimmigen Stücken können Sie auch den Aufwand noch ein wenig weiter treiben und die einzelnen Stimmen voneinander trennen.
Das ist vor allem dann sinnvoll, wenn aus der Einspielung ein Notat erstellt werden soll.
Die automatische Umwandlung von eingespielten MIDI-Dateien in einen Notensatz ist nämlich eine Sache für sich.
Hier trennt sich bei den Programmen schnell die Spreu vom Weizen, und in den meisten Fällen ist eine erhebliche Nachbearbeitung der MIDI-Daten erforderlich.
Beim Umsetzen von Live-Einspielungen sind Programme von Vorteil, die die MIDI-Daten so weit wie möglich unverändert lassen und die Notation separat speichern, da der Originalklang der aufgenommenen MIDI-Signale zur Hörkontrolle erhalten bleibt.
Wenn es ohnehin nur darum geht, Noten z.B. für einen Chor oder eine Band zu setzen, ist man mit einer schrittweisen Eingabe der Noten (Step-Recording) oft schneller am Ziel.


\label{kauf}

Zum Schluß noch ein paar Fragen, die man vor der Entscheidung für und dem Kauf eines Keyboards klären sollte:

\begin{enumerate}[label={\arabic*.}] 
\item Stellen Sie sich zunächst die Frage, was Ihr Ziel bei der ganzen Sache ist.<br>
\textbf{Sie möchten richtig Klavier spielen.} Eigentlich geht das ja nur auf einem guten Klavier.
Zum Glück sind die guten Digitalpianos mittlerweile sehr nah dran am Klavier.
Am wichtigsten ist die Qualität der Tastatur und ihre Ähnlichkeit mit einer Klaviertastatur hinsichtlich Spielgefühl und Ansprechverhalten.
Ein gutes Digitalpiano ist auf jeden Fall besser als ein schlechtes oder verstimmtes Klavier!<br>
\textbf{Sie möchten irgendwann als Alleinunterhalter auftreten.}
Vergessen Sie die Sache mit dem Klavier.
Kaufen Sie sich ein Entertainer-Keyboard mit NNN(N) Klängen und Begleitrhythmen.
Achten Sie nicht zu sehr auf die eingebauten Lautsprecher; für Ihren Auftritt brauchen Sie ohnehin noch einen Verstärker und vernünftige Boxen.<br>
\textbf{Sie möchten das Keyboard nur als Eingabe für ein Sequenzer-Programm benutzen.} Hauptsache MIDI -  Live-Einspielungen können selten so benutzt werden wie sie sind und müssen in der Regel im Sequenzer-Programm kräftig nachbearbeitet werden.
Als Alternative sei hier noch einmal an das Step-Recording erinnert.<br>
\textbf{Sie möchten eigene Klänge \enquote{basteln}.} Willkommen im Wunderland der Sampler, Synthesizer und Workstations.
Sie werden viel Zeit mit Tätigkeiten verbringen, für die Sie die in diesem Buch vorgestellten Methoden nicht brauchen.
Wenn Sie allerdings nicht nur Musik im Studio produzieren, sondern auch mal live auf der Bühne spielen möchten, ist es sehr von Vorteil, die Methoden zu kennen.<br>


\item Da die Soundkarten für PCs auch immer besser werden und zunehmend Funktionalität und ladbare Klänge bieten, ist die Frage, auf welche Eigenschaften und Funktionen beim Keyboard verzichtet werden kann.
Beim Keyboard lassen sich so eventuell ein paar Euro einsparen, die für etwas anderes sinnvoller angelegt sind.
Beim Kopfhörer sollte man z.B. nicht sparen; allerdings auch nicht Geld für Meßwertunterschiede ausgeben, die man nicht hört.
Probieren Sie am besten im Geschäft in Ruhe mehrere Kopfhörer an Ihrem favorisierten Keyboard aus.


\item Wohin mit all den Teilen? Unterschätzen Sie nicht den Platzbedarf.
Zuerst ist es nur das Keyboard, das in der Nähe des PCs steht, weil das MIDI-Kabel schließlich nicht unendlich lang ist.
Irgendwann kommt dann meistens ein vernünftiger Verstärker inkl. Boxen hinzu oder die Stereoanlage zieht um.
Wenn die Zahl der \enquote{Eingabegeräte} steigt und in der PC-Software kein Mischpult enthalten ist, dann kommt das auch noch hinzu.
Auch hier stellt sich wieder die Frage \textbf{\enquote{Was will ich...?}}


 \end{enumerate}
\footnote{Ende der Einfügung.}



<!-- c1iii14.html -->

\subsection{Vorbereitung auf Auftritte und Konzerte}
\label{c1iii14} 

\subsubsection{Nutzen und Risiken von Auftritten und Konzerten}
\label{c1iii14a}

Der Nutzen und die Risiken der Auftritte bestimmen unser tägliches Programm für das Lernen des Klavierspielens.
Für den Amateurklavierspieler ist der Nutzen von Auftritten, auch von zufälligen, unermeßlich.
Der wichtigste Nutzen ist, daß die Technik nie richtig erworben wird, bis man sie in einem Auftritt zeigen kann.
Für junge Schüler ist der Nutzen sogar noch fundamentaler.
Sie lernen, was es bedeutet, eine richtige Aufgabe zu beenden, und sie lernen, was \enquote{Musik machen} bedeutet.
Die meisten Kinder (die keinen Musikunterricht erhalten) lernen diese Fertigkeiten nicht, bis sie aufs College gehen; Klavierschüler müssen sie, unabhängig von ihrem Alter, bei ihrem \textit{ersten Konzert} lernen.
Schüler sind nie so selbstmotiviert wie bei der Vorbereitung auf ein Konzert.
Lehrer, die Konzerte veranstaltet haben, kennen den enormen Nutzen.
Ihre Schüler werden konzentriert, selbstmotiviert und ergebnisorientiert; sie hören dem Lehrer aufmerksam zu und versuchen wirklich, die Bedeutung der Anweisungen des Lehrers zu verstehen.
Die Schüler meinen es todernst damit, alle Fehler zu eliminieren und alles korrekt zu lernen - es ist ein privates Unternehmertum in Vollendung, weil es \textit{ihr} Konzert ist.
Lehrer ohne Konzerte haben am Ende oft Schüler, die vielleicht ein paarmal unmittelbar vor dem Unterrichtstag üben.

Da die Psychologie und Soziologie des Klavierspielens nicht gut entwickelt sind, gibt es Risiken, die wir ernsthaft bedenken müssen.
Das wichtigste ist die \hyperref[c1iii15]{Nervosität} und ihre Auswirkungen auf den Geist, besonders bei Kindern.
Nervosität kann Konzerte zu einer furchtbaren Erfahrung machen, die eine sorgfältige Beachtung erfordert, um nicht nur unglückliche Erfahrungen, sondern auch bleibende psychologische Schäden zu vermeiden.
Die Nervosität zu reduzieren wird zumindest den Streß und die Furcht abschwächen.
Konzerte zu einer angenehmen Erfahrung werden zu lassen und Spannung und Streß zu reduzieren, wird nicht genug Beachtung geschenkt, insbesondere bei Klavierwettbewerben.
Dieses ganze Thema wird im Abschnitt über Nervosität vollständiger behandelt.
Der Punkt ist hier, daß jede Abhandlung des Auftretens eine Diskussion des Lampenfiebers einschließen muß.
Sogar große Künstler haben aus dem einen oder anderen Grund für längere Zeit aufgehört aufzutreten, und einige der Gründe standen zweifellos im Zusammenhang mit dem Streß.
Obwohl gute Klavierlehrer stets Konzerte ihrer Schüler abhalten und sie an Wettbewerben teilnehmen lassen, sind sie oft keine guten Soziologen und Psychologen, konzentrieren sich nur auf das Klavierspielen und ignorieren die Nervosität.
Es ist für jede Person, die Kinder durch Konzerte und Wettbewerbe begleitet, wichtig, die Grundlagen dessen zu lernen, was Nervosität verursacht, wie man mit ihr umgeht und was ihre psychologischen Konsequenzen sind.
Wenn die Lehrer in dieser Hinsicht versagen, ist es die Aufgabe der Eltern, das soziale und psychologische Wohl ihrer Kinder im Auge zu behalten; darum ist der folgende Abschnitt (15) über Nervosität ein notwendiger Begleiter dieses Abschnitts.

Es gibt zahlreiche weitere psychologische und soziologische Gesichtspunkte bei Konzerten und Wettbewerben.
Die Bewertungssysteme in Musikwettbewerben sind bekanntermaßen unfair, und das Bewerten ist eine schwierige und undankbare Aufgabe.
Deshalb müssen Schüler, die an einem Wettbewerb teilnehmen sollen, über diese Unzulänglichkeiten des Systems informiert werden, damit die wahrgenommene Ungerechtigkeit und die Enttäuschung nicht zu psychologischen Problemen führen.
Es ist für Schüler schwierig aber möglich, zu verstehen, daß das Dabeisein das wichtigste Element von Wettbewerben ist, nicht daß man gewinnt.
Es wird zuviel Wert auf die technische Schwierigkeit gelegt und nicht genug auf die Musikalität.
Das System ermutigt nicht die Kommunikation zwischen den Lehrern, um die Lehrmethoden zu verbessern.
Es ist kein Wunder, daß es eine Denkrichtung gibt, die das Abschaffen der Wettbewerbe befürwortet.
Es steht außer Frage, daß Konzerte und Wettbewerbe notwendig sind; aber die derzeitige Situation kann sicher verbessert werden.
Mehr dazu in \hyperref[c1iii15]{Abschnitt (15)}.


\subsubsection{Grundlagen fehlerfreien Vorspielens}
\label{c1iii14b}

Die grundlegenden Voraussetzungen für ein fehlerfreies Vorspielen sind: technische Vorbereitung, musikalische Interpretation, \hyperref[c1ii12mental]{mentales Spielen} und ein gutes Verfahren für die Vorbereitung auf den Auftritt.
Wenn alle diese Elemente zusammenkommen, ist ein perfekter Auftritt im Grunde garantiert.

Natürlich gibt es viele Entschuldigungen dafür, daß man nicht auftreten kann.
Diese Entschuldigungen zu kennen, ist eine der Voraussetzungen dafür, zu lernen wie man auftritt.
Die vielleicht am häufigsten vorgebrachte Entschuldigung ist, daß man immer neue Stücke lernt, so daß ungenügend Zeit vorhanden ist, um ein Stück wirklich abzuschließen oder die fertigen Stücke in spielbarem Zustand zu halten.
Wir haben gesehen, daß ein neues Stück zu lernen die beste Art ist, die alten Stücke zu verschlechtern.
Für diejenigen, die niemals aufgetreten sind, ist der zweite wichtige Grund, daß sie wahrscheinlich nie irgendein Stück wirklich zu Ende gebracht haben.
In jedem \enquote{interessanten} Stück, das es wert ist aufgeführt zu werden, gibt es immer diesen einen schwierigen Abschnitt, den man nicht richtig bewältigen kann.
Eine weitere Entschuldigung ist, daß Stücke, die leicht für Sie sind, irgendwie immer uninteressant sind.
Beachten Sie, daß die Lernmethoden dieses Buchs so konzipiert sind, daß sie jeder dieser Entschuldigungen entgegenwirken und zwar hauptsächlich durch die Beschleunigung des Lernprozesses und durch die Förderung des \hyperref[c1iii6]{Auswendiglernens},
so daß in dem Moment, in dem Sie ein Stück gut in Gedanken spielen können, keine dieser Entschuldigungen mehr berechtigt ist.
Somit befinden sich alle für ein fehlerfreies Vorspielen notwendigen Elemente in diesem Buch.
Wir besprechen nun ein paar weitere Gesichtspunkte des Lernens, wie man auftritt.


\subsubsection{Für Auftritte üben}
\label{c1iii14c}

Direkt vor dem Auftritt benutzen die meisten Pianisten zur Vorbereitung auf den Auftritt eine Übungsgeschwindigkeit, die etwas niedriger als die Aufführungsgeschwindigkeit ist.
Diese Geschwindigkeit gestattet das exakte Üben, ohne daß man unerwartete schlechte Gewohnheiten annimmt, und erzeugt ein klares geistiges Bild der Musik.
Sie konditioniert auch die Hand dafür, bei der schnelleren Aufführungsgeschwindigkeit mit Kontrolle zu spielen, und verbessert die Technik.
Diese langsamere Geschwindigkeit ist nicht notwendigerweise einfacher als die Aufführungsgeschwindigkeit.
Der Grund für die zwei Geschwindigkeiten ist, daß es während des Vorspielens leichter ist, den Ausdruck herauszubringen, wenn man etwas schneller spielt als man beim letzten Mal gespielt hat.
Wenn Sie dieselbe Komposition zweimal hintereinander spielen (oder am selben Tag), kommt die Musik beim zweiten Mal flach heraus, außer wenn sie schneller als beim ersten Mal gespielt wird, weil das langsamere Spielen weniger aufregend klingt und dieses Gefühl - zusätzlich zum \hyperref[fpd]{FPD (Schnellspiel-Abbau)} - eine negative Rückkopplung erzeugt.
Nach solchem wiederholten Spielen (eigentlich nach jedem Vorspielen), sollten Sie so schnell wie möglich langsam spielen, um den FPD zu verhindern und die Musik in Ihrem Kopf \enquote{zurückzusetzen}.
Bei Computern gibt es einen ähnlichen Vorgang: Nach längerem Gebrauch sind die Daten auf der Festplatte zunehmend fragmentiert und müssen defragmentiert werden, damit sie wieder zusammenhängend gespeichert sind.

Wer keine Erfahrung im Vorspielen hat, spielt wegen der \hyperref[c1iii15]{Nervosität} während des Konzerts oft schneller als er aufgrund seiner Fertigkeitsstufe kann. 
Solche unpassenden Geschwindigkeiten kann man mit Hilfe von Videoaufnahmen leicht erkennen.
Für den Fall, daß Ihnen dieser Fehler während des Auftritts unterläuft, ist es deshalb während des routinemäßigen Übens (nicht unmittelbar vor einem Auftritt) wichtig, mit Geschwindigkeiten zu üben, die schneller als die Aufführungsgeschwindigkeit sind.
Offensichtlich muß die Aufführungsgeschwindigkeit niedriger als Ihre höchste Geschwindigkeit sein.
Erinnern Sie sich daran, daß das Publikum dieses Stück nicht wie Sie während des Übens unzählige Male gehört hat und somit nicht genauso mit jedem Detail vertraut ist; es ist wahrscheinlich, daß es für das Publikum viel schneller klingt als für Sie, und Ihre \enquote{endgültige Geschwindigkeit} kann für das Publikum zu schnell sein.
Ein Stück, das mit sorgfältiger Beachtung jeder Note gespielt wird, kann schneller klingen als ein Stück, das mit höherer Geschwindigkeit gespielt wird aber mit nicht deutlich zu erkennenden Noten.
Sie müssen dem Publikum jede Note \enquote{mundgerecht servieren}, da es diese sonst nicht hört.

Üben Sie, über Fehler hinwegzukommen.
Besuchen Sie Konzerte von Schülern und beobachten Sie, wie diese auf ihre Fehler reagieren; Sie werden leicht die richtigen und die falschen Reaktionen erkennen.
Ein Schüler, der nach einem Fehler seine Frustration zeigt oder seinen Kopf schüttelt, macht aus einem Fehler drei: den ursprünglichen Fehler, eine falsche Reaktion, und er vermittelt dem Publikum, daß ein Fehler begangen wurde.
Mehr dazu unten in \hyperref[c1iii14g]{Abschnitt g}.


\label{c1iii14musikalisch}
\subsubsection{Musikalisch üben}
\label{c1iii14d}

Was bedeutet es, musikalisch zu spielen?
Diese Frage kann nur durch eine Unzahl von Mikroregeln definitiv beantwortet werden, die auf bestimmte Abschnitte in bestimmten Kompositionen anwendbar sind; hierbei kann Ihnen ein Lehrer zeigen, was Sie tun müssen.
Wenn Sie die gesamte musikalische Notation inklusive der Zeichen in Ihre Musik einbeziehen, haben Sie eine solide Grundlage.
Es gibt ein paar allgemeine Regeln für das musikalische Spielen:

\begin{enumerate}[label={\roman*.}] 
\item Verbinden Sie jeden Takt sorgfältig mit dem vorangegangenen Takt (oder dem Schlag oder der Phrase).
Diese Takte bzw. Schläge stehen nicht alleine da; einer fließt logisch in den nächsten, und sie unterstützen sich alle gegenseitig.
Sie sind ebenso rhythmisch wie konzeptionell verbunden.
Man könnte meinen, daß dieser Punkt in trivialer Weise offensichtlich ist; aber wenn Sie dies bewußt tun, könnten Sie von der entscheidenden Verbesserungen Ihrer Musik überrascht werden.

\item Es muß immer eine Unterhaltung zwischen der RH und LH vorhanden sein.
Sie spielen nicht unabhängig voneinander.
Und sie werden nicht automatisch miteinander reden, nur weil Sie sie zeitlich perfekt aufeinander abgestimmt haben.
Man muß eine Unterhaltung der beiden Hände oder Stimmen bewußt erzeugen.

\item \enquote{cresc.} bedeutet, daß das meiste der Passage leise gespielt werden sollte; nur die letzten paar Noten sind laut, d.h. es ist wichtig leise anzufangen.
Ähnlich ist es bei den anderen Markierungen dieser Art (rit., accel., dim. usw.); stellen Sie sicher, daß Sie Platz für das Stattfinden der Aktion reservieren, und fangen Sie sie nicht sofort an, warten Sie bis zum letzten Moment.
Diese \enquote{Ausdrucksmittel} sollten ein geistiges Bild erzeugen; wenn Sie z.B. ein Crescendo schrittweise steigern, ist es so, als ob man eine Treppe hinaufsteigt, während es, wenn Sie bis zum letzten Moment warten und exponentiell steigern, so ist, als ob man in die Luft geworfen wird, was einen größeren Effekt erzielt.


\item Streben Sie mehr nach Genauigkeit als nach einem ausdrucksstarken Rubato; \textit{rubato} ist oft zu einfach, inkorrekt und nicht im Einklang mit dem Publikum.
Hier ist der richtige Zeitpunkt, das Timing und den Rhythmus mit dem Metronom zu prüfen!

\item Wenn Sie im Zweifel sind, beginnen und beenden Sie jede Phrase leise, mit den lauteren Noten nahe der Mitte.
Es ist üblicherweise falsch, die lauten Noten am Anfang zu haben; selbstverständlich kann man aber auch Musik machen, indem man diese Regel bricht.

\end{enumerate}
Musikalität hat keine Grenze - Sie können sich unabhängig davon verbessern, wo Sie sich auf der Skala der Musikalität befinden.
Der angsteinflößende Teil davon ist die Kehrseite.
Wenn man nicht achtgibt, kann man unmusikalische Spielgewohnheiten entwickeln, die fortwährend die Musikalität zerstören können.
Darum ist es so wichtig, sich auf die Musikalität zu konzentrieren und nicht nur auf die Technik; es kann den Unterschied ausmachen, ob Sie auftreten können oder nicht.

Hören Sie (beim Üben) stets Ihrer eigenen Musik zu, und führen Sie Ihre Musik mit dem \hyperref[c1ii12mental]{mentalen Spielen} - das ist der einzige Weg, die Aufmerksamkeit des Publikums anzuziehen.
Werden Sie bei einem Fehler nicht bedrückt, weil das es erschweren würde, gut zu spielen.
Wenn Sie einen guten Start haben, wird das Publikum jedoch hineingezogen, die Musik trägt sich selbst, und der Auftritt wird leichter.
Somit wird das Spielen zu einer Rückkopplungsschleife zweier Vorgänge, die sich gegenseitig unterstützen müssen: die Musik mittels des mentalen Spielens führen und der vom Klavier ausgehenden Musik zuhören.

Viele Schüler hassen es, zu üben wenn andere dabei sind, die zuhören; manche sind sogar der Ansicht, daß intensives Klavierüben notwendigerweise unangenehm und eine Strafe für das Ohr sei.
Das sind Symptome verbreiteter falscher Vorstellungen, die aus ineffizienten Übungsmethoden resultieren, und Zeichen einer schwachen mentalen Ausdauer.
Mit den richtigen Übungsmethoden und dem musikalischen Spielen sollte an den Übungseinheiten nichts Unangenehmes sein.
\textbf{Das beste Kriterium dafür, ob Sie richtig üben, ist die Reaktion der anderen - wenn Ihr Üben gut für sie klingt oder sie zumindest nicht stört, dann machen Sie es richtig.}
Das musikalische Spielen verbessert die mentale Ausdauer.



<!-- c1iii14e.html -->

\subsubsection{Zwangloses Vorspielen}
\label{c1iii14e}

Gewöhnliche Arten zwanglosen Vorspielens sind Stücke zu spielen, um in Geschäften Klaviere zu testen, oder bei Partys für Freunde zu spielen usw.
Diese unterscheiden sich von formellen Konzerten aufgrund ihrer größeren Freiheit und dem reduzierten mentalen Druck.
Es gibt üblicherweise kein festgelegtes Programm, Sie picken sich das heraus, was Sie im Moment für angemessen halten.
Es kann voller Änderungen und Unterbrechungen sein.
\hyperref[c1iii15]{Nervosität} ist in der Regel kein Thema, und das zwanglose Vorspielen ist sogar eine der besten Möglichkeiten, Methoden zur Vermeidung von Nervosität zu üben.
Trotz dieser abschwächenden Faktoren ist das am Anfang nicht leicht.
Um einen leichten Start zu bekommen, spielen Sie kleine Auszüge (kurze Ausschnitte einer Komposition).
Beginnen Sie mit leichten Stücken; picken Sie nur die am besten klingenden Abschnitte heraus.
Wenn es nicht so gut funktioniert, beginnen Sie mit einem anderen. Dasselbe wenn Sie hängenbleiben.
Sie können jederzeit anfangen und aufhören.
Das ist eine großartige Möglichkeit zu experimentieren und herauszufinden wie Sie vorspielen und welche  Auszüge funktionieren.
Neigen Sie dazu, zu schnell zu spielen?
Es ist besser, zu langsam anzufangen und schneller zu werden als umgekehrt.
Können Sie ein schönes Legato spielen, oder ist Ihr Klang zu schrill?
Können Sie sich an ein anderes Klavier anpassen, insbesondere an eines, das verstimmt oder schwer zu spielen ist?
Können Sie der Reaktion des Publikums folgen?
Reagiert das Publikum auf Ihr Spielen?
Können Sie die passende Art von Auszügen für die Gelegenheit auswählen?
Können Sie sich selbst in die richtige Gemütsverfassung zum Spielen bringen?
Wie nervös sind Sie, können Sie es kontrollieren?
Können Sie Fehler überspielen ohne sich von ihnen stören zu lassen?
Eine weitere Möglichkeit, das Vorspielen zu üben, ist, Kinder, die nie Klavierunterricht hatten, in das Klavierspielen einzuführen.
Bringen Sie ihnen die Tonleiter, \enquote{Alle meine Entchen} oder \enquote{Happy Birthday} bei.

Auszüge zu spielen hat einen interessanten Vorteil. Das Publikum ist meistens von Ihrer Fähigkeit beeindruckt, irgendwo in der Mitte eines Stücks anzufangen und aufzuhören.
Die meisten Menschen nehmen an, daß alle Amateurklavierspieler Stücke vom Anfang bis zum Ende mit einem \hyperref[c1iii6d]{Hand-Gedächtnis} lernen, und daß die Fähigkeit Auszüge zu spielen ein besonderes Talent erfordert.
Fangen Sie mit kurzen Auszügen an, und versuchen Sie dann schrittweise längere.
Haben Sie dieses zwanglose Auszüge-Vorspielen erst einmal bei vier oder fünf verschiedenen Gelegenheiten gemacht, können Sie Ihre Fähigkeiten zum Vorspielen gut einschätzen.
Offensichtlich sollte das Spielen von Auszügen eines der Dinge sein, die sie regelmäßig \hyperref[c1iii6g]{\enquote{kalt} üben}.

Es gibt ein paar Regeln für die Vorbereitung auf das Auszüge-Vorspielen.
Spielen Sie kein Stück, das Sie gerade gelernt haben.
Lassen Sie es für mindestens sechs Monate schmoren; am besten ein Jahr (üben Sie während dieser Zeit das Auszüge-Vorspielen).
Wenn Sie gerade zwei Wochen damit verbracht haben, ein schwieriges neues Stück zu lernen, dann erwarten Sie nicht, daß Sie in der Lage sind Auszüge zu spielen, die Sie in diesen zwei Wochen überhaupt nicht gespielt haben - seien Sie auf alle Arten von Überraschungen, wie z.B. Gedächtnisblockaden, vorbereitet.
Üben Sie die Auszüge an dem Tag, an dem Sie sie vorführen werden, nicht schnell.
Sie sehr langsam zu üben wird hilfreich sein.
Können Sie sie immer noch HS spielen?
Sie können eine Menge dieser Regeln bei sehr kurzen Auszügen brechen.
Prüfen Sie vor allen Dingen, ob Sie sie \hyperref[c1ii12mental]{in Gedanken (ohne das Klavier) spielen} können - das ist der ultimative Test, ob Sie bereit sind.

Allgemein gesagt: Erwarten Sie nicht, daß Sie etwas gut darbieten können, egal ob informell oder nicht, solange Sie das Stück nicht mindestens dreimal vorgeführt haben; manche behaupten mindestens fünfmal.
Abschnitte, die Sie für einfach hielten, können sich als schwierig vorzuspielen erweisen und umgekehrt.
Deshalb ist der erste Punkt der Tagesordnung, daß Sie Ihre Erwartungen ein wenig verringern und anfangen zu planen, wie Sie das Stück spielen werden, besonders wenn etwas unerwartetes geschieht.
Es wird sicherlich nicht so wie Ihr bester Durchgang beim Üben werden.
Ohne diese mentale Vorbereitung kann es Ihnen passieren, daß Sie schließlich nach jedem Versuch, etwas vorzuspielen, enttäuscht sind und psychologische Probleme bekommen.

Ein paar Fehler oder fehlende Noten werden beim Üben nicht wahrgenommen, und Ihre Einschätzung darüber, wie es während des Übens klingt, ist wahrscheinlich viel optimistischer als Ihre eigene Beurteilung, wenn Sie auf die gleiche Art vor einem Publikum gespielt hätten.
Nach dem Üben neigt man dazu, sich nur an die guten Teile zu erinnern, und nach der Aufführung neigt man dazu, sich nur an die Fehler zu erinnern.
Normalerweise ist man selbst sein schlimmster Kritiker; jeder Ausrutscher klingt für einen selbst viel schlimmer als für das Publikum.
Meistens bekommt das Publikum die Hälfte der Fehler nicht mit und vergißt die meisten, die es mitbekommt, nach kurzer Zeit wieder.
Das zwanglose Vorspielen geht wesentlich entspannter vonstatten, und es bietet eine einfache Möglichkeit, Ihnen schrittweise den Weg zum formellen Auftreten zu ebnen und Sie so auf Konzerte vorzubereiten.

Klassische Musik ist für das formlose Vorspielen nicht immer die beste Wahl.
Deshalb sollte jeder Klavierspieler Pop-Musik, Jazz, Cocktail-Musik, Musik aus \enquote{Fake Books} und das Improvisieren lernen.
Das sind einige der besten Möglichkeiten, für formelle Konzerte zu üben.
Mehr dazu in Abschnitt 23.


\subsubsection{Vorbereitung auf Konzerte}
\label{c1iii14f}

Auch wenn ein Schüler während des Übens perfekt spielen kann, kann er während des Konzerts alle Arten von Fehlern machen und mit der Musikalität ringen, wenn die Vorbereitung inkorrekt ist.
Die meisten Schüler üben in der Woche vor dem Konzert und insbesondere am Tag des Konzerts intuitiv hart und mit voller Geschwindigkeit.
Um das Konzert zu simulieren, stellen sie sich ein Publikum vor und spielen sich die Seele aus dem Leib, indem sie das Stück mehrmals vom Anfang bis zum Ende spielen.
Diese Übungsmethode ist die größte Ursache von Fehlern und schlechten Auftritten.
Die vielsagendste Bemerkung, die ich oft höre, ist: \enquote{Merkwürdig, ich habe den ganzen Morgen so gut gespielt, aber während des Konzerts habe ich Fehler gemacht, die ich während des Übens nicht gemacht habe!}
Für einen erfahrenen Lehrer ist das ein Schüler, der ohne Kontrolle übt und ohne Anleitung über die richtigen und falschen Methoden zur Vorbereitung auf Konzerte.

Lehrer, die jene Konzerte veranstalten, in denen die Schüler wunderbar spielen, halten ihre Schüler an der kurzen Leine und kontrollieren sorgsam deren Übungsablauf.
Wozu die ganze Aufregung?
Weil während eines Konzerts das am meisten angespannte Element das Gehirn ist, nicht der Spielmechanismus.
Und diese Anspannung kann mit keiner Art von simuliertem Auftritt nachgebildet werden.
Deshalb muß das Gehirn für einen einmaligen Auftritt ausgeruht und voll geladen sein; es darf nicht dadurch entladen sein, daß man sich die Seele aus dem Leib gespielt hat.
Alle Fehler haben ihren Ursprung im Gehirn.
Die ganze notwendige Information muß in geordneter Weise, ohne Durcheinander, im Gehirn gespeichert sein.
Deshalb spielen nicht richtig vorbereitete Schüler während eines Konzerts immer schlechter als während des Übens.
Wenn man mit voller Geschwindigkeit übt, dann wird ein großes Maß an Unordnung in das Gedächtnis gebracht.
Die Umgebung ist beim Konzert anders als beim Üben und kann sehr ablenkend sein.
Deshalb müssen Sie ein einfaches, fehlerfreies Gedächtnis des Stücks haben, das trotz aller zusätzlichen Ablenkungen abgerufen werden kann.
Darum ist es schwierig, dasselbe Stück zweimal am selben Tag aufzuführen, ja sogar an aufeinanderfolgenden Tagen.
Die zweite Aufführung ist ausnahmslos schlechter als die erste, obwohl man intuitiv erwarten würde, daß die zweite Aufführung besser wäre, weil man eine zusätzliche Erfahrung in der Aufführung des Stücks hat.
Wie sonst in diesem Abschnitt, ist diese Art von Anmerkungen nur auf Schüler anwendbar.
Professionelle Musiker sollten in der Lage sein, alles zu jeder Zeit beliebig oft vorzuspielen; diese Fertigkeit kommt von den ständigen Auftritten und dem ständigen Feilen an den richtigen Regeln zur Vorbereitung.

Durch Versuch und Irrtum haben erfahrene Lehrer funktionierende Übungsabläufe gefunden.
Die wichtigste Regel ist, das Maß an Übung am Konzerttag zu begrenzen, damit der Geist frisch bleibt.
Das Gehirn ist am Konzerttag völlig unempfänglich.
Es kann nur durcheinandergeraten.
Nur eine kleine Minderheit erfahrener Klavierspieler hat genügend \enquote{starke} musikalische Gehirne, um am Konzerttag etwas neues aufzunehmen.
Das gilt übrigens auch für Tests und Examen in der Schule.
Meistens wird man in einem Examen besser abschneiden, wenn man am Abend vorher ins Kino geht, als wenn man versucht, sich etwas einzutrichtern.
\textbf{Ein typischer empfohlener Übungsablauf beim Klavierspielen ist, einmal fast mit voller Geschwindigkeit zu spielen, dann einmal mit mittlerer Geschwindigkeit und zum Schluß einmal langsam.}
Das war's! Kein weiteres Üben!
Spielen Sie nie schneller als mit Konzertgeschwindigkeit.
Beachten Sie, wie kontraintuitiv das ist.
Da Eltern und Freunde fast immer intuitive Methoden benutzen, ist es für den Lehrer wichtig, sicherzustellen, daß jeder, der mit dem Schüler zu tun hat, die Regeln ebenfalls kennt.
Das gilt insbesondere bei den jüngeren Schülern.
Ansonsten werden die Schüler, trotz allem was der Lehrer sagt, wenn sie zum Konzert kommen, den ganzen Tag mit voller Geschwindigkeit geübt haben, weil ihre Eltern es so wollten.

Selbstverständlich ist das nur der Anfang.
Der Ablauf kann an die Umstände angepaßt werden.
Dieser Ablauf ist für den typischen Schüler und nicht für professionelle Künstler gedacht, die weitaus detailliertere Abläufe haben, die nicht nur von der Art der gespielten Musik abhängen, sondern auch von dem bestimmten Komponisten oder dem bestimmten zu spielenden Stück.
Klar muß, damit dieser Ablauf funktioniert, das Stück einige Zeit vor dem Auftritt fertig sein.
Jedoch ist dies sogar dann der beste Ablauf für den Konzerttag, wenn das Stück noch nicht perfektioniert wurde und mit mehr Übung verbessert werden kann.
Wenn Sie einen Fehler machen, von dem Sie wissen, daß er hartnäckig ist und der fast mit Sicherheit während des Konzerts auftreten wird, fischen Sie die paar Takte heraus, die den Fehler enthalten, und üben Sie diese mit den angemessenen Geschwindigkeiten (enden Sie immer mit langsamem Spielen), wobei Sie schnelles Spielen so weit wie möglich vermeiden.
Wenn Sie sich nicht sicher sind, daß das Stück völlig auswendiggelernt ist, spielen Sie es mehrere Male sehr langsam.
Die Wichtigkeit eines sicheren \hyperref[c1ii12mental]{mentalen Spielens} muß noch einmal betont werden - es ist der ultimative Test für das Gedächtnis und ob Sie zum Auftreten bereit sind.
Üben Sie das mentale Spielen mit jeder Geschwindigkeit und so oft Sie möchten; es kann auch ein nervöses Zittern beruhigen.

Vermeiden Sie auch extreme Anstrengungen, wie z.B. ein Fußballspiel oder etwas schweres zu heben oder zu schieben (wie z.B. einen Konzertflügel!).
Das kann plötzlich die Antwort Ihrer Muskeln auf ein Signal des Gehirns ändern, und Sie machen am Ende beim Spielen völlig unerwartete Fehler.
Selbstverständlich können leichte Aufwärmübungen, Dehnen, Gymnastik, Tai Chi, Yoga usw. sehr nützlich sein.

\textbf{Spielen Sie in der Woche vor dem Konzert immer mit mittlerer Geschwindigkeit und danach mit langsamer Geschwindigkeit, bevor Sie mit dem Üben aufhören}.
Sie können die langsame Geschwindigkeit durch die mittlere ersetzen, wenn Ihnen die Zeit knapp wird, das Stück besonders einfach ist oder wenn Sie ein erfahrenerer Künstler sind.
Übrigens ist diese Regel auf jede Übungseinheit anwendbar, aber sie ist vor einem Konzert besonders entscheidend.
Das langsame Spielen tilgt alle schlechten Angewohnheiten, die Sie eventuell angenommen haben, und stellt das entspannte Spielen wieder her.
Konzentrieren Sie sich deshalb während dieses mittleren bzw. langsamen Spielens auf die Entspannung.
Es gibt keine feste Zahl wie bei der halben Geschwindigkeit, um mittlere und langsame Geschwindigkeit zu definieren, obwohl die mittlere im allgemeinen ungefähr 3/4 der endgültigen Geschwindigkeit ist und die langsame ungefähr 1/2.
Allgemeiner gesagt: Mittlere Geschwindigkeit ist die Geschwindigkeit, mit der man bequem, entspannt und mit viel Zeitersparnis spielen kann.
Langsam ist die Geschwindigkeit, bei der Sie jeder einzelnen Note Beachtung schenken müssen.

Sie können bis zum letzten Tag vor dem Konzert an der Verbesserung des Stücks arbeiten - besonders an der musikalischen.
Aber während der letzten Woche ist es nicht zu empfehlen, neues Material hinzuzufügen oder das Stück zu ändern (z.B. den Fingersatz), obwohl Sie es als Trainingsexperiment versuchen könnten, um zu sehen, wie weit Sie sich treiben können.
In der Lage zu sein, während der letzten Woche etwas Neues hinzuzufügen, ist ein Zeichen, daß Sie starke Fähigkeiten zum Auftreten haben; tatsächlich ist es ein gutes Training für das Auftreten, absichtlich etwas auf die letzte Minute zu ändern.
Vermeiden Sie beim Arbeiten an einem langen Stück, wie z.B. einer Beethoven-Sonate, es viele Male ganz durchzuspielen.
Es ist am besten, es in kleine Abschnitte von wenigen Seiten zu zerteilen und diese Abschnitte zu üben.
HS zu üben ist ebenfalls eine ausgezeichnete Idee, weil jeder sich immer technisch verbessern kann.
Obwohl zu schnelles Spielen in der letzten Woche nicht empfehlenswert ist, können Sie mit jeder Geschwindigkeit HS üben.
Vermeiden Sie es, während dieser letzten Woche neue Stücke zu lernen.
Das bedeutet nicht, daß Sie auf die Konzertstücke beschränkt sind; Sie können weiterhin jedes Stück üben, das Sie zuvor gelernt haben.
Neue Stücke werden oft dazu führen, daß Sie neue Fertigkeiten erwerben, die die Art, wie Sie das Konzertstück spielen, beeinflussen oder ändern.
Im allgemeinen werden Sie es nicht merken, daß dies geschehen ist, bis Sie das Konzertstück spielen und sich fragen, wie sich ein paar neue Fehler eingeschlichen haben.

Machen Sie es sich zur Angewohnheit, Ihre Konzertstücke \enquote{\hyperref[c1iii6g]{kalt}} (ohne Aufwärmen) zu spielen, wenn Sie eine Übungseinheit beginnen.
Die Hände werden sich nach einem oder zwei Stücken erwärmen, so daß Sie eventuell die Reihenfolge der Konzertstücke bei jeder Übungseinheit ändern müssen, wenn Sie viele Stücke spielen.
\enquote{Kalt spielen} muß natürlich innerhalb eines vernünftigen Rahmens stattfinden.
Wenn die Finger von der Untätigkeit völlig träge sind, können Sie nicht, und sollten es auch nicht versuchen, schwieriges Material mit der vollen Geschwindigkeit spielen; es wird zu Streß und sogar Verletzungen führen.
Einige Stücke können nur gespielt werden, wenn die Hände völlig aufgewärmt sind,
insbesondere, wenn man sie musikalisch spielen möchte.
Die Schwierigkeit, musikalisch zu spielen, darf jedoch keine Entschuldigung dafür sein, nicht kalt zu spielen, weil in diesem Fall der Aufwand wichtiger als das Ergebnis ist.
Sie müssen herausfinden, welche Stücke Sie kalt mit voller Geschwindigkeit spielen können und welche nicht.
Verringern Sie die Geschwindigkeit so weit, daß Sie mit kalten Händen spielen können; Sie können aber stets mit der endgültigen Geschwindigkeit spielen, nachdem die Hände aufgewärmt sind. 

Üben Sie mehrere Tage vor dem Konzert nur die ersten paar Takte.
Wann immer Sie Zeit haben, tun Sie so, als ob der Moment des Konzerts wäre, und spielen Sie die ersten paar Takte. 
Wählen Sie die ersten zwei bis fünf Takte, und üben Sie jedesmal eine andere Anzahl.
Halten Sie nicht am Ende eines Taktes an, sondern spielen Sie immer bis zur Note des nächsten Takts.



<!-- c1iii14g.html -->

\subsubsection{Während des Konzerts}
\label{c1iii14g}

Die \hyperref[c1iii15]{Nervosität} ist im allgemeinen unmittelbar bevor man anfängt zu spielen am größten.
Haben Sie erst einmal angefangen, werden Sie so mit dem Spielen beschäftigt sein, daß die Nervosität normalerweise vergessen ist und weniger wird.
Dieses Wissen kann sehr beruhigend sein, so daß nichts falsch daran ist, mit dem Spielen anzufangen, sobald Sie sich für das Konzert an das Klavier gesetzt haben.
Einige verzögern das Starten, indem sie die Bank justieren oder etwas an der Kleidung richten, um genügend Zeit zu haben, das Anfangstempo usw. noch einmal mit Hilfe des \hyperref[c1ii12mental]{mentalen Spielens} zu prüfen.

Nehmen Sie nicht an, daß Sie keine Fehler machen werden; das wird Sie nur in zusätzliche Schwierigkeiten bringen, da Sie sonst auf verlorenem Posten stehen werden, wenn ein Fehler auftritt.
Seien Sie bereit, bei jedem Fehler richtig zu reagieren, oder wichtiger noch, einen drohenden Fehler vorherzusehen, den Sie vermeiden können.
Es ist erstaunlich, wie oft man einen drohenden Fehler erahnen kann, bevor er auftritt, besonders wenn man das mentale Spielen gut beherrscht.
Das Schlechteste, das die meisten Schüler tun, wenn sie einen Fehler machen oder wenn sie einen erwarten, ist, ängstlich zu werden und anzufangen langsamer und leiser zu spielen.
Das kann zur Katastrophe \textit{führen}.
Obwohl das \hyperref[c1iii6d]{Hand-Gedächtnis} nichts ist, von dem man abhängen möchte, ist das ein Zeitpunkt, an dem Sie einen Vorteil daraus ziehen können.
Das Hand-Gedächtnis hängt von der Gewohnheit und von Reizen ab - die Gewohnheit, viele Male geübt zu haben und die Reize durch die vorangegangenen Noten, die zu den nachfolgenden Noten führen.
Um das Hand-Gedächtnis zu verstärken, müssen Sie deshalb etwas schneller und lauter spielen; genau das Gegenteil von dem, was eine verängstigte Person während eines Konzerts tun würde (eine weitere kontraintuitive Situation!).
Das schnellere Spielen nutzt die Spielgewohnheit besser aus und läßt weniger Zeit dafür, einen falschen Muskel zu bewegen, der Sie aus den gewohnten Bahnen lenkt.
Das festere Spielen erhöht den Reiz für das Hand-Gedächtnis.
Nun sind schnelleres und lauteres Spielen während eines Konzerts angsteinflößende Dinge, die Sie deshalb zu Hause genauso wie alles andere üben sollten.
Lernen Sie, Fehler vorauszusehen und sie zu vermeiden, indem Sie diese Vermeidungsmethoden benutzen.
Eine andere Methode für das \enquote{durch Fehler hindurchspielen} ist, sicherzustellen, daß die Melodielinie nicht unterbrochen wird, auch wenn dadurch ein paar Noten der \enquote{Begleitung} ausgelassen werden.
Wenn Sie Übung darin haben, werden Sie es leichter finden als es klingt; am besten üben Sie es beim \hyperref[c1iii11]{Spielen vom Blatt}.
Eine weitere Methode ist, zumindest den Rhythmus zu halten.
Selbstverständlich ist das alles nicht notwendig, wenn Sie über ein sicheres mentales Spielen verfügen.

Falls Sie eine Gedächtnisblockade haben, versuchen Sie nicht, von dort anzufangen, wo Sie den Faden verloren haben, solange Sie nicht genau wissen, wie Sie anfangen müssen.
Fangen Sie bei einem vorangegangenen Abschnitt oder einem nachfolgenden Abschnitt an, den Sie gut kennen (vorzugsweise bei einem nachfolgenden Abschnitt, weil Fehler üblicherweise während eines Konzerts nicht korrigiert werden können und Sie deshalb denselben Fehler erneut machen).
\textbf{Ein sicheres mentales Spielen wird praktisch alle Gedächtnisblockaden eliminieren.}
Wenn Sie sich dafür entscheiden, den Teil mit der Gedächtnisblockade noch einmal zu spielen, dann spielen Sie ihn etwas schneller und lauter, nicht langsamer und leiser, weil das fast mit Sicherheit zu einer Wiederholung der Gedächtnisblockade führen wird.

In einer Konzerthalle mit guter Akustik wird der Schall des Klaviers vom Raum absorbiert, so daß man von seinem eigenen Spielen sehr wenig hört.
Es ist offensichtlich wichtig, vor der Veranstaltung auf dem Konzertklavier in der Konzerthalle zu proben.
Wenn bei einem Flügel der Notenständer aufgestellt ist, wird man sogar noch weniger hören; lassen Sie den Notenständer deshalb immer unten.
Wenn Sie Noten lesen müssen, dann legen Sie sie im Bereich der Stimmwirbel flach hin.


\subsubsection{Das ungewohnte Klavier}
\label{c1iii14h} 

Einige Schüler sind besorgt darüber, daß das Konzertklavier ein großer Flügel ist, während sie auf einem kleinen Klavier üben.
Zum Glück ist es leichter, auf einem großen Klavier zu spielen als auf einem kleinen.
Deshalb muß man sich beim typischen Schülerkonzert üblicherweise keine Gedanken über die unterschiedlichen Klaviere machen.
Größere Klaviere haben im allgemeinen eine bessere Mechanik, und sowohl lautere als auch leisere Töne können auf ihnen leichter erzeugt werden.
Vor allem sind Flügel leichter zu spielen als Klaviere, besonders bei schnellen, schwierigen Passagen.
Deshalb müssen Sie nur dann wegen des Klaviers besorgt sein, wenn das Konzertklavier entschieden minderwertiger als Ihr Übungsklavier ist.
Die schlechteste Situation ist, wenn Ihr Übungsklavier ein sehr guter Flügel ist, Sie aber auf einem qualitativ schlechten Klavier spielen müssen.
In diesem Fall wird es sehr schwierig sein, technisch schwierige Stücke auf dem minderwertigen Klavier zu spielen, und Sie werden dem eventuell dadurch Rechnung tragen müssen, daß Sie z.B. mit einem geringeren Tempo spielen, den Triller verkürzen oder verlangsamen usw.
Die Mechanik von Flügeln kann etwas schwerer als die von Klavieren sein, was einigen Anfängern Probleme bereiten kann.
Es ist immer ratsam, vor dem Konzert auf dem Konzertklavier zu üben.


\label{c1iii14Stimmung}

Ein weiterer wichtiger Faktor ist die \hyperref[c2_1]{Stimmung des Klaviers}.
Ein gut gestimmtes Klavier ist leichter zu spielen als ein verstimmtes.
Deshalb ist es eine gute Idee, das Konzertklavier direkt vor dem Konzert zu stimmen.
Im Gegensatz dazu ist es keine gute Idee, das Übungsklavier direkt vor dem Konzert zu stimmen, außer wenn es stark verstimmt ist.
Wenn das Konzertklavier verstimmt ist, ist es vielleicht am besten, ein wenig schneller und lauter zu spielen als Sie beabsichtigten.


\subsubsection{Nach dem Konzert}
\label{c1iii14i}

Gehen Sie nach dem Konzert die Ergebnisse durch, und ermitteln Sie Ihre Stärken und Schwächen, so daß Sie die Art und Weise des Übens und Ihrer Vorbereitungen auf die Konzerte verbessern können.
Einige wenige Schüler werden in der Lage sein, stets ohne hörbare Fehler zu spielen.
Die meisten anderen werden jedesmal wenn sie spielen ein paar Fehler machen.
Einige werden dazu neigen, auf das Klavier einzuhämmern, während andere schüchtern sind und zu leise spielen.
Es gibt ein Mittel gegen jedes Problem.
Diejenigen, die Fehler machen, haben wahrscheinlich noch nicht gelernt, ausreichend musikalisch zu spielen, und können fast immer nicht \hyperref[c1ii12mental]{in Gedanken spielen}.
Diejenigen, die in der Regel fehlerfrei spielen, haben ohne Ausnahme das mentale Spielen gelernt, egal ob sie es bewußt gebrauchen oder nicht.

Wie bereits an anderer Stelle gesagt, \textbf{ist es das Schwerste, mehrere Konzerte hintereinander zu spielen.
Wenn Sie es aber müssen, dann müssen Sie die Konzertstücke unmittelbar nach dem Konzert überholen.
Spielen Sie sie mit wenig oder keinem Ausdruck und mittlerer Geschwindigkeit, danach langsam.}
Wenn bestimmte Abschnitte oder Stücke während des Konzerts nicht zufriedenstellend waren, arbeiten Sie an diesen, aber nur in kleinen Abschnitten.
Wenn Sie mit voller Geschwindigkeit am Ausdruck arbeiten möchten, tun Sie das ebenfalls in kleinen Abschnitten.



<!-- c1iii15.html -->

\subsection{Ursachen und Kontrolle von Nervosität}
\label{c1iii15}
 
\textbf{Nervosität ist ein natürliches menschliches Gefühl wie Glücklichsein, Angst, Trauer usw.}
Nervosität entsteht aus der geistigen Wahrnehmung einer Situation, in der die Leistung entscheidend ist.
\textbf{Deshalb ist die Nervosität, wie alle Gefühle, eine leistungssteigernde Reaktion auf eine Situation.}
Glücklichsein fühlt sich gut an, weshalb wir versuchen, glückliche Situationen zu erzeugen, die uns helfen; Furcht hilft uns, Gefahren zu entfliehen, und Traurigkeit bringt uns dazu, schmerzliche Situationen zu vermeiden, was dazu führt, daß wir unsere Chancen zu überleben verbessern.
Nervosität läßt uns all unsere Energien auf die anstehende kritische Aufgabe konzentrieren und ist deshalb ein weiteres nützliches Überlebenswerkzeug der Evolution.
Die meisten Menschen haben eine Abneigung gegen die Nervosität, weil sie zu häufig von der Furcht zu versagen begleitet oder verursacht wird.
Obwohl die Nervosität für eine große Leistung erforderlich ist, muß man sie deshalb unter Kontrolle halten; man darf nicht zulassen, daß sie den Auftritt stört.
Die Geschichte der großen Künstler ist voller Berichte sowohl von sehr nervösen als auch von überhaupt nicht nervösen Künstlern.
Das zeigt, daß die bisherigen wissenschaftlichen, medizinischen oder psychologischen Untersuchungen des Phänomens  zu keinen praktischen Ergebnissen führten - sogar auf Konservatorien, bei denen das ein wichtiger Bestandteil des Lehrplans sein sollte.

Gefühle sind grundlegende, primitive, animalische Reaktionen, so etwas wie Instinkt, und sind nicht völlig rational.
Unter normalen Umständen leiten die Gefühle unsere täglichen Aktionen recht ordentlich.
\textbf{Unter extremen Bedingungen können die Gefühle jedoch außer Kontrolle geraten, und sie können dann zu einer Belastung werden.}
Klar, Gefühle waren dazu gedacht, nur unter normalen Umständen zu funktionieren.
So läßt z.B. die Furcht den Frosch flüchten, lange bevor ein Raubtier ihn fangen kann.
Wenn er jedoch in die Enge getrieben wird, erstarrt der Frosch vor Angst, und das läßt ihn für die Schlange zu einer leichteren Beute werden, als wenn ihn die überwältigende Furcht nicht gelähmt hätte.
Ebenso \textbf{ist die Nervosität normalerweise gemäßigt und hilft uns, eine wichtige Aufgabe besser zu bewältigen, als wenn wir gleichgültig wären.}
Unter extremen Bedingungen kann sie jedoch schlagartig außer Kontrolle geraten und unsere Leistung behindern.
Die Anforderung, ein schwieriges Pianosolo vor einem großen Publikum fehlerfrei aufzuführen, kann berechtigtermaßen als extreme Situation bezeichnet werden.
Es ist keine Überraschung, daß die Nervosität außer Kontrolle geraten kann, solange unser Name nicht Wolfgang oder Franz ist (für Frederic traf das offensichtlich nicht zu, da er ein Nervenbündel war und öffentliche Aufführungen nicht ausstehen konnte; in einem Salon fühlte er sich jedoch anscheinend wohler).
Obwohl Geigenspieler ebenfalls nervös werden, gerät dies jedoch nicht außer Kontrolle, wenn sie in einem Orchester spielen, weil die Bedingungen nicht so extrem wie bei Soloauftritten sind.
Kinder, die zuviel Angst davor haben, solo aufzutreten, haben fast immer Spaß daran, in einer Gruppe aufzutreten.
Das zeigt die vorrangige Wichtigkeit der mentalen Wahrnehmung der Situation.

\textbf{Klar ist der Weg, Nervosität zu kontrollieren, zunächst ihre Ursachen und ihre Form zu untersuchen und dann Methoden zu ihrer Kontrolle zu entwickeln, die auf diesem Wissen basieren.}
Da sie ein Gefühl ist, wird jede Methode zur Kontrolle von Gefühlen funktionieren.
Einige haben behauptet, daß unter ärztlicher Aufsicht Medikamente wie Inderal und Atenolol oder sogar Zantac zur Beruhigung der Nerven geeignet sind.\footnote{Beim \enquote{Griff in die Medikamentenkiste} ist äußerste Vorsicht geboten. Wenn überhaupt, dann sollten diese Mittel wirklich nur unter ärztlicher Aufsicht eingenommen werden! Inderal und Atenolol sind Beta-Blocker und somit z.B. zur Senkung von viel zu hohem Blutdruck gedacht; Zantac ist ein Histamin-H2-Blocker und wird z.B. zur Behandlung von Magen- und Zwölffingerdarmgeschwüren eingesetzt. Ich halte den Einsatz solcher Mittel zur Dämpfung von Nervosität bzw. Lampenfieber für übertrieben und bedenklich.
Ein gesundes Maß Lampenfieber ist der Leistung beim Auftritt durchaus förderlich, und alles weitere läßt sich - wie im folgenden beschrieben - auch ohne Chemie gut in den Griff bekommen.}
Umgekehrt kann man die Nervosität verschlimmern, indem man Kaffee oder Tee trinkt, nicht genug Schlaf bekommt oder bestimmte Medikamente gegen Erkältung einnimmt.
Gefühle können auch durch Psychologie, Training oder Konditionierung kontrolliert werden.
Wissen ist das effektivste Mittel zur Kontrolle.
Erfahrene Schlangenbeschwörer leiden z.B. aufgrund ihres Wissens über Schlangen nicht unter einem der Gefühle, die uns überkommen würden, wenn wir in die Nähe einer Giftschlange kämen.

\textbf{Zu dem Zeitpunkt, an dem die Nervosität zum Problem wird, ist sie üblicherweise ein zusammengesetztes Gefühl, das schlagartig außer Kontrolle gerät.}
Zusätzlich zur Nervosität kommen noch andere Gefühle wie Furcht und Sorge hinzu.
Ein Mangel an Verständnis der Nervosität erzeugt ebenfalls Furcht wegen der Furcht vor dem Unbekannten.
Deshalb kann das bloße Wissen, was Lampenfieber ist, durch die Reduzierung der Furcht vor dem Unbekannten ein beruhigender Faktor sein.


\label{ng}

Wie gerät die Nervosität außer Kontrolle, und gibt es Wege, dies zu verhindern?
Eine Möglichkeit, diese Frage anzugehen, ist, einige Prinzipien der Grundlagenforschung zu betrachten.
\textbf{Praktisch alles in unserem Universum wächst durch einen Prozeß der als Kernbildung-Wachstum-Mechanismus (nucleation-growth = NG) bekannt ist.
Die NG-Theorie besagt, daß sich ein Objekt in zwei Stufen bildet: Kernbildung und Wachstum.}
Diese Theorie wurde populär und nützlich, weil es tatsächlich die Art ist, in der die meisten Objekte in unserem Universum gebildet werden, von Regentropfen bis zu Städten, Sternen, Menschen usw.
\textbf{Die beiden Schlüsselelemente der NG-Theorie sind:}

\begin{enumerate}[label={\arabic*.}] 
\item \textbf{Kernbildung}<br>
Es bilden sich ständig Kerne und verschwinden welche.
Es gibt jedoch etwas, das ein kritischer Kern genannt wird, der stabil wird, wenn er sich gebildet hat - er verschwindet nicht.
Im allgemeinen bildet sich der kritische Kern nicht, solange es keine Übersättigung des Materials gibt, das sich verbindet, um ihn zu bilden.

\item \textbf{Wachstum}<br>
Damit das Objekt zu seiner endgültigen Größe anwächst, braucht der kritische Kern einen Wachstumsmechanismus, durch den seine Größe zunimmt.


 \end{enumerate}
Im allgemeinen unterscheidet sich der Wachstumsmechanismus völlig von dem Mechanismus der Kernbildung.
Ein interessanter Aspekt der Kernbildung ist, daß es immer eine Schwelle zur Kernbildung gibt - ansonsten hätten sich bereits vor langer Zeit alle Kerne gebildet.
Die Größenänderung verläuft in beiden Richtungen: Sie kann positiv oder negativ sein.

Lassen Sie uns ein Beispiel untersuchen: Regen.
Regen tritt auf, wenn Wassertropfen kritische Kerne in Luft bilden, die mit Wasserdampf übersättigt ist (relative Feuchtigkeit größer als 100\%).
Gegen die oft falsch zitierte \enquote{wissenschaftliche Wahrheit}, daß die relative Luftfeuchtigkeit nie 100\% überschreitet, wird ständig von der Natur verstoßen, weil diese \enquote{Wahrheit} nur unter Gleichgewichtsbedingungen gültig ist, wenn sich alle Kräfte ausgleichen konnten.
Die Natur ist fast immer dynamisch, und sie kann weit vom Gleichgewicht entfernt sein.
Das geschieht z.B., wenn die Luft sich schnell abkühlt und mit Wasserdampf übersättigt wird.
Sogar ohne Übersättigung bildet Wasserdampf dauernd Wassertropfen, aber diese verdunsten, bevor sie kritische Kerne bilden können.
Bei Übersättigung können sich plötzlich kritische Kerne bilden, besonders wenn Wasser anziehende Staubpartikel in der Luft sind oder bei einer Druckstörung wie z.B. ein Donnerschlag, der die Moleküle näher zusammenbringt und so die Übersättigung steigert.
Die Luft, die mit kritischen Kernen gefüllt ist, nennen wir Wolken oder Nebel.
Wenn die Bildung der Wolke die Übersättigung auf Null reduziert, wird eine stabile Wolke gebildet; wenn nicht, wachsen die Kerne weiter, um die Übersättigung zu reduzieren.
Die Kerne können durch andere Mechanismen wachsen.
Das ist die Wachstumsphase des NG-Prozesses.
Die Kerne können aneinanderstoßen und sich zusammenballen, oder sie beginnen zu fallen und treffen andere Wassermoleküle und Kerne, bis sich Regentropfen bilden.

Wenden wir die NG-Theorie auf die Nervosität an.
Im täglichen Leben kommt und geht das Gefühl der Nervosität, ohne etwas Ernstes zu werden.
In einer ungewöhnlichen Situation, wie kurz vor einem Auftritt, gibt es jedoch eine Übersättigung mit Faktoren, die Nervosität verursachen: Sie müssen fehlerfrei vorspielen, Sie haben nicht genügend Zeit gehabt, das Stück zu üben, es wartet ein großes Publikum da draußen auf Sie, usw.
Das mag immer noch keinerlei Probleme bereiten, weil es bei der Nervosität natürliche Barrieren für die Kernbildung gibt, wie den Fluß des Adrenalins, die Selbstsicherheit oder einfach einen Mangel an Erkennen, daß man nervös ist, oder Sie sind vielleicht zu sehr damit beschäftigt, sich endgültig auf den Auftritt vorzubereiten.
Aber dann sagt ein anderer Künstler \enquote{Mann, ich habe vielleicht Schmetterlinge im Bauch!}, und Sie fühlen plötzlich einen Kloß im Hals und merken, daß Sie nervös sind - der kritische Kern hat sich gebildet!
Das mag immer noch nicht so schlimm sein, bis Sie anfangen sich zu sorgen, daß Ihr Stück vielleicht noch nicht bereit zur Aufführung ist oder daß die Nervosität anfängt, das Spielen zu stören - diese Sorgen lassen die Nervosität anwachsen.
Das sind genau die Prozesse, die durch die NG-Theorie beschrieben werden.
Das schöne an jeder wissenschaftlichen Theorie ist, daß sie nicht nur den Prozeß detailliert beschreibt, sondern auch Lösungen für Probleme anbietet.
Wie hilft uns also die NG-Theorie?

Wir können die Nervosität im Kernbildungsstadium angreifen; wenn wir die Kernbildung verhindern können, wird sich nie ein kritischer Kern bilden können.
Ein bloßes Verzögern der Kernbildung wird hilfreich sein, weil dies die zum Wachsen verfügbare Zeit reduziert.
Leichtere Stücke zu spielen, wird die Übersättigung mit Sorge reduzieren.
Simulierte Konzerte verleihen Ihnen mehr Erfahrung und Selbstsicherheit; beides wird die Angst vor dem Unbekannten verringern.
Im allgemeinen muß man ein Stück dreimal oder öfter vorführen, bevor man weiß, ob man es erfolgreich aufführen kann oder nicht; deshalb ist es hilfreich, Stücke zu spielen, die man mehrmals vorgeführt hat.
\textbf{Die Nervosität ist im allgemeinen vor einem Auftritt am schlimmsten; haben Sie erst einmal angefangen zu spielen, sind Sie so mit den bevorstehenden Aufgaben beschäftigt, daß Sie keine Zeit haben, sich länger mit der Nervosität zu befassen, und der Wachstumsfaktor somit reduziert wird.}
Dieses Wissen hilft, weil es die Furcht abschwächt, daß während des Auftritts alles schlimmer wird.
Sich nicht länger mit der Nervosität zu befassen, ist eine weitere Möglichkeit, sowohl die Kernbildung zu verzögern als auch die Wachstumsphase zu verlangsamen.
Deshalb ist es eine gute Idee, sich selbst beschäftigt zu halten, während man auf den Anfang des Konzerts wartet.\footnote{Sie können sich zusätzlich mit einer \hyperref[c1ii21uebung]{einfachen Atemübung} entspannen.}
\textbf{Das \hyperref[c1ii12mental]{mentale Spielen} ist nützlich, weil Sie gleichzeitig Ihr Gedächtnis prüfen und sich selbst beschäftigt halten können; deshalb ist es das wichtigste Werkzeug zur Vermeidung und Verzögerung der Kernbildung und zur Reduzierung des Wachstums.}
Sehen Sie dazu in den Abschnitten \hyperref[c1iii16c]{16c} und \hyperref[c1iii16d]{16d} einige Vorschläge dafür, wie Lehrer ein Auftrittstraining zur Verfügung stellen können.

Bei einem wichtigen Konzert ist das Vermeiden der Kernbildung wahrscheinlich nicht möglich.
Deshalb sollten wir auch über Wege zur Unterbindung des Wachstums nachdenken.
Da die Nervosität im allgemeinen geringer wird, nachdem der Auftritt beginnt, kann dieses Wissen dazu benutzt werden, die Sorge zu reduzieren und somit die Nervosität.
Das kann sich selbst verstärken, und wenn Sie sich sicherer fühlen, kann sich die Nervosität oftmals völlig auflösen, wenn Sie sie unterhalb des kritischen Kerns reduzieren können.
Weitere wichtige Faktoren sind die geistige Haltung und die Vorbereitung.
Ein Auftritt ist immer ein interaktiver Prozeß zwischen Ihnen selbst und dem Publikum.
Musikalisch zu spielen ist, natürlich, immer die Antwort - wenn Sie Ihr komplettes Gehirn in die Aufgabe Musik zu erzeugen einbeziehen können, bleibt nur sehr wenig Kapazität dafür übrig, sich um die Nervosität zu sorgen.
Das sind alles Maßnahmen, die das Anwachsen der Nervosität reduzieren.

\textbf{Es ist - besonders bei Kindern, da sie leichter langfristige psychologische Schäden erleiden können - keine gute Idee, so zu tun, als ob die Nervosität nicht existieren würde.}
Kinder sind clever, und sie können diese Verstellung leicht durchschauen, und die Notwendigkeit, mit der Verstellung zu spielen, kann den Streß nur verstärken.
Deshalb ist ein Auftrittstraining, in dem offen über Nervosität gesprochen wird, so wichtig.
Im Fall von jungen Schülern müssen ihre Eltern und Freunde, die das Konzert besuchen, ebenfalls Bescheid wissen.
Sätze wie \enquote{Ich hoffe, Du bist nicht nervös!}, oder \enquote{Wie kannst Du auftreten, ohne nervös zu sein?}, führen fast mit Sicherheit zu Kernbildung und Wachstum.
Andererseits ist es jedoch auch unverantwortlich, die Nervosität völlig zu ignorieren und Kinder ohne Auftrittstraining in den Auftritt zu schicken, und kann sogar zu irreparablen psychologischen Schäden führen.

\textbf{Die richtige geistige Haltung zu entwickeln, ist die beste Möglichkeit, das Lampenfieber zu kontrollieren.}
Wenn Sie zu der Auffassung gelangen können, daß aufzutreten die wundervolle Erfahrung ist, Musik für andere zu machen, und die richtigen Reaktionen für den Fall entwickeln, daß Sie Fehler machen, dann wird Nervosität kein Problem sein.
Es ist z.B. ein großer Unterschied, einen Fehler mit Humor zu nehmen bzw. leicht darüber hinwegzukommen oder den Fehler wie eine Katastrophe erscheinen zu lassen, die den ganzen Auftritt verdirbt.
Das Auftrittstraining muß Lektionen über die Reaktion auf verschiedene Umstände beinhalten.
Deshalb ist es so wichtig, früh in der Karriere des Schülers sehr leichte Stücke zu spielen, die ohne Nervosität aufgeführt werden können; eine einzige solche Erfahrung kann der Beweis dafür sein, daß es möglich ist, ohne Nervosität aufzutreten.
Diese eine Erfahrung kann Ihr Verhalten bei Auftritten für den Rest Ihres Lebens beeinflussen.
Um einen solch fehlerfreien Auftritt zu garantieren, entwickeln Sie am besten ein sicheres \hyperref[c1ii12mental]{mentales Spielen}, das Sie dazu befähigt, von jeder beliebigen Note des Stücks aus mit dem Spielen zu beginnen, der Musik immer voraus zu sein, die Musikalität in Ihrem Geist zu erzeugen, ein \hyperref[c1iii12]{absolutes Gehör} zu entwickeln, über Fehler hinwegzukommen oder sie zu kaschieren, jeden Tag in Gedanken Klavier zu spielen, d.h. jeden Teil des Stücks jederzeit und überall zu üben, usw.; all das zu erreichen, wird Ihnen die Zuversicht eines vollendeten Musikers geben.
Das Publikum wird sicherlich der Meinung sein, daß es mit einem großen Talent zusammengekommen ist.

Um es zusammenzufassen: Lampenfieber ist eine Form der Nervosität, die in einer Spirale außer Kontrolle geraten ist.
Ein gewisses Maß an Nervosität ist normal und nützlich.
Man kann die Nervosität minimieren, indem man sich beschäftigt hält und somit ihre Kernbildung verzögert und indem man ihr Wachstum durch musikalisches Spielen reduziert; \hyperref[c1ii12mental]{mentales Spielen} ist dafür das nützlichste Mittel.
Deshalb macht es keinen Sinn, und ist ein Fehler, zu fragen: \enquote{Wirst Du nervös, wenn Du auftrittst?}
Jeder wird es und sollte es auch.
Wir müssen die Nervosität nur eindämmen, so daß sie nicht jenseits unserer Kontrolle anwächst.
\textbf{Zu erkennen, daß ein gewisses Maß an Nervosität normal ist, ist die beste Ausgangslage, um zu lernen, wie man sie kontrolliert.}
Natürlich gibt es einen großen Bereich unterschiedlicher Menschen: von denjenigen, die nicht nervös werden, bis zu denjenigen, die schrecklich unter Lampenfieber leiden.
Am besten begegnet man der Nervosität mit Ehrlichkeit - wir müssen ihre Wirkung auf jeden einzelnen zugeben und entsprechend mit ihr umgehen.
Vertrauen in Ihre Fähigkeiten für das Auftreten zu erlangen, kann üblicherweise die Nervosität eliminieren, und die Kunst des mentalen Spielens zu perfektionieren ist der einzige Weg, wirklich ein solches Vertrauen zu erreichen.



<!-- c1iii16.html -->

\subsection{Unterrichten}
\label{c1iii16}

\subsubsection{Lehrer}
\label{c1iii16a}

Klavierspielen zu unterrichten ist ein schwieriger Beruf, weil praktisch alles, was man zu tun versucht, im Widerspruch zu etwas anderem steht, das getan werden sollte.
Wenn man das \hyperref[c1iii11]{Blattspiel} lehrt, ist der Schüler am Ende vielleicht unfähig \hyperref[c1iii6]{auswendig zu lernen}.
Wenn man \hyperref[c1ii17]{langsames, genaues Spielen} lehrt, erwirbt der Schüler eventuell innerhalb eines vernünftigen Zeitraums nicht genügend Technik.
Wenn man sie zu schnell antreibt, vergessen sie vielleicht alles über die \hyperref[c1ii14]{Entspannung}.
Wenn man sich auf die Technik konzentriert, könnte der Schüler das \hyperref[c1iii14d]{musikalische Spielen} aus den Augen verlieren.
Man muß ein System entwickeln, das erfolgreich durch all diese gegensätzlichen Arten von Anforderungen navigiert und immer noch die individuellen Wünsche und Bedürfnisse jedes einzelnen Schülers befriedigt.
Bevor dieses Buch geschrieben wurde, gab es kein Standardlehrbuch, die Lehrer mußten, wenn Sie ihre Laufbahn begannen, erst ihr eigenes Lehrsystem entwickeln, und sie hatten dafür nur sehr wenige Anhaltspunkte.
\textbf{Klavierspielen zu unterrichten ist eine herkulische Aufgabe, die nichts für Hasenfüße ist.}

\textbf{Historisch gesehen lassen sich Lehrer in mindestens drei Kategorien einteilen: Lehrer für Anfänger, Mittelstufenschüler und Fortgeschrittene.}
Der erfolgreichste Ansatz bezieht eine Gruppe von Lehrern ein, die aus allen drei Kategorien zusammengesetzt ist; die Lehrer sind so koordiniert, daß ihre Art zu lehren zueinander paßt und der passende Schüler zum passenden Lehrer geleitet wird.
Ohne eine solche Zusammenarbeit weigerten sich viele Lehrer für fortgeschrittene Schüler, Schüler von bestimmten Lehrern zu nehmen, weil letztere \enquote{nicht die richtigen Grundlagen lehren}.
Das sollte nicht passieren, wenn die Grundlagen standardisiert sind.
Das letzte, was ein Lehrer für Fortgeschrittene möchte, ist ein Schüler, dem am Anfang lauter \enquote{falsche} Methoden beigebracht wurden.
Somit würde eine Standardisierung mittels eines Lehrbuchs, wie diesem hier, solche Probleme lösen.


\subsubsection{Kinder unterrichten, Eltern einbeziehen}
\label{c1iii16b}

\textbf{Kinder sollten im Alter von zwei bis acht Jahren darauf getestet werden, ob sie bereit für den Klavierunterricht sind.
Die ersten Unterrichtsstunden für Anfänger, besonders für junge Kinder, die weniger als sieben Jahre alt sind, sollten kurz sein, höchstens 10 oder 15 Minuten.}
Verlängern Sie die Unterrichtszeit nur, wenn sich ihre Aufmerksamkeitsspanne und Ausdauer steigert.
Wenn mehr Zeit notwendig ist, teilen Sie den Unterricht in mehrere Einheiten mit Pausen dazwischen (\enquote{Keks-Zeit} o.ä.) auf.
Dieselben Regeln gelten für die Übungszeiten zu Hause.
Man kann in 10 Minuten eine Menge unterrichten; es ist besser, wenn man jeden zweiten Tag für 15 Minuten unterrichtet wird (d.h. dreimal pro Woche), als wenn man nur einmal in der Woche für eine Stunde oder länger Unterricht erhält.
Das gilt für alle Altersstufen, obwohl die Zeit zwischen den Unterrichtsstunden mit zunehmenden Alter und zunehmender Fertigkeitsstufe zunimmt.

Es ist für Kinder wichtig, sich Aufnahmen anzuhören.
Sie können sich in jedem Alter Chopin anhören und spielen.
Sie sollten sich auch \hyperref[c1iii13]{Aufnahmen ihres eigenen Spielens} anhören; sonst verstehen sie vielleicht nicht, warum Sie ihre Fehler kritisieren.
Geben Sie ihnen keine Musik, nur weil sie klassisch ist oder von Bach geschrieben wurde.
Spielen Sie, was Ihnen und den Kindern gefällt.

Kinder entwickeln sich sowohl körperlich als auch geistig in Schüben, und sie können nur das lernen, wofür sie geistig reif genug sind es zu lernen.
Mit anderen Worten: Man kann ihnen nicht etwas beibringen, solange sie nicht dafür bereit sind.
\textbf{Deshalb muß ein Teil des Unterrichtens aus einem ständigen Testen des Grades ihrer Reife bestehen:
Tonhöhe, \hyperref[c1iii1b]{Rhythmus}, \hyperref[c1iii12]{absolutes Gehör}, \hyperref[c1iii11]{vom Blatt spielen}, Kontrolle der Finger, Aufmerksamkeitsspanne, das Interesse an der Musik, welches Instrument das beste ist usw.}
Auf der anderen Seite sind die meisten Kinder für viel mehr Dinge bereit als den meisten Erwachsenen bewußt ist, und wenn sie bereit sind, ist der Himmel die Grenze.
Deshalb ist es auch ein Fehler, anzunehmen, daß alle Kinder ständig als Kinder zu behandeln sind.
Sie können in vielerlei Hinsicht erstaunlich entwickelt sein, und sie als Kinder zu behandeln (z.B. indem man sie nur \enquote{Kinderlieder} hören läßt), hält sie nur zurück und beraubt sie der Möglichkeit, ihr volles Potential auszuschöpfen.
Kinderlieder existieren nur in der Vorstellung der Erwachsenen und richten im allgemeinen mehr Schaden an als sie nutzen.

Die Entwicklung des Gehirns und des Körpers kann mit sehr unterschiedlichen Raten erfolgen.
\textbf{Das Gehirn ist dem Körper im allgemeinen weit voraus.
Wegen dieses körperlichen Rückstands nehmen zu viele Eltern an, daß die Entwicklung des Gehirns ebenfalls langsam ist.}
Es ist wichtig, das Gehirn zu testen, seine Entwicklung zu fördern und die körperliche Entwicklung nicht die Entwicklung des Gehirns verlangsamen zu lassen.
Das ist besonders deshalb wichtig, weil das Gehirn die Entwicklung des Körpers beschleunigen kann.
Sprache, Logik und Musik, sowie optische Reize, sind für die Entwicklung des Gehirns am wichtigsten.

\textbf{Mindestens während der ersten zwei Jahre des Unterrichts (bei Kindern länger) müssen Lehrer darauf bestehen, daß die Eltern am Lehr- bzw. Lernprozeß teilhaben.}
Die erste Aufgabe der Eltern ist, die Methoden zu verstehen, die der Lehrer lehrt.\textbf{Da so viele Übungsmethoden und Abläufe zur \hyperref[c1iii14]{Vorbereitung auf Konzerte} kontraintuitiv sind, müssen die Eltern mit ihnen vertraut sein, so daß sie nicht nur dabei helfen können, die Schüler zu leiten, sondern es auch vermeiden, den Anweisungen des Lehrers zu widersprechen.}
Wenn die Eltern nicht am Unterricht teilnehmen, werden sie nach ein paar Lektionen zurückfallen und können sogar zum Hindernis für die Entwicklung des Kindes werden.
Die Eltern müssen an der Entscheidung beteiligt sein, wie lange der Schüler täglich übt, da sie am besten mit all den Zeitanforderungen des Schülers vertraut sind.
Die Eltern kennen auch die endgültigen Ziele des Schülers am besten - ist der Unterricht nur für das Spielen in der Freizeit gedacht oder um zu viel höheren Stufen zu gelangen?
Welche Arten von Musik möchte der Schüler am Ende spielen?
Anfänger benötigen zu Hause immer Hilfe, beim Herausarbeiten des optimalen Ablaufs für das tägliche Üben genauso wie beim Einhalten des wöchentlichen Pensums.
Wenn der Unterricht angefangen hat, ist es erstaunlich, wie oft die Lehrer die Hilfe der Eltern benötigen: wo und wie die Noten gekauft werden, wie oft das Klavier gestimmt wird oder wann man auf ein besseres Klavier umsteigen soll usw.
Die Lehrer und Eltern müssen darin übereinstimmen, wie schnell die Schüler lernen sollen und daran arbeiten, diese Lernrate zu erreichen.
Die Eltern müssen über die Stärken und Schwächen des Schülers informiert sein, damit sie in der Lage sind, ihre Erwartungen und Pläne damit in Einklang zu bringen, was erreichbar ist und was nicht.
\textbf{Am wichtigsten ist, daß es die Aufgabe der Eltern ist, den Lehrer auszuwählen und die richtige Entscheidung darüber zu fällen, wann der richtige Zeitpunkt ist, den Lehrer zu wechseln.}

Dieses Buch sollte sowohl dem Schüler als auch den Eltern als Lehrbuch dienen.
Das spart dem Lehrer sehr viel Zeit, und der Lehrer kann sich dann darauf konzentrieren, die Fertigkeiten zu demonstrieren und Musik zu lehren.
Eltern müssen dieses Buch lesen, damit sie nicht mit den Lehrmethoden des Lehrers in Konflikt geraten.

Schüler brauchen sehr viel Hilfe von ihren Eltern, und die Art der Hilfe ändert sich mit dem Alter.
Wenn sie jung sind, brauchen die Schüler ständige Hilfe bei den täglichen Übungsabläufen: Die Eltern müssen überwachen, daß sie korrekt üben und den Anweisungen des Lehrers folgen.
Es ist in dieser Phase am wichtigsten, korrekte Übungsgewohnheiten zu etablieren.
\textbf{Die Eltern müssen sicherstellen, daß die Schüler es sich während des Übens zur Gewohnheit machen, durch Fehler hindurchzuspielen statt zurückzugehen, was eine Gewohnheit zu stottern erzeugen und den Schüler anfällig für Fehler während der Auftritte machen würde.}
Die meisten Kinder werden die Anweisungen des Lehrers, die während ihrer Unterrichtsstunden eilig gegeben wurden, nicht verstehen; die Eltern können diese Anweisungen eher verstehen.
Wenn die Schüler Fortschritte machen, brauchen Sie eine Rückmeldung, ob sie musikalisch spielen, ob ihr Tempo und ihr Rhythmus genau sind oder ob sie ein Metronom benutzen müssen und ob sie aufhören sollten zu üben und anfangen, sich Aufnahmen anzuhören.

Die geistige Entwicklung ist der Hauptgrund, warum man Kinder klassische Musik hören lassen sollte - der \enquote{Mozart-Effekt}.
Die Argumentation ist ungefähr folgende:
Nehmen Sie an, der durchschnittliche Elternteil hat eine durchschnittliche Intelligenz; dann gibt es eine 50-prozentige Wahrscheinlichkeit, daß das Kind intelligenter als die Eltern ist.
D.h., daß die Eltern nicht auf derselben intellektuellen Stufe wie ihr Baby konkurrieren können!
Wie sollen nun Eltern einem Baby Musik lehren, dessen musikalisches Gehirn sich schnell auf eine Stufe entwickeln kann, die viel höher ist als die seiner Eltern?
Indem man es die großen Klassiker hören läßt!
Lassen Sie es direkt mit Mozart, Chopin usw. reden und von ihnen lernen.
Musik ist eine universelle Sprache; anders als diese verrückten Erwachsenensprachen, die wir sprechen, ist Musik angeboren, so daß Babys mit Musik kommunizieren können, lange bevor sie \enquote{dada} sagen können.
Deshalb kann klassische Musik das Gehirn eines Babys lange bevor die Eltern mit ihm auf niedrigster Ebene kommunizieren können stimulieren.
Und diese Kommunikationen werden auf den Stufen von genialen Komponisten geführt, etwas, von dem wenige Eltern hoffen können, daß sie dazu in der Lage sind!

\textbf{Wie unterrichten Sie ihr Kind?}
Wir befassen uns hier mit der musikalischen Entwicklung und der des Gehirns.
Die Entwicklung des Gehirns ist bereits lange vor der Geburt wichtig.
Deshalb muß die Mutter für eine möglichst streßfreie Umgebung sorgen und auf eine ausgewogene Ernährung achten, darf nicht rauchen, nicht exzessiv Alkohol trinken usw.
Nach der Geburt ist anerkanntermaßen Muttermilch die beste Ernährung.\footnote{Das ist ein schwieriges Thema. Stichworte sind z.B.: Aufbau des Immunsystems, soziale Bindung, gesamte Dauer des Stillens, Belastung mit Schadstoffen.}
Einige Frauen mit kleinen Brüsten fürchten, daß sie nicht genug Milch produzieren könnten, aber diese Furcht ist unbegründet.
Alle Frauen haben dieselbe Zahl von Milchdrüsen; der Unterschied in der Brustgröße wird nur durch die Abweichung der in der Brust gespeicherten Fettmenge\footnote{bzw. implantierten Silikonmenge} verursacht.
Der wichtige Faktor beim Stillen ist, daß regelmäßig und mit beiden Brüsten zu gleichen Anteilen gestillt wird - jede Unterbrechung kann die Milchproduktion in der einen Brust stoppen.
Babys gedeihen in einer \enquote{normalen} Umgebung am besten; der Raum muß nicht besonders ruhig sein, während das Baby schläft (das erzeugt unruhige Schläfer, die nicht genug Schlaf bekommen, wenn irgendwelche Geräusche vorhanden sind); es gibt sogar einige Argumente dafür, einen gewissen Geräuschpegel im Zimmer des Babys zu haben, um bessere Schlafgewohnheiten zu fördern.
Babys sollten an normale Temperaturschwankungen gewöhnt werden - man muß sie nicht mit zusätzlichen Decken zudecken oder ihnen mehr Kleidungsstücke anziehen als einem Erwachsenen.
Babys können jede Stimulation gebrauchen, die Sie ihnen geben können; die hauptsächlichen sind akustisch, visuell, Geschmack, Geruch, sowie der Druck und die Temperatur bei einer Berührung.
So ist das Herumtragen eines Babys sehr gut für die Reizung der Sinne zur Entwicklung des Gehirns; berühren Sie das Baby überall, und sorgen Sie für viele visuelle und akustische Reize.
Füttern Sie sie so früh wie möglich mit Essen mit so vielen unterschiedlichen Gerüchen und soviel unterschiedlichem Geschmack wie möglich.
Berichten zufolge hat ein Baby bei der Geburt mehr Hirnzellen als ein Erwachsener, obwohl das Gehirnvolumen nur ein Viertel dessen von Erwachsenen beträgt.
Die Stimulation läßt einige Zellen wachsen, und der Mangel an Reizen läßt andere verkümmern und verschwinden.

Zum Unterrichten von Babys ist der wichtigste Schritt, laufend zu testen, was sie zu lernen bereit sind.
Nicht alle Babys werden zu Pianisten, obwohl sie in diesem Stadium praktisch zu jedem Talent hingeführt werden können, und Eltern sind am besten dafür gerüstet, ihre Kinder in eine Karriere zu leiten, für die sie selbst über das Fachwissen verfügen.
\textbf{Babys können sofort nach der Geburt hören.}
Viele Krankenhäuser untersuchen Babys gleich nach der Geburt, um hörbehinderte Babys zu erkennen, die sofort eine spezielle Behandlung benötigen.
Da hörbehinderte Babys keine Klangreize erhalten, wird sich die Entwicklung ihres Gehirns verlangsamen; das ist ein weiterer Beweis dafür, daß Musik der Entwicklung des Gehirns förderlich sein kann.
Bei Babys ist das Gedächtnis für Geräusche von außen fast leer.
Deshalb ist jedes Geräusch, das in diesem Stadium gehört wird, etwas besonderes, und alle nachfolgenden Geräusche werden mit diesen anfänglichen Geräuschen verglichen.
Zusätzlich benutzen Babys (der meisten Spezies, nicht nur menschliche) Geräusche, um die Eltern zu erkennen und sich an sie zu binden (üblicherweise an die Mutter).
Von allen Klangeigenschaften, die das Baby für dieses Erkennen benutzt, ist die absolute Tonhöhe wahrscheinlich eine Haupteigenschaft.
Diese Überlegungen erklären, warum fast jedes Kind sich so leicht ein \hyperref[c1iii12]{absolutes Gehör} aneignen kann.
Einige Eltern setzen ihre Babys bereits vor der Geburt Musik aus, um die Entwicklung des Babys zu beschleunigen, aber ich frage mich, ob das für ein absolutes Gehör hilfreich ist, weil die Schallgeschwindigkeit im Fruchtwasser anders ist als in der Luft, was zu einer scheinbar unterschiedlichen Frequenz führt.
Deshalb kann diese Praxis eventuell das absolute Gehör verwirren, wenn sie denn überhaupt funktioniert.
Um ein absolutes Gehör aufzubauen ist ein elektronisches Klavier besser als ein akustisches, weil es immer richtig gestimmt ist.

\textbf{Praktisch jeder Musiker, Athlet usw. von Weltklasse hatte Eltern\footnote{oder andere Förderer}, die ihn bereits in frühen Jahren unterrichteten}; d.h. \enquote{Wunderkinder} werden erzeugt und nicht geboren, und Eltern\footnote{s.o.} üben einen größeren Einfluß auf das Erzeugen von \enquote{Wunderkindern} aus als Lehrer.
Testen Sie das Kind hinsichtlich Gehör, \hyperref[c1iii1b]{Rhythmus} (in die Hände klatschen), Tonhöhe (singen), Kontrolle der Bewegungsabläufe, Aufmerksamskeitsspanne, was sie interessiert usw.
Sobald sie bereit sind (laufen, sprechen, Musik usw.), muß man sie unterrichten.
Babys zu unterrichten, ist etwas anderes, als Erwachsene zu unterrichten.
Erwachsene müssen unterrichtet werden; bei jungen Kindern muß man nur dem Gehirn das Konzept vermitteln und dann eine unterstützende Umgebung zur Verfügung stellen, wenn das Gehirn diese Richtung einschlägt.
Kinder können schnell so weit voranschreiten, daß Sie sie nicht weiter unterrichten können.
Gute Beispiele sind das \hyperref[c1ii12mental]{mentale Spielen} und die \hyperref[c1iii12]{absolute Tonhöhe}.
Erwecken Sie das mentale Spielen, indem Sie sie Musik hören lassen und sie bitten, Ihnen das Stück vorzusingen.
Vermitteln Sie ihnen die Vorstellung, daß sie Musik im Kopf haben, und daß die Musik nicht bloß durch die Ohren hereinkommt.
Stellen Sie sicher, daß sie sich Musik anhören, die in der richtigen Tonhöhe gespielt wird.
Bringen Sie ihnen dann die Tonleiter bei (benutzen Sie C-D-E usw., nicht do-re-mi, das sollte später kommen), und testen Sie sie danach in der C4-Oktave.
In diesem Alter erfolgt das Lernen der absoluten Tonhöhen automatisch und fast sofort; wenn man ihnen das C4 lehrt, werden sie erkennen, daß keine andere Note ein C4 ist, weil sie keine andere Erinnerung haben, die sie durcheinanderbringen kann.
Deshalb ist es so entscheidend, sie zu unterrichten, sobald sie dazu bereit sind.
Lehren Sie ihnen anschließend die höheren und tieferen Noten, das Konzept der relativen Tonhöhen, wie z.B. Oktaven, dann Intervalle aus zwei Noten (das Kind muß beide Noten identifizieren), dann Akkorde aus drei Noten oder drei beliebige Noten, die gleichzeitig gespielt werden, und immer so weiter, wenn möglich bis zu zehn Noten.
Diese musikalischen Lektionen können im Alter zwischen zwei und acht Jahren gegeben werden.
Unterstützen Sie das mentale Spielen, indem Sie ihnen viel gute Musik zum Zuhören anbieten und sie darauf trainieren, die Namen und den Komponisten der Kompositionen zu kennen.
Singen oder ein einfaches (richtig gestimmtes) musikalisches Spielzeug ist eine gute Möglichkeit, die Tonhöhe, den Rhythmus und die Kontrolle der Bewegungen zu lehren.
Verankern Sie die Vorstellung, daß die Musik ständig in ihrem Kopf ablaufen kann.
Sobald sie mit dem Klavierunterricht beginnen, wird das mentale Spielen durch das \hyperref[c1iii6]{Auswendiglernen} und den Aufbau eines auswendig gelernten Repertoires weiterentwickelt.
Seien Sie darauf vorbereitet, sie zu unterstützen, wenn sie sofort mit dem Komponieren beginnen - bieten Sie ihnen Möglichkeiten, ihre Stücke \hyperref[c1iii13]{aufzuzeichnen} oder lehren Sie ihnen Diktate.
Lange vor ihrer ersten Klavierstunde können Sie ihnen Bilder von vergrößerten Noten zeigen und sie mit dem Notensystem, wo die Noten stehen und wo sie auf dem Klavier zu finden sind vertraut machen.
Das wird die Aufgabe des Lehrers vereinfachen, ihnen das Notenlesen beizubringen.
Wenn Sie kein Klavierspieler sind, können Sie zur selben Zeit Klavierunterricht nehmen, wie Ihr Kind; das ist eine der besten Möglichkeiten, sie zum Anfangen zu bewegen.

Denken Sie vor allem daran, daß jedes Kind Stärken und Schwächen hat.
Es ist die Aufgabe der Eltern, die Stärken herauszufinden und zu unterstützen, und die Stärken werden nicht immer in die Richtung einer Karriere als Pianist weisen.
Sie müssen auch in Sport, Literatur, Wissenschaft, Kunst usw. getestet werden, weil jedes Kind ein Individuum ist.
Seien Sie nicht enttäuscht, wenn die Tests zeigen, daß das Kind die meiste Zeit noch nicht bereit ist - das ist normal.
Eine grundlegende Ausbildung am Klavier, die einer auf Wissen basierenden Methode, ähnlich einer Methode zur Projektsteuerung, wie sie in diesem Buch benutzt wird, folgt, wird dem Kind jedoch, unabhängig davon, welche Karriere es wählt, von Nutzen sein.

Eltern müssen die körperliche und geistige Entwicklung ihrer Kinder in der Balance halten.
Da das Klavierspielenlernen so schnell gehen kann, sind diese alten Zeiten, in denen fleißige Klavierspieler nicht genügend Zeit für Sport und andere Aktivitäten hatten, vorbei.
Techniker und Künstler müssen nicht zwangsläufig zu Weichlingen werden.
Es gibt diese verstörende Neigung, jedes Kind als hirn- oder körperlastig einzuordnen und eine Wand zwischen oder sogar einen gegenseitigen Ausschluß von Kunst und körperlichen Aktivitäten, Wissenschaft usw. zu erzeugen.
In Wahrheit folgen alle ähnlichen Prinzipien.
So sind z.B. die Regeln für das Lernen des Golfspielens und des Klavierspielens so ähnlich, daß dieses Buch mit nur ein paar Änderungen in ein Golfhandbuch umgewandelt werden kann.
Die Griechen hatten bereits vor langer Zeit recht - die geistige und körperliche Entwicklung müssen parallel verlaufen -, heute können wir sogar noch mehr tun.
Wenn die Eltern nicht die richtige Anleitung geben, werden einige Kinder ihre ganze Zeit einer Sache widmen und alles andere vernachlässigen, psychologische Probleme entwickeln und kostbare Zeit verschwenden.
\hyperref[c1iii10krank]{Gesundheit und Verletzungen} sind ein weiteres Thema.
Die Musikgeräte mit Kopfhörern können \hyperref[c1iii10gehoer]{das Gehör schädigen}, so daß man bereits bevor man 40 Jahre alt wird beginnt, das Gehör zu verlieren und unter einem Tinnitus leidet.
Eltern müssen ihren Kindern beibringen, die Lautstärke der Kopfhörer herunterzudrehen, besonders wenn sie eine Art von Musik hören, die oft extrem laut gespielt wird.
 

\subsubsection{Auswendiglernen, Blattspiel, Theorie, mentales Spielen, absolutes Gehör}
\label{c1iii16c}

Der Lehrer muß zu einem frühen Zeitpunkt wählen, ob dem Schüler das \hyperref[c1iii6]{Spielen aus dem Gedächtnis} gelehrt werden sollte oder das \hyperref[c1iii11]{Spielen vom Blatt}.
Diese Wahl ist notwendig, weil die Details des Unterrichtsprogramms und wie der Lehrer mit dem Schüler zusammenarbeitet davon abhängen.
\textbf{Die Suzuki-Violin-Methode betont das Spielen aus dem Gedächtnis zu Lasten des Notenlesens, besonders für Kinder, und das ist auch für das Klavier der beste Ansatz.}
Es ist einfacher, das Blattspiel zu üben, \textit{nachdem} man ziemlich gut spielen kann,
so wie man erst das Sprechen lernt und dann das Lesen.
Die Fähigkeit zu sprechen und zu musizieren sind natürliche evolutionäre Eigenschaften, über die wir alle verfügen; das Lesen kam erst später als Konsequenz unserer Zivilisation hinzu.
Das Sprechenlernen ist einfach ein Prozeß, alle Klänge und logischen Konstrukte jeder einzelnen Sprache auswendig zu lernen.
Deshalb ist das Lesen \enquote{fortgeschrittener} und weniger \enquote{natürlich} und kann dem Auswendiglernen logischerweise nicht vorausgehen.
Man hat sich z.B. (durch das Hören von Aufnahmen) viele musikalische Konzepte gemerkt, die niemals niedergeschrieben werden können, wie Farbe, das Spielen mit Autorität und Überzeugung usw.

Das Notenlesen sollte jedoch am Anfang nicht völlig vernachlässigt werden.
Es ist nur eine Frage der Priorität.
Da die Musiknotation einfacher als jedes Alphabet ist, können Kinder sogar das Notenlesen lernen, bevor sie das Bücherlesen lernen können.
Deshalb sollte das Notenlesen von Anfang an gelehrt werden, aber nur soweit, daß das Kind in der Lage ist, die Noten zu lesen, um ein Stück zu üben und auswendig zu lernen.
\textbf{Das Blattspiel sollte ermutigt werden, solange es nicht das Spielen aus dem Gedächtnis stört.}
Das bedeutet, daß bei einem Stück, das bereits auswendig gelernt wurde, die Notenblätter nicht mehr zum täglichen Üben benutzt werden sollten.
Der Lehrer muß jedoch sicherstellen, daß diese geringe Betonung des Blattspiels nicht zu einem schlechten Blattspieler führt, der automatisch alles auswendig lernt aber keine Noten lesen kann.
Die meisten Anfänger neigen dazu, entweder gute Blattspieler und schlechte Auswendiglernende oder gute Auswendiglernende und schlechte Blattspieler zu werden, denn wenn man bei dem einen gut wird, benötigt man das andere weniger.
Durch die sorgfältige Überwachung des Schülers kann ein Elternteil oder ein Lehrer verhindern, daß der Schüler zu einem schlechten Blattspieler oder einem schlechten Auswendiglernenden wird.
Die Hilfe der Eltern ist oft notwendig, damit diese Überwachung erfolgreich ist, da der Lehrer nicht immer dabei ist, wenn der Schüler übt.
Viele Eltern erzeugen sogar unabsichtlich schlechte Auswendiglernende oder schlechte Blattspieler, weil sie ihren Kindern aushelfen, anstatt sie ihre schwächeren Fertigkeiten trainieren zu lassen.
Da es längere Zeit dauert, bis man zu einem schlechten Blattspieler oder Auswendiglernenden wird, üblicherweise viele Jahre, ist genügend Zeit vorhanden, den Trend zu erkennen und zu korrigieren.
Genau wie Talente, Wunderkinder oder Genies werden Blattspieler und Auswendiglernende nicht geboren sondern erzeugt.

Notenlesen ist für Lehrer ein unverzichtbares Unterrichtsmittel; die Aufgabe des Lehrers kann vereinfacht werden, wenn dem Schüler das Notenlesen beigebracht werden kann.
Lehrer, die das Notenlesen betonen, haben sicherlich wegen der enormen Informationsmenge, die bereits in der einfachsten gedruckten Musik enthalten ist, recht, und praktisch jeder Anfänger wird einen großen Teil dieser Information verpassen.
Sogar fortgeschrittene Klavierspieler kehren oft zu den Notenblättern zurück, um sicherzustellen, daß sie nichts vergessen haben.
Klar basiert das beste Programm auf dem Auswendiglernen, es muß aber ein ausreichendes Training des Notenlesens beinhalten, damit der Schüler kein schlechter Blattspieler wird.

Der normale zum Lernen der neuen Stücke notwendige Aufwand des Notenlesens ist im allgemeinen ausreichend.
Besonders für Anfänger zahlt es sich nicht aus, das Notenlesen zu vertiefen, nur damit man vom Blatt spielen kann (da die Finger die Stücke ohnehin nicht spielen können), obwohl die anfängliche langsame Lesegeschwindigkeit sowohl für den Lehrer als auch den Schüler schrecklich frustrierend sein kann.
Ein wichtiger Lerntrick in der Klavierpädagogik ist, mehrere Fertigkeiten gleichzeitig zu lernen, besonders weil es bei vielen Fertigkeiten so lange dauert sie zu lernen.
So können das Auswendiglernen, das Blattspiel, die Theorie usw. alle gleichzeitig gelernt werden, was auf die Dauer viel Zeit spart.
Zu versuchen, eine dieser Fertigkeiten zu Lasten der anderen schnell zu lernen, kann nur zu Frustrationen führen.

\textbf{Man kann nicht zuviel Musiktheorie (Solfège), Notation, Diktate, \hyperref[c1iii12]{absolute Tonhöhenerkennung}, \hyperref[c1iii1b]{Rhythmus} usw. unterrichten.}
Theorie zu lernen hilft den Schülern beim Erwerb der Technik, Auswendiglernen, Verstehen der Struktur der Komposition und beim richtigen Aufführen.
Es wird auch beim Improvisieren und beim Komponieren hilfreich sein.
Statistisch gesehen komponiert die Mehrheit der erfolgreichen Klavierschüler am Ende selbst Musik.
Das einzige Problem mit Solfège-Stunden ist, daß viele Lehrer ineffizient unterrichten und viel Zeit der Schüler verschwenden.
Moderne Musik (Pop, Jazz usw.) benutzt heutzutage sehr fortgeschrittene musikalische Konzepte, und die Theorie ist für das Verständnis von Akkordprogressionen, Musikstruktur und Improvisation hilfreich.
Deshalb \textbf{ist es vorteilhaft, sowohl klassische als auch moderne Musik zu lernen.
Moderne Musik trägt zeitgenössische Theorie bei, hilft bei der Entwicklung von Rhythmus und den Fertigkeiten zum \hyperref[c1iii14]{Auftreten} und erreicht auch ein breiteres Publikum.}
\label{c040119}
\textit{[Weitere Informationen über das Improvisieren finden Sie u.a. in Marc Sabatellas \enquote{A Jazz Improvisation Primer}: \hyperref[http://www.outsideshore.com/primer/primer/index.html]{das Original in Englisch} <font color=\enquote{blue} size=\enquote{-1}>(extern), \hyperref[http://msjipde.uteedgar-lins.de/index.html]{als deutsche Übersetzung} (extern).]}</font>

\textbf{Das \hyperref[c1ii12mental]{mentale Spielen} sollte von Anfang an gelehrt werden, damit die Schüler stets üben, in Gedanken Musik zu spielen.
Wenn das mit der richtigen Tonhöhe geschieht, dann werden junge Schüler, die oft genug Musik gehört haben, nach nur wenigen Lektionen ohne Aufwand ein \hyperref[c1iii12]{absolutes Gehör} erwerben.}
Das ist ein guter Zeitpunkt, um diejenigen Schüler zu ermitteln, die nur eine geringe Vorstellung von Tonhöhen haben, und ein Verfahren zu entwickeln, um ihnen zu helfen.
Fortgeschrittene Schüler entwickeln automatisch Fähigkeiten zum mentalen Spielen, weil diese so notwendig sind; wenn man es ihnen jedoch von Anfang an lehrt, wird das ihre Lernrate für alles andere beschleunigen.
Wenn das mentale Spielen nicht gelehrt wird, werden die Schüler nicht einmal merken, daß sie es benutzen, und es nicht richtig entwickeln.
Außerdem werden sie, da ihnen nicht bewußt ist was sie tun, dazu neigen, das mentale Spielen zu vernachlässigen, wenn sie älter werden und ihr Gehirn mit anderen wichtigen Dingen beschäftigt ist.
Wenn sie das mentale Spielen vernachlässigen, werden sie ihr absolutes Gehör und die Fähigkeit, mit Leichtigkeit vorzuspielen, verlieren.
Ältere Schüler und Erwachsene, die das mentale Spielen und ein absolutes Gehör erlernen möchten, sollten sich \hyperref[c1iii12]{Abschnitt III.12} ansehen.



<!-- c1iii16d.html -->

\subsubsection{Einige Elemente des Klavierunterrichts}
\label{c1iii16d}

Der Klavierunterricht sollte kein Routineablauf sein, bei dem der Schüler das Unterrichtsstück spielt und der Lehrer ein neues Stück zuweist.
\textbf{Beim Beginnen eines neuen Stücks ist es die Aufgabe des Lehrers, es in Abschnitten durchzugehen, den Fingersatz zu untersuchen, die Noten zu analysieren und im Grunde den Schüler während der Unterrichtsstunde auf die endgültige Geschwindigkeit zu bringen, zumindest mit HS oder abschnittsweise.}
Nachdem die technischen Probleme gelöst sind, ändert sich die Aufgabe zum musikalischen Spielen hin: den musikalischen Inhalt untersuchen, den Ausdruck hervorbringen, die Eigenschaften des Komponisten (Mozart unterscheidet sich von Chopin usw.), die Farbe usw.
Ein guter Lehrer kann den Schülern eine enorme Menge Zeit sparen, indem er ihnen alle notwendigen Elemente der Technik demonstriert.
Es sollte nicht dem Schüler überlassen werden, diese durch Versuch und Irrtum herauszufinden.
Aufgrund dieser Erfordernisse können Unterrichtsstunden jenseits der Anfängerstufe sehr intensiv und zeitaufwendig werden.
\textbf{Tonleitern sollten Anfängern mit dem Daumenuntersatz gelehrt werden, aber innerhalb eines Jahres sollte ihnen auch der \hyperref[c1iii5b]{Daumenübersatz} beigebracht werden.}
Obwohl die meisten Übungen wie \hyperref[c1iii7h]{Hanon} heute als nicht hilfreich angesehen werden, ist es sehr wichtig, in der Lage zu sein, \hyperref[c1iii5]{Tonleitern und Arpeggios} (in allen Tonarten)
gut zu spielen; das wird mehrere Jahre harter Arbeit erfordern.

Jeden zweiten bis dritten Tag 30 Minuten zu üben, ist das absolut notwendige Minimum, um überhaupt Fortschritte zu machen.
Eine halbe Stunde täglich ist bei Kindern für einen bedeutenden Fortschritt angemessen.
Wenn sie älter werden, brauchen sie stetig mehr Zeit.
Das sind die minimalen Übungszeiten; für einen schnelleren Fortschritt wird mehr Zeit benötigt.
Wenn die Übungsmethoden effizient sind und die Schüler gute Fortschritte machen, wird die Frage, wieviel Übungszeit genug ist, bedeutungslos - es gibt soviel Musik und macht soviel Spaß, daß nie genug Zeit vorhanden ist.

\textbf{Der beste Weg, Schüler zum Üben zu motivieren, und die beste Art, die Kunst Musik zu machen zu lehren, ist, \hyperref[c1iii14]{Konzerte} abzuhalten.}
Wenn die Schüler auftreten müssen, bekommen alle Anweisungen des Lehrers, die notwendige Übungszeit usw. eine völlig neue Bedeutung und Dringlichkeit.
\textbf{Die Schüler werden dadurch selbstmotiviert.}
Es ist ein Fehler, Klavier ohne jegliches Programm zum Auftreten zu lehren.
Es gibt zahlreiche Möglichkeiten für solche Programme, und erfahrene Lehrer sind in der Lage, für jeden Schüler jeder Stufe ein angemessenes zu entwickeln.
Formelle Konzerte und Musikwettbewerbe sind voller Fallen und müssen mit Sorgfalt und jeder Menge Planung angegangen werden.
Lehrer können jedoch informelle Konzerte in weniger streßbeladenen Formaten organisieren, die für die Schüler einen enormen Nutzen haben.

\textbf{Obwohl Konzerte und Wettbewerbe wichtig sind, ist es noch wichtiger, ihre Gefahren zu vermeiden.}
Die Hauptgefahr ist, daß Konzerte selbstzerstörerisch sein können, weil der Streß, die Nervosität, der zusätzliche Aufwand, die zusätzliche Zeit und das Gefühl des Versagens auch nach kleinen Fehlern beim Formen der Fähigkeit und der psychologischen Grundlage des Schülers zum Auftreten in jedem Alter mehr Schaden anrichten als Gutes tun können.
Deshalb \textbf{müssen Lehrer ein klar definiertes Programm bzw. eine Vorgehensweise haben, die Kunst des Auftretens zusätzlich zur Kunst des Spielens zu lehren.}
Die vorbereitenden Methoden für Konzerte, die oben in \hyperref[c1iii14]{Abschnitt 14} besprochen wurden, sollten Teil dieses Programms sein.
Pop-Musik oder Musik, \enquote{die Spaß macht}, ist besonders für das Auftrittstraining geeignet.
Vor allem muß das Programm so gestaltet sein, daß es eine belohnende Atmosphäre der Leistungsfähigkeit erzeugt und keine wettbewerbsorientierte, bei der, wenn der Schüler die schwierigsten Stücke spielt, die er bewältigen kann, alles was geringer ist als unglaubliche Perfektion ein Versagen ist.
Für Wettbewerbe muß den Schülern bereits früh beigebracht werden, daß die Beurteilung nie perfekt oder fair ist; daß es nicht der Sieg sondern der Prozeß der Teilnahme ist, der wegen seines pädagogischen Werts am wichtigsten ist.
Ein entspannter und weniger nervöser Schüler wird das gleiche vorgegebene Stück besser ausführen und eine bessere Einstellung zum Auftreten entwickeln.
Die Schüler müssen verstehen, daß der Prozeß das endgültige Ziel eines Wettbewerbs ist, nicht daß man am Ende gewinnt.
Eine der wichtigsten Komponenten dieses Ziels ist es, die Fähigkeit zu entwickeln, die Erfahrung zu genießen, anstatt nervös zu werden.
Eine der wichtigsten Gefahren der meisten Wettbewerbe ist die Betonung des schwierigsten Materials, daß der Schüler spielen kann.
Der korrekte Schwerpunkt sollte die Musik sein, nicht die Akrobatik.

Natürlich müssen wir danach streben, Wettbewerbe zu gewinnen und fehlerlose Konzerte zu spielen.
Es gibt aber streßbeladene und weniger streßbeladene Herangehensweisen für diese Ziele.
\textbf{Es ist die Aufgabe des Lehrers, die Streßkontrolle zu lehren.}
Leider ignoriert die Mehrheit der Lehrer heutzutage völlig die Kontrolle des Stresses bei Auftritten oder schlimmer noch, Eltern und Lehrer tun häufig so, als ob es so etwas wie Nervosität nicht gäbe, sogar wenn sie selbst nervös sind.
Das kann den Effekt haben, ein dauerhaftes Problem mit der Nervosität zu erzeugen.
Sehen Sie dazu oben in Abschnitt 15 eine Besprechung über die \hyperref[c1iii15]{Kontrolle der Nervosität}.

\textbf{Es ist wichtig, einem Schüler zunächst alles über Nervosität und Streß beizubringen und
ihn nicht auf die Bühne zu schubsen, um ohne Vorbereitung aufzutreten, in der vergeblichen Hoffnung, daß er irgendwie von selbst lernen wird, wie man auftritt.}
Solch ein Vorgehen ist so ziemlich das gleiche wie jemanden in der Mitte eines tiefen Sees ins Wasser zu werfen, um ihm das Schwimmen beizubringen; diese Person kann für den Rest ihres Lebens Angst vor dem Wasser haben.
In jeder Unterrichtsstunde für den Lehrer zu spielen, ist zwar ein guter Anfang aber eine beklagenswert ungenügende Vorbereitung.
Deshalb sollte der Lehrer einen Plan für ein \enquote{Auftrittstraining} entwickeln, bei dem der Schüler schrittweise in die Auftritte eingeführt wird.
Dieses Training muß während der ersten Unterrichtsstunden beginnen.
Verschiedene Fertigkeiten, wie über Gedächtnisblockaden hinwegkommen bzw. sie vermeiden, Fehler kaschieren, Fehler erahnen bevor sie auftreten, Auszüge-Spielen, an einer beliebigen Stelle im Stück anfangen, die Auswahl der aufzuführenden Stücke, Kommunikation mit dem Publikum usw., sollten gelehrt werden.
Vor allem müssen sie das \hyperref[c1ii12mental]{mentale Spielen} lernen.
Wir haben gesehen, daß HS-Üben, \hyperref[c1ii17]{langsames Spielen} und \hyperref[c1iii6g]{\enquote{kalt} spielen} die wichtigen Komponenten der Vorbereitung sind.
Die meisten Schüler wissen nicht, welche \enquote{fertigen} Stücke sie zufriedenstellend aufführen können, bevor sie es nicht wirklich mehrere Male tun; deshalb wird jeder Schüler, auch unter den fertigen Stücken, ein \enquote{aufführbares} und ein \enquote{nicht aufführbares} Repertoire haben.
\textbf{Eine der besten Möglichkeiten, für Auftritte zu trainieren, ist, die fertigen Stücke des Schülers \hyperref[c1iii13]{aufzunehmen} und ein Album des fertigen Repertoires herzustellen, das regelmäßig auf den neuesten Stand gebracht wird, wenn der Schüler Fortschritte macht.}
Das sollte von Beginn des Unterrichts an durchgeführt werden, damit diese Fertigkeit so früh wie möglich entwickelt wird.
Der erste Fehler, den die meisten Klavierspieler begehen, ist, zu denken: \enquote{Ich bin noch ein Anfänger, deshalb ist mein Spiel es nicht wert, aufgenommen zu werden.}
Wenn man das am Anfang glaubt, wird man es schließlich für den Rest des Lebens befolgen, weil es zu einer sich selbst erfüllenden Prophezeiung wird.
Diese Behauptung ist falsch, weil Musik das Höchste ist - leichte Kompositionen, die musikalisch gespielt werden, sind kaum zu übertreffen; Horowitz kann \enquote{Alle meine Entchen} auch nicht besser spielen als ein gut unterrichteter Anfänger.

Ohne ein Auftrittstraining werden sogar gute Künstler nicht entsprechend ihrer besten Fähigkeiten vorspielen, und die Mehrheit der Schüler wird am Ende glauben, daß ein Auftritt am Klavier die reine Hölle ist und die Musik oder das Klavier daran schuld ist.
Hat sich diese Einstellung bereits während der Jugend verfestigt, wird sie in das Erwachsenenalter übernommen.
In Wahrheit sollte es genau das Gegenteil sein.
Das Auftreten sollte das endgültige Ziel sein, die endgültige Belohnung für all die harte Arbeit.
Es ist die Demonstration der Fähigkeit, ein Publikum zu beherrschen, die Fähigkeit, die größten Ideen der größten musikalischen Genies, die je gelebt haben, zu übermitteln.
\textbf{Ein sicheres \hyperref[c1ii12mental]{mentales Spielen} ist die effektivste Methode, das Lampenfieber zu reduzieren.}

Eine Möglichkeit, Schüler in das Auftreten bei Konzerten einzuführen, ist, simulierte Konzerte unter den Schülern abzuhalten und sie Ihre Befürchtungen, Schwierigkeiten, Schwächen und Stärken diskutieren zu lassen, um sie alle mit den wichtigsten Punkten vertraut zu machen.
\hyperref[c1ii12mental]{Wie spielt man in Gedanken}?
Tut man es ständig?
Benutzt man das \hyperref[c1iii6foto]{fotografische Gedächtnis} oder das \hyperref[c1iii6tastatur]{Tastatur-Gedächtnis} oder nur das \hyperref[c1iii6musik]{Musik-Gedächtnis}?
Geschieht es automatisch oder macht man es zu bestimmten Zeiten?
Sie werden die einzelnen Punkte besser verstehen, wenn sie sie tatsächlich erfahren können und sie dann mit ihren Mitschülern offen besprechen.
Jeder Streß oder \hyperref[c1iii15]{Nervosität}, die sie fühlen könnten, wird weniger angsteinflößend, wenn sie erkennen, daß jeder dieselben Dinge erlebt, daß Nervosität absolut natürlich ist, und daß es verschiedene Möglichkeiten gibt, sie zu bekämpfen oder sogar einen Vorteil daraus zu ziehen.
Insbesondere wird der ganze Prozeß viel weniger mysteriös und furchterregend, wenn sie erst einmal durch den kompletten Prozeß vom Anfang bis zum Ende eines simulierten Konzerts hindurchgegangen sind.
\textbf{Schülern muß beigebracht werden, daß zu lernen, Spaß am Auftreten zu haben, ein Teil der Kunst des Klavierspielens ist.
Diese \enquote{Kunst des Auftretens} erfordert, so wie die Fingertechnik, ebenfalls Studium und Übung}.
In einer Gruppe von Schülern gibt es immer einen, der gut im Auftreten ist.
Die anderen können durch die Beobachtung der guten Schüler lernen und durch die Diskussion darüber, wie diese die einzelnen Aufgaben bewältigen.
Es gibt aber auch Schüler, die auf der Bühne einfach nur erstarren - diese benötigen besondere Hilfe, wie z.B. sehr einfache Stücke zum Aufführen zu lernen, während eines Konzerts mehrere Gelegenheiten zum Auftreten zu bekommen oder in einer Gruppe oder einem Duo aufzutreten.

\textbf{Eine andere Möglichkeit, Schüler in das Auftreten einzuführen und gleichzeitig etwas Spaß zu haben, ist, ein informelles Konzert anzusetzen, in dem die Schüler das Spiel \enquote{Wer kann am schnellsten spielen?} spielen.}
Bei diesem Spiel spielt jeder Schüler das gleiche Stück aber der Zeitraum zum Üben ist begrenzt, z.B. auf drei Wochen.
Beachten Sie, daß bei dieser List die verborgene Tagesordnung ist, den Schülern beizubringen, wie man Konzerte genießt, nicht ihnen beizubringen, wie man schnell spielt.
Die Schüler stimmen selbst darüber ab, wer der Sieger ist.
Zunächst gibt der Lehrer keine Anweisungen; die Schüler müssen ihre eigenen Übungsmethoden auswählen.
Nach dem ersten Konzert hält der Lehrer eine Gruppenstunde, in der die Schüler ihre Übungsmethoden diskutieren und der Lehrer nützliche Informationen hinzufügt.
Selbstverständlich müssen Klarheit, Genauigkeit und die Musik bei der Wahl des Gewinners berücksichtigt werden.
Man kann die Musik schneller klingen lassen, indem man langsamer aber genauer spielt.
Es wird große Unterschiede in den Übungsmethoden und den erzielten Ergebnissen bei den einzelnen Schülern geben, und auf diese Art werden sie voneinander lernen und die Grundlagen besser verstehen.
Während die Schüler an einem \enquote{Wettbewerb} teilnehmen, muß der Lehrer sicherstellen, daß es eine freudige Erfahrung ist, eine Möglichkeit, die Freude am Auftreten zu erfahren, eine Möglichkeit, die \hyperref[c1iii15]{Nervosität} völlig zu vergessen.
Fehler erzeugen Gelächter, es sollte nicht die Nase über sie gerümpft werden.
Und nachher könnten Erfrischungen gereicht werden.
Der Lehrer darf nicht vergessen, neben den Anweisungen zum Lernen der \enquote{Wettbewerbsfähigkeiten} auch hin und wieder Anweisungen zum Lernen des Auftretens einzustreuen.

Wie sollten Konzerte organisiert sein, nachdem den Schülern die Grundlagen des Auftretens beigebracht wurden?
Sie sollten so gestaltet sein, daß sie die Fähigkeit aufzutreten verstärken.
\textbf{Eines der schwersten Dinge ist, dieselbe Komposition mehrere Male am gleichen Tag oder an aufeinanderfolgenden Tagen aufzuführen.}
Deshalb bieten solche wiederholten Auftritte das beste Training für die Verstärkung der Fähigkeit aufzutreten.
Für Lehrer oder Schulen mit genügender Schülerzahl ist der folgende Plan gut zu verwenden.
Teilen Sie die Schüler in Gruppen für Anfänger, Mittelstufe und Fortgeschrittene auf.
Halten Sie am Freitag ein Konzert für die Anfänger ab, mit ihren Eltern und Freunden als Publikum.
Anfänger sollten ab ihrem ersten Unterrichtsjahr, bereits in einem Alter von 4 oder 5 Jahren, an Konzerten teilnehmen.
Am Ende dieses Konzerts spielen die fortgeschrittenen Schüler ebenfalls, was es für das Publikum wirklich lohnend macht, das Konzert zu besuchen.
Am Samstag spielen die Mittelstufenschüler, mit ihren Eltern und Freunden als Publikum; wieder spielen am Ende die fortgeschrittenen Schüler.
Am Sonntag halten die fortgeschrittenen Schüler ihr Konzert ab, mit ihren Eltern als Publikum; einige besondere Gäste könnten eingeladen werden.
Auf diese Weise müssen die fortgeschrittenen Schüler dasselbe Stück an drei Tagen hintereinander aufführen.
Das Sonntagskonzert der fortgeschrittenen Schüler sollte aufgenommen und auf CD überspielt werden, da diese großartige Souvenirs sind.
Wenn diese Art von Konzert zweimal im Jahr abgehalten wird, dann hat jeder fortgeschrittene Schüler jedes Jahr sechs Konzerte \enquote{in der Tasche}.
Wenn diese Schüler auch zu Wettbewerben geschickt werden (was üblicherweise eine Ausscheidung, ein Finale und wenn man gewinnt noch ein Abschlußkonzert bedeutet), dann haben sie ein angemessenes Auftrittstraining (mindestens 9 Auftritte im Jahr).
Da die meisten Stücke nicht \enquote{sicher} sind, bis sie dreimal aufgeführt wurden, dient dieser Konzertplan auch dazu, das Konzertstück \enquote{sicher} zu machen, so daß es nun, nach nur einem Konzertwochenende, in das \enquote{aufführbare} Repertoire aufgenommen werden kann.

\textbf{Lehrer sollten gewillt sein, mit anderen Lehrern zu kommunizieren, Ideen auszutauschen und voneinander zu lernen.}
Es gibt nichts potentiell schädlicheres für einen Schüler als einen Lehrer, dessen Lehrmethoden inflexibel und an einem gewissen Zeitpunkt stehengeblieben sind.
In diesem Informationszeitalter gibt es so etwas wie geheime Methoden, das Klavierspielen zu unterrichten, nicht, und der Erfolg des Lehrers hängt von der offenen Kommunikation ab.
Ein wichtiger Punkt der Kommunikation ist der Austausch der Schüler.
Die meisten Schüler können in hohem Maß davon profitieren, daß sie von mehr als einem Lehrer unterrichtet werden.
Lehrer von Anfängern sollten ihre Schüler, sobald sie so weit sind, an Lehrer der höheren Stufen weiterreichen.
Natürlich werden die meisten Lehrer versuchen, ihre besten Schüler zu behalten und so viele Schüler zu unterrichten wie sie können.
Eine Möglichkeit, dieses Problem zu lösen, ist für die Lehrer, eine Gruppe von Lehrern mit verschiedenen Spezialgebieten zu bilden, so daß die Gruppe eine komplette Schule bildet.
Das hilft auch den Lehrern, weil es für sie viel einfacher wird, Schüler zu finden.
Für Schüler, die gute Lehrer suchen, ist es aufgrund dieser Überlegungen klar, daß es am besten ist, eher nach Lehrergruppen zu suchen als nach Lehrern, die einzeln arbeiten.
Lehrer können ebenfalls davon profitieren, wenn sie sich zusammenschließen und die Schüler und die Kosten für die Einrichtungen teilen.

\textbf{Lehrer, die gerade beginnen, haben oft Schwierigkeiten, ihre ersten Schüler zu finden.}
Sich einer Gruppe von Lehrern anzuschließen, ist eine gute Möglichkeit anzufangen.
Auch müssen viele etablierte Lehrer oft Schüler aus Zeitmangel abweisen, besonders wenn der Lehrer in seinem Einzugsgebiet einen guten Ruf hat.
Diese Lehrer sind gute Quellen für Schüler.
Eine Möglichkeit, den Vorrat an potentiellen Schülern zu erhöhen, ist, den Schülern anzubieten, sie bei ihnen zu Hause zu unterrichten.
Zumindest für die ersten paar Jahre könnte dies ein guter Ansatz für das Vergrößern des potentiellen Schülerreservoirs sein.


\subsubsection{Warum die größten Pianisten nicht unterrichten konnten}
\label{c1iii16e}

Sehr wenige der großen Pianisten waren gute Lehrer.
Das ist vollkommen natürlich, weil Künstler ihr ganzes Leben trainieren Künstler zu sein und nicht Lehrer.
Ich habe als Physikstudent an der Cornell University eine ähnliche Situation erlebt. Ich nahm Kurse bei Professoren, die auf das Unterrichten spezialisiert waren, und besuchte auch wöchentliche Vorlesungen berühmter Physiker, darunter zahlreiche Nobelpreisgewinner.
Einige dieser berühmten Physiker konnten gewiß spannende Vorlesungen halten, die großes Interesse hervorriefen, aber ich lernte die meisten Fertigkeiten, die notwendig waren, um einen Job als Physiker zu finden, von den unterrichtenden Professoren, nicht von den Nobelpreisträgern.
Dieser Unterschied in der Fähigkeit zu unterrichten zwischen den unterrichtenden und praktizierenden Wissenschaftlern verblaßt - wegen der Natur der wissenschaftlichen Disziplin (s. \hyperref[c3_1]{Kapitel 3}) - im Vergleich zu der Kluft, die in der Welt der Kunst besteht.
Lernen und Unterrichten sind integrale Bestandteile davon, ein Wissenschaftler zu sein.
Im Gegensatz dazu waren die größten Pianisten entweder widerstrebend oder aus wirtschaftlicher Notwendigkeit zum Unterrichten gezwungen, ohne eine bedeutende Ausbildung dafür erhalten zu haben.
Deshalb gibt es viele Gründe, warum große Künstler u.U. keine guten Lehrer waren.

Leider haben wir in der Vergangenheit bei den berühmten Künstlern eine Anleitung in der Annahme gesucht, daß wenn sie es können, sie auch in der Lage sein sollten, uns zu zeigen wie es geht.
Typische historische Berichte zeigen, daß wenn man einen berühmten Pianisten fragt, wie man eine bestimmte Passage spielen muß, er sich an das Klavier setzen und sie spielen wird, weil die Sprache des Pianisten mit den Händen und dem Klavier und nicht mit dem Mund gesprochen wird.
Derselbe große Künstler hat vielleicht nur eine geringe Vorstellung davon, wie die Finger sich bewegen oder wie sie die Klaviertasten handhaben.
Um die Hände auf die richtige Art zu bewegen, muß man lernen, eine Vielzahl von Muskeln und Nerven zu kontrollieren, und dann die Hände darauf trainieren, diese Bewegungen auszuführen.
Es gibt unter den Möglichkeiten, Technik zu erwerben, zwei Extreme.
Ein Extrem ist der analytische Ansatz, bei dem jede Bewegung, jeder Muskel und jede physiologische Information analysiert wird.
Das andere Extrem ist der künstlerische Ansatz, bei dem man sich einfach ein bestimmtes musikalisches Ergebnis vorstellt, und der Körper reagiert auf verschiedene Arten, bis das gewünschte Resultat erreicht ist.
Dieser künstlerische Ansatz kann nicht nur eine schnelle Vereinfachung sein, sondern auch zu unerwarteten Ergebnissen führen, welche die ursprüngliche Idee übersteigen können.
Er hat auch den Vorteil, daß ein \enquote{Genie} ohne analytische Ausbildung Erfolg haben kann.
Der Nachteil ist, daß es keine Garantie für den Erfolg gibt.
Technik, die auf diese Art erworben wird, kann nicht analytisch gelehrt werden, außer indem man sagt, daß \enquote{man die Musik auf diese Art fühlen muß}, um sie zu spielen.
Leider ist diese Art der Anweisung für diejenigen, die noch nicht wissen, wie man etwas spielt, wenig hilfreich, außer um zu zeigen, daß es möglich ist.
Es reicht auch nicht, die Übungsmethoden zu kennen.
Man braucht die richtige Erklärung, warum sie funktionieren.
Diese Erfordernis liegt oft außerhalb der Fachkenntnisse des Künstlers oder Klavierlehrers.
Deshalb gibt es ein grundlegendes Hindernis für die richtige Entwicklung der Werkzeuge für den Klavierunterricht: Künstler und Klavierlehrer haben nicht die Ausbildung, um solche Werkzeuge zu entwickeln; auf der anderen Seite haben Wissenschaftler und Ingenieure, die über eine solche Ausbildung verfügen können, nicht genügend Erfahrung mit dem Klavier, um die Methoden für das Klavierspielen zu erforschen.

Die alten Meister waren selbstverständlich Genies und hatten sowohl eine bemerkenswerte Einsicht und Erfindungsgabe, als auch ein intuitives Gespür für Mathematik und Physik, das sie auf ihr Klavierspiel anwandten.
Deshalb ist es nicht richtig, zu schließen, daß sie keine analytische Herangehensweise an die Technik hatten; praktisch jede analytische Lösung für das Klavierüben, die wir heute kennen, wurde durch diese Genies viele Male erneut erfunden oder zumindest von ihnen benutzt.
Es ist deshalb unglaublich, daß niemand jemals daran gedacht hat, diese Ideen systematisch zu dokumentieren.
Es ist sogar noch erstaunlicher, daß anscheinend sowohl die Lehrer als auch die Schüler nicht einmal in groben Zügen erkannten, daß die Übungsmethoden der Schlüssel für den Erwerb der Technik waren.
Ein paar gute Lehrer haben immer gewußt, daß Talent eher erzeugt wird als angeboren ist (siehe \hyperref[reference]{Quellenverzeichnis}).
Die größte Schwierigkeit scheint die Unfähigkeit des künstlerischen Ansatzes gewesen zu sein, die korrekte theoretische Basis (Erklärung) dafür zu bestimmen, warum diese Übungsmethoden funktionieren.
Ohne eine solide theoretische Erklärung oder Basis kann sogar eine korrekte Methode von verschiedenen Lehrern mißbraucht, mißverstanden, verändert oder herabgewürdigt werden, so daß sie möglicherweise nicht immer funktioniert und als unzuverlässig oder nutzlos angesehen wird.
Diese historischen Tatsachen verhinderten jegliche geordnete Entwicklung der Lehrmethoden für das Klavierspielen.
Deshalb ist das Verständnis - oder die Erklärung, warum eine Methode funktioniert - mindestens so wichtig wie die Methode selbst.
Diese Situation wurde dadurch, daß das \enquote{Talent} als Weg zum Erfolg gepriesen wurde, noch verschlimmert.
Das war eine bequeme Ausrede für erfolgreiche Pianisten, die mehr Anerkennung bekamen als sie verdienten und gleichzeitig von der Verantwortung für ihre Unfähigkeit, die \enquote{weniger talentierten} zu unterrichten, befreit wurden.
Und natürlich trug das Attribut \enquote{Talent} zu ihrem wirtschaftlichen Erfolg bei.

Außerdem neigten die Klavierlehrer dazu, insofern wenig mitteilsam zu sein, als sie ihre Vorstellungen vom Unterrichten kaum mit anderen teilten.
Nur an großen Konservatorien gab es einen bedeutenden Austausch von Ideen, so daß die Qualität des Unterrichts an den Konservatorien besser war als irgendwo sonst.
Die im vorangegangenen Abschnitt beschriebenen Probleme verhinderten jedoch sogar an diesen Organisationen jegliche wirklich systematische Entwicklung der Lehrmethoden.
Ein zusätzlicher Faktor war die Einteilung der Lernenden in Anfänger und fortgeschrittene Schüler.
Konservatorien akzeptierten im allgemeinen nur fortgeschrittene Schüler; ohne eine den Konservatorien entsprechende Ausbildung erreichten jedoch nur wenige Schüler die fortgeschrittenen Stufen, die notwendig waren, um akzeptiert zu werden.
Das verlieh dem Klavierspielenlernen den Ruf, viel schwieriger zu sein als es tatsächlich ist.
Der Engpaß, der durch den Mangel an guten Lehrmethoden erzeugt wurde, wurde in der Vergangenheit dem Mangel an \enquote{Talent} zugeschrieben.
Wenn alle diese historischen Fakten zusammengetragen werden, ist leicht zu verstehen, warum die großen Meister nicht unterrichten konnten und warum sogar hingebungsvolle Klavierlehrer nicht alle Werkzeuge hatten, die sie benötigten.

Anfangs schrieb ich dieses Buch nur als Sammlung einiger bemerkenswert effektiver Lehrwerkzeuge; es hat sich jedoch zu einem Projekt weiterentwickelt, das die historischen Schwächen, die für die meisten Schwierigkeiten beim Erwerb der Technik verantwortlich sind, direkt behandelt.
Das Schicksal hat die Zukunft des Klaviers plötzlich in ein weites, offenes, unbekanntes Land mit unbegrenzten Möglichkeiten verwandelt.
Wir kommen in eine schöne, neue, aufregende Ära, die von jedem genossen werden kann.



<!-- c1iii17.html -->

\subsection{Klaviere, Flügel und elektronische Klaviere; Kauf und Wartung}
\label{c1iii17}
 
\subsubsection{Flügel, akustisches oder elektronisches Klavier?}
\label{c1iii17a}

Flügel haben bestimmte Vorteile gegenüber \hyperref[upright]{Klavieren}.
Diese Vorteile sind jedoch im Vergleich zur Wichtigkeit der Fertigkeitsstufe des Pianisten gering.
\textbf{Es gibt große Pianisten, die technisch fortgeschritten wurden und hauptsächlich auf Klavieren übten.
Es gibt keinen Beweis dafür, daß man für die anfängliche technische Entwicklung einen Flügel braucht}, obwohl es ein paar Klavierlehrer gibt, die darauf bestehen, daß jeder ernsthafte Schüler auf einem Flügel üben muß.
Ein Argument kann, zumindest für Anfänger, zur Bevorzugung der Klaviere angeführt werden, weil Klaviere ein festeres Spielen erfordern und vielleicht für die frühe Fingerentwicklung besser sind (man muß die Tasten fester herunterdrücken, um einen lauteren Ton zu erzeugen).
Sie sind eventuell sogar für Mittelstufenschüler überlegen, weil Klaviere weniger verzeihen und eine größere technische Fertigkeit erfordern.
Diese Argumente sind umstritten, aber sie zeigen, daß für Schüler bis zur Mittelstufe die Unterschiede zwischen Klavieren und Flügeln gegenüber anderen Faktoren wie die Motivation des Schülers, die Qualität der Lehrer, den Übungsmethoden und der richtigen Wartung des Klaviers von geringer Bedeutung sind.
Ein weiterer Faktor ist die Qualität: Gute Klaviere sind Flügeln von geringer Qualität (was die meisten Flügel einschließt, die kleiner als 5,2 ft = 1,58 m sind) überlegen.
Die Regel hinsichtlich aufrechter Klaviere ist einfach: Wenn Sie bereits eines besitzen, gibt es keinen Grund, sich davon zu trennen, bis Sie sich ein elektronisches Klavier oder einen Flügel kaufen; wenn Sie noch kein Klavier besitzen, gibt es keinen zwingenden Grund, ein aufrechtes Klavier zu kaufen.
Schüler oberhalb der Mittelstufe benötigen wahrscheinlich einen Flügel, weil technisch schwierige Musik auf den meisten aufrechten und elektronischen Klavieren viel schwerer (wenn nicht gar unmöglich) zu spielen ist.

Elektronische Klaviere unterscheiden sich grundlegend von akustischen (Flügel und Aufrechte).
Die Konstruktion ihrer Mechanik ist nicht so gut (nicht so teuer) und bei den meisten elektronischen Klavieren sind die Lautsprecher nicht gut genug, um mit den akustischen Klavieren zu konkurrieren.\footnote{Das war ein Grund, warum ich mich damals für ein Stage-Piano ohne Lautsprecher entschieden habe.
Es kostet weniger, beim Üben benutze ich fast immer einen Kopfhörer und zum Vorspielen die vorhandene Stereoanlage.}
Akustische Klaviere erzeugen den Klang deshalb auf grundlegend andere Weise, was dazu führt, daß viele Kritiker akustische Klaviere wegen der besseren Kontrolle des \enquote{Tons} bevorzugen.
Deshalb ist die Frage, welches Instrument das beste ist, komplex und hängt von den Umständen und den Anforderungen des Einzelnen ab.
Wir werden im folgenden jeden einzelnen Typ besprechen, so daß wir eine durchdachte Entscheidung darüber fällen können, welches Klavier für welchen Schüler das beste ist.

 
\subsubsection{Elektronische Klaviere}
\label{c1iii17b}

Heutige elektronische Klaviere sind guten Flügeln hinsichtlich der Entwicklung der Spieltechnik immer noch 
unterlegen, aber sie verbessern sich rapide.
Auch die besten elektronischen Klaviere sind für fortgeschrittene Klavierspieler ungeeignet; ihr mechanisches Ansprechverhalten ist schlechter, das musikalische Ergebnis und ihr Dynamikumfang sind unterlegen, und es wird schwierig, schnelles, technisch fortgeschrittenes Material auszuführen.
Die meisten preisgünstigeren Lautsprecher können nicht mit dem Resonanzboden eines Flügels konkurrieren.
Die elektronischen Klaviere gestatten nicht die Kontrolle des Klangs, der Farbe, des Pianissimo, des Staccato und der besonderen Manipulationsmöglichkeiten des Halte- und Dämpferpedals, die ein guter Flügel bietet.
Deshalb steht außer Frage, daß ein fortgeschrittener Klavierspieler einem Flügel den Vorzug vor einem elektronischen Klavier gibt; das stimmt jedoch nur unter der Voraussetzung, daß der Flügel mindestens zweimal im Jahr gestimmt und, wann immer es notwendig ist, eingestellt und \hyperref[c2_7_hamm]{intoniert} wird.
Die meisten aufrechten Klaviere bieten keinen genügenden Vorteil für die technische Entwicklung, um ihren Gebrauch gegenüber qualitativ guten elektronischen Klavieren, die ohne weiteres verfügbar und vergleichsweise preisgünstig sind und wenig in der Unterhaltung kosten, zu rechtfertigen.

Die elektronischen Klaviere haben einige besondere Vorteile, die wir im folgenden besprechen.
Wegen dieser Vorteile werden die meisten ernsthaften Klavierspieler sowohl ein akustisches als auch ein elektronisches Klavier besitzen.

\begin{enumerate}[label={\arabic*.}] 
%: 1
\item Für weniger als die Hälfte des Preises eines durchschnittlichen akustischen Aufrechten können Sie ein neues elektronisches Klavier mit allen notwendigen Eigenschaften kaufen, z.B. Kopfhöreranschluß, Lautstärkeregler, Anschlagsdynamik, Klänge für Orgel, Saiteninstrumente, Cembalo usw., Metronom, Aufnehmen, Midi-Anschlüsse, Analog-Ausgänge, Transposition, verschiedene Stimmungen und Begleitrhythmen.
Die meisten elektronischen Klaviere bieten viel mehr, aber das sind die minimalen Eigenschaften, die Sie erwarten können.
Das Argument, daß ein akustisches Klavier eine bessere Investition als ein elektronisches sei, ist falsch, weil ein akustisches Klavier keine gute Investition ist, besonders wenn es wesentlich mehr kostet und der anfängliche Wertverlust hoch ist.
Das elektronische Klavier erfordert keine Wartung, während die Wartungskosten eines akustischen beträchtlich sind, da es ungefähr zweimal jährlich gestimmt, intoniert und eingestellt sowie hin und wieder repariert werden muß.

%: 2
\item Elektronische Klaviere sind stets perfekt gestimmt.
Sehr junge Kinder, die genügend oft perfekt gestimmte Klaviere hören, erwerben automatisch ein \hyperref[c1iii12]{absolutes Gehör}, obwohl die meisten Eltern das nicht bemerken, weil es, wenn es nicht erkannt und gepflegt wird, in der Jugendzeit wieder verlorengeht.
Das akustische Klavier fängt an zu verstimmen, sobald der Stimmer Ihr Haus verläßt, und einige Noten werden die meiste Zeit aus der Stimmung sein (tatsächlich werden die meisten Noten die meiste Zeit aus der Stimmung sein).
Diese kleinen Abweichungen von der Stimmung werden den Erwerb des absoluten Gehörs jedoch nicht beeinflussen, solange das Klavier nicht stark verstimmt ist.
Da zu viele akustische Klaviere unzureichend gewartet sind, kann die Tatsache, daß die elektronischen Klaviere immer richtig gestimmt sind, ein großer Vorteil sein.
\textbf{Die Wichtigkeit eines gut gestimmten Klaviers für die musikalische und technische Entwicklung kann nicht überbetont werden, denn ohne die musikalische Entwicklung wird man nie lernen, \hyperref[c1iii14]{vorzuspielen und aufzutreten}.}
Der Klang eines elektronischen Klaviers kann in hohem Maß verbessert werden, indem man es an einen guten Verstärker mit guten Lautsprechern anschließt.

%: 3
\item Sie können Kopfhörer benutzen oder die Lautstärke so einstellen, daß Sie beim Üben niemand anderen stören.
Die Möglichkeit, die Lautstärke herunterzudrehen, ist auch zum Vermindern von \hyperref[c1iii10gehoer]{Gehörschäden} beim Üben lauter Passagen nützlich: ein wichtiger Faktor für jeden, der älter als 60 Jahre ist; ein Alter, in dem viele unter einsetzendem Hörverlust oder Tinnitus leiden.
Wenn man ein fortgeschrittener Spieler ist, erzeugt auch ein elektronisches Klavier (trotz abgeschalteter Lautsprecher) ein erhebliches \enquote{Spielgeräusch}, das in unmittelbarer Nähe ziemlich laut sein kann, und diese Vibrationen können durch den Boden in die unter dem Klavier liegenden Räume übertragen werden.
Deshalb ist es ein Fehler zu glauben, daß die Geräusche eines elektronischen Klaviers (oder eines akustischen \enquote{Silent}-Klaviers) völlig eliminiert werden können.

%: 4
\item Sie sind viel leichter zu transportieren als akustische Klaviere.
Obwohl es leichte Keyboards mit ähnlichen Eigenschaften gibt, ist es für das Klavierüben am besten, ein schwereres elektronisches Klavier zu benutzen, damit es sich beim Spielen von schneller lauter Musik nicht bewegt.
Auch diese schwereren elektronischen Klaviere können leicht von zwei Personen getragen werden und passen in viele Autos.

%: 5
\item \textbf{Ein variables Spielgewicht ist wichtiger als vielen bewußt ist.}
Man muß jedoch wissen, was \enquote{Spielgewicht} bedeutet, bevor man es vorteilhaft einsetzen kann; \hyperref[touchweight]{Details} finden Sie weiter unten.
Im allgemeinen ist das Spielgewicht eines elektronischen Klaviers etwas geringer als das eines akustischen.
Das leichtere Gewicht wurde aus zwei Gründen gewählt: um es Keyboard-Spielern einfacher zu machen, diese elektronischen Klaviere zu spielen (das Spielgewicht von Keyboards ist noch geringer), und um es im Vergleich zu akustischen Klavieren einfacher zu machen, sie zu spielen.
Der Nachteil des leichteren Gewichts ist, daß man es eventuell etwas schwieriger findet, auf einem akustischen Klavier zu spielen, nachdem man auf einem elektronischen geübt hat.
Das Spielgewicht eines akustischen Klaviers muß höher sein, um einen volleren Klang zu erzeugen.
Ein Vorteil des höheren Gewichts ist, daß man die Tasten eines akustischen Klaviers während des Spielens erfühlen kann, ohne aus Versehen falsche Tasten zu drücken.
Das kann jedoch auch zu nachlässigem Spielen mit einigen ungewollten Fingerbewegungen führen, weil man die Tasten eines akustischen Klaviers leicht anschlagen kann, ohne einen Ton zu erzeugen.
Man kann üben, diese unkontrollierten Bewegungen loszuwerden, indem man ein elektronisches Klavier benutzt und ein leichtes Spielgewicht auswählt, so daß ein ungewollter Anschlag einen Ton erzeugt.
Viele Menschen, die nur auf akustischen Klavieren üben, wissen nicht einmal, daß sie solche unkontrollierten Bewegungen haben, bis sie versuchen, auf einem elektronischen Klavier zu spielen, und feststellen, daß sie ziemlich viele zusätzliche Tasten anschlagen.
Der leichte Anschlag ist auch für das schnelle Erwerben schwieriger Technik nützlich.
Wenn man später auf einem akustischen Klavier spielen muß, kann man mit erhöhtem Gewicht üben, nachdem man die Technik bereits erworben hat.
Dieser zweistufige Prozeß ist gewöhnlich schneller, als wenn man versucht, sich die Technik bei einem hohen Spielgewicht anzueignen.

%: 6
\item Klaviermusik \hyperref[c1iii13]{aufzunehmen} ist mit einer konventionellen Ausrüstung eine der schwierigsten Aufgaben.
Mit einem elektronischen Klavier geht das \enquote{auf Knopfdruck}!
Man kann leicht ein Album mit allen gelernten Stücken aufbauen.
Aufzunehmen ist nicht nur eine der besten Möglichkeiten, Ihre Stücke wirklich zu vollenden und auf Hochglanz zu polieren, sondern auch um zu lernen, \hyperref[c1iii14]{wie man für ein Publikum spielt}.
Jeder sollte es sich vom ersten Tag des Unterrichts an zur Gewohnheit machen, jedes fertige Stück aufzunehmen.
Selbstverständlich werden die ersten Vorträge nicht perfekt sein, so daß Sie die Stücke wahrscheinlich noch einmal aufnehmen, wenn Sie besser geworden sind.
Zu viele Schüler nehmen ihre Stücke niemals auf, was der Hauptgrund für übermäßige \hyperref[c1iii15]{Nervosität} und Schwierigkeiten während des Vorspielens ist.

%: 7
\item Die meisten Klavierspieler, die gute Übungsmethoden befolgen und das Klavierspielen in jungen Jahren beherrschen, komponieren irgendwann ihre eigene Musik.
Elektronische Klaviere sind beim Aufnehmen dieser Kompositionen hilfreich, so daß man sie nicht aufschreiben muß, und dafür, sie mit verschiedenen für die jeweilige Komposition geeigneten Instrumenten zu spielen.
Mit etwas zusätzlicher Software oder Hardware kann man ganze Symphonien komponieren und jedes Instrument selbst spielen.
Es gibt sogar Software, die Ihre Musik (wenn auch nicht perfekt) in ein Notat umwandelt.
Es gibt jedoch nichts hilfreicheres für das Komponieren als ein qualitativ hochwertiger Flügel - der Klang eines guten Klaviers ist eine Inspiration für den Kompositionsprozeß; wenn Sie ernsthaft komponieren, werden deshalb die meisten elektronischen Klaviere ungenügend sein.

%: 8
\item Wenn Sie sich die Technik schnell aneignen können, hält Sie nichts davon ab, Ihren Horizont jenseits der klassischen Musik zu erweitern und Pop, Jazz, Blues usw. zu spielen.
Sie werden ein breiteres Publikum ansprechen, wenn Sie die Musikgenres mischen können, und es wird Ihnen mehr Spaß machen.
Die im elektronischen Klavier verfügbaren Begleitrhythmen, Schlagzeuge usw. können bei diesen Arten der Musik hilfreich sein.
Deshalb können diese zusätzlichen Fähigkeiten der elektronischen Klaviere sehr nützlich sein und sollten nicht ignoriert werden.
Elektronische Klaviere sind auch für Auftritte leichter zu transportieren.

%: 9
\item Ein elektronisches Klavier zu kaufen ist ziemlich einfach, besonders wenn man es mit dem \hyperref[c1iii17e]{Kauf eines akustischen Klaviers} vergleicht (s.u.).
Alles, was Sie wissen müssen, ist Ihre Preisspanne, die benötigte Ausstattung und den Hersteller.
Sie brauchen keinen erfahrenen Klaviertechniker, der Ihnen bei der Bewertung des Klaviers hilft.
Es stellt sich nicht die Frage, ob der Klavierhändler alle vorbereitenden Arbeiten am Klavier ausgeführt hat bzw. ausführen ließ, ob der Händler die Vereinbarung einhält, dafür zu sorgen, daß das Klavier nach der Lieferung einwandfrei funktioniert, ob das Klavier während des ersten Jahres richtig \enquote{stabilisiert} wurde oder ob Sie eines mit gutem oder minderwertigem Klang und Anschlag bekommen haben.
Viele renommierte Hersteller, wie Yamaha, Roland, Korg, Technics, Kawai und Kurzweil, produzieren elektronische Klaviere exzellenter Qualität.

%: 10
\item Und das ist nur der Anfang; die elektronischen Klaviere verbessern sich im Laufe der Zeit sprunghaft.
Eine neue Entwicklung ist die Modellierung des Klaviers (z.B. von Pianoteq, siehe www.pianoteq.com), anstelle des Samplens, das zuvor benutzt wurde.
Ein gutes Sampling erfordert eine enorme Menge an Speicher und Rechenleistung, was das Ansprechverhalten des Klaviers verlangsamen kann.
Die Modellierung ist vielseitiger und gestattet Dinge, die man nicht einmal auf einem Flügel machen kann, wie z.B. teilweise getretenes Dämpferpedal, Kontrolle der Biegung des Hammerschafts oder auf Chopins Pleyel zu spielen.

%: 11
\item Wir sollten von der \hyperref[c2_6_et]{gleichschwebenden Stimmung} zu den \hyperref[c2_2_wtk2]{wohltemperierten Stimmungen} (siehe im \hyperref[c2_1]{Kapitel über das Klavierstimmen}) übergehen.
Wenn Sie sich dazu entschlossen haben, die wohltemperierten Stimmungen zu verwenden, benötigen Sie mehrere davon.
Zu lernen, die Farbe einer Tonart zu erkennen und hervorzubringen ist eine sehr wertvolle Fertigkeit.
Die gleichschwebende Stimmung ist dafür am wenigsten geeignet.
Bei elektronischen Klavieren können Sie die meisten der verbreiteten wohltemperierten Stimmungen erhalten.

 \end{enumerate}

\label{touchweight}

Das dynamische Spielgewicht (touch weight) eines Klaviers wird nicht einfach dadurch verändert, daß man Bleigewichte von den Tasten entfernt oder ihnen hinzufügt, um die zum Niederdrücken der Tasten notwendige Kraft zu verändern.
Das dynamische Spielgewicht ist eine Kombination aus dem statischen Spielgewicht (down weight) zur Überwindung der Trägheit der Taste und des Hammers, und der Kraft, die notwendig ist, um einen Ton mit einer bestimmten Lautstärke zu erzeugen.
Das statische Spielgewicht ist das maximale Gewicht, dem die Taste widersteht, bevor sie anfängt, sich abwärts zu bewegen.
Das ist das Gewicht, das mit Bleigewichten justiert wird.
Das statische Spielgewicht aller Klaviere, einschließlich der elektronischen mit \enquote{gewichteter Tastatur}, beträgt in der Regel ungefähr 50 Gramm und variiert geringfügig von Klavier zu Klavier, unabhängig vom dynamischen Spielgewicht.
Wenn man ein Klavier spielt, sind diese 50 Gramm ein kleiner Teil der Kraft, die zum Spielen erforderlich ist - der größte Teil der Kraft wird zum Erzeugen des Tons benutzt.
Bei akustischen Klavieren ist das die Kraft, die notwendig ist, um den Hammer auf Geschwindigkeit zu bringen.
In elektronischen Klaviere ist es die elektronische Reaktion auf die Tastenbewegung und ein fester mechanischer Widerstand.
In beiden Fällen muß man, zusätzlich zum Aufbringen der für die Erzeugung des Tons notwendigen Kraft, auch die Trägheit des Mechanismus überwinden.
Wenn man z.B. staccato spielt, wird der größte Teil der Kraft zur Überwindung der Trägheit benötigt, während beim Legatospiel die Trägheitskomponente klein ist.
Elektronische Klaviere haben eine kleinere Trägheitskomponente, weil sie nur die Trägheit der Tasten haben, während die akustischen Klaviere zusätzlich die Trägheit der Hämmer haben; das macht die akustischen weniger empfindlich für das versehentliche Drücken von Tasten.
Deshalb werden Sie den größten Unterschied zwischen akustischen und elektronischen Klavieren fühlen, wenn Sie schnell oder staccato spielen und wenig Unterschied, wenn Sie legato spielen.
Für den Klavierspieler ist das dynamische Spielgewicht nur die Kraft, die erforderlich ist, um eine bestimmte Lautstärke des Tons zu erzeugen und hat wenig mit dem statischen Spielgewicht zu tun.
Bei akustischen Klavieren wird das dynamische Spielgewicht hauptsächlich von der Masse und dem \hyperref[c2_7_hamm]{Intonieren der Hämmer} (Härte) bestimmt.
Es gibt nur einen schmalen Bereich der Hammermassen, der ideal ist, weil man schwerere Hämmer für einen stärkeren Klang aber leichtere Hämmer für eine schnelle Mechanik möchte.
Deshalb kann ein großer Teil des dynamischen Spielgewichts vom Klaviertechniker eher durch das Intonieren der Hämmer als durch das Ändern der Gewichte justiert werden.
Bei elektronischen Klavieren wird das dynamische Spielgewicht auf folgende Weise durch die Software kontrolliert, um zu simulieren, was in einem Flügel geschieht.
Für ein höheres dynamisches Spielgewicht wird der Klang auf den eines weicheren Hammers umgeschaltet und umgekehrt.
Es erfolgt keine mechanische Veränderung des statischen Spielgewichts der Tasten oder der Trägheitskomponente.
Wenn Sie auf das höchste Spielgewicht umschalten, werden Sie deshalb den Klang eventuell als gedämpft empfinden, und wenn Sie auf das geringste Gewicht umschalten, könnte der Klang zu schrill sein.
Bei elektronischen Klavieren ist es einfacher, das dynamische Spielgewicht zu vermindern, ohne den Klang nachteilig zu beeinflussen, weil keine Hämmer bewegt werden müssen.
Auf der anderen Seite wird der maximale Dynamikumfang der meisten elektronischen Klaviere durch die Elektronik und die Lautsprecher begrenzt, so daß der Flügel für die lautesten Töne ein geringeres dynamisches Spielgewicht haben kann.
\textbf{Zusammengefaßt ist das dynamische Spielgewicht ein subjektives Urteil des Klavierspielers darüber, wieviel Kraft notwendig ist, um eine bestimmte Lautstärke zu erzeugen; es ist nicht das feste Gewicht oder der Widerstand der Tasten gegenüber dem Anschlag.}

Man kann diese subjektive Beurteilung demonstrieren, indem man die Lautstärke eines elektronischen Klaviers hoch- oder herunterdreht.
Wenn man längere Zeit auf einem elektronischen Klavier mit heruntergedrehter Lautstärke übt und dann auf einem  akustischen Klavier spielt, kann sich das akustische geradezu leicht anfühlen.
Leider sind die Dinge etwas komplizierter, denn wenn man ein elektronisches Klavier auf ein höheres Spielgewicht umschaltet, erzeugt es den Klang eines weicheren Hammers.
Um den Klang eines richtig intonierten Hammers zu erzeugen, muß man härter anschlagen.
Das ergibt zusammen die Wahrnehmung des höheren Spielgewichts, und dieser Effekt kann nicht durch ein Drehen am Lautstärkeregler simuliert werden.
Anhand dieser Überlegungen können wir folgende Schlüsse ziehen:
Es gibt geringe Unterschiede im Spielgewicht zwischen Flügeln und elektronischen Klavieren, wobei das der Flügel meistens höher ist, aber diese Unterschiede reichen nicht aus, um größere Probleme zu bereiten, wenn man vom einen zum anderen wechselt.
Deshalb ist die Befürchtung, daß das Üben auf einem elektronischen Klavier es erschwert, auf einem Flügel zu spielen, unbegründet; in Wahrheit ist es wahrscheinlich sogar einfacher, obwohl es eventuell ein paar Minuten dauern kann, bis man sich an das Spielen auf dem Flügel gewöhnt hat.

\textbf{Wenn Sie ein Anfänger sind und Ihr erstes Klavier kaufen möchten, ist ein\footnote{qualitativ gutes} elektronisches Klavier die offensichtliche Wahl, es sei denn, Sie können sich einen qualitativ guten Flügel leisten und haben den Platz dafür.}
Sogar in diesem Fall möchten Sie wahrscheinlich ein elektronisches Klavier, weil die Kosten im Vergleich zu einem Flügel gering sind und es Ihnen Ausstattungsmerkmale bietet, die ein Flügel nicht hat.
Die meisten akustischen aufrecht stehenden Klaviere sind nun obsolet, außer Sie sind bereit, Preise zu zahlen, die mit denen eines guten Flügels vergleichbar sind.


\subsubsection{Klaviere}
\label{c1iii17c}

Akustische Klaviere haben ihre eigenen Vorteile.
Sie sind weniger teuer als Flügel.
Sie benötigen weniger Platz, und für kleine Räume erzeugen große Flügel unter Umständen zuviel Schall, so daß sie bei ganz geöffnetem Deckel nicht mit voller Lautstärke gespielt werden können, ohne daß die \hyperref[c1iii10gehoer]{Ohren schmerzen oder geschädigt} werden.
Die elektronischen Klaviere haben jedoch dieselben, sowie viele weitere, Vorteile.
Besitzer eines Klaviers vernachlässigen zu häufig das \hyperref[c2_7_hamm]{Intonieren der Hämmer} völlig, da dieses Vernachlässigen zu einem stärkeren Ton führt.
Da Klaviere im Grunde geschlossene Instrumente sind, ist das Vernachlässigen des Intonierens weniger wahrnehmbar.
Klaviere sind meistens weniger teuer in der Wartung, hauptsächlich weil teure Reparaturen sich nicht lohnen und deshalb nicht durchgeführt werden.
Selbstverständlich gibt es qualitativ hochwertige Klaviere, die im Spielgefühl und in der Klangqualität mit Flügeln vergleichbar sind; aber ihre Zahl ist klein.

Unter den Klavieren sind die Kleinklaviere die mit der geringsten Höhe und im allgemeinen die billigsten; die meisten erzeugen keinen zufriedenstellenden Klang, auch für Schüler.
Die geringe Höhe der Kleinklaviere begrenzt die Saitenlänge, was die hauptsächliche Begrenzung der Schallerzeugung ist.
Theoretisch sollte der Diskantbereich einen ausreichenden Schall erzeugen (es gibt auch bei Kleinklavieren keine Einschränkung der Saitenlänge), aber die meisten Kleinklaviere sind im Diskant wegen der schlechten Qualität der Konstruktion schwach; testen Sie deshalb unbedingt die höheren Noten, wenn Sie ein Kleinklavier beurteilen - vergleichen Sie es einfach mit einem größeren Klavier.
Konsolenklaviere oder größere Klaviere können sehr gute Schülerklaviere sein.
Alte Klaviere mit schlechtem Klang sind im allgemeinen nicht zu retten, egal wie groß sie sind.
In einem solchen Alter ist der Wert des Klaviers geringer als die Kosten für das Restaurieren; es ist billiger, ein neueres Klavier mit einem zufriedenstellenden Klang zu kaufen.
\textbf{Die meisten Klaviere wurden von den elektronischen Klavieren überholt.
Deshalb gibt es keinen Grund, ein neues Klavier zu kaufen, obwohl einige Klavierlehrer und die meisten Verkäufer in den Klaviergeschäften etwas anderes vorschlagen.}
Viele Klavierlehrer haben nicht genug Erfahrung mit elektronischen Klavieren, sind mit dem Gefühl und dem Klang der akustischen Klaviere vertrauter und neigen dazu, die akustischen als \enquote{richtige Klaviere} zu empfehlen, was im allgemeinen ein Fehler ist.
Der geringe Unterschied im \enquote{Klang} (wenn es ihn gibt) wiegt die Schwierigkeit des Kaufs eines qualitativ guten Klaviers, die Probleme, die man oft damit hat, es vor und nach der Auslieferung richtig \enquote{herzurichten}, und die Notwendigkeit, es reguliert und in der Stimmung zu halten, nicht auf.


\subsubsection{Flügel}
\label{c1iii17d}

Die Vorteile der meisten Flügel sind: größerer Dynamikbereich (laut / leise), offene Struktur, die dem Schall gestattet, frei zu entweichen (was mehr Kontrolle und Ausdruck bietet), vollerer Klang, schnellere Repetierung, weichere Mechanik (Benutzung der Schwerkraft statt Federn), ein \enquote{wahres} Dämpferpedal (\hyperref[c1ii24]{s. Abschnitt II.24}), klarerer Klang (leichter exakt zu stimmen) und eine eindrucksvollere Erscheinung.
Eine Ausnahme bildet die Klasse der Stutzflügel (kleiner als ca. 5'-2``\footnote{ca. 1,57m}), deren erzeugter Klang üblicherweise nicht zufriedenstellend ist und die hauptsächlich als dekorative Möbelstücke gesehen werden sollten.
Ein paar Firmen (Yamaha, Kawai) beginnen damit, Stutzflügel mit akzeptablem Klang zu produzieren.\footnote{Ich habe die hier und im folgenden genannten Firmen jeweils ohne Prüfung der aktuellen Gegebenheiten aus dem Originaltext übernommen.
Falls sich eine Firma dadurch nicht angemessen berücksichtigt sieht, kann sie gerne mit dem Autor oder mir \hyperref[kontakt]{Kontakt} aufnehmen.}
Sie sollten also diese sehr neuen Flügel nicht abschreiben, ohne sie getestet zu haben.
Größere Flügel können in zwei Hauptklassen unterteilt werden: die \enquote{Schülerflügel}  (kleiner als ca. 6 bis 7 ft\footnote{ca. 1,83m - 2,13m}) und die Konzertflügel.
Die Konzertflügel bieten einen größeren Dynamikbereich, bessere Klangqualität und mehr Tonkontrolle.

Nehmen wir die Steinway-Flügel als ein Beispiel für dieses Thema \enquote{Qualität gegen Größe}:

\begin{itemize} 
\item Das Stutzflügelmodell, Modell S (5'-2``\footnote{ca. 1,57m}), ist im Grunde ein dekoratives Möbelstück, und sehr wenige erzeugen einen qualitativ genügenden Klang, um als spielbar angesehen zu werden, und sind vielen Klavieren unterlegen.
\item Die nächste Größengruppe besteht aus den Modellen M, O und L (5'-7`` bis 5'-11``\footnote{ca. 1,57m - 1,80m}).
Diese Modelle sind einander ziemlich ähnlich und exzellente Schülerklaviere.
Fortgeschrittene Klavierspieler würden sie jedoch wegen des geringeren Sustains, des zu stark perkussiven Klangs und den Noten mit zu hohem harmonischen Gehalt nicht als wahre Flügel betrachten.
\item Das nächste Modell, A (6'-2``\footnote{ca. 1,88m}), ist ein Grenzfall.
\item B (6'-10``\footnote{ca. 2,08m}), C (7'-5``\footnote{ca. 2,26m}) und D (9'\footnote{ca. 2,74m}) sind richtige Flügel.
 \end{itemize}
Ein Problem beim Beurteilen von Steinways ist, daß die Qualität innerhalb eines Modells sehr unterschiedlich ist; im Durchschnitt gibt es jedoch mit jeder Steigerung in der Größe eine deutliche Verbesserung der Klangqualität und -stärke.

Flügel erfordern ein häufigeres \hyperref[c2_7_hamm]{Intonieren der Hämmer} als Klaviere; sonst werden sie zu \enquote{brillant} oder \enquote{schrill}, und an diesem Punkt spielen die meisten Besitzer schließlich nur noch bei geschlossenem Deckel.
Viele Besitzer vernachlässigen das Intonieren völlig.
Das Ergebnis ist, daß solche Flügel zu viel und einen zu schrillen Klang erzeugen und deshalb mit geschlossenem Deckel gespielt werden.
Es ist aus technischer Sicht nichts falsch daran, einen Flügel mit geschlossenem Deckel zu spielen.
Einige Puristen werden jedoch über eine solche Praxis bestürzt sein, und man verschenkt sicherlich etwas wundervolles, für das man eine bedeutende Investition getätigt hat.
Vorführungen bei Konzerten erfordern fast immer, daß der Deckel offen ist, was dazu führt, daß der Flügel empfindlicher reagiert.
Deshalb sollten Sie vor einem Auftritt immer mit offenem Deckel üben, auch wenn Sie normalerweise bei geschlossenem Deckel üben.
In einem größeren Raum oder in einer Konzerthalle gibt es jedoch viel weniger Mehrfachreflexionen der Töne, so daß man nicht den ohrenbetäubenden Lärm hört, der in einem kleinen Raum daraus resultieren kann.
Eine Konzerthalle wird den Schall des Klaviers absorbieren, so daß man, wenn man gewohnt ist, in einem kleinen Raum zu üben, in einer Konzerthalle Schwierigkeiten haben wird, sein eigenes Spielen zu hören.

Einer der größten Vorteile von Flügeln ist die Ausnutzung der Schwerkraft als Kraft für das Zurückstellen der Hämmer.
Bei Klavieren wird die Rückstellkraft von Federn zur Verfügung gestellt.
Die Schwerkraft ist immer konstant und über die ganze Tastatur hinweg gleichförmig, während Ungleichmäßigkeiten in den Federn und Reibung Ungleichmäßigkeiten in dem Gefühl für die Tasten eines Klaviers erzeugen können.
Gleichmäßigkeit im Gefühl ist eine der wichtigsten Eigenschaften von gut eingestellten Qualitätsklavieren.
Viele Schüler sind von der Erscheinung großer Flügel bei Konzerten und Wettbewerben eingeschüchtert, aber diese Flügel sind in Wahrheit leichter zu spielen als Klaviere.
Eine Furcht, die diese Schüler in bezug auf jene Flügel haben, ist, daß deren Mechanik schwerer sei.
Das Spielgewicht wird jedoch durch den Techniker, der das Piano einstellt, reguliert und kann sowohl bei einem Flügel als auch bei einem Klavier auf jeden Wert eingestellt werden.
Fortgeschrittene Schüler werden es natürlich leichter finden, anspruchsvolle Stücke auf einem Flügel als auf einem Klavier zu spielen; hauptsächlich wegen der schnelleren Mechanik und der Gleichmäßigkeit.
Folglich können Flügel eine Menge Zeit sparen, wenn man versucht, fortgeschrittene Fertigkeiten zu erwerben.
Der Hauptgrund dafür ist, daß es leicht ist, schlechte Angewohnheiten zu entwickeln, wenn man auf Klavieren mit schwierigem Material kämpft.
Anspruchsvolles Material ist auf elektronischen Klavieren sogar noch schwieriger (und bei Modellen ohne richtiges Spielgewicht unmöglich), weil sie nicht die Robustheit und das Ansprechverhalten auf den Anschlag haben, die bei höheren Geschwindigkeiten erforderlich sind.

Einige Menschen mit kleinen Räumen zermartern sich den Kopf darüber, ob ein großer Flügel an so einem Platz zu laut wäre.
Lautstärke ist üblicherweise nicht das wichtigste Thema, und Sie haben immer die Option, den Deckel in unterschiedlichem Maß zu schließen.
Die maximale Lautstärke von mittleren und großen Flügeln ist nicht so unterschiedlich, und man kann mit den größeren Flügeln leiser spielen.
Es sind die Mehrfachreflexionen, die am lästigsten sind.
Mehrfachreflexionen können leicht durch einen Teppich auf dem Boden und durch die Schalldämmung einer oder zweier Wände eliminiert werden.
Wenn das Klavier von der Größe her ohne offensichtliche Schwierigkeiten in den Raum paßt, dann kann es deshalb hinsichtlich des Schalls akzeptabel sein.



<!-- c1iii17e.html -->

\subsubsection{Ein akustisches Klavier kaufen}
\label{c1iii17e}

Ein akustisches Klavier zu kaufen, kann für die Nichteingeweihten eine anstrengende Erfahrung sein, egal ob sie ein neues oder ein gebrauchtes kaufen.
Wenn man einen Händler mit einem guten Ruf finden kann, ist es gewiß sicherer ein neues zu kaufen, aber auch dann sind die Kosten für die anfängliche Wertminderung hoch.
Viele Klaviergeschäfte werden Ihnen ein Klavier mit einer Vereinbarung leihen, daß die Miete auf den Kaufpreis angerechnet wird, wenn Sie sich dafür entscheiden, es zu behalten.
In diesem Fall sollten Sie über den besten Kaufpreis verhandeln \textit{bevor} Sie über die Miete reden; wenn Sie sich bereits auf das Mieten geeinigt haben, haben Sie wenig Verhandlungsmöglichkeiten.
Sie werden am Ende einen höheren Anfangspreis haben, so daß der endgültige Preis, auch wenn Sie die Miete abziehen, kein günstiges Angebot ist.
Auch bei teuren Klavieren finden es viele Händler zu teuer, sie in gutem Zustand und gestimmt zu halten.
Bei diesen Händlern ist es schwierig, das Klavier durch Spielen zu testen.
Deshalb werden Klaviere oft aufs Geratewohl gekauft.
Bei in Massen produzierten Klavieren wie Yamaha oder Kawai ist die Qualität der neuen Klaviere meistens einheitlich, so daß man ziemlich genau weiß, was man bekommt.
Die Klangqualität der teureren \enquote{handgefertigten} Klaviere kann spürbar variieren, so daß es schwieriger ist, solch ein Klavier zu kaufen, wenn Sie ein gutes finden möchten.

Gute gebrauchte akustische Klaviere findet man nur schwer in Klaviergeschäften, weil spielbare Klaviere als erstes verkauft werden und die meisten Geschäfte auf einem ausgedehnten Inventar an unspielbaren sitzenbleiben.
\textbf{Offensichtlich findet man die besten Schnäppchen unter den privaten Angeboten.
Jemand, der sich nicht auskennt, wird sich einen Klavierstimmer oder -techniker engagieren müssen, um ein gebrauchtes Klavier aus dem privaten Markt zu bewerten.}
Man braucht auch jede Menge Geduld, weil gute private Angebote nicht immer dann zur Verfügung stehen, wenn man sie braucht.
Das Warten kann sich jedoch rentieren, weil das gleiche Klavier im privaten Verkauf voraussichtlich nur die Hälfte des Preises in einem Geschäft kosten wird (oder weiniger).
Es gibt eine ständige Nachfrage nach guten Klavieren, die einen vernünftigen Preis haben.
Das bedeutet, daß es nicht leicht ist, gute Angebote an gut zugänglichen Orten, wie Internet-Klaviermärkten, zu finden, weil gute Klaviere schnell verkauft sind.
Umgekehrt sind solche Orte hervorragend zum Verkaufen, besonders wenn man ein gutes Klavier anbietet.
Der beste Ort, um gute Angebote zu finden, ist die Kleinanzeigensparte von Zeitungen, insbesondere in Großstädten.
Die meisten solcher Anzeigen werden am Freitag, Samstag oder Sonntag aufgegeben.

Nur wenige Markenklaviere \enquote{behalten ihren Wert}, wenn man sie viele Jahre besitzt.
Der Rest verliert schnell an Wert, so daß es keine lohnende Alternative ist, zu versuchen, sie Jahre nach dem Kauf wieder zu verkaufen.
\enquote{Ihren Wert behalten} bedeutet, daß ihr Wiederverkaufswert mit der Inflation Schritt hält; das bedeutet nicht, daß man sie mit Gewinn verkaufen kann.
D. h. wenn Sie ein Klavier für 1.000 Euro gekauft haben und es 30 Jahre später für 2.500 Euro verkaufen, haben Sie keinen Gewinn erzielt, wenn die Inflation über diese 30 Jahre hinweg im Durchschnitt etwas über 3\% betragen hat.
Außerdem haben Sie noch die Kosten für das Stimmen und die Wartung.
So ist es z.B. billiger, alle 30 bis 40 Jahre einen nagelneuen 7ft-Flügel von Yamaha zu kaufen, als einen neuen Steinway M zu kaufen und ihn alle 30 bis 40 Jahre zu restaurieren; deshalb ist die Wahl, welches Klavier sie kaufen, keine Frage der Wirtschaftlichkeit, sondern hängt davon ab, welche Art von Klavier Sie benötigen.
Von wenigen Ausnahmen abgesehen, sind Klaviere keine gute Investition; man muß ein erfahrener Klaviertechniker sein, um auf dem Gebrauchtklaviermarkt ein Schnäppchen zu finden, das mit Gewinn verkauft werden kann.
Selbst wenn Sie so ein Schnäppchen finden, ist Klaviere zu verkaufen eine zeit- und arbeitsintensive Aufgabe.
Ziehen Sie für nähere Einzelheiten darüber, wie man ein Klavier kauft, das Buch von Larry Fine zu Rate.
Auch bei den berühmtesten Marken wird ein neu gekauftes Klavier bei der Auslieferung bereits 20 bis 30\% seines Kaufpreises verlieren, und wird im allgemeinen nach 5 Jahren nur noch die Hälfte eines vergleichbaren neuen Klaviers wert sein.
Als grobe Regel wird ein gebrauchtes Klavier in einem Klaviergeschäft ungefähr die Hälfte eines neuen Klaviers desselben Modells kosten und von Privat ungefähr ein Viertel.

Die Preise der Klaviere lassen sich grob danach ordnen, ob die Klaviere es wert sind, neu aufgebaut zu werden.
Jene die es wert sind, kosten meistens das Doppelte, wenn sie neu sind.
Praktisch alle Klaviere und alle Flügel, die in Massen produziert werden (Yamaha, Kawai usw.), werden nicht wieder aufgebaut, weil die Kosten ungefähr genauso hoch sind wie der Preis für ein neues Klavier desselben Modells.
Solche Klaviere wieder aufzubauen ist oft unmöglich, weil der Handel und die notwendigen Teile für den Wiederaufbau nicht existieren.
Klaviere, bei denen sich der Wiederaufbau lohnt, sind die von Steinway, Bösendorfer, Bechstein, Mason und Hamlin, einige von Knabe und ein paar andere.
Grob gesagt kostet der Wiederaufbau ungefähr 1/4 des Preises eines neuen Klaviers, und der Wiederverkaufswert ist ungefähr die Hälfte eines neuen; deshalb können sich die Kosten sowohl für den Restaurator als auch für den Käufer rechnen.


\subsubsection{Pflege und Wartung des Klaviers}
\label{c1iii17f}

Alle neuen Klaviere müssen nach dem Kauf mindestens ein Jahr speziell gepflegt und gestimmt werden, damit die Spannung der Saiten nicht mehr nachläßt und die Mechanik und die Hämmer sich ausbalancieren.
Die meisten Klavierhändler versuchen, die Kosten für die Pflege des Klaviers nach der Auslieferung zu minimieren.
Das setzt voraus, daß das Klavier vor der Lieferung gut vorbereitet wurde.
Viele Händler verschieben einen großen Teil der vorbereitenden Arbeiten auf die Zeit nach dem Kauf, und wenn der Käufer nichts darüber weiß, lassen sie einige Schritte eventuell ganz weg.
In dieser Hinsicht ist es bei den weniger teueren Modellen leichter, eines von Yamaha, Kawai, Petroff und ein paar anderen zu kaufen, weil das meiste der vorbereitenden Arbeiten bereits in der Fabrik durchgeführt wird.
Ein neues Klavier muß im ersten Jahr mindestens viermal gestimmt werden, damit sich die Spannung der Saiten stabilisiert.

Alle Klaviere erfordern zusätzlich zum regelmäßigen Stimmen eine Wartung.
Je besser die Qualität des Klaviers ist, desto leichter ist es im allgemeinen, die Verschlechterung, die durch normalen Verschleiß verursacht wird, zu erkennen, und deshalb sollte es auch mehr gewartet werden.
D.h. teurere Klaviere sind teurer im Unterhalt.
Typische Wartungsarbeiten sind: die Tasten richten, die Reibung reduzieren (z.B. die Piloten polieren), zusätzliche Töne eliminieren, die Hämmer in Form bringen und sie intonieren (nadeln), die unzähligen Buchsen überprüfen usw.
Die \hyperref[c2_7_hamm]{Hämmer zu intonieren} ist wahrscheinlich die am meisten vernachlässigte Wartungsarbeit.
Abgenutzte, harte Hämmer können einen Saitenbruch, den Verlust der musikalischen Kontrolle und ein erschwertes leises Spielen verursachen (die letzten zwei Punkte sind schlecht für die technische Entwicklung).
Sie ruinieren auch die Klangqualität des Klaviers, machen es schrill und unangenehm für das Ohr.
Wenn die Mechanik genügend abgenutzt ist, braucht sie eventuell eine Generalüberholung, d.h. alle Teile der Mechanik werden wieder gemäß der ursprünglichen Spezifikation hergerichtet.

Wenn die drahtumwickelten Baßsaiten rostig sind, kann das diese Noten absterben lassen.
Diese Saiten zu ersetzen kann sich sehr lohnen, wenn diese Baßnoten schwach sind und keinen Sustain haben.
Die oberen, nicht umwickelten Saiten müssen im allgemeinen nicht ersetzt werden, auch wenn sie rostig sind.
Bei extrem alten Klavieren können diese Saiten jedoch so auseinandergezogen sein, daß sie ihre ganze Elastizität verloren haben.
Solche Saiten sind anfällig für Brüche, können nicht richtig schwingen, erzeugen einen blechernen Klang und sollten ersetzt werden.

Klavierspieler sollten sich mit etwas Grundwissen über das \hyperref[c2_1]{Stimmen} vertraut machen, wie z.B. den Teilen eines Klaviers, Stimmungen, Stabilität der Stimmung und Auswirkungen von Temperatur- und Luftfeuchtigkeitsänderungen, damit sie in der Lage sind, sich mit einem Stimmer zu unterhalten und zu verstehen, was er tun muß.
Zu viele Klavierbesitzer wissen nichts über diese Grundlagen; infolgedessen frustrieren sie den Stimmer und arbeiten in Wahrheit gegen ihn, mit dem Ergebnis, daß das Klavier nicht richtig gewartet wird.
Einige Besitzer gewöhnen sich so sehr an ihr \enquote{verfallenes} Klavier, daß Sie, wenn der Stimmer eine gute Arbeit dabei leistet, dem Klavier wieder seinen ursprünglichen Glanz zu verleihen, sehr unglücklich mit dem fremdartigen neuen Klang und Gefühl des Klaviers sind.
Abgenutzte Hämmer neigen dazu, übermäßig helle und laute Töne zu erzeugen; das hat den unerwarteten Effekt, daß sich die Mechanik leicht anfühlt.
Deshalb können richtig intonierte Hämmer am Anfang den Eindruck erwecken, daß die Mechanik nun schwerer ist und weniger gut anspricht.
Natürlich hat der Stimmer nicht die Kraft geändert, die notwendig ist, um die Tasten niederzudrücken.
Haben sich die Besitzer erst einmal an die neu intonierten Hämmer gewöhnt, werden sie finden, daß sie eine viel bessere Kontrolle über den Ausdruck und den Ton haben, und daß sie nun sehr leise spielen können.

Klaviere müssen mindestens einmal im Jahr gestimmt werden; besser wäre zweimal, während des Frühjahrs und im Herbst, wenn die Temperatur und die Luftfeuchtigkeit in der Mitte zwischen ihren jährlichen Extremen sind.
Viele fortgeschrittene Klavierspieler lassen sie sogar öfter stimmen.
Zusätzlich zu den offensichtlichen Vorteilen, daß man in der Lage ist bessere Musik zu erzeugen und seine Musikalität schärft, gibt es viele zwingende Gründe, das Klavier gestimmt zu halten.
Einer der wichtigsten ist, daß es Ihre technische Entwicklung beeinflussen kann.
Verglichen mit einem verstimmten Klavier spielt sich ein gut gestimmtes Klavier wie von selbst - Sie werden es überraschend leichter finden, es zu spielen.
Deshalb kann ein gestimmtes Klavier tatsächlich Ihre technische Entwicklung beschleunigen.
Ein verstimmtes Klavier kann zu Spielfehlern führen und zur Angewohnheit zu stottern, d.h. bei jedem Fehler anzuhalten.
Viele wichtige Aspekte des Ausdrucks lassen sich nur auf einem gut gestimmten Klavier richtig herausarbeiten.
Da wir stets darauf bedacht sein müssen, \hyperref[c1iii14d]{musikalisch zu üben}, macht es keinen Sinn, auf einem Klavier zu üben, das keine richtige Musik erzeugen kann.
Das ist einer der Gründe, warum ich \hyperref[c2_2_wtk2]{wohltemperierte Stimmungen} (mit ihren kristallklaren Intervallen) der \hyperref[c2_6_et]{gleichmäßigen Stimmung} vorziehe, in welcher nur die Oktaven rein sind.
Sehen Sie dazu in \hyperref[c2_1]{Kapitel 2} mehr über die Vorzüge der verschiedenen Stimmungen.
Klaviere höherer Qualität haben einen eindeutigen Vorteil, weil sie nicht nur die Stimmung besser halten, sondern auch genauer gestimmt werden können.
Klaviere niedrigerer Qualität haben oft zusätzliche Schwebungen und Töne, die ein genaues Stimmen unmöglich machen.

Diejenigen, die ein \hyperref[c1iii12]{absolutes Gehör} haben, haben mit verstimmten Klavieren große Schwierigkeiten.
Wenn man das absolute Gehör hat, können sehr verstimmte Klaviere den altersbedingten schrittweisen Verlust des absoluten Gehörs beschleunigen.
Babys und sehr junge Kinder können das absolute Gehör automatisch erwerben, wenn sie den Klang des Klaviers oft genug hören, auch wenn sie keine Vorstellung davon haben, was das absolute Gehör ist.
Damit sie das richtige absolute Gehör erwerben, muß das Klavier gestimmt sein.

Wenn Sie immer auf einem gestimmten Klavier üben, werden Sie es schwer haben, auf einem verstimmten zu spielen.
Die Musik kommt nicht heraus, man macht unerwartete Fehler und hat Gedächtnisblockaden.
Das trifft auch dann zu, wenn man nichts über das Stimmen weiß und nicht sagen kann, ob eine einzelne Note verstimmt ist.
Für einen Klavierspieler ohne Erfahrung im Stimmen ist ein Stück zu spielen der beste Weg, die Stimmung zu testen.
Eine gute Stimmung ist für jeden Klavierspieler phantastisch.
Durch das Spielen eines Musikstücks können die meisten Klavierspieler leicht den Unterschied zwischen einer schlechten und einer ausgezeichneten Stimmung hören, sogar wenn sie nicht den Unterschied durch das Spielen einzelner Noten oder das Testen von Intervallen angeben können (unter der Annahme, daß sie nicht auch Klavierstimmer sind).
Deshalb muß jeder Klavierspieler, neben der technischen Entwicklung, lernen, die Vorteile einer guten Stimmung zu hören.
Es ist vielleicht eine gute Idee, ab und zu auf einem verstimmten Klavier zu üben, damit man weiß, was einen erwartet, wenn man gebeten wird, auf einem Klavier mit zweifelhafter Stimmung zu spielen.
Bei Konzerten sollte das Konzertklavier direkt vor dem Konzert gestimmt werden, so daß das Konzertklavier eine bessere Stimmung hat als das Übungsklavier.
Versuchen sie, den umgekehrten Fall zu vermeiden, bei dem das Übungsklavier besser gestimmt ist als das Konzertklavier.
Das ist ein weiterer Grund, warum Schüler, die auf preisgünstigen Klavieren üben, wenig Probleme damit haben, auf großen, ungewohnten Flügeln zu spielen, solange die Flügel gestimmt sind.

Insgesamt gesehen sind Flügel für die technische Entwicklung ungefähr bis zur Mittelstufe nicht notwendig, obwohl sie in jeder Stufe nützlich sind.
Oberhalb der Mittelstufe werden die Argumente, die Flügel gegenüber Klavieren favorisieren, stichhaltiger.
Flügel sind besser, weil ihre Mechanik schneller ist, sie genauer gestimmt werden können, einen größeren Dynamikumfang haben, über ein wahres Dämpferpedal verfügen, mehr Kontrolle über Ausdruck und Klang gestatten können (man kann den Deckel öffnen) und so eingestellt werden können, daß sie eine größere Gleichmäßigkeit der Noten bieten (durch den Gebrauch der Schwerkraft statt von Federn).
Diese Vorteile sind jedoch zunächst verglichen mit der Liebe des Schülers zur Musik, seinem Fleiß und den korrekten Übungsmethoden gering.
Flügel werden für fortgeschrittene Schüler wünschenswerter, weil technisch herausforderndes Material auf einem Flügel leichter auszuführen ist.
Für diese fortgeschrittenen Klavierspieler werden das richtige \hyperref[c2_1]{Stimmen}, das Einstellen des Klaviers und das \hyperref[c2_7_hamm]{Intonieren der Hämmer} wesentlich, denn wenn die Wartung des Klaviers vernachlässigt wird, gehen die ganzen Vorteile praktisch verloren.
 

\label{c1iii17g}
\label{digital}

\footnote{Die folgenden Absätze sind wieder eine Einfügung, die ich wegen der besseren Lesbarkeit in normaler Schrift gelassen habe.}


\subsubsection{Anmerkungen zu Digitalpianos}

An dieser Stelle muß ich noch einmal eine Lanze für die Digitalpianos brechen.
Klar hat Chuan C. Chang Recht, daß fortgeschrittene Techniken nur auf einem akustischen Klavier richtig zu lernen und anzuwenden sind.
Wer also höhere Ambitionen hat, der sollte auf alle Fälle ein qualitativ hochwertiges akustisches Klavier oder besser einen Flügel kaufen und regelmäßig stimmen und warten lassen.
Für alle anderen Klavierspieler (mich eingeschlossen) ist ein gutes - wohlgemerkt ein gutes! - Digitalpiano völlig ausreichend.
Es gibt mittlerweile einige Digitalpianos, die hinsichtlich des Ansprechverhaltens und Spielgefühls (Stichwort \textit{gewichtete Hammertastatur}) nicht oder kaum noch von den akustischen Klavieren zu unterscheiden sind, die man sich im allgemeinen gönnt.
Bis jetzt hatte ich jedenfalls nie Schwierigkeiten, bei Bekannten \enquote{mal was vorzuspielen}.

Vorteile von Digitalpianos sind z.B. die sehr geringen Unterhaltskosten (kein Stimmen, normalerweise selten bis nie Wartung oder Reparaturen, im Grunde nur ein paar Cent für den Strom), daß man mit Kopfhörern üben kann ohne jemanden zu stören und sie fast alle MIDI-fähig sind (s. Anmerkungen zum \hyperref[c1iii13MIDI]{Aufnehmen}).
Der Preis eines Digitalpianos ist meistens wesentlich niedriger als der eines akustischen Klaviers, hängt aber auch stark von der Optik ab.
Ein Digitalpiano im hochglanzpolierten Holz(imitat)gehäuse ist gewöhnlich teurer als ein \enquote{Stage-Piano} auf einem möglichst stabilen Keyboardständer aber deshalb nicht zwangsläufig auch technisch besser.

Digitalpianos bieten mehrere Klänge: neben diversen Klavieren, Flügeln und Orgeln teilweise auch völlig andere Instrumente wie z.B. Streich- und Blasinstrumente, Synthesizerklänge und Schlagzeuge.
Die Qualität der einzelnen Klänge ist sehr unterschiedlich.
Teilweise sind sie wirklich erstklassig, teilweise von den Herstellern anscheinend nur als Zugabe gedacht; letzteres gilt vor allem für die Klänge, die keine Klaviere oder Flügel sind.
Bei manchen Geräten lassen sich die Klänge auch noch verändern (Hüllkurven, Effektgeneratoren usw.), bzw. \enquote{layern}, d.h. die einzelnen Noten werden mit mehreren verschiedenen Klängen gleichzeitig wiedergegeben.
Dadurch kann man interessante Effekte erzielen und zum Teil sogar die Klänge verbessern.
Sehr gut ist, wenn man dann seine ermittelten Einstellungen noch als \enquote{Preset} speichern kann, damit man beim nächsten Einschalten nicht wieder von vorne anfangen muß, bzw. auf der Bühne schnell umschalten kann.


\footnote{Ende der Einfügung.}



<!-- c1iii18.html -->

\subsection{Wie man das Klavierspielenlernen beginnt - vom jüngsten Kind bis zum ältesten Erwachsenen}
\label{c1iii18}

\subsubsection{Benötigt man einen Lehrer?}
\label{c1iii18a}

Viele Anfänger möchten sich das Klavierspielen gerne selbst beibringen, und es gibt viele stichhaltige Gründe dafür.
Es steht jedoch völlig außer Frage, daß es für die ersten sechs Monate (und wahrscheinlich viel länger) keinen schnelleren Weg zum Anfangen gibt, als Stunden bei einem Lehrer zu nehmen, sogar bei einem, der die intuitive Methode lehrt.
Die einzigen Lehrer, die man gänzlich meiden sollte, sind diejenigen, die nicht das lehren können, was man spielen möchte (z.B. wenn Sie Pop, Jazz oder Blues spielen möchten, während der Lehrer nur klassische Musik unterrichtet), oder diejenigen, die strenge, inflexible Methoden lehren, die für den Schüler nicht angemessen sind (eine Methode kann für sehr junge Kinder entwickelt worden sein, aber Sie sind ein älterer Anfänger).
Warum sind Lehrer am Anfang so hilfreich?
Erstens sind die grundlegendsten Dinge, die Sie jedesmal beim Spielen benutzen, wie \hyperref[c1ii2]{Haltung der Hand}, \hyperref[c1ii3]{Sitzposition}, \hyperref[c1iii4]{Handbewegungen} usw., in einem Lehrbuch schwer zu erklären, während Ihnen ein Lehrer sofort zeigen kann, was richtig und was falsch ist.
Sie möchten sich keine dieser falschen Angewohnheiten aneignen und Ihr ganzes Leben damit zurechtkommen müssen.
Zweitens macht ein Anfänger, der sich ans Klavier setzt und zum ersten Mal spielt, mindestens 20 Fehler gleichzeitig (\hyperref[c1ii25]{Koordination der rechten und linken Hand}, Kontrolle der \hyperref[c1iii14d]{Lautstärke}, \hyperref[c1iii1b]{Rhythmus}, Arm- und \hyperref[c1iii4c]{Körperbewegungen}, \hyperref[c1ii13]{Geschwindigkeit}, Timing, \hyperref[c1ii18]{Fingersatz}, der Versuch das Falsche zuerst zu lernen, völliges Vernachlässigen der \hyperref[c1iii14d]{Musikalität} usw.).
Es ist die Aufgabe des Lehrers, alle Fehler zu erkennen und eine gedankliche Prioritätenliste derer zu erstellen, die als erste korrigiert werden müssen, so daß die schlimmsten schnell beseitigt werden können.
Die meisten Lehrer wissen auch, welche grundlegenden Fertigkeiten man benötigt, und lehren sie in der richtigen Reihenfolge.
Lehrer sind auch dabei hilfreich, das richtige Lehrmaterial zu finden.
Lehrer sorgen für eine strukturierte Lernumgebung, ohne die ein Student am Ende die falschen Dinge tut und nicht merkt, daß er keinen Fortschritt macht.
Kurz gesagt: Lehrer sind für Anfänger definitiv ihr Geld wert.


\subsubsection{Bücher für Anfänger; Keyboards}
\label{c1iii18b}

Wenn man anfängt, ist die Auswahl der Lehrbücher der erste Tagesordnungspunkt.
Diejenigen, die mit dem Erlernen der allgemeinen Technik anfangen möchten (kein Spezialgebiet wie Jazz oder Gospel), können jedes der zahlreichen Bücher für Anfänger, wie Michael Aaron, Alfred, Bastien, Faber und Faber, Schaum oder Thompson, benutzen.
Von diesen bevorzugen viele Faber und Faber.
Die meisten haben Bücher für Anfänger, die für Kinder oder Erwachsene entwickelt wurden.\footnote{Erfahrungsberichte und Vorschläge für den deutschen Sprachraum lasse ich gerne hier einfließen.}
Es gibt eine exzellente Website für Klavier (www.amsinternational.org/piano_pedagogy.htm), welche die meisten dieser Lehrbücher auflistet und viele davon bespricht.
In Abhängigkeit von Ihrem Alter und Ihrer bisherigen musikalischen Ausbildung können Sie diese Bücher in Ihrem eigenen Tempo durchgehen und Ihre Lernrate optimieren.

Diese Bücher für den Anfang werden Ihnen die Grundlagen beibringen: \hyperref[c1iii11]{Notenlesen}, verschiedene allgemeine \hyperref[c1ii18]{Fingersätze} wie \hyperref[c1iii5a]{Tonleitern}, \hyperref[Arpeggios]{Arpeggios} und Begleitungen usw.
Sobald Sie mit den meisten Grundlagen vertraut sind, können Sie damit beginnen, Stücke zu lernen, die Sie spielen möchten.
Hierbei sind Lehrer wieder äußerst wertvoll, weil sie die meisten Stücke kennen, die Sie vielleicht spielen möchten und 
Ihnen sagen können, ob sie zu der Schwierigkeitsstufe gehören, die Sie bewältigen können.
Sie können Ihnen die schwierigen Abschnitte herausstellen und Ihnen zeigen, wie Sie über diese Schwierigkeiten hinwegkommen.
Sie können Ihnen die Unterrichtsstücke vorspielen, um Ihnen zu zeigen, was sie versuchen müssen zu erreichen; meiden Sie Lehrer, die Ihnen nicht vorspielen können oder möchten.
Nach ein paar Monaten bis zu einem Jahr Unterricht werden Sie soweit sein, daß Sie mit dem Material im Buch weitermachen können.
Um die zahlreichen Fallen zu vermeiden, die auf sie lauern, sollten Sie das Buch zumindest einmal kurz durchlesen, bevor Sie mit der ersten Lektion beginnen.

Ganz am Anfang, vielleicht bis zu einem Jahr lang, ist es möglich, mit einem Keyboard zu lernen, auch mit einem kleineren mit weniger als den 88 Tasten des Standardklaviers.
Wenn Sie beabsichtigen, Ihr ganzes Leben lang elektronische Keyboards zu spielen, ist es sicherlich in Ordnung, wenn Sie nur auf Keyboards üben.
Im Grunde haben jedoch alle Keyboards eine Mechanik, die zu leicht ist, um wirklich ein \hyperref[c1iii17c]{akustisches Klavier} zu simulieren\footnote{\hyperref[c1iii17b]{elektronische Klaviere} (Digitalpianos) sind in dieser Hinsicht bedeutend besser als Keyboards}.
Sie werden so bald wie möglich auf ein Digitalpiano mit 88 gewichteten Tasten (oder ein akustisches Klavier) umsteigen wollen - siehe oben in \hyperref[c1iii17]{Abschnitt 17}.


\subsubsection{Anfänger im Alter von 0 bis über 65}
\label{c1iii18c}

Viele Eltern fragen: \enquote{In welchem Alter können unsere Kinder mit dem Klavierspielen beginnen?},
während ältere Anfänger fragen: \enquote{Bin ich zu alt, um das Klavierspielen zu lernen?
Wie gut werde ich spielen können?
Wie lange wird es dauern?}
Wir beginnen zunehmend zu erkennen, daß das, was wir dem \enquote{Talent} zugeschrieben haben, in Wahrheit das Ergebnis unserer Ausbildung war.
Diese relativ neue \enquote{Entdeckung} verändert die Landschaft der Klavierpädagogik radikal.
\textbf{Deshalb ist es legitim, wenn wir in Frage stellen, daß das Talent solch ein wichtiger Faktor dafür sei, wie schnell man das Spielen lernen könne.}
Was ist also ein wichtiger Faktor?
Das Alter ist einer, weil das Klavierspielenlernen ein Prozeß ist, bei dem - insbesondere im Gehirn - Nervenzellen gebildet werden.
Der Prozeß des Wachstums von Nervenzellen verlangsamt sich mit zunehmendem Alter.
Betrachten wir also die Kategorien der Anfänger und die Auswirkungen des sich verlangsamenden Zellwachstums in Abhängigkeit vom Alter.


\label{c1iii18c0}

\textbf{Alter von 0 bis 6:} Babys können hören, sobald sie geboren sind, und auf den meisten Geburtsstationen wir das Gehör der Babys gleich nach der Geburt getestet.
Die Gehirne tauber Babys entwickeln sich aufgrund des Mangels an akustischen Reizen langsam, und bei solchen Babys muß die Fähigkeit zur Verarbeitung akustischer Reize (wenn möglich) wieder hergestellt oder müssen andere Verfahren angewandt werden, um eine normale Entwicklung des Gehirns zu fördern.
Deshalb wird ein frühes musikalisches Stimulieren die Gehirnentwicklung normaler Babys beschleunigen, nicht nur hinsichtlich der Musik, sondern auch allgemein.
Im Alter von 6 bis 10 Monaten haben die meisten Babys eine genügende Menge Töne und Sprache gehört, die eine für den Beginn des Sprechens ausreichende Gehirnentwicklung stimulierte.
Sie können innerhalb weniger Minuten nach der Geburt schreien und mit uns kommunizieren.
Musik kann eine zusätzliche Stimulation bieten, um Babys innerhalb eines Jahres nach der Geburt einen enormen Vorsprung in der Gehirnentwicklung zu verschaffen.
Alle Eltern sollten eine gute Sammlung von Klaviermusik, Orchestermusik, Klavier- und Violinkonzerten, Opern usw. haben und sie im Zimmer des Babys oder irgendwo im Haus, wo das Baby die Musik immer noch hören kann, abspielen.
Viele Eltern flüstern und gehen leise, während das Baby schläft, aber das ist ein schlechtes Training.
Babys kann man darauf trainieren, in einer (normal) lauten Umgebung zu schlafen, und das ist die gesunde Alternative.

Bis zu einem Alter von ungefähr 6 Jahren erwerben sie neue Fertigkeiten schrittweise; d.h., sie erwerben plötzlich eine neue Fertigkeit wie das Laufen und werden schnell gut darin.
Aber jeder einzelne erwirbt diese Fertigkeiten zu unterschiedlichen Zeiten und in einer unterschiedlichen Reihenfolge.
Die meisten Eltern machen den Fehler, dem Baby nur Babymusik vorzuspielen.
Denken Sie daran: Kein Baby hat jemals Babymusik komponiert; Erwachsene haben es getan - Babymusik verlangsamt nur die Entwicklung des Gehirns.
Es ist keine gute Idee, sie lauten Trompeten und Schlagzeugen auszusetzen, die das Baby erschrecken können, aber Babys können Bach, Beethoven, Chopin usw. verstehen.
Musik ist ein erworbener Geschmack; deshalb hängt, wie sich das Gehirn der Babys musikalisch entwickelt, von der Art der Musik ab, die sie hören.
Ältere klassische Musik enthält mehr grundlegende Akkordstrukturen und Harmonien, die vom Gehirn auf natürliche Weise erkannt werden.
Später wurden komplexere Akkorde und Dissonanzen hinzugefügt, als wir im Laufe der Jahre vertrauter mit ihnen wurden.
Deshalb ist die ältere klassische Musik geeigneter für Babys, weil sie mehr stimulierende Logik und weniger Dissonanzen und Betonungen enthält, die später eingeführt wurden, um die \enquote{moderne Zivilisation} widerzuspiegeln.
Klaviermusik ist besonders geeignet, denn wenn sie schließlich Klavierunterricht nehmen, werden sie ein höheres Verständnis der Musik haben, die sie als Baby gehört haben.


\label{c1iii18c3}

\textbf{Alter von 3 bis 12:} Im Alter von weniger als drei Jahren sind die Hände der meisten Kinder für das Klavierspielen zu klein, die Finger können sich nicht unabhängig voneinander krümmen oder bewegen, und das Gehirn und der Körper (Stimmbänder, Muskel usw.) sind eventuell noch nicht genügend entwickelt, um die musikalischen Konzepte zu bewältigen.
Im Alter von mehr als vier Jahren können die meisten Kinder eine bestimmte Art von Musikerziehung erhalten, besonders wenn sie Musik seit der Geburt gehört haben; deshalb \textbf{sollten sie laufend auf ihr Verständnis der Tonhöhe (\hyperref[c1iii12]{relatives und absolutes Gehör}; können sie \enquote{eine Melodie halten}?), des \hyperref[c1iii1b]{Rhythmus}, von laut und leise, von langsam  und schnell, sowie des Notenlesens, das leichter zu lernen ist als jedes Alphabet, getestet werden}.
Diese Gruppe kann aus dem enormen Gehirnwachstum, das in dieser Altersspanne stattfindet, den vollen Nutzen ziehen;
das Lernen geschieht ohne Aufwand und wird eher von der Fähigkeit des Lehrers begrenzt, das entsprechende Material zu bieten, als durch die Fähigkeit des Schülers, das Material in sich aufzunehmen.
Ein bemerkenswerter Aspekt (es gibt viele!) dieser Altersgruppe ist ihre \enquote{Formbarkeit}; ihre \enquote{Talente} können geformt werden.
Deshalb können sie, auch wenn sie von alleine nie zu Musikern geworden wären, durch das richtige Training zu Musikern gemacht werden.
Das ist das ideale Alter, um mit dem Klavierspielen zu beginnen.
\hyperref[c1ii12mental]{Mentales Spielen} ist nichts besonderes - bei dieser Altersgruppe kommt es wie von selbst.
Viele Erwachsene halten das mentale Spielen für eine seltene Fertigkeit, weil sie es - wie das \hyperref[c1iii12]{absolute Gehör} - während ihrer Jugendzeit zu wenig benutzt und deswegen verloren haben.
Achten Sie deshalb darauf, daß die Schüler das mentale Spielen ausführen und die Musik in Gedanken spielen.


\label{c1iii18c13}

\textbf{Alter von 13 bis 19:} Die Jugendzeit.
Diese Gruppe hat immer noch eine ausgezeichnete Chance, die Stufe eines Konzertpianisten zu erreichen.
Sie haben aber eventuell die Chance verpaßt, zu diesen Superstars zu werden, zu denen die jüngeren Anfänger werden können.
Obwohl die Entwicklung des Gehirns sich verlangsamt hat, wächst der Körper ungefähr bis zum Alter von 16 Jahren weiterhin schnell und danach langsamer.
Die wichtigsten Faktoren sind hier die Liebe zur Musik und zum Klavier.
Mitglieder dieser Altersgruppe können praktisch alles erreichen, was sie möchten, solange sie ein intensives Interesse an der Musik haben.
Sie sind jedoch nicht mehr so formbar; sie anzuregen, das Klavierspielen zu lernen, funktioniert  nicht, wenn sie am Cello oder Saxophon mehr interessiert sind, und die Rolle der Eltern wechselt von der Vorgabe der Richtung hin zur Unterstützung dessen, was der Jugendliche tun möchte.
Das ist die Altersstufe, in der Jugendliche lernen, was es bedeutet, Verantwortung zu übernehmen und was es bedeutet, ein Erwachsener zu werden - alles Lektionen, die durch die Erfahrungen mit dem Klavier gelernt werden können.
Um sie zu beeinflussen, muß man fortgeschrittenere Methoden anwenden, wie z.B. Logik, Wissen und Psychologie.
Sie werden wahrscheinlich niemals etwas vergessen, das sie in diesem oder einem jüngeren Alter \hyperref[c1iii6]{auswendig gelernt} haben.
Oberhalb dieser Altersstufe wird die Einteilung nach dem Alter schwierig, weil es zwischen den einzelnen Menschen so viele Unterschiede gibt.


\label{c1iii18c20}

\textbf{Alter von 20 bis 35:}
Einige Mitglieder dieser Altersgruppe haben immer noch die Chance, die Stufe eines Konzertpianisten zu erreichen.
Sie können die Erfahrungen, die sie im Leben gemacht haben, nutzen, um sich die Fertigkeiten für das Klavierspielen effektiver als jüngere Schüler anzueignen.
Diejenigen, die sich in diesem Alter dafür entscheiden, das Klavierspielen zu lernen, haben im allgemeinen eine größere Motivation und eine klarere Vorstellung davon, was sie wollen.
Aber sie werden sehr hart arbeiten müssen, weil der Fortschritt sich nur mit einem ausreichenden Maß an Arbeit einstellt.
In dieser Altersgruppe kann die \hyperref[c1iii15]{Nervosität} für einige zu einem großen Problem werden.
Obwohl jüngere Schüler nervös werden können, scheint die Nervosität im Laufe der Jahre zuzunehmen.
Das geschieht, weil eine starke Nervosität aus der Angst zu versagen resultiert, und die Angst erwächst aus Assoziationen mit Erinnerungen an schreckliche Erlebnisse, egal ob diese real sind oder nur in der Vorstellung existieren.
Diese schrecklichen Erinnerungen oder Vorstellungen sammeln sich im Laufe der Zeit an.
Wenn Sie \hyperref[c1iii14]{vorspielen} möchten, sollten Sie sich deshalb mit der Kontrolle der Nervosität auseinandersetzen, z.B. indem Sie selbstsicherer werden oder indem Sie bei jeder Gelegenheit das öffentliche Vorspielen üben usw.
Die Nervosität kann sowohl aus dem Bewußtsein als auch dem Unterbewußtsein kommen; deshalb werden Sie sich mit beiden befassen müssen, um zu lernen, die Nervosität zu kontrollieren.
Diejenigen, die nur technisch versiert genug werden möchten, um das Spielen der großen Klavierwerke zu genießen, sollten, wenn sie in dieser Altersstufe beginnen, keine Probleme haben.
Obwohl einige \hyperref[c1iii6c]{Pflege} notwendig sein wird, können Sie alles ein Leben lang behalten, was Sie in dieser Altersstufe auswendig gelernt haben.


\label{c1iii18c35}

\textbf{Alter von 35 bis 45:}
Mitglieder dieser Altersgruppe können sich nicht zu Konzertpianisten entwickeln, können aber für einfacheres Material, wie leichte Klassik und Cocktail-Musik (\enquote{Fake Books}, Jazz), gut genug \hyperref[c1iii14]{vorspielen}.
Sie können genug Fertigkeiten erwerben, um die meisten berühmten Kompositionen zur eigenen Freude oder bei informellen Auftritten zu spielen.
Das anspruchsvollste Material wird wahrscheinlich außer Reichweite sein.
Die \hyperref[c1iii15]{Nervosität} erreicht irgendwo im Alter von 40 bis 60 Jahren ein Maximum und nimmt danach oft langsam ab. 
Das mag erklären, warum viele berühmte Pianisten irgendwann in dieser Altersstufe mit dem Auftreten aufgehört haben.
Das \hyperref[c1iii6]{Auswendiglernen} beginnt in dem Sinne zu einem Problem zu werden, daß es zwar möglich ist, praktisch alles auswendig zu lernen, man aber dazu neigt, es fast völlig zu vergessen, wenn man es nicht richtig \hyperref[c1iii6c]{pflegt}.
Das Notenlesen kann für einige zum Problem werden, die stark korrigierende Gläser benötigen, weil sich der Abstand der Augen zur Tastatur oder dem Notenständer zwischen dem Abstand zum Lesen und dem Fernblick befindet.
Deshalb benötigen Sie vielleicht eine Brille für den Zwischenbereich.
Gleitsichtgläser könnten das Problem lösen, aber manche finden sie wegen ihres kleinen Fokussierbereichs lästig.


\label{c1iii18c45}

\textbf{Alter von 45 bis 65:}
Das ist das Alter, in dem es je nach der Person zunehmende Einschränkungen dafür gibt, was man spielen lernen kann.
Man kann wahrscheinlich bis zur Stufe der Beethoven-Sonaten kommen, obwohl die schwierigsten eine große Herausforderung sein werden und es mehrere Jahre dauern wird, sie zu lernen.
Sich ein genügend großes Repertoire anzueignen wird schwierig sein, und man wird immer nur ein paar Stücke \hyperref[c1iii14]{vorspielen} können.
Aber es gibt unzählige Kompositionen, die man zur eigenen Freude spielen kann.
Da es mehr wunderbare Kompositionen zu lernen gibt, als man Zeit zum Lernen hat, muß man nicht unbedingt das Gefühl haben, daß man bei dem, was man spielen möchte, eingeschränkt ist.
Es gibt noch immer keine großen Probleme beim Lernen neuer Stücke, aber man muß sie ständig \hyperref[c1iii6c]{pflegen}, wenn man sie in seinem Repertoire behalten möchte.
Das wird Ihr spielbares Repertoire einschränken, da Sie beim Lernen neuer Stücke die alten völlig vergessen, wenn Sie diese nicht in viel jüngerem Alter gelernt haben.
Außerdem wird Ihre Lernrate definitiv anfangen abzunehmen.
Durch das mehrfache Vergessen und erneute \hyperref[c1iii6]{Auswendiglernen} können Sie trotzdem eine bedeutsame Menge Material auswendig lernen.
Es ist am besten, wenn Sie sich auf ein paar Stücke konzentrieren und lernen, diese gut zu spielen.
Es ist wenig Zeit für Bücher und Übungen für Anfänger - diese sind nicht schädlich, aber Sie sollten innerhalb weniger Monate nach dem Beginn des Unterrichts damit anfangen, Stücke zu lernen, die Sie spielen möchten.


\label{c1iii18c65}

\textbf{Alter von mehr als 65:}
Es gibt keinen Grund, warum man in irgendeinem Alter nicht mit dem Klavierspielenlernen beginnen können soll.
Diejenigen, die in diesem Alter anfangen, sehen es realistisch, was sie spielen lernen können und haben im allgemeinen keine unerfüllbaren Erwartungen.
Es gibt jede Menge einfache aber wundervolle Musik zum Spielen, und die Freude am Spielen bleibt genauso hoch wie sie in jungen Jahren war.
Solange Sie leben und nicht stark behindert sind, können Sie in jedem Alter das Klavierspielen lernen und zufriedenstellende Fortschritte machen.
Eine Komposition \hyperref[c1iii6]{auswendig zu lernen}, die man gerade übt, ist für die meisten kein Problem.
Die größte Schwierigkeit beim Auswendiglernen resultiert aus der Tatsache, daß man längere Zeit braucht, bis man bei schwierigem Material zur endgültigen Geschwindigkeit gekommen ist, und mit langsamem Spielen auswendig zu lernen ist die schwierigste Arbeit beim Auswendiglernen.
Deshalb werden Sie, wenn Sie leichte Stücke auswählen, die man leicht auf die endgültige Geschwindigkeit bringen kann, diese schneller auswendig lernen.
Die Hände zu dehnen, um weite Akkorde oder Arpeggios zu greifen, sowie schnelle Läufe werden schwieriger, und das \hyperref[c1ii14]{Entspannen} wird ebenfalls schwerer.
Wenn Sie sich jeweils nur auf eine Komposition konzentrieren, können Sie immer eine oder zwei Kompositionen haben, die Sie \hyperref[c1iii14]{vorspielen} können.
Es gibt keinen Grund, die Übungsmethoden zu ändern - es sind dieselben, die auch für jüngere Schüler benutzt werden.
Und Sie werden nicht so \hyperref[c1iii15]{nervös} sein, wie Sie eventuell in den mittleren Altersstufen waren.
Das Klavierspielenlernen, insbesondere die Arbeit am Gedächtnis, ist eine der besten Übungen für das Gehirn; deshalb sollten die ernsthaften Bemühungen beim Klavierspielenlernen den Alterungsprozeß verzögern, so wie das richtige körperliche Training notwendig ist, um die Gesundheit zu erhalten.
Nehmen Sie sich keinen Lehrer, der Sie wie einen jungen Anfänger behandelt und Ihnen nur Übungen gibt - dafür haben Sie keine Zeit.
Fangen Sie sofort damit an, Musik zu spielen.



<!-- c1iii19.html -->

\subsection{Der \enquote{ideale} Übungsablauf (Bachs Invention \#4)}
\label{c1iii19}

\textbf{Gibt es einen idealen, universellen Übungsablauf?
Nein,} weil jeder bei jeder Übungseinheit seinen eigenen Übungsablauf entwickeln muß.
Mit anderen Worten: \textbf{Dieses Buch handelt davon, wie Sie Ihre eigenen Übungsabläufe entwickeln können.}
Einige Unterschiede zwischen einem durchdachten und dem in \hyperref[c1ii1]{Abschnitt II.1} gezeigten intuitiven Ablauf werden im letzten Absatz dieses Abschnitts besprochen.
Ein guter Klavierlehrer wird während des Unterrichts die richtigen Übungsabläufe für die Übungsstücke mit Ihnen besprechen.
Diejenigen, die bereits wissen, wie man Übungsabläufe erstellt, werden diesen Abschnitt trotzdem interessant finden, weil wir zusätzlich zu den Übungsabläufen viele nützliche Punkte (wie Bachs Lehren und Details über das Üben der Invention \#4) besprechen.


\subsubsection{Die Regeln lernen}
\label{c1iii19a}

Deshalb ist der erste \enquote{Übungsablauf}, den Sie benutzen sollten, Kapitel 1 zu verfolgen.
Fangen Sie vorne an und wenden Sie die Konzepte auf eine Komposition an, die Sie spielen möchten.
Das Ziel ist, mit allen verfügbaren Übungsmethoden vertraut zu werden.
Wenn Sie erst ein wenig mit den meisten Übungsmethoden vertraut sind, sind wir bereit, Übungsabläufe zu entwickeln.
Um allgemein nützliche Abläufe zu entwickeln, nehmen wir an, daß Sie das Klavierspielen mindestens ein Jahr ernsthaft geübt haben.
Unser Ziel ist es, Bachs Invention \#4 zu lernen.


\subsubsection{Ein neues Stück lernen (Invention \#4)}
\label{c1iii19b}

\enquote{Ein neues Stück lernen} bedeutet, es auswendig zu lernen.
Fangen Sie deshalb ohne Aufwärmen usw. direkt mit dem \hyperref[c1iii6]{Auswendiglernen} von Bachs Invention \#4 an; zuerst mit der RH, beginnend mit Abschnitten von einem bis zu drei Takten, die eine Phrase bilden, dann mit der LH; weitere Details zu den einzelnen Schritten finden Sie in \hyperref[c1iii6l]{Abschnitt III.6l}.
Fahren Sie mit dem Vorgang fort, bis Sie das ganze Stück - nur HS - auswendig gelernt haben.
Diejenigen, die bereits gut darin sind, die Methoden dieses Buchs zu benutzen, sollten in der Lage sein, die ganze Invention (nicht perfekt) am ersten Tag innerhalb von einer bis zwei Stunden üben HS auswendig zu lernen (das gilt für eine durchschnittliche Person mit einem IQ von ungefähr 100).
Konzentrieren Sie sich nur auf das Auswendiglernen, machen Sie sich keine Gedanken, daß Sie etwas \enquote{nicht zufriedenstellend spielen können} (wie z.B. den 1,3-Triller in der LH), und spielen Sie mit einer beliebigen Geschwindigkeit, mit der Sie gut zurechtkommen.
Wenn Sie dieses Stück so schnell wie möglich auswendig lernen möchten, ist es am besten, wenn Sie sich nur auf dieses Stück konzentrieren und keine anderen Stücke spielen.
Anstelle einer langen Sitzung von 2 Stunden, könnten Sie zweimal am Tag eine Stunde üben.
Beginnen Sie am zweiten Tag das HT langsam und immer noch in Abschnitten von wenigen Takten; verbinden Sie diese dann.
Wenn Sie dieses Stück so schnell wie möglich auswendig lernen möchten, üben Sie wieder nichts anderes; sogar Fingerübungen zum Aufwärmen zu spielen wird dazu führen, daß Sie etwas von dem vergessen, was Sie gerade auswendig gelernt haben.


\subsubsection{\enquote{Normale} Übungsabläufe und Bachs Lehren}
\label{c1iii19c}  

Nach 3 oder 4 Tagen können Sie zu Ihrem \enquote{normalen} Übungsablauf zurückkehren.
Beim Ablauf für das Auswendiglernen haben wir im Grunde nichts anderes getan als auswendig zu lernen, weil der Prozeß des Auswendiglernens verlangsamt wird, wenn man das Auswendiglernen mit anderen Übungen mischt.
Beim \enquote{normalen} Ablauf können wir einen Vorteil aus dem Anfang ziehen, wenn die Hände noch kalt sind, und einige fertige Stücke \hyperref[c1iii6g]{\enquote{kalt} spielen}.
Natürlich können Sie schwierige, schnelle Stücke nicht kalt spielen.
Spielen Sie entweder leichtere Stücke oder spielen Sie die schwierigen langsam.
Ein gutes Verfahren ist, mit leichteren Stücken zu beginnen und schrittweise schwierigere zu spielen.
Wenn Sie im \hyperref[c1iii14]{Aufführen} stark genug geworden sind, so daß Sie keine Probleme haben kalt zu spielen (das mag ein Jahr dauern), wird dieser Schritt, besonders wenn Sie täglich Klavier spielen, optional.
Wenn Sie nicht täglich spielen, verlieren Sie eventuell die Fähigkeit kalt zu spielen, wenn Sie aufhören es zu üben.
Sie können während dieser Aufwärmphase auch Tonleitern und Arpeggios üben; Sie finden dazu in den Abschnitten \hyperref[c1iii4b]{III.4b (Mit flachen Fingern spielen)} und \hyperref[c1iii5]{III.5 (Schnelle Tonleitern und Arpeggios)} nähere Details.
Sie könnten auch die Übungen zur \hyperref[c1iii7d]{Unabhängigkeit der Finger und dem Anheben der Finger} in Abschnitt III.7d versuchen; einige Klavierspieler führen diese Übungen regelmäßig ein- oder zweimal täglich aus.
Beginnen Sie damit, zusätzlich zu dem Bach-Stück andere Kompositionen zu lernen.

Zu diesem Zeitpunkt sollten Sie in der Lage sein, die gesamte Bach-Invention ohne Probleme in Gedanken HS zu spielen.
Das ist der richtige Zeitpunkt, um Stücke, die Sie bereits auswendig gelernt haben, zu überarbeiten (s. Abschnitte \hyperref[c1iii6c]{III.6c} und \hyperref[c1iii6f]{III.6f}), weil ein neues Stück zu lernen oft dazu führt, daß man Teile von zuvor gelernten Stücken vergißt.
Wechseln Sie beim Üben zwischen der Bach-Invention und Ihren alten Stücken.
Sie sollten die Invention die meiste Zeit HS üben, bis Sie sich die gesamte notwendige Technik angeeignet haben.
Steigern Sie die Geschwindigkeit - indem Sie kurze Abschnitte spielen - so schnell Sie es können auf Geschwindigkeiten, die schneller als die endgültige Geschwindigkeit sind.
Üben sie hauptsächlich die Abschnitte, die Ihnen Schwierigkeiten bereiten; es besteht keine Notwendigkeit, Abschnitte zu üben, die Ihnen leichtfallen.
Wenn Sie mit HS eine bestimmte Geschwindigkeit erreicht haben, fangen Sie damit an, HT mit einer niedrigeren Geschwindigkeit zu üben.
Sobald Sie mit dem HT bei niedrigeren Geschwindigkeiten zurechtkommen, können Sie es - wieder mit kurzen Abschnitten - auf höhere Geschwindigkeit bringen.
\textbf{Um die Geschwindigkeit zu steigern (HS oder HT), benutzen Sie nicht das Metronom oder zwingen Ihre Finger schneller zu spielen. Warten Sie, bis Sie das Gefühl bekommen, daß Ihre Finger schneller spielen \textit{wollen}, und erhöhen Sie dann die Geschwindigkeit um einen leicht zu bewältigenden Betrag.}
Das gestattet Ihnen, \hyperref[c1ii14]{entspannt} zu üben und alle Geschwindigkeitsbarrieren zu vermeiden.

\textbf{Entwickeln Sie für den Übergang vom HS- zum HT-Üben das Gefühl, daß die beiden Hände einander brauchen um zu spielen.}
Das wird Ihnen dabei helfen, die Bewegungen zu finden, die für das HT-Spielen hilfreich sind.
Das HS-Spielen ist sogar während des HT-Spielens nützlich; wenn Sie z.B. beim HT-Spielen mit der einen Hand einen Fehler machen, können Sie mit der anderen Hand weiterspielen und das Spielen mit der Hand, die den Fehler begangen hat, wieder aufnehmen, wann immer es möglich ist.
Ohne ausgedehntes HS-Üben wäre eine solche Leistung unmöglich.
Sie können solch ein Manöver als Teil des Auswendiglernens üben - warten Sie nicht bis zum Auftritt damit, zu versuchen es auszuführen!

Um die besonderen Techniken zu erwerben, die Bach im Sinn hatte, müssen wir die Invention detaillierter analysieren.
Bachs Inventionen wurden hauptsächlich als Übungsstücke für die Technik komponiert, und jede Invention lehrt uns bestimmte Techniken.
Deshalb müssen wir wissen, welche Arten von Techniken uns diese Invention lehren soll.
\textbf{Bach lehrt uns nicht nur besondere Fertigkeiten, sondern auch \textit{wie man sie übt}! Indem wir die Inventionen analysieren, können wir deshalb viele Übungsmethoden dieses Buchs lernen!}
Spielen Sie vor allem das ganze Stück mit \hyperref[c1iii5b]{Daumenübersatz}.
Beachten Sie, daß Bach ein Maximum an Kreuzungen des Daumens eingefügt hat, so daß wir viele Gelegenheiten haben, sie zu üben - offenbar ein absichtliches Konstrukt.
Üben Sie beim 212345 der RH in Takt 1 das Drehen um die 2 mit der Hand in der \hyperref[c1iii5c]{Glissando-Position}, um den Daumenübersatz zu erleichtern.

Das Hauptthema dieser Invention wird in den ersten 4 Takten der RH eingeführt.
Es wird dann von der LH wiederholt.
\textbf{Bach sagt uns, wir sollen HS üben!}
Beide Hände spielen im Grunde dasselbe, was uns die Gelegenheit gibt, die technische Fertigkeit der beiden Hände einander anzugleichen; das kann nur dadurch erreicht werden, daß man HS übt und der schwächeren Hand mehr Arbeit gibt.
Es gibt keine bessere Möglichkeit, die Unabhängigkeit der Hände zu üben - die wichtigste Lektion der Inventionen -, als die Hände getrennt zu üben.
Der Abschnitt, in dem eine Hand trillert, wäre unheimlich schwierig, wenn man ihn von Anfang an HT üben würde, während er HS ziemlich einfach ist.
Einige Schüler, die das HS-Üben nicht kennen, werden versuchen, die beiden Hände zur Deckung zu bringen, indem Sie die Noten des Trillers vorher ermitteln und sie dann für das HT-Üben verlangsamen.
Das mag für Anfänger oder Kinder angemessen sein, die das Trillern noch nicht gelernt haben.
Die meisten Schüler sollten von Anfang an (HS) trillern und daran arbeiten, die Triller so rasch wie möglich zu beschleunigen.
Es ist nicht notwendig, die beiden Hände mathematisch zur Deckung zu bringen; das ist Kunst, keine Mechanik!
Bach möchte, daß Sie mit jeder Hand unabhängig von der anderen trillern.
Das wird Ihnen gestatten, diese Invention mit jeder Geschwindigkeit zu spielen, ohne daß Sie die Trillergeschwindigkeit spürbar ändern müssen.
Der Grund, warum man die Noten nicht zur Deckung bringen muß, ist, daß diese Triller nur ein Mittel sind, die Noten längere Zeit auszuhalten und die einzelnen Noten keine rhythmische Bedeutung haben.
Was tun Sie, wenn Sie am Ende des Trillers mit der falschen Note aufhören?
Sie sollten in der Lage sein, das zu kompensieren, indem Sie entweder kurz warten oder die Geschwindigkeit des Trillers am Ende ändern - das ist die Art von Fertigkeit, die diese Invention lehrt.
Deshalb würde es die Lektion aus dieser Invention zunichte machen, wenn man üben würde, den Triller mit der anderen Hand zur Deckung zu bringen.
Das Staccato in den Takten 3 und 4 der RH ist ein weiteres Mittel, die Unabhängigkeit der Hände zu üben; Staccato in der einen Hand gegen Legato in der anderen erfordert mehr Kontrolle als Legato in beiden Händen.
Das Staccato sollte im ganzen Stück benutzt werden, obwohl es in vielen Ausgaben nur am Anfang angegeben ist.

Die meisten Unterrichtsstücke von Bach lehren nicht nur die Unabhängigkeit der Hände, sondern auch die Unabhängigkeit der Finger einer Hand, besonders des vierten Fingers.
So sind in den Takten 11 und 13 der RH sechs Noten, die als zwei Triolen gespielt werden könnten aber in Wirklichkeit wegen des 3/8-Taktes drei Zweiergruppen sind.
Diese Takte können für den Anfänger schwierig sein, weil sie die Koordination von drei verschiedenen Bewegungen erfordern:

\begin{enumerate}[label={\roman*.}] 
 \item Der Aufbau des Fingersatzes der RH ist der von zwei Triolen (\textbf{3}45\textbf{3}45 Rhythmus), muß aber als drei Zweiergruppen gespielt werden (\textbf{3}4\textbf{5}3\textbf{4}5).
 \item Gleichzeitig muß die LH etwas völlig anderes spielen.
 \item All das muß hauptsächlich mit den drei schwächsten Fingern - 3, 4 und 5 - durchgeführt werden.
\end{enumerate}
Bach benutzte dieses Mittel häufig, um uns zu zwingen, einen Rhythmus zu spielen, der sich vom Aufbau des Fingersatzes unterscheidet, um die Unabhängigkeit der Finger zu entwickeln.
Er versucht auch, dem vierten Finger soviel Arbeit wie möglich zu geben, wie z.B. im \textbf{4}5 am Ende.

Die Triolen sind mit dem Fingersatz 234 leichter zu spielen als mit 345, besonders mit großen Händen, und die meisten Ausgaben empfehlen den Fingersatz 234, weil die meisten Herausgeber das Konzept der parallelen Sets nicht kannten.
Die Kenntnis der \hyperref[c1iii7b]{Übungen für parallele Sets} zeigt jedoch, daß Bachs ursprüngliche Absicht 345 war (für einen maximalen Wert für die technische Entwicklung), und es ist eine \enquote{musikalische Freiheit}, den Fingersatz in 234 zu ändern, um die Musikalität zu vereinfachen.
In jeder anderen Komposition als dieser Invention wäre 234 der korrekte Fingersatz.
Der Gebrauch von 234 kann hier gerechtfertigt werden, weil es den Schüler das Prinzip lehrt, den Fingersatz mit der größten Kontrolle auszuwählen.
Deshalb kann der Schüler beide Fingersätze wählen.
Eine ähnliche Situation tritt in Takt 38 auf, in dem Bachs ursprüngliche Absicht für die LH wahrscheinlich 154321 war (ein vollständigeres paralleles Set), während die musikalische Freiheit 143212 anzeigt, was technisch weniger herausfordernd ist.
Ohne die Hilfe der Übungen für parallele Sets wäre die offensichtliche Wahl die musikalische Freiheit.
Durch die Anwendung der Übungen für parallele Sets kann der Schüler lernen, beide Fingersätze mit gleicher Leichtigkeit zu benutzen.

Die \enquote{Triolen im 3/8-Takt} sind ein gutes Beispiel, wie das fehlerhafte Lesen der Noten es schwierig macht, auf Geschwindigkeit zu kommen, und zu Geschwindigkeitsbarrieren führt.
Wenn man HT spielt, trifft man auf Probleme, wenn man die RH-Triolen auf zwei Schläge spielt (falsch) und die LH in drei (richtig).
Sogar wenn man einen zweiten Fehler begeht und die LH in zwei Schlägen spielt, um sie an die RH anzupassen, gibt es wegen der rhythmischen Änderung ein Problem mit den nachfolgenden Takten.
Man ist vielleicht bei niedriger Geschwindigkeit in der Lage, durch diese Fehler hindurchzuspielen, aber wenn man schneller wird, dann wird es unmöglich es zu spielen, und man baut eine Geschwindigkeitsbarriere auf.
Das ist ein Beispiel für die \hyperref[c1iii1b]{Wichtigkeit des Rhythmus}.
Es ist erstaunlich, wie viele Lektionen Bach in etwas hineinstecken konnte, das so einfach aussieht, und diese Komplexität erklärt teilweise, warum \textbf{viele Schüler es ohne die richtigen Übungsmethoden oder die Anleitung eines erfahrenen Lehrers unmöglich finden, Bach \hyperref[c1iii6]{auswendig zu lernen} oder seine Kompositionen jenseits einer bestimmten Geschwindigkeit zu spielen.
Der Mangel an richtigen Übungsmethoden ist der Hauptgrund, warum so viele Schüler so wenige Stücke von Bach spielen.}

Die Inventionen sind ausgezeichnete technische Unterrichtsstücke.
\hyperref[c1iii7h]{Hanon}, Czerny usw. versuchten dasselbe zu erreichen, indem sie das benutzten, was sie für einen einfacheren, systematischeren Ansatz hielten, aber sie versagten, weil sie versuchten, etwas zu vereinfachen, was außerordentlich komplex ist.
Im Gegensatz dazu packte Bach, wie oben gezeigt, so viele Lektionen in jeden Takt wie er konnte.
Hanon, Czerny usw. muß die Schwierigkeit Bach zu lernen bewußt gewesen sein, aber sie kannten die guten Übungsmethoden nicht und versuchten, indem sie ihren intuitiven Instinkten folgten, einfachere Methoden zum Erwerb der Technik zu finden.
Das ist eines der besten historischen Beispiele für die Fallen der intuitiven Vorgehensweise.

Weil die Inventionen für das Lehren bestimmter Fertigkeiten komponiert wurden, können sie etwas gezwungen klingen. 
Trotz dieser Gezwungenheit enthalten alle von Bachs Unterrichtsstücken mehr Musik als praktisch alles, was jemals komponiert wurde, und es gibt genug davon, um die Bedürfnisse von Schülern aller Stufen zu befriedigen, einschließlich von Anfängern.
Wenn die Inventionen zu schwierig sind, denken Sie darüber nach, die große Zahl wunderbarer (und vorzüglich aufführbarer) einfacheren Unterrichtsstücke, die Bach komponiert hat, zu studieren.
Die meisten davon finden Sie im \enquote{Notenbüchlein für Anna Magdalena Bach} (seiner zweiten Frau).
Da es so viele davon gibt, enthalten die meisten Bücher nur eine kleine Auswahl davon.
\textbf{Da die Inventionen Unterrichtsstücke sind, werden in fast jeder Ausgabe die kritischen Stellen der Fingersätze angegeben.}
Deshalb sollte es kein Problem sein, die Fingersätze herauszufinden, was extrem wichtig ist.
\enquote{J. S. Bach, Inventions and Sinfonias} von Willard A. Palmer, Alfred, CA, (www.alfredpub.com) zeigt alle nicht offensichtlichen Fingersätze und enthält auch einen Abschnitt über das Spielen der Verzierungen.

Die Inventionen wurden komponiert, indem wohldefinierte Abschnitte aneinandergefügt wurden, die üblicherweise nur ein paar Takte lang sind.
Das macht sie für das abschnittsweise HS-Üben ideal, einem weiteren Schlüsselelement der Methoden dieses Buchs.
Diese und viele andere Eigenschaften von Bachs Kompositionen machen sie zu einer idealen Musik, um die Methoden dieses Buchs zu lernen, und es ist ziemlich wahrscheinlich, daß sie mit dem Gedanken an diese Übungsmethoden komponiert wurden.
Bach war wohl das meiste Material in diesem Buch bekannt!

\textbf{Eine weitere wichtige Lektion von Bachs Inventionen sind die parallelen Sets.
Die hauptsächliche technische Lektion dieser Invention \#4 ist das parallele Set 12345, das Basis-Set, das benötigt wird, um die \hyperref[c1iii5]{Tonleitern} und Läufe zu spielen.}
Bach wußte jedoch, daß ein einziges paralleles Set von einem technischen Standpunkt aus zu gefährlich ist, weil man durch \hyperref[c1iii7b2]{Phasenkopplung} schummeln kann ohne Technik zu erwerben.
Um eine Phasenkopplung zu verhindern, fügte er dem parallelen Set eine oder zwei Noten hinzu.
Wenn man nun versucht zu schummeln, wird man sofort erwischt, weil die Musik nicht gleichmäßig herauskommt: Bach hat uns keine andere Wahl gelassen, als die erforderliche Technik zu erwerben, wenn man das musikalisch spielen will!
Hier ist ein weiteres Beispiel, in dem Bach uns lehrt, warum Musik und Technik untrennbar sind (indem er Musik als Kriterium für den Erwerb der Technik benutzt).
Deshalb ist der schnellste Weg, diese Invention spielen zu lernen, die parallelen Sets 12345 und 54321 zu üben und den \hyperref[c1iii5b]{Daumenübersatz} zu lernen.
\textbf{Sobald sie Ihre Finger mit Hilfe dieser parallelen Sets testen, werden Sie verstehen, warum Bach diese Invention komponiert hat.}
Wenn Sie diese Übung für parallele Sets zufriedenstellend ausführen können, wird dieses Stück ziemlich einfach sein, aber Sie werden finden, daß die parallelen Sets überhaupt nicht einfach sind und wahrscheinlich jede Menge Arbeit erfordern, auch wenn Sie zur Mittelstufe gehören.
Arbeiten Sie zunächst an diesen Sets, indem Sie nur die weißen Tasten benutzen; arbeiten Sie dann an den anderen, die schwarze Tasten beinhalten, wie von Bach vorgeschlagen.
Ein gutes Beispiel ist das parallele Set 12345 in der LH in den Takten 39-40 mit dem schwierigen vierten Finger auf einer weißen Taste, der auf 3 auf einer schwarzen Taste folgt.
Bach zieht den schwierigsten Teil dieses parallelen Sets, 2345, heraus und wiederholt ihn in Takt 49.

Bach hat klar den Wert davon gesehen, für die Entwicklung der Technik (Geschwindigkeit) eine kleine Anzahl Noten, wie Verzierungen und Triller, sehr schnell zu spielen.
Somit sind seine Verzierungen ein weiteres wichtiges Mittel für den Erwerb der Technik, und sie sind im Grunde eine kleine Ansammlung paralleler Sets.
Es gibt zahlreiche Diskussionen darüber, wie man Bachs Verzierungen spielen sollte (siehe Palmer, 3 Absätze zuvor); diese Diskussionen sind vom Standpunkt des korrekten musikalischen Ausdrucks wichtig, aber \textbf{wir dürfen nicht vergessen, daß die Verzierungen in Unterrichtsstücken technisch gesehen ein wesentliches Mittel zum Erwerb der Geschwindigkeit und nicht bloß musikalische Verzierungen sind.}
Spielen Sie sowohl die RH- als auch die LH-Triller mit den Fingern 1 und 3, was das Lernen des LH-Trillers vereinfacht.
Die meisten Schüler werden den RH-Triller zunächst besser spielen als den LH-Triller; \hyperref[c1ii20]{benutzen Sie in diesem Fall die RH, um die LH zu unterrichten}.
Dieser \enquote{Techniktransfer} von einer Hand zur anderen ist einfacher, wenn beide Hände einen ähnlichen Fingersatz benutzen.
Da der Zweck des Trillers einfach ein Aushalten der Noten ist, ist für den Triller keine bestimmte Geschwindigkeit erforderlich; versuchen Sie jedoch, die Triller mit beiden Händen mit der gleichen Geschwindigkeit auszuführen.
Wenn Sie sehr schnell trillern möchten, benutzen Sie die parallelen Sets, um die Triller wie in \hyperref[c1iii3]{Abschnitt III.3a} beschrieben zu üben.
Es ist wichtig, daß Sie die ersten beiden Noten schnell beginnen, wenn Sie schnell trillern möchten.
Beachten Sie die Haltung der Finger 2, 4 und 5 während Sie trillern.
Diese sollten stillstehen, nahe an den Tasten und leicht gebogen sein.

Die meisten Schüler finden es schwierig, diese Inventionen jenseits einer bestimmten Geschwindigkeit zu spielen.
Lassen Sie uns deshalb einen Übungsablauf für das Steigern der Geschwindigkeit ansehen.
Wenn Sie diese Art von Ablauf benutzen, sollten Sie irgendwann in der Lage sein, praktisch mit jeder vernünftigen Geschwindigkeit zu spielen, einschließlich der hohen Geschwindigkeiten von Glenn Gould und anderen berühmten Pianisten.
Wir werden lernen, wie man die Takte 1 und 2 schnell spielt, und danach sollten Sie in der Lage sein, selbst herauszufinden, wie man den Rest beschleunigt.
Beachten Sie, daß diese beiden Takte \hyperref[c1iii2]{selbst-zirkulierend} sind (s. Abschnitt III.2).
Versuchen Sie, diese schnell zu zirkulieren.
Es besteht die Wahrscheinlichkeit, daß Sie es nicht schaffen, weil sich mit steigender Geschwindigkeit sehr schnell Streß aufbaut.
Üben Sie dann nur 212345 von Takt 1, bis es gleichmäßig und schnell ist.
Üben Sie dann 154, dann 54321 des zweiten Takts.
Verbinden Sie sie nun und zirkulieren Sie am Ende die beiden Takte.
Sie sind am ersten Tag vielleicht noch nicht in der Lage alles zu vollenden, aber die \hyperref[c1ii15]{PPI} wird es am zweiten Tag einfacher machen.
Benutzen Sie ähnliche Methoden, um Ihre technischen Probleme im ganzen Stück zu lösen.
Die Hauptschwierigkeit der LH ist das 521 in Takt 4, üben Sie deshalb das parallele Set 521, bis Sie es mit jeder Geschwindigkeit völlig \hyperref[c1ii14]{entspannt} spielen können.
Beachten Sie, daß das 212345 der RH und das 543212 der LH Übungen für das Vorbeigehen des Daumens sind.
Bach erkannte sicherlich, daß das \hyperref[c1iii5a]{Über- und Untersetzen des Daumens} bei hohen Geschwindigkeiten kritische Elemente der Technik sind, und er entwickelte zahlreiche geniale Gelegenheiten für uns, es zu üben.
Bevor man schnell HT spielen kann, muß man mit HS zu Geschwindigkeiten kommen, die viel schneller sind als die gewünschte HT-Geschwindigkeit.
\enquote{Auf Geschwindigkeit kommen} bedeutet nicht nur, in der Lage zu sein, die Geschwindigkeit zu erreichen, sondern man muß die \hyperref[ruhig]{ruhigen Hände} fühlen und die volle Kontrolle über jeden einzelnen Finger haben.
Anfänger müssen eventuell monatelang HS üben, um höhere Geschwindigkeiten zu erreichen.
Vielen Schülern gelingt es schneller zu spielen, wenn sie laut spielen.
Das ist ebenfalls nicht die wahre Geschwindigkeit; spielen Sie deshalb während der Übungseinheiten alles leise.
Wenn Sie damit beginnen, HT schnell zu spielen, übertreiben Sie den Rhythmus - das macht es wahrscheinlich einfacher.
Obwohl die meisten Kompositionen von Bach mit verschiedenen Geschwindigkeiten gespielt werden können, ist die minimale Geschwindigkeit für die Inventionen jene, bei der man die ruhigen Hände fühlen kann, sobald man die notwendige Technik erworben hat, denn wenn man nicht bis zu dieser Geschwindigkeit kommt, dann hat man eine von Bachs wichtigsten Lektionen versäumt.

Ein Schüler der Mittelstufe sollte die technischen Schwierigkeiten dieser Invention innerhalb einer Woche meistern.
\textbf{Nun sind wir bereit, es als Musikstück zu üben!}
Hören Sie sich verschiedene Aufnahmen an, um eine Vorstellung davon zu bekommen, was man tun kann und was Sie tun möchten.
Probieren Sie verschiedene Geschwindigkeiten aus, und entscheiden Sie sich für Ihre endgültige Geschwindigkeit.
\hyperref[c1iii13]{Nehmen Sie sich auf Video auf}, und prüfen Sie, ob das Ergebnis optisch und musikalisch zufriedenstellend ist; üblicherweise ist es das nicht, und Sie werden vieles finden, das Sie verbessern möchten. Sie werden vielleicht nie ganz zufrieden sein, sogar wenn Sie dieses Stück Ihr ganzes Leben lang geübt haben.

Um musikalisch zu spielen, müssen Sie jede Note mit den Fingern fühlen, bevor Sie sie spielen, auch wenn es nur den Bruchteil einer Sekunde früher ist.
Das wird Ihnen nicht nur mehr Kontrolle verleihen und Fehler eliminieren, sondern Ihnen auch gestatten, über den ganzen Tastenweg zu beschleunigen, so daß der Hammerstiel genau im richtigen Maß gebogen wird, wenn der Hammer die Saiten anschlägt.
Tun Sie so, als ob es keinen unteren Punkt für den Tastenweg gäbe, und lassen Sie Ihren Finger durch den unteren Punkt stoppen.
Sie können das tun und trotzdem leise spielen.
Das wird \enquote{in die Tasten hineingehen} genannt.
Man kann nicht \enquote{die Finger gut anheben und anschlagen} wie \hyperref[c1iii7h]{Hanon} empfahl und erwarten Musik zu machen.
Solch eine Bewegung kann den Hammerstiel schwingen statt sich biegen lassen und einen nicht vorhersagbaren und schrillen Klang erzeugen.
Üben Sie deshalb, wenn Sie HS üben, auch die Musikalität.
Benutzen Sie die \enquote{\hyperref[c1iii4b]{flachen Fingerhaltungen}} aus Abschnitt III.4b.
Kombinieren Sie diese mit einem geschmeidigen Handgelenk.
Spielen Sie soviel wie möglich mit dem fleischigen Teil der Finger (gegenüber dem Fingernagel), nicht mit der knochigen Fingerspitze.
Wenn Sie Ihr Spielen auf Video aufnehmen, dann wird die gebogene Fingerhaltung kindlich und amateurhaft aussehen.
Sie können nicht entspannt spielen, bis Sie die Streckmuskeln der ersten 2 oder 3 Glieder der Finger 2 bis 5 völlig entspannen können.
Diese Entspannung ist das Wesentliche der flachen Fingerhaltungen.
Zunächst werden Sie nur bei niedrigeren Geschwindigkeiten in der Lage sein, diese Überlegungen zu berücksichtigen.
Sobald Sie jedoch die ruhigen Hände entwickeln, werden Sie die Fähigkeit erlangen, sie bei höheren Geschwindigkeiten einzuschließen.
Tatsächlich werden Sie, weil diese Fingerhaltungen Ihnen die völlige Entspannung und Kontrolle gestatten, in der Lage sein, mit viel höherer Geschwindigkeit zu spielen.
Das ist einer der (vielen) Gründe, warum ruhige Hände so wichtig sind.


<h3><br>Klang und Farbe</h3>

Der verbesserte Klang wird am deutlichsten, wenn man leise spielt; das leisere Spielen ist auch für die \hyperref[c1ii14]{Entspannung} und Kontrolle hilfreich.
Es ist die \hyperref[c1iii4b]{flache Fingerhaltung}, die ein leiseres Spielen mit Kontrolle ermöglicht.
Wie leise ist leise?
Das hängt von der Musik, der Geschwindigkeit usw. ab, aber für Übungszwecke ist immer leiser zu spielen, bis Sie anfangen, einige Noten auszulassen, ein nützliches Kriterium; diese Lautstärke (oder ein wenig lauter) ist meistens die beste um leise zu üben.
Wenn Sie die Kontrolle über den Klang erlangt haben (Klang jeder einzelnen Note), versuchen Sie, Ihrer Musik Farbe hinzuzufügen (Effekt von Notengruppen).
Die Farbe ist bei jedem Komponisten anders.
Chopin erfordert Legato, besonderes Staccato, Rubato usw.
Mozart erfordert die äußerste Aufmerksamkeit auf die Ausdrucksbezeichnungen.
Beethoven erfordert ununterbrochene \hyperref[c1iii1b]{Rhythmen}, die über viele, viele Takte gehen; deshalb müssen Sie die Fertigkeit entwickeln, aufeinanderfolgende Takte zu \enquote{verbinden}.
Bachs Inventionen sind etwas künstlich und \enquote{eingeengt}, weil sie hauptsächlich auf einfache \hyperref[c1iii7b]{parallele Sets} beschränkt sind.
Man kann dieses Handicap leicht überwinden, indem man die Vielzahl der musikalischen Konzepte betont, die seiner Musik fast unendliche Tiefe verleihen.
Die offensichtliche Musikalität kommt von der Harmonie bzw. Konversation zwischen beiden Händen.
Der Schluß jedes Stücks muß etwas besonderes sein, und Bachs Schlüsse sind immer überzeugend.
Lassen Sie deshalb den Schluß nicht einfach so daherkommen; stellen Sie sicher, daß der Schluß einen Zweck hat.
Seien Sie bei dieser Invention in Takt 50, bei dem die beiden Hände in Gegenbewegung sind, besonders aufmerksam, wenn Sie zu dem respekteinflößenden Schluß kommen.
Wenn Sie die Musik auf Geschwindigkeit bringen und \hyperref[ruhig]{ruhige Hände} entwickeln, sollten die 6-notigen Läufe (z.B. 212345 usw.) wie aufsteigende und fallende Wellen klingen.
Der RH-Triller ist einer Glocke ähnlich, weil die Noten einen Ganzton umfassen, während der LH-Triller düsterer ist, weil er ein Halbtonschritt ist.
Wenn Sie HS üben, beachten Sie, daß der RH-Triller nicht nur ein Triller ist, sondern an Lautstärke zunimmt.\footnote{Das gilt zumindest für Ausgaben mit Dynamikzeichen. Im \enquote{Urtext} findet man davon nichts, was zeigt, wie sich die Bach-Ausgaben über die Jahrhunderte hinweg geändert haben und was man alles mit den Inventionen anstellen kann.}
Ähnlich ist der LH-Triller eine Einführung zu dem darauffolgenden Kontrapunkt zur RH.
\textbf{Man kann die Farbe nicht herausbringen, wenn man nicht jeden Finger präzise im richtigen Moment anhebt.}
Die meisten Unterrichtsstücke Bachs enthalten Übungen zum richtigen Anheben der Finger.
Selbstverständlich sollte die Färbung zunächst HS untersucht werden.
Ruhige Hände werden am leichtesten HS erworben; deshalb ist eine angemessene Vorbereitung mit HS vor dem HT-Üben für den Klang und die Farbe von entscheidender Wichtigkeit.
Wenn die Vorbereitungen abgeschlossen sind, können Sie mit HT beginnen und den unglaublichen Reichtum von Bachs Musik hervorbringen!

Klang und Farbe haben in dem Sinn keine Grenze, daß es, sobald es einmal gelungen ist, leichter wird, mehr davon hinzuzufügen, und die Musik sogar leichter spielbar wird.
Sie entdecken vielleicht plötzlich, daß Sie die ganze Komposition ohne einen einzigen hörbaren Fehler spielen können.
Das ist wahrscheinlich das deutlichste Beispiel für die Aussage, daß man die Musik nicht von der Technik trennen kann.
Das Erzeugen guter Musik macht Sie tatsächlich zu einem besseren Klavierspieler.
Das bietet eine der Erklärungen dafür, warum man gute und schlechte Tage haben kann - wenn Ihre geistige Verfassung und die Konditionierung Ihrer Finger genau so sind, daß Sie den Klang und die Farbe kontrollieren können, werden Sie einen guten Tag haben.
Das lehrt uns, daß man an schlechten Tagen in der Lage sein kann \enquote{darüber hinwegzukommen}, indem man versucht, sich an die Grundlagen zu erinnern, wie man den \hyperref[c1iii1]{Klang und die Farbe kontrolliert}.
Damit beenden wir die Besprechung der Invention \#4.
Wir wenden uns nun wieder dem Übungsablauf zu.

Sie haben nun über eine Stunde geübt, und die Finger fliegen.
Das ist der Zeitpunkt, wenn Sie wirklich Musik machen!
Sie müssen jede Anstrengung unternehmen, um mindestens während der Hälfte der gesamten Übungszeit das Musikmachen zu üben.
Wenn Sie ein genügend großes Repertoire aufgebaut haben, dann sollten Sie versuchen, diese \enquote{Musikzeit} von 50\% auf 90\% zu steigern.
Deshalb müssen Sie bewußt diesen Anteil Ihres Übungsablaufs für die Musik reservieren.
Spielen Sie sich die Seele aus dem Leib, mit allen Emotionen und dem Ausdruck, den Sie zustande bringen.
\textbf{Den musikalischen Ausdruck zu finden ist sehr erschöpfend; deshalb wird es zunächst viel mehr Konditionierung und Aufwand erfordern als alles, was man mit \hyperref[c1iii7h]{Hanon} machen kann.}
Wenn Sie keinen Lehrer haben, dann ist der einzige Weg, Musikalität zu lernen, sich Aufnahmen anzuhören und Konzerte zu besuchen.
Wenn Sie in nächster Zukunft einen Auftritt mit einer bestimmten Komposition planen, spielen Sie sie einmal langsam oder zumindest mit einer bequemen und völlig kontrollierbaren Geschwindigkeit, bevor Sie an etwas anderes herangehen.
Wenn man langsam spielt, ist der Ausdruck nicht wichtig.
Es mag sogar nützlich sein, absichtlich mit wenig Ausdruck zu spielen, wenn man langsam spielt, bevor man zu etwas anderem übergeht.

Bach zu lernen wird in diesem Buch sehr betont.
Warum?
\textbf{Weil die Musik, die Bach für die technische Entwicklung schrieb, in der Klavierpädagogik in ihrem gesunden, vollständigen, effizienten und korrekten Herangehen an den Erwerb der Technik einmalig ist - es gibt nichts vergleichbares.}
Jeder erfahrene Lehrer wird einige Stücke von Bach zu Studienzwecken zuweisen.
Wie oben erwähnt, ist der einzige Grund, warum Schüler nicht mehr Stücke von Bach lernen, daß sie ohne die richtigen Übungsmethoden so schwierig erscheinen.
Sie können sich selbst den Nutzen von Bachs Lektionen beweisen, indem Sie fünf seiner technischen Kompositionen lernen und sie ein halbes Jahr oder länger üben.
Kehren Sie dann zurück und spielen Sie die schwierigsten Stücke, die Sie zuvor gelernt haben, und Sie werden über die größere Leichtigkeit und die Kontrolle, die Sie erworben haben, überrascht sein.
Bachs Kompositionen wurden entworfen, um Konzertpianisten mit einer gesunden Grundlage der Technik hervorzubringen.
Chopins Etüden wurden nicht für eine schrittweise, vollständige technische Entwicklung entworfen, und viele von Beethovens Kompositionen können \hyperref[c1iii10hand]{Handverletzungen} und \hyperref[c1iii10gehoer]{Gehörschäden} verursachen, wenn man nicht die richtige Anleitung bekommt (anscheinend haben sie Beethovens Gehör geschädigt).
Keine davon lehrt, wie man übt.
Deshalb ragen Bachs Kompositionen gegenüber allen anderen hinsichtlich der technischen Entwicklung heraus.
Mit den Übungsmethoden dieses Buchs können wir nun den vollen Vorteil aus Bachs Quellen für die technische Entwicklung ziehen, die in der Vergangenheit sträflich vernachlässigt wurden.

Natürlich sind \hyperref[c1iii7h]{Hanon} usw. (es gibt viele andere, wie z.B. Cramer-Bülow) es nicht Wert, hier besprochen zu werden, weil sie den wichtigsten Punkt außer Acht gelassen haben: Ohne Musik ist Technik nicht möglich.
Aber \hyperref[c1iii7d]{Tonleitern und Arpeggios} sind notwendig, weil sie die Grundlage von praktisch allem bilden, das wir spielen.
Die Erfordernis der Musikalität bedeutet, daß man sie auf eine Weise üben muß, daß andere, die einem beim Spielen von Tonleitern zuhören, sagen: \enquote{Enorm!}
Warum soll man Hanon nicht genauso üben können?
Man kann, aber es ist nicht notwendig; es gibt viel besseres Material, mit dem man die Kunst des Musikers üben kann.

\textbf{Alles in allem gibt es keinen Standard-Übungsablauf.
Das Konzept eines festen Übungsablaufs kam auf, weil diejenigen, die die intuitiven Methoden praktizierten und nicht wußten, wie man Übungsmethoden lehrt, ihn benutzten, weil sie nicht wußten, was sie sonst lehren sollten.}
Für diejenigen, die die Übungsmethoden kennen, wird das Konzept eines Standard-Übungsablaufs irgendwie zu einer dummen Idee.
Ein typischer Standardablauf mag z.B. mit \hyperref[c1iii7h]{Hanon-Übungen} anfangen; man kann die Hanon-Übungen jedoch mit den Methoden dieses Buchs leicht auf lächerliche Geschwindigkeiten bringen.
Und wenn man das vollbracht hat, fragt man sich, warum man das ständig wiederholen soll.
Was wird man erreichen, wenn man diese lächerlich schnellen Hanon-Stücke jeden Tag spielt?
Statt eines Standard-Übungsablaufs müssen Sie jeweils das Ziel Ihrer Übungseinheit definieren und die Übungsmethoden auswählen, die Sie benötigen, um dieses Ziel zu erreichen.
Ihr Übungsablauf wird sich im Laufe der Übungseinheiten kontinuierlich weiterentwickeln.
Deshalb ist der Schlüssel zum Entwickeln eines guten Übungsablaufs eine genaue Kenntnis aller Übungsmethoden.
Wie sich das von dem intuitiven Ablauf, der in \hyperref[c1ii1]{Abschnitt II.1} beschrieben wurde, unterscheidet!



<!-- c1iii20.html -->

\subsection{Bach: der größte Komponist und Lehrer (15 Inventionen)}
\label{c1iii20} 

Wir analysieren kurz Bachs fünfzehn zweistimmige Inventionen aus einfacher struktureller Sicht, um zu untersuchen, wie und warum er sie komponierte.
Das Ziel ist, besser zu verstehen, wie man Bachs Kompositionen übt und von ihnen profitiert.
Als Nebenprodukt können wir diese Ergebnisse benutzen, um darüber zu spekulieren, was Musik ist und wie Bach aus (wie wir zeigen werden) grundlegendem \enquote{Unterrichtsmaterial}, das sich nicht von Czerny oder Cramer-Bülow unterscheiden sollte, eine solch unglaubliche Musik erzeugte.
Klar hat Bach fortgeschrittene musikalische Konzepte der Harmonie, des Kontrapunkts usw. benutzt, die von Musiktheoretikern bis  zum heutigen Tag diskutiert werden, während andere \enquote{Unterrichtsmusik} hauptsächlich wegen ihres Werts für das Fingertraining geschrieben haben.
Hier untersuchen wir die Inventionen nur auf der einfachsten strukturellen Stufe.
Sogar auf dieser grundlegenden Stufe gibt es ein paar lehrreiche und faszinierende Ideen, die wir erforschen können und zu der Erkenntnis führen, daß Musik und Technik untrennbar sind.
 
Es gibt einen guten Aufsatz über Bachs Inventionen, ihre Geschichte usw. von Dr. Yo Tomita von der Queen's University in Belfast, Irland (www.music.qub.ac.uk/~tomita/essay/inventions.html).
Jede Invention benutzt eine andere der Tonleitern, die in den zu Bachs Zeiten favorisierten \hyperref[c2_2_wtk2]{Wohltemperierten Stimmungen} wichtig waren.
Die Inventionen wurden - ungefähr 1720 - ursprünglich für seinen ältesten Sohn Wilhelm Friedemann Bach geschrieben, als dieser 9 Jahre alt war.
Sie wurden später verändert und anderen Schülern gelehrt.

Die bemerkenswerteste Eigenschaft aller Inventionen ist, daß jede sich auf eine kleine Zahl \hyperref[c1iii7b]{paralleler Sets} konzentriert, gewöhnlich weniger als drei.
Nun könnte man sagen: \enquote{Das ist nicht fair - da praktisch jede Komposition in parallele Sets zerlegt werden kann, müssen die Inventionen natürlich aus lauter parallelen Sets bestehen. Was ist also neu daran?}
Das neue Element ist, daß jede Invention nur auf einem bis drei bestimmten parallelen Sets basiert, die Bach zum Üben ausgewählt hat.
Um das zu zeigen, listen wir diese parallelen Sets weiter unten für jede Invention auf.
Um sich ganz auf einfache parallele Sets zu konzentrieren, vermied Bach völlig den Gebrauch von Terzen und komplexeren Intervallen (in einer Hand), die \hyperref[c1iii7h]{Hanon} in seinen Übungen mit höheren laufenden Nummern benutzt hat.
Bach wollte, daß seine Schüler die parallelen Sets vor den Intervallen beherrschen.

Einzelne parallele Sets sind aus technischer Sicht fast trivial.
Deshalb sind sie so nützlich - sie sind leicht zu lernen.
Jeder mit einiger Erfahrung am Klavier kann lernen, sie sehr schnell zu spielen.
Die wahre technische Herausforderung tritt auf, wenn man zwei davon mit einer Verbindung dazwischen vereinen muß.
Bach wußte das offensichtlich und benutzte deshalb nur Kombinationen paralleler Sets als Bausteine.
Somit lehren uns die Inventionen, wie man parallele Sets und Verbindungen spielt - parallele Sets zu lernen macht keinen Sinn, wenn man sie nicht verbinden kann.
Im folgenden benutze ich den Begriff \enquote{lineare} parallele Sets, um Sets zu beschreiben, in denen die Finger nacheinander spielen (z.B. 12345), und \enquote{alternierende} Sets, wenn die Finger abwechselnd spielen (z.B. 132435).
Diese verbundenen parallelen Sets bilden das, was man in diesen Inventionen normalerweise \enquote{Motiv} nennt.
Die Tatsache, daß sie aus den grundlegendsten parallelen Sets erzeugt wurden, läßt jedoch darauf schließen, daß die \enquote{Motive} nicht wegen ihres musikalischen Gehalts ausgewählt wurden, sondern wegen ihres pädagogischen Werts, und die Musik wurde dann durch die Genialität Bachs hinzugefügt.
Dadurch konnte nur Bach eine solche Meisterleistung vollbringen; das erklärt, warum Hanon scheiterte.
Der Hauptgrund für das Scheitern Hanons ist natürlich, daß er im Gegensatz zu Bach die guten Übungsmethoden nicht kannte.
Ich habe im folgenden jeweils nur eine repräsentative Kombination der parallelen Sets für jede Invention angeführt; Bach benutzte sie in vielen Variationen, wie z.B. umgekehrt, gespiegelt usw.
Beachten Sie, daß Hanon seine Übungen im Grunde aus den gleichen parallelen Sets aufbaute, obwohl er das wahrscheinlich eher dem Zufall verdankte, daß er diese Motive aus Bachs Werken entnahm.
Vielleicht ist der überzeugendste Beweis dafür, daß Bach die parallelen Sets kannte, die mit steigender Nummer der Inventionen zunehmende Komplexität der von ihm ausgewählten parallelen Sets.


\label{c1iii20ps}

<h3><br>Liste der parallelen Sets in den einzelnen Inventionen (für die RH)</h3>

\begin{enumerate}[label={\arabic*.}] 

\item \label{c1iii20ps01}

1234 und 4231 (linear gefolgt von alternierend); das war ein Fehler, weil die erste Invention nur die einfachsten (linearen) Sets behandeln sollte.
Dementsprechend ersetzte Bach in einer späteren Änderung dieser Invention das alternierende Set 4231 durch zwei lineare Sets, 432 und 321.
Diese Änderung ist ein, meine These unterstützendes, Indiz dafür, daß Bach die parallelen Sets als grundlegende Einheiten für das Strukturstudium benutzte.
Die Reihenfolge der Inventionen in bezug auf die Schwierigkeit mag jedoch für die meisten von uns nicht dieselbe wie die in bezug auf die Komplexität der parallelen Sets sein, weil die strukturelle Einfachheit der parallelen Sets nicht immer ein leichteres Spielen bedeutet.


\item \label{c1iii20ps02}

Lineare Sets wie in \hyperref[c1iii20ps01]{\#1} aber mit einer größeren Vielfalt in den Verbindungen.
Eine zusätzliche Komplexität ist, daß das gleiche Motiv an den verschiedenen Stellen unterschiedliche Fingersätze erfordert.
Somit behandeln die ersten beiden Inventionen hauptsächlich lineare Sets, aber die zweite ist komplexer.


\item \label{c1iii20ps03}

324 und 321 (alternierend gefolgt von linear).
Ein kurzes alternierendes Set wird eingeführt.


\item \label{c1iii20ps04}

12345 und 54321 mit einer ungewöhnlichen Verbindung.
Diese längeren linearen Sets mit der ungewöhnlichen Verbindung erhöhen die Schwierigkeit.


\item \label{c1iii20ps05}

4534231.
Völlig alternierende Sets.


\item \label{c1iii20ps06}

545, 434, 323 usw.
Das einfachste Beispiel der grundlegenden zweinotigen parallelen Sets, die mit einer Verbindung verknüpft werden; diese sind schwierig, wenn schwache Finger einbezogen werden.
Obwohl sie einfach sind, sind sie extrem wichtige technische Grundelemente, und sie zwischen den beiden Händen zu wechseln, ist eine großartige Möglichkeit, zu lernen sie zu kontrollieren (indem man \hyperref[c1ii20]{eine Hand benutzt, um die andere zu unterrichten}, s. Abschnitt II.20).
Sie führt auch die arpeggioartigen Sets ein.


\item \label{c1iii20ps07}

543231.
Das ist wie eine Kombination von \hyperref[c1iii20ps03]{\#3} und \hyperref[c1iii20ps04]{\#4} und ist deshalb komplexer als beide.


\item \label{c1iii20ps08}

14321 und die erste Einführung der \enquote{Alberti}-artigen Kombination 2434.
Hier wird die Steigerung der Schwierigkeit durch die Tatsache erzeugt, daß die 14 am Anfang nur einen oder zwei Halbtöne entfernt sind, was es für Kombinationen, die schwache Finger einbeziehen, schwieriger macht.
Es ist erstaunlich, daß Bach nicht nur alle Kombinationen mit den schwachen Fingern kannte, sondern auch, wie er sie in reale Musik einband.
Darüber hinaus wählte er Situationen, in denen er den schwierigen Fingersatz benutzen mußte.


\item \label{c1iii20ps09}

Der Lehrstoff ist hier ähnlich dem in \hyperref[c1iii20ps02]{\#2} (lineare Sets) aber schwieriger.


\item \label{c1iii20ps10}

Dieses Stück besteht fast vollständig aus arpeggioartigen Sets.
Da die Finger bei arpeggioartigen Sets eine größere Strecke zwischen den Noten zurücklegen müssen, stellen sie eine weitere Steigerung in der Schwierigkeit dar.


\item \label{c1iii20ps11}

Ähnlich wie \hyperref[c1iii20ps02]{\#2} und \hyperref[c1iii20ps09]{\#9}.
Die Schwierigkeit wird erneut gesteigert, indem das Motiv gegenüber den vorangegangenen Stücken verlängert ist.
Beachten Sie, daß es in allen anderen Stücken nur ein kurzes Motiv gibt, dem ein einfacher Kontrapunktabschnitt folgt, was es vereinfacht, sich auf die parallelen Sets zu konzentrieren.


\item \label{c1iii20ps12}

Diese kombiniert lineare und arpeggioartige Sets und wird schneller gespielt als die vorherigen Stücke.


\item \label{c1iii20ps13}

Arpeggioartige Sets, wird schneller gespielt als \hyperref[c1iii20ps10]{\#10}.


\item \label{c1iii20ps14}

12321, 43234.
Eine schwierigere Version von \hyperref[c1iii20ps03]{\#3} (fünf Noten statt drei und schneller).


\item \label{c1iii20ps15}

3431, 4541.
Schwierige Kombinationen, die Finger 4 einbeziehen.
Diese Fingerkombinationen sind besonders schwer zu spielen, wenn mehrere von ihnen aneinandergereiht werden.

 \end{enumerate}
Die obige Liste zeigt, daß:

\begin{enumerate}[label={\roman*.}] 
\item es eine systematische Einführung in zunehmend komplexere parallele Sets gibt.
\item die Tendenz zu einer schrittweisen Steigerung der Schwierigkeit besteht, mit der Betonung der Entwicklung der schwächeren Finger.
\item die \enquote{Motive} in Wahrheit sorgfältig ausgewählte parallele Sets und Verbindungen sind, die wegen ihres technischen Werts ausgewählt wurden.

\end{enumerate}
Die Tatsache, daß Motive, die einfach wegen ihrer technischen Nützlichkeit ausgewählt wurden, benutzt werden können, um einige der größten Musikstücke, die jemals komponiert wurden, zu erzeugen, ist faszinierend.
Diese Tatsache ist für Komponisten nichts Neues.
Für den durchschnittlichen Musikliebhaber, der von Bachs Musik begeistert ist, scheinen diese Motive wegen der Vertrautheit, die sie  bei mehrfachem Anhören erzeugen, eine besondere Bedeutung mit scheinbar tiefem musikalischen Wert zu gewinnen.
In Wahrheit sind es nicht die Motive selbst, sondern wie sie in der Komposition benutzt werden, was den Zauber erzeugt.
Wenn Sie sich einfach die reinen Grundmotive ansehen, können sie kaum einen Unterschied zwischen Hanon und Bach feststellen, und trotzdem wird niemand die Hanon-Übungen als Musik ansehen.
Das ganze Motiv besteht eigentlich nur aus den parallelen Sets und dem angefügten Kontrapunktabschnitt, der so genannt wird, weil er als Kontrapunkt zu dem wirkt, was von der anderen Hand gespielt wird.
Bachs geschickte Verwendung des Kontrapunkts dient offensichtlich mehreren Zwecken, von denen einer das Erzeugen der Musik ist.
Es mag so erscheinen, als ob der Kontrapunkt (der in den Hanon-Übungen fehlt) keine technischen Lehren hinzufügt (der Grund, warum Hanon ihn ignorierte), aber Bach benutzte ihn, um Fertigkeiten wie z.B. Triller, Verzierungen, Staccatos, Unabhängigkeit der Hände usw. zu üben, und der Kontrapunkt macht es sicher viel einfacher, die Musik zu komponieren und ihre Schwierigkeitsstufe anzupassen.

Musik wird so durch eine \enquote{logische} Reihenfolge von Noten oder Gruppen von Noten erzeugt, die vom Gehirn erkannt werden, so wie Ballett, schöne Blumen oder eine herrliche Landschaft visuell erkannt werden.
Was ist diese \enquote{Logik}?
Ein großer Teil davon ist wie im visuellen Fall eine automatisierte, fast so etwas wie eine festverdrahtete, Datenverarbeitung des Gehirns; sie beginnt mit einer angeborenen Komponente (neugeborene Babys schlafen ein, wenn sie ein Wiegenlied hören), aber eine große Komponente kann erlernt werden (z.B. Bach von Rock 'n' Roll zu unterscheiden).
Aber sogar der erlernte Teil ist größtenteils automatisiert.
Mit anderen Worten: Wenn ein beliebiger Ton das Gehör erreicht, fängt das Gehirn sofort damit an, die Klänge zu verarbeiten und zu interpretieren, egal ob wir bewußt versuchen, die Information zu verarbeiten oder nicht.
Ein sehr großer Teil dieser automatisierten Verarbeitung geschieht, ohne daß wir Notiz davon nehmen, wie Tiefenwahrnehmung, Fokussierung der Augen, Bestimmung der Richtung und Quellen von Tönen, Bewegungen zum Gehen und Halten der Balance, Töne in beängstigende und beruhigende zu unterscheiden usw.
Das meiste dieser Verarbeitung ist angeboren oder erlernt, ist aber im Grunde außerhalb unserer bewußten Kontrolle.
Das Ergebnis dieser mentalen Verarbeitung ist, was wir Sinn für Musik nennen.
Akkordprogressionen und andere Elemente der Musiktheorie geben uns eine Vorstellung davon, was diese Logik ist.
Aber das meiste dieser \enquote{Theorie} ist heutzutage eine einfache Zusammenstellung von verschiedenen Eigenschaften wirklich existierender Musik.
Diese bilden keine ausreichend grundlegende Theorie, um uns zu gestatten, Musik zu erzeugen, obwohl sie uns erlauben, Fallen zu vermeiden und eine Komposition zu erweitern bzw. zu vervollständigen, wenn man erst einmal auf beliebige Weise ein realisierbares Motiv erzeugt hat.
Es scheint so, als ob die Musiktheorie heute immer noch sehr unvollständig wäre.
Hoffentlich können wir, indem wir weiterhin Musik von großen Meistern analysieren, langsam, Schritt für Schritt das Ziel erreichen, ein besseres Verständnis der Musik zu entwickeln.



<!-- c1iii21.html -->

\subsection{Klavierspielen und die Psychologie}
\label{c1iii21}

Uns allen ist bewußt, daß die Psychologie nicht nur in der Musik, sondern auch beim Erlernen des Klavierspielens eine wichtige Rolle spielt.
Es gibt zahlreiche Möglichkeiten, einen Vorteil aus unserem Verständnis der Psychologie zu ziehen, und wir werden einige dieser Methoden in diesem Abschnitt besprechen.
Die wichtigere, sofort zu erledigende Aufgabe ist jedoch, die psychologischen Fallen aufzudecken, die zu scheinbar unüberwindlichen Hindernissen für das Lernen des Klavierspielens führen, wie z.B. \enquote{Mangel an Talent} oder \enquote{\hyperref[c1iii15]{Nervosität}} beim \hyperref[c1iii14]{Auftreten}.
Ein weiteres Beispiel ist das Phänomen der \hyperref[c1iii16e]{Unfähigkeit großer Künstler zu unterrichten}, wie oben in Abschnitt 16.e beschrieben.
Dieses Phänomen wurde durch die psychologische Herangehensweise der Künstler an das Unterrichten erklärt, die ihr Herangehen an die Musik widerspiegelt.
Da die Psychologie der Musik nur minimal verstanden wird, erzeugen die Komponisten die Musik in ihrem Geist quasi \enquote{aus dem Nichts} - es gibt keine Formel für das Erzeugen von Musik.
Auf ähnliche Weise haben sie Ihre Technik dadurch erworben, daß sie sich das musikalische Ergebnis vorgestellt haben und  die Hände eine Möglichkeit finden ließen, dieses zu erreichen.
Es ist eine unheimliche Verkürzung des Wegs zu einem komplexen Ziel, wenn es funktioniert.
Für die meisten Schüler ist es jedoch ein höchst ineffizienter Weg, sich die Technik anzueignen, und wir wissen nun, daß es bessere Vorgehensweisen gibt.
Offensichtlich ist die Psychologie sowohl für das Lernen, das Üben und das Auftreten als auch für das Komponieren von Musik wichtig.

Die Psychologie wird hauptsächlich durch das Wissen kontrolliert, und es ist oft schwierig, zwischen Psychologie und Wissen zu unterscheiden.
In den meisten Fällen kontrolliert das Wissen, wie wir an ein Thema herangehen.
Aber die Psychologie bestimmt, wie wir dieses Wissen anwenden.
Es ist nun an der Zeit, einige besondere Punkte zu untersuchen.

Der vielleicht wichtigste Punkt ist, wie wir das Klavierspielenlernen sehen, oder unsere generelle Haltung dem Prozeß des Spielenlernens gegenüber.
Die Methoden dieses Buchs sind von den \enquote{intuitiven} Methoden diametral verschieden.
Wenn ein Schüler z.B. beim Lernen versagte, war es gemäß des alten Systems wegen eines Mangels an Talent, so daß das Versagen die Schuld des Schülers war.
Im System dieses Buchs ist das Versagen die Schuld des Lehrers, weil es die Aufgabe des Lehrers ist, alle für den Erfolg notwendigen Informationen zur Verfügung zu stellen.
Es gibt kein blindes Vertrauen mehr darin, daß jeden Tag eine Stunde lang \hyperref[c1iii7h]{Hanon} zu üben jemanden in einen Virtuosen verwandelt.
Es sollte tatsächlich nichts aufgrund bloßen Vertrauens angenommen werden, und es ist die Verpflichtung des Lehrers, jede Methode zu erklären, so daß der Schüler sie versteht.
Das erfordert, daß der Lehrer sich in einer breiten Vielzahl von Disziplinen auskennt, in der Kunst und in den Wissenschaften.
Wir sind an einem Punkt in der Geschichte angekommen, an dem Kunstlehrer die Wissenschaft nicht mehr ignorieren können.
Deshalb erfordert die Psychologie des Klavierspielenlernens sowohl für den Schüler als auch den Lehrer tiefgreifende Veränderungen der Grundhaltung.

Für die Schüler - besonders für diejenigen, die nach dem alten System mit Regeln ausgebildet wurden - reicht das Erleben des Übergangs vom alten zum neuen System von \enquote{sehr leicht} bis zur völligen Verwirrung.
Einige Schüler werden sofort die neue Befähigung und Freiheit genießen und - innerhalb einer Woche - den vollen Nutzen aus den Methoden ziehen.
Auf der anderen Seite wird es Schüler geben, die erkennen, daß die alten Regeln nicht mehr gelten und nach \enquote{neuen Regeln} Ausschau halten, die sie befolgen können.
Sie haben jede Menge Fragen: Wenn ich eine Hand \hyperref[c1iii2]{zirkuliere}, sind zehnmal genug oder muß ich es 10.000 mal tun?
Zirkuliere ich so schnell wie ich kann oder mit einer langsameren, genaueren Geschwindigkeit?
Ist HS-Üben auch dann notwendig, wenn ich bereits HT spielen kann?
Bei einfacher Musik kann HS-Üben schrecklich langweilig sein - warum brauche ich es?
Solche Fragen enthüllen das Ausmaß, in dem der Schüler sich an die neue Denkweise angepaßt hat oder nicht.
Lassen Sie uns zur Verdeutlichung die letzte Frage näher analysieren.
Um solch eine Frage zu stellen, muß diese Person blind geübt haben, weil sie gelesen hat, daß es notwendig ist, HS zu üben.
Mit anderen Worten: Sie hat eine Regel blind befolgt.
Das ist nicht die Methode dieses Buchs.
Wir definieren ein Ziel und benutzen dann das HS-Üben, um dieses Ziel zu erreichen.
Dieses Ziel kann ein sichereres \hyperref[c1iii6]{Gedächtnis} sein, um Gedächtnisblockaden während der Aufführung zu vermeiden, oder die technische Entwicklung, so daß man, wenn man HT spielt, hören kann, daß das Spielen auf überlegenen technischen Fertigkeiten basiert.
Wenn diese Ziele erreicht werden, ist das Üben überhaupt nicht langweilig!

Für den Lehrer steht außer Frage, daß alles in der modernen Gesellschaft auf einer breiten Ausbildung basiert.
Es ist nicht notwendig, ein Wissenschaftler zu werden oder fortgeschrittene Konzepte der Psychologie zu studieren.
Erfolg in der realen Welt ist nicht an akademische Leistungen gebunden; die meisten erfolgreichen Unternehmer sind keine diplomierten Wirtschaftswissenschaftler.
Der vielleicht wichtigste Fortschritt der modernen Gesellschaft ist, daß all diese Konzepte, die als Spezialwissen fortgeschrittener Gebiete angesehen wurden, leichter verständlich werden; nicht weil sie sich geändert haben, sondern weil ein besseres Verständnis die Dinge immer vereinfacht und die Lehrmethoden immer besser werden.
Außerdem werden wir vertrauter mit ihnen, weil wir sie zunehmend in unserem täglichen Leben benötigen.
Es wird einfacher, auf die Informationen zuzugreifen.
Deshalb muß ein Lehrer nur neugierig sein und gewillt, mit anderen zu kommunizieren, und die Ergebnisse werden automatisch folgen.

Viele von uns brauchen ein psychologisches Mittel, um die unbegründete Angst vor der Unfähigkeit \hyperref[c1iii6]{auswendig zu lernen} zu überwinden.
In diesem Buch sprechen wir nicht darüber, nur \enquote{Für Elise} auswendig zu lernen.
Wir sprechen über ein Repertoire von mehr als 5 Stunden richtiger Musik, wobei Sie sich bei den meisten Stücken einfach ans Klavier setzen und sie sofort spielen können.
Um ein solches Repertoire zu \hyperref[c1iii6k]{pflegen}, ist es nur erforderlich, daß Sie jeden Tag auf dem Klavier spielen.
Einige Menschen haben keine Schwierigkeiten mit dem Auswendiglernen, aber die meisten haben die vorgefaßte Meinung, daß ein bedeutendes Repertoire auswendig zu lernen nur etwas für die wenigen \enquote{Begabten} ist.
Der Hauptgrund für diese unbegründete Angst ist die frühere Angewohnheit, Schülern zuerst beizubringen, ein Stück gut zu spielen, und erst danach, es auswendig zu lernen, was die schwierigste Art auswendig zu lernen ist (wie in Abschnitt III.6 beschrieben).
Für Schüler, die von Anfang an richtig unterrichtet wurden, ist das Auswendiglernen zur zweiten Natur geworden; es ist ein integraler  Bestandteil davon, eine neue Komposition zu lernen.
Diese Vorgehensweise wird Sie automatisch zu einem guten Auswendiglernenden machen, obwohl das bei älteren Menschen viele Jahre dauern kann.

\hyperref[c1iii15]{Nervosität} ist eine besonders schwer zu überwindende psychologische Barriere.
Um erfolgreich zu sein, muß man verstehen, daß Nervosität ein rein psychologischer Prozeß ist.
Das derzeitige System, junge Schüler ohne richtige \hyperref[c1iii14]{Vorbereitung} in Konzerte zu hetzen, ist kontraproduktiv und erzeugt im allgemeinen Schüler, die für Probleme mit der Nervosität anfälliger sind als zu Beginn ihres Unterrichts.
Wenn ein Schüler beim Klavierspielen erst einmal intensive Nervosität erfahren muß, kann es einen negativen Einfluß auf alle ähnlichen Situationen haben, wie in Theaterstücken zu spielen oder jede andere Art öffentlichen Auftretens.
Deshalb ist das jetzige System für die psychologische Gesundheit allgemein schlecht.
Wie oben in Abschnitt 15 besprochen, ist die Nervosität für die meisten Menschen ein gut zu lösendes Problem, und ein gutes Programm für das Überwinden der Nervosität wird wegen des Stolzes, der Freude und dem Gefühl der Vollendung, das man hat, zur Stärkung der Persönlichkeit beitragen.

Die Psychologie zieht sich durch alles, was wir im Zusammenhang mit dem Klavier tun, von der Motivation der Schüler bis zu den Grundlagen der Musik und des Komponierens von Musik.
Schüler motiviert man am besten, indem man Übungsmethoden lehrt, deren Nutzen so groß ist, daß die Schüler nicht aufhören möchten.
Wettbewerbe und Konzerte sind große Motivatoren, aber sie müssen mit Vorsicht und mit einem richtigen Verständnis der Psychologie durchgeführt werden.
Am interessantesten sind die psychologischen Aspekte der Grundlagen von Musik.
Bach benutzte das einfachste thematische Material, \hyperref[c1iii20ps]{parallele Sets}, und zeigte, daß sie benutzt werden können, um die tiefste jemals geschriebene Musik zu komponieren und uns gleichzeitig zu lehren, wie man übt.
\hyperref[c1iv4]{Mozart benutzte eine Formel} zur Massenproduktion von Musik;
wir verstehen nun, wie er innerhalb eines so kurzen Lebens so viel schreiben konnte.
Beethoven benutzte Konzepte der Gruppentheorie als tragende Säule seiner Musik.
Er zeigte, wie man die Aufmerksamkeit des Publikums mit einer eingängigen Melodie in der RH aufrecht erhalten kann, während man die Emotionen mit der LH kontrolliert, so wie es die Fernsehindustrie heute auch tut, indem Sie uns ein aufregendes Video zeigt und dabei die Emotionen mit Klangeffekten kontrolliert.
Chopin, der wie kein anderer für seinen Romantizismus und seine einzigartige Musikalität bekannt ist, benutzte in seiner Fantaisie Impromptu mathematische Mittel, um Musik zu schreiben, die die \enquote{Klangmauer} durchbricht (s. \hyperref[c1iii2]{Abschnitt III.2}) und besondere Effekte im Gehirn erzeugt, die das Publikum zu fesseln vermögen.
Die \hyperref[c2_2]{chromatische Tonleiter} wurde von Intervallen abgeleitet, und die Musik folgt Akkordprogressionen, weil diese Beziehungen der hörbaren Frequenzen die Verfahren für die Informationsverarbeitung und das Gedächtnis im Gehirn vereinfachen.
Die Technik kann nicht von der Musik getrennt werden, und die Musik kann nicht von der Psychologie getrennt werden; deshalb ist Klavierüben nicht mit dem Aufbau von Fingermuskeln oder mit wiederholenden Übungen gleichzusetzen: Bei der Technik dreht sich letztendlich alles um das menschliche Gehirn.
Die Kunst und die Künstler bringen uns das alles nahe, lange bevor wir analytisch erklären können, warum es funktioniert.


\subsection{Zusammenfassung der Methoden}
\label{c1iii22}

Die Methoden basieren auf sieben Hauptprinzipien:

\begin{enumerate}[label={\arabic*.}] 
 \item \hyperref[c1ii7]{mit getrennten Händen üben} (HS, Abschnitt II.7),
 \item \hyperref[c1ii6]{abschnittsweise üben} (II.6),
 \item Entspannung (\hyperref[c1ii10]{II.10} und \hyperref[c1ii14]{14}),
 \item parallele Sets (\hyperref[c1ii11]{II.11}, \hyperref[c1iii7b]{III.7b}, \hyperref[c1iv2a]{IV.2a})
 \item \hyperref[c1iii6]{Auswendiglernen} (III.6),
 \item mentales Spielen (\hyperref[c1iii6tastatur]{III.6} und \hyperref[c1iii12]{12})
 \item und musikalisches Spielen (im ganzen Buch).
 \end{enumerate}
\begin{itemize} 
\item Lernen Sie nur musikalische Kompositionen, kein \hyperref[c1iii7h]{Hanon}, Czerny usw., aber Tonleitern, Arpeggios und die chromatische Tonleiter sind wichtig (siehe \hyperref[c1iii5a]{Abschnitt III.5}).
Ihr erstes Klavier sollte ein \hyperref[c1iii17b]{Digitalpiano} mit gewichteten Tasten sein; kaufen Sie dann so bald wie möglich einen qualitativ guten \hyperref[c1iii17d]{Flügel}.

\item Hören Sie sich Auftritte und Aufnahmen an.
Es ist nicht möglich, andere exakt nachzuahmen, und es wird Ihnen einige Anregungen geben, die Ihnen beim musikalischen Üben helfen.

\item \hyperref[c1iii6g]{Üben Sie alte, fertige Stücke kalt} (ohne Aufwärmen, siehe Abschnitt III.6g), um Ihre Fähigkeiten für das \hyperref[c1iii14]{Auftreten} zu stärken.

\item Wenn Sie ein neues Stück beginnen, schauen Sie sich die Notenblätter an, um die schwierigen Stellen zu ermitteln, und \hyperref[c1ii5]{üben Sie die schwierigsten Abschnitte zuerst}; dann:

 \begin{enumerate}[label={\alph*.}] 
 <li>Üben Sie \hyperref[c1ii7]{mit getrennten Händen} und mit sich überschneidenden Abschnitten (\hyperref[c1ii8]{Kontinuitätsregel}, Abschnitt II.8); wechseln Sie oft die Hände, wenn notwendig, alle fünf Sekunden.
Die gesamte technische Entwicklung sollte HS erfolgen.
 
 \item Lernen Sie das Stück zuerst HS \hyperref[c1iii6]{auswendig}, und beginnen Sie erst dann mit dem Üben für die Technik; versuchen Sie, so schnell wie möglich \hyperref[c1iii7i]{die endgültige Geschwindigkeit zu erreichen}.
Auswendiglernen nachdem man gelernt hat, das Stück gut zu spielen, funktioniert nicht.
Lernen Sie das \hyperref[c1ii12mental]{mentale Spielen}, sobald Sie mit dem Auswendiglernen beginnen, und benutzen Sie es, um sich ein \hyperref[c1iii12]{relatives und abolutes Gehör} anzueignen (Abschnitt III.12).
 
 \item Benutzen Sie die \hyperref[c1ii11]{parallelen Sets}, um Ihre Schwachstellen zu diagnostizieren; \hyperref[c1iii2]{zirkulieren} (Abschnitt III.2) Sie die parallelen Sets, um diese Schwächen zu eliminieren und um schnell zur endgültigen Geschwindigkeit zu kommen.
 
 \item \hyperref[c1ii6]{Teilen Sie schwierige Passagen} in kleine Abschnitte auf, die leicht zu spielen sind, und benutzen Sie diese Abschnitte für die \hyperref[c1ii14]{Entspannung} und die Geschwindigkeit.
 
  \end{enumerate}
</li>
\item Spielen Sie den letzten Durchlauf jeder wiederholenden Übung immer \hyperref[c1ii17]{langsam}, bevor Sie die Hände wechseln oder zu einem neuen Abschnitt übergehen.

\item Üben Sie stets das \hyperref[c1ii14]{Entspannen}, besonders HS; das schließt den gesamten Körper ein, inklusive der \hyperref[c1ii21]{Atmung und des Schluckens} (Abschnitt II.21).

\item \hyperref[c1ii22]{Spielen Sie durch Fehler hindurch}; halten Sie nicht an, um sie zu korrigieren, weil Sie dadurch ein Stottern entwickeln.
Korrigieren Sie die Fehler später, indem Sie im Bereich der Fehler abschnittsweise üben.

\item Benutzen Sie das \hyperref[c1ii19]{Metronom} kurz (üblicherweise ein paar Sekunden), um den \hyperref[c1iii1b]{Rhythmus} und die Geschwindigkeit zu prüfen; benutzen Sie es nicht, um die Geschwindigkeit schrittweise zu steigern oder für längere Zeit (mehr als einige Minuten).
\item 
Benutzen Sie das \hyperref[c1ii23]{Pedal} nur, wenn es in den Noten angezeigt ist; üben Sie ohne Pedal, bis Sie zufriedenstellend HT spielen können, und fügen Sie erst dann das Pedal hinzu.

\item Um das \hyperref[c1ii25]{beidhändige Spielen} (Abschnitt II.25) zu lernen, üben Sie HS bis zu einer Geschwindigkeit, die höher als die endgültige HT-Geschwindigkeit ist, bevor Sie mit dem HT-Üben beginnen.
Um schwierige Passagen HT zu üben, nehmen Sie einen kurzen Abschnitt davon, spielen Sie mit der schwierigeren Hand, und fügen Sie nach und nach die Noten der anderen Hand hinzu.

\item Üben Sie musikalisch, ohne Lautstärke aber mit Festigkeit, Autorität und Ausdruck.
Das Klavierspielenüben ist kein Training der Fingerstärke; es ist die Entwicklung von Fertigkeiten des Gehirns und von Nervenverbindungen für die Kontrolle und die Geschwindigkeit.
Lernen Sie bei Fortissimo-Passagen zuerst das \hyperref[c1ii14]{Entspannen}, die Technik und die Geschwindigkeit, und fügen Sie erst dann das Fortissimo hinzu.
Die Kraft für das Fortissimo kommt aus dem Körper und den Schultern, nicht aus den Armen.

\item Bevor Sie mit dem Üben aufhören, spielen Sie das, was Sie gerade geübt haben, \hyperref[c1ii17]{langsam}, um eine korrekte \hyperref[c1ii15]{automatische Verbesserung nach dem Üben (PPI, Abschnitt II.15)} zu garantieren, die hauptsächlich während des Schlafes stattfindet.
Das letzte, das Sie gebrauchen können, ist, daß die PPI Ihre Fehler einschließt (besonders die aus dem \hyperref[fpd]{Schnellspiel-Abbau} resultierenden - Abschnitt II.25).
 \end{itemize}




<!-- c1iv1.html -->

\section{Mathematische Theorien des Klavierspielens}
\label{c1iv1}

\subsection{Wozu braucht man mathematische Theorien?}

Jede Disziplin kann von einer grundlegenden mathematischen Theorie profitieren, wenn eine gültige Theorie formuliert werden kann.
Jedes Feld, das erfolgreich mathematisch aufbereitet wurde, hat zwangsläufig sprunghafte Fortschritte gemacht.
Wenn die Theorie erst einmal korrekt formuliert ist, dann können die mächtigen mathematischen Werkzeuge und Schlußfolgerungen mit großer Sicherheit angewandt werden.
Im folgenden finden Sie meinen ersten Versuch einer solchen Formulierung für das Klavierspielen.
Soweit ich weiß, ist es der erste seiner Art in der Geschichte der Menschheit.
Solche unerforschten Gebiete haben in der Vergangenheit sehr schnell einen enormen Nutzen erfahren.
Ich war selbst überrascht, wie viele nützliche, und manchmal bisher unbekannte, Schlußfolgerungen wir aus sehr rudimentären Theorien ziehen können.
Egal welche Mathematik ich im folgenden benutzen werde, es wird eine wirklich einfache Mathematik sein.
Bereits in diesem frühen Stadium können wir mit den einfachsten Konzepten viel erreichen.
Weitere Fortschritte sind offensichtlich durch die Anwendung höherer Mathematik möglich.
Ich werde auch ein paar dieser Möglichkeiten besprechen.

Es wird kaum in Frage gestellt, daß die Kunst des Klavierspielens unter einem totalen Mangel an mathematischer Analyse leidet.
Außerdem bezweifelt niemand, daß Geschwindigkeit, Beschleunigung, Impuls, Kraft, usw. beim Klavierspielen eine entscheidende Rolle spielen.
Unabhängig davon, welch ein Genie in dem Künstler steckt, muß die Kunst durch Fleisch und Knochen und durch eine mechanische Vorrichtung aus Holz, Filz und Metall übermittelt werden.
Deshalb befassen wir uns hier nicht nur mit einem mathematischen, sondern auch mit einem völlig wissenschaftlichen Ansatz, der die menschliche Physiologie, Psychologie, Mechanik und Physik einbezieht, die vereint das repräsentieren, was wir am Klavier tun.

Die Notwendigkeit für ein solches Vorgehen zeigt sich anhand der Tatsache, daß es viele Fragen gibt, auf die wir immer noch keine Antworten wissen.
Was ist eine Geschwindigkeitsbarriere?
Wie viele gibt es?
Was verursacht sie?
Gibt es eine Formel für das Überwinden von Geschwindigkeitsbarrieren?
Was tun Klavierspieler, wenn sie schrill bzw. sanft spielen oder flach bzw. tief?
Ist es möglich, zwei verschiedenen Klavierspielern beizubringen, dieselbe Passage in genau derselben Art zu spielen?
Gibt es irgendeine Möglichkeit, die verschiedenen Fingerbewegungen in der gleichen Weise wie die Gangarten der Pferde zu klassifizieren?
Wir werden alle diese Fragen im folgenden beantworten.

Die Vorteile einer mathematischen Theorie sind offensichtlich.
Wenn wir z.B. die Frage, was eine Geschwindigkeitsbarriere ist (oder was Geschwindigkeitsbarrieren sind - wenn die Theorie davon ausgeht, daß es mehr als eine gibt!), mathematisch beantworten können, dann sollte die Theorie uns sofort mögliche Lösungen dafür zur Verfügung stellen, diese Geschwindigkeitsbarriere(n) zu überwinden.
Heute weiß niemand, wie viele Geschwindigkeitsbarrieren es gibt.
Zu wissen, wie viele es gibt, wäre schon ein sagenhafter Fortschritt.
Es kann wichtig sein, mathematisch zu beweisen, daß zwei Klavierspieler niemals dasselbe Stück auf genau die gleiche Art spielen können (oder sogar ein einzelner Klavierspieler dasselbe Stück nicht zweimal auf die gleiche Art spielen kann).
Wenn das der Fall ist, kann es nicht schädlich sein, jemand anderem beim Spielen zuzuhören, weil man es sowieso nicht exakt imitieren kann (unter der Annahme, daß exakte Imitation nicht wünschenswert ist), und es ist dann als unmöglich bewiesen, einem Schüler beizubringen, einen berühmten Künstler exakt zu imitieren.
Das wird sicher einen Einfluß darauf haben, wie Lehrer die Schüler darin unterrichten, Beispiele von Aufnahmen berühmter Künstler zu benutzen.

Bis vor kurzem machten sich die Chemiker über die Physiker lustig, die zwar in der Lage waren, ihre Gleichungen auf viele Dinge anzuwenden, aber einfache chemische Reaktionen nicht einmal annähernd erklären konnten.
Die Biologie und die Medizin entwickelten sich anfangs ebenfalls eigenständig - mit wenig Mathematik und mit Methoden, die von fundamentaler Wissenschaft weit entfernt waren.
Medizin, Biologie und Chemie begannen ursprünglich als reine Kunst.
Mittlerweile sind alle drei Disziplinen äußerst mathematisch und basieren auf den fortgeschrittensten wissenschaftlichen Prinzipien.
Die sich daraus ergebenden Leistungen auf diesen Gebieten sind zu zahlreich, um sie hier alle aufzuführen.
Ein Beispiel: In der Chemie wurde die absolute Grundlage der Chemiker, das Periodensystem der Elemente, von den Physikern mit Hilfe der Quantenmechanik erklärt.
Als Ergebnis davon, daß sie wissenschaftlicher wurden, sind alle drei Disziplinen enorm erfolgreich und erzielen große Fortschritte.
Die \enquote{Verwissenschaftlichung} jeder Disziplin ist unvermeidlich; es ist wegen des möglichen zu erwartenden Nutzens nur eine Frage der Zeit.
Diese Verwissenschaftlichung wird sich auch für die Musik als nützlich erweisen.

Wie wenden wir also die exakte Wissenschaft der Mathematik auf etwas an, das als Kunst wahrgenommen wird?
Sicherlich wird das Ergebnis anfangs etwas grob sein, aber Verfeinerungen werden garantiert folgen.
Klaviertechniker wissen bereits, daß das Klavier selbst in seinem Design ein Wunder im Gebrauch der physikalischen Grundlagen ist.
Klaviertechniker müssen mit einem enormen Maß an Wissenschaft, Mathematik und Physik vertraut sein, um ihrer \enquote{Kunst} nachzugehen.
Eine mathematische Theorie des Klavierspielens muß mit einem wissenschaftlichen Ansatz beginnen, in dem jedes zur Diskussion stehende Element klar als Objekt definiert und klassifiziert wird; s. \enquote{\hyperref[c3_2]{Der wissenschaftliche Ansatz}} in Kapitel 3.
Ist das erst einmal erreicht, suchen wir nach allen relevanten Beziehungen zwischen diesen Objekten.
Diese Prozeduren bilden den Kern der Gruppentheorie. Sie ist elementar! Lassen Sie uns anfangen.



<!-- c1iv2.html -->

\subsection{Die Theorie der Fingerbewegung}
\label{c1iv2}

\subsubsection{Serielles und paralleles Spielen}
\label{c1iv2a}

Die Fingerbewegungen für das Klavierspielen können auf unterster Stufe in serielle und parallele Bewegungen unterteilt werden.
Beim seriellen Spielen werden die Finger nacheinander gesenkt um zu spielen.
Eine Tonleiter ist ein Beispiel für etwas, das seriell gespielt werden kann.
Beim parallelen Spielen bewegen sich alle Finger gleichzeitig.
Ein Akkord ist ein Beispiel für paralleles Spielen.
Wie wir später sehen werden, kann eine Tonleiter auch parallel gespielt werden.

Serielles Spielen kann durch eine oszillierende Funktion, wie z.B. eine trigonometrische Funktion, beschrieben werden.
Es ist im Grunde durch eine Amplitude (die Entfernung, die der Finger sich auf und ab bewegt) und eine Frequenz (wie schnell man spielt) charakterisiert.
Außer bei Akkorden und schnellen Rollungen können die meisten langsamen Stücke seriell gespielt werden, und Anfänger neigen dazu, mit seriellem Spielen zu beginnen.
Beim parallelen Spielen gibt es eine wohldefinierte Phasenbeziehung zwischen den verschiedenen Fingern.
Deshalb müssen wir nun die Phase etwas ausführlicher behandeln.

Die Phase ist ein Maß dafür, wo sich der Finger relativ zu den anderen Fingern befindet.
Angenommen, wir benutzen die trigonometrischen Funktionen (Sinus, Cosinus, usw.), um die Fingerbewegung zu beschreiben.
Dann befindet sich der Finger in seiner Ruheposition z. B. bei 0 Grad im Phasenraum.
Da wir wissen, wie das Klavier gespielt werden sollte, werden wir etwas von diesem Wissen in unsere Definition der Phase einbauen.
Weil das Abheben der Finger von den Tasten im allgemeinen nicht die richtige Art zu spielen ist, werden wir den Nullpunkt der Phase als die obere Ruheposition der Tasten definieren.
Somit liegt der Phasennullpunkt der schwarzen Tasten um die zusätzliche Höhe der schwarzen Tasten höher als der Phasennullpunkt der weißen Tasten.
Weiterhin nehmen wir an, daß es bezüglich der Phase nicht berücksichtigt wird, wenn man die Finger von den Tasten abhebt.
Diese Konventionen befinden sich in Übereinstimmung mit einer guten Technik und vereinfachen auch die Mathematik.
Dann können die Phasen dieser Bewegung folgendermaßen definiert werden:

\begin{itemize} 
 \item der Finger drückt den halben Weg nach unten = 90 Grad
 \item niederdrücken bis zur unteren Position = 180 Grad
 \item den halben Weg anheben = 270 Grad
 \item zurück in die Ausgangsposition anheben = 360 Grad, was wieder 0 Grad ist.
 \end{itemize}
Wenn nun beim parallelen Spielen der zweite Finger seine Bewegung beginnt, wenn der erste Finger bei 90 Grad ist, der dritte Finger beginnt, wenn der erste Finger bei 180 Grad ist, usw., dann werden bei diesem parallelen Spielen die Noten 4 mal so schnell gespielt wie beim seriellen Spielen mit derselben Fingergeschwindigkeit.
In diesem Fall beträgt die Phasendifferenz zwischen den Fingern 90 Grad.
Wenn man die Phasendifferenz auf 9 Grad senken würde, könnte man die Noten 40 mal so schnell spielen - dieses Beispiel zeigt die Stärke des parallelen Spielens für das Steigern der Spielgeschwindigkeit.
Bei einem Akkord ist die Phasendifferenz 0.

Serielles Spielen kann als paralleles Spielen definiert werden, bei dem die Phasendifferenz zwischen aufeinanderfolgenden Fingern ungefähr 360 Grad oder mehr ist oder bei dem die Phasen in keinem Zusammenhang stehen.
Eine Bewegung der Hand nützt sowohl dem seriellen als auch dem parallelen Spielen aber auf unterschiedliche Weise.
Sie hilft dem seriellen Spielen durch das Vergrößern der Amplitude.
Sie beeinflußt jedoch das parallele Spielen in erheblicherem Maß, indem sie Ihnen bei der Kontrolle der Phase hilft.
Mit diesen einfachen Definitionen können wir damit beginnen, einige nützliche Resultate zu erzeugen.


\subsubsection{Geschwindigkeitsbarrieren}
\label{c1iv2b}

Nehmen wir an, jemand beginnt ein Musikstück zu üben, indem er zunächst langsam spielt und überwiegend serielles Spielen benutzt, weil das der einfachste Weg ist (ignorieren wir erst einmal die Akkorde).
Wenn die Geschwindigkeit der Finger schrittweise gesteigert wird, wird er natürlich auf eine Geschwindigkeitsbarriere stoßen, weil menschliche Finger sich nicht unendlich schnell bewegen können.
Somit haben wir eine Geschwindigkeitsbarriere mathematisch entdeckt und zwar die Geschwindigkeitsbarriere des seriellen Spielens.
Wie überwindet man diese Geschwindigkeitsbarriere?
Wir müssen eine Spielmethode finden, die keine Geschwindigkeitsbegrenzung hat.
Das ist das parallele Spielen.
Beim parallelen Spielen steigert man die Geschwindigkeit, indem man die Phasendifferenz verringert.
D.h. die Geschwindigkeit ist der Phasendifferenz umgekehrt proportional.
Da wir wissen, daß die Phasendifferenz bis auf Null vermindert werden kann (was einen Akkord bedeutet), wissen wir, daß paralleles Spielen das Potential für unendliche Geschwindigkeit und deshalb keine theoretische Geschwindigkeitsbeschränkung hat.
Wir sind bei einer mathematischen Grundlage des \hyperref[c1ii9]{Akkord-Anschlags} angekommen!

Die Unterscheidung zwischen seriellem und parallelem Spielen ist in gewisser Weise künstlich und stark vereinfacht.
In Wahrheit wird praktisch alles parallel gespielt.
Deshalb diente uns die obige Diskussion nur zur Illustration, wie man eine Geschwindigkeitsbarriere definiert oder erkennt.
Die tatsächliche Situation jedes einzelnen ist zu komplex, um sie zu beschreiben (weil Geschwindigkeitsbarrieren durch schlechte Angewohnheiten, Streß und HT-Spielen verursacht werden), aber es ist klar, daß falsche Spielmethoden Geschwindigkeitsbarrieren erzeugen und jeder seine eigenen Fehler hat, die zu Geschwindigkeitsbarrieren führen.
Das wird durch die Benutzung der \hyperref[c1iii7b]{Übungen für Parallele Sets} gezeigt, mit denen man die Geschwindigkeitsbarrieren überwindet.
Das bedeutet, daß Geschwindigkeitsbarrieren nicht immer von sich aus vorhanden sind, sondern von jedem einzelnen \textit{erzeugt} werden.
Deshalb gibt es für jeden eine beliebige Zahl möglicher Geschwindigkeitsbarrieren und jeder hat einen unterschiedlichen Satz Geschwindigkeitsbarrieren.
Es gibt selbstverständlich allgemeine Klassen von Geschwindigkeitsbarrieren, wie z.B. jene, die durch Streß, falsche Fingersätze, Mangel an HS-Technik, Mangel an HT-Koordination, usw. erzeugt werden.
Meiner Meinung nach wäre es sehr kontraproduktiv, zu sagen, daß solche komplexen Konzepte nicht irgendwann wissenschaftlich oder mathematisch behandelt werden.
Wir müssen es tun.
So spielt z.B. beim parallelen Spielen die Phase eine sehr wichtige Rolle.
Indem man die Phase auf Null vermindert, können wir im Prinzip unendlich schnell spielen.

Können wir wirklich unendlich schnell spielen? Natürlich nicht.
Was ist also dann die Höchstgeschwindigkeit beim parallelen Spielen und welcher Mechanismus erzeugt diese Grenze?
Wir wissen, daß verschiedene Menschen verschiedene Geschwindigkeitsbeschränkungen haben, deshalb muß die Antwort einen Parameter einschließen, der von diesem Menschen abhängt.
Wenn wir diesen Parameter kennen, können wir erklären, wie man schneller spielt!
Sicherlich wird die schnellste Geschwindigkeit durch die kleinste Phasendifferenz bestimmt, die der einzelne kontrollieren kann.
Wenn die Phasendifferenz so klein ist, daß sie nicht kontrolliert werden kann, dann verliert die \enquote{parallele Spielgeschwindigkeit} ihre Bedeutung.
Wie mißt man diese winzige Phasendifferenz beim einzelnen Menschen?
Das kann durch das Anhören seiner Akkorde erreicht werden.
Die Genauigkeit des Akkord-Spiels, d.h. wie genau alle Noten des Akkords gleichzeitig gespielt werden können, ist ein gutes Maß für die Fähigkeit des einzelnen, die kleinsten Phasendifferenzen zu kontrollieren.
Deshalb muß man, um schnell parallel zu spielen, in der Lage sein, genaue Akkorde zu spielen.
Das bedeutet, daß Sie bei der Anwendung des \hyperref[c1ii9]{Akkord-Anschlags} zuerst in der Lage sein müssen, genaue Akkorde zu spielen, bevor Sie zum nächsten Schritt übergehen.

Es ist klar, daß es viele weitere Geschwindigkeitsbarrieren gibt, und die bestimmte Geschwindigkeitsbarriere sowie die Methoden für das Überwinden jeder Barriere hängen von der Art der Finger- oder Handbewegung ab.
So kann man unendliche Geschwindigkeit mit parallelem Spielen nur erreichen, wenn man eine unendliche Zahl von Fingern hat (z.B. für einen langen Lauf).
Leider haben wir nur zehn Finger, und oft stehen nur fünf für eine bestimmte Passage zur Verfügung, weil die anderen fünf benötigt werden, um andere Teile der Musik zu spielen.
Als eine grobe Näherung kann gelten, daß wenn serielles Spielen es gestattet, mit einer maximalen Geschwindigkeit von M zu spielen, dann kann man mit zwei Fingern mit einer Geschwindigkeit von 2M spielen, mit drei Fingern 3M, usw.
Die maximale Geschwindigkeit wird dadurch begrenzt, wie schnell man die Finger zirkulieren kann.
In Wahrheit stimmt das wegen des Impulsausgleichs nicht ganz (er erlaubt es, schneller zu spielen), was weiter unten gesondert behandelt wird.
Somit führt jede Zahl zur Verfügung stehender Finger zu einer anderen neuen Geschwindigkeitsbarriere.
Deshalb kommen wir zu zwei weiteren nützlichen Ergebnissen: 

\begin{enumerate} 
 \item Es existieren beliebig viele Geschwindigkeitsbarrieren.
 \item Man kann seine Geschwindigkeitsbarriere dadurch ändern, daß man den Fingersatz ändert.
\end{enumerate}
Allgemein gesagt: Je mehr Finger Sie beim parallelen Spielen benutzen können, bevor Sie zirkulieren müssen, desto schneller können Sie spielen.
Anders gesagt: Die meisten Verbindungen führen zu ihrer eigenen Geschwindigkeitsbarriere.


\subsubsection{Die Geschwindigkeit steigern}
\label{c1iv2c}

Diese Ergebnisse bieten uns auch die mathematische Grundlage, um den wohlbekannten Trick zu erklären, die Finger abzuwechseln, wenn man dieselbe Note mehrmals spielt.
Man mag zunächst denken, daß nur einen Finger zu benutzen einfacher wäre und mehr Kontrolle bieten würde, aber diese Note kann schneller wiederholt gespielt werden, indem man parallel spielt und so viele Finger benutzt wie man in dieser Situation kann, als wenn man seriell spielen würde.

Die Notwendigkeit für paralleles Spielen läßt schnell gespielte Triller ebenfalls zu einer besonders großen Herausforderung werden, weil Triller im allgemeinen mit nur zwei Fingern ausgeführt werden müssen.
Wenn man versuchen würde, mit einem Finger zu trillern, würde man bei einer Geschwindigkeit von sagen wir M auf eine Geschwindigkeitsbarriere treffen; wenn man mit zwei Fingern trillert, wird die Geschwindigkeitsbarriere bei 2M liegen (wobei wir wieder den Impulsausgleich vernachlässigen).
Schlägt die Mathematik einen anderen Weg vor, um noch höhere Geschwindigkeiten zu erreichen?
Ja: Phasenkürzung.

Sie können den Finger für das Spielen der Note senken aber ihn nur so weit heben, wie es notwendig ist, um den Repetiermechanismus zurückzusetzen, bevor Sie die nächste Note spielen.
Sie müssen den Finger vielleicht nur um 90 Grad statt der normalen 180 Grad anheben.
Das ist es, was ich mit Phasenkürzung meine; der unnötige Teil der kompletten Phase wird abgeschnitten.
Wenn die ursprüngliche Amplitude des Fingerwegs für die Bewegung von 360 Grad 2 cm betragen hat, dann muß der Finger bei einer Verkürzung um 180 Grad nur 1 cm bewegt werden.
Dieser eine Zentimeter kann weiter bis zu der Grenze reduziert werden, an der der Repetiermechanismus nicht mehr funktioniert, d.h. bei ungefähr 5 mm.
Phasenkürzung ist die mathematische Basis für die schnelle Repetierung des Flügels und erklärt, warum die schnelle Repetierung so konstruiert ist, daß sie mit einer kurzen Strecke bis zum Umkehrpunkt funktioniert.

Eine gute Analogie dafür, auf diese Art auf Geschwindigkeit zu kommen, ist die Bewegung eines Basketballs beim Dribbeln im Gegensatz zur schwingenden Bewegung eines Pendels.
Ein Pendel hat eine feste Schwingungsfrequenz unabhängig von der Schwingungsamplitude.
Ein Basketball \enquote{schwingt} jedoch schneller, wenn man näher am Boden dribbelt (wenn man die Dribbelamplitude reduziert).
Ein Basketballspieler wird es im allgemeinen schwer haben zu dribbeln, bis er diese Veränderung der Dribbelfrequenz in Abhängigkeit von der Dribbelhöhe gelernt hat.
Ein Klavier verhält sich (zum Glück!) mehr wie ein Basketball als ein Pendel, und die Trillerfrequenz steigt mit sinkender Amplitude, bis man die Grenze des Repetiermechanismus erreicht.
Beachten Sie, daß der Fänger auch beim schnellsten Triller eingehakt sein muß, d.h. die Taste muß immer vollständig heruntergedrückt sein.
Der Triller wird möglich, weil die mechanische Antwort des Fängers schneller ist als die schnellste Geschwindigkeit, die der Finger erreichen kann.

Die Trillergeschwindigkeit wird, außer durch die Höhe, bei der die Repetierung aufhört zu funktionieren, nicht durch den Klaviermechanismus begrenzt.
Deshalb ist es bei den meisten Klavieren \textit{[im Gegensatz zu den Flügeln]} schwieriger, schnell zu trillern, weil die Phasenkürzung hier keine so große Auswirkung hat.
Diese mathematische Schlußfolgerungen sind mit der Tatsache konsistent, daß wir um schnell zu trillern die Finger auf den Tasten halten und die Bewegungen auf das für das Funktionieren des Repetiermechanismus notwendige Minimum reduzieren müssen.
Die Finger müssen \enquote{tief in das Klavier} drücken und dürfen nur gerade genügend angehoben werden, um den Repetiermechanismus zu aktivieren.
Außerdem hilft es, die Saiten zu benutzen, um den Hammer zurückspringen zu lassen, genauso wie man einen Basketball vom Boden wegspringen läßt.
Beachten Sie, daß man einen Basketball bei einer gegebenen Amplitude schneller dribbeln kann, wenn man ihn fester herunterdrückt.
Auf dem Klavier wird dies dadurch erreicht, daß man die Finger fest auf die Tasten drückt und sie nicht \enquote{hochschweben} läßt während man trillert.

Ein weiterer wichtiger Faktor ist die Abhängigkeit der Kontrolle des Klangs, Staccatos und anderer Eigenschaften des Klavierklangs im Zusammenhang mit dem Ausdruck von der Funktion der Fingerbewegung (rein trigonometrisch oder hyperbolisch usw.).
Mit einfachen elektronischen Instrumenten ist es eine leichte Aufgabe, die exakte Fingerbewegung, zusammen mit der Tastengeschwindigkeit, Beschleunigung, usw., zu messen.
Diese Spieleigenschaften eines jeden Klavierspielers können mathematisch analysiert werden, um die charakteristischen elektronischen Signaturen zu ermitteln, die damit verbunden werden können, wie wir das Gehörte wahrnehmen, z.B. als wütend, gefällig, angeberisch, tief, flach, usw.
Die Bewegung der Taste kann z.B. unter Benutzung der schnellen Fourier-Transformation analysiert werden, und es sollte möglich sein, anhand der Ergebnisse jene Elemente der Bewegung zu identifizieren, die die entsprechenden hörbaren Merkmale erzeugen.
Indem man sich von diesen Merkmalen aus rückwärts arbeitet, sollte es dann möglich sein, zu ermitteln wie man spielen muß, um diese Effekte zu erzeugen.
Das ist ein völlig neues Gebiet des Klavierspielens, das bisher noch nicht erforscht worden ist.
Diese Art der Analyse ist nicht möglich, indem man sich nur eine Aufnahme eines berühmten Pianisten anhört, und mag das wichtigste Thema für die zukünftige Forschung sein.



<!-- c1iv3.html -->

\subsection{Die Thermodynamik des Klavierspielens}
\label{c1iv3}

Ein wichtiges Feld der Mathematik ist das Studium großer Anzahlen.
Auch wenn einzelne Ereignisse eines bestimmten Typs nicht vorhersagbar sind, verhalten sich große Anzahlen solcher Ereignisse oft gemäß strikter Gesetze.
Obwohl die Energien einzelner Wassermoleküle in einem Glas Wasser erheblich voneinander abweichen können, kann die Wassertemperatur sehr konstant bleiben und mit sehr großer Genauigkeit gemessen werden.
Hat Klavierspielen eine analoge Situation, die es uns erlaubt, die Gesetze großer Anzahlen anzuwenden und dabei einige nützliche Schlüsse zu ziehen?


\label{canonic}

Klavierspielen ist wegen der großen Anzahl von Variablen, die in das Erzeugen von Musik eingehen, ein komplexer Vorgang.
Das Studium großer Anzahlen wird durch das Zählen der \enquote{Anzahl der Zustände} eines Systems erreicht.
Die Gesamtzahl der so gezählten bedeutungsvollen Zustände könnte die \enquote{\hyperref[ueb-canonic]{kanonische Gesamtheit}} genannt werden, eine bedeutungsvolle Ansammlung, die zusammen ein Lied singt, das wir entziffern können.
Deshalb müssen wir nur die kanonische Gesamtheit berechnen, und wenn wir damit fertig sind, wenden wir einfach die bekannten mathematischen Regeln großer Systeme (z.B. der Thermodynamik) an und voilà!
Wir sind fertig!

Die fraglichen Variablen sind hier klar die unterschiedlichen Bewegungen des menschlichen Körpers, besonders jener Teile, die beim Klavierspielen wichtig sind.
Unsere Aufgabe ist es, all die Arten zu zählen, in denen der Körper beim Klavierspielen bewegt werden kann; das ist sicherlich eine sehr große Zahl; die Frage ist, ob sie groß genug für eine bedeutungsvolle kanonische Gesamtheit ist.

Da niemand jemals versucht hat, diese kanonische Gesamtheit zu berechnen, erforschen wir hier ein neues Gebiet, und ich werde nur eine ungefähre Schätzung abgeben.
Das Schöne an den kanonischen Gesamtheiten ist, daß am Ende, wenn die Berechnungen richtig sind (eine berechtigte Sorge bei etwas so neuem), die Methode, die benutzt wurde, um dorthin zu gelangen, üblicherweise unerheblich ist - man kommt immer zu denselben Antworten.
Wir berechnen die Gesamtheit, indem wir alle relevanten Variablen auflisten und den gesamten Parameterraum dieser Variablen zählen.
Fangen wir also an.

Beginnen wir mit den Fingern.
Finger können sich auf, ab und seitwärts bewegen sowie gekrümmt und gestreckt werden (drei Variablen).
Sagen wir, es gibt 10 meßbare verschiedene Positionen für jede Variable (Parameterraum = 10); wenn wir nur die Anzahl der 10er-Gruppen berücksichtigen, haben wir bisher 4, einschließlich der Tatsache, daß wir 10 Finger haben.
In Wahrheit gibt es viel mehr Variablen (wie das Rotieren der Finger um jede Fingerachse) und mehr als 10 meßbare Parameter je Variable, aber wir zählen nur die Zustände, die vernünftig benutzt werden können, um ein bestimmtes Stück auf dem Klavier zu spielen.
Der Grund für diese Einschränkung ist, daß wir die Resultate dieser Berechnungen benutzen werden, um zu vergleichen, wie zwei Personen dasselbe Stück spielen oder wie eine Person es zweimal hintereinander spielt.
Das wird später klarer werden.

Nun können die Handflächen angehoben oder gesenkt, seitwärts gebeugt und um die Achse des Unterarms gedreht werden.
Das sind 3 weitere 10er-Gruppen, womit wir insgesamt 7 hätten.
Der Unterarm kann angehoben oder gesenkt und seitwärts geschwungen werden; die neue Summe ist 9.
\textit{[Anatomisch genau genommen ist die Drehung des Handgelenks ja eine Drehung des Unterarms im Ellbogen, aber das ist hier egal, weil wir trotzdem jetzt bei 9 sind.]}
Der Oberarm kann vor oder zurück und seitwärts geschwungen werden; die neue Summe ist 11.
Der Körper kann vorwärts oder rückwärts und seitwärts bewegt werden; die neue Summe ist 13.
Dann gibt es noch die Variablen der Kraft, Geschwindigkeit und Beschleunigung; macht insgesamt mindestens 16.
Somit umfaßt der gesamte Parameterraum eines Klavierspielers viel mehr als 10 hoch 16 Zustände (eine 1 gefolgt von 16 Nullen!).
Die tatsächliche Anzahl für ein bestimmtes Musikstück ist um mehrere Zehnerpotenzen höher, weil die obige Berechnung nur für eine Note gilt und ein typisches Musikstück tausende oder zehntausende Noten enthält.
Der daraus resultierende Parameterraum ist deshalb ungefähr 10 hoch 20.
Das nähert sich dem Gesamtheitsraum für Moleküle; so hat z.B. ein Kubikzentimeter Wasser 10 hoch 23 Moleküle, von denen jedes viele Freiheitsgrade in der Bewegung und viele mögliche Energiezustände hat.
Da die Thermodynamik auf Wasservolumina von wesentlich weniger als 0,0001 Kubikzentimeter anwendbar ist, kommt die kanonische Gesamtheit eines Klavierspielers thermodynamischen Bedingungen sehr nah.

Wenn die kanonische Gesamtheit des Klavierspielers fast thermodynamische Eigenschaften hat, welchen Schluß können wir daraus ziehen?
Das wichtigste Ergebnis ist, daß jeder einzelne Punkt in diesem Phasenraum völlig irrelevant ist, weil die Wahrscheinlichkeit, daß man diesen Punkt reproduzieren kann, nahe Null ist.
Aus diesem Ergebnis können wir sofort einige Schlüsse ziehen:

\textbf{Erstes Gesetz der Klavierdynamik}: Keine zwei Personen können dasselbe Musikstück auf exakt die gleiche Weise spielen.
Eine Folgerung aus diesem ersten Gesetz ist, daß dieselbe Person, die dasselbe Musikstück zweimal spielt, es niemals exakt auf die gleiche Weise spielen wird.

Na und? Nun, das bedeutet, daß die Vorstellung, man könnte, wenn man jemandem beim Spielen zuhört, seine Kreativität dadurch verringern, daß man diesen Künstler imitiert, keine gültige Vorstellung ist, da ein exaktes Imitieren niemals möglich ist.
Es unterstützt wirklich die Lehrmeinung, nach der behauptet wird, daß einem guten Künstler zuzuhören nicht schädlich sein kann.
Jeder Pianist ist ein einzigartiger Künstler, und niemand wird jemals seine Musik reproduzieren.
Die Folgerung bietet eine wissenschaftliche Erklärung für den Unterschied zwischen dem Anhören einer Aufnahme (die eine Aufführung exakt reproduziert) und dem Zuhören bei einem Live-Konzert, das niemals exakt reproduziert werden kann (außer als Aufnahme).

\textbf{Zweites Gesetz der Klavierdynamik}: Wir können niemals jeden Aspekt davon, wie wir ein bestimmtes Stück spielen, völlig kontrollieren.

Dieses Gesetz ist nützlich für das Verständnis dafür, wie wir uns unbewußt schlechte Angewohnheiten aneignen können und wie die Musik wenn wir auftreten ihr Eigenleben bekommt und uns auf manche Arten außer Kontrolle gerät.
Die mächtigen Gesetze der Klavierdynamik übernehmen in diesen Fällen die Führung, und es ist nützlich, unsere Grenzen und die Quellen unserer Schwierigkeiten zu kennen, um sie soviel wie möglich zu kontrollieren.
Es ist ein wahrhaft demütigender Gedanke, nach langem, harten Üben festzustellen, daß wir ohne den leisesten Hauch einer Ahnung jede beliebige Zahl von schlechten Gewohnheiten angenommen haben könnten.
Das mag in der Tat eine Erklärung dafür bieten, warum es so nützlich ist, beim letzten Durchlauf während des Übens langsam zu spielen.
Indem man langsam und exakt spielt, verringert man den Gesamtheitsraum in hohem Maß und schließt die \enquote{schlechten} Bewegungen aus, die weit von dem Raum der \enquote{richtigen} entfernt sind.
Wenn diese Prozedur tatsächlich bestimmte schlechte Angewohnheiten eliminiert und sich der Effekt von Übungseinheit zu Übungseinheit kumuliert, dann kann das auf lange Sicht einen großen Unterschied in der Rate erzeugen, mit der Sie die Technik erwerben.



<!-- c1iv4.html -->

\subsection{Mozarts Formel, Beethoven und Gruppentheorie}
\label{c1iv4}

Es gibt eine enge, wenn nicht sogar unabdingbare, Beziehung zwischen Mathematik und Musik.
Zumindest sind ihnen eine große Zahl der fundamentalsten Eigenschaften gemeinsam, angefangen mit der Tatsache, daß die chromatische Tonleiter eine einfache logarithmische Gleichung ist \hyperref[c2_2]{(s. Kapitel Zwei, Abschnitt 2)}, und daß die Grundintervalle Verhältnisse der kleinsten ganzen Zahlen sind.
Bis jetzt interessieren sich wenige Musiker um der Mathematik willen für die Mathematik.
Praktisch jeder ist jedoch neugierig und hat sich von Zeit zu Zeit gefragt, ob die Mathematik irgendwie in das Erzeugen von Musik verwickelt ist.
Gibt es ein tiefes, zugrunde liegendes Prinzip, dem sowohl die Mathematik als auch die Musik unterworfen ist?
Außerdem gibt es die feststehende Tatsache, daß jedesmal, wenn wir die Mathematik erfolgreich auf ein Gebiet angewandt haben, wir auf diesem Gebiet mit gewaltigen Schritten vorangekommen sind.
Eine Möglichkeit, diese Beziehung zu untersuchen, ist, die Arbeit der größten Komponisten von einem mathematischen Standpunkt aus zu studieren.

Die folgenden Analysen beinhalten keine Eingaben seitens der Musiktheorie.
Als ich das erste Mal von Mozarts Formel hörte, war ich sehr begeistert, weil ich dachte, daß es ein wenig Licht in die Musiktheorie und die Musik an sich bringen könnte.
Sie mögen zunächst enttäuscht sein, so wie ich es war, wenn Sie herausfinden, daß Mozarts Formel sich als streng strukturell erweist.
Strukturelle Analysen haben bisher noch nicht viel Information darüber erbracht, wie man berühmte Melodien hervorbringt; aber dann macht das die Musiktheorie genauso wenig.
Die heutige Musiktheorie hilft nur dabei, \enquote{korrekte} Musik zu komponieren oder sie weiter auszuführen, wenn man erst einmal eine musikalische Idee bekommen hat.
Musiktheorie ist eine Klassifikation von Notenfamilien und ihren Arrangements in bestimmte Muster.
Wir können noch nicht die Möglichkeit ausschließen, daß Musik letzten Endes auf bestimmten nachweisbaren Arten von strukturellen Mustern basiert.
Ich habe zuerst von Mozarts Formel in einer Vorlesung eines Musikprofessors erfahren.
Ich habe die Quelle verloren - wenn jemand, der dieses Buch liest, eine Quelle kennt, lassen Sie es mich bitte wissen.

Es ist nun bekannt, daß Mozart praktisch seine ganze Musik, seit er sehr jung war, nach einer einzigen Formel komponiert hat, die seine Musik um einen Faktor von mehr als zehn erweiterte.
D.h., wann immer er sich eine neue Melodie ausdachte, die eine Minute dauerte, wußte er, daß seine endgültige Komposition mindestens 10 Minuten lang sein würde.
Manchmal war sie \textit{viel} länger.
Der erste Teil seiner Formel war, jedes Thema zu wiederholen.
Diese Themen waren im allgemeinen sehr kurz - nur 4 bis 10 Noten, viel kürzer als man annehmen würde, wenn man an ein musikalisches Thema denkt.
Diese Themen, die viel kürzer sind als die gesamte Melodie, verschwinden einfach in der Melodie, weil sie zu kurz sind, um erkannt zu werden.
Deshalb nehmen wir sie normalerweise nicht wahr, und das ist fast mit Sicherheit ein absichtliches Konstrukt des Komponisten.
Das Thema wird dann zwei- oder dreimal verändert und noch einmal wiederholt, um das zu erzeugen, was das Publikum als fortlaufende Melodie wahrnimmt.
Diese Änderungen bestehen aus der Anwendung verschiedener mathematischer und musikalischer Symmetrien wie Inversionen, Umkehrungen, harmonischen Veränderungen, geschickte Positionierung von Verzierungen usw.
Diese Wiederholungen werden zu einem Abschnitt zusammengestellt, und der ganze Abschnitt wird wiederholt.
Die erste Wiederholung trägt einen Faktor von zwei bei, die verschiedenen Veränderungen bringen einen weiteren Faktor von zwei bis sechs (oder mehr) und die Wiederholung des ganzen Abschnitts am Schluß bringt einen weiteren Faktor von zwei, was mindestens zu 2x2x2 = 8 führt.
Auf diese Art war er in der Lage, große Kompositionen mit einem Minimum an thematischem Material zu schreiben.
Zusätzlich folgten seine Änderungen des ursprünglichen Themas einer bestimmten Reihenfolge, so daß gewisse Stimmungen oder Färbungen der Musik in jeder Komposition in derselben Reihenfolge angeordnet waren.

Wegen dieser vorherbestimmten Struktur war er in der Lage, seine Kompositionen von überall in der Mitte beginnend oder Stimme für Stimme niederzuschreiben, weil er bereits im voraus wußte, wo jeder Teil hingehört.
Und er mußte nicht das ganze Stück niederschreiben, bevor nicht das letzte Teil des Puzzles an seinem Platz war.
Er konnte auch mehrere Stücke gleichzeitig komponieren, weil sie alle dieselbe Struktur hatten.
Diese Formel ließ ihn als ein größeres Genie erscheinen als er tatsächlich war.
Das führt uns natürlich zu der Frage, wieviel von seinem angeblichen \enquote{Genie} einfach eine Illusion war, die von solchen Manipulationen hervorgerufen wurde.
Das soll seine Genialität nicht in Frage stellen - die Musik beweist diese schließlich!
Viele der wundervollen Dinge, die diese Genies taten, waren jedoch das Ergebnis von relativ einfachen Mitteln, und wir können alle einen Vorteil daraus ziehen, indem wir die Details dieser Mittel herausfinden.
So vereinfacht Mozarts Formel zu kennen es z. B., seine Kompositionen zu zergliedern und auswendig zu lernen.
Der erste Schritt zu einem Verständnis seiner Formel ist die Fähigkeit, seine Wiederholungen zu analysieren.
Es sind keine einfachen Wiederholungen; Mozart benutzte sein Genie, um die Wiederholungen zu verändern und zu tarnen, so daß sie Musik erzeugten und, wichtiger noch, die Tatsache der Wiederholung nicht wahrgenommen wird.


\label{KV525}

Lassen Sie uns als Beispiel für die Wiederholungen die berühmte Melodie im Allegro seiner \enquote{Eine Kleine Nachtmusik}\footnote{KV525} untersuchen.
Das ist die Melodie, die am Anfang des Films \enquote{Amadeus} von Salieri gespielt und dem Pastor erkannt wurde.
Diese Melodie ist eine Wiederholung, die als eine Frage und eine Antwort gesetzt ist.
Die Frage ist die einer männlichen Stimme \hyperref[ueb-KV525]{\enquote{Oh, mein Schatz, kommst Du nachher zu mir?}} und die Antwort ist eine weibliche Stimme: \hyperref[ueb-KV525]{\enquote{Ja, oh ja, ich komm nachher zu Dir!}}
Die männliche Aussage wird mit nur zwei Noten erzeugt, die eine gebieterische Quarte bilden und dreimal wiederholt werden (sechs Noten), und die Frage wird erzeugt, indem am Ende drei ansteigende Noten hinzugefügt werden (das scheint für die meisten Sprachen universell zu sein - Fragen werden durch das Anheben der Stimme am Ende des Satzes gestellt).
Somit besteht der erste Teil aus 9 Noten.
Die Wiederholung ist eine Antwort in einer weiblichen Stimme, weil die Tonhöhe höher ist, und besteht wieder aus zwei Noten, dieses Mal in einer sanfteren kleinen Terz, die (Sie haben es geahnt!) dreimal wiederholt wird (sechs Noten).
Es ist eine Antwort, weil die letzten drei Noten sich abwärts schlängeln.
Wieder sind es insgesamt 9 Noten.
Die Effizienz, mit der er dieses Konstrukt erzeugt hat, ist erstaunlich.
Es ist sogar noch unglaublicher, wie er die Wiederholung tarnt, so daß man es nicht als Wiederholung ansieht, wenn man sich das Ganze anhört.
Praktisch seine ganze Musik kann auf diese Art analysiert werden, d.h. meistens als Wiederholungen.
Wenn Sie noch nicht überzeugt sind, nehmen Sie ein beliebiges seiner Stücke und analysieren es, und Sie werden dieses Muster finden.

Lassen Sie uns ein anderes Beispiel betrachten: die Sonate No. 16 in A, K 300 (KV 331, mit dem Rondo \enquote{Alla Turca} am Ende).
Die Grundeinheit des Anfangsthemas ist eine Viertelnote, gefolgt von einer Achtelnote.
Die erste Einführung dieser Einheit wird durch das Hinzufügen der Sechzehntelnote getarnt, auf die die Grundeinheit folgt.
Auf diese Art wird die Einheit im ersten Takt zweimal wiederholt.
Er übersetzt danach (in der Tonhöhe) die ganze verdoppelte Einheit des ersten Takts und wiederholt sie im zweiten Takt.
Der dritte Takt besteht nur aus einer zweimaligen Wiederholung der Grundeinheit.
Im vierten Takt tarnt er wieder die erste Einheit mit Hilfe der Sechzehntelnoten.
Die Takte 1 bis 4 werden dann mit kleinen Änderungen in den Takten 5 bis 8 wiederholt.
Von einem strukturellen Standpunkt aus ist jeder der ersten 8 Takte dem Muster des ersten Takts nachgebildet.
Von einem melodischen Standpunkt aus erzeugen diese 8 Takte zwei lange Melodien mit ähnlichen Anfängen aber verschiedenen Endungen.
Da alle 8 Takte wiederholt werden, hat er im Grunde seine anfängliche Idee, die im ersten Takt enthalten ist, mit 16 multipliziert!
Wenn man in Begriffen der Grundeinheit denkt, hat er sie mit 32 multipliziert.
Aber dann fährt er damit fort, diese Grundeinheit zu nehmen und unglaubliche Variationen zu erzeugen, um die ganze Sonate zu schaffen, so daß der endgültige Multiplikationsfaktor sogar noch größer ist.
Das ist mit der Feststellung gemeint, daß er Wiederholungen von Wiederholungen benutzt.
Indem er die Wiederholungen der veränderten Einheiten aneinanderreiht, erzeugt er am Ende eine Melodie, die sich wie eine lange Melodie anhört, bis man sie in Ihre Komponenten zerlegt.

In der zweiten Hälfte dieser Exposition führt er neue Veränderungen der Grundeinheit ein.
In Takt 10 fügt er zunächst eine Verzierung mit melodischem Wert hinzu, um die Wiederholung zu tarnen und führt danach eine weitere Änderung ein, indem er die Grundeinheit als Triole spielt.
Nachdem die Triole eingeführt ist, wird sie zweimal in Takt 11 wiederholt.
Takt 12 ist Takt 4 ähnlich; er ist eine Wiederholung der Grundeinheit aber auf eine solche Weise strukturiert, daß er als Verbindung zwischen den in Zusammenhang dazu stehenden vorangegangenen und nachfolgenden drei Takten fungiert.
Somit sind die Takte 9 bis 16 den Takten 1 bis 8 ähnlich, es steht aber eine andere musikalische Idee dahinter.
Die letzten zwei Takte (17 und 18) bilden das Ende der Exposition.
Mit diesen Analysen als Beispiele sollten Sie nun in der Lage sein, den Rest des Stückes zu zergliedern.
Sie werden dasselbe Muster von Wiederholungen das ganze Stück hindurch finden.
Wenn Sie mehr von seiner Musik analysieren, werden Sie weitere Komplexitäten einbeziehen müssen; er mag drei- oder sogar viermal wiederholen und andere Änderungen dazumischen, um die Wiederholungen zu tarnen.
Klar ist, daß er ein Meister der Tarnung ist, so daß die Wiederholungen und andere Strukturen üblicherweise nicht offensichtlich sind, wenn man nur der Musik ohne jegliche Absicht zur Analyse zuhört.

Mozarts Formel wurde wahrscheinlich hauptsächlich dazu entwickelt, seine Produktivität zu steigern.
Trotzdem mag er bestimmte magische (hypnotische?, \enquote{süchtig machende}?) Kräfte in den Wiederholungen der Wiederholungen gefunden haben, und er hatte wahrscheinlich seine eigenen musikalischen Gründe, die Stimmungen seiner Themen in der von ihm benutzten Reihenfolge anzuordnen.
D.h., wenn man seine Themen weiter nach den Stimmungen, die sie hervorrufen, klassifiziert, dann findet man, daß er die Stimmungen immer in derselben Reihenfolge anordnet.
Es stellt sich hier die Frage: \enquote{Wenn wir tiefer und tiefer graben, werden wir nur mehr von diesen einfachen strukturellen bzw. mathematischen Mitteln finden, die einfach aufeinander gestapelt sind, oder steckt mehr in der Musik?}
Es muß fast mit Sicherheit mehr dahinter stecken, aber bis jetzt hat noch niemand Hand daran gelegt, nicht einmal die großen Komponisten selbst - zumindest, soweit sie es uns gesagt haben.
Deshalb scheint es so, als ob das einzige, das wir Normalsterblichen tun können, weitergraben ist.

Der oben erwähnte Musikprofessor, der eine Vorlesung über Mozarts Formel hielt, behauptete auch, daß die Formel so streng befolgt wurde, daß man sie benutzen kann, um Mozarts Kompositionen zu identifizieren.
Elemente dieser Formel sind jedoch unter Komponisten wohlbekannt.
Somit ist Mozart nicht der Erfinder dieser Formel und ähnliche Formeln wurden wahrscheinlich von den Komponisten seiner Zeit ausgiebig benutzt.
Insbesondere einige von Salieris Kompositionen befolgen eine sehr ähnliche Formel; vielleicht war das ein Versuch Salieris, Mozart nachzuahmen.
Deshalb muß man einige Details der Mozart eigenen Formel kennen, damit man sie benutzen kann, um seine Kompositionen zu identifizieren.

Es gibt wenig Zweifel daran, daß eine starke Wechselwirkung zwischen Musik und Genie besteht.
Wir wissen nicht einmal, ob Mozart ein Komponist war, weil er ein Genie war, oder ob der Umstand, daß er von Geburt an beträchtlichen Kontakt mit Musik hatte, die Genialität erzeugte.
Die Musik trug sicherlich zur Entwicklung seines Gehirns bei.
Es kann sehr gut sein, daß Wolfgang Amadeus selbst das beste Beispiel für den \enquote{Mozart-Effekt} war, obwohl er nicht den Nutzen aus seinen eigenen Meisterwerken hatte.
In diesen ersten paar Jahren des neuen Jahrtausends fangen wir gerade an, einige Geheimnisse der Funktion des Gehirns zu verstehen.
Zum Beispiel dachte man bis vor kurzem teilweise zu unrecht, daß ein bestimmter Teil der geistig behinderten Menschen ein ungewöhnliches musikalisches Talent hätte.
Es stellt sich heraus, daß Musik einen starken Effekt auf die tatsächliche Funktionsweise des Gehirns und seiner motorischen Kontrolle hat.
Das ist einer der Gründe, warum wir immer Musik beim Tanzen oder Trainieren benutzen.
Der beste Beweis dafür kommt von Alzheimer-Patienten, die ihre Fähigkeit sich selber anzuziehen verloren haben, weil sie die verschiedenen Arten der Kleidungsstücke nicht erkennen können.
Es wurde entdeckt, daß diese Patienten sich oft selbst anziehen können, wenn man diese Prozedur mit der richtigen Musik begleitet!
\enquote{Richtige Musik} ist normalerweise Musik, die sie in früher Jugend gehört haben oder ihre Lieblingsmusik.
Deshalb können geistig behinderte Menschen, die extrem unbeholfen sind, wenn sie alltägliche Verrichtungen ausführen, sich plötzlich hinsetzen und Klavier spielen, wenn die Musik von der richtigen Art ist, die ihr Gehirn stimuliert.
Sie müssen deshalb nicht musikalisch talentiert sein; es ist die Musik, die ihnen neue Fähigkeiten verleiht.
In einem größeren Maßstab ist es natürlich nicht nur die Musik, die diese magischen Auswirkungen auf das Gehirn hat, wie es von einigen Behinderten bewiesen wird, die unglaubliche Mengen an Informationen auswendig lernen können oder mathematische Kunststücke ausführen können, die normale Menschen nicht ausführen können.
Es existiert ein grundlegenderer interner Rhythmus im Gehirn, den die Musik anregen kann.
Ich weiß nicht, was dieser Mechanismus ist, aber er muß irgendwie dem Taktzyklus von Computerchips analog sein.
Ohne den Taktzyklus würden diese Chips nicht arbeiten, und das Maß für ihre Leistung ist die Taktrate - 3 GHz-Chips sind besser als 1 GHz-Chips.

Wenn Musik solch nachhaltige Wirkungen auf Behinderte erzeugen kann, stellen Sie sich vor, was sie mit dem Gehirn eines aufblühenden Genies tun könnte, insbesondere während der Entwicklung des Gehirns in früher Kindheit.
Diese Auswirkungen gelten für jeden, der Klavier spielt, nicht nur für Behinderte oder Genies.
Haben Sie jemals gute Tage und schlechte Tage gehabt?
Haben Sie jemals bemerkt, daß wenn Sie ein Stück das erste Mal lernen, Sie es plötzlich unglaublich gut spielen können, und es dann verlieren, wenn Sie weiterüben, oder daß Sie viel besser spielen, wenn Sie mit guten Spielern - z.B. in einer Kammermusikgruppe - spielen?
Haben Sie es schwer gefunden aufzutreten, wenn das Publikum aus Leuten besteht, die das Stück besser spielen können als Sie es können?
Spielen Sie besser, wenn die Musik gut herauskommt und schlechter, sobald Sie Fehler machen?
Wahrscheinlich ja, und die Antworten auf diese Fragen liegen in der Beziehung zwischen der Musik und dem Gehirn.
Deshalb sollte uns das Verständnis dieser Beziehung sehr dabei helfen, einige dieser Schwierigkeiten zu überwinden.


\label{c1iv4Gruppe}

Die Benutzung der mathematischen Mittel ist tief in Beethovens Musik eingebettet.
Deshalb ist das einer der besten Plätze, um nach Informationen über die Beziehung zwischen Mathematik und Musik zu graben.
Ich sage nicht, daß andere Komponisten keine mathematischen Mittel benutzen.
Praktisch jede musikalische Komposition hat mathematische Grundlagen.
Beethoven war jedoch in der Lage, diese mathematischen Mittel ins Extreme zu erweitern.
Durch das Analysieren dieser Extremfälle können wir überzeugendere Beweise dafür finden, welche Arten von Mitteln er benutzte.

Wir wissen alle, daß Beethoven niemals wirklich höhere Mathematik studiert hat.
Trotzdem nahm er eine erstaunliche Menge Mathematik in seine Musik auf und das auf sehr hohen Stufen.
Der Anfang seiner Fünften Symphonie ist ein erstklassiger Fall, aber Beispiele wie dieses sind überaus zahlreich.
Er \enquote{benutzte} Konzepte der Gruppentheorie, um diese berühmte Symphonie zu komponieren.
Tatsächlich benutzte er, was Kristallographen die Raumgruppe der Symmetrie-Transformationen nennen!
Diese Gruppe bestimmt viele fortgeschrittene Technologien wie die Quantenmechanik, Kernphysik und Kristallographie, die die Fundamente der heutigen technischen Revolution sind.
Auf dieser Abstraktionsstufe \textbf{sind ein Diamantkristall und Beethovens 5. Symphonie ein und dasselbe!}
Ich werde diese bemerkenswerte Beobachtung im folgenden erklären.

Die Raumgruppe, die Beethoven \enquote{benutzte} (er hatte sicher einen anderen Namen dafür) wurde angewandt, um Kristalle zu charakterisieren, wie z.B. Silizium oder Diamant.
Es sind die Eigenschaften der Raumgruppe, die Kristallen gestatten, makellos zu wachsen, und deshalb ist die Raumgruppe die absolute Grundlage für die Existenz von Kristallen.
Da Kristalle durch die Raumgruppe charakterisiert sind, führt ein Verständnis der Raumgruppe zu einem grundlegenden Verständnis der Kristalle.
Das war hervorragend für Materialforscher, die daran arbeiteten, Kommunikationsprobleme zu lösen, weil die Raumgruppe den Rahmen bot, von dem aus sie ihre Studien starten konnten.
Es ist so, als ob die Physiker von New York nach San Francisco fahren müßten und die Mathematiker ihnen eine Straßenkarte geben würden!
So haben wir den Siliziumtransistor perfektioniert, der zu integrierten Schaltkreisen und der Computer-Revolution geführt hat.
Was ist also die Raumgruppe? Und warum war diese Gruppe so nützlich, um diese Symphonie zu komponieren?

Gruppen werden durch eine Reihe von Eigenschaften definiert.
Mathematiker haben herausgefunden, daß Gruppen, die auf diese Art definiert werden, mathematisch manipuliert werden können, und Physiker fanden sie nützlich: d.h., diese bestimmten Gruppen, die Mathematiker und Wissenschaftler interessierten, bieten uns einen Pfad zur Realität.
Eine der Eigenschaften von Gruppen ist, daß Sie aus Elementen und Operationen bestehen.
Eine weitere Eigenschaft ist, daß wenn man eine Operation auf ein Element ausführt, man ein anderes Element derselben Gruppe erhält.
Eine vertraute Gruppe ist die Gruppe der ganzen Zahlen: \-1, 0, 1, 2, 3 usw.
Eine Operation dieser Gruppe ist die Addition: 2 + 3 = 5.
Beachten Sie, daß die Anwendung der Operation + auf die Elemente 2 und 3 zu einem anderen Element der Gruppe, 5, führt.
Da die Operationen auch ein Element in ein anderes transformieren, werden sie auch Transformationen genannt.
Ein Element der Raumgruppe kann alles in jedem Raum sein: ein Atom, ein Frosch oder eine Note in jeder musikalischen Raumdimension wie Tonhöhe, Geschwindigkeit oder Lautstärke.
Die Operationen der Raumgruppe, die für die Kristallographie relevant sind, sind Translation, Rotation, Spiegelung, Inversion und die Unäre Operation.
Diese sind fast selbsterklärend (Translation bedeutet, daß man das Element eine bestimmte Entfernung im Raum bewegt), außer bei der Unären Operation, welche das Element im Grunde unverändert läßt.
Diese ist jedoch ein wenig subtil, weil sie nicht das gleiche ist wie die Gleichheitstransformation, und wird deshalb in den Lehrbüchern immer am Schluß aufgelistet.
Unäre Operationen sind im allgemeinen mit dem speziellsten Element der Gruppe verbunden, das wir das Unäre Element nennen könnten; in der oben erwähnten Gruppe der ganzen Zahlen wäre das die 0 für die Addition und die 1 für die Multiplikation (5+0 = 5x1 = 5).

Lassen Sie mich demonstrieren, wie man diese Raumgruppe im täglichen Leben benutzen könnte.
Können Sie erklären, warum beim Blick in den Spiegel die linke Hand zur rechten herumgedreht wird (und umgekehrt) aber Ihr Kopf nicht hinunter zu Ihren Füßen rotiert?
Die Raumgruppe sagt uns, daß man nicht die rechte Hand drehen und eine linke Hand bekommen kann, weil links-rechts eine Spiegelungsoperation ist, keine Rotation.
Beachten Sie, daß dies eine merkwürdige Transformation ist: obwohl Ihre rechte Hand im Spiegel Ihre linke Hand ist, ist die Warze auf Ihrer rechten Hand nun auf der linken Hand Ihres Spiegelbildes.
Die Spiegelungsoperation ist der Grund, warum die rechte Hand zur linken Hand wird, wenn man in einen flachen Spiegel schaut; ein Spiegel kann jedoch keine Rotation ausführen, deshalb bleibt Ihr Kopf oben und die Füße bleiben unten.
Gekrümmte Spiegel, die optische Streiche spielen (wie die Umkehrung der Position von Kopf und Füßen) sind komplexere Spiegel, die zusätzliche Operationen der Raumgruppe durchführen können, und die Gruppentheorie ist beim Analysieren von Bildern in einem gekrümmten Spiegel genauso hilfreich.
Die Lösung des Problems des Bildes im flachen Spiegel erschien ziemlich einfach, weil wir einen Spiegel zu Hilfe hatten und wir so vertraut mit Spiegeln sind.
Dasselbe Problem kann noch einmal anders dargelegt werden, und es wird sofort viel schwieriger, so daß die Notwendigkeit der Gruppentheorie für die Hilfe zur Lösung des Problems offensichtlicher wird.
Wenn Sie einen rechtshändigen Handschuh von innen nach außen kehren, wird er ein rechtshändiger bleiben oder wird er zu einem linkshändigen Handschuh?
Ich überlasse es Ihnen, das herauszufinden (Tip: benutzen Sie einen Spiegel).

Lassen Sie uns sehen, wie Beethoven sein intuitives Verständnis der räumlichen Symmetrie benutzte, um seine 5. Symphonie zu komponieren.
Dieser berühmte erste Satz ist zu einem großen Teil aus einem einzigen musikalischen Thema aufgebaut, das aus vier Noten besteht, von denen die ersten drei Wiederholungen derselben Note sind.
Da die vierte Note anders ist, wird sie Überraschungsnote genannt und trägt den Schlag.
Dieses musikalische Thema kann schematisch durch die Folge 555\textbf{3} repräsentiert werden, wobei \textbf{3} die Überraschungsnote ist.
Das ist eine auf der Tonhöhe basierende Raumgruppe; Beethoven benutzte einen Raum mit 3 Dimensionen: Tonhöhe, Zeit und Lautstärke.
Ich werde in der folgenden Diskussion nur die Tonhöhen- und Zeitdimension berücksichtigen.
Beethoven beginnt seine Fünfte Symphonie indem er zunächst ein Element seiner Gruppe vorstellt: drei wiederholte Noten und eine Überraschungsnote, 555\textbf{3}.
Nach einer kurzen Pause, um uns Zeit zu geben, sein Element zu erkennen, führt er eine Translationsoperation aus: 444\textbf{2}.
Jede Note wird nach unten verschoben.
Das Ergebnis ist ein weiteres Element derselben Gruppe.
Nach einer weiteren Pause, so daß wir seinen Translationsoperator erkennen können, sagt er \enquote{Ist das nicht interessant? Laß uns Spaß haben!} und demonstriert das Potential dieses Operators mit einer Serie von Translationen, die Musik erzeugt.
Um sicherzustellen, daß wir sein Konstrukt verstehen, mischt er dieses Mal keine anderen, komplizierteren Operatoren darunter.
In der darauffolgenden Serie von Takten fügt er zuerst den Rotationsoperator hinzu, was \textbf{3}555 erzeugt, und anschließend den Spiegelungsoperator, was zu \textbf{7}555 führt.
Irgendwo nahe der Mitte des ersten Satzes führt er schließlich das ein, was als Unäres Element interpretiert werden kann: 555\textbf{5}.
Beachten Sie, daß die Noten einfach wiederholt werden, was die Unäre Operation ist.

In den letzten schnellen Takten kehrt er zur selben Gruppe zurück, benutzt aber nur das Unäre Element, und zwar auf eine Weise, die eine Stufe komplexer ist.
Es wird immer dreimal wiederholt.
Das seltsame daran ist, daß diesem eine vierte Sequenz folgt - eine Überraschungssequenz 765\textbf{4}, die kein Element ist.
Zusammen mit dem dreifach wiederholten Unären Element bildet die Überraschungssequenz eine Supergruppe der ursprünglichen Gruppe.
Er hat sein Gruppenkonzept verallgemeinert!
Die Supergruppe besteht nun aus drei Elementen und einem Nichtelement der anfänglichen Gruppe, was die Bedingungen der anfänglichen Gruppe erfüllt (drei Wiederholungen und eine Überraschung).

Somit liest sich der Anfang von Beethovens Fünfter Symphonie, wenn man ihn in eine mathematische Sprache übersetzt, fast Satz für Satz wie das erste Kapitel eines Lehrbuchs für Gruppentheorie!
Erinnern Sie sich daran, daß die Gruppentheorie eine der höchsten Formen von Mathematik ist.
Das Material wird sogar in der richtigen Reihenfolge präsentiert, wie es in Lehrbüchern auftritt, von der Einführung des Elements bis zum Gebrauch der Operatoren, angefangen mit dem einfachsten, der Translation und am Schluß der subtilste, der Unäre Operator.
Beethoven demonstriert sogar die Allgemeingültigkeit des Konzepts, indem er eine Supergruppe aus der ursprünglichen Gruppe erzeugt.

Beethoven war von diesem 4-notigen Thema besonders angetan und benutzte es in vielen seiner Kompositionen, so z.B. im ersten Satz der Klaviersonate \enquote{Appassionata}, siehe LH in Takt 10.
Als Meister seines Fachs vermied er bei der Appassionata sorgsam die auf der Tonhöhe basierende Raumgruppe und benutzte andere Räume - er transformierte das Thema in einem Temporaum und einem Lautstärkenraum (Takte 234 bis 238).
Das ist eine weitere Unterstützung der Vorstellung, daß er einen intuitiven Begriff von der Gruppentheorie gehabt haben mußte und bewußt zwischen diesen Räumen unterschieden hat.
Es scheint eine mathematische Unmöglichkeit zu sein, daß diese vielen Übereinstimmungen seiner Konstrukte mit der Gruppentheorie nur durch Zufall entstanden sind und ist quasi ein Beweis, daß er irgendwie mit diesen Konzepten gespielt hat.

Warum war dieses Konstrukt in dieser Einführung so nützlich?
Es bietet mit Sicherheit eine einheitliche Plattform, an die man seine Musik anknüpfen kann.
Die Einfachheit und Einheitlichkeit gestatten dem Publikum, sich ohne Ablenkung nur auf die Musik zu konzentrieren.
Es hat auch einen süchtig machenden Effekt.
Diese unterschwelligen Wiederholungen (mal angenommen, das Publikum weiß nicht, daß er dieses bestimmte Mittel einsetzte) können einen großen emotionalen Effekt erzeugen.
Es ist wie bei einem Zaubertrick - er hat einen viel größeren Effekt, wenn wir nicht wissen, wie der Zauberer es macht.
Es ist eine Möglichkeit, das Publikum ohne sein Wissen zu kontrollieren.
So wie Beethoven ein intuitives Verständnis dieses gruppenartigen Konzepts hatte, so können wir alle spüren, daß irgendeine Art von Muster existiert, ohne daß wir es ausdrücklich erkennen.
Mozart erreichte durch die Verwendung von Wiederholungen einen ähnlichen Effekt.

Das Wissen um diese gruppenartigen Mittel, die er benutzt, ist für das Spielen seiner Musik sehr nützlich, weil es Ihnen genau sagt, was Sie tun sollten und was nicht.
Ein weiteres Beispiel davon findet man im dritten Satz seiner Waldstein-Sonate, in der der ganze Satz auf einem 3-notigen Thema basiert, das durch 15\textbf{5} repräsentiert wird (das erste CG\textbf{G} am Anfang).
Er macht das gleiche mit dem anfänglichen Arpeggio des ersten Satzes der Appassionata, mit einem Thema, das durch 53\textbf{1} repräsentiert wird (das erste CAb\textbf{F}).
In beiden Fällen verliert die Musik ihre Struktur, Tiefe und Spannung, sofern man nicht den Schlag auf der letzten Note beibehält.
Das ist bei der Appassionata besonders interessant, weil man bei einem Arpeggio normalerweise den Schlag auf die erste Note setzt, und viele Schüler machen tatsächlich diesen Fehler.
Wie in der Waldstein-Sonate wird dieses anfängliche Thema den ganzen Satz hindurch wiederholt und wird zunehmend offensichtlich, wenn der Satz voranschreitet.
Aber zu diesem Zeitpunkt ist das Publikum süchtig danach und merkt nicht einmal, daß es die Musik dominiert.
Diejenigen, die es interessiert, können gegen Ende des ersten Satzes der Appassionata nachsehen, wo Beethoven das Thema zu 31\textbf{5} transformiert und es in Takt 240 auf eine extreme und fast lächerliche Stufe anhebt.
Trotzdem wird der größte Teil des Publikums keine Vorstellung davon haben, welches Mittel Beethoven benutzt hat, und den wilden Höhepunkt genießen, der offensichtlich lächerlich extrem ist aber inzwischen eine mysteriöse Vertrautheit in sich trägt, weil das Konstrukt dasselbe ist und Sie es hunderte Male gehört haben.
Beachten Sie, daß dieser Höhepunkt viel von seinem Effekt verliert, wenn der Pianist nicht das Thema herausstellt (das im ersten Takt eingeführt wurde!) und die Schlagnote betont.

Beethoven liefert uns die Begründung für das unerklärliche 53\textbf{1}-Arpeggio am Anfang der Appassionata, wenn sich das Arpeggio in Takt 35 in das Hauptthema umwandelt.
Dann entdecken wir, daß das Arpeggio am Anfang eine von seinem Hauptthema abgeleitete Inversion ist und warum der Schlag dort ist, wo er ist.
Deshalb ist der Anfang dieses Stücks, bis Takt 35, eine psychologische Vorbereitung auf eines der schönsten Themen, das er komponiert hat.
Er wollte die Vorstellung des Themas in unser Gehirn einpflanzen, noch bevor wir es hören!
Das mag eine Erklärung dafür sein, warum dieses fremdartige Arpeggio am Anfang unter Benutzung einer unlogischen Akkordprogression zweimal wiederholt wird.
Durch eine Analyse dieser Art wird die Struktur des ganzen ersten Satzes offenbar, was uns dabei hilft, das Stück auswendig zu lernen, zu interpretieren und korrekt zu spielen.

Die Benutzung gruppentheoretischer Konzepte könnte eine zusätzliche Dimension sein, die Beethoven in seine Musik eingeflochten hat, vielleicht um uns wissen zu lassen, wie schlau er war, falls wir die Botschaft immer noch nicht empfangen haben.
Es kann der Mechanismus sein oder nicht sein, mit dem er die Musik generiert hat.
Deshalb gibt uns die obige Analyse nur einen kleinen Blick auf die mentalen Prozesse, die Musik inspirieren.
Einfach diese Mittel zu benutzen, führt nicht zu Musik.
Oder kommen wir nahe an etwas heran, daß Beethoven wußte aber niemandem verraten hat?



<!-- c1iv5.html -->

\subsection{Berechnung der Lernrate}
\label{c1iv5}

Es folgt mein grober Entwurf zur Berechnung der Lernrate der Methoden dieses Buchs.
Das Ergebnis deutet darauf hin, daß sie ungefähr 1000 mal schneller sind als die intuitive Methode.
Der große Faktor von 1000 macht es unnötig, den genauen Wert zu ermitteln, um zu zeigen, daß ein großer Unterschied besteht.
Das Ergebnis erscheint angesichts der Tatsache plausibel, daß viele Schüler, die ihr ganzes Leben lang hart gearbeitet und die intuitive Methode benutzt haben, nicht in der Lage sind, irgend etwas bedeutsames vorzuspielen, während ein glücklicher Schüler, der die richtigen Lernmethoden benutzt, in weniger als 10 Jahren ein Konzertpianist werden kann.
Es ist klar, daß der Unterschied in den Übungsmethoden den Unterschied zwischen einem Leben voller Frustrationen und einer lohnenden Karriere am Klavier ausmachen kann.
Nun bedeutet \enquote{1000 mal schneller} nicht, daß man innerhalb einer Millisekunde ein Pianist werden kann; es bedeutet, daß die intuitiven Methoden 1000 mal \textit{langsamer} sind als die guten Methoden.
Der Schluß, den wir daraus ziehen sollten, ist, daß unsere Lernrate mit den richtigen Methoden sehr nahe an denen von berühmten Komponisten wie Mozart, Beethoven, Liszt und Chopin liegen sollte.
Erinnern Sie sich daran, daß wir bestimmte Vorteile haben, die sich diesen verstorbenen Genies nicht boten.
Sie hatten nicht diese wundervollen Beethoven-Sonaten, Liszt- und Chopin-Etüden, usw., mit denen wir heute die Technik erwerben, oder die Kompositionen von Mozart, mit denen wir vom \enquote{Mozart-Effekt} profitieren, oder Bücher wie dieses, mit einer geordneten Liste von Übungsmethoden.
Zudem gibt es nun hunderte bewährter Methoden, um diese Kompositionen für den Technikerwerb zu nutzen (Beethoven hatte oft Schwierigkeiten, seine eigenen Kompositionen zu spielen, weil niemand die richtigen oder falschen Methoden kannte, sie zu üben).
Eine faszinierende historische Anmerkung ist, daß das einzige allgemein verfügbare Material zum Üben für alle diese großen Pianisten Bachs Kompositionen waren.
Das führt uns zu dem Gedanken, daß Bach zu studieren ausreichend sein mag, um die meisten grundlegenden Fertigkeiten des Klavierspielens zu erwerben.

Ich werde versuchen, eine detaillierte Berechnung durchzuführen, indem ich mit den fundamentalsten Prinzipien anfange und bis zum Endresultat voranschreite ohne unbekannte Schritte auszulassen.
Auf diese Weise können eventuelle Fehler in dieser Berechnung bereinigt werden, wenn wir unser Verständnis darüber, wie wir die Technik erwerben, verbessern.
Das ist, offensichtlich, der wissenschaftliche Ansatz.
Es gibt nichts neues in diesen Berechnungen, außer daß sie auf das musikalische Lernen angewandt werden.
Das mathematische Material ist einfach ein Rückgriff auf bekannte Algebra und Infinitesimalrechnung.

Die Mathematik kann benutzt werden, um Probleme auf die folgende Weise zu lösen.
Als erstes definiert man die Bedingungen, die die Natur des Problems bestimmen.
Wenn diese Bedingungen korrekt bestimmt wurden, gestatten Sie es, Differentialgleichungen aufzustellen; diese sind getreue, mathematische Aussagen über die Bedingungen.
Wenn die Differentialgleichungen aufgestellt sind, bietet die Mathematik Methoden sie zu lösen, um eine Funktion zur Verfügung zu stellen, die die Antworten auf die Probleme durch die Parameter beschreibt, die diese Antworten bestimmen.
Die Lösungen der Probleme können dann durch Einsetzen der geeigneten Parameterwerte in die Funktion berechnet werden.

Das physikalische Prinzip, das wir benutzt haben, um unsere Lerngleichung herzuleiten, ist die Linearität mit der Zeit.
Solch ein abstraktes Konzept mag so erscheinen, als hätte es nichts mit dem Klavier zu tun und ist sicherlich unbiologisch, es stellt sich aber heraus, daß das genau das ist, was wir brauchen.
Lassen Sie mich also das Konzept der \enquote{Linearität mit der Zeit} erklären.
Es bedeutet einfach proportional zur Zeit.
Wenn wir z.B. in der Zeit T eine Menge Technik L lernen (L steht für Lernen), dann sollten wir, wenn wir diesen Prozeß ein paar Tage später wiederholen, eine weitere Menge L in der gleichen Zeit T lernen.
Deshalb sagen wir, daß L in bezug auf T in dem Sinne linear ist, daß sie proportional sind; in 2T sollten wir 2L lernen.
Natürlich wissen wir, daß Lernen in hohem Maß nichtlinear ist.
Wenn wir denselben kurzen Abschnitt 4 Stunden lang üben, dann werden wir wahrscheinlich während der ersten 30 Minuten mehr erreichen als während der letzten 30 Minuten.
Wir sprechen aber über eine optimierte Übungssitzung, die einen Durchschnitt vieler Übungssitzungen darstellt, die über einen Zeitraum von mehreren Jahren ausgeführt wurden (in einer optimierten Übungssitzung werden wir nicht dieselben 4 Noten 4 Stunden lang üben!).
Wenn wir den Durchschnitt über all diese Lernprozesse bilden, dann neigen sie dazu ziemlich linear zu sein.
Innerhalb eines Faktors von 2 oder 3 ist Linearität sicherlich eine gute Näherung, und dieses Maß an Genauigkeit ist alles was wir brauchen.
Beachten Sie, daß die Linearität in der ersten Näherung nicht davon abhängt, ob man ein schneller oder langsamer Lerner ist; das ändert nur die Proportionalitätskonstante.
Deshalb kommen wir zur ersten Gleichung

<table border>L = kT \\ </table>
<h5>(Gleichung 1.1)</h5>
wobei L die Zunahme des Lernens im Zeitintervall T und k die Proportionalitätskonstante ist.
Wir versuchen, die Abhängigkeit von L von der Zeit zu finden, oder L(t), wobei t die Zeit ist (im Gegensatz zu T, was ein Zeitintervall ist).
Genauso ist L eine Zunahme des Lernens, während L(t) eine Funktion ist.

Nun kommt das erste interessante neue Konzept.
Wir können L kontrollieren; wenn wir 2L möchten, üben wir einfach zweimal.
Aber das ist nicht das L, das wir behalten, weil wir im Laufe der Zeit etwas L \textit{verlieren}, nachdem wir üben.
Leider können wir um so mehr vergessen, je mehr wir wissen; d.h. die Menge, die wir vergessen, ist zu der ursprünglichen Menge Wissen L(U) proportional.
Angenommen, wir haben L(U) erworben, ist die Menge L, die wir in T verlieren, deshalb:

<table border>L = \-kTL(U) \\ </table>
<h5>(Gleichung 1.2)</h5>
wobei die \enquote{k}s in den Gleichungen 1.1 und 1.2 unterschiedlich sind, aber wir benennen sie der Einfachheit halber nicht um.
Beachten Sie, daß k ein negatives Vorzeichen hat, weil wir L verlieren.
Gleichung 1.2 führt zu der Differentialgleichung
<table border>dL(t)/dt = \-kL(t) \\ </table>
<h5>(Gleichung 1.3)</h5>
wobei d für Differential steht (das ist alles normale Infinitesimalrechnung), und die Lösung dieser Differentialgleichung ist:

<table border>L(t) = Ke<sup>\-kt</sup> \\ </table>
<h5>(Gleichung 1.4)</h5>
wobei e die Basis des natürlichen Logarithmus ist (ca. 2,71828), und K ist eine neue Konstante, die mit k verbunden ist (der Einfachheit halber haben wir einen weiteren Term in der Lösung ignoriert, der im jetzigen Stadium unwichtig ist).
Gleichung 1.4 sagt uns, daß wir, wenn wir L gelernt haben, sofort anfangen, es exponentiell zur Zeit zu vergessen, wenn der Prozeß des Vergessens linear zur Zeit ist.

Da der Exponent nur eine Zahl ist, hat k in Gleichung 1.4 die Einheit 1/Zeit.
Wir werden k = 1/T(k) setzen, wobei T(k) die charakteristische Zeit genannt wird.
Hier bezieht sich k auf einen spezifischen Prozeß des Lernens und Vergessens.
Wenn wir Klavierspielen lernen, dann lernen wir durch eine Unzahl von Prozessen, von denen die meisten nicht vollständig verstanden werden.
Deshalb ist es im allgemeinen nicht möglich, für jeden Prozeß genaue Werte für T(k) zu bestimmen, so daß wir in den Berechnungen ein paar \enquote{intelligente Schätzungen} machen müssen.
Beim Klavierüben müssen wir schwieriges Material viele Male wiederholen, bevor wir es gut spielen können, und wir müssen jeder Wiederholung beim Üben eine Zahl (sagen wir i) zuordnen.
Dann wird Gleichung 1.4 zu

<table border>L(i,t,k) = K(i)e<sup>\-t(i)/T(k)</sup> \\ </table>
<h5>(Gleichung 1.5)</h5>
für jede Wiederholung i und jeden \enquote{Lernen/Vergessen}-Prozeß k.
Lassen Sie uns ein paar relevante Beispiele untersuchen.
Angenommen, Sie üben nacheinander 4 Noten eines parallelen Sets, spielen schnell und wechseln die Hände, usw. 10 Minuten lang.
Wir weisen der Ausführung eines parallelen Sets, das nur ungefähr eine halbe Sekunde dauert i = 0 zu.
Sie haben das eventuell zehn- oder hundertmal während der zehnminütigen Übungseinheit wiederholt.
Sie haben L(U) nach dem ersten parallelen Set gelernt.
Wir müssen aber den Betrag von L(U) berechnen, den wir nach der zehnminütigen Übungseinheit behalten.
Da wir viele Male wiederholen, müssen wir in Wahrheit das kumulierte Lernen von allen berechnen.
Gemäß Gleichung 1.5, ist dieser kumulative Effekt durch die Summe aller \enquote{L}s über alle Wiederholungen des parallelen Sets gegeben:

<table border>L(Total) = &sum; über i von K(i)e<sup>\-t(i)/T(k)</sup> \\ </table>
<h5>(Gleichung 1.6)</h5>
Lassen Sie uns nun ein paar Werte in Gleichung 1.6 einsetzen, damit wir ein paar Antworten bekommen.
Nehmen Sie eine Passage, die Sie (mit der intuitiven Methode) langsam in ungefähr 100 Sekunden HT spielen können.
Diese Passage enthält vielleicht 2 oder 3 parallele Sets, die schwierig sind und die Sie in weniger als einer Sekunde schnell spielen können, so daß Sie die Sets in diesen 100 Sekunden (mit den Methoden dieses Buchs) mehr als hundertmal wiederholen können.
Typischerweise sind diese 2 oder 3 Stellen die einzigen, die sie ausbremsen, so daß Sie, wenn Sie diese gut spielen können, die ganze Passage mit der endgültigen Geschwindigkeit spielen können.
Natürlich werden Sie die Stellen auch mit der intuitiven Methode viele Male wiederholen, aber lassen Sie uns den Unterschied im Lernen für jede der 100 Sekunden dauernden Wiederholungen vergleichen.
Für diesen schnellen Lernprozeß ist unsere Neigung das Gelernte zu \enquote{verlieren} ebenfalls schnell, so daß wir eine \enquote{Vergessenszeitkonstante} von ungefähr 30 Sekunden annehmen können; d.h. alle 30 Sekunden vergessen Sie fast 30\% von dem, was Sie von einer Wiederholung gelernt haben.
Beachten Sie, daß Sie auch nach langer Zeit niemals alles vergessen, weil der Prozeß des Vergessens exponentiell ist - exponentielle Abfälle erreichen niemals den absoluten Nullpunkt.
Auch können Sie mit parallelen Sets innerhalb kurzer Zeit viele Wiederholungen ausführen, so daß sich diese Lernereignisse schnell ansammeln.
Diese Vergessenszeitkonstante von 30 Sekunden hängt von dem Mechanismus des Lernens und Vergessens ab, und ich habe eine relativ kurze für schnelle Wiederholungen ausgewählt; wir werden unten eine viel längere untersuchen.

Eine Wiederholung von einem parallelen Set pro Sekunde angenommen, ist die Lernmenge durch die erste Wiederholung e<sup>\-100/30</sup> pprox 0,04 (Sie haben 100 Sekunden, um die erste Wiederholung zu vergessen), während Ihnen die letzte Wiederholung e<sup>\-1/30</sup> pprox 0,97 gibt und die durchschnittliche Lernmenge ungefähr dazwischen liegt: ungefähr 0,4 (wie wir sehen werden, müssen wir nicht genau sein). Wir kommen so mit Hilfe der parallelen Sets bei mehr als 100 Wiederholungen auf eine gelernte Menge von mehr als 40.
Bei der intuitiven Methode haben wir eine einzige Wiederholung oder e<sup>\-100/30</sup> pprox 0,04.
Der Unterschied ist ein Faktor von 40/0,04 = 1.000!
Bei einem solch großen Faktor brauchen wir keine große Genauigkeit, um zu zeigen, daß es einen großen Unterschied gibt.
Die tatsächliche Differenz in der Lernmenge kann sogar noch größer sein, weil die Wiederholung mit der intuitiven Methode mit niedriger Geschwindigkeit stattfindet, während die Wiederholrate der parallelen Sets mit der endgültigen Geschwindigkeit oder sogar einer noch höheren Geschwindigkeit erzielt wird.

Die Zeitkonstante von 30 Sekunden, die oben benutzt wurde, war für einen \enquote{schnellen} Lernprozeß, wie er mit dem Lernen \textit{während} einer einzelnen Übungseinheit verbunden ist.
Es gibt viele andere, wie den Technikerwerb durch die \hyperref[c1ii15]{Automatische Verbesserung nach dem Üben (PPI)}.
Nach jeder strengen Konditionierung wird sich Ihre Technik durch die PPI für eine Woche oder mehr verbessern.
Die Rate des Vergessens, oder des Technikverlusts, beträgt bei einem solch langsamen Prozeß nicht 30 Sekunden sondern viel mehr, wahrscheinlich mehrere Wochen.
Um die gesamte Differenz in der Lernrate zu berechnen, müssen wir deshalb die Differenz für alle bekannten Methoden des Technikerwerbs berechnen, wobei wir die entsprechende Zeitkonstante benutzen, die von Methode zu Methode beträchtlich abweichen kann.
Die PPI wird in hohem Maß durch das Konditionieren bestimmt, und das Konditionieren ist der oben berechneten Wiederholung der parallelen Sets ähnlich.
Deshalb sollte der Unterschied in der PPI ebenfalls ungefähr 1000 betragen.

Wenn wir die wichtigsten Raten wie oben beschrieben berechnen, können wir die Ergebnisse verfeinern, indem wir andere Faktoren berücksichtigen, die die endgültigen Ergebnisse beeinflussen.
Es gibt Faktoren, die die Methoden dieses Buchs langsamer machen (das Auswendiglernen kann zunächst länger dauern als das Spielen vom Blatt, oder HS kann länger dauern als HT, weil man jede Passage dreimal statt einmal lernen muß, usw.) und Faktoren, die sie schneller machen (wie das Lernen in kurzen Abschnitten, schnell auf Geschwindigkeit kommen, Geschwindigkeitsbarrieren vermeiden, usw.).
Es gibt viele weitere Faktoren, die die intuitiven Methoden langsamer machen, so daß das obige Resultat \enquote{1000 mal schneller} eine Unterschätzung sein kann.
Es ist jedoch wahrscheinlich nicht möglich, den vollen Vorteil aus dem 1000fachen Faktor zu ziehen, da die meisten Schüler bereist einige der Ideen dieses Buchs benutzen werden.

Die Auswirkungen der Geschwindigkeitsbarrieren sind schwer zu berechnen, weil Geschwindigkeitsbarrieren von jedem Klavierspieler künstlich erzeugt werden und ich nicht weiß, wie ich eine Gleichung dafür schreiben soll.
Die Erfahrung zeigt uns, daß die intuitive Methode für Geschwindigkeitsbarrieren anfällig ist.
Die Methoden dieses Buchs bieten viele Möglichkeiten, sie zu vermeiden.
Zudem werden die Geschwindigkeitsbarrieren hier klar definiert, so daß es möglich ist, sie während des Übens von vornherein zu vermeiden.
Parallele Sets sind das mächtigste Werkzeug, um sie zu vermeiden, weil Geschwindigkeitsbarrieren im allgemeinen nicht entstehen, wenn man die Geschwindigkeit von hoher Geschwindigkeit aus verringert.
Deshalb verzögern Geschwindigkeitsbarrieren die Lernrate für intuitive Methoden in hohem Maß.
Einige Lehrer, die die Geschwindigkeitsbarrieren nicht ausreichend verstehen, verbieten ihren Schülern, etwas gewagtes und schnelles zu üben, was den Fortschritt noch mehr verlangsamt, sogar wenn dieses langsame Spielen die Geschwindigkeitsbarrieren erfolgreich völlig vermeidet.
Wenn alle diese Faktoren berücksichtigt werden, kommen wir zu dem Schluß, daß das Ergebnis \enquote{bis zu 1000 mal schneller} im Grunde korrekt ist.
Wir sehen auch, daß der Gebrauch der parallelen Sets, schwierige Abschnitte zuerst üben, kurze Abschnitte üben und schnell auf Geschwindigkeit kommen die Hauptfaktoren sind, die das Lernen beschleunigen.
HS-Üben, Entspannung und frühzeitiges Auswendiglernen sind einige der Wegzeuge, die uns in die Lage versetzen, den Gebrauch dieser beschleunigenden Methoden zu optimieren.



<!-- c1iv6.html -->

\subsection{Noch zu erforschende Themen}
\label{c1iv6}

Die einzelnen Punkte dieses Abschnitts sind unvollständig; ich habe nur einige erste Ideen niedergeschrieben.

Dieses Buch basiert auf einem wissenschaftlichen Ansatz, was sicherstellt, daß Fehler so schnell wie möglich korrigiert werden, alle bekannten Fakten erklärt, dokumentiert und in einer nützlichen Weise geordnet werden und wir insgesamt nur voranschreiten.
In der Vergangenheit war es oft so, daß ein Klavierlehrer eine sehr nützliche Methode lehrte und ein anderer Lehrer nichts davon wußte oder zwei Lehrer völlig entgegengesetzte Methoden lehrten.
Das sollte nicht geschehen.
Ein wichtiger Teil des wissenschaftlichen Ansatzes ist eine Diskussion darüber, was immer noch unbekannt ist und was noch erforscht werden muß.
Es folgt eine Sammlung solcher Themen.


\subsubsection{Impulstheorie des Klavierspielens}
\label{c1iv6a}

Langsames Klavierspielen wird \enquote{Spielen im statischen Gleichgewicht} genannt.
Das bedeutet, daß die Kraft des herunterkommenden Fingers beim Herunterdrücken der Taste die hauptsächliche zum Spielen benutzte Kraft ist.
Wenn wir schneller werden, gehen wir vom statischen Gleichgewicht zum dynamischen Gleichgewicht über.
Das bedeutet, daß der Impuls der Hände, Arme, Finger, usw. eine viel wichtigere Rolle spielt als die Kraft, mit der die Tasten heruntergedrückt werden.
Selbstverständlich wird eine Kraft benötigt, um die Taste herunterzudrücken, aber im dynamischen Gleichgewicht sind die Phasen der Kraft und der Bewegung üblicherweise um 180  Grad zueinander versetzt, d.h. ihr Finger bewegt sich nach oben, wenn Ihre Fingermuskeln versuchen ihn herunterzudrücken!
Das geschieht bei hoher Geschwindigkeit, weil Sie den Finger vorher so schnell angehoben haben, daß Sie auf seinem Weg nach oben anfangen müssen herunterzudrücken, so daß Sie seine Bewegung für den nächsten Anschlag umkehren können.
Die wahren Bewegungen sind komplex, weil Sie die Hände, die Arme und den Körper benutzen, um die Impulse abzugeben und abzufangen.
Das ist einer der Gründe, warum der ganze Körper am Spielen beteiligt wird, besonders wenn man schnell spielt.
Beachten Sie, daß das Schwingen des Pendels und das Dribbeln des Basketballs im dynamischen Gleichgewicht stattfinden.
Beim Klavierspielen befinden Sie sich im allgemeinen irgendwo zwischen dem statischen und dem dynamischen Gleichgewicht - mit einer zunehmenden Tendenz zum dynamischen Gleichgewicht bei steigender Geschwindigkeit.

Beim statischen Spielen sind der Kraftvektor und die Bewegung des Fingers in Phase.
Wenn wir zum dynamischen Spielen übergehen, baut sich eine Phasendifferenz auf, bis sie im reinen dynamischen Gleichgewicht 180 Grad beträgt, wie es beim Pendel der Fall ist.

Die Wichtigkeit des dynamischen Spielens ist offensichtlich; es bezieht viele neue Finger- bzw. Handbewegungen ein, die im statischen Spielen nicht möglich sind.
Deshalb trägt das Wissen, welche Bewegungen der statischen oder dynamischen Art sind, viel zu dem Verständnis dafür bei, wie sie auszuführen sind und wann man sie benutzen muß.
Da das dynamische Spielen bis jetzt nie in der Literatur besprochen wurde, gibt es im Klavierspielen ein großes Gebiet, von dem wir sehr wenig verstehen.


\subsubsection{Die Physiologie der Technik}
\label{c1iv6b}

Wir haben immer noch ein sehr primitives Verständnis der biomechanischen Prozesse, die der Technik zugrunde liegen.
Sie hat ihren Ursprung sicherlich im Gehirn und ist wahrscheinlich damit verbunden, wie die Nerven mit den Muskeln kommunizieren, besonders mit den schnellen Muskeln.
Was sind die biologischen Veränderungen, die mit der Technik einhergehen?
Wann sind Finger \enquote{aufgewärmt}?


\subsubsection{Gerhirnforschung (HS- und HT-Spielen, usw.)}
\label{c1iv6c}

Die Gehirnforschung wird in naher Zukunft eines der wichtigsten Gebiete der medizinischen Forschung sein.
Diese Forschung wird sich anfänglich auf die Verhinderung des geistigen Verfalls durch das Altern (z.B. die Heilung von Alzheimer) konzentrieren.
Es werden sicherlich gleichzeitig Anstrengungen zur tatsächlichen Kontrolle des Wachstums der geistigen Fähigkeiten unternommen.
Die Musik sollte bei solchen Entwicklungen eine wichtige Rolle spielen, weil wir mit Kindern aural \textit{[d.h. über die Ohren]} kommunizieren können, lange bevor wir es mit einer anderen Methode können, und es ist bereits klar, daß die Resultate um so besser sind, je früher man den Kontrollprozeß beginnt.

Wir sind alle mit der Tatsache vertraut, daß sogar wenn wir ziemlich gut HS spielen können, HT trotzdem sehr schwierig sein kann.
Warum ist HT soviel schwieriger?
Einer der Gründe mag sein, daß die beiden Hände von den verschiedenen Hälften des Gehirns gesteuert werden.
Wenn das so ist, dann erfordert das Lernen von HT, daß das Gehirn Wege entwickelt, die beiden Hälften zu koordinieren.
Das würde bedeuten, daß das HS- und das HT-Üben völlig verschiedene Arten von Gehirnfunktionen benutzen und stützt die Behauptung, daß diese Fertigkeiten separat entwickelt werden sollten, so daß wir jeweils an einer Fertigkeit arbeiten können.
Eine faszinierende Möglichkeit wäre, wenn wir parallele Sets für HT entwickeln könnten, die dieses Problem lösen.


\subsubsection{Was verursacht Nervosität?}
\label{c1iv6d}

In der Klavierpädagogik wurde die Nervosität zu lange \enquote{unter den Teppich gekehrt} (ignoriert).
Wir müssen sie von einem medizinischen und psychologischen Standpunkt aus untersuchen.
Wir müssen wissen, ob einzelne Menschen von einer angemessenen Medikation profitieren können.
Gibt es darüber hinaus eine medizinische oder psychologische Behandlung, mit deren Hilfe die Nervosität schließlich überwunden werden kann?
Von einem formalen psychologischen Standpunkt aus müssen wir eine Lehrprozedur entwickeln, die die Nervosität reduziert.
Nervosität ist sicherlich das Resultat einer mentalen Haltung, Reaktion und Wahrnehmung und ist deshalb der aktiven Kontrolle sehr zugänglich.
Klavierspieler, die Pop- oder Jazz-Musik spielen, scheinen im allgemeinen viel weniger nervös zu sein als diejenigen, die klassische Musik spielen.
Es gibt keinen Grund, warum wir nicht untersuchen sollten, warum das so ist, und einen Vorteil aus diesem Phänomen ziehen.


\subsubsection{Ursachen von und Mittel gegen Tinnitus}
\label{c1iv6e}

Struktur der Cochlea, hoch- und niederfrequenter Tinnitus.

Es gibt Anzeichen dafür, daß die moderate Einnahme von Aspirin den altersbedingten Gehörverlust verlangsamen kann.
Es gibt jedoch auch Anzeichen dafür, daß Aspirin einen Tinnitus unter bestimmten Bedingungen verschlimmern kann \textit{[auch kann es z.B. dem Magen und den Nieren schaden]}.
Es gibt anscheinend keinen Beweis dafür, daß Tinnitus nur durch das Altern verursacht wird; statt dessen gibt es zahlreiche Beweise, daß er durch Infektionen, Krankheiten und Mißbrauch des Gehörs verursacht wird.
Deshalb können in den meisten dieser Fälle die Ursachen und die Arten der Schäden direkt studiert werden.


\subsubsection{Was ist Musik?}
\label{c1iv6f}

Struktur der Cochlea und die Beziehung zu Tonleitern und Akkorden.
Parameter: Timing (Rhythmus), Tonhöhe, Muster (Sprache, Gefühle), Lautstärke, Geschwindigkeit.
Musikalische Informationsverarbeitung im Gehirn.


\subsubsection{In welchem Alter soll bzw. darf man mit dem Klavierspielen anfangen?}
\label{c1iv6g}

Wir brauchen medizinische, psychologische und soziologische Studien darüber, wie bzw. wann Kinder anfangen sollten.
Einzelne Sportorganisationen haben, zumindest informell, bereits diese Art der Forschung für den Bereich des Sports begonnen und Methoden zum Unterricht von Kindern bis herunter auf ein Alter von ungefähr zwei Jahren entwickelt.
In der Musik können wir beginnen, sobald ein Baby geboren ist, indem wir es die angemessenen Arten von Musik hören lassen.
Bei der Musik sind wir wahrscheinlich mehr an der Entwicklung des Gehirns als am Erwerb von Fertigkeiten der Bewegung interessiert.
Da wir erwarten, daß die Gehirnforschung in naher Zukunft rapide zunimmt, ist das ein günstiger Zeitpunkt, um einen Vorteil aus dieser Forschung zu ziehen und die Resultate für das Klavierspielenlernen zu benutzen.


\subsubsection{Die Zukunft des Klavierspielens}
\label{c1iv6h}

Zum Schluß ein Ausblick in die Zukunft.
Der Abschnitt mit den \hyperref[testimonials]{Leserkommentaren} ist ein deutlicher Hinweis darauf, daß unser neuer Ansatz zum Klavierüben praktisch jeden befähigen wird, das Klavierspielen zu seiner Zufriedenheit zu lernen.
Er wird sicherlich die Zahl der Klavierspieler erhöhen.
Deshalb sind die folgenden Fragen sehr wichtig:

\begin{enumerate} 
 \item Können wir die zu erwartende Zunahme an Klavierspielern berechnen?
 \item Was bedeutet dieser Anstieg für die wirtschaftliche Seite des Klavierspielens: Künstler, Lehrer, Techniker und Hersteller?
 \item Wenn die Popularität des Klaviers rapide zunimmt, was wird die hauptsächliche Motivation für so viele Menschen sein, das Klavierspielen zu lernen?
\end{enumerate}
Klavierlehrer werden zustimmen, daß 90\% der Klavierschüler das Klavierspielen in dem Sinn niemals richtig lernen, daß sie nicht in der Lage sind, zu ihrer Zufriedenheit zu spielen und es im Grunde aufgeben, vollendete Klavierspieler zu werden.
Da das ein wohlbekanntes Phänomen ist, hält es Kinder und ihre Eltern davon ab, sich dafür zu entscheiden, mit dem Klavierunterricht zu beginnen.
Da ernsthaftes Befassen mit dem Klavier das Verdienen des Lebensunterhalts wesentlich beeinträchtigt, hält der wirtschaftliche Faktor ebenfalls vom Einstieg in das Klavierspielen ab.
Es gibt viele weitere negative Faktoren, die die Beliebtheit des Klaviers begrenzen (Mangel an guten Lehrern, hohe Kosten für gute Klaviere und ihre Wartung, usw.),
die letztendlich fast alle mit der Tatsache zusammenhängen, daß es so schwierig war, Klavierspielen zu lernen.
Wahrscheinlich haben nur 10\% von denen, die versucht haben könnten, das Klavierspielen zu lernen, sich auch dazu entschieden.
Deshalb können wir ernsthaft erwarten, daß die Beliebtheit des Klaviers hundertfach gesteigert wird, wenn sich die Erwartungen dieses Buchs erfüllen.

Eine solche Steigerung würde bedeuten, daß ein großer Teil der Bevölkerung in den entwickelten Ländern das Klavierspielen lernt.
Da dieses ein bedeutender Teil ist, brauchen wir keine exakte Zahl, nehmen wir also eine vernünftige Zahl, z.B. 30\%.
Das würde mindestens eine zehnfache Zunahme der Zahl der Klavierlehrer erfordern.
Das wäre großartig für die Schüler, weil es heutzutage eines der großen Probleme ist, gute Lehrer zu finden.
In jeder Region gibt es zur Zeit nur ein paar Lehrer, und die Schüler haben wenig Auswahl.
Die Zahl der verkauften Klaviere würde ebenfalls zunehmen, wahrscheinlich um mehr als 300\%.
Viele Haushalte besitzen zwar bereits ein Klavier, viele davon sind jedoch nicht spielbar.
Da die meisten der neuen Klavierspieler auf einem fortgeschrittenen Niveau sein werden, wird die Zahl der notwendigen guten Flügel um einen noch größeren Prozentsatz steigen.

Indem sie dieses Buch als Grundlage für die Übungsmethoden benutzen, können sich die Klavierlehrer auf das konzentrieren, was sie am besten tun: lehren Musik zu machen.
Da Lehrer das die ganze Zeit getan haben, werden nur geringfügige neue Änderungen in der Art wie die Lehrer unterrichten notwendig sein.
Das einzige neue Element ist das Hinzufügen der Übungsmethoden, die innerhalb kurzer Zeit zu lernen sind.
Die größte Veränderung ist natürlich, daß Lehrer von dem alten langsamen Prozeß befreit werden, Technik zu lehren.
Es wird für Lehrer viel leichter sein, zu entscheiden was sie lehren, weil technische Schwierigkeiten ein viel geringeres Hindernis sein werden.
Innerhalb weniger Generationen von Lehrern und Schülern wird sich die Qualität der Lehrer dramatisch verbessern, was die Lernraten zukünftiger Schüler weiter beschleunigen wird.

Ist eine hundertfache Zunahme der Anzahl der Klavierspieler realistisch?
Was würden sie tun?
Sie können mit Sicherheit nicht alle Konzertpianisten und Klavierlehrer sein.
Die ganze Art, wie wir das Klavierspielen sehen, wird sich verändern.
Vor allem wird das Klavier bis dahin zu einem Standard-Zweitinstrument für alle Musiker werden, weil es so einfach sein wird es zu lernen und es überall Klaviere geben wird.
Die Freude am Klavierspielen wird für viele Belohnung genug sein.
Die Vielzahl der Musikliebhaber, die bisher nur Aufnahmen hören konnte, kann nun ihre eigene Musik spielen - eine viel befriedigendere Erfahrung.
Jeder, der ein vollendeter Klavierspieler geworden ist, wird Ihnen bestätigen, daß man, wenn man dieses Niveau erreicht hat, gar nicht anders kann, als anzufangen Musik zu komponieren.
Somit wird die Klavierrevolution auch eine Revolution in der Komposition in Gang setzen, und es wird eine große Nachfrage nach neuen Kompositionen bestehen, weil viele Klavierspieler nicht damit zufrieden sind, immer \enquote{dieselben alten Sachen} zu spielen.
Klavierspieler werden wegen der Entwicklung der Keyboards mit leistungsstarker Software Musik für jedes Instrument komponieren, und jeder Klavierspieler wird ein akustisches Klavier und ein elektronisches Keyboard oder ein Doppelinstrument besitzen (s.u.).
Die große Versorgung mit guten Keyboardspielern würde bedeuten, daß ganze Orchester aus Keyboardspielern bestehen werden.
Ein weiterer Grund, warum das Klavier allgemein beliebt werden würde, ist, daß es als eine Methode zur Steigerung des IQ von heranwachsenden Kindern benutzt wird.
Die Gehirnforschung wird sicherlich offenbaren, daß die Intelligenz durch die richtige Stimulation des Gehirns während der frühen Entwicklungsstadien verbessert werden kann.
Da es nur zwei Eingangskanäle zum Gehirn kleiner Kinder gibt, akustisch und visuell, und der akustische Teil zu Beginn weiter entwickelt ist als der visuelle, ist Musik das logischste Mittel, um das Gehirn während der frühen Entwicklung zu beeinflussen.

Wenn derart starke Kräfte am Werk sind, wird sich das Klavier selbst rasch weiterentwickeln.
Zunächst wird das elektronische Keyboard in zunehmendem Maß in den Klaviersektor vorstoßen.
Die Unzulänglichkeiten der elektronischen Klaviere werden weiter abnehmen, bis die elektronischen von den akustischen nicht mehr zu unterscheiden sind.
Unabhängig davon, welches Instrument benutzt wird, werden die technischen Erfordernisse dieselben sein.
Bis dahin werden die akustischen Klaviere viele Merkmale der elektronischen haben: Sie werden jederzeit gestimmt sein (statt 99\% der Zeit nicht richtig gestimmt zu sein, wie sie es heute sind), man wird die Temperaturen durch Umlegen eines Schalters ändern können, und sie werden MIDI-fähig sein.
Die akustischen Klaviere werden nie völlig verschwinden, weil die Kunst, Musik mit mechanischen Geräten zu machen, so faszinierend ist.
Um auf diesem neuen Gebiet erfolgreich zu sein, müssen Klavierhersteller viel flexibler und innovativer werden.

Die Klavierstimmer werden sich ebenfalls an diese Veränderungen anpassen müssen.
Alle Klaviere werden selbststimmend sein, so daß die Einkünfte aus dem Stimmen abnehmen werden.
Klaviere, die immer hundertprozentig gestimmt sind, müssen öfter intoniert werden, und
wie Hämmer gemacht sind und intoniert werden, wird sich ändern müssen.
Es ist nicht so, daß die heutigen Klaviere nicht genauso viel intoniert werden müßten, aber wenn die Saiten perfekt gestimmt sind, wird jede Abnutzung der Hämmer zu einem begrenzenden Faktor der Klangqualität.
Klavierstimmer werden schließlich in der Lage sein, Klaviere richtig einzustellen und zu intonieren, statt sie nur zu stimmen; sie können sich auf die Qualität des Klavierklangs konzentrieren, statt nur die Dissonanzen zu beseitigen.
Da die neue Generation der vollkommeneren Klavierspieler aural anspruchsvoller sein wird, werden sie nach einem besseren Klang verlangen.
Die stark gestiegene Zahl an Klavieren und ihr ständiger Gebrauch wird eine Vielzahl neuer Klaviertechniker erfordern, um sie einzustellen und zu reparieren.
Klavierstimmer werden auch mehr daran beteiligt sein, akustischen Klavieren elektronische Fähigkeiten (MIDI, usw.) hinzuzufügen und diese zu warten.
Deshalb wird sich das Geschäft der Klavierstimmer in Richtung Wartung und Ausbau der elektronischen Klaviere erweitern.
Somit werden die meisten Menschen entweder ein Hybridklavier oder sowohl ein akustisches als auch ein elektronisches Klavier besitzen.
 

\subsubsection{Die Zukunft des Unterrichts}
\label{c1iv6i}

Das Internet verändert offensichtlich die Natur der Ausbildung.
Eines meiner Ziele beim Schreiben dieses Buchs im WWW ist, Möglichkeiten für eine Steigerung der Kosteneffizienz der Ausbildung zu erforschen.
Wenn ich auf meine erste Ausbildung und meine Tage auf dem College zurückblicke, wundere ich mich über die Effizienz des Ausbildungsprozesses, den ich durchlaufen habe.
Die Aussicht auf eine viel größere Effizienz durch das Internet ist jedoch im Vergleich dazu schwindelerregend.
Meine bisherige Erfahrung war sehr lehrreich.
Hier sind einige der Vorteile internetbasierter Ausbildung:

\begin{enumerate}[label={\roman*.}] 
\item Kein Warten auf Schulbusse oder Laufen von Klassenzimmer zu Klassenzimmer mehr; keine Kosten mehr für Schulgebäude und zugehörige Einrichtungen.
\item Keine teuren Lehrbücher.
Alle Bücher sind auf dem neuesten Stand, im Gegensatz zu vielen Lehrbüchern, die in Universitäten benutzt werden, die über 10 Jahre alt sind.
Querverweise, Inhaltsverzeichnisse, Stichwortsuche, usw. können elektronisch vorgenommen werden.
Jedes Buch ist überall verfügbar, solange man einen Computer und eine Internetverbindung hat.
\item Viele Menschen können gemeinsam an einem Buch arbeiten, und die Aufgabe, es in andere Sprachen zu übersetzen, wird sehr effizient, insbesondere wenn die Übersetzer von einer guten Übersetzungssoftware unterstützt werden.
\item Fragen und Vorschläge können per E-Mail gesandt werden, und der Lehrer hat reichlich Zeit, sich eine detaillierte Antwort zu überlegen.
Dieser Austausch kann an jeden versandt werden, der sich dafür interessiert, und für den späteren Gebrauch gespeichert werden.
\item Der Lehrberuf wird sich drastisch verändern.
Auf der einen Seite wird es eine regere direkte Kommunikation per E-Mail, Videokonferenzen und Datenaustausch (wie z.B. Audiodateien vom Schüler zum Lehrer) geben,
aber auf der anderen Seite wird es weniger Gruppenkontakte geben, bei denen die Gruppe der Studenten in einem Klassenzimmer zusammenkommt.
Jeder Lehrer kann mit dem \enquote{Hauptlehrbuchzentrum} zusammenarbeiten, um Verbesserungen vorzuschlagen, die in das System eingebunden werden können.
Und die Schüler können auf viele verschiedene Lehrer zurückgreifen, sogar für das gleiche Thema.
\item Ein solches System würde bedeuten, daß ein Experte auf dem Gebiet nicht durch das Schreiben des besten Lehrbuchs der Welt reich werden kann.
Das ist aber so wie es sein sollte - Ausbildung muß für jeden zu den niedrigsten Kosten verfügbar sein.
Wenn die Ausbildungskosten sinken, müssen deshalb Institutionen, die auf die alte Art Geld verdienten, etwas ändern und sich an die neue Lage anpassen.
Würde das nicht die Experten davon abhalten, Lehrbücher zu schreiben?
Ja, aber man braucht nur einen solchen \enquote{Freiwilligen} für die ganze Welt; außerdem hat das Internet bereits genügend solcher freien Systeme wie z.B. Linux, Browser, usw. hervorgebracht, so daß dieser Trend nicht nur unumkehrbar sondern auch bewährt ist.
Mit anderen Worten: Der Wunsch, der Gesellschaft einen Dienst zu erweisen, wird zu einem großen Faktor beim Beitragen zur Ausbildung.
Für Projekte, die der Gesellschaft einen erheblichen Nutzen bringen, werden sich sicherlich Geldgeber finden (Regierungen, Philanthropen, Sponsoren).
\item Dieses neue Paradigma, etwas zur Gesellschaft beizutragen, kann sogar noch tiefgreifendere Veränderungen für die Gesellschaft bringen.
Eine Art, das heutige Geschäftsleben zu sehen, ist die der Wegelagerei.
Man verlangt soviel wie möglich, ungeachtet wie viel oder wie wenig Gutes das Produkt dem Käufer bringt.
In einem akkuraten Rechnungsparadigma sollte der Käufer immer den Gegenwert seines Geldes bekommen.
Das ist die einzige Situation, in der die Geschäftswelt auf lange Sicht überleben kann.
Das funktioniert in beide Richtungen; gut funktionierende Geschäfte sollten nicht einfach nur wegen übermäßigen Wettbewerbs Bankrott gehen.
In einer offenen Gesellschaft, in der alle relevanten Informationen sofort verfügbar sind, können wir ein Rechnungswesen haben, daß die Preise dem Service angemessen gestalten kann.
Die Philosophie ist hier, daß eine Gesellschaft, die aus Mitgliedern besteht, die sich verpflichtet fühlen, sich gegenseitig zu helfen, besser funktionieren wird als eine, die aus Räubern besteht, die sich gegenseitig bestehlen.
Insbesondere sollte in der Zukunft alle Grundausbildung im Prinzip kostenlos sein.
Das bedeutet nicht, daß die Lehrer ihre Arbeit verlieren, weil Lehrer die Lernrate in hohem Maß steigern können und entsprechend bezahlt werden sollten.
\end{enumerate}
Anhand der obigen Überlegungen ist klar, daß ein freier Informationsaustausch das Feld der Ausbildung (so wie jedes andere auch) verwandeln wird.
Dieses Buch ist einer der Versuche, den vollen Vorteil aus diesen neuen Möglichkeiten zu ziehen.





<!-- c21.html -->

\chapter{Stimmen des Klaviers}
\label{c2}

\subsection{Einleitung}
\label{c2_1} 

\textbf{\textit{[Achtung: Beim Stimmen, Intonieren und anderen Wartungsarbeiten kann ein Klavier durch Mangel an Vorsicht, unsachgemäßes Vorgehen, usw. beschädigt werden!
Wenn jemand zum ersten Mal die Mechanik ausbaut und daran arbeitet, stehen die Chancen ungefähr 1:1, \underline{daß}  er etwas beschädigt.
Sie sollten sich deshalb eingehend mit dem Thema beschäftigen, bevor (!) Sie Hand an Ihr eigenes oder ein fremdes Klavier legen.
Lassen Sie sich ggf. von jemandem beraten, der sich mit der Materie auskennt, und beschaffen Sie sich gute weiterführende Literatur zu dem Thema, wie z.B. das von Chuan C. Chang angeführte Buch \enquote{Piano Servicing, Tuning, and Rebuilding} von Arthur Reblitz.]}}

\textbf{Dieses Kapitel ist für diejenigen bestimmt, die ihr Klavier noch nie selbst gestimmt haben und sehen möchten, ob sie der Aufgabe gewachsen sind.}
Das Buch \textit{Piano Servicing, Tuning, and Rebuilding} von Arthur Reblitz ist dabei ein sehr hilfreiches Nachschlagewerk.\footnote{Hinweise für deutsche Bücher gleicher Qualität nehme ich gerne hier auf.}
Der schwierigste Teil beim Lernen des Stimmens ist das Anfangen.
Wer in der glücklichen Lage ist, jemanden zu haben, der ihn unterrichtet, ist natürlich am besten dran.
Unglücklicherweise sind Lehrer für das Klavierstimmen nicht ohne weiteres verfügbar.
Probieren Sie die Vorschläge in diesem Kapitel aus, und sehen Sie, wie weit sie kommen.
Nachdem Sie sich darüber im klaren sind, was Ihnen Probleme bereitet, können Sie mit Ihrem Stimmer über 30-minütige Lektionen mit einer vereinbarten Vergütung verhandeln oder ihn bitten zu erklären, was er tut, wenn er stimmt.
Seien Sie darauf bedacht, Ihrem Stimmer nicht zuviel aufzubürden; Stimmen und Unterrichten kann mehr als viermal so lang dauern wie nur zu stimmen.
Seien Sie auch vorgewarnt, daß Klavierstimmer keine ausgebildeten Lehrer sind, und einige von ihnen mögen unberechtigte Befürchtungen hegen, daß sie einen Kunden verlieren könnten.
Diese Befürchtungen sind unbegründet, weil die tatsächliche Zahl der Menschen, die professionelle Stimmer erfolgreich ersetzt haben, vernachlässigbar klein ist.
Am wahrscheinlichsten bekommen Sie am Ende ein besseres Verständnis dafür, was es bedeutet, ein Klavier zu stimmen.

Für Klavierspieler bietet das Vertrautwerden mit der Kunst des Stimmens eine Ausbildung, die für ihre Fähigkeit Musik zu erzeugen und ihre Instrumente zu warten sehr wichtig ist.
Es versetzt sie auch in die Lage, vernünftig mit ihren Stimmern zu kommunizieren.
So kannte z.B. die Mehrzahl der Klavierlehrer, denen ich die Frage stellte, noch nicht einmal den Unterschied zwischen \hyperref[et1]{gleichschwebender Temperatur} und historischen Stimmungen.
Der Hauptgrund, warum die meisten Menschen versuchen das Stimmen zu lernen, ist aus Neugier - für die meisten ist das Klavierstimmen ein rätselhaftes Geheimnis.
Sind sie erst einmal über die Vorteile eines gestimmten (gewarteten) Klaviers unterrichtet, ist es wahrscheinlicher, daß sie ihren Stimmer regelmäßig rufen.
Klavierstimmer können bestimmte Töne, die vom Klavier kommen, hören, die die meisten Menschen, sogar Pianisten, nicht wahrnehmen.
Diejenigen, die das Stimmen üben, werden für die Töne von verstimmten Klavieren sensibilisiert.
Unter der Annahme, daß Sie die Zeit haben, mindestens einmal alle ein oder zwei Monate für mehrere Stunden zu üben, wird es wahrscheinlich ungefähr ein Jahr dauern, bis Sie anfangen mit dem Stimmen zurechtzukommen.

Lassen Sie mich hier ein wenig abschweifen, um zu besprechen, wie wichtig es unter dem Gesichtspunkt, von dem Stimmer einen Gegenwert für Ihr Geld zu bekommen, so daß Ihr Klavier richtig gewartet werden kann, ist, die Lage des Klavierstimmers zu verstehen und richtig mit ihm zu kommunizieren.
Diese Überlegungen haben sowohl eine direkte Auswirkung auf Ihre Fähigkeit, sich Klaviertechnik anzueignen als auch auf Ihre Entscheidungen darüber, was oder wie Sie bei einem Auftritt vorspielen, wenn Sie ein bestimmtes Klavier zur Verfügung haben.
Eine der verbreitetsten Schwierigkeiten, die ich z.B. bei Schülern festgestellt habe, ist ihre Unfähigkeit, pianissimo zu spielen.
Aus meinem Verständnis des Klavierstimmens heraus gibt es dafür eine einfache Erklärung - die meisten Klaviere dieser Schüler werden zu wenig gewartet.
Die Hämmer sind zu abgenutzt bzw. verdichtet und die Mechanik so sehr verstellt, daß pianissimo zu spielen unmöglich ist.
Diese Schüler werden nicht einmal in der Lage sein, pianissimo zu üben!
Das gilt auch für den musikalischen Ausdruck und die Tonkontrolle.
Diese zu wenig gewarteten Klaviere sind wahrscheinlich eine der Ursachen für die Ansicht, daß Klavierüben eine Qual für die Ohren ist, aber das sollte es nicht sein.

Ein weiterer Faktor ist, daß man sich im allgemeinen das Klavier nicht aussuchen kann, wenn man gebeten wird, etwas vorzuspielen.
Man kann auf alles treffen: von einem wundervollen Konzertflügel über ein Kleinklavier bis zu (Schock!) einem billigen Stutzflügel, der völlig vernachlässigt wurde, seit er vor 40 Jahren gekauft wurde.
Ihr Verständnis dafür, was man mit jedem dieser Klaviere tun kann bzw. nicht tun kann, sollte der erste Punkt bei der Entscheidung sein, was und wie man spielt.

Wenn Sie erst einmal angefangen haben, das Stimmen zu üben, werden Sie schnell verstehen, warum es einem genauen und qualitativen Stimmen nicht förderlich ist, wenn jemand dabei Staub saugt, Kinder herumspringen, der Fernseher oder die Stereoanlage plärrt oder in der Küche die Töpfe klappern, und warum ein Stimmen auf die Schnelle für 70 Euro kein Schnäppchen ist im Vergleich zu einem Stimmen für 150 Euro, bei dem der Stimmer die Hämmer neu formt und nadelt.
Wenn man aber die Besitzer fragt, was der Stimmer an ihrem Klavier getan hat, haben sie im allgemeinen keine Vorstellung davon.
Eine Beschwerde, die ich oft von Besitzern höre, ist, daß das Klavier nach dem Stimmen tot oder schrecklich klingen würde.
Das geschieht oft, wenn der Besitzer keinen festen Bezugspunkt hat, von dem aus er den Klavierklang beurteilen kann - das Urteil basiert darauf, ob der Besitzer den Klang mag oder nicht.
Solche Wahrnehmungen sind allzuoft durch die Vorgeschichte des Besitzers falsch beeinflußt.
Der Besitzer kann sich tatsächlich an den Klang eines verstimmten Klaviers mit verdichteten Hämmern gewöhnen, so daß wenn der Stimmer den Klang wiederherstellt, der Besitzer diesen nicht mag, weil er sich nun zu sehr von dem gewohnten Klang oder Gefühl unterscheidet.
Der Stimmer könnte sicherlich Schuld daran haben; der Besitzer braucht jedoch ein minimales Wissen über technische Details des Stimmens, um solch ein Urteil hieb- und stichfest zu machen.
Der Nutzen eines Verständnisses für das Stimmen und der richtigen Wartung des Klaviers wird offensichtlich von der Allgemeinheit unterschätzt.
Vielleicht ist das wichtigste Ziel dieses Kapitels, dieses Bewußtsein zu vergrößern.

\textbf{Klavierstimmen erfordert - im Gegensatz zum \hyperref[c1iii12]{absoluten Gehör} - keine guten Ohren, weil das ganze Stimmen durch den Vergleich mit einer Referenz ausgeführt wird, bei dem Schwebungen benutzt werden und man mit der Bezugsfrequenz einer Stimmgabel beginnt.}
Tatsächlich kann die Fähigkeit des absoluten Gehörs bei einigen Menschen mit dem Stimmen in Konflikt geraten.
Deshalb ist die \enquote{einzige} notwendige Hörfertigkeit die Fähigkeit, die verschiedenen Schwebungen zu hören und zwischen ihnen zu unterscheiden, wenn zwei Saiten angeschlagen werden.
Diese Fähigkeit entwickelt sich durch Übung und ist nicht mit dem Wissen über Musiktheorie oder mit Musikalität verknüpft.
Größere Flügel sind leichter zu Stimmen als \enquote{\hyperref[upright]{Aufrechte}}; die meisten Stutzflügel sind jedoch schwieriger zu stimmen als gute \enquote{Aufrechte}.
Obwohl man seine Übungen logischerweise an einem qualitativ schlechteren Klavier beginnen sollte, wird dieses deshalb schwieriger zu stimmen sein.
 


<!-- c22.html -->

\subsection{Chromatische Tonleiter und Temperaturen}
\label{c2_2}

\subsubsection{Einleitung}
\label{c2_2a} 

Die meisten von uns sind mit der chromatischen Tonleiter einigermaßen vertraut und wissen, daß sie temperiert sein muß, aber was sind die präzisen Definitionen der beiden Begriffe?
\textbf{Warum ist die chromatische Tonleiter so besonders, und warum ist das Temperieren notwendig?}
Wir erforschen zunächst die mathematische Grundlage der chromatischen Tonleiter und des Temperierens, weil der mathematische Ansatz die knappste, klarste und präziseste Behandlung ist.
Wir besprechen dann die historischen und musikalischen Gesichtspunkte, damit wir die relativen Vorzüge der verschiedenen Stimmungen besser verstehen.
Ein grundlegendes mathematisches Fundament dieser Konzepte ist entscheidend für ein gutes Verständnis dafür, wie Klaviere gestimmt werden.
Informationen über das Stimmen finden Sie bei White, Howell, Fischer, Jorgensen oder Reblitz (s. \hyperref[reference]{Quellenverzeichnis}).
 

\label{c2_2b}

\subsubsection{Mathematische Behandlung}
\label{c2_2_math} 

In Tabelle 2.2a sind drei Oktaven aufgeführt.
Die schwarzen Tasten des Klaviers werden mit Kreuzen dargestellt, z.B. steht das \# rechts von C für ein C\#, und ist nur bei der höchsten Oktave ausgewiesen.
\textbf{Jede der aufeinanderfolgenden Frequenzänderungen in der chromatischen Tonleiter wird ein Halbton genannt, und eine Oktave besteht aus 12 Halbtönen.}
Die Hauptintervalle und die Ganzzahlen, die die Frequenzverhältnisse dieser Intervalle repräsentieren, sind jeweils oberhalb und unterhalb der chromatischen Tonleiter aufgeführt.
Das Wort Intervall wird hierbei im Sinne von zwei Noten benutzt, deren Frequenzverhältnis der Quotient kleiner ganzer Zahlen ist.
Außer für Vielfache dieser Grundintervalle erzeugen Ganzzahlen, die größer als ungefähr 10 sind, Intervalle, die für das Ohr nicht einfach zu erkennen sind.
Gemäß Tabelle 2.2a ist das grundlegendste Intervall die Oktave, bei der die Frequenz der höheren Note das Doppelte der Frequenz der tieferen Note ist.
Das Intervall zwischen C und G ist eine Quinte, und die Frequenzen von C und G stehen in einem Verhältnis von 2 zu 3 zueinander.
Die große Terz hat vier Halbtöne, und die kleine Terz hat drei.
\textbf{Die Zahl, die jedem Intervall zugeordnet ist, z.B. vier in der Quarte, ist bei der C-Dur-Tonleiter die Zahl der weißen Tasten inkl. der beiden Tasten am Anfang und Ende des Intervalls und hat keine weitere mathematische Bedeutung.}
Beachten Sie, daß das Wort \enquote{Tonleiter} bzw. \enquote{Skala} in \enquote{chromatische Tonleiter}, \enquote{C-Dur-Tonleiter} und \enquote{logarithmische oder Frequenz-Skala} (s.u.) eine völlig unterschiedliche Bedeutung hat; die zweite ist eine Untermenge der ersten.

<table border>
 <tr>
  <td bgcolor=\enquote{\#E0E0E0}>\textbf{Oktave}</td>
  <td bgcolor=\enquote{\#E0E0E0}>\textbf{Quinte}</td>
  <td bgcolor=\enquote{\#E0E0E0}>\textbf{Quarte}</td>
  <td bgcolor=\enquote{\#E0E0E0}>\textbf{Gr. Terz}</td>
  <td bgcolor=\enquote{\#E0E0E0}>\textbf{Kl. Terz}</td>
  <td bgcolor=\enquote{\#E0E0E0}> </td>
  <td bgcolor=\enquote{\#E0E0E0}>  \\ 
 CDEFGAH & C D E F & G A H & C \# D \# & E F \# & G \# A \# H & C \\ 
 1 & 2 & 3 & 4 & 5 & 6 & 8 \\ 
</table>
<h5>(Tabelle 2.2a: Frequenzverhältnisse der Intervalle in der chromatischen Tonleiter)</h5>
Wir können oben sehen, daß eine Quarte und eine Quinte sich zu einer Oktave \enquote{aufaddieren} und eine große Terz und eine kleine Terz sich zu einer Quinte \enquote{aufaddieren}.
Beachten Sie, daß diese Addition im logarithmischen Raum erfolgt, wie unten erklärt wird.
Die fehlende Ganzzahl 7 wird ebenfalls unten erklärt.
 

\label{et1}

\textbf{Die \hyperref[et]{gleichschwebend temperierte} (ET) chromatische Tonleiter besteht aus \enquote{gleichen} Halbtonschritten für jede nachfolgende Note.}
Sie sind in dem Sinne gleich, daß das Verhältnis der Frequenzen von zwei aufeinanderfolgenden Noten immer das gleiche ist.
Diese Eigenschaft stellt sicher, daß jede Note (außer in der Tonhöhe) mit allen anderen identisch ist.
Diese Gleichförmigkeit der Noten gestattet es dem Komponisten oder Künstler, jede Tonhöhe und jede Tonart zu benutzen, ohne auf große Dissonanzen zu treffen, wie unten weiter erklärt wird.
In einer Oktave einer ET-Tonleiter gibt es 12 gleiche Halbtöne und jede Oktave ist ein genauer Faktor von 2 im Frequenzverhältnis.
Deshalb beträgt die Frequenzänderung für jeden Halbton:


\label{gleich21}
<table border>&\#9;Halbtonschritt<sup>12</sup>&\#9;= 2&\#9;oder &\#9;Halbtonschritt&\#9;= 2<sup>1/12</sup> pprox 1,05946 \\ </table>
<h5>(Gleichung 2.1)</h5>

Gleichung 2.1 definiert die ET chromatische Tonleiter und erlaubt die Berechnung der Frequenzverhältnisse von \enquote{Intervallen} in dieser Tonleiter.
Wie verhalten sich die \enquote{Intervalle} bei ET zu den Frequenzverhältnissen der reinen Intervalle?
\textbf{Der Vergleich ist in Tabelle 2.2b aufgeführt und zeigt, daß die Intervalle der ET-Tonleiter den reinen Intervallen sehr nah kommen.}

<table border cellpadding=3>
 <tr>
  <td bgcolor=\enquote{\#E0E0E0}>\textbf{Intervall}</td>
  <td bgcolor=\enquote{\#E0E0E0}>\textbf{Freq.-Verh.}</td>
  <td bgcolor=\enquote{\#E0E0E0}>\textbf{ET-Tonleiter}</td>
  <td bgcolor=\enquote{\#E0E0E0}>\textbf{Differenz} \\ 
 <tr>
  <td valign=\enquote{bottom}>Kleine Terz</td>
  <td valign=\enquote{bottom}>6/5 = 1,2000 & Halbtonschritt<sup>3</sup> pprox 1,1892</td>
  <td valign=\enquote{bottom}>+0,0108 \\ 
 <tr>
  <td valign=\enquote{bottom}>Große Terz</td>
  <td valign=\enquote{bottom}>5/4 = 1,2500 & Halbtonschritt<sup>4</sup> pprox 1,2599</td>
  <td valign=\enquote{bottom}>-0,0099 \\ 
 <tr>
  <td valign=\enquote{bottom}>Quarte</td>
  <td valign=\enquote{bottom}>4/3 pprox 1,3333 & Halbtonschritt<sup>5</sup> pprox 1,3348</td>
  <td valign=\enquote{bottom}>-0,0015 \\ 
 <tr>
  <td valign=\enquote{bottom}>Quinte</td>
  <td valign=\enquote{bottom}>3/2 = 1,5000 & Halbtonschritt<sup>7</sup> pprox 1,4983</td>
  <td valign=\enquote{bottom}>+0,0017 \\ 
 <tr>
  <td valign=\enquote{bottom}>Oktave</td>
  <td valign=\enquote{bottom}>2/1 = 2,0000 & Halbtonschritt<sup>12</sup> = 2,0000</td>
  <td valign=\enquote{bottom}>0,0000 \\ 
</table>
<h5>(Tabelle 2.2b: Vergleich der reinen Intervalle mit der gleichschwebend temperierten Tonleiter)</h5>
\textbf{Die Abweichung ist bei den Terzen am größten, mehr als fünfmal so groß wie die Abweichung bei den anderen Intervallen aber trotzdem nur ungefähr 1\%.}
Nichtsdestoweniger sind diese Abweichungen leicht zu hören, und einige Klavierliebhaber haben sie großmütig als \enquote{die rollenden Terzen} tituliert, während sie in Wahrheit inakzeptable Dissonanzen sind.
Es ist ein Mangel, mit dem wir leben müssen, wenn wir die ET-Tonleiter akzeptieren wollen.
Die Abweichungen bei den Quarten und Quinten erzeugen um das mittlere C Schwebungen von ungefähr 1 Hz, was bei den meisten Musikstücken kaum zu hören ist; diese Schwebungsfrequenz verdoppelt sich jedoch mit jeder höheren Oktave.

Wäre die Ganzzahl 7 in Tabelle 2.2a aufgenommen worden, hätte sie ein Intervall mit dem Verhältnis 7/6 repräsentiert und würde dem Quadrat eines Halbtonschritts entsprechen.
Die Abweichung zwischen diesen beiden Zahlen beträgt mehr als 4\% und ist zu groß, um ein musikalisch akzeptables Intervall zu bilden; sie wurde deshalb nicht in Tabelle 2.2a aufgeführt.
Es ist nur ein mathematischer Zufall, daß die aus 12 Tönen bestehende chromatische Tonleiter so viele Verhältnisse nahe an den reinen Intervallen erzeugt.
\textbf{Von den 8 kleinsten Ganzzahlen führt nur die Zahl 7 zu einem völlig inakzeptablen Intervall.
Die chromatische Tonleiter basiert auf einem glücklichen mathematischen Zufall der Natur!
Sie wird durch die kleinste Anzahl von Noten gebildet, die die maximale Anzahl von Intervallen ergeben.}
Kein Wunder, daß frühe Zivilisationen glaubten, es läge etwas mystisches in dieser Tonleiter.
Die Zahl der Noten in einer Oktave zu erhöhen, führt zu keiner großen Verbesserung der Intervalle, bis die Zahlen sehr groß werden, was diesen Ansatz für die meisten Musikinstrumente undurchführbar macht.

Beachten Sie, daß die Frequenzverhältnisse der Quarten und Quinten sich nicht zu dem der Oktave aufaddieren (1,5000 + 1,3333 = 2,8333 statt 2,0000).
Sie addieren sich allerdings im logarithmischen Maßstab, weil (3/2)x(4/3) = 2.
Im logarithmischen Raum wird die Multiplikation zur Addition.
Warum ist das so wichtig?
Weil die Geometrie der Cochlea (Ohrschnecke) anscheinend eine logarithmische Komponente hat.
Akustische Frequenzen auf einer logarithmischen Skala wahrzunehmen erreicht zwei Dinge: man kann bei gegebener Größe der Cochlea einen breiteren Frequenzbereich hören, und das Analysieren der Frequenzverhältnisse wird einfach, weil man, anstatt die zwei Frequenzen zu dividieren oder multiplizieren, nur ihre Logarithmen subtrahieren oder addieren muß.
Wenn z.B. das C3 von der Cochlea an einer Stelle erkannt wird und das C4 an einer anderen Stelle, die 2 mm weiter aufwärts liegt, dann wird das C5 an einer Stelle erkannt, die 4 mm aufwärts liegt, genau wie bei einem Rechenschieber.
Um zu zeigen, wie nützlich das ist: bei einem gegebenen F5 weiß das Gehirn, daß das F4 2 mm weiter unten zu finden ist!
Deshalb sind Intervalle (erinnern Sie sich daran, daß Intervalle Divisionen von Frequenzen sind) von einer logarithmisch aufgebauten Cochlea besonders einfach zu analysieren.
Wenn wir Intervalle spielen, üben wir mathematische Operationen im logarithmischen Raum auf einem mechanischen Computer genannt Klavier aus, ähnlich wie es früher mit dem Rechenschieber getan wurde.
\textbf{Deshalb hat die logarithmische Natur der chromatischen Tonleiter viel mehr Konsequenzen, als nur einen größeren hörbaren Frequenzbereich zur Verfügung zu stellen.}
Die logarithmische Skala stellt sicher, daß die beiden Noten jedes Intervalls, unabhängig davon wo man sich auf dem Klavier befindet, immer denselben Abstand voneinander haben.
Durch die Übernahme einer logarithmischen Skala wird die Tastatur mechanisch auf das menschliche Ohr abgebildet!
Das ist wahrscheinlich der Grund, warum Harmonien für das Ohr so angenehm sind - Harmonien werden durch das menschliche Gehör am leichtesten entschlüsselt und erinnert.

Angenommen, wir würden \hyperref[gleich21]{Gleichung 2.1} nicht kennen; können wir die ET chromatische Tonleiter aus den Beziehungen der Intervalle erzeugen?
Wenn die Antwort ja ist, kann ein Klavierstimmer ein Klavier stimmen, ohne Berechnungen durchführen zu müssen.
Diese Intervallbeziehungen, so stellt sich heraus, bestimmen die Frequenzen aller Noten der zwölfnotigen chromatischen Tonleiter.
Eine Temperatur ist eine Gruppe von Intervallbeziehungen, die diese Bestimmung ermöglicht.
Von einem musikalischen Standpunkt aus gibt es keine chromatische Tonleiter, die besser wäre als alle anderen, obwohl ET die einmalige Eigenschaft hat, daß sie ein freies Transponieren erlaubt.
Unnötig zu sagen, daß \textbf{ET nicht die einzige musikalisch nützliche Temperatur ist, und wir werden unten weitere Temperaturen besprechen.}
Die Temperatur ist keine Option, sondern eine Notwendigkeit; wir \textit{müssen} eine Temperatur wählen, um diese mathematischen Schwierigkeiten zu überwinden.
\textbf{Kein musikalisches Instrument, das auf der chromatischen Tonleiter basiert, ist völlig frei von Temperatur.}
So müssen z.B. die Löcher eines Blasinstruments und die Bünde der Gitarre für eine bestimmte temperierte Tonleiter in einem entsprechenden Abstand angeordnet sein.
Die Geige ist ein ziemlich cleveres Instrument, weil es alle Probleme mit der Temperatur dadurch vermeidet, daß die leeren Saiten zueinander einen Abstand von einer Quinte haben.
Wenn man die A(440)-Saite richtig stimmt und alle anderen Saiten dazu in Quinten, dann sind die anderen rein und nicht temperiert.
Man kann Probleme mit der Temperatur auch vermeiden, indem man alle Noten mit Ausnahme des A(440) greift.
Außerdem ist das Vibrato größer als die Korrekturen der Temperatur, was die Differenzen der Temperatur unhörbar werden läßt.

\textbf{Die Erfordernis für das Temperieren entsteht, weil eine chromatische Tonleiter, die auf eine Tonart gestimmt ist (z.B. C-Dur mit reinen Intervallen), in anderen Tonarten keine akzeptablen Intervalle erzeugt.}
Wenn man eine Komposition in C-Dur, die viele reine Intervalle enthält, transponiert, kann das zu schrecklichen Dissonanzen führen.
Es gibt ein noch grundlegenderes Problem.
Reine Intervalle in einer Tonart erzeugen auch Dissonanzen in anderen Tonarten, die im selben Musikstück benutzt werden.
Die Temperierschemata wurden deshalb dafür entwickelt, diese Dissonanzen zu minimieren, indem man die Verstimmung der reinen Intervalle bei den wichtigsten Intervallen minimierte und die größten Dissonanzen in die weniger benutzten Intervalle verschob.
Die zum schlimmsten Intervall gehörende Dissonanz wurde \enquote{Wolfsquinte} genannt.

Das Hauptproblem ist natürlich die Intervallreinheit; die obige Diskussion macht klar, daß egal was man tut, irgendwo eine Dissonanz auftreten wird.
\textbf{Es mag für manche ein Schock sein, daß das Klavier im Grunde ein unvollkommenes Instrument ist!}
Wir werden deshalb in jeder Tonleiter immer mit einigen Kompromissen bei den Intervallen leben müssen.

Der Name \enquote{chromatische Tonleiter} wird im allgemeinen auf jede zwölfnotige Tonleiter mit beliebiger Temperatur angewandt.
Natürlich erlaubt die chromatische Tonleiter des Klaviers nicht die Benutzung von Frequenzen zwischen den Noten (wie man das bei der Geige tun kann), so daß es eine unendliche Zahl fehlender Noten gibt.
In diesem Sinne ist die chromatische Tonleiter unvollständig.
Nichtsdestoweniger ist die zwölfnotige Tonleiter für die meisten musikalischen Anwendungen genügend vollständig.
Die Situation ist einer digitalen Fotografie analog.
Wenn die Auflösung ausreichend ist, kann man den Unterschied zwischen einem digitalen Foto und einem analogen Foto mit viel höherer Informationsdichte nicht sehen.
Ähnlich \textbf{hat die zwölfnotige Tonleiter offenbar für eine genügend große Anzahl musikalischer Anwendungen eine ausreichende Auflösung in der Tonhöhe.}
Diese zwölfnotige Tonleiter ist für ein bestimmtes Instrument oder musikalisches Notationssystem mit begrenzter Zahl zur Verfügung stehender Noten ein guter Kompromiß zwischen \enquote{mehr Noten je Oktave für eine größere Vollständigkeit haben} und \enquote{genug Frequenzbereich haben, um den Bereich des menschlichen Gehörs abzudecken}.

Es gibt eine fruchtbare Debatte darüber, welche Temperatur musikalisch gesehen am besten ist.
ET war von der frühesten Geschichte des Temperierens an bekannt.
Es hat definitiv Vorteile, auf eine Temperatur zu standardisieren, aber das ist hinsichtlich der Unterschiedlichkeit der Meinungen über Musik und der Tatsache, daß viel der zur Zeit existierenden Musik mit dem Gedanken an eine bestimmte Temperatur geschrieben wurde, wahrscheinlich nicht möglich oder sogar nicht wünschenswert.
Deshalb werden wir nun die verschiedenen Temperaturen erforschen.
 

\label{c2_2c}
\subsubsection{Temperatur und Musik}
\label{c2_2_temp} 

Der obige \hyperref[c2_2b]{mathematische Ansatz} ist nicht die Art und Weise, in der die chromatische Tonleiter entwickelt wurde.
Musiker begannen zunächst mit Intervallen und versuchten, eine Tonleiter mit einer minimalen Anzahl Noten zu finden, die diese Intervalle erzeugen würde.
Die Erfordernis einer minimalen Anzahl von Noten ist offensichtlich wünschenswert, weil diese die Anzahl der Tasten, Saiten, Löcher, usw. bestimmt, die für die Konstruktion eines Musikinstruments notwendig sind.
Intervalle sind notwendig, denn wenn man mehr als eine Note gleichzeitig spielen möchte, werden diese Noten für das Ohr unangenehme Dissonanzen erzeugen, außer wenn sie harmonische Intervalle bilden.
Der Grund, warum Dissonanzen so unangenehm für das Ohr sind, hat eventuell etwas mit der Schwierigkeit zu tun, mit dem Gehirn dissonante Informationen zu verarbeiten.
Es ist sicherlich hinsichtlich des Gedächtnisses und Verständnisses leichter, sich mit harmonischen Intervallen als mit Dissonanzen zu befassen, wobei es bei einigen davon für die meisten Menschen fast unmöglich ist, herauszufinden, ob zwei dissonante Noten gleichzeitig gespielt werden.
Deshalb wird es, wenn das Gehirn mit der Aufgabe komplexe Dissonanzen zu erkennen überlastet ist, unmöglich zu entspannen und die Musik zu genießen oder die musikalische Idee zu verfolgen.
Sicherlich muß jede Tonleiter gute Intervalle erzeugen, wenn wir fortgeschrittene, komplexe Musik komponieren sollen, die mehr als eine Note gleichzeitig erfordert.

\textbf{Die optimale Anzahl Noten in einer Tonleiter stellte sich als 12 heraus.
Leider gibt es keine zwölfnotige Tonleiter, die überall reine Intervalle erzeugt.
Musik würde besser klingen, wenn eine Tonleiter, die nur aus reinen Intervallen besteht, gefunden werden könnte.}
Viele solcher Versuche wurden bereits unternommen, hauptsächlich durch das Erhöhen der Notenanzahl je Oktave und besonders bei Gitarren und Orgeln, aber keine dieser Tonleitern hat eine Akzeptanz erreicht\footnote{zumindest in der \enquote{westlichen} Musik, in der Musik anderer Kulturen gibt es durchaus Tonleitern mit mehr als 20 Tönen je Oktave}.
Es ist relativ leicht, die Zahl der Noten mit einem gitarrenähnlichen Instrument zu erhöhen, weil man nur Saiten und Bünde hinzufügen muß.
Die neuesten Verfahren, die heute entwickelt werden, beziehen computergenerierte Tonleitern mit ein, bei denen der Computer die Frequenzen bei jeder Transposition justiert; dieses Verfahren wird adaptives Stimmen (Sethares) genannt.

\textbf{Das grundlegendste Konzept, das benötigt wird, um Temperaturen zu verstehen, ist das Konzept des Quintenzirkels.}
Nehmen Sie, um einen Quintenzirkel zu beschreiben, eine beliebige Oktave.
Beginnen Sie mit der tiefsten Note, und gehen Sie in Quinten aufwärts.
Nach zwei Quinten kommen Sie über diese Oktave hinaus.
Wenn das geschieht, gehen Sie eine Oktave nach unten, so daß Sie weiter in Quinten aufwärts gehen können und immer noch in der ursprünglichen Oktave bleiben.
Machen Sie das für zwölf Quinten, und Sie werden bei der höchsten Note der Oktave ankommen!
D.h. wenn Sie mit C4 anfangen, kommen Sie am Ende zu C5, und deshalb wird es ein Zirkel\footnote{lat. circulus = Kreis} genannt.
Nicht nur das, sondern jede Note, auf die Sie treffen, wenn Sie die Quinten spielen, ist eine andere Note.
Das bedeutet, daß der Quintenzirkel jede Note trifft, und das nur einmal, was eine nützliche Schlüsseleigenschaft für das Stimmen der Tonleiter ist und dafür, sie mathematisch zu untersuchen.
 

\label{c2_2_hist}

\textbf{Historische Entwicklungen sind ein zentrales Thema der Diskussionen über Temperaturen, weil die Musik aus einer Zeit mit der Temperatur aus dieser Zeit verbunden ist.
Pythagoras wird zugeschrieben, daß er ungefähr 550 v. Chr. unter Benutzung des Quintenzirkels die \enquote{pythagoreische Stimmung} erfunden hat, bei der die chromatische Tonleiter durch das Stimmen mit reinen Quinten erzeugt wird.}
Die zwölf reinen Quinten im Quintenzirkel bilden keinen exakten Faktor 2.
Deshalb ist die letzte Note, die man bekommt, nicht genau die Oktavnote, sondern ist in der Frequenz um den Wert zu hoch, den man \enquote{pythagoreisches Komma} nennt, d.h. ungefähr 23 Cent (ein Cent ist ein Hundertstel eines Halbtonschritts).
Da eine Quarte und eine Quinte eine Oktave bilden, resultiert die pythagoreische Stimmung in einer Tonleiter mit reinen Quarten und Quinten, wobei man allerdings am Ende eine sehr schlechte Dissonanz bekommt.
Es stellt sich heraus, daß mit reinen Quinten zu stimmen, zu unreinen Terzen führt.
Das ist ein weiterer Nachteil der pythagoreischen Stimmung.
Wenn nun jemand stimmen sollte, indem er jede Quinte um 23/12 Cent zusammenzieht, dann hätte er am Ende genau eine Oktave, und das ist eine Möglichkeit, eine \hyperref[et1]{ET-Tonleiter} zu stimmen.
Wir werden im Abschnitt über das Stimmen tatsächlich \hyperref[c2_6_et]{solch eine Methode benutzen}.
Die ET-Tonleiter war schon ca. 100 Jahre nach der Erfindung der pythagoreischen Stimmung bekannt.
Deshalb ist die ET keine \enquote{moderne Temperatur}.

Alle neueren Temperaturen, die auf die Einführung der pythagoreischen Stimmung folgten, waren Bemühungen, diese zu verbessern.
Die erste Methode war, das pythagoreische Komma zu halbieren und auf die letzten beiden Quinten zu verteilen.
\textbf{Eine wichtige Entwicklung war die mitteltönige Stimmung, bei der die Terzen statt der Quinten rein gemacht wurden.}
Musikalisch spielen Terzen eine bedeutendere Rolle als die Quinten, so daß die mitteltönige Stimmung sinnvoll war, besonders in einer Zeit, in der die Musik mehr Gebrauch von den Terzen machte.
Unglücklicherweise hat die mitteltönige Stimmung eine Wolfsquinte, die schlimmer als die der pythagoreischen Stimmung ist.
 

\label{c2_2_wtk2}

Der nächste Meilenstein wird von Bachs \enquote{Das Wohltemperirte Clavier} markiert, in dem er Musik für verschiedene Wohltemperierte Stimmungen (WT) geschrieben hat.
Das waren Temperaturen, die einen Kompromiß zwischen mitteltöniger und pythagoreischer Stimmung darstellten.
Dieses Konzept funktionierte, weil die pythagoreische Stimmung zu Noten führt, die zu \textit{hoch} sind, während die mitteltönige zu Noten führt, die zu \textit{tief} sind.
Außerdem boten die WT nicht nur die Möglichkeit guter Terzen, sondern auch von guten Quinten.
\textbf{Die einfachste WT wurde von Kirnberger, einem Schüler Bachs, entworfen.
Der größte Vorteil der Temperatur von Kirnberger ist ihre Einfachheit.
Bessere WTs wurden von Werckmeister und von Young entwickelt.
Wenn wir die Stimmungen allgemein in mitteltönig, WT und pythagoreisch einteilen, dann ist ET eine WT, weil ET weder erhöht noch erniedrigt ist.}
Es gibt keine Aufzeichnungen darüber, welche Temperatur(en) Bach benutzte.
Wir können die Temperatur(en) nur anhand der Harmonien in seinen Kompositionen vermuten, insbesondere seines \enquote{Wohltemperierten Klaviers}, und diese Studien zeigen, daß im Grunde alle Details des Temperierens bereits zu Bachs Zeiten (vor 1700) bekannt waren, und daß Bach eine Temperatur benutzte, die sich von der von Werckmeister nicht sehr unterschied.

Die Violine scheint einen Vorteil aus ihrem einzigartigen Aufbau zu ziehen, um diese Temperaturprobleme zu umgehen.
Die leeren Saiten bilden miteinander Quintintervalle, so daß sie von Natur aus pythagoreisch gestimmt ist.
Da die Terzen immer rein gespielt werden können, hat sie alle Vorteile der pythagoreischen, mitteltönigen und WT-Stimmung, und weit und breit ist keine Wolfsquinte in Sicht!

In den letzten ca. 100 Jahren wurde ET fast überall akzeptiert.
Deshalb werden die anderen Temperaturen im allgemeinen als \enquote{historische Temperaturen} eingestuft, was klar ein falsche Bezeichnung ist.
Der historische Gebrauch der WT führte zu dem Konzept der Tonartfarbe, bei dem jede Tonart in Abhängigkeit von der Stimmung der Musik besondere Farbe verlieh, und zwar hauptsächlich durch die kleinen Verstimmungen, die \enquote{Spannung} und andere Effekte erzeugen.
Das komplizierte die Lage sehr, weil die Musiker sich nun nicht nur mit reinen Intervallen und Wolfsquinten befassen mußten, sondern auch mit Farben, die nicht so leicht zu definieren waren.
Das Ausmaß, in dem die Farben herausgebracht werden können, hängt vom Klavier, dem Pianisten und dem Zuhörer genauso ab wie vom Stimmer.
Beachten Sie, daß der Stimmer die Streckung (s. \enquote{\hyperref[c2_5_stre]{Was ist Streckung?}} am Ende von Abschnitt 5) mit der Temperatur verbinden kann, um die Farbe zu kontrollieren.
Nachdem man Musik gehört hat, die auf einem Klavier gespielt wird, das WT gestimmt ist, klingt ET eher trüb und farblos.
Deshalb ist die Farbe der Tonart wichtig.
Wichtiger sind die wundervollen Klänge von reinen (gestreckten) Intervallen bei WT.
Auf der anderen Seite gibt es in den WTs immer eine Art von Wolfsquinte, die bei der ET reduziert ist.

Zum Spielen der meisten Musik, die um die Zeit von Bach, Mozart und Beethoven komponiert wurde, ist WT am besten geeignet.
So hat Beethoven z.B. für die dissonanten Nonen im ersten Satz seiner Mondschein-Sonate Akkorde gewählt, die in WT am wenigsten dissonant und in ET viel schlechter sind.
Diese großen Komponisten waren sich der Temperatur genauestens bewußt.
Die meisten Werke aus der Zeit von Chopin oder Liszt wurden im Hinblick auf ET komponiert, so daß die Tonartfarbe kein Thema ist.
Obwohl diese Kompositionen für das geschulte Ohr in ET und WT unterschiedlich klingen, ist nicht klar, daß gegen WT etwas einzuwenden ist, weil reine Intervalle immer besser klingen als verstimmte.

Meine persönliche Ansicht hinsichtlich des Klaviers ist, daß wir von ET abkommen sollten, weil sie uns eines der angenehmsten Aspekte der Musik beraubt - reinen Intervallen.
Sie werden eine dramatische Demonstration davon erleben, wenn Sie den letzten Satz von Beethovens Waldstein-Sonate in ET und WT hören.
Mitteltönige Stimmung kann ziemlich extrem sein, es sei denn Sie spielen Musik dieser Periode (vor Bach), so daß uns die WTs bleiben.
Hinsichtlich der Einfachheit und der leichten Stimmbarkeit ist Kirnberger nicht zu schlagen.
Ich glaube, daß wenn Sie sich an WT gewöhnt haben, ET nicht genauso gut klingen wird.
Deshalb sollte die Welt die WTs zum Standard erheben.
Welche man auswählt, macht für die meisten Menschen keinen großen Unterschied, weil diejenigen, die nicht in den Temperaturen ausgebildet sind, im allgemeinen keinen großen Unterschied zwischen den hauptsächlichen Temperaturen bemerken, geschweige denn zwischen den unterschiedlichen WTs.
Das soll nicht heißen, daß wir alle Kirnberger benutzen sollten, sondern daß wir in den Temperaturen ausgebildet werden und eine Wahl haben sollten, anstatt in die Zwangsjacke der farblosen ET gesteckt zu werden.
Das ist nicht nur eine Frage des Geschmacks oder die Frage, ob die Musik besser klingt.
Wir sprechen darüber, unsere musikalische Sensibilität zu entwickeln und zu wissen, wie man diese wirklich reinen Intervalle benutzt.
Ein Nachteil von WT ist, daß es hörbar wird, wenn das Klavier auch nur ein wenig verstimmt ist.
Es würde mich jedoch freuen, wenn alle Klavierschüler ihre Sensibilität bis zu dem Punkt entwickeln würden, an dem sie bereits erkennen können, wenn das Klavier auch nur ein wenig verstimmt ist.



<!-- c23.html -->

\subsection{Werkzeuge zum Stimmen}
\label{c2_3}

\textbf{Sie werden einen Stimmhammer, mehrere Gummikeile, einen Filzstreifen zum Dämpfen, eine oder zwei Stimmgabeln und Ohrstöpsel oder Ohrenschützer benötigen.}
Professionelle Stimmer benutzen heutzutage auch eine elektronische Stimmhilfe; wir werden diese aber hier nicht berücksichtigen, weil sie für den Amateur nicht rentabel ist und ihre richtige Anwendung ein fortgeschrittenes Wissen über die Feinheiten des Stimmens erfordert.
Die Stimmethode, die wir hier behandeln, wird aurales Stimmen genannt - Stimmen nach Gehör.
Alle guten professionellen Stimmer müssen gute aurale Stimmer sein, auch wenn sie oft elektronische Stimmhilfen verwenden.

Für Flügel brauchen Sie größere Gummikeile zum Dämpfen, während \enquote{\hyperref[upright]{Aufrechte}} kleinere mit Metallgriffen erfordern.
Vier Keile jeden Typs werden ausreichen.
Sie können diese per Versand bestellen oder Ihren Stimmer bitten, den ganzen Satz Werkzeuge, den Sie benötigen, für Sie zu kaufen.

Die verbreitetsten Dämpfungsstreifen sind aus Filz, ungefähr 4ft lang und 5/8`` breit\footnote{ca. 1,22m x 1,6cm}.
Sie werden benutzt, um die 2 Nebensaiten der 3-saitigen Noten in der Oktave zu dämpfen, die für das \enquote{\hyperref[c2_4]{Einstellen des Bezugspunkts}} benutzt wird (s.u.).
Es gibt die Streifen auch als verbundene Gummikeile, aber diese funktionieren nicht genauso gut.
Die Streifen gibt es auch in Gummi, aber Gummi dämpft nicht so gut und ist nicht so stabil wie Filz (die Streifen können sich während des Stimmens verschieben oder herausspringen).
Der Nachteil der Filzstreifen ist, daß sie auf dem Resonanzboden eine Filzfaserschicht hinterlassen, die abgesaugt werden muß.

Ein Stimmhammer hoher Qualität besteht aus einem verlängerbaren Griff, einem an der Spitze des Griffs befestigten Kopf und einem auswechselbaren, in den Kopf geschraubten Einsatz.
Es ist eine gute Idee, einen Stimmwirbel zu haben, den Sie in den Einsatz stecken können, so daß Sie den Einsatz mit Hilfe eines Schraubstocks fest in den Kopf schrauben können.
Ansonsten könnten Sie den Einsatz verkratzen, wenn Sie ihn mit dem Schraubstock fassen.
Wenn der Einsatz nicht fest im Kopf sitzt, wird er sich während des Stimmens lösen.
Die meisten Klaviere erfordern einen Einsatz \#2, es sei denn, das Klavier wurde mit größeren Stimmwirbeln neu besaitet.
Der Standardkopf ist ein 5-Grad-Kopf.
Diese \enquote{5 Grad} sind der Winkel zwischen der Einsatzachse und dem Griff.
Sowohl die Köpfe als auch die Einsätze gibt es in verschiedenen Längen, aber \enquote{Standard-} oder \enquote{mittlere} Länge wird genügen.
 

\label{c2_3_gabel}

Besorgen Sie sich zwei Stimmgabeln - A440 und C523,3 - von guter Qualität.
Entwickeln Sie die gute Angewohnheit, sie am schmalen Hals des Griffs zu halten, so daß Ihre Finger nicht die Schwingungen der Stimmgabeln stören.
Klopfen Sie die Spitze der Gabel fest gegen einen muskulösen Teil Ihres Knies und testen Sie die Aushaltezeit (Sustain).
Sie sollte für 10 bis 20 Sekunden deutlich zu hören sein, wenn Sie sie nahe an Ihr Ohr halten.
Die beste Art, die Gabel zu hören, ist, die Spitze des Griffs auf den dreieckigen Knorpel (Tragus) zu setzen, der zur Mitte des Ohrlochs hin herausragt.
Sie können die Lautstärke der Gabel anpassen, indem Sie den Tragus mit dem Ende der Gabel ein- oder auswärts drücken.
Benutzen Sie keine Pfeifen; diese sind zu ungenau.

Ohrenschützer sind eine notwendige Schutzvorrichtung, da Gehörschäden das Berufsrisiko eines Stimmers sind.
Wie weiter unten erklärt wird, ist es notwendig \hyperref[c2_5_infi]{die Tasten hart anzuschlagen} (auf die Tasten zu hämmern - um den Jargon der Stimmer zu benutzen), um richtig zu stimmen, und die Klangintensität eines solchen Hämmerns kann das Ohr sehr leicht schädigen, was zu Gehörverlust und Tinnitus führt.
 

\subsection{Vorbereitung}
\label{c2_4}

Bereiten Sie das Stimmen vor, indem Sie den Notenständer entfernen, so daß sie an die Stimmwirbel herankommen (Flügel).
Für den folgenden Abschnitt brauchen Sie keine weiteren Vorbereitungen.
Um \enquote{die Bezugspunkte einzustellen} müssen Sie alle Nebensaiten der 3-fachen Saiten in der \enquote{Bezugsoktave} mit den Dämpfungsstreifen dämpfen, so daß, wenn Sie eine Note in dieser Oktave spielen, nur die mittlere Saite vibriert.
Sie werden wahrscheinlich je nach Stimmalgorithmus fast zwei Oktaven dämpfen müssen.
Probieren Sie zunächst den ganzen Stimmalgorithmus aus, um die höchste und die tiefste Note zu bestimmen, die Sie dämpfen müssen.
Dämpfen Sie dann alle Noten dazwischen.
Benutzen Sie das gerundete Ende des Drahtgriffs eines Dämpfungskeils für \enquote{Aufrechte}, um den Filz in den Raum zwischen den äußeren Saiten zweier nebeneinander liegenden Noten zu pressen.
 


<!-- c25.html -->

\subsection{Wie man anfängt}
\label{c2_5}

\subsubsection{Einleitung}
\label{c2_5a}

\textbf{Ohne einen Lehrer können Sie nicht einfach mit dem Stimmen anfangen.}
Sie werden schnell Ihren Bezugspunkt verlieren und keine Ahnung haben, wie Sie wieder zurückkommen.
\textbf{Deshalb müssen Sie zunächst bestimmte Stimmverfahren lernen und üben, damit Sie am Ende nicht mit einem unspielbaren Klavier dastehen, das Sie nicht wiederherstellen können.}
Dieser Abschnitt ist ein Versuch, Sie auf die Stufe zu bringen, bei der Sie ein richtiges Stimmen versuchen können, ohne auf Schwierigkeiten dieser Art zu stoßen.

\textbf{Als erstes müssen Sie lernen, was man tun kann und was nicht, um zu vermeiden, daß Sie das Klavier zerstören, was leicht geschehen kann.
Wenn man eine Saite zu stark spannt, dann bricht sie.\footnote{Verletzungsgefahr!}
Die anfänglichen Anweisungen sind dafür gedacht, Saitenbrüche aufgrund von amateurhaftem Vorgehen zu minimieren, lesen Sie sie deshalb sorgfältig.}
Sie müssen im voraus planen, was Sie tun, wenn eine Saite bricht.
Eine gebrochene Saite ist, auch wenn Sie über längere Zeit nicht ersetzt wird, für sich genommen keine Katastrophe für das Klavier.
Es ist jedoch wahrscheinlich klug, die ersten Übungen zu machen, kurz bevor man die Absicht hat, seinen Stimmer zu sich zu bitten.
Wenn Sie erst wissen wie man stimmt, ist ein Saitenbruch - außer bei sehr alten oder mißhandelten Klavieren -  ein seltenes Problem.
Die Stimmwirbel werden während des Stimmens um solch kleine Beträge gedreht, daß die Saiten fast nie brechen.
Ein verbreiteter Fehler, der von Anfängern begangen wird, ist, den Stimmhammer am falschen Stimmwirbel anzusetzen.
Da das Drehen des Wirbels keine hörbare Veränderung bewirkt, wird dann weitergedreht, bis die Saite bricht.
Eine Möglichkeit, das zu vermeiden, ist, \textbf{immer damit zu beginnen, \textit{tiefer} zu stimmen, wie es unten empfohlen wird, und niemals den Wirbel zu drehen, ohne den Ton anzuhören.}

\textbf{Die wichtigste Aufgabe für einen beginnenden Stimmer ist, den Zustand des Stimmstocks zu bewahren.}
Der Druck des Stimmstocks auf die Wirbel ist enorm.
Sie dürfen das natürlich niemals tun, aber angenommen, Sie würden den Wirbel sehr schnell um 180 Grad drehen, wäre die dabei an der Fläche zwischen Wirbel und Stimmstock erzeugte Hitze ausreichend, um das Holz zu verbrennen und seine molekulare Struktur zu verändern.
Es ist klar, daß alle Drehungen des Wirbels in langsamen, kleinen Schritten ausgeführt werden müssen.
Wenn Sie den Wirbel durch Drehen entfernen müssen, drehen Sie ihn nur eine viertel Drehung (gegen den Uhrzeigersinn), warten Sie einen Moment, bis sich die Hitze von der Grenzfläche weg verteilt hat, wiederholen Sie dann den Vorgang, usw., um eine Beschädigung des Stimmstock zu vermeiden.

\textbf{Ich werde alles am Beispiel des Flügels erklären, die entsprechenden Bewegungen für \enquote{\hyperref[upright]{Aufrechte}} sollten aber offensichtlich sein.}
\textbf{Es gibt beim Stimmen zwei grundlegende Bewegungen.
Die erste ist die Drehung des Wirbels, so daß die Saite entweder angezogen oder entspannt wird.\footnote{Mit \enquote{wiegen} ist im folgenden kein wildes Hin- und Herschaukeln des Stimmwirbels gemeint, sondern jeweils das behutsame Ziehen von der Saite weg bzw. behutsame Nachgeben zur Saite hin!}
Die zweite ist, den Wirbel rückwärts zu Ihnen hin zu wiegen (um an der Saite zu ziehen) oder ihn vorwärts zu wiegen, in Richtung der Saite, um sie nachzulassen.}
Wenn die wiegende Bewegung extrem ausgeführt wird, vergrößert sie das Loch und beschädigt den Stimmstock.
Beachten Sie, daß das Loch an der Oberseite des Stimmstocks ein wenig elliptisch ist, weil die Saite den Wirbel in Richtung der Hauptachse der Ellipse zieht.
Darum vergrößert ein kleines Wiegen rückwärts die Ellipse nicht, weil der Wirbel durch die Saite immer in das vordere Ende der Ellipse gezogen wird.
Auch ist der Wirbel nicht gerade, sondern wird durch den Zug der Saite elastisch zur Saite hin gebogen.
Deshalb kann die wiegende Bewegung für das Bewegen der Saite sehr effektiv sein.
Sogar ein geringes Maß an Vorwärtswiegen, innerhalb der Elastizität des Holzes, ist unschädlich.
Anhand dieser Überlegungen wird deutlich, daß \textbf{Sie die Drehung benutzen müssen, wann immer sie möglich ist, und die wiegende Bewegung nur, wenn sie absolut notwendig ist}.
Nur sehr kleine wiegende Bewegungen sollten angewandt werden.
Bei den höchsten Noten (die zwei obersten Oktaven), ist die für das Stimmen der Saite notwendige Bewegung so gering, daß Sie eventuell nicht in der Lage sind, sie angemessen durch das Drehen des Wirbels zu kontrollieren.
Das Wiegen bietet eine viel feinere Kontrolle und kann für diese \hyperref[c2_5_infi]{abschließende, winzige Bewegung} benutzt werden, um die Saite in perfekte Stimmung zu bringen.

Was ist nun der einfachste Weg, mit dem Üben zu beginnen?
Lassen Sie uns zunächst die am einfachsten zu stimmenden Noten auswählen.
Diese liegen in der \hyperref[Noten]{C3-C4}-Oktave.
Tiefere Noten sind wegen ihres hohen harmonischen Gehalts schwieriger zu stimmen, und die höheren Noten sind schwierig, weil das für das Stimmen notwendige Maß der Wirbeldrehung mit steigender Tonhöhe abnimmt.
Beachten Sie, daß C4 für das mittlere C steht; das H direkt darunter ist H3, und das D direkt über dem mittleren C ist D4.
Die Oktavnummer 1, 2, 3, . . . ändert sich somit beim C, nicht beim A.
Wählen wir das G3 als unsere Übungsnote, und fangen wir mit dem Numerieren der Saiten an.
Jede Note in diesem Bereich hat 3 Saiten.
Von der linken Seite beginnend, numerieren wir die Saiten 123 (für G3), 456 (für G\#3), 789 (für A3), usw.
Fügen Sie zwischen den Saiten 3 und 4 einen Keil ein, um die Saite 3 zu dämpfen, so daß nur 1 und 2 schwingen können, wenn Sie G3 spielen.
Plazieren Sie den Keil ungefähr in der Mitte zwischen Steg und Agraffe.

\textbf{Es gibt zwei grundlegende Arten zu stimmen: unisono und harmonisch.}
Beim Unisono werden die beiden Saiten identisch gestimmt.
Beim harmonischen Stimmen wird eine Saite harmonisch zur anderen gestimmt, z.B. im Abstand einer Terz, Quarte, Quinte oder Oktave.
Die drei Saiten jeder Note unisono zu stimmen, ist einfacher, als harmonisch zu stimmen; lassen Sie uns also damit beginnen.
 

\label{c2_5b}
\subsubsection{Einsetzen und Bewegen des Stimmhammers}
\label{c2_5_hamm} 

Wenn Ihr Stimmhammer eine justierbare Länge hat, ziehen Sie ihn ungefähr 3 Zoll\footnote{7,5cm} heraus, und stellen Sie ihn fest.
Halten Sie den Griff des Stimmhammers in Ihrer RH und den Einsatz mit Ihrer LH, und setzen Sie den Einsatz oben am Wirbel an.
Richten Sie den Griff so aus, daß er ungefähr senkrecht zu den Saiten steht und nach rechts zeigt.
Wackeln Sie mit Ihrer RH leicht mit dem Griff um den Stimmwirbel, und schieben Sie den Einsatz mit Ihrer LH über den Wirbel nach unten, so daß der Einsatz sicher so weit eingeschoben ist wie es geht.
\textbf{Entwickeln Sie vom ersten Tag an die Angewohnheit, mit dem Einsatz zu wackeln, so daß er sicher eingeschoben ist.}
An diesem Punkt ist der Griff wahrscheinlich nicht perfekt senkrecht zu den Saiten; wählen Sie einfach die Position des Einsatzes so, daß der Griff so gut wie möglich senkrecht steht.
Finden Sie nun eine Position, in der Sie Ihre RH so abstützen, daß Sie einen festen Druck auf den Hammer ausüben können.
Sie können z.B. die Spitze des Griffs mit dem Daumen und einem oder zwei Fingern greifen und den Arm auf dem hölzernen Klavierrahmen abstützen oder den kleinen Finger auf den Stimmwirbeln direkt unter dem Griff abstützen.
Wenn der Griff näher an der Platte (dem Metallrahmen) über den Saiten ist, könnten Sie Ihre Hand an der Platte abstützen.
Sie sollten den Griff nicht so greifen, wie Sie einen Tennisschläger halten, und ziehen bzw. drücken, um den Wirbel zu drehen - das gibt Ihnen nicht genügend Kontrolle.
Sie werden vielleicht nach mehreren Jahren Übung dazu in der Lage sein, aber am Anfang ist eine exakte Kontrolle zu schwierig, wenn man den Griff packt und drückt, ohne sich an etwas abzustützen.
\textbf{Entwickeln Sie deshalb die Angewohnheit, je nach der Griffposition gute Stellen zum Abstützen zu finden.}
Üben Sie diese Positionen und stellen Sie sicher, daß Sie einen kontrollierten, konstanten und kräftigen Druck auf den Griff ausüben können, aber drehen Sie noch keine Wirbel.

Der Hammergriff muß nach rechts zeigen, so daß Sie, wenn Sie ihn zu sich hin drehen (die Saite wird gespannt), gegen die Kraft der Saite arbeiten und den Wirbel aus der Vorderseite des Lochs (zur Saite hin) befreien.
Das gestattet - wegen der reduzierten Reibung - dem Wirbel, sich freier zu drehen.
Wenn Sie \textit{tiefer} stimmen, versuchen sowohl Sie als auch die Saite, den Wirbel in die gleiche Richtung zu drehen.
Der Wirbel würde dabei zu leicht drehen, wenn nicht sowohl Ihr Druck als auch der Zug der Saite den Wirbel gegen die Vorderseite des Lochs drücken, somit den Druck (Reibung) erhöhen und es verhindern würden.
Würden Sie den Griff nach links stellen, bekämen Sie sowohl bei der Bewegung zum Höher- als auch zum Tieferstimmen Probleme.
Beim Höherstimmen drücken sowohl Sie als auch die Saite den Wirbel gegen die Vorderseite des Lochs, was es um so schwieriger macht, den Wirbel zu drehen, und das Loch beschädigt.
Beim Tieferstimmen neigt der Hammer dazu, den Wirbel von der Vorderkante des Lochs abzuheben und reduziert die Reibung.
Außerdem drehen sowohl der Hammer als auch die Saite den Wirbel in die gleiche Richtung.
Jetzt dreht sich der Wirbel zu leicht.
Der Hammergriff muß bei \enquote{Aufrechten} nach links zeigen.
Wenn man von oben auf den Stimmwirbel schaut, sollte der Hammer bei Flügeln nach 3 Uhr und bei \enquote{Aufrechten} nach 9 Uhr zeigen.
In beiden Fällen befindet sich der Hammer auf der Seite der letzten Windung der Saite.

Professionelle Stimmer benutzen diese Hammerpositionen nicht.
Die meisten benutzen 1-2 Uhr für Flügel und 10-11 Uhr für \enquote{Aufrechte}, und Reblitz empfiehlt 6 Uhr für Flügel und 12 Uhr für \enquote{Aufrechte}.
Um zu verstehen warum, betrachten wir zunächst das Einsetzen des Hammers bei einem Flügel bei 12 Uhr (bei 6 Uhr ist es ähnlich).
Nun ist die Reibung des Wirbels mit dem Stimmstock beim Höher- und Tieferstimmen die gleiche.
Beim Höherstimmen arbeiten Sie jedoch gegen die Saitenspannung und beim Tieferstimmen hilft Ihnen die Saite.
Deshalb ist die Differenz der benötigten Kraft zwischen Höher- und Tieferstimmen viel größer als die Differenz ist, wenn der Hammer auf 3 Uhr steht, was ein Nachteil ist.
Anders als bei der 3-Uhr-Position, wiegt der Wirbel während des Stimmens nicht vor und zurück, so daß, wenn Sie den Druck auf den Stimmhammer nachlassen, der Wirbel nicht zurückspringt - er ist stabiler - und Sie können eine höhere Genauigkeit erreichen.

Die 1-2-Uhr-Position ist ein guter Kompromiß, der sowohl die Vorteile der 3-Uhr-Position als auch der 12-Uhr-Position ausnutzt.
Anfänger haben nicht die Genauigkeit, um den vollen Vorteil aus der 1-2-Uhr-Position zu ziehen; mein Vorschlag ist deshalb, mit der 3-Uhr-Position anzufangen, was zunächst einfacher sein sollte, und zur 1-2-Uhr-Position überzugehen, wenn Ihre Genauigkeit steigt.
Wenn Sie gut werden, kann die höhere Genauigkeit der 1-2-Uhr-Position Ihr Stimmen beschleunigen, so daß Sie jede Saite in wenigen Sekunden stimmen können.
Bei der 3-Uhr-Position werden Sie raten müssen wieviel der Wirbel zurückspringt und um diesen Betrag überstimmen müssen, was mehr Zeit benötigt.
Klar wird es wichtiger, wo Sie den Hammer plazieren, sobald Sie besser werden.
 

\label{c2_5c}
\subsubsection{Den Wirbel einstellen}
\label{c2_5_wirb} 

\textbf{Es ist wichtig, den Stimmwirbel richtig \enquote{einzustellen}, damit die Stimmung hält.}
Wenn man von oben auf den Wirbel schaut, kommt die Saite von der rechten Seite des Wirbels (bei Flügeln - sie ist bei \enquote{Aufrechten} auf der linken Seite) und ist um ihn herumgewickelt.
Deshalb stimmen Sie \textit{höher}, wenn Sie den Wirbel im Uhrzeigersinn drehen, und \textit{tiefer}, wenn Sie den Wirbel gegen den Uhrzeigersinn drehen.
Die Saitenspannung versucht immer, den Wirbel gegen den Uhrzeigersinn zu drehen (oder \textit{tiefer}).
Normalerweise verstimmt sich ein Klavier \textit{tiefer}, wenn man es spielt.
Da der Stimmstock den Wirbel so stark umklammert, ist der Wirbel jedoch niemals gerade sondern verdreht.

Wenn man ihn im Uhrzeigersinn dreht und anhält, wird die Oberseite des Wirbels in bezug auf den Boden im Uhrzeigersinn verdreht.
In dieser Position möchte die Oberseite des Wirbels gegen den Uhrzeigersinn drehen (der Wirbel möchte sich zurückdrehen), aber er kann nicht, weil er vom Stimmstock gehalten wird.
Erinnern Sie sich daran, daß die Saite ebenfalls versucht, ihn gegen den Uhrzeigersinn zu drehen.
Die beiden Kräfte zusammen können genügen, um das Klavier schnell \textit{tiefer} zu verstimmen, wenn man etwas laut spielt.

Wenn der Wirbel gegen den Uhrzeigersinn gedreht wird, geschieht das Gegenteil - der Wirbel will sich im Uhrzeigersinn zurückdrehen, was der Saitenkraft entgegenwirkt.
Das reduziert das Nettodrehmoment am Wirbel, was die Stimmung stabiler macht.
Tatsächlich kann man den Wirbel so weit gegen den Uhrzeigersinn verdrehen, daß die zurückdrehende Kraft viel größer als die Saitenkraft ist, und das Klavier kann sich dann beim Spielen selbst \textit{höher} verstimmen.
Klar muß man den Wirbel richtig einstellen, damit man eine stabile Stimmung erzeugt.
Diese Erfordernis wird bei den folgenden Stimmanweisungen berücksichtigt.
 

\label{c2_5d}
\subsubsection{Unisono stimmen}
\label{c2_5_unis}

Stecken Sie nun den Stimmhammer auf den Wirbel für Saite 1.
Wir werden die Saite 1 nach Saite 2 stimmen.
\textbf{Die Stimmbewegung, die Sie üben werden, ist:}

\begin{enumerate}[label={\arabic*.}] 
 \item \textbf{\textit{tiefer}}
 \item \textbf{\textit{höher}}
 \item \textbf{\textit{tiefer}}
 \item \textbf{\textit{höher}}
 \item \textbf{\textit{tiefer}}
 \end{enumerate}
Außer bei (1) muß jede Bewegung kleiner als die vorhergehende sein.
Wenn Sie besser werden, werden Sie Schritte passend hinzufügen oder weglassen.
Wir nehmen an, daß die beiden Saiten fast gestimmt sind.
Während Sie stimmen, müssen Sie zwei Regeln beachten:

\begin{itemize} 
 \item \textbf{Drehen Sie nie einen Wirbel, wenn Sie nicht gleichzeitig auf den Ton hören.}
 \item \textbf{Lassen Sie nie den Druck auf den Griff des Stimmhammers nach, bis diese Bewegung komplett ist.}
 \end{itemize}
Fangen wir z.B. mit Bewegung (1) \textit{tiefer} an: spielen Sie die Note alle ein oder zwei Sekunden mit der LH, so daß es einen dauernden Ton gibt, während Sie das Ende des Hammergriffs mit dem Daumen und dem Zeigefinger von sich weg drücken.
Spielen Sie die Note so, daß Sie einen fortwährenden Ton aufrecht erhalten.
Heben Sie die Taste nicht an, egal wie lang, da dies den Ton stoppt.
Halten Sie die Taste unten, und spielen Sie mit einer schnellen Auf- und Abbewegung, so daß der Ton nicht unterbrochen wird.
Der kleine Finger und der Rest Ihrer RH sollten gegen das Klavier abgestützt werden.
Die erforderliche Bewegung des Hammers beträgt nur ein paar Millimeter.
Zunächst werden Sie einen steigenden Widerstand spüren, und dann wird der Wirbel anfangen sich zu drehen.
Bevor der Wirbel anfängt sich zu drehen, sollten Sie eine Veränderung im Ton hören.
Hören Sie beim Drehen des Wirbels darauf, wie die Saite 1 \textit{tiefer} wird und eine Schwebung mit der mittleren Saite erzeugt; die Schwebungsfrequenz nimmt während Sie drehen zu.
Hören Sie bei einer Schwebungsfrequenz von 2 bis 3 je Sekunde auf.
Das äußere Ende des Hammergriffs sollte sich weniger als einen cm bewegen.
Erinnern Sie sich daran, daß Sie nie den Wirbel drehen, wenn kein Ton zu hören ist, weil Sie sonst in bezug auf die Änderung der Schwebungen sofort die Orientierung verlieren.
Halten Sie aus demselben Grund immer einen konstanten Druck auf den Hammer aufrecht, bis die Bewegung abgeschlossen ist.

Was ist die rationale Erklärung für die o.a. 5 Bewegungen?
Angenommen, die beiden Saiten sind vernünftig gestimmt, dann stimmen Sie bei Schritt (1) die Saite 1 \textit{tiefer}, um sicherzustellen, daß Sie in Schritt (2) den Stimmpunkt passieren\footnote{d.h. den Punkt, an dem die Saite genau richtig gestimmt ist}.
Das schützt auch gegen die Möglichkeit, daß Sie den Hammer auf den falschen Stimmwirbel gesetzt haben; solange Sie \textit{tiefer} stimmen, werden Sie niemals eine Saite zerbrechen.

Nach (1) sind Sie mit Sicherheit \textit{tiefer}, so daß Sie in Schritt (2) auf den Stimmpunkt hören können, während Sie durch ihn hindurchkommen.
Gehen Sie darüber hinaus, bis Sie eine Schwebungsfrequenz von ungefähr 2 bis 3 je Sekunde auf der \textit{höher}en Seite hören und stoppen Sie.
Sie wissen nun, wo der Stimmpunkt ist und wie er klingt.
Der Grund dafür, so weit über den Stimmpunkt hinaus zu gehen, ist, daß Sie den Wirbel wie oben erklärt einstellen möchten.

Kehren Sie nun zu \textit{tiefer} zurück, Schritt (3), aber stoppen Sie dieses Mal direkt hinter dem Stimmpunkt, sobald Sie irgendwelche einsetzenden Schwebungen hören können.
Der Grund, warum man nicht zu weit hinter den Stimmpunkt kommen möchte, ist, daß man nicht das \enquote{Einstellen des Wirbels} aus Schritt (2) rückgängig machen möchte.
Achten Sie wieder genau darauf, wie der Stimmpunkt klingt.
Er sollte perfekt sauber und rein klingen.
Dieser Schritt stellt sicher, daß Sie den Wirbel nicht zu weit eingestellt haben.

Führen Sie nun das endgültige Stimmen durch, indem Sie in Richtung \textit{höher} gehen (Schritt 4), dabei so wenig wie möglich über die perfekte Stimmung hinausgehen und die Saite dann durch Drehen nach \textit{tiefer} (Schritt 5) in die endgültige Stimmung bringen.
Beachten Sie, daß Ihre letzte Bewegung immer \textit{tiefer} sein muß, um den Wirbel einzustellen.
Wenn Sie gut darin werden, könnten Sie in der Lage sein, das Ganze in drei Bewegungen (\textit{tiefer, höher, tiefer}) durchzuführen.

Idealerweise sollten Sie von Schritt (1) bis zur endgültigen Stimmung den Ton ohne Unterbrechung aufrechterhalten, immer Druck auf den Griff ausüben und niemals den Hammer loslassen.
Am Anfang werden Sie das wahrscheinlich Bewegung für Bewegung ausführen müssen.
Wenn Sie es beherrschen, wird die ganze Durchführung nur ein paar Sekunden dauern.
Aber zunächst wird es \textit{viel} länger dauern.
Bis Sie Ihre \enquote{Stimmuskeln} entwickelt haben, werden Sie schnell ermüden und von Zeit zu Zeit aufhören müssen, um sich zu erholen.
Das gilt nicht nur für die Hand- und Armmuskeln, auch die erforderliche Konzentration des Geistes und des Gehörs auf die Schwebungen kann eine große Anstrengung sein und schnell Ermüdung verursachen.
Sie müssen schrittweise eine \enquote{Stimmausdauer} entwickeln.
Die meisten kommen besser zurecht, wenn Sie nur mit einem statt mit beiden Ohren hören; drehen Sie deshalb Ihren Kopf, um festzustellen, welches Ohr besser ist.

\textbf{Der häufigste Fehler, den Anfänger in diesem Stadium begehen, ist, bei dem Versuch, die Schwebungen zu hören, die Stimmbewegung zu unterbrechen.}
Schwebungen sind schwer zu hören, wenn sich nichts ändert.
Wenn der Wirbel nicht gedreht wird, ist schwer zu entscheiden, welche der vielen Dinge, die man hört, die Schwebung ist, auf die man sich konzentrieren muß.
\textbf{Stimmer bewegen den Hammer weiter und hören dann auf \underline{die Veränderungen der Schwebungen}.}
Wenn die Schwebungen sich ändern, ist es einfacher, die einzelne Schwebung zu identifizieren, die man für das Stimmen dieser Saite benutzt.
Deshalb wird es nicht einfacher, wenn man die Stimmbewegung verlangsamt.
Somit bewegt sich der Anfänger auf einem schmalen Grat.
Wenn man den Wirbel zu schnell dreht, bricht die Hölle los und man verliert die Orientierung.
Wenn man auf der anderen Seite zu langsam dreht, wird es schwierig, die Schwebungen zu identifizieren.
Arbeiten Sie deshalb daran, den Bereich der Bewegung zu bestimmen, den Sie benötigen, um die Schwebungen zu erkennen und die richtige Geschwindigkeit, mit der Sie den Wirbel beständig drehen können, um die Schwebungen entstehen und verschwinden zu lassen.
Falls Sie sich hoffnungslos verirrt haben, dämpfen Sie die Saiten 2 und 3, indem Sie einen Keil zwischen sie setzen, spielen Sie die Note, und hören Sie, ob Sie eine andere Note auf dem Klavier finden, die der Note nahe kommt.
Wenn die andere Note tiefer ist als G3, dann müssen Sie \textit{höher} stimmen, um zurückzukommen, und umgekehrt.

Wenn Sie nun die Saite 1 mit Saite 2 gleich gestimmt haben, bringen Sie den Keil in eine neue Position, so daß Saite 1 gedämpft wird und die Saiten 2 und 3 frei schwingen können.
Stimmen Sie Saite 3 nach Saite 2.
Wenn Sie zufrieden sind, entfernen Sie den Keil und hören Sie, ob das G nun frei von Schwebungen ist.
Sie haben eine Note gestimmt!
Wenn das G ziemlich gut gestimmt war, bevor Sie angefangen haben, haben Sie nicht viel erreicht; finden Sie eine Note in der Nähe, die aus der Stimmung ist, um zu sehen, ob Sie sie \enquote{reinigen} können.
Beachten Sie, daß Sie bei diesem Schema immer eine einzelne Saite nach einer anderen einzelnen Saite stimmen.
Im Prinzip sind, wenn Sie wirklich gut sind, die Saiten 1 und 2 perfekt gestimmt, nachdem Sie mit dem Stimmen von 1 fertig sind, so daß Sie den Keil nicht mehr brauchen.
Sie sollten in der Lage sein, Saite 3 nach den zusammen schwingenden 1 und 2 zu stimmen.
In der Praxis funktioniert das nicht, bis Sie es wirklich beherrschen.
Das kommt von einem Phänomen, das man \hyperref[mitschwingung]{Mitschwingung} nennt.
 


<!-- c25e.html -->

\label{c2_5e}
\subsubsection{Mitschwingung}
\label{c2_5_mits} 

Die Genauigkeit, die erforderlich ist, um zwei Saiten in perfekte Stimmung zu bringen, ist so hoch, daß es eine fast unmögliche Aufgabe ist.
Es stellt sich heraus, daß es in der Praxis einfacher ist: \textbf{Wenn die Frequenzen sich in einem Bereich einander annähern, der \enquote{Mitschwingungsbereich} genannt wird, dann ändern die beiden Saiten ihre Frequenzen aufeinander zu, so daß Sie mit der gleichen Frequenz schwingen.}
Das geschieht, weil die beiden Saiten nicht unabhängig sind, sondern am Steg miteinander gekoppelt.
Wenn sie gekoppelt sind, dann bringt die Saite, die mit einer höheren Frequenz schwingt, die langsamere Saite dazu, mit einer etwas höheren Frequenz zu schwingen und umgekehrt.
Der Nettoeffekt ist, daß beide Frequenzen zur Durchschnittsfrequenz der beiden hin getrieben werden.
Somit wissen Sie, wenn Sie die Saiten 1 und 2 unisono stimmen, überhaupt nicht, ob sie perfekt gestimmt sind oder nur im Mitschwingungsbereich (außer wenn Sie ein erfahrener Stimmer sind).
Am Anfang werden sie wahrscheinlich nicht perfekt gestimmt sein.

Wenn Sie nun versuchen müßten, die dritte Saite nach den beiden Saiten zu stimmen, die in Mitschwingung sind, würde die dritte Saite die Saite, die ihr in der Frequenz am nächsten ist, in Mitschwingung versetzen.
Die andere Saite kann aber in bezug auf die Frequenz zu weit entfernt sein.
Sie wird aus der Mitschwingung ausbrechen und dissonant klingen.
Das Resultat ist, daß Sie, egal wo Sie sind, immer Schwebungen hören werden - der Stimmpunkt verschwindet!
Man könnte meinen, daß wenn die dritte Saite in der Durchschnittsfrequenz der beiden Saiten, die in Mitschwingung sind, gestimmt wäre, alle drei zur Mitschwingung übergehen sollten.
Es stellt sich heraus, daß das nicht geschieht, außer wenn alle drei Frequenzen perfekt gestimmt sind.
Wenn die ersten beiden Saiten genügend weit auseinander sind, erfolgt ein komplexer Energietransfer zwischen allen drei Saiten.
Sogar wenn die ersten beiden nah beieinander sind, gibt es höhere harmonische Schwingungen, die verhindern, daß alle Schwebungen verschwinden, wenn eine dritte Saite hinzukommt.
Zusätzlich gibt es häufig Fälle, in denen man nicht alle Schwebungen völlig eliminieren kann, weil die beiden Saiten nicht identisch sind.
Deshalb würde sich ein Anfänger völlig verirren, wenn er eine dritte Saite nach einem Paar Saiten stimmen sollte.
\textbf{Bis Sie es beherrschen, den Mitschwingungsbereich herauszufinden, stimmen Sie immer eine Saite nach einer, niemals eine nach zwei.}
Außerdem bedeutet, daß Sie 1 nach 2 und 3 nach 2 gestimmt haben, nicht, daß die drei Saiten \enquote{sauber} zusammen klingen werden.
Prüfen Sie es immer; wenn die Saiten nicht völlig \enquote{sauber} sind, müssen Sie die störende Saite finden und es erneut versuchen.

Beachten Sie den Gebrauch des Ausdrucks \enquote{sauber}.
Mit genügender Übung werden Sie bald aufhören, auf die Schwebungen zu hören; statt dessen werden Sie nach einem reinen Klang suchen, der sich irgendwo innerhalb des Mitschwingungsbereichs ergibt.
Dieser Punkt hängt davon ab, welche Arten von Obertönen jede Saite erzeugt.
Im Prinzip versuchen wir, wenn wir unisono stimmen, die Grundschwingungen zur Deckung zu bringen.
In der Praxis ist ein kleiner Fehler in den Grundschwingungen verglichen mit demselben Fehler in einer hohen Oberschwingung unhörbar.
Leider sind diese hohen Obertöne im allgemeinen keine exakten harmonischen Obertöne, sondern sind von Saite zu Saite unterschiedlich.
Wenn die Grundtöne übereinstimmen, erzeugen deshalb diese hohen Obertöne hochfrequente Schwebungen, die die Note \enquote{schmutzig} oder \enquote{blechern} machen.
Wenn die Grundtöne gerade so verstimmt sind, daß die Obertöne keine Schwebungen erzeugen, \enquote{versäubert} sich die Note.
\textbf{Die Realität ist sogar noch komplizierter, weil einige Saiten, besonders bei Klavieren niedrigerer Qualität, eine zusätzliche Eigenresonanz haben, was es unmöglich macht, bestimmte Schwebungen völlig zu eliminieren.}
Diese Schwebungen werden sehr ärgerlich, wenn man diese Note benutzen muß, um eine andere zu stimmen.
 

\label{c2_5f}
\subsubsection{Diese letze infinitesimale Bewegung ausführen}
\label{c2_5_infi}

Wir kommen nun zur nächsten Schwierigkeitsstufe.
Finden Sie eine Note nahe G5, die leicht außerhalb der Stimmung ist, und wiederholen Sie das oben für G3 angegebene Verfahren.
Die Stimmbewegungen für diese höheren Noten sind viel kleiner, was sie schwieriger macht.
Sie werden vielleicht nicht in der Lage sein, durch das Drehen des Wirbels eine ausreichende Genauigkeit zu erreichen.
Wir müssen eine neue Fertigkeit erlernen.
\textbf{Diese Fertigkeit erfordert, daß Sie auf die Tasten \enquote{hämmern}, benutzen Sie deshalb Ihre Ohrenschützer oder Ohrstöpsel.}

Typischerweise werden Sie bei Bewegung (4) erfolgreich sein, aber bei Bewegung (5) wird sich der Wirbel entweder nicht bewegen oder über den Stimmpunkt hinwegspringen.
\textbf{Damit die Saite sich in kleineren Schritten vorwärts bewegt, müssen Sie einen Druck auf den Stimmhammer ausüben, der knapp unter dem Punkt liegt, an dem der Wirbel springt.
Schlagen Sie nun die Note fest an, während Sie den gleichen Druck auf den Stimmhammer aufrechterhalten.}
Die zusätzliche Saitenspannung durch den harten Hammerschlag\footnote{Hammer der Klaviermechanik, nicht der Stimmhammer!} läßt die Saite ein kleines Stück vorwärtsgehen.
Wiederholen Sie das, bis sie perfekt gestimmt ist.
Es ist wichtig, niemals den Druck auf den Stimmhammer nachzulassen und den Druck während dieser wiederholten Vorwärtssprünge konstant zu halten, oder Sie werden\footnote{in bezug auf die Saitenfrequenz} schnell die Orientierung verlieren.
Wenn die Saite perfekt gestimmt ist und Sie den Hammer loslassen, könnte der Wirbel zurückspringen und die Saite leicht \textit{tiefer} werden lassen. Sie werden aus der Erfahrung heraus lernen müssen, wie weit er zurückspringt, und es während des Stimmvorgangs entsprechend kompensieren müssen.

Die Notwendigkeit, auf die Saite zu hämmern, damit sie sich vorwärts bewegt, ist ein Grund, warum man Stimmer oft auf die Tasten hämmern hört.
Es ist eine gute Idee, sich anzugewöhnen, die meisten Noten zu hämmern, weil das die Stimmung stabilisiert.
Der daraus resultierende Ton kann so laut sein, daß das Ohr geschädigt wird, und eines der Berufsrisiken von Stimmern ist ein Gehörschaden wegen des Hämmerns.
Die Lösung ist die Benutzung von Ohrenstöpseln.
Beim Hämmern werden Sie auch mit Ohrstöpseln die Schwebungen problemlos hören.
Das verbreitetste anfängliche Symptom eines Gehörschadens ist der Tinnitus (Klingeln im Ohr).
Sie können die zum Hämmern notwendige Kraft minimieren, indem Sie den Druck auf den Stimmhammer erhöhen.
Ein geringeres Hämmern ist auch erforderlich, wenn der Stimmhammer parallel zu den Saiten statt rechtwinklig dazu steht, und ein noch geringeres, wenn Sie ihn nach links zeigen lassen.
Das ist ein weiterer Grund, warum viele Stimmer ihren Stimmhammer eher parallel zu den Saiten benutzen als rechtwinklig dazu.
Beachten Sie, daß es zwei Möglichkeiten gibt, ihn parallel zu halten: zu den Saiten hin (12 Uhr) und von den Saiten weg (6 Uhr).
Experimentieren Sie mit unterschiedlichen Hammerpositionen, wenn Sie an Erfahrung gewonnen haben, da Ihnen das viele Möglichkeiten für das Lösen verschiedener Probleme eröffnet.
Mit dem beliebten 5-Grad-Kopf auf dem Hammer sind Sie z.B. nicht in der Lage, bei der höchsten Oktave den Griff nach rechts zeigen zu lassen, weil er auf den hölzernen Klavierrahmen treffen kann.
 

\label{c2_5g}
\subsubsection{Ausgleich der Saitenspannung}
\label{c2_5_span} 

\textbf{Das Hämmern hilft auch dabei, die Saitenspannung gleichmäßiger auf die ganzen nicht klingenden Abschnitte der Saite zu verteilen, wie z.B. den Bereich im Duplex, aber besonders den Abschnitt zwischen dem Capotasto (Druckstab) und der Agraffe.}
Es gibt eine Kontroverse darüber, ob der Ausgleich der Saitenspannung den Klang verbessert.
Es steht außer Frage, daß eine gleichmäßige Spannung die Stimmung stabiler macht.
Es kann jedoch fraglich sein, ob sie einen \textit{wesentlichen} Unterschied in der Stabilität ausmacht, besonders wenn die Wirbel während des Stimmens korrekt eingestellt wurden.
Bei vielen Klavieren sind die Duplex-Abschnitte fast völlig mit Filz gedämpft, weil sie unerwünschte Schwingungen erzeugen könnten.
Tatsächlich sind die \enquote{nicht klingenden} Abschnitte bei fast jedem Klavier gedämpft.
Anfänger müssen sich über die Spannung in diesen Abschnitten der Saiten keine Gedanken machen.
Deshalb ist das schwere Hämmern für den Anfänger nicht notwendig, obwohl es nützlich ist, diese Fertigkeit zu erlernen.

\textbf{\textit{Meiner persönlichen Meinung nach trägt der Klang des Duplex-Abschnitts nichts zum Klavierklang bei.}}
In Wahrheit ist dieser Klang unhörbar und wird im Baß, wo er hörbar würde, völlig abgedämpft.
Deshalb ist die \enquote{Kunst des Stimmens des Duplex-Abschnitts} ein Mythos, obwohl den meisten Klavierstimmern (einschließlich Reblitz!) von den Herstellern beigebracht wurde, daran zu glauben, weil es ein gutes Verkaufsargument abgibt.
Der einzige Grund, warum man den Duplex-Abschnitt stimmen sollte, ist, daß der Steg sowohl im Knoten des klingenden als auch des nicht klingenden Bereichs sein sollte; ansonsten wird das Stimmen schwierig, der Sustain wird eventuell verkürzt, und man verliert die Gleichmäßigkeit.
Wenn man Begriffe der Mechanik benutzt, kann man sagen, daß den Duplex-Abschnitt zu stimmen die Schwingungsimpedanz des Stegs optimiert.
Mit anderen Worten: Der Mythos ändert nichts an der Fähigkeit der Stimmer, ihren Job zu machen.
Nichtsdestoweniger ist ein gutes Verständnis sicher förderlich.
Der Duplex-Abschnitt wird benötigt, damit der Steg sich freier bewegen kann, nicht für die Tonerzeugung.
Offensichtlich verbessert der Duplex-Abschnitt die Klangqualität (des klingenden Bereichs), weil er die Impedanz des Stegs optimiert, aber nicht, weil er einen Ton erzeugt.
Die Tatsache, daß der Duplex-Abschnitt im Baß gedämpft und im Diskant völlig unhörbar ist, beweist, daß der Klang des Duplex-Abschnitts nicht benötigt wird.
Sogar im unhörbaren Diskant ist der Duplex-Abschnitt - um die Impedanz zu optimieren - in gewissem Sinne \enquote{gestimmt}, d.h. die Aliquotleiste ist so angebracht, daß die Länge des Duplex-Abschnitts der Saite eine harmonische Länge des klingenden Abschnitts der Saite ist (\enquote{aliquot} bedeutet \enquote{ohne Rest teilend}).
Wenn der Ton des Duplex-Abschnitts hörbar wäre, dann müßte der Duplex-Abschnitt genauso sorgfältig gestimmt werden wie der klingende Abschnitt der Saite.
Für das Anpassen der Impedanz muß das Stimmen jedoch nur annähernd genau sein, was in der Praxis auch der Fall ist.
Manche Hersteller haben diesen Mythos des Duplex-Abschnitts ins Lächerliche gesteigert, indem sie auf der Seite des Stimmwirbels einen zweiten Duplex-Abschnitt vorsehen.
Da der Hammer auf diesen Bereich (wegen des festen Capotasto) nur Zugkräfte übertragen kann, kann dieser Bereich der Saite nicht schwingen, um einen Klang zu erzeugen.
Folglich gibt praktisch kein Hersteller ausdrücklich an, daß der nicht klingende Abschnitt auf der Seite der Stimmwirbel gestimmt werden soll.


\label{c2_5h}
\subsubsection{Wiegen im Diskant}
\label{c2_5_disk}

\textbf{Die am schwierigsten zu stimmenden Noten sind die höchsten.}
Hier brauchen Sie beim Bewegen der Saiten eine unglaubliche Genauigkeit, und die Schwebungen sind schwer zu hören.
Anfänger können leicht den Bezugspunkt verlieren und es schwer haben, den Weg zurück zu finden.
Ein Vorteil der Notwendigkeit für solch kleine Bewegungen ist, daß Sie nun die wiegende Bewegung des Wirbels für das Stimmen benutzen können.
Da die Bewegung so klein ist, schädigt das Wiegen des Wirbels nicht den Stimmstock.
\textbf{Um den Wirbel zu wiegen, plazieren Sie den Stimmhammer parallel zu den Saiten, und lassen Sie ihn auf die Saiten zeigen (weg von Ihnen selbst).
Um \textit{höher} zu stimmen, ziehen Sie am Hammer nach oben, und um \textit{tiefer} zu stimmen, drücken Sie nach unten.}
Stellen Sie zuerst sicher, daß der Stimmpunkt nahe dem Mittelpunkt der wiegenden Bewegung ist.
Wenn er es nicht ist, dann drehen Sie den Wirbel so, daß er es ist.
Da diese Drehung viel größer ist als jene, die für das endgültige Stimmen benötigt wird, ist es nicht schwierig, aber denken Sie daran, den Wirbel richtig einzustellen.
Es ist besser, wenn der Stimmpunkt vor der Mitte ist (nach der Saite zu), aber ihn zu weit nach vorne zu bringen, würde das Risiko bedeuten, den Stimmstock zu beschädigen, wenn man versucht \textit{tiefer} zu stimmen.
Beachten Sie, daß \textit{höher} zu stimmen für den Stimmstock nicht so schädlich ist wie \textit{tiefer} zu stimmen, weil der Wirbel bereits gegen die Vorderseite des Lochs gedrückt ist.
 

\label{c2_5i}
\subsubsection{Grollen im Baß}
\label{c2_5_bass}

\textbf{Die tiefsten Baßsaiten sind (nach den höchsten Noten) jene, die am zweitschwierigsten zu stimmen sind.}
Diese Saiten erzeugen einen Ton, der zum größten Teil aus höheren Obertönen besteht.
Nahe dem Stimmpunkt sind die Schwebungen so langsam und leise, daß sie nur schwer zu hören sind.
Manchmal kann man sie besser \enquote{hören}, indem man sein Knie gegen das Klavier drückt, um die Vibrationen zu fühlen, als zu versuchen, sie mit den Ohren zu hören, besonders im einsaitigen Abschnitt.
Sie können das Unisono-Stimmen nur bis zum letzten zweisaitigen Abschnitt hinunter üben.
\textbf{Stellen Sie fest, ob sie die hochtönenden, metallischen, klingelnden Schwebungen erkennen können, die in diesem Abschnitt vorherrschend sind.}
Versuchen Sie, diese zu eliminieren, und sehen Sie, ob Sie ein wenig verstimmen müssen, um sie zu eliminieren.
Wenn Sie diese hohen, klingelnden Schwebungen hören können, bedeutet das, daß Sie auf dem richtigen Weg sind.
Machen Sie sich keine Gedanken, wenn Sie sie zunächst nicht einmal erkennen können - von Anfängern wird das nicht erwartet.
 

\label{c2_5j}
\subsubsection{Harmonisches Stimmen}
\label{c2_5_harm}

Wenn Sie mit Ihrer Fähigkeit unisono zu stimmen zufrieden sind, fangen Sie an, das Stimmen von Oktaven zu üben.
Nehmen Sie eine Oktave nahe des mittleren C und dämpfen Sie die beiden oberen Saiten jeder Note, indem Sie einen Keil zwischen ihnen einfügen.
Stimmen Sie die obere Note nach der Note eine Oktave unterhalb davon und umgekehrt.
Beginnen Sie wie beim Unisono nahe dem mittleren C, arbeiten Sie sich dann bis zu den höchsten Noten im Diskant vor, und üben Sie dann im Baß.
Wiederholen Sie die gleiche Übung mit den Quinten, Quarten und den großen Terzen.

\textbf{Nachdem Sie perfekte Harmonien stimmen können, versuchen Sie sie zu verstimmen, um festzustellen, ob Sie die zunehmenden Schwebungsfrequenzen hören können, wenn Sie ganz leicht von der perfekten Stimmung abweichen.}
Versuchen Sie, verschiedene Schwebungsfrequenzen zu identifizieren, insbesondere 1 bps (beats per second = Schwebungen je Sekunde) und 10 bps, indem Sie Quinten benutzen.
Diese Fertigkeiten werden sich später als nützlich erweisen.
 

\label{c2_5k}
\subsubsection{Was ist Streckung?}
\label{c2_5_stre}

Harmonisches Stimmen ist immer mit einem Phänomen verbunden, das Streckung genannt wird.
Harmonische Obertöne in Klaviersaiten sind niemals exakt, weil reale Saiten, die an realen Enden befestigt sind, sich nicht wie ideale mathematische Saiten verhalten.
Diese Eigenschaft der nicht exakten Obertöne nennt man Inharmonizität.
Die Differenz zwischen den tatsächlichen und den theoretischen harmonischen Frequenzen nennt man Streckung.
Experimentell findet man, daß die meisten harmonischen Obertöne im Vergleich zu ihren idealen theoretischen Werten \textit{höher} sind, obwohl es ein paar geben kann, die \textit{tiefer} sind.

Gemäß eines Untersuchungsergebnisses (Young, 1952) wird Streckung durch Inharmonizität verursacht, die aus der Steifheit der Saiten resultiert.
Ideale mathematische Saiten haben eine Steifheit von Null.
Steifheit ist eine extrinsische Eigenschaft - sie hängt von den Abmessungen des Drahtes ab.
Wenn diese Erklärung richtig ist, dann muß Streckung ebenfalls extrinsisch sein.
Wenn eine bestimmte Art Stahl vorgegeben ist, dann ist der Draht steifer, wenn er dicker oder kürzer ist.
Eine Konsequenz aus dieser Abhängigkeit von der Steifheit ist eine Steigerung der Frequenz mit der Zahl der \hyperref[moden]{Schwingungsmoden}; d.h. der Draht erscheint bei harmonischen Obertönen mit kürzeren Wellenlängen steifer.
Steifere Drähte vibrieren schneller, weil sie zusätzlich zur Saitenspannung eine weitere Rückstellkraft haben.
Diese Inharmonizität wurde mit einer Genauigkeit von einigen Prozent berechnet, so daß die Theorie richtig erscheint, und dieser einzelne Mechanismus scheint für den größten Teil der beobachteten Streckung verantwortlich zu sein.

Diese Berechnungen zeigen, daß die Streckung für die zweite Schwingungsmode bei C4 ungefähr 1,2 Cent beträgt und sich ungefähr alle 8 Halbtöne bei höheren Frequenzen verdoppelt (C4 = mittleres C, die erste Mode ist die tiefste oder Grundfrequenz, ein Cent ist ein hundertstel Halbton, und es gibt 12 Halbtöne in einer Oktave).
Die Streckung wird für tiefere Noten kleiner, besonders unterhalb von C3, weil die drahtumwickelten Saiten ziemlich flexibel sind.
Die Streckung nimmt schnell mit steigender Modenzahl zu und nimmt mit steigender Saitenlänge noch schneller ab.
Prinzipiell ist die Streckung bei größeren Klavieren kleiner und bei Klavieren mit geringerer Spannung größer, wenn Saiten mit dem gleichen Durchmesser benutzt werden.
Streckung führt zu Problemen beim Entwerfen von Tonleitern, weil abrupte Veränderungen des Saitentyps, Saitendurchmessers, der Länge, usw. eine diskontinuierliche Veränderung in der Streckung erzeugen.
Obertöne sehr hoher Moden bereiten, wenn Sie ungewöhnlich laut sind, wegen ihrer großen Streckung Probleme beim Stimmen - ihre Schwebungen herauszustimmen könnte die unteren, wichtigeren Obertöne hörbar aus der Stimmung bringen.

Da größere Klaviere oft eine geringere Streckung haben, aber auch dazu neigen, besser zu klingen, könnte man daraus schließen, daß eine kleinere Streckung besser ist.
Die Differenz der Streckung ist jedoch im allgemeinen gering, und die Klangqualität eines Klaviers wird zu einem großen Teil von anderen Eigenschaften als der Streckung kontrolliert.

Beim harmonischen Stimmen stimmt man z.B. die Grundfrequenz oder einen Oberton der oberen Note nach einem höheren Oberton der tieferen Note.
Die resultierende neue Note ist kein genaues Vielfaches der tieferen Note, sondern ist um den Betrag der Streckung \textit{höher}.
Das interessante an der Streckung ist, daß eine Tonleiter mit Streckung \enquote{lebhaftere} Musik erzeugt als eine ohne!
Das hat einige Stimmer veranlaßt, mit doppelten Oktaven statt mit einzelnen Oktaven zu stimmen, was die Streckung vergrößert.

Der Betrag der Streckung ist für jedes Klavier einzigartig und, in Wahrheit, einzigartig für jede Note des Klaviers.
Moderne elektronische Stimmhilfen sind genügend mächtig, um die Streckung für alle gewünschten Noten eines bestimmten Klaviers aufzuzeichnen.
Stimmer mit elektronischen Stimmhilfen können auch die durchschnittliche Streckung oder die Streckungsfunktion für jedes Klavier berechnen und das Klavier entsprechend stimmen.
Tatsächlich gibt es anekdotenhafte Berichte über Pianisten, die eine Streckung weit über der natürlichen Streckung des Klaviers wünschen.
Beim auralen Stimmen wird die Streckung natürlich und genau berücksichtigt.
Deshalb muß der Stimmer, obwohl die Streckung ein wichtiger Aspekt des Stimmens ist, nichts besonderes tun, um die Streckung einzubeziehen, wenn man nur die natürliche Streckung des Klaviers möchte.
 

\label{c2_5l}
\subsubsection{Präzision, Präzision, Präzision}
\label{c2_5_prae} 

\textbf{Das, worum es beim Stimmen geht, ist Präzision.}
Alle Stimmverfahren sind so angeordnet, daß man nacheinander die erste Note nach einer Stimmgabel stimmt, die zweite nach der ersten, usw.
Deshalb werden sich eventuelle Fehler schnell aufaddieren.
Tatsächlich wird ein Fehler an einem Punkt oft einige nachfolgende Schritte unmöglich machen.
Das geschieht, weil man auf den kleinsten Hinweis auf eine Schwebung hört, und wenn die Schwebungen in einer Note nicht vollständig eliminiert wurden, kann man sie nicht benutzen, um eine andere zu stimmen, weil diese Schwebungen klar zu hören sein werden.
Das wird bei Anfängern, bevor sie gelernt haben, wie präzise man sein muß, tatsächlich oft geschehen.
Wenn das geschieht, hört man Schwebungen, die man nicht eliminieren kann.
Gehen Sie in diesem Fall zu Ihrer Referenznote zurück, und stellen Sie fest, ob sie die gleiche Schwebung hören; wenn das so ist, ist dort der Ursprung Ihres Problems - beseitigen Sie es.

\textbf{Der beste Weg, die Präzision sicherzustellen, ist, die Stimmung zu prüfen.}
Fehler treten auf, weil jede Saite anders ist und Sie nie sicher sind, daß die Schwebung, die Sie hören, jene ist, nach der Sie suchen; das gilt besonders für Anfänger.
Ein weiterer Faktor ist, daß Sie die Schwebungen pro Sekunde (bps) zählen müssen, und Ihre Vorstellung von sagen wir 2 bps wird an verschiedenen Tagen oder zu verschiedenen Zeiten desselben Tags unterschiedlich sein, bis Sie sich diese \enquote{Schwebungsgeschwindigkeiten} gut gemerkt haben.
Wegen der entscheidenden Wichtigkeit der Präzision zahlt es sich aus, jede gestimmte Note zu prüfen.
Das gilt besonders, wenn Sie \enquote{\hyperref[c2_6]{die Bezugsnoten einstellen}}, was unten erklärt wird.
Unglücklicherweise ist die Note genauso schwierig zu prüfen, wie sie richtig zu stimmen ist; d.h. ein Person, die nicht hinreichend genau stimmen kann, ist üblicherweise unfähig, eine sinnvolle Prüfung durchzuführen.
Außerdem funktioniert das Prüfen nicht, wenn die Stimmung weit genug daneben ist.
Deshalb \textbf{habe ich Methoden gewählt, die ein Minimum an Prüfungen benutzen.}
Die resultierende Stimmung wird für die Gleichschwebende Temperatur zunächst nicht sehr gut sein.
Die \hyperref[c2_6_kirn]{Kirnberger-Stimmung} (s.u.) ist einfacher akkurat zu stimmen.
Auf der anderen Seite können Anfänger ohnehin keine guten Stimmungen erzeugen, unabhängig davon, welche Methoden Sie benutzen.
Zumindest werden die Verfahren, die unten vorgestellt werden, eine Stimmung bieten, die keine Katastrophe sein sollte und die besser wird, sobald sich Ihre Fertigkeiten verbessern.
\textbf{Tatsächlich ist das wahrscheinlich der schnellste Weg zum Lernen.}
Nachdem Sie sich genug verbessert haben, können Sie die Prüfungsverfahren untersuchen, wie jene, die bei Reblitz oder in \enquote{Tuning} von Jorgensen angegeben sind.
 


<!-- c26.html -->

\subsection{Stimmverfahren}
\label{c2_6}

\subsubsection{Einleitung}
\label{c2_6a}

Stimmen besteht aus dem \enquote{Einstellen der Bezugsnoten} in einer Oktave in der Nähe des mittleren C und daraus, diese Oktave in geeigneter Weise auf alle anderen Tasten zu \enquote{kopieren}.
Sie werden verschiedene harmonische Stimmungen benötigen, um die Bezugsnoten einzustellen, und zunächst wird nur die mittlere Saite jeder Note der \enquote{Bezugsoktave} gestimmt.
Das \enquote{Kopieren} wird durch das Stimmen in Oktaven durchgeführt.
Wenn eine Saite jeder Note auf diese Art gestimmt ist, werden die restlichen Saiten jeder Note unisono gestimmt.

Beim Einstellen der Bezugsnoten müssen wir uns entscheiden, welche Temperatur wir benutzen möchten.
Wie oben in \hyperref[c2_2]{Abschnitt 2} erklärt wurde, sind die meisten Klaviere heutzutage auf \hyperref[et1]{gleichschwebende Temperatur (ET)} gestimmt, aber die historischen Temperaturen, insbesondere die \hyperref[c2_2_wtk2]{Wohltemperierten Stimmungen (WT)} erfreuen sich zunehmender Beliebtheit.
Deshalb habe ich ET und eine WT, \hyperref[c2_6_kirn]{Kirnberger II (K-II)}, für dieses Kapitel ausgewählt.
K-II ist eine der am leichtesten zu stimmenden Temperaturen; deshalb werden wir diese zuerst ansehen.
Die meisten, die nicht mit den verschiedenen Temperaturen vertraut sind, werden zunächst keinen Unterschied zwischen ET und K-II bemerken; sie werden beide im Vergleich zu einem verstimmten Klavier hervorragend klingen.
Auf der anderen Seite sollten die meisten Klavierspieler einen deutlichen Unterschied hören und in der Lage sein, eine Meinung oder eine Vorliebe zu entwickeln, wenn man ihnen bestimmte Musikstücke vorspielt und die Unterschiede aufzeigt.
Der einfachste Weg für Außenstehende, sich die Unterschiede anzuhören, ist, ein modernes elektronisches Klavier zu benutzen, das alle diese Temperaturen eingebaut hat, und dasselbe Stück mit jeder der Temperaturen zu spielen.
Benutzen Sie als ein leichtes Teststück z.B. den ersten Satz von Beethovens Mondschein-Sonate; als ein schwierigeres Stück können Sie den dritten Satz seiner Waldstein-Sonate benutzen.
Probieren Sie auch ein paar Ihrer Lieblingsstücke von Chopin aus.
Mein Vorschlag für einen Anfänger ist, zuerst K-II zu lernen, so daß man ohne zu viele Schwierigkeiten anfangen kann, und dann ET zu lernen, wenn man schwierigere Aufgaben in Angriff nehmen kann.
Ein Nachteil dieses Plans ist, daß man eventuell K-II so sehr gegenüber ET bevorzugt, daß man sich nie dazu entschließt, ET zu lernen.
Wenn man sich an K-II gewöhnt hat, wird ET ein wenig ungenügend oder \enquote{schmutzig} klingen.
Man kann jedoch nicht wirklich als Stimmer angesehen werden, bevor man nicht ET stimmen kann.
Auch gibt es viele WTs, auf die Sie vielleicht einen Blick werfen möchten, die in verschiedener Hinsicht K-II überlegen sind.

WT-Stimmungen sind wünschenswert, weil sie perfekte Harmonien haben, die der Kern der Musik sind.
Sie haben jedoch einen großen Nachteil.
Weil die perfekten Harmonien so schön sind, treten die Dissonanzen in den \enquote{Wolfs}-Tonleitern hervor und sind sehr unangenehm.
Nicht nur das, sondern jede Saite, die ein wenig aus der Stimmung ist, ist sofort zu erkennen.
Deshalb erfordern WT-Stimmungen ein viel häufigeres Stimmen als ET.
Man könnte meinen, daß ein leichtes Verstimmen der Unisono-Saiten bei ET genauso unangenehm wäre, aber offenbar sind, wenn die Intervalle wie bei der ET aus der Stimmung sind, die geringfügigen Abweichungen in der Stimmung der Unisono-Saiten bei ET weniger wahrnehmbar.
Deshalb kann für Klavierspieler, die ein sensibles Gehör für das Stimmen haben, WT ziemlich unangenehm sein, solange sie ihr Klavier nicht selbst stimmen können.
Das ist ein wichtiger Punkt, weil die meisten Klavierspieler, die die Vorteile der WT hören können, empfindlich auf die Stimmung reagieren.
Die Erfindung des selbststimmenden Klaviers kann vielleicht der Retter der WT sein, weil das Klavier immer richtig gestimmt sein wird.
Deshalb wird WT eventuell nur durch elektronische und selbststimmende Klaviere (wenn sie verfügbar werden - s. \hyperref[c1iv6h]{Abschnitt IV.6h \enquote{Die Zukunft des Klaviers}}) eine breite Zustimmung finden.

Sie können das Stimmen in ET überall beginnen, aber die meisten Stimmer benutzen die Stimmgabel A440 um anzufangen, weil Orchester im allgemeinen nach A440 stimmen.
Das Ziel bei K-II ist, C-Dur und so viele Tonarten \enquote{in der Nähe} wie möglich rein zu haben (mit reinen Intervallen), weshalb das Stimmen mit dem mittleren C (C4 = 261,6 - die meisten Stimmer benutzen die C523,3-Stimmgabel um das mittlere C zu stimmen) begonnen wird.
Nun ist das aus K-II resultierende A, wenn man vom richtigen C aus stimmt, nicht das A440.
Deshalb benötigen Sie zwei Stimmgabeln (A und C), um sowohl ET als auch K-II stimmen zu können.
Alternativ können Sie nur mit einer C-Gabel beginnen und fangen das Stimmen in ET bei C an.
Zwei Stimmgabeln zu haben ist ein Vorteil, denn egal ob Sie von C oder von A aus starten, können Sie sich selbst überprüfen, wenn Sie bei ET bei der anderen Note ankommen.


\label{c2_6b}
\subsubsection{Das Klavier nach der Stimmgabel stimmen}
\label{c2_6_gabe}

Einer der schwierigsten Schritte beim Stimmvorgang ist das Stimmen des Klaviers nach der Stimmgabel.
Diese Schwierigkeit hat zwei Ursachen:

\begin{enumerate}[label={\arabic*.}] 
 \item Die Stimmgabel hat eine andere - üblicherweise kürzere - Aushaltezeit (Sustain) für den Ton als das
 Klavier, so daß die Gabel ausklingt, bevor Sie einen genauen Vergleich machen können.
 \item Die Gabel erzeugt eine reine Sinuswelle ohne die lauten Obertöne der Klaviersaiten.
 \end{enumerate}
Deshalb kann man keine Schwebungen mit höheren Obertönen benutzen, um die Genauigkeit des Stimmens zu erhöhen, wie man es mit zwei Klaviersaiten tun kann.
Ein Vorteil von elektronischen Stimmgeräten ist, daß sie so programmiert werden können, daß sie Referenztöne mit Rechteckschwingungen liefern, die eine große Anzahl von höheren harmonischen Obertönen beinhalten.
Diese hohen harmonischen Obertöne (die notwendig sind, um die scharfen Ecken der Rechteckschwingungen zu erzeugen) sind für eine höhere Stimmgenauigkeit nützlich.
Wir müssen deshalb diese beiden Probleme lösen, damit wir das Klavier genau nach der Stimmgabel stimmen können.

Beide Schwierigkeiten können beseitigt werden, wenn wir das Klavier als Stimmgabel benutzen können und diesen Übergang von der Stimmgabel zum Klavier durchführen, indem wir einige hohe harmonische Obertöne des Klaviers benutzen.
Finden Sie, um diesen Übergang zu erreichen, eine Note innerhalb der gedämpften Noten, die laute Schwebungen mit der Gabel erzeugt.
Wenn Sie keine Note finden können, benutzen Sie die Note einen Halbton höher oder tiefer; benutzen Sie z.B. für eine Stimmgabel A das Ab oder A\# auf dem Klavier.
Wenn diese Schwebungsfrequenzen etwas zu hoch sind, versuchen Sie die gleichen Noten eine Oktave tiefer.
Stimmen Sie nun das A auf dem Klavier so, daß es Schwebungen der gleichen Frequenz mit diesen Bezugsnoten erzeugt (Ab, A\#, oder jede andere Note, die Sie gewählt haben).
Die beste Möglichkeit, die Stimmgabel zu hören, ist, sie \hyperref[c2_3_gabel]{wie oben beschrieben} gegen den Tragus zu drücken oder sie auf irgendeine große, harte, flache Oberfläche zu drücken.
 

\label{c2_6c}
\subsubsection{Kirnberger II}
\label{c2_6_kirn}

\begin{itemize} 
 \item Dämpfen Sie alle Nebensaiten von F3 bis F4.
 \item Stimmen Sie C4 (das mittlere C) nach der Gabel.
 \item Benutzen Sie dieses C4, um G3 (Quarte), E4 (Terz), F3 (Quinte), und F4 (Quarte) zu stimmen.
 \item Benutzen Sie G3, um D4 (Quinte) und H3 (Terz) zu stimmen.
 \item Benutzen Sie H3, um F\#3 (Quarte) zu stimmen.
 \item Benutzen Sie F\#3, um Db4 (Quinte) zu stimmen.
 \item Benutzen Sie F3, um B3 (Quarte) zu stimmen.
 \item Benutzen Sie B3, um Eb4 (Quarte) zu stimmen.
 \item Benutzen Sie Eb4, um Ab3 (Quinte) zu stimmen.
 \item Alle Stimmungen bis hierhin sind \textit{rein}.<br>
Stimmen Sie nun A3 so, daß die Schwebungsfrequenzen von F3-A3 und A3-D4 die gleichen sind.
 \end{itemize}

Sie sind fertig mit dem Einstellen der Bezugsnoten!


\label{c2_6_kirn2}

Stimmen Sie nun in \textit{reinen} Oktaven aufwärts, bis zu den höchsten Noten.
Stimmen Sie dann abwärts, bis zu den tiefsten Noten.
Beginnen Sie dabei mit der Bezugsoktave als Referenz.
Stimmen Sie bei all diesen Stimmungen nur eine neue Oktavsaite, während Sie die anderen dämpfen.
Stimmen Sie dann die eine bzw. zwei verbleibenden Saiten mit der neu gestimmten Saite \hyperref[c2_5_unis]{unisono}.

Das ist ein Moment, in dem Sie die Regel \enquote{Stimmen Sie eine Saite nach einer anderen.} brechen sollten.
Wenn z.B. Ihre Referenznote eine (gestimmte) 3-saitige Note ist, benutzen Sie sie wie sie ist, ohne eine Saite davon zu dämpfen.
Das dient als ein Test der Qualität Ihres Stimmens.
Wenn es Ihnen schwer fällt, sie zu benutzen, um eine neue einzelne Saite zu stimmen, dann war u.U. Ihr Unisono-Stimmen der Referenznote nicht genügend genau, und Sie sollten zurückgehen und sie bereinigen.
Wenn Sie auch nach erheblicher Mühe nicht 3 gegen 1 stimmen können, haben Sie selbstverständlich keine andere Chance, als zwei der drei Saiten zu dämpfen, damit Sie vorwärtskommen.
Sie gefährden jedoch die Qualität des Stimmens.
Wenn alle Noten im Diskant und Baß fertig sind, dann sind die einzigen ungestimmten Noten jene, die Sie für das Einstellen der Bezugsnoten gedämpft haben.
Stimmen Sie diese - mit der tiefsten Note beginnend - unisono mit ihrer mittleren Saite, indem Sie vom Filz jeweils eine Schleife wegziehen.
 

\label{c2_6d}
\label{c2_6_et}

<h3><br>6d. Gleichschwebende Temperatur \hyperref[et]{(gleichstufige Temperatur, gleichmäßige
 Temperatur)}</h3>

Ich zeige hier das leichteste, angenäherte Verfahren für die gleichschwebende Temperatur.
Genauere Algorithmen kann man in der Literatur finden (Reblitz, Jorgensen).
Kein professioneller Stimmer, der etwas auf sich hält, würde dieses Schema benutzen; wenn man jedoch gut darin wird, kann man eine annehmbare gleichschwebende Temperatur erzeugen.
Bei einem Anfänger werden die vollständigeren und präziseren Schemata nicht notwendigerweise zu besseren Ergebnissen führen.
Mit den komplexeren Methoden kann ein Anfänger schnell durcheinander kommen, ohne eine Vorstellung davon zu haben, was er falsch gemacht hat.
Mit der hier gezeigten Methode kann man schnell die Fähigkeit entwickeln, herauszufinden, was man falsch gemacht hat:

\begin{itemize} 
 \item Dämpfen Sie die Nebensaiten von G3 bis C\#5.
 \item Stimmen Sie A4 nach der A440 Gabel.
 \item Stimmen Sie A3 nach A4.
 \item Stimmen Sie dann mit verkürzten Quinten von A3 aus aufwärts, bis Sie nicht mehr weiter aufwärts gehen können, ohne den gedämpften Bereich zu verlassen, dann eine Oktave tiefer, und wiederholen Sie dieses \enquote{aufwärts in Quinten und eine Oktave abwärts}-Verfahren bis Sie zu A4 kommen.
Sie beginnen z.B. mit einer verkürzten A3-E4, dann einer verkürzten E4-H4.
Die nächste Quinte würde Sie über die höchste gedämpfte Note, C\#5, hinausführen; stimmen Sie deshalb eine Oktave abwärts, H4-H3.
 \end{itemize}

Alle Oktaven sind selbstverständlich \textit{rein}.
Die verkürzten Quinten sollten am unteren Ende des gedämpften Bereichs mit etwas weniger als 1 Hz schweben und ungefähr 1,5 Hz am oberen Ende.
Die Schwebungsfrequenzen der Quinten zwischen dieser oberen und unteren Grenze sollten langsam mit zunehmender Tonhöhe steigen.

Wenn Sie in Quinten aufwärts gehen, stimmen Sie von \textit{rein} zu \textit{tiefer}, um eine verkürzte Quinte zu erzeugen.
Deshalb können Sie mit \textit{rein} beginnen und \textit{tiefer} stimmen, um gleichzeitig die Schwebungsfrequenz auf den gewünschten Wert zu steigern und \hyperref[c2_5_wirb]{den Wirbel richtig einzustellen}.
Wenn Sie alles perfekt getan haben, sollte das letzte D4-A4 ohne neu zu stimmen eine verkürzte Quinte mit einer Schwebungsfrequenz von 1 Hz sein.
Dann sind Sie fertig.
Sie haben gerade einen \enquote{Quintenzirkel} beendet.
Das Wunder des Quintenzirkels ist, daß er jede Note einmal stimmt, ohne irgendeine in der A3-A4-Oktave zu überspringen!

Wenn die abschließende D4-A4 nicht richtig ist, haben Sie irgendwo einen Fehler begangen.
Kehren Sie in diesem Fall die Prozedur um; beginnen Sie bei A4, gehen Sie in verkürzten Quinten abwärts und in Oktaven aufwärts, bis Sie A3 erreichen, wobei die abschließende A3-E4 eine verkürzte Quinte mit einer Schwebungsfrequenz von etwas weniger als 1 Hz sein sollte.
Um in Quinten abwärts zu gehen, erzeugen Sie eine verkürzte Quinte, indem Sie von \textit{rein} nach \textit{höher} stimmen.
Dieser Schritt des Stimmens wird jedoch den Wirbel nicht einstellen.
Um den Wirbel korrekt einzustellen, müssen Sie deshalb zunächst \textit{zu hoch} stimmen und dann die Schwebungsfrequenz auf den gewünschten Wert vermindern.
Deshalb ist in Quinten abwärts zu gehen eine schwierigere Operation als in Quinten aufwärts zu gehen.

Eine alternative Methode ist, mit A anzufangen, mit Quinten aufwärts bis zum C zu stimmen und dieses C mit einer Stimmgabel zu prüfen.
Wenn Ihr C \textit{zu hoch} ist, waren Ihre Quinten nicht ausreichend verkürzt und umgekehrt.
Eine weitere Variation ist, in Quinten von A3 aus etwas mehr als die Hälfte der Strecke aufwärts zu stimmen und dann von A4 bis zur letzten Note, die Sie beim Aufwärtsgehen gestimmt haben, abwärts zu stimmen.

Wenn die Bezugsnoten eingestellt sind, fahren Sie wie oben im Abschnitt über \hyperref[c2_6_kirn2]{Kirnberger II} beschrieben fort.



<!-- c27.html -->

\subsection{Kleinere Reparaturen durchführen}
\label{c2_7}

Wenn man mit dem Stimmen angefangen hat, muß man zwangsläufig kleinere Reparaturen und ein paar Wartungsarbeiten durchführen.
 

\label{c2_7a}
\subsubsection{Intonieren der Hämmer}
\label{c2_7_hamm}

\textbf{Ein verbreitetes Problem, das man bei vielen Klavieren findet, sind verdichtete Hämmer.
Ich bringe diesen Punkt zur Sprache, weil der Zustand der Hämmer für die richtige Entwicklung der Klaviertechnik und der Fertigkeiten für das Auftreten viel wichtiger ist, als vielen Menschen bewußt ist.}
Zahlreiche Stellen in diesem Buch weisen auf die Wichtigkeit des musikalischen Übens für das Erwerben der Technik hin.
Man kann aber nicht musikalisch spielen, wenn die Hämmer ihre Aufgabe nicht erfüllen können - ein entscheidender Punkt, der sogar von vielen Stimmern übersehen wird (oftmals weil sie fürchten, daß die zusätzlichen Kosten die Kunden vergraulen würden).
Bei einem Flügel ist, daß man es für notwendig hält, den Deckel zumindest teilweise zu schließen, um leise Passagen zu spielen, ein sicheres Zeichen für verdichtete Hämmer.
Ein weiteres sicheres Zeichen ist, daß man dazu neigt, das Dämpferpedal zu Hilfe zu nehmen, um leise zu spielen.
Verdichtete Hämmer erzeugen entweder einen lauten Ton oder überhaupt keinen.
Jede Note neigt dazu, mit einem lästigen, perkussiven Schlag zu beginnen, der zu stark ist, und der Klang ist übermäßig hell.
Es sind diese perkussiven Schläge, die für das Gehör des Stimmers so schädlich sind.
Ein richtig intoniertes Klavier erlaubt die Kontrolle über den ganzen Dynamikbereich und erzeugt einen gefälligeren Klang.

Lassen Sie uns zunächst sehen, wie ein verdichteter Hammer zu so extremen Resultaten führen kann.
Wie können kleine, leichte Hämmer laute Töne erzeugen, wenn sie mit relativ geringer Kraft auf eine Saite treffen, die unter einer solch hohen Spannung steht?
Wenn man versuchen würde, die Saite herunterzudrücken oder zu zupfen, müßte man eine ziemlich große Kraft aufwenden, um nur einen kleinen Ton zu erzeugen.
Die Antwort liegt in einem unglaublichen Phänomen, das auftritt, wenn straff gespannte Saiten im rechten Winkel angeschlagen werden.
\textbf{Es stellt sich heraus, daß die vom Hammer erzeugte Kraft im Moment des Aufpralls theoretisch unendlich ist!}
Diese fast unendliche Kraft ist es, was den leichten Hammer in die Lage versetzt, praktisch jede erreichbare Spannung der Saite zu überwinden und sie zum Schwingen zu bringen.

Hier ist die Berechnung dieser Kraft.
Stellen Sie sich vor, daß der Hammer an seinem höchsten Punkt ist, nachdem er die Saite angeschlagen hat (Flügel).
Die Saite bildet zu diesem Zeitpunkt mit ihrer ursprünglichen horizontalen Position ein Dreieck (das ist nur eine idealisierte Näherung, s.u.).
Die kürzeste Seite dieses Dreiecks ist der Abstand zwischen der Agraffe und dem Aufschlagspunkt des Hammers.
Die zweitkürzeste Seite ist die vom Hammer bis zum Steg.
Die längste ist die ursprüngliche horizontale Lage der Saite, eine gerade Linie vom Steg zur Agraffe.
Wenn wir nun eine vertikale Linie vom Aufschlagspunkt des Hammers nach unten zur ursprünglichen Saitenposition ziehen, erhalten wir zwei aneinanderliegende rechtwinklige Dreiecke.
Das sind zwei extrem spitze rechtwinklige Dreiecke, die sehr kleine Winkel an der Agraffe und dem Steg haben; wir werden diese kleinen Winkel \enquote{theta} nennen.

Das einzige, das wir zu dieser Zeit kennen, ist die Kraft des Hammers, aber das ist nicht die Kraft, die die Saite bewegt, weil der Hammer die Saitenspannung überwinden muß, bevor die Saite nachgibt.
D.h. die Saite kann sich nicht aufwärts bewegen, solange sie nicht länger werden kann.
Das ist verständlich, wenn man sich die beiden oben beschriebenen rechtwinkligen Dreiecke ansieht.
Die Saite hatte, bevor der Hammer auftraf, die Länge der langen Katheten der rechtwinkligen Dreiecke, aber nach dem Auftreffen bildet die Saite die Hypotenusen, welche länger sind.
D.h., wenn die Saite absolut unelastisch wäre und die Enden der Saiten wären fest fixiert, könnte keine noch so große Hammerkraft die Saite dazu bringen, sich zu bewegen.

Es ist eine einfache Angelegenheit, mit Vektordiagrammen zu zeigen, daß die \textit{zusätzliche} Spannungskraft F (zusätzlich zu der ursprünglichen Saitenspannung), die vom Hammeraufschlag erzeugt wird, durch f = F * sin(theta) gegeben ist, wobei f die Kraft des Hammers ist.
Es ist egal, welches rechtwinklige Dreieck wir für diese Berechnung verwenden (das auf der Seite des Stegs oder das auf der Seite der Agraffe).
Deshalb ist die Saitenspannung F = f / sin(theta).
Im ersten Moment des Auftreffens ist theta = 0, und deshalb F = unendlich!
Das geschieht, weil sin(0) = 0.
Selbstverständlich kann F nur unendlich werden, wenn die Saite sich nicht strecken kann und sich nichts anderes bewegt.
In der Realität geschieht folgendes: F steigt in Richtung unendlich an, irgend etwas gibt nach (die Saite streckt sich, der Steg bewegt sich, usw.), so daß der Hammer anfängt, die Saite zu bewegen, und theta größer als Null wird, was F endlich werden läßt.

Diese Vervielfachung der Kraft erklärt, warum ein kleines Kind auf einem Klavier trotz der mehreren hundert Pfund Spannung auf den Saiten einen ziemlich lauten Ton erzeugen kann.
Es erklärt auch, warum eine normale Person eine Saite beim Klavierspielen zerbrechen kann, besonders wenn die Saite alt ist und ihre Elastizität verloren hat.
Der Mangel an Elastizität führt dazu, daß F weitaus mehr ansteigt, als wenn die Saite elastischer ist, die Saite kann sich nicht strecken, und theta bleibt nahe Null.
Diese Situation wird außerordentlich verschärft, wenn der Hammer ebenfalls verdichtet ist, so daß er eine große, flache, harte Kerbe hat, die die Saite berührt.
In diesem Fall gibt die Oberfläche des Hammers nicht nach, und die anfängliche Kraft \enquote{f} in der obigen Gleichung wird sehr groß.
Da das bei einem verdichteten Hammer alles nahe theta = 0 geschieht, wird der Vervielfachungsfaktor der Kraft ebenfalls vergrößert.
Das Resultat ist eine gebrochene Saite.

Die obige Berechnung ist eine starke Vereinfachung und nur qualitativ richtig.
In Wirklichkeit sendet ein Hammerschlag zunächst eine wandernde Welle in Richtung des Stegs, ähnlich dem was geschieht, wenn man das Ende eines Seils nimmt und es schnalzen läßt.
Um solche Wellenformen zu berechnen, muß man bestimmte wohlbekannte Differentialgleichungen lösen.
Der Computer hat die Lösung solcher Differentialgleichungen zu einer einfachen Angelegenheit werden lassen, und realistische Berechnungen dieser Wellenformen können nun routinemäßig erfolgen.
Deshalb führen die obigen Ergebnisse, obwohl sie nicht genau sind, zu einem qualitativen Verständnis dafür, was geschieht und was die wichtigen Mechanismen und kontrollierenden Faktoren sind.

Zum Beispiel zeigt die obige Berechnung, daß es nicht die Energie der Transversalschwingung der Saite ist, sondern die Zugspannung der Saite, die für den Klang des Klaviers verantwortlich ist.
Die Energie, die durch den Hammer abgegeben wird, wird im gesamten Klavier gespeichert, nicht nur in den Saiten.
Das ist weitgehend analog zu Pfeil und Bogen: Wenn die Sehne gezogen wird, dann wird die gesamte Energie im Bogen gespeichert, nicht in der Sehne.
Und die gesamte Energie wird durch die Spannung in den Saiten übertragen.
In diesem Beispiel ist der mechanische Vorteil und die oben berechnete Vervielfachung der Kraft (nahe theta = 0) leicht zu sehen.
Es ist das gleiche Prinzip, auf dem die Harfe basiert.

Warum verdichtete Hämmer höhere harmonische Obertöne erzeugen, ist am einfachsten zu verstehen, wenn man erkennt, daß das Auftreffen in kürzerer Zeit stattfindet.
Wenn es schneller geschieht, generiert die Saite als Antwort auf das schnellere Ereignis Komponenten mit höherer Frequenz.

Die obigen Abschnitte machen deutlich, daß ein verdichteter Hammer zunächst einen großen Aufschlag auf den Saiten erzeugt, während ein richtig intonierter Hammer sanfter auf die Saite trifft und somit mehr seiner Energie an die niedrigeren Frequenzen als an die harmonischen Obertöne abgibt.
Da die gleiche Menge an Energie bei einem verdichteten Hammer in einem kürzeren Zeitraum verteilt wird, kann der anfängliche Lautstärkegrad viel höher als bei einem richtig intonierten Hammer sein, besonders bei den höheren Frequenzen.
Solche kurzen Tonspitzen können das Gehör schädigen, ohne Schmerzen zu verursachen.
Verbreitete Symptome solcher Schäden sind Tinnitus (Klingeln im Ohr) und Hörverlust bei hohen Frequenzen.
Klavierstimmer, die ein Klavier mit solchen abgenutzten Hämmern stimmen müssen, tun gut daran, Ohrenstöpsel zu tragen.
Es ist klar, daß das Intonieren der Hämmer mindestens genauso wichtig ist wie das Stimmen des Klaviers, besonders weil wir über potentielle Gehörschäden sprechen.
Ein verstimmtes Klavier mit guten Hämmern schädigt das Ohr nicht.
Trotzdem lassen viele Klavierbesitzer ihr Klavier zwar stimmen, vernachlässigen aber das Intonieren.

\textbf{Die beiden wichtigsten Prozeduren beim Intonieren sind das Wiederherstellen der Form und das Nadeln.}

Wenn der verflachte Auftreffpunkt des Hammers größer als ungefähr 1 cm ist, ist es Zeit, die Form des Hammers wieder herzustellen.
Beachten Sie, daß Sie zwischen der Länge der Saitenkerbe und dem flachgedrückten Bereich unterscheiden müssen; sogar bei gut intonierten Hämmern können die Kerben mehr als 5 mm lang sein.
Bei der endgültigen Beurteilung werden sie anhand des Klangs entscheiden müssen.
Das Formen wird durch das Schleifen der \enquote{Schultern} des Hammers erreicht, so daß er seine ursprüngliche, gerundete Form am Auftreffpunkt wiedergewinnt.
Das wird üblicherweise mit 1 Zoll\footnote{ca. 2,5 cm} breiten Streifen Sandpapier ausgeführt, die mit Leim oder doppelseitigem Klebeband auf Holz- oder Metallstreifen befestigt sind.
Sie könnten mit Papier der Körnung 80 beginnen und zum Schluß Papier der Körnung 150 verwenden.
Die Schleifbewegung muß in der Ebene des Hammers ausgeführt werden; schleifen Sie niemals quer zur Ebene.
Es besteht fast nie die Notwendigkeit den Auftreffpunkt zu schleifen.
Lassen Sie deshalb ungefähr 2 mm vom Zentrum des Auftreffpunkts unberührt.

\footnote{Eine detaillierte Beschreibung findet man z.B. in der amerikanischen Ausgabe von Reblitz auf den Seiten 137 bis 140:</font></i>

\begin{itemize} 
\item \textit{Schleifen Sie nur an der schmalen umlaufenden Fläche, die durch den Auftreffpunkt (und die Saitenkerben) hindurchgeht.}
\item \textit{Führen Sie dabei das Schleifpapier immer mit einer bogenförmigen Bewegung vom Stiel zum Auftreffpunkt hin.}
\item \textit{Der Filz muß auf beiden Seiten des Auftreffpunkts symmetrisch geformt sein, damit die Spitze des Hammers beim Auftreffen auf die Saiten nicht schrittweise in die Richtung der geringeren Unterstützung hin verformt wird und sich der Auftreffpunkt verschiebt.}
\item \textit{Die Fläche darf nicht nach der Seite abgerundet werden und muß auch rechtwinklig zu den beiden großen Seitenflächen sein, damit die Saiten einer Note gleichzeitig angeschlagen werden.}
\item \textit{Es muß genügend Filz stehenbleiben, so daß die Saite beim Anschlag nicht den Filz durchschlägt und auf das Holz des Hammers trifft.
Deshalb soll der Filz der schmalen Hämmer für die hohen Töne überhaupt nicht geschliffen werden.}
 \end{itemize}
\textit{\textbf{Also alles andere als einfach und somit nur jemandem mit wirklicher handwerklicher Begabung zu empfehlen, der stets größte Sorgfalt walten läßt!}}

Nadeln ist nicht einfach, weil die richtige Stelle zum Nadeln und die richtige Tiefe vom jeweiligen Hammer bzw. Hammerhersteller abhängen und davon, wie der Hammer ursprünglich intoniert war.
Besonders im Diskant werden beim Intonieren der Hämmer in der Fabrik oft Härter wie Lack, usw. benutzt.
Fehler beim Nadeln sind im allgemeinen nicht rückgängig zu machen.
Tiefes Nadeln ist üblicherweise an den Schultern unmittelbar außerhalb des Auftreffpunkt erforderlich.
Sehr sorgfältiges und flaches Nadeln kann im Bereich des Auftreffpunkts nötig werden.
Der Klang des Klaviers reagiert auf das flache Nadeln am Auftreffpunkt sehr empfindlich, so daß man sehr genau wissen muß, was man tut.
Wenn er richtig genadelt ist, sollte der Hammer Ihnen erlauben, sowohl sehr leise Töne zu kontrollieren als auch laute Töne zu produzieren, die nicht schrill sind.
Sie bekommen das Gefühl der totalen klanglichen Kontrolle.
Sie können nun Ihren Flügel ganz öffnen und ohne das Dämpferpedal sehr leise spielen!
Sie können auch diese lauten, reichen, respekteinflößenden Töne erzeugen.
 

\label{c2_7b}
\subsubsection{Polieren der Piloten}
\label{c2_7_pilo} 

\textbf{<i><font size=\enquote{+1} color=\enquote{navy}>[Die Beschreibung des  Aus- und Einbaus der Mechanik und der Tastatur ist z.Zt. (14.2.2005) im Originaltext relativ knapp gehalten.
Ich erinnere deshalb an dieser Stelle noch einmal an das \enquote{\hyperref[c2_1]{Achtung: ...}} am Anfang dieses Kapitels!]}}

Das Polieren der Piloten kann eine lohnende Pflegearbeit sein.
Sie müssen eventuell poliert werden, wenn Sie mehr als 10 Jahre nicht gereinigt wurden, manchmal auch früher.
Drücken Sie die Tasten langsam herunter und stellen Sie fest, ob Sie eine Reibung in der Mechanik fühlen können.
Eine reibungslose Mechanik wird sich anfühlen, als ob man mit einem geölten Finger über ein glattes Glas fährt.
Wenn Reibung vorhanden ist, fühlt es sich wie die Bewegung eines sauberen Fingers über quietschendes sauberes Glas an.
Um an die Piloten zu kommen, muß man die Mechanik von den Tasten abheben, indem man bei einem Flügel die Schrauben löst, die die Mechanik unten halten.
Bei \enquote{\hyperref[upright]{Aufrechten}} muß man im allgemeinen die Knöpfe losschrauben, die die Mechanik an ihrem Ort halten; stellen Sie sicher, daß die Pedalstangen usw. losgelöst sind.

Wenn die Mechanik entfernt wurde, können die Tasten herausgehoben werden, nachdem man die Tastendeckleiste entfernt hat.
Stellen Sie zuerst sicher, daß alle Tasten numeriert sind, so daß Sie sie wieder in der richtigen Reihenfolge einsetzen können.
Das ist ein guter Zeitpunkt, um alle Tasten zu entfernen und alle vorher unzugänglichen Bereiche sowie die Seiten der Tasten zu reinigen.
Sie können ein mildes Reinigungsmittel wie ein mit Xxxxxx befeuchtetes Tuch für das Reinigen der Seiten der Tasten benutzen.

Stellen Sie fest, ob die oberen, kugelförmigen Kontaktflächen der Piloten stumpf sind.
Wenn sie keine glänzende Politur haben, sind sie stumpf.
Benutzen Sie eine gute Messing-, Bronze- bzw. Kupferpolitur (wie z.B. Xxxxx), um die Kontaktflächen zu polieren und blank zu putzen.
Bauen Sie alles wieder zusammen, und die Mechanik sollte nun viel leichtgängiger sein.
 




<!-- c31.html -->

\chapter{Wissenschaftliche Methode, Theorie des Lernens und das Gehirn}
\label{c3_1}

\subsection{Einleitung}

Der erste Teil dieses Kapitels beschreibt meine Vorstellung davon, was eine wissenschaftliche Methode ist, und wie ich sie benutzt habe, um dieses Buch zu schreiben.
Dieser wissenschaftliche Ansatz ist der Hauptgrund, warum sich dieses Buch von allen anderen Büchern über das Thema des Klavierspielenlernens unterscheidet.

Die anderen Abschnitte behandeln Themen des Lernens im allgemeinen, und die Gleichung für die Lernrate wird hergeleitet.
Das ist die Gleichung, die benutzt wurde, um die Lernraten in \hyperref[c1iv5]{Kapitel 1, Abschnitt IV.5} zu berechnen.
Ich bespreche auch Themen, die das Gehirn betreffen, weil das Gehirn offensichtlich ein integraler Bestandteil des Spielmechanismus ist.
Mit Ausnahme der anfänglichen Diskussion darüber, wie sich das Gehirn im Laufe des Älterwerdens entwickelt und wie diese Entwicklung das Lernen des Klavierspielens beeinflußt, haben die Themen über das Gehirn jedoch nur eine geringe direkte Verbindung zum Klavier.
Die Rolle des Gehirns beim Lernen des Klavierspielens muß natürlich viel mehr erforscht werden.
Ich habe auch eine Diskussion über die Interpretation von Träumen eingefügt, die mehr Licht in die Frage bringt, wie das Gehirn arbeitet.
Zum Schluß beschreibe ich meine Erfahrungen mit meinem Unterbewußtsein, welches mir in zahlreichen Fällen gute Dienste geleistet hat.
 

\subsection{Der wissenschaftliche Ansatz}
\label{c3_2}

\subsubsection{Einleitung}
\label{c3_2a}

Dieses Buch wurde mit dem besten mir möglichen wissenschaftlichen Ansatz geschrieben, wobei ich benutzt habe, was ich während meiner 31-jährigen Karriere als Wissenschaftler lernte.
Ich befaßte mich nicht nur mit Grundlagenforschung (es wurden mir sechs Patente erteilt), sondern auch mit Materialwissenschaft (Mathematik, Physik, Chemie, Biologie, Maschinenbau, Elektronik, Optik, Akustik, Metallen, Halbleitern, Isolatoren), industrieller Problemlösung (Fehlermechanismen, Ausfallsicherheit, Fertigung) und wissenschaftlichen Veröffentlichungen (ich habe über 100 gegengeprüfte Artikel in den meisten großen Wissenschaftsmagazinen veröffentlicht).
Sogar nachdem ich meinen Doktortitel in Physik von der Cornell University verliehen bekam, investierten meine Arbeitgeber im Laufe meiner Karriere über eine Million Dollar, um meine Ausbildung zu fördern.
Zurückblickend war diese ganze wissenschaftliche Ausbildung für das Schreiben dieses Buchs unentbehrlich.
Diese Notwendigkeit, die wissenschaftliche Methode zu verstehen, läßt darauf schließen, daß es den meisten Klavierspielern schwerfallen würde, das gleiche Ergebnis zu erzielen.
Ich erkläre unten genauer, daß die Ergebnisse wissenschaftlicher Anstrengungen für jeden nützlich sind, nicht nur für Wissenschaftler.
\textbf{Deshalb bedeutet die Tatsache, daß dieses Buch von einem Wissenschaftler geschrieben wurde, daß jeder in der Lage sein sollte, es leichter zu verstehen, als wenn es nicht von einem Wissenschaftler geschrieben wäre.}
Ein Ziel dieses Abschnitts ist es, diese Botschaft zu erläutern.


\subsubsection{Lernen}
\label{c3_2b}

Klavier, Algebra, Bildhauerei, Golf, Physik, Biologie, Quantenmechanik, Tischlerei, Kosmologie, Medizin, Politik, Wirtschaftswissenschaft usw. - was haben diese gemeinsam?
Sie sind alle wissenschaftliche Disziplinen und haben deshalb eine große Zahl grundlegender Prinzipien gemeinsam.
In den folgenden Abschnitten \textbf{werde ich viele der wichtigen Prinzipien der wissenschaftlichen Methode erklären und zeigen, wie sie für das Erzeugen eines nützlichen Produkts benötigt werden}, z.B. für ein Handbuch zum Lernen des Klavierspielens.
Diese Erfordernisse für ein Klavierbuch unterscheiden sich nicht von den Erfordernissen für das Schreiben eines fortgeschrittenen Lehrbuchs über Quantenmechanik; die Erfordernisse sind ähnlich, obwohl die Inhalte Welten voneinander entfernt sind.
Ich beginne mit der Definition der wissenschaftlichen Methode, weil sie von der Öffentlichkeit so oft mißverstanden wird.
Danach beschreibe ich den Beitrag der wissenschaftlichen Methode zum Schreiben dieses Buchs.
Bei dieser Gelegenheit stelle ich heraus, wann die Klavierlehre in der Vergangenheit wissenschaftlich oder unwissenschaftlich war.
Während der letzten Jahrhunderte gab es durch das Anwenden der wissenschaftlichen Methode auf fast alle wichtigen Disziplinen enorme Fortschritte; ist es nicht an der Zeit, daß wir dasselbe mit dem Lernen und Unterrichten des Klavierspielens tun?

Dieser Abschnitt wurde hauptsächlich geschrieben, um die wissenschaftliche Methode zu skizzieren, in der Hoffnung, anderen dabei zu helfen, sie auf den Klavierunterricht anzuwenden.
Ein weiteres Ziel ist, zu erklären, warum es einen Wissenschaftler wie ich einer bin erforderte, um zu einem solchen Buch zu kommen.
Warum konnten Musiker ohne wissenschaftliche Ausbildung nicht bessere Bücher über das Klavierlernen schreiben?
Schließlich sind sie die führenden Experten auf diesem Gebiet!
Ich werde unten ein paar der Antworten darauf geben.

Ich vermute, Sie werden beim Lesen der folgenden Abschnitte Konzepte finden, die sich von Ihren Vorstellungen von der Wissenschaft unterscheiden.
\textbf{Wissenschaft an sich besteht nicht aus Mathematik, Physik oder Gleichungen.
Sie befaßt sich mit menschlichen Interaktionen, die andere Menschen zu etwas befähigen} (s.u.).
Ich habe viele \enquote{Wissenschaftler} gesehen, die nicht verstehen was Wissenschaft ist, und deshalb in ihrer eigenen Berufung versagten (d.h. entlassen wurden).
So wie täglich 8 Stunden zu üben Sie nicht notwendigerweise zu einem vollendeten Pianisten werden läßt, macht Sie das Bestehen aller Physik- und Chemieexamen nicht zu einem Wissenschaftler; Sie müssen etwas mit diesem Wissen vollbringen.
Ich war von vielen Klaviertechnikern besonders beeindruckt, die ein praktisches Verständnis der Physik haben, obwohl sie kein Wissenschaftsdiplom haben.
Diese Techniker müssen wissenschaftlich sein, weil das Klavier so tief in der Physik verwurzelt ist.
So definieren Mathematik, Physik usw. nicht die Wissenschaft (ein verbreitetes Mißverständnis); diese Gebiete haben sich lediglich als nützlich für Wissenschaftler erwiesen, weil sie in einer absolut vorhersagbaren Weise befähigen.
\textbf{Ich habe vor, Ihnen im folgenden die Ansicht eines Insiders darüber zu zeigen, wie Wissenschaft ausgeführt wird.}

Kann jemand, der keinerlei wissenschaftliche Ausbildung besitzt, das folgende lesen und sofort damit beginnen, den wissenschaftlichen Ansatz zu benutzen?
Wahrscheinlich nicht.
Es gibt keinen anderen Weg, als Wissenschaft zu studieren.
Sie werden sehen, daß die Erfordernisse und Komplexitäten der wissenschaftlichen Methode die meisten Menschen vor unüberwindbare Schwierigkeiten stellen.
Das ist natürlich eine Erklärung dafür, daß dieses Buch so einmalig ist.
Sie werden aber zumindest eine Vorstellung davon bekommen, was einige der nützlichen Vorschläge sind, wenn Sie den wissenschaftlichen Ansatz verfolgen möchten.

Lassen Sie uns, bevor wir die Definition der Wissenschaft in Angriff nehmen, ein verbreitetes Beispiel dafür untersuchen, wie Menschen die Wissenschaft mißverstehen, weil uns das ermitteln hilft, warum wir eine Definition benötigen.
Sie können einen Klavier- oder Tanzlehrer sagen hören, daß er ein Gefühl beschreibt, den Flug eines Vogels oder die Bewegung einer Katze, und seine Schüler bekommen sofort auf eine Art eine Vorstellung davon, wie sie spielen oder tanzen müssen, die der Lehrer unmöglich erreicht hätte, wenn er die Bewegung der Knochen, Muskeln, Arme usw. beschrieben hätte.
Der Lehrer behauptet dann, daß die Vorgehensweise des Künstlers besser ist als die wissenschaftliche.
Dieser Lehrer bemerkt nicht, daß er wahrscheinlich eine sehr gute wissenschaftliche Methode benutzt hat.
Indem man eine Analogie herstellt oder das Endprodukt der Musik beschreibt, kann man oft viel mehr Informationen übermitteln als durch das detaillierte Beschreiben jeder Komponente der Bewegung.
Es ist so, als ob man von Schmalband- zu Breitbandkommunikation übergeht und ist ein gültiges wissenschaftliches Vorgehen; es hat wenig mit der Unterscheidung zwischen Wissenschaft und Kunst zu tun.
Diese Art von Mißverständnis entsteht oft, weil die Menschen glauben, daß Wissenschaft schwarz oder weiß ist - daß etwas entweder wissenschaftlich ist oder nicht; die meisten Dinge im richtigen Leben sind mehr oder weniger wissenschaftlich, es ist nur eine Frage des Ausmaßes.
Was diese Lehrmethoden wissenschaftlicher macht oder nicht, hängt davon ab, wie gut sie die notwendigen Informationen übermitteln.
In dieser Hinsicht sind viele berühmte Künstler, die gute Lehrer sind, Meister dieser Art von Wissenschaft.
Ein weiteres häufiges Mißverständnis ist, daß Wissenschaft zu schwierig für Künstler sei.
Das verwundert doch sehr.
Die geistigen Prozesse, die Künstler beim Erzeugen der höchsten Stufen von Musik oder anderen Künsten durchlaufen, sind mindestens so komplex wie jene von Wissenschaftlern, die über den Ursprung des Universums nachdenken.
Das Argument, daß die Menschen mit unterschiedlichen Talenten für Kunst oder Wissenschaft geboren werden, mag teilweise gültig sein; ich stimme dieser Ansicht jedoch nicht zu - für den größten Teil der Menschen gilt, daß sie Künstler oder Wissenschaftler sein können, je nachdem in welchem Ausmaß sie, besonders in früher Kindheit, mit jedem Gebiet in Berührung gekommen sind.
Deshalb haben die meisten Menschen, die gute Musiker sind, die Fähigkeit, große Wissenschaftler zu sein.
Wenn man sein ganzes Leben Kunst studieren würde, hätte man nicht viel Zeit Wissenschaft zu studieren, wie kann man also beides miteinander kombinieren?
So wie ich es verstehe, sind Kunst und Wissenschaft komplementär; die Kunst hilft den Wissenschaftlern und umgekehrt.
Künstler, die der Wissenschaft aus dem Weg gehen, schaden sich nur selbst, und Wissenschaftler, die der Kunst aus dem Weg gehen, neigen dazu, weniger erfolgreiche Wissenschaftler zu sein.
Was mich in meiner Zeit am College am meisten beeindruckte, war die große Zahl von Wissenschaftsstudenten, die Musiker waren.
 


<!-- c33.html -->

\subsection{Was ist die Wissenschaftliche Methode?}
\label{c3_3}

\subsubsection{Einleitung}
\label{c3_3a}

Eine häufige falsche Vorstellung ist, daß Klavierspielen eine Kunst ist und deshalb der wissenschaftliche Ansatz nicht möglich und nicht anwendbar sei.
Diese falsche Vorstellung ist auf ein falsches Verständnis dafür zurückzuführen, was Wissenschaft ist.
Es mag viele Menschen überraschen, daß Wissenschaft in Wahrheit eine Kunst ist; Wissenschaft und Kunst können nicht voneinander getrennt werden, so wie Klaviertechnik und musikalisches Spielen nicht voneinander getrennt werden können.
Wenn Sie es nicht glauben, gehen Sie einfach zu irgendeiner großen Universität.
Sie wird immer eine herausragende Abteilung besitzen: die Abteilung für Kunst und Wissenschaft.
Beide erfordern Vorstellungskraft, Originalität und die Fähigkeit zur Ausführung.
Zu sagen, daß eine Person die Wissenschaft nicht kenne und deshalb einen wissenschaftlichen Ansatz nicht benutzen könne, ist so, als ob man sagen würde, daß man, wenn man weniger weiß, weniger lernen sollte.
Das macht keinen Sinn, weil es genau die Person, die weniger weiß, ist, die mehr lernen muß.
Offensichtlich müssen wir klar definieren, was Wissenschaft ist.


\subsubsection{Definition}
\label{c3_3b}

\textbf{Die einfachste Definition der wissenschaftlichen Methode ist, daß sie jede Methode ist, die funktioniert}.
Die wissenschaftliche Methode ist eine, die in völliger Harmonie mit der Realität oder Wahrheit ist.
Wissenschaft ist Befähigung.
Deshalb ist zu sagen, daß \enquote{Wissenschaft nur etwas für Wissenschaftler ist}, so, als ob man sagen würde, daß Jumbo Jets nur etwas für Luftfahrtingenieure sind.
Es ist wahr, daß  Flugzeuge nur von Luftfahrtingenieuren gebaut werden können, aber das hindert nicht einen von uns daran, Flugzeuge für unsere Reisen zu benutzen - in Wahrheit sind diese Flugzeuge für uns gebaut worden.
Genauso ist der Zweck der Wissenschaft, das Leben für alle leichter zu machen, nicht nur für Wissenschaftler.

Obwohl kluge Wissenschaftler benötigt werden, um die Wissenschaft voran zu bringen, kann jeder von der Wissenschaft profitieren.
Deshalb \textbf{ist eine weitere Möglichkeit, Wissenschaft zu definieren, daß sie zuvor unmögliche Aufgaben ermöglicht und schwierige Aufgaben vereinfacht.}
Von diesem Standpunkt aus nützt Wissenschaft den Unwissenderen unter uns mehr als den besser Informierten, die Dinge selbst herausfinden können.
Dazu ein Beispiel: Wenn ein Analphabet gebeten würde, zwei sechsstellige Zahlen zu addieren, hätte er keine Möglichkeit, es von selbst zu tun.
Jeder Drittklässler jedoch, der Rechnen gelernt hat, kann diese Aufgabe ausführen, wenn man ihm einen Stift und Papier gibt.
Heute kann man dem Analphabeten innerhalb weniger Minuten beibringen, diese Zahlen auf einem Taschenrechner zu addieren.
Nachweislich hat die Wissenschaft eine zuvor unmögliche Aufgabe für jeden leicht gemacht.

Die obigen Definitionen der wissenschaftlichen Methode liefern keine direkte Information darüber, wie man ein wissenschaftliches Projekt durchführt.
\textbf{Eine praktische Definition des wissenschaftlichen Ansatzes ist, daß er eine Gruppe von eindeutig definierten Objekten und deren Beziehungen zueinander ist}.
Eine der nützlichsten Beziehungen ist ein Klassifizierungsschema, das Objekte in Klassen und Unterklassen einteilt.
Beachten Sie, daß das Wort \enquote{definieren} eine sehr spezielle Bedeutung bekommt.
Objekte müssen in einer solchen Art definiert werden, daß sie nützlich sind und auf eine solche Art, daß die Beziehungen zwischen ihnen präzise beschrieben werden können.
Und all diese Definitionen und Beziehungen müssen wissenschaftlich korrekt sein (hierbei bekommen Nichtwissenschaftler Probleme).

Lassen Sie uns ein paar Beispiele ansehen.
Musiker haben grundlegende Objekte, wie z.B. \hyperref[c1iii5a]{Tonleitern}, \hyperref[c1iii7e]{Akkorde}, Harmonien, Verzierungen usw., definiert.
In diesem Buch wurden wichtige Konzepte, wie z.B. \hyperref[c1ii7]{Üben mit getrennten Händen}, \hyperref[c1ii9]{Akkord-Anschlag}, \hyperref[c1ii11]{parallele Sets}, \hyperref[c1ii5]{abschnittsweises Üben}, \hyperref[c1ii15]{automatische Verbesserung der Fähigkeiten nach dem Üben (PPI)} usw., präzise definiert.
Damit diese wissenschaftliche Methode, dieses Buch zu schreiben, funktioniert (d.h. damit ein nützliches Lehrbuch herauskommt), ist es notwendig, alle nützlichen Beziehungen zwischen diesen Objekten zu kennen.
Insbesondere ist es wichtig, vorauszusehen was der Leser \textit{benötigt}.
Der Akkord-Anschlag wurde als Antwort auf eine Notwendigkeit zur Lösung eines Geschwindigkeitsproblems definiert.
Man kann hier sehen, warum die Physik nicht so wichtig ist wie die menschliche Befähigung.
Ich habe verschiedene Bücher gelesen, die das Staccato besprechen, ohne es jemals zu definieren.
Die Wissenschaft spielt bereits auf den grundlegendsten Stufen der Definitionen, Erklärungen und Anwendungen eine Rolle.
Der Autor muß bestens mit den besprochenen Themen vertraut sein, damit er keine fehlerhaften Aussagen macht.
Das ist der Kern der Wissenschaft, nicht Mathematik oder Physik.

Eines der Probleme mit \hyperref[Whiteside]{Whitesides Buch} ist der Mangel an präzisen Definitionen.
Sie benutzt viele Worte und Konzepte, wie z.B. \hyperref[c1iii1b]{Rhythmus} und \hyperref[c1iii8]{Konturieren}, ohne sie zu definieren.
Das macht es für den Leser schwierig, zu verstehen was sie sagt oder ihre Anweisungen anzuwenden.
Natürlich mag es zunächst unmöglich erscheinen, diese komplexen Konzepte, auf die wir in der Musik oft treffen, zu definieren, besonders wenn man alle Komplexitäten und Nuancen im Umfeld eines schwierigen Konzeptes einschließen möchte.
Es ist jedoch die normale wissenschaftliche Vorgehensweise, Bestimmungsgrößen zu benutzen, um die Definition zu begrenzen, wenn man bestimmte Beispiele benutzt und andere Bestimmungsgrößen, um die Definition auf andere Möglichkeiten auszudehnen.
Es ist nur eine Frage sowohl des Verständnisses des Themas als auch der Bedürfnisse des Lesers.
\hyperref[Fink]{Finks} und \hyperref[Sandor]{Sandors} Buch bieten Beispiele von ausgezeichneten Definitionen.
Was ihnen fehlt, sind die Beziehungen: ein systematischer, strukturierter Ansatz, wie man diese Definitionen benutzt, um die Technik Schritt für Schritt zu erwerben.
Sie haben auch ein paar der wichtigen Definitionen vergessen, die in diesem Buch enthalten sind.

Der Hauptbestandteil der wissenschaftlichen Methode ist Wissen, aber Wissen alleine ist nicht genug.
Dieses Wissen muß in eine Struktur gebracht werden, so daß wir die Beziehungen zwischen den Objekten sehen, verstehen und ausnutzen können.
Ohne diese Beziehungen weiß man nicht, ob man alle notwendigen Teile hat oder gar wie man sie benutzt.
So sind z.B. \hyperref[c1ii11]{parallele Sets} ziemlich nutzlos, solange man das HS-Üben nicht kennt.
Die häufigste Methode, diesen Überbau herzustellen, ist ein Klassifizierungsschema.
In diesem Buch werden die verschiedenen Verfahren in Anfängermethoden, mittlere Stufen des Lernens, Methoden zum Auswendiglernen, Methoden zur Steigerung der Geschwindigkeit, schlechte Angewohnheiten usw. eingeteilt.
Hat man erst einmal die Definitionen und das Klassifizierungsschema, muß man anschließend die Details darüber, wie alles zusammengehört und ob es fehlende Elemente gibt, hinzufügen.
Wir besprechen nun einige besondere Komponenten der wissenschaftlichen Methode.


\subsubsection{Forschung}
\label{c3_3c}

Ein Handbuch über das Klavierspielen ist im Grunde eine Liste von Entdeckungen, wie man einige technische Probleme löst.
Es ist ein Produkt der Forschung.
In der wissenschaftlichen Forschung führt man Experimente durch, sammelt die Daten und schreibt die Resultate auf eine Art nieder, daß andere verstehen können, was man getan hat, und die Resultate reproduzieren können.
Klavierspielen zu lehren ist nicht anders.
Man muß zunächst verschiedene Übungsmethoden erforschen, die Resultate sammeln und sie aufschreiben, so daß andere davon profitieren können.
Klingt trivial einfach.
Aber wenn man sich umschaut, ist das \textit{nicht} das, was in bezug auf den Klavierunterricht geschehen ist.
Liszt hat seine Übungsmethoden niemals schriftlich festgehalten.
Die \enquote{intuitive Methode} (wie sie in diesem Buch beschrieben wird) erfordert keine Forschung; sie ist die am wenigsten informierte Art zu üben.
Deshalb war \hyperref[Whiteside]{Whitesides Buch} so erfolgreich - sie führte Forschungen durch und hielt ihre Ergebnisse fest.
Leider hatte sie keine wissenschaftliche Ausbildung und versagte bei wichtigen Aspekten, wie z.B. einem klaren, kurzen Schreibstil (besonders bei den Definitionen) und der Ordnung (Klassifizierung und Beziehungen).
Wenn es uns gelingt, diese Unzulänglichkeiten zu korrigieren, dann besteht natürlich einige Hoffnung, daß wir wissenschaftliche Methoden auf das Lehren des Klavierspielens anwenden können.
Offensichtlich wurde von allen großen Pianisten ein enormes Maß an Forschung durchgeführt.
Unglücklicherweise wurde sehr wenig davon dokumentiert; es fiel dem unwissenschaftlichen Ansatz der Klavierpädagogik zum Opfer.


\subsubsection{Dokumentation und Kommunikation}
\label{c3_3d}

Das oberste Ziel der Dokumentation ist die Aufzeichnung allen Wissens auf einem Gebiet - es ist ein unschätzbarer Verlust, daß Bach, Chopin, Liszt usw. ihre Übungsmethoden nicht niedergeschrieben haben.
Eine weitere Funktion der wissenschaftlichen Dokumentation ist das Eliminieren von Fehlern.
Eine korrekte Idee, die von einem großen Meister formuliert und mündlich von den Lehrern an die Schüler weitergegeben wurde, ist fehleranfällig und völlig unwissenschaftlich.
Wenn die Idee niedergeschrieben ist, kann man sie auf ihre Genauigkeit überprüfen und alle Fehler beseitigen sowie neue Erkenntnisse hinzufügen.
D.h., Dokumentation erzeugt eine Einbahnstraße, bei der sich eine Idee im Laufe der Zeit in ihrer Genauigkeit nur verbessern kann.

Eine Erkenntnis, die sogar Wissenschaftler überrascht hat, ist, daß ungefähr die Hälfte aller neuen Entdeckungen nicht während der Durchführung der Forschungen gemacht werden, sondern wenn die Resultate niedergeschrieben werden.
Aus diesem Grund hat sich das wissenschaftliche Schreiben zu einem Gebiet mit besonderen Erfordernissen entwickelt, die so beschaffen sind, daß nicht nur die Fehler minimiert werden, sondern auch der Entdeckungsprozeß maximiert wird.
Während des Schreibens dieses Buchs entdeckte ich die Erklärung für die \hyperref[c1iv2b]{Geschwindigkeitsbarrieren}.
Ich war damit konfrontiert, etwas über Geschwindigkeitsbarrieren zu schreiben und begann mich natürlich zu fragen, was sie sind und was sie erzeugt.
Es ist wohlbekannt, daß man, wenn man erst einmal die richtigen Fragen stellt, auf dem besten Weg ist, eine Antwort zu finden.
Ähnlich wurde das Konzept der \hyperref[c1iii7b]{parallelen Sets} mehr während des Schreibens entwickelt als während meiner Forschungen (Bücher lesen, mit Lehrern sprechen und das Internet benutzen) und persönlicher Experimente am Klavier.
Das Konzept der parallelen Sets wurde jedesmal benötigt, wenn bestimmte Übungsverfahren zu Schwierigkeiten führten.
Deshalb wurde es notwendig, dieses Konzept präzise zu definieren, damit man es wiederholt bei so vielen Gelegenheiten benutzen kann.

Es ist wichtig, mit allen anderen Wissenschaftlern, die ähnliche Arbeiten durchführen, zu kommunizieren und jegliche neuen Resultate der Forschung offen zu diskutieren.
In dieser Hinsicht war die Klavierwelt beklagenswert unwissenschaftlich.
Die meisten Bücher über das Klavierspielen haben nicht einmal ein Quellenverzeichnis (einschließlich der ersten Ausgabe meines Buchs, weil es innerhalb einer begrenzten Zeit geschrieben wurde - diese Unzulänglichkeit wurde in dieser zweiten Ausgabe korrigiert), und sie bauen selten auf den bisherigen Arbeiten von anderen auf.
Lehrer an den bedeutenden Musikinstitutionen kommen der Aufgabe zu kommunizieren besser nach als private Lehrer, weil sie an einer Institution versammelt sind und zwangsläufig in Kontakt kommen.
Als Folge davon ist die Klavierpädagogik an einer solchen Institution der der meisten privaten Lehrer überlegen.
Zu viele Klavierlehrer sind in bezug auf das Annehmen oder Erforschen verbesserter Lehrmethoden inflexibel und stehen oftmals allem kritisch gegenüber, das von \textit{ihren} Methoden abweicht.
Das ist eine sehr unwissenschaftliche Situation.

Beispiele der offenen Kommunikation in meinem Buch sind das miteinander Verflechten der Konzepte von: den \hyperref[c1ii10]{Armgewichtsmethoden} und der \hyperref[c1ii14]{Entspannung} (Ansatz nach der Art von Taubman), Ideen aus \hyperref[Whiteside]{Whitesides Buch} (Kritik an den Übungen der Art von \hyperref[c1iii7h]{Hanon} und der Methode des Daumenuntersatzes), Einschluß der verschiedenen von \hyperref[Sandor]{Sandor} usw. beschriebenen Handbewegungen.
Da das Internet die absolute Form der offenen Kommunikation ist, ist das Aufkommen des Internets eventuell das wichtigste Ereignis, das am Ende dazu führen wird, daß die Klavierpädagogik wissenschaftlicher durchgeführt wird.
Dafür gibt es kein besseres Beispiel als dieses Buch.

Ein Mangel an Kommunikation ist offensichtlich die Hauptursache, warum so viele Klavierlehrer immer noch die intuitive Methode lehren, obwohl die meisten der in diesem Buch beschriebenen Methoden während der letzten zweihundert Jahre von dem einen oder anderen Lehrer gelehrt wurden.
Wenn der wissenschaftliche Ansatz der völlig offenen Kommunikation und der richtigen Dokumentation von der Klavierlehrergemeinde früher angenommen worden wäre, dann wäre die jetzige Situation sicher eine ganz andere und eine große Zahl Klavierschüler würde mit Raten lernen, die im Vergleich zu den heutigen Standards unglaublich erscheinen.

Beim Schreiben der ersten Ausgabe meines Buchs wurde mir die Wichtigkeit der richtigen Dokumentation und des Ordnens der Ideen durch die Tatsache demonstriert, daß ich, obwohl ich die meisten Ideen in meinem Buch bereits ungefähr 10 Jahre kannte, nicht in vollem Umfang von ihnen profitieren konnte, bis ich dieses Buch fertiggestellt hatte.
Mit anderen Worten: Nachdem ich das Buch fertiggestellt hatte, las ich es erneut und probierte es systematisch aus.
Dann erst erkannte ich, wie effektiv die Methode war!
Obwohl ich die meisten Bestandteile der Methode kannte, gab es offenbar einige Lücken, die erst gefüllt wurden, als ich damit konfrontiert wurde, alle Ideen in eine nützliche und organisierte Struktur zu bringen.
Es ist so, als ob ich alle Einzelteile eines Autos hätte, sie aber solange nutzlos wären, bis ein Mechaniker sie zusammenbaut und das Auto einstellt.

So verstand ich z.B. nicht ganz, warum die Methode so schnell war (1000mal schneller als die intuitive Methode), bevor ich nicht die Berechnung der Lernrate durchgeführt hatte \hyperref[c1iv5]{(s. Kapitel 1, Abschnitt IV.5)}.
Ich führte die Berechnungen zunächst aus Neugierde aus, weil ich hoffte, ein Kapitel über die Lerntheorie zu schreiben.
Tatsächlich dauert es fast ein Jahr, bis ich mich selbst überzeugen konnte, daß die Berechnung ungefähr richtig war - eine Lernrate von 1000mal schneller schien zunächst ein lachhaft absurdes Ergebnis zu sein, bis ich feststellte, daß Schüler, die die intuitive Methode benutzen, oftmals während ihres ganzen Lebens nicht über die Mittelstufe hinauskommen, während andere in weniger als zehn Jahren zu Konzertpianisten werden können.
Die meisten Menschen neigten dazu, solche Unterschiede der Lernrate dem Talent zuzuschreiben, was nicht zu meinen Beobachtungen paßte.
Ein Nebenprodukt dieser Berechnung war ein besseres Verständnis dafür, \textit{warum} die Methode schneller war, weil man keine Gleichung schreiben kann, ohne zu wissen, welche physikalischen Prozesse beteiligt sind.
Als die mathematischen Formeln mir verrieten, welche Teile die Lernrate am meisten beschleunigten, konnte ich effektivere Übungsmethoden entwickeln.

Ein erstklassiges Beispiel einer neuen Entdeckung, die aus dem Schreiben dieses Buchs resultierte, ist das Konzept der \hyperref[c1ii11]{parallelen Sets}.
Ohne dieses Konzept fand ich es unmöglich, alle Ideen auf eine stimmige Weise zusammenzustellen.
Als das Konzept der parallelen Sets eingeführt war, führte es natürlich zu den \hyperref[c1iii7b]{Übungen für parallele Sets}.
Nichts davon wäre geschehen, wenn ich das Buch nicht geschrieben hätte, obwohl ich Übungen für parallele Sets die ganze Zeit benutzt hatte, ohne es bewußt wahrzunehmen.
Das kommt daher, daß der \hyperref[c1ii9]{Akkord-Anschlag} eine primitive Form der Übungen für parallele Sets ist; sogar \hyperref[Whiteside]{Whiteside} beschreibt Methoden für das Üben des \hyperref[c1iii3]{Trillers}, die im Grunde Übungen für parallele Sets sind.


\subsubsection{Konsistenzprüfungen}
\label{c3_3e}

Viele wissenschaftliche Entdeckungen werden als Resultat von Konsistenzprüfungen gemacht.
Diese Prüfungen funktionieren folgendermaßen.
Nehmen Sie an, Sie würden 10 Fakten über Ihr Experiment kennen, und Sie entdecken ein elftes.
Sie haben nun die Möglichkeit, dieses neue Ergebnis gegen alle alten Resultate zu prüfen, und oftmals führt diese Prüfung zu einer weiteren Entdeckung.
Eine einzige Entdeckung kann ohne jegliche weitere Experimente potentiell zu 10 weiteren Ergebnissen führen.
Die neuen Methoden dieses Buchs brachten z.B. ein viel schnelleres Lernen hervor, was dann darauf schließen ließ, daß die intuitive Methode Übungsverfahren beinhalten muß, die in Wahrheit das Lernen behindern.
Mit diesem Wissen wurde es eine einfache Sache, Gesichtspunkte der intuitiven Methode zu finden, die den Fortschritt verlangsamen.
Diese Aufdeckung der Schwächen der intuitiven Methode wären fast unmöglich gewesen, wenn man nur die intuitive Methode gekannt hätte.
Das ist eine Konsistenzprüfung, denn wenn beide Methoden korrekt wären, müßten Sie gleich effektiv sein.
Solch ein geistiger Prozeß, automatisch von allem auf das man trifft die Konsistenz zu prüfen, mag vielen Menschen nicht selbstverständlich erscheinen.
Als Wissenschaftler hatte ich das jedoch während meiner Laufbahn aus schierer Notwendigkeit bewußt getan.

Konsistenzprüfungen sind der ökonomischste und schnellste Weg, Fehler zu finden und neue Entdeckungen zu machen, weil man neue Ergebnisse erhält, ohne weitere Experimente durchzuführen.
Es kostet wenig extra, außer Ihrer Zeit.
Sie können nun sehen, warum der Prozeß des Dokumentierens so produktiv sein kann.
Jedesmal, wenn ein neues Konzept eingeführt wird, kann es gegen alle anderen bekannten Konzepte des Klavierübens geprüft werden, um potentiell zu neuen Ergebnissen zu führen.
Die Methode ist wegen der großen Zahl der Fakten, die bereits bekannt sind, mächtig.
Lassen Sie uns annehmen, daß man diese bekannten Wahrheiten zählen könnte und es 1000 wären.
Dann bedeutet eine neue Entdeckung, daß man nun 1000 weitere Möglichkeiten hat, um zu prüfen, ob sich neue Entdeckungen daraus ergeben!

Konsistenzprüfungen sind für das Eliminieren von Fehlern am wichtigsten und wurden benutzt, um Fehler in diesem Buch zu minimieren.
Langsames Üben ist z.B. sowohl nützlich als auch schädlich.
Diese Inkonsistenz muß irgendwie beseitigt werden; das geschieht durch sorgfältiges Definieren derjenigen Bedingungen, die langsames Üben erfordern (Auswendiglernen, HT-Üben), und der Bedingungen, unter denen langsames Üben abträglich ist (intuitive Methode ohne HS-Üben).
Klar ist jedes Pauschalurteil, das sagt \enquote{Langsames Üben ist gut, weil immer schnell zu spielen zu Problemen führt.}, nicht mit allen bekannten Fakten konsistent.
Wann immer ein Autor eine falsche Behauptung aufstellt, ist eine Konsistenzprüfung oft der leichteste Weg, diesen Fehler herauszufinden.


\subsubsection{Grundlegende Theorie}
\label{c3_3f}

Wissenschaftliche Resultate müssen immer auf einer Theorie oder einem Prinzip basieren, das durch andere verifiziert werden kann.
Sehr wenige Konzepte stehen allein, unabhängig von allem anderen.
Mit anderen Worten: Für alles, von dem jemand behauptet, daß es funktioniert, muß es eine gute Erklärung geben, warum es funktioniert; anderenfalls ist es suspekt.
Erklärungen wie \enquote{Es hat bei mir funktioniert.} oder \enquote{Ich habe das 30 Jahre lang so unterrichtet.} oder sogar \enquote{Das ist, wie Liszt es getan hat.} sind einfach nicht gut genug.
Wenn ein Lehrer ein Verfahren 30 Jahre unterrichtet hat, sollte er genügend Zeit gehabt haben, herauszufinden, warum es funktioniert.
Die \textit{Erklärungen} sind oft wichtiger als die Verfahren, die sie erklären.
HS-Üben funktioniert z.B., weil es eine schwierige Aufgabe vereinfacht.
Wenn dieses Prinzip der Vereinfachung eingeführt ist, kann man nach weiteren Dingen dieser Art Ausschau halten, wie z.B. schwierige Passagen zu kürzen oder das \hyperref[c1iii8]{Konturieren}.
Ein Beispiel für eine grundlegende Erklärung ist der Zusammenhang zwischen der Schwerkraft und der Armgewichtsmethode und ihrer Beziehung zum Tastengewicht.
Im Beispiel der schweren Hand des Sumoringers und der leichten Hand des Kindes \hyperref[c1ii10]{(Kapitel 1, II.10)} müssen beide bei einem korrekten Anschlag mit Freiem Fall einen Ton gleicher Intensität erzeugen, wenn ihre Hände aus der gleichen Höhe auf das Klavier herunterfallen.
Das ist offensichtlich  für den Sumoringer wegen seiner Neigung, sich auf das Klavier zu stützen, um seine schwerere Hand anzuhalten, schwieriger.
Deshalb ist der korrekte Freie Fall für den Sumoringer schwieriger auszuführen.
Diese Feinheiten auf theoretischer Grundlage zu verstehen führt zur Ausführung eines wirklich korrekten Freien Falls.
Mit anderen Worten: Bei einem korrekten Freien Fall darf man sich nicht auf dem Klavier abstützen, um die Hand anzuhalten, bis der Anschlag vollständig ist.
Man braucht ein sehr geschmeidiges Handgelenk, um diese Meisterleistung zu vollbringen.

Selbstverständlich gibt es immer ein paar Konzepte, die sich der Erklärung widersetzen, und es ist extrem wichtig, sie klar als \enquote{gültige Prinzipien ohne Erklärungen} zu klassifizieren.
Wie können wir in diesen Fällen wissen, daß sie gültig sind?
Sie können nur als gültig angesehen werden, nachdem man eine unbestreitbare Aufzeichnung der experimentellen Überprüfung erstellt hat.
Es ist wichtig, diese klar zu kennzeichnen, weil Verfahren ohne Erklärungen schwieriger anzuwenden sind und diese Verfahren sich während wir dazulernen und sie besser verstehen ändern.
Das beste an den Methoden, für die es gute Erklärungen gibt, ist, daß man uns nicht jedes Detail, wie man das Verfahren durchführt, sagen muß - wir können die Details oft anhand unseres eigenen Verständnisses der Methode selbst einfügen.

Leider ist die Geschichte der Klavierpädagogik voller Verfahren für das Erwerben der Technik, die keine theoretische Grundlage haben, die aber trotzdem eine breite Akzeptanz erfahren haben.
Die \hyperref[c1iii7h]{Hanon-Übungen} sind das beste Beispiel dafür.
Die meisten Anweisungen, wie man etwas tun soll, die ohne eine Erklärung dafür gegeben werden, warum sie funktionieren, haben in einem wissenschaftlichen Ansatz einen geringen Wert.
Das nicht nur wegen der hohen Wahrscheinlichkeit, daß solche Verfahren falsch sind, sondern auch, weil es die Erklärung ist, die dabei hilft, das Verfahren korrekt anzuwenden.
Weil es keine theoretische Grundlage für die Hanon-Übungen gibt, wenn er uns ermahnt, \enquote{die Finger stark anzuheben} und \enquote{eine Stunde täglich zu üben}, können wir in keinster Weise wissen, ob diese Verfahren tatsächlich hilfreich sind.
In jedem Verfahren des täglichen Lebens ist es für jeden fast unmöglich, alle notwendigen Schritte eines Verfahrens für alle denkbaren Fälle zu beschreiben.
Es ist ein Verständnis dafür, warum es funktioniert, das jedem gestattet, das Verfahren abzuändern, damit es den besonderen Bedürfnissen des einzelnen und der sich ändernden Umstände gerecht wird.

So empfehlen z.B. Lehrer, die die intuitive Methode benutzen, daß man das Spielen langsam und genau anfängt und die Geschwindigkeit schrittweise steigert.
Andere Lehrer mögen das langsame Spielen so weit wie möglich zu unterbinden suchen, weil es eine solche Zeitverschwendung ist.
Keines dieser Extreme ist das Beste.
Das \hyperref[c1ii16]{langsame Spielen des intuitiven Ansatzes} ist unerwünscht, weil man eventuell Bewegungen verfestigt, die das schnellere Spielen stören.
Auf der anderen Seite ist langsames Spielen, wenn man erst einmal mit der endgültigen Geschwindigkeit spielen kann, sehr nützlich für das \hyperref[c1iii6h]{Auswendiglernen} und für das Üben der \hyperref[c1ii14]{Entspannung} und Genauigkeit.
Deshalb ist die einzige Möglichkeit, die richtige Übungsgeschwindigkeit auszuwählen, im Detail zu verstehen, warum man diese Geschwindigkeit nehmen muß.
In diesem Zeitalter der Informationstechnologie und des Internets sollte es fast keinen Platz mehr für blindes Vertrauen geben.

Das heißt nicht, daß es Regeln ohne Erklärungen nicht gibt.
Schließlich gibt es immer noch viele Dinge in dieser Welt, die wir nicht verstehen.
Beim Klavierspielen ist die Regel, \hyperref[c1ii17]{vor dem Aufhören langsam zu spielen}, ein Beispiel dafür.
Es muß eine gute Erklärung geben, aber ich habe noch keine gehört, die ich für zufriedenstellend halte.
In der Wissenschaft sind Paulis Ausschließungsprinzip\footnote{oder kurz Pauli-Prinzip} (zwei Fermionen können nicht die gleichen Quantenzahlen haben) und die Heisenbergsche Unschärferelation Beispiele von Regeln, die nicht von einem tieferen Prinzip abgeleitet werden können.
Deshalb ist es genauso wichtig, etwas zu verstehen, wie zu wissen, was wir nicht verstehen.
Die sachkundigsten Physikprofessoren sind diejenigen, die alle Dinge benennen können, die wir immer noch nicht verstehen.


\subsubsection{Dogma und Lehre}
\label{c3_3g}

Wir wissen alle, daß man nicht jede Regel brechen kann, von der man glaubt, sie brechen zu können, und immer noch musikalisch spielen kann, es sei denn, man hat Initialen wie LvB.
Die dogmatischen Lehrmethoden, die in der Klavierpädagogik so weit verbreitet sind, haben sich in diesem restriktiven Umfeld der Schwierigkeit, Schüler zum Erzeugen von Musik anzuleiten, entwickelt.
Um es zynisch zu sagen: Der dogmatische Ansatz ist ein angenehmer Weg, die Unwissenheit des Lehrers dadurch zu verbergen, daß alles unter den Dogma-Teppich gekehrt wird.
Alle großen Vorträge, die ich von berühmten Künstlern gehört habe, sind voller exzellenter wissenschaftlicher Erklärungen, warum man auf eine bestimmte Art vorspielen oder nicht vorspielen sollte.
Es sind jedoch nicht alle großen Künstler auch gute Lehrer oder in der Lage, zu erklären was sie tun.
Die Lektion daraus ist für die Schüler, daß sie im allgemeinen nichts akzeptieren sollten, das sie nicht verstehen können; das wird dazu führen, daß die Ausbildungsstufe, die sie erreichen, ansteigt.
Ich bin überzeugt, daß sogar die Interpretation der Musik mit der Zeit ebenfalls wissenschaftlicher wird, genauso wie die Alchemie sich schließlich zur Chemie entwickelte.

Leider ist ein dogmatischen Herangehen an das Unterrichten nicht immer ein Zeichen für einen schlechteren Lehrer.
In Wahrheit scheint es, vermutlich aus historischen Gründen, eher das Gegenteil zu sein.
Zum Glück sind viele gute junge Lehrer, und besonders diejenigen an großen Institutionen, weniger dogmatisch - sie können erklären.
Wenn die Lehrer besser ausgebildet sind, sollten sie in der Lage sein, Dogma vermehrt durch ein tieferes Verständnis für die zugrunde liegenden Prinzipien zu ersetzen.
Das sollte die Effizienz und die Leichtigkeit des Lernens für den Schüler deutlich verbessern.

Den meisten Menschen ist bewußt, daß Wissenschaftler ihr ganzes Leben lang lernen müssen, nicht nur wenn sie an der Universität für ihre Abschlüsse arbeiten.
Den meisten ist jedoch nicht bewußt, in welchem Ausmaß Wissenschaftler ihre Zeit der Ausbildung widmen, nicht nur um zu lernen, sondern auch um alle anderen zu unterrichten, insbesondere andere Wissenschaftler.
Tatsächlich muß, um das Maß der Entdeckungen zu maximieren, die Ausbildung zu einer ganztägigen, alles verschlingenden Passion werden.
Wissenschaftler entwickeln sich deshalb oftmals mehr zu Lehrern als z.B. Klavier- oder Schullehrer, sowohl wegen des breiteren Bereichs an \enquote{Schülern}, auf die sie treffen, als auch wegen der Breite der Themen, die sie abdecken müssen.
Es ist wirklich erstaunlich, wie viel man wissen muß, um nur eine kleine neue Entdeckung zu machen.
Deshalb muß ein notwendiger Teil der wissenschaftlichen Dokumentation die höchsten Fertigkeiten des Unterrichtens einschließen.
Ein wissenschaftlicher Forschungsbericht ist nicht so sehr eine Dokumentation dessen, was getan wurde, als vielmehr ein Lehrbuch darüber, wie man das Experiment reproduziert und die zugrunde liegenden Prinzipien versteht.
Deshalb ist die wissenschaftliche Methode für das Unterrichten ideal.
Und es ist eine Lehrmethode, die zur dogmatischen Methode diametral verschieden ist.


\subsubsection{Schlußfolgerungen}
\label{c3_3h}

Der wissenschaftliche Ansatz ist mehr als nur eine präzise Art, die Ergebnisse eines Experiments zu dokumentieren.
Er ist ein Verfahren, das zur Beseitigung von Fehlern und zur Erzeugung von Entdeckungen entwickelt wurde.
Vor allem ist er im Grunde ein Mittel zur Befähigung des Menschen.
Wenn der wissenschaftliche Ansatz früher übernommen worden wäre, dann wäre die Klavierpädagogik heute aller Wahrscheinlichkeit nach völlig anders.
Das Internet wird sicherlich die Übernahme von wissenschaftlicheren Vorgehensweisen in das Lernen des Klavierspielens beschleunigen.
 

\subsection{Theorie des Lernens}
\label{c3_4}

\footnote{Abschnitt 4 ist im Original z.Zt. (30.5.2005) noch \enquote{preliminary draft} also ein \enquote{Rohentwurf}.}

Ist es nicht seltsam, daß wir, wenn wir auf die Universität gehen, finden, daß \enquote{101 Lernen} kein erforderlicher Kurs ist (wenn er überhaupt existiert!)?
Von Colleges und Universitäten erwartet man, daß sie Lernzentren sind.
Psychologische Abteilungen haben oft einführende Kurse über Studiengewohnheiten usw., aber man sollte meinen, daß die Wissenschaft des Lernens der erste Punkt auf der Tagesordnung an jedem Lernzentrum wäre.
Beim Schreiben dieses Buchs fand ich, daß es notwendig ist, über den Lernprozeß nachzudenken und eine - wie auch immer näherungsweise - Gleichung für die Lernrate abzuleiten.
 


<!-- c35.html -->

\subsection{Was Träume erzeugt und Methoden zu ihrer Kontrolle}
\label{c3_5}

\subsubsection{Einleitung}
\label{c3_5a}

Dieser Abschnitt hat nichts mit Klavieren zu tun.
Er ist hier eingefügt, weil er ein wenig Klarheit darüber bringt, wie das Gehirn funktioniert.
Ich kenne keine Untersuchungen über die Ursachen von Träumen und Methoden für ihre Kontrolle, wie ich sie unten beschreibe.
Wenn Sie eine solche Quelle kennen, schicken Sie mir bitte eine Mail.

Haben Sie jemals wiederkehrende Träume gehabt und sich gefragt, was sie verursacht?
Oder Alpträume, die Sie gerne losgeworden wären?
Es scheint so, als hätte ich Antworten auf diese beiden Fragen gefunden und bei dem Prozeß einige Einsichten darüber gewonnen, wie das Gehirn während wir schlafen funktioniert.

Die meisten Traumdeuter sind heutzutage wie Handleser.
Sie bemühen sich, Ihre Zukunft vorherzusagen und schreiben Träumen magische Kräfte oder Botschaften zu, die wundervoll wären, wenn sie wahr wären, aber leider so realistisch sind wie Séancen oder Kaffeesatzlesen.
Ich habe gefunden, daß eine Interpretation der Träume, die auf körperlichen Anzeichen basiert, uns eine Menge darüber sagen kann, wie unser Gehirn funktioniert.
Ich bespreche hier vier Arten von Träumen, die ich hatte und für die ich eine körperliche Erklärung gefunden habe.
In Diskussionen mit Freunden habe ich entdeckt, daß viele ähnliche Träume haben und diese, fast mit Sicherheit, ähnliche Ursachen.
Im letzten Abschnitt bespreche ich, was diese Träume uns darüber sagen, wie unser Gehirn funktioniert.
Ich bin zu dem Schluß gekommen, daß dieses Herangehen an Träume viel lohnender ist als das der Wahrsager und ähnlicher Traumdeuter.
Die vier Träume, die unten besprochen werden sind:

\begin{itemize} 
 \item \hyperref[c3_5b]{fallen},
 \item \hyperref[c3_5c]{unfähig sein zu laufen},
 \item \hyperref[c3_5d]{zu spät zu Besprechungen oder Prüfungen kommen oder unfähig sein, das Ziel zu finden}, und
 \item \hyperref[c3_5e]{ein langer, komplexer Traum, der für mich spezifisch ist}.
 \end{itemize}
Die ersten drei sind, so glaube ich, ziemlich verbreitete Träume, die viele Menschen haben.
 

\subsubsection{Der Fall-Traum}
\label{c3_5b}

In diesem Traum falle ich, nicht von einem bestimmten Ort oder hinunter auf einen bestimmten Platz, aber ich falle definitiv und habe Angst.
Und ich bin absolut unfähig, den Fall zu stoppen.
Stets bin ich, wenn ich lande, unverletzt.
Es gibt keine Schmerzen.
Tatsächlich fühlt es sich, obwohl ich auf dem Boden aufgeschlagen bin, wie eine weiche Landung an, und der Traum hört immer auf, sobald ich lande.
Die weiche Landung ist besonders seltsam, denn bei jedem Fall auf fast jede Fläche gibt es im allgemeinen am Ende irgendeine Art von Katastrophe.
Was würde alle diese Details jenes Traums erklären?
Ich habe eines Tages die körperliche Ursache dieses Traums entdeckt, als ich unmittelbar nach dem Traum aufwachte und feststellte, daß meine Knie heruntergefallen waren.
Ich hatte auf dem Rücken geschlafen, hatte beide Knie hochgestellt und als ich meine Beine streckte, führte das Gewicht der Bettdecke dazu, daß meine Füße ausrutschten und die Knie herunterfielen.
Diese fallenden Knie brachten mein Gehirn dazu, den Fall-Traum zu erzeugen!
Zunächst war das nur eine hypothetische Erklärung und eine offensichtlich dumme noch dazu.
Warum konnte mein Gehirn nicht erkennen, daß meine Knie gefallen waren?
Nachdem die Hypothese aber erst einmal aufgestellt war, konnte ich sie jedesmal überprüfen, wenn ich diesen Traum hatte (über einen Zeitraum von mehreren Jahren), und es gelang mir mehrere Male, sie zu bestätigen.
Beim Aufwachen konnte ich mich deutlich daran erinnern, daß meine Knie gerade eben heruntergefallen waren.
Die Tatsache, daß die Knie auf das weiche Bett fallen, erklärt die weiche Landung, und da hinterher nichts passiert, endet der Traum.
Warum bin ich unfähig, den Fall der Knie zu stoppen?
Wie weiter unten wiederholt gezeigt wird, haben wir während wir schlafen manchmal eine sehr geringe Kontrolle über unsere Muskeln.
Nicht nur das, das schlafende Gehirn kann nicht einmal die einfache Tatsache erkennen, daß das Knie fällt.
Zusätzlich denkt es sich das aus, was normalerweise ein unglaubliches Szenario eines Falls sein sollte, und tatsächlich glaube ich es am Ende.
Dieser letzte Teil ist der absurdeste, weil ich mich doch tatsächlich selbst hereinlege!


\subsubsection{Der Unfähig-zu-laufen-Traum}
\label{c3_5c}

Das ist ein sehr frustrierender Traum.
Ich möchte laufen, aber ich kann es nicht.
Es macht keinen Unterschied, ob mich jemand verfolgt oder ob ich bloß schnell irgendwo hinlaufen will; ich kann nicht laufen.
Wenn man läuft, muß man sich vorwärts beugen.
Deshalb versuche ich im Traum, mich nach vorne zu beugen, aber ich kann es nicht.
Irgend etwas schiebt mich fast zurück.
Im Traum habe ich sogar überlegt, daß wenn ich nicht vorwärts laufen oder mich vorwärts beugen kann, warum dann nicht zurücklehnen oder rückwärts laufen?
Auf diese Art kann ich mich zumindest bewegen.
Ich kann mich auch nicht zurücklehnen, meine Füße sind wie gelähmt, und ich komme weder vorwärts noch rückwärts richtig voran.
Wenn man läuft, muß man zunächst seine Knie nach vorne und oben bringen, so daß man nach hinten treten kann, aber ich kann auch das nicht.
Was würde ein solches Gefühl auslösen während ich schlafe?
Ich habe die Ursache dieses Traumes entdeckt, nachdem ich den \hyperref[c3_5b]{Fall-Traum} gelöst hatte, so daß die Erklärung leichter zu finden war.
Wieder kam ich auf die Erklärung, als ich unmittelbar nach dem Traum aufwachte und mich selber mit dem Gesicht nach unten auf dem Bauch liegend fand. Heureka!
Wenn man auf dem Bauch liegt, kann man den Winkel des Körpers zum Bett nicht verändern; man kann sich nicht vorwärts lehnen.
Man kann auch nicht die Knie nach oben ziehen, weil das Bett im Weg ist.
Man kann sich nicht zurücklehnen, weil man von der Schwerkraft nach unten gedrückt wird.
Man kann nicht rückwärts gehen, weil das Bettzeug im Weg ist.
Das zeigt erneut, daß man während man schläft keine große Kontrolle über die Muskeln hat, denn wenn man wach wäre, dann wäre das Hochziehen der Knie nicht so schwierig, sogar wenn man mit dem Gesicht nach unten liegen würde.
Nachdem ich die Erklärung gefunden hatte, konnte ich sie wieder mehrmals bestätigen; d.h., als ich wach wurde, lag ich mit dem Gesicht nach unten.
An diesem Punkt fing ich an zu erkennen, daß vielleicht die meisten unserer Träume eine körperliche Erklärung haben.
Das Ganze machte jedoch irgendwie keinen Sinn - warum sollte mein Gehirn nicht wissen, daß meine Knie herunterfallen, oder daß ich mit dem Gesicht nach unten schlafe?
Wie kann mein Gehirn einen so komplexen Traum träumen und trotzdem nicht in der Lage sein, solch einfache Dinge zu erkennen?
Und wieder hat sich mein Gehirn eine Geschichte ausgedacht und sie mich erfolgreich glauben lassen, während ich träumte.


\subsubsection{Der Zu-spät-zur-Prüfung-kommen- oder Sich-verlaufen-Traum}
\label{c3_5d}

Dies ist ein weiterer frustrierender Traum.
Können Sie sehen, wie ein Muster zum Vorschein kommt?
Ich werde weiter unten spekulieren, warum Träume dazu neigen, negativ oder alptraumhaft zu sein.
Dies ist kein bestimmter Traum, sondern eine ganze Klasse von Träumen, in denen ich versuche, zu einer Prüfung oder irgendwo anders hin zu kommen, aber spät dran bin und nicht hingelangen oder es nicht finden kann.
Ich muß z.B. einen steilen Hang überwinden oder um Gebäude herumlaufen.
Oder wenn ich in einem Gebäude bin, gehe ich durch einen Irrgarten aus Rampen, Treppen, Türen, Aufzügen usw., aber ich kann noch nicht einmal zum Ausgangspunkt zurück.
Tatsächlich wird es immer schlimmer und komplexer.
Nach einer Weile kann ich ziemlich erschöpft sein.
Dieser Traum könnte auftreten, wenn ich in einer ungünstigen oder unbequemen Position schlafe, aus der ich nicht leicht herauskomme, wie z.B. auf meiner Hand schlafend oder im Laken oder Bettzeug eingewickelt.
Jede Art von Schlafposition, die unbequem ist, aus der ich gerne herauskommen möchte, es aber nicht leicht tun kann während ich schlafe.
Wenn ich in den Laken eingewickelt bin, kann ich mich nicht so leicht daraus befreien während ich schlafe, und je mehr ich damit kämpfe, desto mehr verwickle ich mich darin, und es kann sehr anstrengend werden.
Ich war bisher nicht in der Lage, diese Traumfamilie oder eines seiner Mitglieder direkt mit einer bestimmten Ursache zu verbinden, wie bei den anderen drei Träumen.
Ich habe jedoch eine leichte Schlafapnoe und das erste Auftreten dieser Art von Träumen fällt damit zusammen, was ich für das erste Auftreten der Schlafapnoe halte.
Somit könnte der Traum durch meine Unfähigkeit zu atmen verursacht worden sein.

Was auch immer die genaue Ursache ist, ob eine unbequeme Position oder Schlafapnoe, so ist klar, daß ich, wenn ich wach gewesen wäre, leicht eine Lösung gefunden hätte.
Somit ist das Muster, das zum Vorschein kommt, daß mein logisches Denkvermögen und meine Fähigkeit zur Lösung von Problemen stark beeinträchtigt sind; sehr einfache Probleme können mich in die Klemme bringen während ich schlafe.
 

\subsubsection{Die Lösung für meinen langen und komplexen Traum}
\label{c3_5e}

Nachdem ich die Lösung für die drei oben genannten Träume gefunden hatte, war ich überzeugt, daß ein anderer wiederkehrender Traum ebenfalls eine körperliche Ursache hatte.
Dieser Traum war lang und komplex aber immer derselbe.
Er fängt angenehm an.
Ich gehe für eine Klettertour nach draußen, und vor mir ist eine sanfter Hügel oder eine wogende Wiese, die in der Ferne zu einem Berg führen.
Das erste Anzeichen, daß etwas nicht stimmt, kommt von diesem Berg.
Er geht mit steilen Felswänden nach oben, und die Spitze ist so hoch, daß ich sie kaum sehen kann.
Ich mache mich trotzdem auf den Weg, aber sofort tritt eine furchterregende Situation ein: Ich bin an der Kante einer vertikalen Felswand, und ich kann nicht einmal den Boden darunter sehen!
Ich bekomme Angst, drehe mich sofort um und versuche zurück zu gehen, aber der Vorsprung, auf dem ich weitergehe, wird schmaler, als ob ich auf einem Schwebebalken gehen würde.
Schließlich merke ich, daß ich fast am Ende bin aber eine letzte Hürde nehmen muß: einen Fluß!
Bevor ich über Felsen springe, um über den Fluß zu kommen, prüfe ich ihn mit der Hand, und das Wasser ist kalt und tief.
Ungefähr in diesem Stadium endet der Traum.
Wie würde ich solch einen komplexen Traum erklären?
Ich löste das Rätsel wieder, nachdem ich unmittelbar nach dem Traum erwachte.
Ich hatte am Rand des Betts geschlafen, und eine Hand schaute unter der Bettdecke hervor und hing herunter.
Nun konnte ich jedes Detail meines Traums erklären!
Mein Traum beginnt offensichtlich damit, daß ich auf dem Bauch schlafe, mit meinem Kinn auf dem Bett, und ich schaue auf das Kissen vor mir (die wogende Wiese).
Hinter dem Kissen ist das vertikale hölzerne Kopfende, aus kanadischer Walnuß hergestellt, das wie eine steile Felswand aussieht, welches der Berg ist.
Mit meinem Kinn auf dem Bett kann ich kaum die Spitze des Kopfendes sehen.
Bis hierher ist interessant, daß ich offensichtlich Sachen ansehe während ich schlafe.
Da ich an der Bettkante schlafe, fällt eine Hand über die Kante, und das ist die Kante der Felswand, an der ich stehe.
Ungefähr sieben Zoll\footnote{\lt 20 cm} von meinem Bett entfernt steht mein Nachttisch mit einer schmalen abgestuften Kante wie die Oberseite eines Schwebebalkens (schwer zu beschreiben).
Meine Hand tastet also offensichtlich herum.
Da meine Hand nun nicht mehr unter der Bettdecke liegt, fühlt sie sich kalt an (der kalte Fluß). Das ist es!
Diese Erklärungen tragen jedem Detail meines Traums Rechnung!
Diese Erklärungen haben mich überzeugt, daß Träume interpretiert werden KÖNNEN, und daß die meisten von ihnen körperliche Ursachen haben.
Wenn das alles wahr ist, dann sollten wir in der Lage sein, die Ursachen und Erklärungen zu benutzen, um daraus abzuleiten, was das Gehirn während des Schlafens tut.
Das ist eine aufregende Aussicht, von deren Verwirklichung nicht einmal die Wahrsager und Traumdeuter träumen konnten.


\subsubsection{Die Kontrolle der Träume}
\label{c3_5f}

Das erstaunlichste an der Erklärung dieser Träume war, daß ich etwas Kontrolle über sie entwickelte.
Nachdem ich völlig überzeugt war, daß jede Erklärung richtig ist, verschwanden diese Träume!
Ich konnte mich nicht mehr selbst hereinlegen!
Zu denken, daß fallende Knie dasselbe sind wie von einem Dach oder einer Klippe zu stürzen, heißt ganz klar, mich selbst hereinzulegen.
Wenn der Mechanismus erst einmal verstanden wird, dann wird das Gehirn nicht mehr getäuscht.
Obwohl das Gehirn hinreichend abgeschaltet ist, so daß es während des Schlafs leicht getäuscht werden kann, hat es demnach genügend Kapazität, um die Wahrheit zu erkennen, wenn der Mechanismus erst bekannt ist.

Trotzdem erschien es mir irgendwie weit hergeholt, daß ich mich selbst hereingelegt hatte.
Um mich selbst davon zu überzeugen, daß diese Art von Täuschung möglich ist, mußte ich ein Beispiel aus dem richtigen Leben finden.
Glücklicherweise habe ich eins gefunden.
Es ist das, was Magier tun.
Wenn man einen Zaubertrick beobachtet, weiß man, daß es keine Zauberei ist, man fällt aber jedesmal in dem Sinn darauf herein, daß es völlig verwirrend und sehr aufregend ist.
Nun ändert sich die Geschichte gänzlich, wenn Ihnen jemand erklären sollte, wie der Trick funktioniert.
Dann verschwindet plötzlich das Mysterium und die Spannung, und man konzentriert sich am Ende darauf, wie der Magier den Trick ausführt.
Man kann nicht dazu verleitet werden zu denken, daß es Zauberei ist.
Somit kann unser Gehirn in einem Traum so lange getäuscht werden, wie es nicht weiß, daß es getäuscht wird.
Da die meisten Menschen die Erklärung für den Traum nicht kennen, sind sie sich der stattfindenden Täuschung offensichtlich nicht bewußt, und die Träume gehen weiter.
Kennt man erst einmal die Ursache des Traums, weiß man auch, daß das Gehirn getäuscht wurde; deshalb ist es dann für das Gehirn viel einfacher, die Wahrheit herauszufinden, und der Traum verschwindet.
Bevor man die Wahrheit herausgefunden hat, wußte das Gehirn nicht einmal, daß es getäuscht wurde, so daß es keinen Grund hatte, nach der Wahrheit zu suchen.
Nun scheint alles Sinn zu machen.


\subsubsection{Was uns diese Träume über das Gehirn lehren}
\label{c3_5g}

Diese vier Beispiele lassen darauf schließen, daß die meisten Träume einen konkreten körperlichen Ursprung haben.
Ich habe diese Art von Erklärung niemals zuvor gesehen, trotzdem erscheint alles vernünftig.
Soweit ich weiß, ist der \hyperref[c3_5b]{Fall-Traum} weit verbreitet - viele haben diesen Traum.
Bei mir war es das fallende Knie; bei jemand anderem mag es ein Arm oder ein Bein sein, daß über die Bettkante gleitet.

Die obigen Resultate bieten eine Unmenge an Möglichkeiten, über die Funktionsweise des Gehirns zu spekulieren.
Dazu ein paar Ideen.
Während des Schlafs ist der größte Teil des Gehirns abgeschaltet, so daß es nicht überraschend ist, wenn das Gehirn leicht getäuscht werden kann.
Es scheint, daß die höheren Funktionen vollständiger abgeschaltet sind, so daß das logische Denken am stärksten beeinträchtigt ist.
Es kann sein, daß Angst das Gefühl ist, das beim Einschlafen als letztes abgeschaltet und beim Aufwachen als erstes eingeschaltet wird - wahrscheinlich aus Gründen des Überlebens.
Wenn ein Feind während des Schlafs angreift, ist Angst das erste Gefühl, das aufgeweckt werden muß.
Das läßt darauf schließen, daß die meisten Träume tendenziell alptraumhaft sein könnten.
Aber selbstverständlich kann das von Person zu Person unterschiedlich sein, und einige Menschen können hauptsächlich angenehme Träume haben, je nach der Veranlagung der Person.
In meinem Fall legen die Anhaltspunkte nahe, daß die Träume, die ich entschlüsselt habe, unmittelbar bevor ich aufgewacht bin aufgetreten sind.
Das läßt darauf schließen, daß die meisten Träume während der kurzen Zeit zwischen Schlaf und Erwachen auftreten.
Obwohl es Schlafwandler gibt, die ihre starken Muskeln während des Schlafs kontrollieren können, zeigt das oben gesagte, daß die Bemühungen, während eines Traums die Muskeln zu bewegen, nicht gut in tatsächliche Bewegungen umgesetzt werden.
Das \hyperref[c3_5e]{vierte Beispiel} zeigt jedoch, daß man sich während des Schlafs viel bewegt - zusätzlich zu den normalen Bewegungen, die notwendig sind, um den Körper periodisch in eine andere Lage zu bringen, damit man einen ausgedehnten Durchblutungsmangel an den Stellen, an denen man aufliegt, vermeidet usw.
Somit ist die Bewegung des Körpers während des Schlafs ein völlig normaler Prozeß als Antwort auf die Schmerzen, die entstehen, wenn man zu lange in einer Position bleibt.
Eine Minderheit scheint in der Lage zu sein, im Grunde die ganze Nacht in einer Position zu schlafen; solche Menschen müssen eine Methode haben, die Auflagepunkte mit Sauerstoff usw. zu versorgen, so daß keine wunden Stellen entstehen (vielleicht bewegen sie sich unmerklich nach der einen oder anderen Seite, um den Druck zeitweilig zu verringern).

Ich glaube, daß ich hier ein paar überzeugende Beispiele dafür aufgeführt habe, wie Träume eher auf der Wirklichkeit basierend interpretiert werden können als auf den falschen übernatürlichen Kräften, die historisch bedingt mit der Traumdeutung verbunden werden.
Dieser Ansatz scheint eine Einsicht in die Arbeitsweise des Gehirns während des Schlafs zu liefern.
Eine mögliche Anwendung von Träumen, die mit der Realität verbunden werden können, ist, daß sie zu nützlichen Diagnosewerkzeugen für Störungen wie z.B. Schlafapnoe werden können.
Sie können uns viel über unsere Bewegungen während des Schlafs sagen und darüber, wie man etwas ändern kann, damit man besser schlafen kann.



<!-- c36.html -->

\subsection{Das Unterbewußtsein}
\label{c3_6}

\subsubsection{Einleitung}
\label{c3_6a}

Das Gehirn hat einen bewußten und einen unterbewußten Teil.
Die meisten Menschen haben es nicht gelernt, das Unterbewußtsein zu benutzen, aber das Unterbewußtsein ist wichtig, weil es

\begin{itemize} 
 \item die Emotionen kontrolliert,
 \item 24 Stunden am Tag funktioniert, egal ob man wach ist oder schläft, und
 \item einige Dinge tun kann, die das Bewußtsein nicht kann, einfach weil es eine andere Art Gehirn ist.
 \end{itemize}
Obwohl es schwierig ist, das bewußte Gehirn mit dem unterbewußten zu vergleichen, weil sie verschiedene Funktionen ausüben und verschiedene Fähigkeiten haben, könnten wir statistisch vermuten, daß das Unterbewußtsein bei der Hälfte der menschlichen Bevölkerung cleverer ist als das bewußte.
Zusätzlich zur Tatsache, daß man eine zusätzliche Fähigkeit des Gehirns hat, macht es deshalb keinen Sinn, diesen Teil des Gehirns, der eventuell cleverer ist als der bewußte Teil, nicht zu benutzen.
In diesem Abschnitt präsentiere ich meine Ideen dazu, wie das Unterbewußte eventuell funktioniert und zeige, wie wir mit Hilfe des Unterbewußtseins einige erstaunliche Leistungen vollbringen können.


\subsubsection{Emotionen}
\label{c3_6b}

Das Unterbewußtsein kontrolliert Emotionen auf mindestens zwei Arten.
Die erste ist eine schnelle Kampf- oder Fluchtreaktion - das Erzeugen von sofortiger Wut oder Furcht.
Wenn solche Situationen aufkommen, muß man schneller reagieren können als man denken kann, so daß das bewußte Gehirn durch etwas umgangen werden muß, das für eine sofortige Reaktion fest verdrahtet und vorprogrammiert ist.
Die zweite ist ein sehr langsames, schrittweises Erkennen einer tiefen oder grundlegenden Situation.
Ob der erste und der zweite Teil des unterbewußten Gehirns Teile desselben Unterbewußtseins sind, ist eine akademische Frage, da wir fast mit Sicherheit viele Arten eines unterbewußten Verhaltens besitzen.
Gefühle der Depression während einer Midlife-Krise könnten das Ergebnis von Vorgängen der zweiten Art von Unterbewußtsein sein: Das unterbewußte Gehirn hat während man älter wird Zeit gehabt, alle negativen Situationen herauszufinden, die sich entwickeln, und die Zukunft fängt an weniger hoffnungsvoll auszusehen.
Solch ein Prozeß erfordert die Auswertung von Myriaden guter und schlechter Möglichkeiten, die die Zukunft bringen mag.
Wenn man versuchen wollte, solch eine zukünftige Situation zu bewerten, müßte das bewußte Gehirn alle diese Möglichkeiten auflisten, jede bewerten und versuchen, sie zu behalten.
Das Unterbewußtsein funktioniert anders.
Es bewertet verschiedene Situationen auf eine unsystematische Weise; wie es eine bestimmte Situation für die Beurteilung auswählt, unterliegt nicht unserer Kontrolle; das wird mehr von alltäglichen Ereignissen kontrolliert.
Das Unterbewußtsein speichert seine Schlußfolgerungen auch in etwas, was man \enquote{Emotionsfach} nennen könnte.
Für jede Emotion gibt es ein Fach und jedesmal, wenn das Unterbewußtsein zu einem Entschluß kommt, sagen wir zu einem glücklichen, dann deponiert es den Entschluß in einem \enquote{Glücklichfach}.
Der Füllgrad jedes Fachs bestimmt Ihren emotionalen Zustand.
Das erklärt, warum Menschen oftmals spüren können, was richtig oder falsch ist oder ob eine Situation gut oder schlecht ist, ohne daß sie genau wissen, was die Gründe dafür sind.
So beeinflußt das Unterbewußtsein unser Leben viel mehr als die meisten von uns merken.


\subsubsection{Das Unterbewußtsein benutzen}
\label{c3_6c}

Üblicherweise geht das Unterbewußtsein seine eigenen Wege; man kontrolliert normalerweise nicht, welche Möglichkeiten es in Betracht zieht, weil die meisten von uns nicht gelernt haben, mit ihm zu kommunizieren.
Die Ereignisse, denen man im täglichen Leben begegnet, machen es jedoch in der Regel ziemlich deutlich, welche Faktoren wichtig sind und welche unwichtig, und es zieht das Unterbewußtsein ganz natürlich zu den wichtigsten Ideen.
Wenn diese wichtigen Ideen zu wichtigen Schlüssen führen, wird es interessierter.
Wenn sich eine genügende Zahl solcher wichtiger Schlüsse aufstapeln, wird es sich mit Ihnen in Verbindung setzen.
Das erklärt, warum manchmal plötzlich eine unerwartete Eingebung in unserem Bewußtsein aufblitzt.
Darum ist hier die wichtige Frage, wie man am besten mit seinem Unterbewußtsein kommunizieren kann.

Jede Idee, von der man sich selber überzeugen kann, daß sie wichtig ist, oder jedes Rätsel oder Problem, das man mit großer Mühe zu lösen versucht hat, wird offensichtlich ein Kandidat zur Überprüfung durch das Unterbewußtsein sein.
Das ist deshalb eine Art, wie man sein Problem dem Unterbewußtsein präsentieren kann.
Außerdem muß das Unterbewußtsein, um in der Lage zu sein, ein Problem zu lösen, alle notwendigen Informationen besitzen.
Deshalb ist es wichtig, alles zu untersuchen und so viele Informationen über das Problem zu sammeln wie man kann.
Im College habe ich auf diese Art viele Probleme meiner Hausaufgaben gelöst, die meine klügeren Klassenkameraden nicht lösen konnten.
Sie haben versucht, sich einfach hinzusetzen, ihre Aufgabe zu bearbeiten und hofften, das Problem zu lösen.
Probleme in einer schulischen Umgebung sind solche, die immer mit den Informationen, die im Klassenzimmer oder Lehrbuch gegeben wurden, lösbar sind.
Man muß nur die richtigen Teile zusammenfügen, um auf die Antwort zu kommen.
Ich habe mir deshalb keine Gedanken darüber gemacht, ob ich in der Lage wäre, das Problem sofort zu lösen, sondern habe nur intensiv darüber nachgedacht, um sicherzustellen, daß ich das ganze Kursmaterial studiert hatte.
Wenn ich ein Problem nicht sofort lösen konnte, wußte ich, daß mein Unterbewußtsein weiter daran arbeiten würde, so daß ich das Problem einfach vergessen und später dazu zurückkehren konnte.
Somit war es nur erforderlich, daß ich nicht bis zur letzten Minute wartete, um zu versuchen solche Probleme zu lösen.
Einige Zeit danach würde die Antwort plötzlich in meinem Kopf auftauchen, oftmals zu merkwürdigen und unerwarteten Gelegenheiten.
Sie tauchten meistens am frühen Morgen auf, wenn mein Geist erholt und frisch war.
Man kann also der Erfahrung nach sowohl lernen, dem Unterbewußtsein das Material zu präsentieren, als auch die Schlußfolgerungen daraus zu empfangen.
Im allgemeinen kam die Antwort nicht, wenn ich mein Unterbewußtsein absichtlich darum bat, sondern sie kam, wenn ich etwas tat, das mit dem Problem nicht in Zusammenhang stand.
Man kann das Unterbewußtsein auch benutzen, um sich an etwas zu erinnern, das man vergessen hat.
Versuchen Sie zunächst, sich so gut Sie können daran zu erinnern, und bemühen Sie sich dann für eine Weile überhaupt nicht mehr.
Nach einiger Zeit wird sich Ihr Gehirn oftmals für Sie daran erinnern.

Selbstverständlich kennen wir bis jetzt noch keinen direkten Weg, uns mit unserem Unterbewußtsein zu unterhalten.
Und diese Kommunikationskanäle sind von Person zu Person sehr verschieden, so daß jede Person experimentieren muß, um herauszufinden was am besten funktioniert.
Klar kann man die Kommunikation mit ihm sowohl verbessern, als auch die Kommunikationskanäle blockieren.
Viele meiner clevereren Freunde im College wurden sehr frustriert, wenn sie herausfanden, daß ich die Antwort ohne Anstrengung gefunden hatte, während sie es nicht konnten; und sie wußten, daß sie cleverer waren.
Diese Art der Frustration kann jegliche Kommunikation zwischen den verschiedenen Teilen des Gehirns blockieren.
Es ist besser, eine entspannte, positive Einstellung aufrechtzuerhalten und das Gehirn seine Sache erledigen zu lassen.
Das ist wahrscheinlich der Grund, warum Techniken wie Meditation und Qi Gong so gut funktionieren.
Das sind effektive, lange Zeit getestete, Methoden der Kommunikation mit den verschiedenen Teilen des Gehirns und des Körpers.
Beachten Sie, daß die verschiedenen Teile des Gehirns viele Körperfunktionen direkt kontrollieren, wie z.B. die Herzfrequenz, den Blutdruck, die Atmung, Verdauung, Speichelbildung, die Funktion der inneren Organe, sexuelle Reaktionen usw.
Das sind mächtige Funktionen, die große Mengen von Energie erzeugen oder verschwenden können, so daß wie die Teile reibungslos zusammenarbeiten oder gegeneinander agieren einen wichtigen Effekt auf Ihre allgemeine Gesundheit und geistige Funktionen hat.
Eine weitere wichtige Methode, einen maximalen Nutzen aus dem Unterbewußtsein zu ziehen, ist, es ohne Störung durch das bewußte Gehirn sich selbst zu überlassen, nachdem man ihm das Problem präsentiert hat.
Mit anderen Worten: Sie sollten das Problem vergessen und sich sportlich betätigen, ins Kino gehen oder etwas anderes tun, das Ihnen Spaß macht, und das Unterbewußtsein wird seine Aufgabe besser erfüllen, weil es ein völlig anderer Teil Ihres Gehirns ist.
Wenn Sie die ganze Zeit bewußt über das Problem nachdenken, dann beeinflussen Sie das Unterbewußtsein und erlauben ihm nicht, seine eigenen freien Forschungen zu betreiben.

Das Gehirn hat viele Teile, und es ist von Vorteil, jedes Teil zu kennen und zu lernen, wie man es benutzt.
Das unterbewußte Gehirn ist wahrscheinlich eines der am meisten zu wenig genutzten Teile unseres Gehirns, weil zu vielen von uns seine Existenz nicht bewußt ist.
Es muß bestimmt noch viele andere nützliche Teile unseres Gehirns geben.
So gibt es z.B. zahlreiche automatische Gehirnprozesse, die unser tägliches Leben beeinflussen.
Wenn wir ein Bild mit unseren Augen sehen, geschehen viele Dinge sofort und automatisch.
Wenn man ein Bild empfängt, wird das Gehirn vorübergehend mit der Informationsverarbeitung überladen, so daß es andere Aufgaben nicht gut ausführen kann.
Deshalb spürt man auch mit geöffneten Augen weniger Schmerzen als mit geschlossenen.
Ein ähnlicher Effekt tritt bei Geräuschen auf.
Deshalb vermindert Schreien bei Schmerzen tatsächlich die Schmerzen.
Der angenehme Klang der Musik ist eine weitere automatische Reaktion, genauso ist es bei Reaktionen auf visuelle Eingaben wie schönen Blumen, beruhigenden Panoramablicken von Bergen oder Seen, oder der Wirkung von unangenehmen oder angenehmen Düften.
Es ist eine dieser automatischen Reaktionen, die wir aufrufen, wenn wir Musik anhören;  trotzdem, gerade so wie wir nicht genau erklären können, warum eine schöne Blume schön aussieht, können wir nicht genau erklären, warum Musik so gut klingt.
Vielleicht ist es eine von diesen festverdrahteten unterbewußten Reaktionen.

Die Identifizierung der verschiedenen Teile des Gehirn muß sicherlich eine der zukünftigen Revolutionen sein.
Die medizinische Wissenschaft schreitet immer schneller voran, und das Gehirn zu verstehen wird einer der größten Durchbrüche sein, angefangen damit, wie es sich in der Kindheit entwickelt und wie wir diese Entwicklung erleichtern können.
Deshalb ist es voll und ganz möglich, daß Mozart kein musikalisches Genie war, sondern ein Genie, das durch die Musik erzeugt wurde.
 





<!-- reference.html --> 

\label{reference}

<h2><br>\underline{Quellenverzeichnis}</h2>

<h3><br>\underline{Buchbesprechungen}</h3>

\paragraph{Allgemeine Schlußfolgerungen aus den besprochenen Büchern}
\label{allgemein}

\begin{enumerate}[label={\arabic*.}] 
\item 
Die Klavierliteratur hat sich in den letzten 100 Jahren von der Aufmerksamkeit auf die Finger und Fingerübungen zum Gebrauch des ganzen \hyperref[c1iii4c]{Körpers}, zur \hyperref[c1ii14]{Entspannung} und der \hyperref[c1iii14d]{musikalischen Aufführung} hin entwickelt.
Deshalb enthalten die älteren Veröffentlichungen oftmals Konzepte, die nun angezweifelt werden.
Das bedeutet nicht, daß Mozart, Beethoven, Chopin und Liszt nicht die richtige Technik hatten, sondern daß in der Literatur hauptsächlich die großen Auftritte aufgezeichnet wurden aber nicht das, was man tun mußte, um so gut zu werden.
Kurz gesagt: Die Klavierliteratur war bis in die heutige Zeit in beklagenswerter Weise unzulänglich.

\item Ein Konzept, das sich nicht geändert hat, ist, daß musikalische Gesichtspunkte, wie \hyperref[c1iii1b]{Rhythmus}, \hyperref[c1iii1]{Klang}, Phrasierung usw., nicht von der Technik getrennt werden können.

\item Fast jedes Buch beschäftigt sich mit einer Auswahl derselben Themen; die Hauptunterschiede liegen im Ansatz und im Detailreichtum.
Fast alle behandeln nur einzelne Aspekte und sind unvollständig.
Sie behandeln zunächst den menschlichen Geist und die menschliche Anatomie sowie ihr Verhältnis zum Klavier: geistige Haltung und Vorbereitung, \hyperref[c1ii3]{Sitzposition, Bankhöhe}, Rolle der \hyperref[c1ii10]{Arme}, \hyperref[c1iii4]{Hände} und \hyperref[c1iii4b]{Finger} - oft mit entsprechenden Übungen und einer Besprechung von \hyperref[c1iii10hand]{Verletzungen}.
Sodann Konzepte der Technik und Musikalität: Anschlag, \hyperref[c1iii1]{Klang}, \hyperref[c1iii5a]{Daumen}, Legato, \hyperref[c1iii1c]{Staccato}, \hyperref[c1ii18]{Fingersätze}, \hyperref[c1iii5a]{Tonleitern}, \hyperref[Arpeggios]{Arpeggios}, Oktaven, \hyperref[c1iii7e]{Akkorde}, wiederholte Noten, \hyperref[c1ii13]{Geschwindigkeit}, Glissando, \hyperref[c1ii23]{Pedal}, Übungszeit, \hyperref[c1iii6]{Auswendiglernen} usw.
Es gibt erstaunlich wenig Literatur über das \hyperref[c1iii11]{Blattspiel}.
\label{c030530}\footnote{Eine sehr aufschlußreiche und verständliche Arbeit zum Vom-Blatt-Spielen mit einer eigenen Konzeption ist die Dissertation \enquote{Zur Methodik des elementaren Prima-Vista-Spiels} (2001/2002, 448 S.) von Dr. Bernd Sommer.
Eine Zusammenfassung der Arbeit und die Möglichkeit zum Herunterladen der Dissertation finden Sie \hyperref[http://www.dissertation.de/buch.php3?buch=1405]{hier} (extern) (mit freundlicher Genehmigung von Dr. Sommer).}

\item Von ein paar älteren Ausnahmen abgesehen, raten die meisten vom Gebrauch des \hyperref[c1iii5a]{Daumenuntersatzes} zum Spielen von Tonleitern ab; der Daumenuntersatz ist jedoch für einige bestimmte Anwendungen eine wertvolle Bewegung.
Chopin bevorzugte das Untersetzen des Daumens für sein Legato, lehrte aber das \hyperref[c1iii5a2]{Übersetzen} wo es technisch vorteilhaft war.

\item Der Mangel an Quellenangaben in vielen Büchern spiegelt die Tatsache wieder, daß die Lehrmethoden für das Klavierspielen niemals ausreichend oder richtig dokumentiert wurden.
Jeder Autor mußte im Prinzip jedesmal das Rad neu erfinden.
Das zeigt sich auch in den aktuellen Lehrmethoden.
Die Lehrmethoden für das Klavier wurden im Grunde durch die Worte aus dem Mund des Lehrers an den Schüler weitergegeben, was an die Art erinnert, in der prähistorische Menschen ihre Überlieferungen und medizinische Praktiken über Generationen hinweg weitergaben.
Dieser grundlegende Makel brachte die Entwicklung der Lehrmethoden fast zum Stillstand, und sie haben sich im Grunde über Jahrhunderte hinweg nicht verändert.

\hyperref[Whiteside]{Whitesides Buch} wurde deshalb weithin anerkannt, weil es der erste wirkliche Versuch war, mit einer wissenschaftlichen Vorgehensweise die besten Übungsmethoden zu entdecken.
Gemäß den Überlieferungen wurden jedoch die meisten ihrer \enquote{Entdeckungen} von Chopin gelehrt; offenbar stand diese Information Whiteside nicht zur Verfügung.
Es mag jedoch mehr als bloßer Zufall sein, daß sie Chopins Musik ausgiebig in Ihren Lehren benutzte.
Whitesides Buch versagte kläglich, denn obwohl sie Experimente durchführte und ihre Ergebnisse dokumentierte, benutzte sie keine klare Sprache, ordnete ihre Ergebnisse nicht, führte keine Analyse von Ursache und Wirkung durch usw., was für ein gutes wissenschaftliches Projekt notwendig ist.
Trotzdem war ihr Buch zur Zeit seiner Veröffentlichung - wegen der minderen Qualität aller anderen - eines der besten erhältlichen Bücher.

Eine ungeheure Anzahl Lehrer behauptet, die Liszt-Methode zu lehren, aber es gibt nur eine fragmentarische und herzlich wenig Dokumentation darüber, was diese Methode ist.
Es gibt reichlich Literatur darüber, wohin Liszt reiste, wen er traf und unterrichtete, was er spielte und welche wunderbaren Meisterleistungen des Klavierspiels er vollführte, aber es gibt praktisch keine Aufzeichnungen davon, was ein Schüler tun muß, damit er in der Lage ist so zu spielen.

\item \hyperref[Chang]{Changs Buch} (besonders diese zweite Ausgabe) ist das einzige, das die Übungsmethoden dafür zur Verfügung stellt, bestimmte anfängliche technische Probleme (Überwindung von \hyperref[c1iv2b]{Geschwindigkeitsbarrieren}, \hyperref[c1ii14]{Entspannung}, \hyperref[c1ii21]{Ausdauer}, \hyperref[c1iii6]{Auswendiglernen}, \hyperref[c1iii6h]{langsames} gegenüber \hyperref[c1ii13]{schnellem} Üben usw.) zu lösen, die im Anfängerstadium gelernt werden sollten aber nicht immer gelehrt werden.
Die anderen Bücher behandeln hauptsächlich die \enquote{höheren} Stufen des Klavierspielens und nehmen an, daß der Schüler die grundlegenden Techniken auf irgendeine magische Weise erworben hat.
Offensichtlich ist es wichtig, diese Fertigkeiten der \enquote{höheren Stufen} von Anfang an zu lernen, so daß Changs Buch eine große Lücke in der Literatur über das Erwerben von Technik füllt.

 \end{enumerate}
<br>\textbf{Format der Darstellung:} Autor, Titel, Erscheinungsjahr, Anzahl der Seiten und ob Quellen im Buch angegeben werden.<br>
Die Quellen sind ein Indiz dafür, wie wissenschaftlich das Buch ist.
Nach diesem Kriterium ist Changs erste Ausgabe nicht wissenschaftlich; dieser Mangel wurde in der zweiten Ausgabe beseitigt.
Die Besprechungen erheben nicht den Anspruch der Objektivität und sind nicht umfassend; sie befassen sich hauptsächlich damit, wie relevant diese Bücher für den Klavierschüler sind, der sich für Klaviertechnik interessiert.
Das meiste \enquote{irrelevante} Material wurde ignoriert.


\label{Bree}

<br>\textbf{Bree, Malwine}, \enquote{The Leschetizky Method}. 1997 (1913), 92 S., keine Quellenangaben.<br>
Obwohl dieses Buch 1997 erschienen ist, ist es eine Wiederveröffentlichung des Materials von 1913.<br>
Abstammungslinie der Unterrichtsmethode: Beethoven-Czerny-Leschetizky-Bree.<br>  Buch mit Übungen für das Entwickeln der Technik, Fotos der Fingerpositionen.
Befürwortet den Daumenuntersatz.
Handposition, Übungen für die Unabhängigkeit der Finger, Tonleitern, Akkorde, Anschlag, Glissando, Pedal, Auftritte usw., eine ziemlich vollständige Abhandlung.
Lesen Sie das, um etwas über die älteren \enquote{etablierten} Methoden herauszufinden.


\label{Bruser}

<br>\textbf{Bruser, Madeline}, \enquote{The Art of Practicing}. 1997, 272 S., Quellenangaben (artofpracticing.com).<br>
Basiert darauf, zunächst den Geist (Meditation) und Körper (Dehnungsübungen) vorzubereiten, und geht dann zu einigen nützlichen Einzelheiten der Fertigkeiten für das Klavierspielen über.
Die Menge der Anweisungen für das Klavierspielen wird leider dadurch reduziert, daß ebenfalls Anweisungen für andere Instrumente (hauptsächliche Saiten- und Blasinstrumente) gegeben werden.
Obwohl körperliches Training (leichte Gymnastik) gut ist, sind Übungen wie Tonleitern nicht hilfreich.
Enthält wenige nützliche Informationen.


\label{Chang}

<br>\textbf{Chang, Chuan C.}, \enquote{\hyperref[http://www.pianopractice.org]{Fundamentals of Piano Practice}} (extern), erste Ausgabe. 1994, 130 S., keine Quellenangaben (2. Ausgabe mit Quellenangaben).<br> Abstammungslinie der Unterrichtsmethode: Long-Combe<br>
Lehrt die grundlegendsten Methoden für das schnelle Erwerben der Technik (Üben mit getrennten Händen, Akkord-Anschlag [parallele Sets], schwierige Passagen kürzen, Auswendiglernen, Entspannung, Geschwindigkeitsbarrieren eliminieren usw.).
Kein anderes Buch bespricht alle dieser wesentlichen Elemente, die für einen schnellen Fortschritt und eine korrekte Technik notwendig sind.
Behandelt auch das Spielen vom Blatt, Vorbereitung auf Konzerte, Kontrolle der Nervosität, Anschlag mit freiem Fall, welche Übungen gut sind und welche nutzlos oder schädlich, Lernen des absoluten Gehörs, Konturieren usw.
Enthält ein Kapitel über das Klavierstimmen für den Amateur, erklärt die chromatische Tonleiter und das Temperieren.
Die auf dieser Website \hyperref[Inhalt]{zum freien Download vorliegende zweite Ausgabe} ist eine aktualisierte und erweiterte Fassung der ersten Ausgabe.
\textbf{Muß man gelesen haben.}


\label{Eigeldinger}

<br>\textbf{Eigeldinger, Jean-Jacques,}\enquote{Chopin, pianist and teacher as seen by his pupils}. 1986, 324 S., Quellenangaben.<br>
Die wissenschaftlichste und vollständigste Zusammenfassung relevanten Materials über Chopin in bezug auf Unterricht, Technik, Interpretation und Geschichte.
Wegen des Mangels an direkter Dokumentation zu Chopins Zeit ist praktisch das ganze Material anekdotenhaft.
Trotzdem scheint die Genauigkeit wegen der umfassenden Dokumentation, dem Fehlen jeglicher erkennbarer systematischer Fehler und der offensichtlichen Tatsache, daß solch ein tiefes Verständnis nur von Chopin selbst gekommen sein kann, unzweifelhaft zu sein - die Ergebnisse sind in verblüffender Übereinstimmung mit dem besten heute verfügbaren Material.
Eigeldinger hat die Themen in hilfreichen Gruppen angeordnet (Technik, Interpretation, Zitate, kommentierte Notenblätter und Fingersätze, Chopins Stil).
Ich würde mir wünschen, daß es mehr Übungsmethoden enthielte, aber wir müssen alle zur Kenntnis nehmen, daß der Mangel an Dokumentation zu Chopins Zeit zum Verlust eines großen Teils dessen was er gelehrt hat führte.
Im Fall von Franz Liszt ist die Situation weitaus schlechter.

Die technischen Lehren werden kurz und prägnant auf den Seiten 23-64 präsentiert.
Diese Lehren sind fast in völliger Übereinstimmung mit denen der besten Quellen, von \hyperref[Walker]{Liszt} und \hyperref[Whiteside]{Whiteside} bis zu \hyperref[Fink]{Fink}, \hyperref[Sandor]{Sandor}, \hyperref[Suzuki]{Suzuki} und \hyperref[Chang]{diesem Buch (Chang)}.
Die Präsentation steht in starkem Kontrast zu Whiteside; hier ist sie autoritativ (Whiteside nimmt manchmal ihre eigenen Ergebnisse zurück), kurz (nur 41 Seiten verglichen mit 350 Seiten bei Whiteside!), organisiert und klar, wobei ein ähnlicher Themenbereich abgedeckt wird.
Der zweite Teil, Seiten 65-89, behandelt die Interpretation und enthält deshalb viel weniger Informationen über Technik, ist jedoch genauso informativ wie der erste Abschnitt.
Er befaßt sich (sehr!) kurz damit, wie man jede einzelne von Chopins Hauptkompositionen interpretiert.
Die verbleibenden 200 Seiten widmen sich der Dokumentation, Illustrationen, Chopins Anmerkungen zu seinen eigenen Kompositionen und Fingersätzen und einem 10 Seiten umfassenden \enquote{Entwurf} von grundlegendem Material für den Anfängerunterricht.

Anmerkungen zur Technik: Chopin war Autodidakt; es ist wenig darüber bekannt, wie er lernte als er jung war, außer, daß er von seiner Mutter unterrichtet wurde, die eine vollendete Pianisten war.
Chopin glaubte nicht an Drill und Übungen (er empfahl nicht mehr als 3 Stunden Übung pro Tag).
Chopins Methoden stehen nicht in dem Maß, wie es zunächst erscheinen mag, im Gegensatz zu Liszts, obwohl Liszt häufig mehr als 10 Stunden täglich übte und Übungen \enquote{bis zur Erschöpfung} empfahl.
Chopin schrieb wie Liszt Etüden, und Liszts \enquote{Übungen} waren keine stupiden Wiederholungen, sondern zum Erwerb der Technik bestimmte Methoden.

Man soll lernen, Musik zu machen, \textit{bevor} man Technik erlernt.
Der ganze Körper muß einbezogen werden, und der Gebrauch des Armgewichts (Freier Fall) ist ein Schlüsselelement der Technik.
Er lehrte sowohl den Daumenübersatz (besonders wenn die passierte Note schwarz ist!) als auch den Daumenuntersatz und gestattete es sogar, jeden Finger über jeden anderen rollen zu lassen, wann immer es vorteilhaft war - der Daumen war nicht einzigartig und mußte \enquote{frei} sein.
Jeder Finger war jedoch unterschiedlich.
Das Übersetzen des Daumens (genau wie das anderer Finger) war besonders nützlich bei beidhändigen chromatischen Tonleitern (Terzen usw.).
Bei Chopin mußte das Klavier sprechen und singen; für Liszt war es ein Orchester.
Da die C-Dur-Tonleiter schwieriger ist, benutzte er die H-Dur-Tonleiter, um Entspannung und Legato zu lehren; ironischerweise ist es besser, die Tonleiter zunächst staccato zu lernen, um die schwierigen Probleme mit dem Legato zu eliminieren, obwohl er am Ende immer zu seiner Spezialität zurückkommt - dem Legato.
Große Arpeggios erfordern eher eine geschmeidige Hand als eine große Reichweite.
Beim Rubato wird der Rhythmus streng eingehalten, während die Zeit im Verlauf der Melodie ausgeliehen und wieder zurückgegeben wird.
[Meiner Meinung nach wird diese Definition oft falsch zitiert und falsch verstanden; nur weil er das ein paarmal gesagt hat, bedeutet es nicht, daß er es auf alles angewandt hat.
Diese Definition des Rubato gilt besonders für die Situation, in der die RH rubato spielt, während die LH im strikten Zeitmaß bleibt.
Chopin hat sicherlich auch erlaubt, daß Rubato eine Freiheit vom strikten Tempo zugunsten des Ausdrucks war.]
Chopin bevorzugte den Pleyel, ein Klavier mit sehr leichter Mechanik.
Seine Musik ist auf modernen Instrumenten eindeutig schwieriger zu spielen, besonders das Pianissimo und Legato.
\textbf{Muß man gelesen haben}.


\label{Fink}

<br>\textbf{Fink, Seymour}, \enquote{Mastering Piano Technique}, 1992, 187 S., ausgezeichnetes Quellenverzeichnis; Video ebenfalls erhältlich.<br>
Das wissenschaftlichste der hier aufgeführten Bücher, wie es sich für einen Universitätsprofessor gebührt.
Wissenschaftliche Abhandlung, die die korrekte Terminologie benutzt (im Gegensatz zu \hyperref[Whiteside]{Whiteside}, die häufig nichts von der Standardterminologie wußte), leicht verständlich, beginnt mit der menschlichen Anatomie und ihrer Beziehung zum Klavier, gefolgt von einer Auflistung der Bewegungen, die ins Spielen einbezogen sind, einschließlich des Pedals.
Tonleitern dürfen nicht mit Daumenuntersatz gespielt werden, aber der Daumenuntersatz ist eine wichtige Bewegung (S. 115).
Veranschaulicht jede Bewegung und die zugehörigen Klavierübungen.
Gute Beschreibung des Freien Falls.
Strikter mechanischer Ansatz, aber das Buch betont das Erzeugen eines volleren Tons und das emotionale Spielen.
Die Bewegungen sind aus den Diagrammen schwer zu entnehmen und machen den Kauf des Videos wünschenswert.
Sie müssen entweder Fink oder \hyperref[Sandor]{Sandor} lesen; vorzugsweise beide, da sie ähnliche Themen von verschiedenen Standpunkten aus angehen.
Einige Leser werden den einen mögen und den anderen ablehnen.
Fink basiert auf Übungen, Sandor mehr auf Beispielen von klassischen Kompositionen.

Die erste Hälfte ist eine Abhandlung aller grundlegenden Bewegungen und von Übungen für diese Bewegungen.
Diese schließen ein: Pronation, Supination, Wegführen, Heranziehen, Handpositionen (ausgestreckt, Handfläche, krallen), Fingerschläge, Bewegungen des Unterarms, Oberarms, der Schulter (Schub, Zug, Zirkulieren) usw.
Der zweite Teil wendet diese Bewegungen auf Beispiele aus berühmten Klassikern von Ravel, Debussy und Rachmaninoff bis Chopin, Beethoven, Mozart und vielen anderen an. \textbf{Man muß entweder dieses Buch oder Sandor gelesen haben}.


\label{Gieseking}

<br>\textbf{Gieseking, Walter und Leimer, Karl}, \enquote{Modernes Klavierspiel}, 2 Bücher in einem, 1972, keine Quellenangaben.<br>
Abstammungslinie der Unterrichtsmethode: Leimer-Gieseking.
\textbf{Erstes Buch}: Gieseking, \enquote{Modernes Klavierspiel}, 77 S.<br>
Wichtigkeit des Zuhörens, \enquote{Ganzkörper}-Methode (wie bei der Armgewichtsschule), Konzentration, präzises Üben, Aufmerksamkeit auf die Details.
Hervorragende Behandlung, wie man eine Komposition für das Üben und Auswendiglernen analysiert.
Dieses Buch ist ein Vertreter der meisten Bücher, die von jenen großen Künstlern geschrieben wurden.
Ein typischer Rat in bezug auf die Technik ist \enquote{Konzentration, präzises Üben und Aufmerksamkeit auf die Details führt automatisch zur Technik.} oder \enquote{Benutzen Sie Ihre Ohren.} oder \enquote{Alle Noten eines Akkords müssen zusammen klingen.} ohne jeden Hinweis darüber, wie man jede einzelne Fertigkeit tatsächlich erlangt.

Führt Ihnen vor, wie man Bachs zweistimmige C-Dur-Invention (\#1) übt, die dreistimmige C-Dur-Invention (\#1) und Beethovens Sonate \#1, aber mehr von der Analyse und Interpretation her als vom Standpunkt der technischen Fertigkeiten.
Er führt Sie durch die ersten 3 Sätze von Beethovens Sonate, entläßt aber den technisch anspruchsvollsten 4. Satz mit \enquote{bringt keine weiteren neuen Probleme}!
Beachten Sie, daß dieser letzte Satz einen starken, schwierigen und sehr schnellen 5,2,4-Fingersatz, gefolgt von einem absteigenden Arpeggio mit Daumenübersatz in der LH sowie schnelle und akkurate weite Akkordsprünge in der RH erfordert.
Bei diesen hätten wir uns ein wenig Rat von Gieseking gewünscht.
Changs Buch schließt diese Lücke, indem es die Anleitung in \hyperref[c1iii8]{Kapitel 1, Abschnitt III.8} liefert.
Lesenswert, sogar wenn es nur wegen der speziellen Führung durch die oben angegebenen Stücke ist.

\textbf{Zweites Buch}: Leimer, \enquote{Rhythmik, Dynamik, Pedal und andere Probleme des Klavierspiels}, 56 S.<br>
Wichtigkeit von Rhythmus, Zählen, akkuratem Timing, Phrasieren.
Ausgezeichneter Abschnitt über den Gebrauch des Pedals.
Enthält einige spezielle Informationen, die woanders schwer zu finden sind.


\label{Green}

<br>\textbf{Green, Barry, und Gallwey, Timothy}, \enquote{The Inner Game of Music}, 1986, 225 S., keine Quellenangaben.<br>
 Mentales Herangehen an Musik; Entspannung, Bewußtsein, Vertrauen.
Fast keine technischen Anweisungen zum Klavierspielen.
Nur für diejenigen, die glauben, daß geistige Haltung der Schlüssel zum Klavierspielen ist.
Wer an bestimmten Rezepten für das Üben interessiert ist, wird wenige nützliche Informationen finden.


\label{Hofman}

<br>\textbf{Hofman, Josef}, \enquote{Piano Playing, With Piano Questions Answered}, 1909, 183 S., keine Quellenangaben.<br>
Abstammungslinie der Unterrichtsmethode: Moszkowki, Rubinstein.<br>
Die erste Hälfte behandelt sehr nützliche allgemeine Regeln, und die zweite Hälfte ist in Frage-und-Antwort-Form.
Der größte Teil des Buchs bespricht generelle Konzepte; nicht viele detaillierte technische Anweisungen.
Kein unentbehrliches Buch für Technik, ist aber gut nebenbei zu lesen.


\label{Lhevine}

<br>\textbf{Lhevine, Josef}, \enquote{Basic Principles in Piano Playing}, 1972, 48 S., keine Quellenangaben.<br> Ausgezeichnete Behandlung wie man einen guten Klang erzeugt.
Kurze Besprechung von: Grundwissen der Tonarten, Tonleitern usw., Rhythmus, Gehörtraining, leise und laut, Genauigkeit, Staccato, Legato, Auswendiglernen, Übungszeit, Geschwindigkeit, Pedal.
Meistens oberflächlich - das Buch ist zu kurz.
Gute allgemeine Zusammenfassung, es fehlen aber nähere Einzelheiten und es enthält kein Material, das man nicht auch woanders findet.


\label{Prokop}

<br>\textbf{Prokop, Richard}, \enquote{Piano Power, a Breakthrough Approach to Improving your Technique}, 1999, 108 S., sehr wenige Quellenangaben.<br>
Die Einführung liest sich, als ob das das Buch wäre, auf das jeder gewartet hat.
Je mehr man jedoch liest, desto desillusionierter wird man.
Der Autor - Pianist, Klavierlehrer und Komponist - begann das Klavierspielen mit der \enquote{intuitiven Methode} (s. \hyperref[c1ii1]{Kapitel 1 Abschnitt II} von Chang) zu lernen, und seine Lehren bestehen immer noch zu 50\% daraus.
Er kennt z.B. nicht den Daumenübersatz und trifft deshalb auf viele \enquote{Probleme}.
Die Lehren bestehen aus \enquote{Theoremen}, die er \enquote{beweist}.
Wenn man nur ein paar solcher Theoreme liest, zeigt sich, daß man bei der Klaviertechnik Theoreme nicht wie in der Mathematik beweisen kann, was im Grunde das ganze Buch widerlegt.
Er bringt ein paar nützliche Ideen zur Sprache:

\begin{enumerate} 
 \item Wichtigkeit der Streckmuskeln (Anheben der Finger); akkurates Anheben der Finger (und Pedale) ist genauso wichtig wie ein akkurater Anschlag.
Er stellt Übungen für das Anheben jedes Fingers zur Verfügung und gibt die beste Beschreibung der Knochen, Sehnen und Muskeln der Finger, Hände und Arme und wie bzw. welche Bewegungen durch sie kontrolliert werden.
 \item Detaillierte Analyse der Vor- und Nachteile von kleinen, mittelgroßen und großen Händen.
\end{enumerate}
Da gute Ideen mit den falschen vermischt sind, kann dieses Buch den weniger informierten Schüler in die Irre führen oder verwirren.
Es gibt keine \enquote{Durchbrüche} (s. Titel); empfehlenswert nur für diejenigen, die nützliche Ideen von den falschen unterscheiden können.


\label{Richman}

<br>\textbf{Richman, Howard,} \enquote{Super Sight-Reading Secrets}, 1986, 48 S., keine Quellenangaben.<br>
Dieses Buch ist das beste Buch über das Notenlesen.
Es enthält alle Grundlagen; sie werden in allen Einzelheiten beschrieben, und wir lernen die ganze korrekte Terminologie und Methodik.
Es beginnt damit, wie man Noten liest (für den Anfänger), und geht auf logische Weise vorwärts bis zu fortgeschrittenen Stufen des Blattspiels; es ist für den Anfänger besonders hilfreich.
Es ist auch kurz und prägnant; Sie sollten deshalb das ganze Buch einmal durchlesen, bevor Sie mit dem eigentlichen Üben anfangen.
Beginnt damit, wie man psychologisch an das Notenlesen herangeht.
Grundlegende Komponenten des Notenlesens sind Tonhöhe, Rhythmus und Fingersatz.
Nach einer hervorragenden Einführung in die Notation werden geeignete Übungen gezeigt.
Anschließend wird der Vorgang des Spielens vom Blatt in die einzelnen Schritte der visuellen, neuralen, muskulären und auralen\footnote{also Auge, Gehirn/Nerven, Muskeln, Ohren} Prozesse zerlegt, die mit dem Notenblatt beginnen und als Musik enden.
Übungen für das Lernen der \enquote{Orientierung auf der Tastatur} (Noten finden, ohne auf die Tastatur zu sehen) und \enquote{visuelle Wahrnehmung} (sofort erkennen, was zu spielen ist) schließen sich an.
In Abhängigkeit von der Person kann es von 3 Monaten bis zu 4 Jahren dauern, es zu lernen; Sie sollten es täglich üben.
Schließlich ungefähr eine Seite Gedanken über fortgeschrittenes Blattspiel.
\textbf{Muß man gelesen haben}.


\label{Sandor}

<br>\textbf{Sandor, Gyorgy}, \enquote{On Piano Playing}, 1995, 240 S., keine Quellenangaben.<br>  Abstammungslinie der Unterrichtsmethode: Bartok-Kodaly-Sandor.<br>  Das vollständigste, wissenschaftlichste und teuerste Buch.
Enthält den größten Teil des Materials aus \hyperref[Fink]{Fink}, betont die \enquote{Armgewichts}-Methoden.
Bespricht: Freier Fall, Tonleitern (Daumenübersatz; hat die detaillierteste Beschreibung von Tonleiter- und Arpeggiospiel, S. 52-78), Drehung, Staccato, Schub, Pedale, Klang, Üben, Auswendiglernen, Auftritte.
Begleitet Sie durch das Lernen der gesamten Waldstein-Sonate (Beethoven).

Zahlreiche Beispiele, wie man die Grundsätze des Buchs auf Kompositionen von Chopin, Bach, Liszt, Beethoven, Haydn, Brahms, Schumann und vielen anderen anwendet.
Dieses Buch ist ziemlich vollständig; es behandelt Themen von der Auswirkung der Musik auf Emotionen bis zu Besprechungen des Klaviers, der menschlichen Anatomie und grundlegenden Spielbewegungen, sowie Auftritte und Aufnahmen; viele Themen werden jedoch nicht hinreichend ausführlich behandelt.
\textbf{Muß man gelesen haben}, aber \hyperref[Fink]{Fink} wird Ihnen ähnliche Informationen zu geringeren Kosten bieten.


\label{Sherman}

<br>\textbf{Sherman, Russell}, \enquote{Piano Pieces}, 1997, keine Quellenangaben.<br>  Besteht aus 5 Abschnitten, die das Spielen, Unterrichten, kulturelle Gesichtspunkte, Notenblätter und \enquote{alles andere} behandeln.
Die Inhalte sind in keiner besonderen Reihenfolge angeordnet, mit keinen wirklichen Lösungen oder Schlußfolgerungen.
Bespricht die Politik der Künste (Musik), Meinungen, Urteile und Beobachtungen, zu denen Pianisten einen Bezug haben; ob Nicht-Pianisten diese Gedankengänge verstehen können, ist fraglich, sie gewähren aber Einblicke.
Sitzposition, Daumen dient als Impulsausgleich.
Finger = Truppen; Körpermitte = Versorgungslinie, Unterstützung, Transportschiff und Herstellung.
Finger gegen Körper = Vertrieb gegen CEO; deshalb resultiert die Kontrolle der Finger nicht in Musik.
Leichte Stücke sind wertvoll zum Lernen, wie man Musik macht.
Was hat man davon, Klavierspielen zu lernen?
Es ist, finanziell gesehen, nicht einmal eine gute Karriere.
Sollte man den Finger gleiten lassen?
Was spielt eine Rolle für die Schönheit oder den Charakter des Klavierklangs?
Wie wichtig sind Qualitätsklaviere und gute Stimmer?
Pro und Kontra von Wettbewerben (hauptsächlich Kontras): Vorbereitung auf Wettbewerbe ist kein Musikmachen und wird oftmals zu einem athletischen Wettbewerb; ist es den Streß und den Aufwand wert?; Urteile sind niemals perfekt.

Handelt von Themen, denen sich Pianisten, Lehrer und Eltern gegenübersehen; beschreibt viele der Hauptprobleme, präsentiert aber wenige Lösungen.
Dieses Buch berührt viele Themen, ist aber so ziellos wie der Titel.
Lesen Sie es nur, wenn Sie Zeit totzuschlagen haben.


\label{Suzuki}

<br>\textbf{Suzuki, Shinichi (et al)}, zwei Bücher (es gibt mehr): \enquote{The Suzuki Concept: An Introduction to a Successful Method for Early Music Education}, 1973, 216 S., keine Quellenangaben, hat eine große, ausgezeichnete Bibliographie.<br>
Hauptsächlich für einen Geigenunterricht, der in jungen Jahren beginnt.
Nur ein kurzes Kapitel (7 Seiten) über Methoden des Klavierunterrichts.

\enquote{HOW TO TEACH SUZUKI PIANO}, 1993, 21 S., keine Quellenangaben.<br>  Eine kurze allgemeine Skizzierung der Suzuki-Klavier-Methoden.
Die von Chang beschriebenen Methoden sind in genereller Übereinstimmung mit den Suzuki-Methoden.
Lassen Sie ein Baby zuhören; kein Beyer, Czerny, Hanon oder Etüden (sogar Chopin!); Auftritte sind ein Muß; Lehrer müssen einheitliche Lehrmethoden haben und offene Diskussionen führen (Forschungsgruppen); Auswendiglernen und Blattspiel müssen ausgewogen sein, aber Auswendiglernen ist wichtiger.
Lehrern wird eine kleine Anzahl abgestufter Stücke angegeben, auf denen Ihr Unterricht basieren soll.
Suzuki ist eine zentral kontrollierte Schule; als solche hat sie viele der Vorteile von den Fakultäten der etablierten Musikhochschulen und Colleges, aber die akademische Stufe ist im allgemeinen niedriger.
Suzuki-Lehrer sind mindestens eine Klasse besser als der durchschnittliche Privatlehrer, weil sie bestimmten Mindestanforderungen genügen müssen.
Beschreibt viele allgemeine Vorgehensweisen beim Unterrichten aber wenig Einzelheiten darüber, wie man Klavier für die Technik übt.
Klassisches Beispiel dafür, wie ein autoritäres System schlechte Lehrer durch das Festlegen von Mindestanforderungen ausschließen kann.


\label{Walker}

<br>\textbf{Walker, Alan}, \enquote{Franz Liszt, The Virtuoso Years, 1811-1847}, 1983, 481 S., Quellenangaben.<br>
Das ist das erste von drei Büchern; es behandelt die Periode von Liszts Geburt bis zu der Zeit, als er sich mit 36 Jahren entschied, nicht mehr aufzutreten.
Das zweite Buch behandelt die Jahre 1848-1861, in denen er sich hauptsächlich dem Komponieren widmete.
Das dritte Buch behandelt die Jahre 1861-1886, seine letzten Jahre.
Ich bespreche hier nur das erste Buch.

Liszt ist als der größte Pianist aller Zeiten bekannt.
Deshalb würden wir erwarten, von ihm am meisten darüber zu lernen, wie man Technik erwerben kann.
Unglücklicherweise ist jedes Buch über Liszt in dieser Hinsicht eine absolute Enttäuschung.
Meine Vermutung ist, daß Technik zu Liszts Zeit so etwas wie ein \enquote{Berufsgeheimnis} war und seine Lektionen nie dokumentiert wurden.
Paganini übte völlig im Geheimen und stimmte sogar seine Geige verdeckt anders, um Resultate zu erzielen, die kein anderer erzielen konnte.
Chopin hingegen war ein Komponist und professioneller Lehrer - das waren die Quellen seiner Einkünfte, und es gibt zahlreiche Berichte über seinen Unterricht.
Liszts Grundlage seines Ruhms waren seine Auftritte.
Sein Erfolg in dieser Hinsicht spiegelt sich in der Tatsache wieder, daß praktisch jedes Buch über Liszt eine endlose und wiederholte Chronik seiner unglaublichen Auftritte ist.
Meine Vermutung über diese Heimlichtuerei würde erklären, warum so viele der Pianisten dieser Zeit behaupten, Schüler von Liszt gewesen zu sein, obwohl sie Liszts Lehrmethoden selten in nützlicher Weise detailliert beschreiben.
Wenn man jedoch diesen Details unter den heutigen Lehrern der \enquote{Liszt-Schule} nachspürt, findet man heraus, daß sie ähnliche Methoden benutzen (getrennte Hände, schwierige Passagen kürzen, Akkord-Anschlag usw.).
Was immer die wahren Gründe sind, Liszts Lehrmethoden wurden nie ausreichend dokumentiert.
Ein Vermächtnis, daß Liszt uns hinterlassen hat, ist die gut dokumentierte Tatsache, daß die Art Meisterleistungen, die er vorführte, menschlich möglich sind.
Das ist wichtig, weil es bedeutet, daß wir alle ähnliche Dinge tun können, wenn wir wieder entdecken, wie er es gemacht hat.
Vielen Pianisten ist das gelungen, und ich hoffe, daß mein Buch ein Schritt in die richtige Richtung dafür ist, die besten bekannten Übungsmethoden zu dokumentieren.

Walkers Buch ist typisch für die anderen Bücher über Liszt, die ich gelesen habe, und ist im Grunde eine Chronik von Liszts Leben und kein Lehrbuch darüber, wie man Klavier lernt.
Als solches ist es eine der besten Biographien und enthält zahlreiche Besprechungen einzelner Kompositionen mit besonderen pianistischen Ansprüchen und Schwierigkeiten.
Leider lehrt uns die Beschreibung einer unmöglichen Passage als \enquote{die mit größter Leichtigkeit ausgeführt wurde} nicht, wie man es machen muß.
Dieses Fehlen der technischen Lehrinformation ist, angesichts der Tatsache, daß die Zahl der bibliographische Berichte über Liszt 10.000 bei weitem übersteigt, überraschend!
Tatsächlich muß jede technische Information, die wir aus diesem Buch entnehmen können, mit Hilfe unseres eigenen Klavierwissens aus dem Inhalt abgeleitet werden (s. u. das Beispiel über die Entspannung).
Der Abschnitt mit der Überschrift \enquote{Liszt und die Tastatur} (S. 285-318) enthält ein paar Fingerzeige darauf, wie man spielt.
Wie in allen drei Büchern, wird Liszt als ein Halbgott verehrt, der nicht fehlgehen kann, ja sogar mit Superhänden ausgestattet ist, die für das Klavier irgendwie ideal gestaltet sind - er konnte eine Dezime leicht erreichen.
Diese Voreingenommenheit vermindert die Glaubwürdigkeit, und die unablässigen, wiederholten Berichte der übermenschlichen Darbietungen erzeugen eine Langeweile, die von der enormen Menge an aufschlußreichen und faszinierenden historischen Details in diesen Büchern ablenkt.

Vom Standpunkt der Klaviertechnik ist vielleicht der interessanteste Punkt, daß Liszt von früher Jugend an ein dünner, kränklicher Mann war.
Tatsächlich wurde er im Alter von drei Jahren nach einer Krankheit als so gut wie tot aufgegeben, und man hatte sogar schon einen Sarg bestellt.
Er fing mit dem Klavierspielen erst mit sechs Jahren an und hatte nicht einmal ein eigenes Klavier zum Üben, bis er sieben war, weil seine Familie so arm war.
Er wurde von seinem Vater unterrichtet, einem talentierten Musiker und passablen Pianisten und wurde von Geburt an mit Musik getränkt.
Czerny war sein erster \enquote{richtiger} Lehrer, im Alter von 11, und Czerny behauptet, er habe Franz dessen ganzen grundlegenden Fertigkeiten beigebracht.
Er gibt jedoch zu, daß Franz bereits ein offensichtliches Wunderkind war, als sie einander vorgestellt wurden - was verdächtig widersprüchlich scheint.
Franz rebellierte sogar gegen Czernys Drill, machte aber von Übungen für seine technische Entwicklung ausgiebigen Gebrauch.
Was er übte waren die Grundlagen: Läufe, Sprünge, wiederholte Noten.
Meine Interpretation ist, daß dieses keine stupiden Wiederholungen für den Muskelaufbau waren, sondern Übungen mit bestimmten Zielen hinsichtlich der Fertigkeiten, und wenn die Ziele erreicht waren, wandte er sich neuen zu.

Aber wie führt ein schwächlicher Mensch \enquote{unmögliche} Übungen bis zur Erschöpfung aus?
Indem er entspannt!
Liszt könnte, aus der Notwendigkeit heraus, der größte Experte der Welt für Entspannung gewesen sein.
Was die Entspannung angeht, mag es kein Zufall sein, daß Paganini ebenfalls ein kränklicher Mann war.
Zu der Zeit als er berühmt wurde, in seinen Dreißigern, hatte er Syphilis, und seine Gesundheit verschlechterte sich aufgrund seiner Leidenschaft für das Glücksspiel und der Ansteckung mit Tuberkulose.
Und doch waren diese beiden Männer mit schlechter Gesundheit die beiden größten Meister ihrer Instrumente.
Die Tatsache, daß beide körperlich schwach waren, zeigt, daß die Energie für übernatürliche Darbietungen nicht von einer athletischen Muskelkraft kommt, sondern eher von einer völligen Beherrschung der Entspannung.
Chopin war ebenfalls eher schwächlich und steckte sich mit Tuberkulose an.
Eine traurige historische Notiz sind - zusätzlich zu Paganinis schlechter Gesundheit und den Auswirkungen der primitiven chirurgischen Versuche dieser Zeit - die Umstände seines schrecklichen Todes, da es eine Verzögerung bei seinem Begräbnis gab und man ihn in einem betonierten Wasserbehälter verwesen ließ.

Ein weiterer bemerkenswerter Lehrer von Liszt war Salieri, der ihn Komposition und Theorie lehrte.
Damals war Salieri über 70 Jahre alt und hatte jahrelang unter dem Verdacht gelitten, er hätte Mozart aus Eifersucht vergiftet.
Liszt verbesserte sich im Alter von 19 Jahren immer noch.
Seinen Meisterleistungen wird die Popularisierung des Klaviers zugeschrieben.
Ihm wird die Erfindung des Klavierkonzerts zugeschrieben (indem er es aus dem Salon in die Konzerthalle brachte).
Eines seiner Mittel war sowohl der Gebrauch mehrerer Klaviere als auch der Auftritt mehrerer Pianisten.
Er spielte sogar Konzerte mit mehreren Klavieren mit Chopin und anderen Koryphäen seiner Zeit.
Dies gipfelte in phantastischen Kompositionen mit bis zu 6 Klavieren, die als \enquote{Konzert mit 60 Fingern} beworben wurden.
Innerhalb von 10 Wochen spielte er 21 Konzerte und 80 Werke, 50 aus dem Gedächtnis.
Daß er sein Publikum so begeistern konnte, war das überraschendere, weil angemessene Klaviere (Steinway, Bechstein) bis in die 1860er Jahre nicht verfügbar waren - fast 20 Jahre nachdem er aufhörte, Konzerte zu geben.

Ich habe dieses Buch mit der Absicht gelesen, Information darüber herauszuziehen, wie man Klavier übt.
 Wie Sie sehen, können wir heute vom größten Pianisten aller Zeiten fast nichts darüber erfahren,
 wie man Klavier übt, obwohl seine Lebensgeschichte ein faszinierender Lesestoff ist.


\label{Werner}

<br>\textbf{Werner, Kenney}, \enquote{Effortless Mastery}, 191 S., plus Meditations-CD, ein paar Quellenangaben und viele Vorschläge für Material zum Anhören.<br>
Mentaler bzw. spiritueller Ansatz zum Musizieren; fast keine Beschreibungen der Mechanik des Spielens, Schwerpunkt ist die Meditation.
Das Buch ist wie eine Autobiographie, und die Lektionen werden so gelehrt, wie er sie während seines Lebens gelernt hat.
In derselben Kategorie wie \hyperref[Green]{Green und Gallwey} aber ein anderer Ansatz.


\label{Whiteside}

<br>\textbf{Whiteside, Abby}, \enquote{On Piano Playing}, 2 Bücher in einem, 1997, keine Quellenangaben.<br>
Das ist eine Neuauflage von \enquote{Indispensables of Piano Playing} (1955), und \enquote{Mastering Chopin Etudes and Other Essays} (1969).<br>
Abstammungslinie der Unterrichtsmethode: Ganz-Whiteside.

\textbf{Erstes Buch}: \enquote{Indispensables of Piano Playing}, 155 S.<br>
Benutzt kein Standard-Englisch, verworrene Logik, biblische Phrasen, unnötig langatmig.
Die Inhalte sind ausgezeichnet aber der fürchterliche Schreibstil macht das Lernen unproduktiv.
Viele der Ideen, die sie beschreibt, tauchen in anderen Büchern auf, aber es kann sein, daß sie die meisten hervorgebracht (oder wiederentdeckt) hat.
Obwohl ich Schwierigkeiten hatte, das Buch zu lesen, haben andere behauptet, daß es leichter zu verstehen sei, wenn man es schnell lesen kann.
Das kommt zum Teil daher, daß sie sich ständig wiederholt und oft einen Absatz oder sogar eine Seite braucht, um etwas zu beschreiben, das man in einem Satz sagen kann.

Fast das ganze Buch ist so (S. 54): \enquote{Frage: Kann Gewicht - ein regloser Druck - dabei helfen, Gewandtheit zu erlangen?
Antwort: Es ist genau der reglose Druck des Gewichts, der nicht für die Geschwindigkeit benutzt werden kann.
Worte sind beim Unterrichten wichtig.
Worte der Bewegung sind notwendig, um die Koordination für die Geschwindigkeit zu bewirken.
Gewicht bewirkt nicht die Muskelaktivität, die das Gewicht des Arms bewegt.
Es bewirkt einen reglosen Druck.}
Ich habe diesen Abschnitt nicht deshalb gewählt, weil er besonders verworren wäre - ich habe ihn zufällig ausgewählt, indem ich das Buch mit geschlossenen Augen geöffnet habe.

Inhalt: Man muß ihren Methoden religiös folgen; warum Rhythmus wichtig ist; es gibt unendlich viele Möglichkeiten für die Kombination des Körpers, der Arme, Hände und Finger, von denen uns die meisten nicht bewußt sind; Tonleitern mit Daumenuntersatz werden geschmäht; Funktionen jedes Teils der Anatomie für das Klavierspielen (horizontale, Einwärts-, Auswärts-, vertikale Bewegungen); Besprechung des Erzeugens von Emotionen, Auswendiglernen, Pedale, Phrasieren, Triller, Tonleitern, Oktaven, Lehrmethoden.
Stellt die Wichtigkeit des Rhythmus für die Musik heraus und wie man diesen durch Konturieren erreicht (S. 141).
Czerny und Hanon sind nutzlos oder schlimmer.

Das folgende ist ihr Angriff auf das Passieren mit Daumenuntersatz für das Spielen von Tonleitern (in verständlicherer Sprache), herausgezogen aus über zwei Seiten; die ( ) sind meine Klarstellungen:

\enquote{\textit{Passieren.} Hier sehen wir uns beim traditionellen Unterricht hinsichtlich der exakten Bewegungen, die mit den Fingern und den Daumen stattfinden sollten, mit einem Wust von Streß konfrontiert ...
Wenn ich diese Konzepte einfach hinwegblasen könnte, würde ich nicht zögern es zu tun.
Für so fehlerhaft und übel halte ich sie.
Sie können einen Pianisten buchstäblich verkrüppeln ...
Wenn es (perfekte Tonleitern spielen) hoffnungslos unmöglich erscheint und Sie keinen Schimmer einer Idee haben, wie es vollbracht werden kann, dann versuchen Sie es mit einer Koordination, die eine Tonleiter tatsächlich zu einer unmöglichen Meisterleistung macht.
Es bedeutet, daß der Daumen unter die Handfläche klappt und man nach der Position sucht; und die Finger versuchen, über den Daumen zu reichen und suchen nach der Verbindung der Tasten für das Legato.
Es ist egal, ob der Künstler, der bereits die  raschen und schönen Tonleitern und Arpeggios erreicht hat, Ihnen erzählt, daß er nur das (Daumenuntersatz) macht - es ist nicht wahr.
Ich will nicht unterstellen, daß er lügt, sondern daß er die Koordination, die er gelehrt bekam als die Gelegenheit, die es unangemessen machte, auftrat, erfolgreich verdrängt hat.
Sie (die Daumenuntersatz-Spieler) müssen wieder körperlich zu einem neuen Koordinationsmuster ausgebildet werden; und diese erneute Ausbildung kann für sie eine Periode des schrecklichen Elends bedeuten ...
Die Bewegung (für das Passieren mit Daumenübersatz) kann mittels des Schultergelenks in jede Richtung ausgeführt werden.
Der Oberarm kann sich so bewegen, daß das Ellbogenende des Humerus\footnote{d.h. des Oberarmknochens} einen Kreisabschnitt beschreiben kann und zwar sowohl auf- und abwärts als auch ein- und auswärts, vor und zurück oder um sich selbst herum ... (usw., eine ganze Seite mit Anweisungen dieser Art, wie man mit Daumenübersatz spielt) ...
Mit Kontrolle von der Mitte funktioniert die ganze Koordination so, daß es einfacher wird, einen Finger in dem Moment zur Verfügung zu haben, in dem er benötigt wird ...
Der beste Beweis für diese Aussage ist eine schöne Tonleiter oder Arpeggio, die mit völliger Nichtbeachtung jeglichen konventionellen Fingersatzes gespielt werden.
Das geschieht oftmals bei einem begabten, nicht ausgebildeten Klavierspieler ...
Beim Passieren (Daumenübersatz) agiert der Oberarm als Drehpunkt für all die}anderen Techniken\enquote{und bezieht den Unterarm und die Hand mit ein; Beugung und Streckung am Ellbogen, Drehbewegung und seitliche Bewegung beim Handgelenk und zu guter Letzt seitliche Bewegungen der Finger und Daumen ...
Durch die Drehbewegung und abwechselnde Bewegung wird das Passieren so leicht gemacht, wie es aussieht, wenn der Experte es macht.}

\textbf{Zweites Buch}: \enquote{Mastering the Chopin Etudes and Other Essays}, 206 S.<br>
Zusammenfassung von bearbeiten Manuskripten Whitesides; viel lesbarer, weil sie von ihren Schülern bearbeitet wurden, und enthält die meisten der Ideen des ersten Buchs, basierend auf dem Spielen der Chopin-Etüden, welche sowohl aufgrund ihres unerreichten musikalischen Gehalts als auch wegen ihrer technischen Herausforderung ausgewählt wurden.
Das ist wie ein Katechismus zur obigen Bibel; es mag eine gute Idee sein, dieses Buch zu lesen, bevor man das erste o.a. Buch liest.
Beschreibt das Konturieren einigermaßen detailliert: S. 54-61 grundlegende Beschreibung und S. 191-193 grundlegende Definition, mit mehr Beispielen auf den S. 105-107 und S. 193-196.
Obwohl das Konturieren dazu benutzt werden kann, technische Schwierigkeiten zu überwinden, ist es wertvoller dafür, das musikalische Konzept der Komposition kennenzulernen oder es spielen zu lernen.

Diese beiden Bücher \textit{sind} eine Diamantenmine an praktischen Ideen; aber wie bei einer Diamantenmine muß man tief schürfen, und man weiß nie, wo sie verborgen sind.
Die Verwendung der Chopin-Etüden stellt sich hier nicht als zufällige Wahl heraus; die meisten von Whitesides Grundsätzen wurden bereits von Chopin gelehrt (s. \hyperref[Eigeldinger]{Eigeldinger}); Eigeldingers Buch wurde jedoch lange Zeit nach Whitesides Buch geschrieben, und ihr waren wahrscheinlich viele von Chopins Methoden nicht bekannt.

Es gibt keinen Mittelweg - Sie werden Whitesides Buch entweder für die Fundgrube an Information lieben oder es hassen, weil es unlesbar, eintönig und ungeordnet ist.
 

\label{American}

<br>\textbf{Scientific American, Jan. 1979, S. 118-127, \textit{The Coupled Motions of Piano Strings}, von G. Weinreich}

Das ist ein guter Artikel über die Bewegungen von Klaviersaiten, wenn man die absoluten Grundlagen lernen muß.
Der Artikel ist jedoch nicht gut geschrieben, und die Experimente sind nicht sorgfältig ausgeführt; aber wir sollten die begrenzten Mittel berücksichtigen, die der Autor wahrscheinlich zur Verfügung hatte.
Noch weiter gehende Untersuchungen wurden sicherlich lange vor 1979 von Klavierherstellern und Akustikwissenschaftlern durchgeführt.
Ich werde im folgenden einige der Mängel, die ich in diesem Artikel gefunden habe, in der Hoffnung besprechen, daß die Kenntnis dieser Mängel den Leser in die Lage versetzt, hilfreichere Informationen aus dieser Publikation zu entnehmen und zu vermeiden, in die Irre geführt zu werden.

Es gibt keinerlei Information über die Frequenzen der Noten, die untersucht wurden.
Da das Verhalten von Klaviersaiten so frequenzabhängig ist, ist diese fehlende Information von entscheidender Bedeutung.
Behalten Sie dies im Gedächtnis, wenn Sie den Artikel lesen, da viele der Ergebnisse ohne die Kenntnis der Frequenzen, bei denen die Experimente ausgeführt wurden, schwer zu interpretieren und deshalb von fragwürdigem Wert sind.

Das mittlere Diagramm in der unteren Reihe der Abbildungen auf S. 121 (es gibt im ganzen Artikel keine Abbildungsnummern!) wird nicht ausreichend erklärt.
Der Artikel schlägt später vor, daß die vertikalen \hyperref[moden]{Moden} den Anschlagsklang erzeugen.
Die Abbildung könnte deshalb so interpretiert werden, daß sie das Ausschwingen einer einzelnen Saite zeigt.
Ich kenne keine Note auf einem Flügel, bei der die Ausschwingzeit einer einzelnen Saite weniger als 5 Sekunden beträgt, wie es durch die Abbildung suggeriert wird.
Die linke Abbildung der oberen Reihe der Diagramme für eine einzelne Saite zeigt, in Übereinstimmung mit meinen flüchtigen Messungen bei einem Flügel, ein Ausschwingen von mehr als 15 Sekunden.
Somit scheinen sich die beiden Diagramme für einzelne Saiten zu widersprechen.
Für das obere Diagramm wurde der Schalldruck gemessen, für das untere hingegen die Verschiebung der Saite, so daß sie nicht richtig miteinander vergleichbar sind, aber man hätte es gerne gesehen, wenn der Autor zumindest diese offensichtliche Diskrepanz ein wenig erklärt hätte.
Ich habe den Verdacht, daß für die beiden Diagramme Saiten mit sehr verschiedenen Frequenzen benutzt wurden.

In bezug auf diese Diagramme gibt es den Satz \enquote{Ich benutzte eine empfindliche elektronische Sonde, um die vertikalen und horizontalen Bewegungen einer einzelnen Saite getrennt zu messen.} ohne weitere Informationen.
Nun wäre jeder Forscher auf diesem Gebiet sehr daran interessiert, wie der Autor es gemacht hat.
In richtigen wissenschaftlichen Veröffentlichungen ist es gängige (allgemein \textit{verlangte}) Praxis, die Ausrüstung zu benennen (üblicherweise inkl. der Hersteller und Modellnummern) und sogar wie sie benutzt wurde.
Die resultierenden Daten sind einige der wenigen neuen Informationen, die in diesem Papier präsentiert werden, und sind deshalb in diesem Artikel von größter Wichtigkeit.
Zukünftige Forscher werden wahrscheinlich dieser Vorgehensweise folgen müssen, indem sie die Saitenverschiebungen detaillierter messen und werden diese Information über die Ausrüstung benötigen.

Auf die vier Abbildungen auf Seite 122 wird nirgends im Artikel verwiesen.
Somit bleibt es uns überlassen, zu vermuten, welche Teile des Artikels dazu gehören.
Auch ist meine Vermutung, daß die beiden unteren Diagramme, die Oszillationen zeigen, nur schematische Darstellungen sind und nichts repräsentieren, das nahe an tatsächliche Daten herankommt.
Ansonsten wäre der Anschlagsklang gemäß dieser Diagramme ungefähr nach 1/40 Sekunde zu Ende.
Die in diesen beiden Diagrammen gezeichneten Kurven sind nicht nur schematisch, sondern zusätzlich rein imaginär.
Es gibt keine Daten, die sie untermauern.
Tatsächlich präsentiert der Artikel keine weiteren neuen Daten und die Diskussionen auf den darauffolgenden 5 Seiten (aus einem achtseitigen Artikel) sind im Grunde eine Übersicht bekannter akustischer Prinzipien.
Deshalb sollten die Beschreibungen der federnden, massiven und standfesten Enden genau wie die der \hyperref[mitschwingung]{Mitschwingungen} qualitativ gültig sein.

Die Hauptthese dieses Artikels ist, daß das Klavier einmalig ist, weil es einen Nachklang hat, und daß das richtige Stimmen des Nachklangs das Wesentliche einer guten Stimmung ist und die einmalige Klaviermusik erzeugt.
Meine Schwierigkeit mit dieser These ist, daß der Anschlagsklang typischerweise mehr als 5 Sekunden dauert.
Sehr wenige Klaviernoten werden derart lange gespielt.
Deshalb wird im Grunde die ganze Klaviermusik nur mit dem Anschlagsklang gespielt.
Tatsächlich benutzen Klavierstimmer hauptsächlich den Anschlagsklang (so wie er hier definiert wird) zum Stimmen.
Außerdem ist der Nachklang mindestens um 30 db schwächer; er beträgt nur ein paar Prozent des anfänglichen Klangs.
Er wird in den anderen Noten eines jeden Musikstücks völlig untergehen.
In Wirklichkeit ist es so, daß was immer die Qualität des Klavierklangs kontrolliert, sowohl den Anschlags- als auch den Nachklang kontrolliert, und was wir brauchen ist eine Abhandlung, die Licht in diesen Mechanismus bringt.

Schließlich brauchen wir eine Publikation mit richtigen Quellenangaben, so daß wir wissen können, was bereits untersucht wurde und was nicht.
(Zur Verteidigung des Autors: Scientific American erlaubt keine Quellenangaben außer zu bereits im Scientific American veröffentlichten Artikeln.
Das macht es notwendig, Artikel zu schreiben, die \enquote{selbstbezüglich} sind, was dieser Artikel nicht ist.
Gemäß Reblitz [S. 14], gibt es einen Artikel im Scientific American von 1965 über \enquote{The Physics of the Piano}, aber auf diesen Artikel wird in diesem Bericht nicht verwiesen.)


\label{Lectures}

<br>\textbf{Five Lectures on the Acoustics of the Piano}
<br>(www.speech.kth.se/music/5_lectures/contents.html)

Eine sehr moderne Vorlesungsreihe darüber, wie das Klavier seinen Klang erzeugt.
Die Einführung bespricht die Geschichte des Klaviers und präsentiert die Terminologie und Hintergrundinformation, die notwendig sind, um die Vorlesungen zu verstehen.

Die erste Vorlesung bespricht Faktoren des Klavierdesigns, die den Klang und die akustische Leistung beeinflussen.
Hämmer, Resonanzboden, Rahmen, Platte, Saiten, Stimmwirbel und wie sie zusammenarbeiten.
Stimmer stimmen die transversalen Schwingungsmoden der Saite, aber die longitudinalen \hyperref[moden]{Moden} sind durch den Aufbau der Saite und der Tonleiter festgelegt und können vom Stimmer nicht gesteuert werden, haben aber hörbare Effekte.

Die zweite Vorlesung konzentriert sich auf den Klang des Klaviers.
Der Hammer hat zwei Biegemoden, eine Schaftbiegungsmode und eine schnellere Vibrationsmode.
Die erste wird durch die rasche Beschleunigung des Hammers verursacht, ähnlich der Biegung des Golfschlägers.
Die zweite ist am ausgeprägtesten, wenn der Hammer von den Saiten zurückspringt, kann aber auch auf seinem Weg zu den Saiten angeregt werden.
Klar ist der Fänger ein wichtiges Werkzeug, das der Klavierspieler benutzen kann, um diese zusätzlichen Hammerbewegungen zu reduzieren oder kontrollieren und dadurch den Klang zu kontrollieren.
Die tatsächliche zeitabhängige Saitenbewegung ist völlig anders als die Bewegung schwingender Saiten, wie sie in Lehrbüchern gezeigt wird, mit Grundschwingungen und harmonischen Obertönen, die ganzzahlige Bruchteile der Wellenlängen sind, die fein säuberlich zwischen die befestigten Enden der Saiten passen.
Sie ist in Wahrheit eine Gruppe von wandernden Wellen, die durch den Hammer in Richtung der Brücke und der Agraffe entsendet werden.
Diese wandern so schnell, daß der Hammer für einige Durchläufe - vorwärts und rückwärts - auf den Saiten \enquote{steckenbleibt}, und es ist die Energie einer dieser Wellen, die auf den Hammer treffen, die ihn schließlich in Richtung Fänger zurückwirft.
Wie werden nun die Grund- und Partialschwingungen erzeugt?
Einfach - sie sind nur die Fourier-Komponenten der wandernden Wellen!
Nichtmathematisch ausgedrückt: Die einzigen in diesem System möglichen wandernden Wellen sind Wellen, die hauptsächlich die Grund- und Partialschwingungen enthalten, weil das System durch die festen Enden eingegrenzt wird.
Das Ausklingen und die harmonische Verteilung reagieren auf die genauen Eigenschaften des Hammers, wie Größe, Gewicht, Form, Härte usw., sehr empfindlich.

Die Saiten übertragen ihre Schwingungen über die Brücke auf den Resonanzboden (RB) und die Effektivität dieses Vorgangs kann durch Messen der Übereinstimmung der akustischen Impedanz bestimmt werden.
Diese Energieübertragung wird durch die Resonanzen kompliziert, die im RB durch seine  Eigenschwingungen erzeugt werden, weil die Resonanzen Spitzen und Täler in der Impedanz/Frequenz-Kurve erzeugen.
Die Effizienz der Klangerzeugung ist bei niedrigen Frequenzen gering, weil die Luft einen \enquote{Schlußspurt} um das Klavier herum machen kann, so daß eine Druckwelle über dem RB das Vakuum unter ihm aufheben kann, wenn der RB aufwärts vibriert (und umgekehrt, wenn er sich abwärts bewegt).
Bei einer hohen Frequenz erzeugen die Vibrationen des RBs zahlreiche kleine Gebiete, die sich in verschiedene Richtungen bewegen.
Wegen ihrer Nähe kann komprimierte Luft in einem Bereich das Vakuum in einem angrenzenden Bereich aufheben, was zu einem geringeren Schall führt.
Das erklärt, warum ein kleiner Anstieg in der Klaviergröße, besonders bei niedrigen Frequenzen, die Schallerzeugung stark erhöhen kann.
Diese Komplikationen machen klar, daß es eine monumentale Aufgabe ist, die akustischen Effizienzen über alle Noten des Klaviers hinweg aufeinander abzustimmen und erklärt, warum gute Klaviere so teuer sind.

Das obige ist mein Versuch einer kurzen Übersetzung von hochtechnischem Material und ist wahrscheinlich nicht 100\% richtig.
Mein Hauptzweck ist, dem Leser eine gewisse Vorstellung vom Inhalt der Vorlesungen zu vermitteln.
Diese Website enthält eindeutig sehr wissenschaftliches Material.


<h3><br>\underline{Weitere Quellen}</h3>

\begin{itemize} 
 \item Bach Bibliography (www.music.qub.ac.uk/tomita/bachbib/).
 \item Bertrand, OTT., \textit{Liszt et la Pédagogie du Piano, Collection Psychology et Pédagogie de la Musique}, (1978) E. A. P. France.
 \item Boissier, August., \textit{A Diary of Franz Liszt as Teacher 1831-32}, übersetzt von Elyse Mach.
 \item Chan, Angela, \textit{Comparative Study of the Methodologies of Three Distinguished Piano Teachers of the Nineteenth Century: Beethoven, Czerny and Liszt} (www.geocities.com/Paris/Metro/5453/maped.htm).
 \item Fay, Amy, \textit{Music Study in Germany.}
 \item Fine, Larry, \textit{The Piano Book}, Brookside Press, 4. Ausgabe, Nov. 2000.
 \item Fischer, J. C., \textit{Piano Tuning}, Dover, N.Y., 1975.
 \item Howell, W. D., \textit{Professional Piano Tuning}, New Era Printing Co., Conn. 1966.
 \item Jaynes, E. T., \textit{The Physical Basis of Music} (bayes.wustl.edu/etj/music.html).
Erklärt, warum Liszt nicht unterrichten konnte.
Beste Erklärung des \hyperref[c1iii5a]{Daumenübersatzes} in der 
Literatur.
 \item Jorgensen, Owen H, \textit{Tuning}, Michigan St. Univ. Press, 1991.
 \item Reblitz, Arthur, \textit{Piano Servicing, Tuning, and Rebuilding,}2. Ausgabe, 1993.
 \item Moscheles, \textit{Life of Beethoven.}
 \item Sethares, William A., \textit{Adaptive tunings for musical scales}, J. Acoust. Soc. Am. 96 (1), Juli 1994, P. 10.
 \item Tomita, Yo, \textit{J. S. Bach: Inventions and Sinfonia} (www.music.qub.ac.uk/~tomita/essay/inventions.html), 1999.
 \item White, W. B., \textit{Piano Tuning and Allied Arts}, Tuners' Supply Co., Boston, Mass, 1948.
 \item Young, Robert W., \textit{Inharmonicity of Plain Wire Piano Strings}, J. Acoust. Soc. Am., 24 (3), 1952.
 \item \footnote{Zahlreiche Beiträge des Usenet-Forums rec.music.makers.piano.}
 \end{itemize}

\label{Websites}

<h3><br>\underline{Websites, Bücher, Videos}</h3>

\footnote{Wegen der unklaren deutschen Rechtslage hinsichtlich der Mitverantwortung für Inhalte von Seiten, zu denen man Links in seinen Seiten anbietet, führe ich die Links aus der Originalseite hier nicht mit auf.
Wer also wissen möchte, welche Websites Chuan C. Chang für weitergehende Informationen empfiehlt, den verweise ich auf das \hyperref[http://www.pianopractice.org]{Original dieser Seite} (extern).
Ich erspare mir (und den LeserInnen) auch die Wiederholung der umfangreichen Liste der Bücher und Videos und verweise wiederum auf das Original.<br>
Zu den einzelnen dort angeführten Büchern und Videos kann ich nicht viel sagen, da ich die meisten nicht kenne.
In meinem Bücherregal stehen (das soll jetzt keine Empfehlung sein, sondern nur als Beispiel dienen, und ich verdiene auch nichts damit, daß ich das hier schreibe!):<br>
- aus den Anfängen \enquote{Der junge Pianist}<br>
- die deutsche Ausgabe von \hyperref[Gieseking]{Leimer/Gieseking}<br>
- die amerikanische Ausgabe von Arthur Reblitz: \enquote{Piano Servicing, Tuning, and Rebuilding} (habe ich bisher noch nicht in Deutsch gesehen)<br>
- für den Einstieg in die Musiktheorie \enquote{der Ziegenrücker}<br>
- ein paar Bücher mit Analysen<br>
- und natürlich Noten, Noten, Noten}
 





<!-- anmerkungen.html -->

\label{anmerkungen}

<h2>Anmerkungen</h2>

Diese Website bietet Ihnen einen kostenlosen Klavierunterricht, Lehrmaterial für das Klavierspielen und Anweisungen zum \hyperref[c2_1]{Stimmen des Klaviers}.
Sie können das Klavierspielen im Vergleich zu anderen Methoden\footnote{bis zu} \textbf{1000mal schneller (!)} lernen (s. \hyperref[c1iv5]{Kapitel 1, IV.5}).
Dieses ist das erste Buch, das jemals darüber geschrieben wurde, wie man das Klavierspielen übt.
Hunderte von Jahren lehrten viele Lehrer und andere Bücher, welche Techniken man erwerben muß, aber das nutzt wenig, wenn man im Gegensatz zu Mozart, Liszt usw. nicht weiß, wie man sich diese schnell aneignet.
Sie können sich entweder \hyperref[copy]{hier} das ganze Buch oder mit Hilfe der Links im \hyperref[Inhalt]{Inhaltsverzeichnis} oder in der \hyperref[./dateien.html\#copy]{Übersicht der Dateien} Teile davon kostenlos herunterladen.

Benutzen Sie dieses Buch zum Klavierspielenlernen als ergänzendes Lehrbuch, wenn Sie einen Lehrer haben.
Wenn Sie keinen Lehrer haben, suchen Sie sich ein beliebiges Musikstück aus, das Sie lernen möchten (das innerhalb Ihrer Fertigkeitsstufe liegt), und beginnen Sie, es mit den hier beschriebenen Methoden zu lernen; die Methoden sind grob in der Reihenfolge angeordnet, in der Sie sie benötigen, wenn Sie ein neues Stück lernen.
In beiden Fällen (mit oder ohne Lehrer), sollten Sie das ganze Buch zunächst einmal durchlesen.
Beginnen Sie mit dem \hyperref[preface]{Vorwort}, das Ihnen einen kurzen Überblick vermittelt.
Überspringen Sie alle Abschnitte, von denen Sie glauben, sie seien nicht relevant oder zu ausführlich.
Versuchen Sie nicht, jedes Konzept zu verstehen oder sich alles zu merken - lesen Sie es so, wie Sie einen Roman lesen würden, nur zum Spaß.
Machen Sie sich nur mit dem Buch vertraut, damit Sie eine Vorstellung davon bekommen, wo bestimmte Themen besprochen werden.
Lesen Sie zuletzt die \hyperref[testimonials]{Leserkommentare}, soweit Sie diese interessant finden.
Beginnen Sie dann erneut ab einer Stelle, die Ihnen das benötigte Material bietet; die meisten werden die kompletten Abschnitte \hyperref[c1i1]{I} und \hyperref[c1ii1]{II} des ersten Kapitels benötigen.
Sie können danach zu den einzelnen Themen springen, die die Komposition betreffen, die Sie gerade lernen.
Für den Fall, daß Sie nicht genau wissen, welche Kompositionen Sie lernen sollen, werden in diesem Buch zahlreiche Beispiele vorgestellt: von Material für Anfänger (s. \hyperref[c1iii18]{Kapitel 1, III.18}) bis zur Mittelstufe; merken Sie sich deshalb beim ersten Lesen diese Stellen mit Beispielen und Vorschlägen.

\textbf{Ein Wunsch} an diejenigen, die dieses Material nützlich fanden: Bitte lassen Sie mindestens zwei Menschen von meiner Website wissen, so daß wir eine Kettenreaktion von noch mehr Leuten, die über diese Website informiert werden, starten können.

Ich suche Freiwillige, die das Buch in jede andere Sprache übersetzen.
Senden Sie bitte eine E-Mail an \hyperref[mailto:cc88m@aol.com?subject=foppde:\%20Translation\%20request]{cc88m@aol.com} um die Einzelheiten zu besprechen.


\label{HinUeber}

Übersetzer sollten sich etwas mit HTML auskennen und in der Lage sein, eine eigene Site für die Webseiten zu unterhalten (XXX und Xxxxx sollten gut sein).
Meine Vision ist, daß dieses Buch irgendwann zu einer dauerhaften Site umziehen wird.
Alle Übersetzungen sollten zur gleichen Site umziehen können.
Der Speicherbedarf aller potentiellen Übersetzungen ist bescheiden, da jede Sprache nur etwas mehr als 1 MB erfordert.

Übersetzer sind für ihre eigene Website verantwortlich und sollten nach Möglichkeit mit den Updates des Originals Schritt halten.
Es gibt jede Menge Software, um geänderte Version mit älteren Versionen zu vergleichen, so daß dies kein Problem darstellen sollte.
Sie werden aber eine Kopie der älteren Version auf Ihrem Computer behalten müssen, weil die älteren Versionen von meiner Website verschwinden werden, wenn sie geändert werden\footnote{und falls mal die Platte samt Backup \enquote{abraucht}, gibt es ja noch den netten Kollegen von der anderen Sprache.}

Übersetzer sollten vorzugsweise Klavierspieler oder Klavierlehrer sein und etwas über das Klavier selbst wissen (\footnote{Mechanik,} Stimmen, Einstellen, Aufarbeiten).
Wenn der Übersetzer bei einem bestimmten Thema Lücken hat, können wir immer Helfer für dieses Thema finden, so daß ein Mangel an Fachkenntnissen eines Übersetzers kein Problem ist.

Ich schreibe dieses Buch auf einer Freiwilligenbasis und kann deshalb Übersetzern nichts zahlen, bis irgendein Stifter auftaucht.
Wir haben ein Programm zur Aufteilung der Aufwandsentschädigungen; ich werde es mit Ihnen besprechen, sobald Sie sich anbieten.
Ich würde mich selbstverständlich freuen, soweit wie möglich dabei zu helfen, die Übersetzung zu beschleunigen und werde einen Link zur Übersetzung auf meiner Inhaltsverzeichnisseite zur Verfügung stellen.

Wir leisten hier Pionierarbeit für eine neue Art von Internet-Buch. Das ist ein Wink der Zukunft und wirklich aufregend.
Dieses Buch sollte sich zu dem vollständigsten Lehrbuch für das Klavierspielen entwickeln, das kostenlos ist, immer auf dem neuesten Stand, in dem Fehler eliminiert werden, sobald sie entdeckt werden, und das in allen verbreiteten Sprachen verfügbar sein wird.
Es gibt keinen Grund, warum Schulen und Schüler für elementare Lehrbücher von Arithmetik bis Zoologie zahlen müssen.
In der Zukunft werden sie alle kostenlos zum Download zur Verfügung stehen.
Der Weltwirtschaft wird enorm dadurch geholfen, daß das Ausbildungsmaterial jedem frei zugänglich ist.
Es ist einfach unglaublich, sich die Zukunft der Ausbildung im Internet vorzustellen.
Da alles, was man braucht, ein paar der besten Experten der Welt sind, die Lehrbücher schreiben und weitere Freiwillige, um sie zu übersetzen, sind die dafür notwendigen Mittel vernachlässigbar im Vergleich zum ökonomischen Nutzen.
Deshalb bringen Übersetzer nicht nur ihren Landsleuten einen Nutzen, sondern nehmen auch an einem prächtigen neuen Experiment teil, das alle Klavierspieler, Klavierlehrer, Klavierstimmer und die Klavierindustrie fördert.





<!-- testimonials.html -->

\label{testimonials}

<h2><br>\underline{Leserkommentare}</h2>

Diese Leserkommentare veranschaulichen sowohl die Hoffnungen, Versuche und den Kummer als auch die Erfolge von Klavierspielern und Klavierlehrern.
Die Leserkommentare sind nicht nur eine Sammlung schmeichelhafter Lobeshymnen, sondern eine offene Diskussion darüber, was es bedeutet, das Klavierspielen zu lernen.
Die Zahl der Lehrer, die mir Kommentare gegeben haben, und ihre Berichte, daß sie durch die Benutzung dieser Art von Methoden mehr Erfolg mit ihren Schülern haben, sind eine Ermutigung für mich.
Es scheint unausweichlich, daß Lehrer, die Nachforschungen anstellen und ihre Lehrmethoden verbessern, erfolgreicher sind.
Zahlreiche Klavierspieler haben erwähnt, daß sie durch ihre vorherigen Lehrer völlig falsch unterrichtet wurden.
Viele, die ihre Lehrer mochten, merkten an, daß diese Lehrer Methoden benutzten, die denen in diesem Buch ähnlich sind.
Es gibt eine fast einmütige Übereinstimmung darüber, was richtig und was falsch ist; wenn man dem wissenschaftlichen Ansatz folgt, kommt man deshalb nicht in die Situation, bei der man sich nicht darüber einigen kann, was richtig ist.
Ich war beeindruckt, wie schnell einige Leser diese Methoden aufgenommen haben.

Die Auszüge wurden kaum bearbeitet, irrelevante Details jedoch weggelassen.
Eintragungen in [...] sind meine Kommentare.
Ich nehme die Gelegenheit wahr, jedem zu danken, der mir geschrieben hat; Sie haben mir geholfen, das Buch zu verbessern.
Ich kann nicht über die Tatsache hinweggehen, daß die Leser das Buch für mich weiterschreiben (d.h., ich könnte ihre Anmerkungen in mein Buch einfügen und sie würden perfekt passen!).
Im folgenden habe ich nicht nur die schmeichelhaften Bemerkungen ausgewählt; ich wählte Material, das bedeutend (lehrreich) erschien, egal ob es positiv oder kritisch ist.


<ol type=1>

\item \label{testimonials01}
[Von einem Pfarrer:]<br>
Dieses Buch ist die Klavierbibel.
Ich habe solch enorme Fortschritte gemacht, seit ich es gekauft habe [die erste Ausgabe].
Ich werde es weiterhin anderen empfehlen.


\item \label{testimonials02}
[Jan. 2003 erhielt ich diese E-Mail (mit freundlicher Genehmigung):]<br>
Mein Name ist Marc, und ich bin 17 Jahre alt.
Ich habe gerade vor einem Monat mit dem Klavierspielen angefangen und habe Ihr Buch \enquote{The Fundamentals of Piano Practice} gelesen. . .
Ich habe noch keinen Lehrer, aber ich bin dabei, einen zu suchen. . .
[Gefolgt von einer Reihe Fragen, die bei jemand so jungem mit so wenig Klaviererfahrung von früher Reife zeugen.
Ich beantwortete die Fragen so gut ich - damals - konnte.]

[Mai 2004 erhielt ich diese erstaunliche E-Mail:]<br>
Ich erwarte nicht, daß Sie sich an mich erinnern, aber ich sandte Ihnen vor ungefähr einem Jahr eine E-Mail. . .
Ich möchte Sie wissen lassen, wie ich mit Hilfe Ihrer Methode mit dem Klavierspielen zurechtgekommen bin.
Ich begann mit dem Klavierspielen ungefähr Weihnachten 2002 und benutzte Ihre Methode von Anfang an.
Mitte März 2003 nahm ich am Konzertwettbewerb meiner High-School teil - aus Spaß und der Erfahrung wegen, nicht in der Hoffnung, ihr 500\$-Stipendium zu gewinnen.
Ich kam - in einer Konkurrenz mit reiferen Klavierspielern, die seit bis zu 10 Jahren spielten - unerwartet auf den ersten Platz.
Es war ein Schock für die Richter, als ich ihnen sagte, daß ich seit 3 Monaten spiele.
Vor ein paar Tagen gewann ich auch den diesjährigen Wettbewerb.
Mit anderen Worten: Der Fortschritt ist sehr schnell gekommen.
Ein solcher Fortschritt ist einer der größten Motivatoren (neben der generellen Liebe zur Musik), 
so daß ich nun für den Rest meines Lebens Klavier spielen und mich darin verbessern kann.
Und obwohl ich meinen Lehrern ebenfalls Anerkennung zollen muß, ist Ihre Methode meine Grundlage, auf der sie aufbauen, und ich glaube, daß die Methode der Hauptgrund für meinen Fortschritt ist.
Trotzdem halte ich mich immer noch für einen Anfänger. . .
Meine Website (www.mtm-piano.tk) enthält alle Aufnahmen, die ich bis heute (18) gemacht habe. . .
Vor kurzem habe ich Chopins Regentropfen-Präludium, Scarlattis K.466 und Bachs F-Dur-Invention aufgenommen. . .
Meine nächste Aufnahme wird Bachs e-Moll-Sinfonie\footnote{dreistimmige Invention} sein, und ich möchte das bis Ende nächster Woche fertig haben.
Ihr Buch ist weit mehr als ein Liebhaber der Musik und des Klaviers erwarten könnte, und ich kann Ihnen nicht genug für die Hilfe danken, die Sie mir und so vielen anderen aufstrebenden Klavierspielern gegeben haben. . .
[Gehen Sie auf seine Website und hören Sie sich diese erstaunlichen Aufnahmen an!
Sie können ihn sogar auf der Music-Download-Website (music.download.com) finden; suchen Sie nach \enquote{Marc McCarthy}.]


\item \label{testimonials03}
[Von einer angesehenen, erfahrenen Klavierlehrerin:]<br>
Ich habe gerade Ihren neuen Abschnitt überflogen [über Übungen für parallele Sets] und dachte, ich sollte Ihnen meine erste Reaktion mitteilen.
Als Prinzregentin der Übungshasser habe ich mich lautstark für die Kriminalisierung von Hanon und anderen eingesetzt und hatte zunächst bestürzt angenommen, daß Sie sich den unterdrückten Massen der pseudo-voodoo-haft Übenden angeschlossen hätten - hoffnungslos, hilflos, wiederholend, wiederholend, . . .
Jedenfalls, um auf den Punkt zu kommen, sehe ich einen Punkt des Verdienstes in Ihrem Ansatz, WENN WENN WENN der Schüler Ihre GANZEN Anweisungen befolgt und die beschriebenen Schlüsselkombinationen als Diagnosewerkzeug benutzt - NICHT um jede einzelne Kombination als tägliche Routine zu wiederholen.
Als Diagnosewerkzeug und daraus folgendes Heilmittel; das ist Ihnen wunderbar gelungen!
Es war etwas vertrautes in Ihren Übungen, deshalb habe ich heute im Studio herumgesucht und fand die \enquote{Technische Studien} von Louis Plaidy, Edition Peters, die ca. 1850 das erste Mal gedruckt wurden.
Obwohl sich Plaidys Philosophie bezüglich des Gebrauchs seiner Übungen sehr von Ihrer unterscheidet, folgen die tatsächlich abgedruckten Noten buchstabengetreu (ich sollte notengetreu sagen) dem, was Sie in Ihrem Übungskapitel beschreiben.
Plaidys Übungen waren in den späten 1800er Jahren in Europa hoch angesehen und wurden während dieser Zeit am Konservatorium in Leipzig benutzt.
Plaidy selbst war ein ziemlich vielgefragter Lehrer, von dessen Protegés einige in Liszts innerem Kreis akzeptiert wurden und/oder einigen Erfolg auf den Konzertbühnen hatten.
Sie sind in großartiger Gesellschaft!


\item \label{testimonials04}
Ich möchte gerne wissen, ob Sie die Arbeit von Guy Maier kennen.
Geht sein Ansatz des \enquote{Impuls}-Übens von 5-fingrigen Mustern in die gleiche Richtung wie die von Ihnen besprochenen \enquote{parallelen Sets}?
Maier benutzt das Prinzip, eine Note mit jedem Finger zu wiederholen, während die anderen ruhig auf der Tastenoberfläche gehalten werden, als eine der 5-fingrigen Übungen.
\textit{Thinking Fingers} war eines der Übungsbücher, das Maier mit Herbert Bradshaw Anfang der 1940er Jahre schrieb.
Eine seiner ersten 5-fingrigen Übungen, die widerzuspiegeln scheint, was Sie über \enquote{Quadrupel}-Wiederholungen auf einer Note gesagt haben, geht folgendermaßen:<br>
a. Einzelne Finger mit wiederholten 1, 2, 3, 4, 8, und 16 Noten-Anschlägen.<br>
b. Üben Sie jeden Finger einzeln, drücken Sie die anderen Tasten leicht herunter oder halten Sie die Finger still in der Position auf den Tasten.<br>
c. Benutzen Sie CDEFG mit der rechten Hand, setzen Sie 5 Finger auf diese Noten eine Oktave oberhalb des mittleren C, mit dem Daumen der rechten Hand auf dem C.<br>
d. Genauso mit der linken Hand, eine Oktave unter dem mittleren C, mit dem fünften Finger auf dem C.<br>
e. Üben Sie die Hände getrennt; beginnen Sie mit dem Daumen der rechten Hand, spielen Sie einen Anschlag C, lassen Sie los, dann zwei Anschläge usw. bis zu 16.
Wiederholen Sie das mit jedem Finger, benutzen Sie dann die linke Hand.<br>
[Sehen Sie dazu meine Übungen in \hyperref[c1iii7b]{Abschnitt III.7b}; es ist erstaunlich, daß wir unabhängig voneinander für diese Übung zu Gruppen von \enquote{Quadrupeln} (vier Wiederholungen) gekommen sind und bis zu 4 Quadrupeln (16 Wiederholungen), was fast mit meiner Übung \#1 identisch ist.]<br>
f. Anfänger werden die Anschläge langsam ausführen und sich an die volle Geschwindigkeit heranarbeiten müssen (und hier kommen glaube ich Ihre \enquote{Quadrupel} ins Spiel - so viele Wiederholungen je Sekunde sind das Ziel).<br> Maier bezeichnet 16 als seine Grenze.
Er gibt eine Vielzahl von Mustern dafür, diese Vorgehensweise bei 5-fingrigen Anschlagsübungen zu verwenden, in Buch 1 und 2 von \textit{Thinking Fingers}, 1948 herausgegeben von Belwin Mills Inc., NY.
Ich glaube, Maier war bemüht, den Schülern dabei zu helfen, die nötige Gewandtheit ohne die endlosen Wiederholungen von Hanon, Pischna und anderen zu bekommen.


\item \label{testimonials05}
Bitte senden Sie mir Ihr Buch - ich bin seit mehr als 50 Jahren Klavierlehrer und immer noch bereit dazuzulernen.


\item \label{testimonials06}
[Diese Zuschrift öffnet einem die Augen: Sie zeigt uns eines der am meisten falsch diagnostizierten Probleme, das uns daran hindert, schnell zu spielen.]<br>
In jungen Jahren begann ich mit dem Klavier und hörte dann wieder auf.
Als Teenager ging ich dann auf ein [berühmtes] Konservatorium und versuchte jahrelang, Technik zu erwerben, aber versagte kläglich und machte am Ende Karriere als Ingenieur.
Jahre später kehrte ich zum Klavier (Clavinova) zurück und versuche nun, das zu tun, was mir Jahre zuvor nicht gelang.
Einer der Gründe, warum ich mit dem Üben aufhörte, ist, daß meine Frau und mein Sohn ärgerlich wurden, wenn sie mich Passagen ständig wiederholen hörten; das Clavinova erlaubt mir, ohne Schuldgefühle zu jeder Stunde zu üben\footnote{mit Kopfhörern!}.
Ich las Ihre Website und war fasziniert.
Ich wünschte, ich hätte einige Jahre zuvor an einige Ihrer Ideen gedacht.
Ich habe eine Frage und kann scheinbar keine sinnvolle Antwort darauf finden, und doch ist es eine solch grundlegende Frage.
Man hat mir beigebracht, daß man, wenn man Klavier spielt, das Gewicht des Arms auf jeden Finger der spielt überträgt.
Schwerkraft.
Man drückt niemals nach unten, man muß entspannt sein.
Deshalb fragte ich meine Lehrer, wie man pianissimo spielt.
Die Antwort war, daß man näher an den Tasten spielt.
Das funktioniert bei mir nicht.
[Lange Diskussion verschiedener Methoden zu versuchen, pianissimo mit dem Armgewicht zu spielen und warum sie nicht funktionieren.
Anscheinend kann er nur pianissimo spielen, wenn er bewußt die Hände von den Tasten hebt.
Auch ist, da alles oftmals als forte herauskommt, die Geschwindigkeit ein Problem.]
Würden Sie mir diese Frage bitte beantworten?
Was macht man mit dem Armgewicht, wenn man pianissimo spielt?
Ich habe viele Bücher über das Klavierspielen gelesen und mit vielen vollendeten Pianisten gesprochen.
Es ist eine Sache, zu wissen wie man alles spielt, und es ist eine ziemlich andere Sache, in der Lage zu sein, jemand anderem beizubringen, wie man spielt.
[Ich hätte das nicht besser sagen können!]
Ihre Schriften sind brillant und auf vielfältige Weise revolutionär, ich wußte instinktiv, daß wenn irgend jemand mir helfen könnte, dann Sie.<br>
[Nach solch einem Kompliment mußte ich etwas tun, weshalb ich den Bericht seiner Schwierigkeiten sorgfältig las und zu dem Schluß kam, daß er nach so vielen Jahren der Versuche unwissentlich auf das Klavier herunterdrücken mußte, fast so als wenn er hypnotisiert wäre.
Ich sagte ihm, er solle einen Weg finden, um festzustellen, ob er tatsächlich herunterdrückte - keine leichte Aufgabe.
Dann kam diese Antwort:]<br>
Danke für Ihre Antwort.
Die Wahrheit wird am besten anhand von Extremen untersucht.
Ihr Vorschlag brachte mich auf die Idee, daß ich vielleicht IMMER so spielen sollte, wie ich MEIN pianissimo spiele - indem ich meine Hände von den Tasten hebe.
Ich eilte zu meinem Hanon, und JA!
Ich kann viel schneller spielen!
Ich eilte schnell zum Bach-Präludium II, das ich niemals mit der richtigen Geschwindigkeit (144) spielen konnte, und ich hatte immer Schwierigkeiten, die Finger gleichzeitig landen zu lassen, wenn ich schnell spielte, und bei Geschwindigkeiten über 120 landeten die Finger zusammen, wie eine einzige Note.
Kein Herumfingern, keine Anstrengung.
Nicht nur das, ich kann piano oder forte so schnell spielen wie ich möchte.
Es fühlt sich so unglaublich LEICHT an!
Ich habe es gerade entdeckt!
Ich kann es nicht glauben.
[Lange Diskussion darüber, wie er über die Jahre dazu kam, Armgewicht mit herunterdrücken gleichzusetzen, was hauptsächlich durch die Angst, die Lehrerin nicht zu verstehen, verursacht wurde, die eine strikte Verfechterin des Armgewichts war.
Das ist tatsächlich etwas, was mir sehr verdächtig bei der Armgewichtsmethode vorkam: daß ein so starkes Betonen des Armgewichts und übermäßig strikte Disziplin eine Art Neurose oder Mißverstehen verursachen könnten - vielleicht sogar eine Art Hypnose.]
Eine große Mauer ist gerade eingestürzt und nun, nach so vielen Jahren des Nachdenkens und Stunden des Übens (ich übte am Konservatorium bis zu 10 Stunden täglich, und immer noch lernte ich nur die Musik auswendig, ohne jemals meine Technik zu verbessern), kann ich dahinter sehen.
Ich entdeckte, daß ich die Fähigkeit habe, schneller zu spielen als ich je geträumt hätte (habe gerade die C-Dur-Tonleiter versucht und war schockiert, daß ich das war, der spielte) und mit dem vollen Dynamikbereich, den ich möchte, OHNE ANSPANNUNG.
[Eine lange Beschreibung all der neuen Dinge, die er nun kann, und ein Vergleich mit seinen vorherigen Jahren des Kämpfens und der Kritik von anderen.]
Ich muß Ihnen dafür danken.
Ihr Buch war das einzige, das ich jemals gelesen habe, das genug Variationen von der Hauptlinie bot, um meinen Geist schließlich von einem großen Irrtum zu befreien.
Ich drückte herunter, ließ nicht los.
Meine Arme wiegen einfach keine Tonne, sondern sie sind frei.
Weil ich Angst vor meiner Lehrerin hatte und einzig auf das Gewicht meiner Arme achtete, drückte ich unbewußt nach unten.
Ich wagte nie, PPP für sie zu spielen.
Ich wußte wie, aber ich war sicher, daß es die falsche Technik war.
[Ich fürchte, daß das jungen Menschen häufig passiert; sie verstehen den Lehrer nicht aber fürchten sich davor zu fragen und nehmen am Ende das Falsche an.]
Was sie mir hätte sagen sollen, war: \enquote{DRÜCKE NIEMALS HERUNTER!}; statt dessen fixierte ich mich auf das Gewicht meines Arms als den Schlüssel für alles.
[Ein junger Mensch muß herunterdrücken, um \enquote{Gewicht} auf seine Arme zu geben!
Wie erklärt man einem Kind, das keinen Physikunterricht hatte, daß das falsch ist?]
Sie erlaubte mir nie, schnell zu spielen.
[Das ist ein weiterer Kommentar, den ich von Schülern strikter Armgewichtslehrer gehört habe - Geschwindigkeit kommt nicht in Frage, bis bestimmte Meilensteine erreicht sind; obwohl wir vorsichtig sein müssen, wenn wir für die Geschwindigkeit üben, ist langsamer zu spielen nicht der schnellste Weg zur Geschwindigkeit.]
Weil ich verspannt war, und sie sagte, ich würde niemals schnell spielen, wenn ich verspannt wäre.
In Ihrem Buch sagen Sie, daß wir schnell spielen müssen, um die Technik zu entdecken.
Es wurde mir nie erlaubt!
Ihr Buch und Ihre E-Mail befreiten mich von den Ketten in meinem Geist, die mich all die Jahre gefangen gehalten hatten.
Ich danke Ihnen vielmals.
Ich kann nicht beschreiben, wie dankbar ich Ihnen und Ihren Einblicken bin.

[Obwohl meine obigen Kommentare gegen die Armgewichtsschule gerichtet scheinen, ist das nicht der Fall - ähnliche Schwierigkeiten gelten für jede Lehre, die auf ungenügendem Wissen basiert und sich in den Händen eines strengen disziplinarischen Lehrers befindet.
Unglücklicherweise hat in der Vergangenheit eine große Zahl Klavierlehrer aufgrund eines Mangels an theoretischem Verständnis und rationaler Erklärungen inflexible Lehrmethoden übernommen.
Eine systematische Behandlung der Geschwindigkeit finden Sie in \hyperref[c1ii13]{Abschnitt II.13} und besonders in Abschnitt III.7i.]


\item \label{testimonials07}
Ich fand Ihr Buch im Internet und schätze mich sehr glücklich.
Danke vielmals, daß Sie eine solch große Anstrengung unternommen haben, die Klaviertechnik und sinnvolle Übungsgewohnheiten zu beschreiben.
Ich bin Klavierlehrer.
Ich habe erst angefangen, Ihr Buch zu lesen, und habe bereits einige Übungstechniken bei meinen Schülern angewandt.
Sie mochten es und ich mag es ebenfalls.
Das Üben wird dadurch viel interessanter.
Kennen Sie das Buch \enquote{The Amateur Pianist's Companion} von James Ching, herausgegeben 1956 von Keith Prowse Music Publishing Co. in London?
Dieses Buch wird vielleicht nicht mehr gedruckt, aber ich habe es gebraucht gefunden.
Es könnte Sie interessieren, denn \enquote{die Details der korrekten Haltungen, Bewegungen und Bedingungen, wie sie in diesem Buch skizziert werden, sind das Ergebnis von ausgedehnten Forschungen auf dem Gebiet der physiologischen Mechanik der Klaviertechnik, die vom Autor in Zusammenarbeit mit Professor H. Hartridge, Professor der Physiologie, und H. T. Jessop, Dozent in Mechanik und Angewandter Mathematik an der Universität von London, durchgeführt wurden}.


\item \label{testimonials08}
Ich bin so dankbar, daß ich Ihre Website gefunden habe.
Ich bin ein erwachsener Klavierspieler, der als junger Mensch auf die falsche Art unterrichtet wurde.
Ich versuche immer noch, meine schlechten Techniken und Angewohnheiten abzutrainieren.
Ich nehme nun Stunden bei einem sehr guten Lehrer.


\item \label{testimonials09}
Vor ein paar Wochen habe ich mir Ihr Buch aus dem Internet heruntergeladen und es ausprobiert.
Ich bin ungefähr zur Hälfte damit durch und weit davon entfernt, alles völlig anzuwenden, aber ich bin bis jetzt so erfreut über die Resultate, daß ich dachte, ich gebe Ihnen ein paar spontane Rückmeldungen.<br>
Zunächst ein paar Hintergrundinformationen.
Ich lernte Klavier bis zu einer fortgeschrittenen Stufe und begann ein Musikstudium, das ich nach einem Jahr abbrach, um Mathematik zu studieren.
Nach dem Diplom war ich ein enthusiastischer Amateur, aber während der letzten 20 Jahre spielte ich immer weniger, hauptsächlich wegen meiner Frustration über den Mangel an Fortschritt und weil ich überzeugt war, daß ich nie soviel Zeit für das Üben haben würde, wie nötig wäre, um besser zu spielen.<br>
Ich suchte nach ein paar Tips für den Kauf eines Klaviers und kam zu Ihrer Website.
Nachdem ich ein paar Kapitel gelesen hatte, lud ich mir alles herunter und fing an es auszuprobieren.
Das ist nicht das erste Mal, daß ich versucht habe, mich mit einem Buch oder dem Rat eines Lehrers zu verbessern.
Hier sind meine Erfahrungen nach drei Wochen.
[Beachten Sie, wie schnell man diese Methoden lernen und sie nutzen kann.]<br>
Ich habe mich darauf konzentriert, 4 Stücke zu lernen, die ich sehr gern mag:<br>
- Ravels Prélude<br>
- Chopins Prélude Nr. 26 in As-Dur<br>
- Poulencs Novelette Nr. 1<br>
- Ravels Alborada del Graziosa aus Miroirs<br>
Die Ravel-Prélude ist ein kleines Stück mit keiner offensichtlichen technischen Schwierigkeit.
Das ist ein Stück, das ich immer vom Blatt gespielt habe, aber nie richtig gut.
Es gibt in der Mitte einen Abschnitt mit gekreuzten Händen mit einer exquisiten Dissonanz, die einige Schwierigkeiten bereitet, aber das war es dann.
Ich wandte die Übungsmethoden in diesem Buch auf das Stück an, und es wurde plötzlich weitaus nuancierter lebendig als ich ihm jemals zugetraut hätte.
Es ist alles andere als das beiläufig spielbare Stück, für das ich es immer gehalten hatte, aber ohne die richtigen Übungsmethoden wird es immer als das erscheinen.<br>
Die Poulenc-Novelette ist eines der Stücke, die ich 20 Jahre lang mindestens einmal die Woche gespielt habe und die ich sehr mag.
Ich habe es nie wirklich ganz zu meiner Zufriedenheit gespielt, aber ich nahm immer an, daß es wegen des Mangels an Übungszeit war.
Ich begann unter Benutzung Ihrer Vorschläge zu analysieren was falsch war.
Neben einigen offensichtlichen Fehlern, die ich nie wirklich korrekt gelernt hatte, war das überraschendste Ergebnis, daß es für mich unmöglich war, mit dem Metronom im Takt zu bleiben!
Eine detailliertere Analyse offenbarte die Ursache - ein großer Teil von Poulencs Werk erfordert schnelle und merkwürdige Verschiebungen der Handposition bei Melodien, die über diese Verschiebungen hinweg durchgehalten werden müssen.
Die schlechte Angewohnheit, die ich gelernt hatte, war, während dieser Verschiebungen nach den Tasten zu \enquote{greifen} und so die Melodielinie zu zerstören und das Stück schrittweise zu beschleunigen.
Die Offenbarung für mich war, daß das Problem nicht durch Üben mit dem Metronom behoben werden konnte!
Es konnte nur durch die Analyse des Problems und die Ausarbeitung einer Strategie zur Behandlung der Verschiebungen behoben werden.
Nun bin ich sehr zufrieden mit der Art wie ich spiele und habe sogar viel Zeit übrig, um über die Musik nachzudenken.<br>
Alborada del Graziosa ist ein Fall für sich.
Das ist ein höllisch schwieriges Stück, das ich in der Vergangenheit versucht habe zu lernen, aber bei dem ich nicht in der Lage war, die meisten Passagen auf die korrekte Geschwindigkeit zu bringen.
Meine Annahme war immer, daß mehr Üben nötig war, und daß ich nie die Zeit finden konnte.
Wieder habe ich die Methoden in Ihrem Buch angewandt, um das zu lernen, und nach drei Wochen bin ich zwar noch nicht am Ziel, aber ich kann nun das meiste davon mit der richtigen Geschwindigkeit spielen und auch ziemlich musikalisch.
Ich schätze, daß ich es innerhalb von ein paar Wochen in den Fingern habe und mich auf die Musik konzentrieren kann.<br>
Zu guter Letzt die Chopin-Prélude.
Ich hatte diese für eine Prüfung gelernt als ich 16 Jahre alt war aber seit damals nie wirklich gespielt.
Ich begann, sie erneut zu lernen und machte ein paar Entdeckungen.
Als erstes hatte ich sie nie mit der endgültigen Geschwindigkeit gespielt, sogar bei der Prüfung, das war also etwas, was ich beheben mußte.
Das funktionierte jedoch nicht - ich entdeckte, daß ich aus zwei Gründen nicht schneller werden konnte.
Erstens hatte ich gelernt, das Legato mit dem Pedal nachzuahmen - aber wenn man schneller wird, erhält man nur ein Durcheinander von Tönen, und wenn ich versuchte, das Pedal richtig zu benutzen, konnte ich das Legato nicht hinbekommen.
Zweitens enthält der mittlere Abschnitt einige weit gestreckte, gebrochene Akkorde in der linken Hand, die auf jedem Schlag verschoben werden.
Langsam gespielt ist das OK aber bei der richtigen Geschwindigkeit wird das höllisch schwierig und sogar schmerzhaft zu spielen.
Im Grunde mußte ich das Stück erneut lernen - neuer Fingersatz, neue Handpositionen, anderes Pedalieren usw.
Nun kann ich das mit jeder Geschwindigkeit, die ich mag, ohne Streß spielen.
Ich fand das einen interessanten Beweis für das, was Sie im Buch sagen - das ist ein sehr kleines Stück, das ziemlich einfach erscheint aber bei der richtigen Geschwindigkeit seinen Charakter völlig ändert und jeden Schüler frustrieren wird, der die intuitive Methode benutzt, es sei denn er ist mit einer Spanne von mehr als 1,5 Oktaven gesegnet.<br>
Abschließend möchte ich Ihnen für das Schreiben dieses Buchs danken und noch mehr dafür, daß Sie es über das Internet verfügbar gemacht haben.
Ich habe in der Vergangenheit enorme Summen Geld für sehr angesehene Lehrer ausgegeben und nicht einer davon, obwohl ich nicht bezweifle, daß sie diese Techniken selbst beherrschen, konnte mich lehren, wie man übt.


\item \label{testimonials10}
Ich denke, Ihr Buch ist es wert, daß ich es lese, obwohl ich viele der Regeln (wie z.B. mit getrennten Händen üben, Akkord-Anschlag . . . ) von meinen Lehrern gelernt habe.
Sogar wenn nur eine Regel, die ich aus Ihrem Buch gelernt habe, funktioniert, dann ist das meines Erachtens weitaus mehr wert als die 15\$, die ich für die erste Ausgabe bezahlt habe.
Ich mag auch den Abschnitt über die Vorbereitung auf Konzerte.
Ich stimme zu, daß vor dem Konzert mit voller Geschwindigkeit zu spielen verboten ist.
Ich habe das mit meinem Lehrer besprochen, und wir sehen verschiedene Gründe warum [ausgedehnte Diskussionen darüber, warum am Tag des Konzerts mit voller Geschwindigkeit zu spielen, zu Problemen führen kann; hier nicht angegeben, weil ich sie nicht verstehen kann].
Deshalb ist vor dem Konzert schnell zu üben eine Situation, in der man nichts gewinnen kann.
Ich würde gerne mehr darüber sehen, wie man auf Geschwindigkeit kommt und wie man die Hände effizienter zusammenbringt.
Manche Musik (ich denke an Bachs Inventionen) ist mit getrennten Händen leicht zu spielen aber schwierig mit beiden Händen zusammen.
Alles in allem macht es mir Spaß, Ihr Buch zu lesen.


\item \label{testimonials11}
Ich empfehle jedem, das Üben mit getrennten Händen zu versuchen, wie es in Ihrem Buch erklärt wird.
Während ich bei Robert Palmieri an der Kent State University studierte, ließ er es mich als Teil meiner Übungen tun.
Es half mir, über das Amateurstadium hinauszukommen und zu einer viel besseren Technik und musikalischem Spielen.


\item \label{testimonials12}
Auf der Grundlage dessen, was ich Ihrer Website entnehmen konnte, wandte ich eines der Prinzipien - das Spielen mit getrennten Händen bei voller Geschwindigkeit - auf eine Reihe schwieriger Passagen in zwei völlig unterschiedlichen Stücken an, die ich spielte, eines ein Kirchenlied, das andere ein Jazz-Stück.
Interessanterweise fand ich, als ich gestern in der Kirche war und es Zeit wurde, die Gemeinde zu begleiten, daß die schwierigen Teile, die ich mit der Methode der getrennten Hände gelernt hatte, unter den festesten und sichersten des ganzen Lieds waren.
Es schien, daß jedesmal, wenn ich zu einem dieser schwierigen Punkte kam, ein geistiger Schalter anging, so daß mein Gehirn und Nervensystem diese Teile mit besonderer Sorgfalt und Genauigkeit ausführten.
Dasselbe gilt für den schwierigen Punkt in dem Jazz-Stück, der nun überhaupt kein Problem mehr ist.


\item \label{testimonials13}
Ungefähr vor eineinhalb Jahren kaufte ich das Buch \enquote{Fundamentals of Piano Practice} von Ihnen.
Ich wollte Ihnen einfach persönlich für Ihren Beitrag danken.
Es hat mir ziemlich viel geholfen!
Vor Ihrem Buch wußte ich nie, wie man üben soll, weil es mir nie beigebracht wurde.
Ich nahm durchaus Unterricht, aber meine Lehrer hatten mich nie gelehrt wie man übt.
Ist das nicht erstaunlich!
Ich habe den Verdacht, daß das alltäglich ist.
Der nützlichste Rat für mich ist Ihr Vorschlag, beim letzten Durchgang des Stücks, das man übt, mit viel geringerer Geschwindigkeit zu spielen.
Ich muß zugeben, daß es am schwierigsten für mich war, diese Angewohnheit zu entwickeln.
Aber ich versuche es.
Ich finde, daß langsames Üben eine große Hilfe ist.
Nur jeweils einen Takt oder zwei zu üben war auch wertvoll!
Ich wünschte, daß es mir leichter fallen würde, Noten auswendig zu lernen; wenn Sie irgendwelche neuen Ideen zum Auswendiglernen haben, lassen Sie es mich bitte wissen.
[Ich habe seit dieser Korrespondenz beträchtliches Material über das Auswendiglernen hinzugefügt.]


\item \label{testimonials14}
Danke, daß Sie meine Fragen zum Klavierüben beantwortet haben.
Ich muß Ihnen sagen, daß es eine besonders verzwickte Prélude von Chopin gibt - die in Cis-Moll.
Als ich Ihr Buch erhielt, bewältigte ich diese Prélude innerhalb eines Tages und schneller als mit ihrer hohen Geschwindigkeit.
Zugegeben, es ist eine kurze, aber viele Klavierspieler ringen damit.
Diese Erfahrung war sehr ermutigend.


\item \label{testimonials15}
Ich spiele nun seit 8 Jahren Klavier und habe Ihr Buch vor ungefähr einem Jahr gekauft.
Nachdem ich dieses Buch gelesen habe, sind meine täglichen einstündigen Übungssitzungen viel produktiver.
Ich lerne auch neue Stücke viel schneller.
Sie zeigen Einblicke in folgendes:<br>
Korrekte Übungsmethoden.<br>
Wie man ein neues Stück anfängt.<br>
Langsames Üben (wann man es benutzt und warum).<br>
Wann man schneller spielen soll als normal.<br>
Wie man sich auf ein Konzert vorbereitet.<br>
Ich stimme nicht allem zu was Sie schreiben, aber ich lese Ihr Buch ungefähr alle zwei Monate, damit ich die richtige Art zu üben nicht aus den Augen verliere.
[Das wird oft gesagt: Mein Buch ist eine solch verdichtete Zusammenfassung, daß man es mehrere Male lesen muß.]


\item \label{testimonials16}
Nach einer Woche war ich sehr zufrieden mit mir und der Methode, da ich dachte, daß ich eine ganze Seite mit HS erfolgreich AUSWENDIGGELERNT!!! hätte.
Das war eine absolut unbekannte Leistung, soweit es mich betraf.
Probleme kamen aber auf, als ich versuchte, die beiden Hände zusammenzubringen, was ich dann in der Zeit versuchte, in der ich den Rest des Stücks lernte.
Als ich versuchte, den Rest des Stücks zu lernen, fand ich auch heraus, daß ich die erste Seite falsch \enquote{auswendiggelernt} hatte, und ich machte schließlich Anmerkungen für mich selbst.
[Das geschieht wahrscheinlich häufiger als die meisten von uns zugeben - wenn Sie Schwierigkeiten haben, HT zur endgültigen Geschwindigkeit zu kommen - PRÜFEN SIE DIE NOTEN! - könnte die Ursache ein Fehler beim Notenlesen sein.
Fehler beim Rhythmus sind besonders schwer zu entdecken.]
Ihr Buch HAT mir genau das gegeben, wonach ich gesucht habe - d.h. eine Grundlage um herauszuarbeiten, wie man schneller und effizienter lernt.
Kein Lehrer war jemals in der Lage, mir einen Anhaltspunkt dafür zu geben, wie man an das Lernen eines Stücks herangeht.
Der einzige Vorschlag, den ich jemals bekam, war \enquote{Schau es Dir an, und sieh, was Du daraus machen kannst.}, und dafür, wie man die Genauigkeit und/oder die Geschwindigkeit verbessert \enquote{Üben, Üben, . . .} WAS?????
Ich habe nun die Antworten auf diese entscheidenden Fragen erhalten. Danke.


\item \label{testimonials17}
Ich habe Ihr Buch auf Ihrer Website gelesen und habe viel für mich herausgeholt.
Sie haben mich dazu inspiriert, auf die Art zu üben, von der ich immer gewußt habe, daß es die beste Art ist aber niemals die Geduld dazu hatte.
Was Sie über gleichmäßige Akkorde vor dem Versuch schnell zu spielen schreiben, hat mir sehr geholfen.
Ich glaube, daß meine Unfähigkeit, über eine bestimmte Geschwindigkeit hinaus zu spielen, von einer grundlegenden Ungleichmäßigkeit in meinen Fingern kommt, um die ich mich nie wirklich gekümmert habe.
Ich sagte immer nur: \enquote{Ich kann nicht gut schnell spielen.}
Ich habe ein kleines Stück einer Etüde mit dem Akkord-Anschlag aufgearbeitet und kann es tatsächlich ziemlich flüssig und gleichmäßig spielen!
Ich bin neugierig auf Ihre Theorien über die Entwicklung eines absoluten Gehörs.
Die Lager scheinen in bezug auf dieses Thema sehr weit auseinander zu sein: Genetik und Umwelt.
[Seit dieser Korrespondenz habe ich die Übungen für parallele Sets zum Üben der Akkorde hinzugefügt und habe einen ausgedehnten Abschnitt über das Aneignen eines absoluten Gehörs geschrieben.]


\item \label{testimonials18}
Ich wollte Sie einfach wissen lassen, wie gut meiner Familie von Musikern Ihr Buch über das Klavierspielen gefallen hat.
Ohne Zweifel haben Sie in Ihrem Buch einige innovative, unorthodoxe Ideen vorgetragen, die trotz der Tatsache, daß sie gemessen an den Standards der meisten Klavierlehrer extrem klingen, wirklich funktionieren.
[Ich stimme zu!]
Die Methode, die Hände getrennt zu üben, scheint genauso gut zu funktionieren wie die Methode nicht alles soooooo langsam zu spielen!
Auch hat es sich als nützlich erwiesen, nicht so viel Betonung auf das Metronom zu legen.
Mit Sicherheit haben Ihre Methoden geholfen, den ganzen Lernprozeß für neue Stücke zu beschleunigen, und ich kann mir nun nicht vorstellen, wie wir jemals ohne das Wissen um Ihre \enquote{musikalischen Wahrheiten} zurechtgekommen sind.
Danke nochmals, daß Sie ein solch wunderbares JUWEL von einem Buch geschrieben haben!


\item \label{testimonials19}
Ich habe die Online-Kapitel gelesen und bin der Meinung, daß von jedem Klavierlehrer verlangt werden müßte, dieses Buch gelesen zu haben.
Ich bin einer der Unglücklichen, der 7 Jahre damit verbracht hat, ohne die geringste Ahnung von Entspannung oder effizienten Übungsmethoden Tonleitern und Hanon zu üben.
Ich fing damit an, gute Übungshinweise aus Diskussionsforen im Internet und verschiedenen Büchern zu sammeln, aber Ihr Buch ist die bei weitem umfassendste und überzeugendste Quelle, die ich bisher gefunden habe.


\item \label{testimonials20}
Ich bin ein Klavierspieler der Mittelstufe.
Vor einem Monat habe ich Teile Ihres Buchs heruntergeladen, und ich muß in einem Wort sagen, daß es fabelhaft ist!
Als Wissenschaftler schätze ich die strukturierte Art, in der das Material des Themas präsentiert und auf einer grundlegenden Stufe erklärt wird.
Es hat meine Betrachtungsweise des Klavierübens verändert.
Besonders der Teil über das Auswendiglernen hat mir bereits geholfen, den Aufwand für das Auswendiglernen erheblich zu verringern.
Mein Privatlehrer (ein auftretender Solist) benutzt den einen oder anderen Teil Ihrer Methode.
Der Lehrer ist jedoch ein Czerny-Anhänger und hat noch nie vom Daumenübersatz gehört.
Sie müssen dem Daumenübersatz mehr Beachtung schenken, besonders dem flüssigen Aneinanderfügen von parallelen Sets.
Ich habe das Buch an meinen Lehrer weitergegeben und empfehle es jedem.<br>
[Ein Jahr später:]<br>
Ich habe Ihnen vor über einem Jahr einmal wegen Ihres phantastischen Buchs im Internet geschrieben.
Die Methoden funktionieren wirklich.
Durch das Benutzen Ihrer Methoden war ich in der Lage, einige Stücke viel schneller zu lernen und zu meistern.
Ihre Methoden funktionieren wirklich bei Stücken, die bekanntermaßen schwierig auswendig zu lernen sind, wie einige Mozart-Sonaten und Stücken, von denen mein Klavierlehrer sagte, daß sie schwierig auswendig zu lernen sind, wie die Bach-Inventionen oder einige Préludes von Chopin.
Total einfach, wenn man Ihre Methode benutzt.
Ich nehme nun die Fantaisie Impromptu in Angriff, und dieses scheinbar unmögliche Stück liegt offensichtlich innerhalb meiner Reichweite!
Ich mag auch Ihren Beitrag über das Unterbewußtsein.
Ich frage mich, ob Sie das Buch \enquote{The Mindbody Prescription} von J. D. Sarno kennen.
Dieses Buch behandelt das Unterbewußtsein genau wie Sie es tun.
Als ich an meiner Doktorarbeit arbeitete, löste ich meine scheinbar unlösbaren theoretischen Rätsel genau wie Sie es getan haben.
Ich fütterte mein Gehirn damit, und ein paar Tage später platzte die Lösung einfach heraus.
Was Sie schreiben ist also absolut richtig!


\item \label{testimonials21}
Ihre Vorschläge, wie man Musik auswendig lernt, indem man Assoziationen erzeugt (z.B. eine Geschichte), klangen für mich unklug.
Aber als ich übte, konnte ich nicht anders als zu fragen, was ich mit einer bestimmten musikalischen Phrase assoziieren könnte, die einen problematischen F-Akkord hatte.
\enquote{Gib Dir selbst ein F für falsch gespielt.} kam mir in den Sinn.
Ich dachte, das wäre kein sehr ermutigender Gedanke!
Aber jedesmal, wenn ich nun zu dieser Phrase komme, erinnere ich mich an das F.
Ich hab's. Meine Herren! Danke. Ihr Buch ist sehr nützlich.
Es spiegelt die Vorschläge meiner Lehrerin wider aber mit mehr Details.
Wenn ich nicht Klavierspielen kann, macht nichts mehr Spaß als etwas über das Klavierspielen zu lesen . . .
In den letzten Wochen vor meinem letzten Konzert schlug meine Lehrerin vor, während des Übens durch meine Fehler hindurchzuspielen.
Dann zurückzugehen und an den problematischen Takten zu arbeiten, größtenteils so wie Sie vorschlagen, obwohl es das einzige Mal war, daß das vorkam.
Sie sagt, daß die meisten Menschen den Fehler nicht einmal bemerken, solange er die Musik nicht unterbricht.
Ihr Punkt ist, die Musik nicht zu unterbrechen und das Problem an der Quelle zu beheben, indem man zu dem Takt zurückgeht.
Ich finde, daß ich mich sehr oft korrigiere (stottere); ich werde mich darauf konzentrieren, es nicht zu tun.
Dieser Rat ist nicht intuitiv, wie Sie wissen.
Man korrigiert Fehler wie von selbst, wenn sie auftreten.
Aber ich sehe ein, daß dies ständig zu tun in Wahrheit die Fehler verfestigt.


\item \label{testimonials22}
Ich stolperte über Ihr Online-Buch über das Klavierüben, als ich nach Artikeln über das absolute Gehör suchte.
Als ich es las, war ich von dem verwendeten wissenschaftlichen Vorgehen beeindruckt.
Besonders das Konzept der \enquote{Geschwindigkeitsbarrieren} und wie man sie überwindet half mir sehr.
Ich fand Ihr Buch gerade zur richtigen Zeit.
Viele Probleme, denen ich beim Klavierspielen begegne, werden in Ihrem Buch besprochen.
Viele Klavierlehrer haben anscheinend kein klares wissenschaftliches Konzept dafür, wie man bestimmte Probleme von Klavierspielern der Mittelstufe behandelt.
Deshalb arbeite ich mich Abschnitt für Abschnitt mit gutem Erfolg durch das Buch.
Es gibt verschiedene Dinge, die ich in Ihrem Buch vermisse.
In manchen Kapiteln wären Bilder hilfreich, wie z.B. korrekte Handpositionen, Daumenübersatz, Übungen für parallele Sets.
Etwas wie eine chronologische Tabelle für den Übungsablauf könnte nützlich sein.
\enquote{Kalt üben} wäre z.B. an der ersten Position.
Sie weisen immer auf die Wichtigkeit hin, WANN man WAS tun soll.
Könnten Sie die Übungen, die Sie erklären, in einer Weise ordnen, die sie am effizientesten macht?
Auf alle Fälle möchte ich meine tiefe Dankbarkeit für Ihr Projekt aussprechen!


\item \label{testimonials23}
Den ganzen Winter hindurch habe ich selbst weiter Klavier gelernt und ich muß sagen, daß jedes Wort in Ihrem Buch wahr ist.
Ich habe das Klavierspielen mehrere Jahre gelernt und nur einen durchschnittlichen Fortschritt gemacht.
Weil ich das Klavier und romantische Musik liebe, macht mich das manchmal verrückt und zutiefst frustriert.
Ich wende Ihre Methoden seit ungefähr einem Jahr an und machte enorme Fortschritte.
Ich arbeite nun an mehreren Stücken gleichzeitig, Kompositionen von denen ich nie zuvor gedacht hatte, daß ich sie spielen könnte.
Es ist wunderbar.
Heute habe ich ein kleines Repertoire, daß ich mit großer Befriedigung spielen kann.


\item \label{testimonials24}
Ich habe Ihr Buch der ersten Ausgabe bestellt und erhalten und habe Teile Ihrer zweiten Ausgabe gelesen.
Ich fand Ihre Information extrem wertvoll.
Ich sende Ihnen diese E-Mail in der Hoffnung, ein paar Ratschläge für mein kommendes Konzert zu bekommen.
Ich bin extrem nervös, aber nachdem ich Ihre Abschnitte über Konzerte gelesen habe, verstehe ich deren Wichtigkeit.
Ich wünschte, ich hätte Ihre Anmerkungen über das Auswendiglernen gehabt als ich anfing, weil ich extrem viel Zeit gebraucht habe, es endlich (auf die falsche Art) auswendig zu lernen.
Ich bin mir nicht sicher, wie ich das Stück beim Konzert vorspielen soll.
Bei den wenigen Gelegenheiten, bei denen ich für andere gespielt habe, stolperte ich über bestimmte Abschnitte, weil ich wegen meiner Nerven vergaß, wo ich im Stück war.
Das ist mein erstes Konzert, so daß ich nicht weiß, was mich erwartet.
Jeder Tip oder Rat über Übungsabläufe wäre mir sehr willkommen.<br>
[Nachdem wir uns ein paarmal darüber ausgetauscht hatten, was er spielte, usw., gab ich ihm ein Szenario von typischen Übungsabläufen für die Konzertvorbereitung und was er während des Konzerts erwarten sollte.
Nach dem Konzert erhielt ich die folgende E-Mail:]<br>
Ich wollte Sie nur wissen lassen, daß mein Konzert, dafür daß es das erste Mal war, extrem gut verlaufen ist.
Der Rat, den Sie mir gegeben haben, war sehr hilfreich.
Ich war nervös als ich das Stück begann, wurde dann aber extrem fokussiert (so wie Sie es sagten, daß es geschehen würde).
Ich konnte mich sogar auf die Musik konzentrieren und nicht nur die Bewegungen durchgehen.
Das Publikum war von meiner Fähigkeit, es aus dem Gedächtnis heraus zu tun, beeindruckt (wie sie sagten, daß sie es würden).
Sie hatten Recht damit, daß eine positive Erfahrung wie diese mir mit meinem Selbstvertrauen helfen würde.
Ich fühle mich aufgrund der Erfahrung großartig!
Mein Lehrer ist von [ein berühmtes Konservatorium] und lehrt Hanon und anderes technisches Material.
Deshalb war und ist Ihr Buch eine Goldmine für mich.
Ich möchte in der Lage sein, die Stücke zu spielen, die ich mag, ohne 20 Jahre damit verbringen zu müssen, sie zu lernen.
Ich fühle aber auch, daß ich einen Lehrer brauche.


\item \label{testimonials25}
[Und schließlich hunderte von Zuschriften der Art:]

Ich muß sagen, daß Ihr Buch hervorragend ist . . .

Seit ich C. C. Chang's Fundamentals of Piano Practice gelesen habe, habe ich seine Vorschläge ausprobiert; Dank an diejenigen, die es empfohlen haben und an Herrn Chang, daß er sich die Zeit genommen hat, es zu schreiben und dafür, daß er es verfügbar gemacht hat.

Ich fand Ihre Webseiten bei meinem Arbeiten für das Klavierspielen sehr nützlich.

Ihre Arbeit ist einfach wunderbar!

Das ist hilfreich und ermutigend, da ich zum Klavier zurückkehre, nachdem ich viele Jahre keines zur Verfügung hatte; danke!

Sie haben mir enorm geholfen.

Nach dem, was ich bisher gelesen habe, macht es viel Sinn, und ich bin darauf gespannt es auszuprobieren.

Usw., usw.


</ol> 



<p align=\enquote{center}>Danksagung</p>

Dieses Buch ist meiner Frau Merry gewidmet, deren Liebe, Unterstützung und grenzenlose Energie mich in die Lage versetzt haben, für dieses Projekt so viel Zeit zu verwenden.
 

\textbf{\textit{Ende der Übersetzung dieser Seite}}

<table>
 Original: & \hyperref[http://www.pianopractice.org]{http://www.pianopractice.org} (extern) \\ 
</table>





<!-- ueberset.html --> 

\label{ueberset}

<h2 align=\enquote{center}>Anmerkungen zur Übersetzung</h2>

Ich hatte das Buch zunächst satzweise übersetzt und dabei den Sinn der von Chuan C. Chang benutzten Worte soweit wie möglich erhalten.
Mir ist bewußt, daß der Text dadurch an manchen Stellen etwas ungewöhnlich klang, aber insbesondere bei den Anweisungen zur Spieltechnik war mir die \enquote{Werktreue} wichtiger als die Sprachgewohnheiten.
Nachdem das Buch nun (mit Ausnahme der fortlaufenden Ergänzungen durch Chuan C. Chang) komplett übersetzt ist und ich inzwischen wesentlich besser verstehe, worauf es in den einzelnen Passagen ankommt, passe ich die Wortwahl während der Überarbeitung zunehmend an den alltäglichen Sprachgebrauch an, wenn ich mir sicher bin, den tieferen Sinn eines Abschnitts nicht zu verfälschen.

Wie übersetzt man das \textbf{you}?
 Da ich - von ein paar Ausnahmen abgesehen - auch nicht möchte, daß mich einfach jemand duzt, stand es für mich außer Frage, daß ich in meiner Übersetzung \textbf{you} in der Regel mit \textbf{Sie} übersetze.
Damit der Text dadurch nicht zu förmlich wird, habe ich mich dazu entschieden, nur bei Anleitungen, Empfehlungen usw.
den Leser durch die Verwendung von \textbf{Sie} direkt anzusprechen.
Bei allgemeinen Ausführungen, Erklärungen und Hintergrundinformationen habe ich dagegen oft \textbf{man} benutzt.


Ursprünglich hatte ich an manchen Stellen des Texts eine Mischung aus der weiblichen und der männlichen Form eines Wortes oder Ausdrucks benutzt.
Bei der Überarbeitung ersetze ich diese nun durch die derzeit immer noch häufiger benutzte männliche Form.
Das bedeutet nicht, daß ich jetzt zum Macho mutiert bin, sondern dient der flüssigeren Lesbarkeit, weil dadurch Konstrukte wie z.B. \enquote{die/der Klavierlehrer/in} entfallen.
Vielleicht gibt es ja bei der nächsten Rechtschreibreform auch zu diesem Thema etwas Neues.
Bis dahin bleibe ich jedenfalls bei der alten Rechtschreibung.


\label{Pedale}

Das \textbf{linke Pedal} wird meistens \textbf{Dämpfer- bzw. Dämpfungspedal oder Sordino} genannt und das \textbf{rechte Pedal} als \textbf{Haltepedal oder Verlängerungspedal} bezeichnet.
Neuerdings gibt es aber auch Quellen, in denen die Namen der Pedale an die englischen Ausdrücke angeglichen sind.
Darin wird nun das rechte Pedal als Dämpfer- bzw. Dämpfungspedal (damper pedal) bezeichnet und das linke u.a. als Pianopedal (soft pedal).
Es ist mir bisher nicht gelungen, den Ursprung dieser neuen Bezeichnungen ausfindig zu machen.
Solange ich nicht weiß, ob es sich hierbei wirklich um eine neue Wortbildung in der deutschen Sprache handelt oder die Änderung nur durch eine weitergetragene Falschübersetzung der Ausdrücke \enquote{damper pedal} und \enquote{soft pedal} entstanden ist, werde ich die bisherigen Begriffe im Text beibehalten.

Teilweise gibt es englische Wörter, für die es zwar im Deutschen ein Wort gibt, das jedoch negativ belegt ist, wie z.B. \enquote{exposed = ausgesetzt}.
Wurden solche Wörter in einem positiven Sinn gebraucht, habe ich versucht sie so zu umschreiben, daß der Sinn \underline{und} der positive Eindruck erhalten bleiben.<br>
\label{memorizer}
Außerdem gibt es Wörter, für die es - ähnlich wie für das \enquote{Hier-kommt-der-Einkauf-des-nächsten-Kunden-Holz} an der Supermarktkasse - keinen vernünftigen Ausdruck im Deutschen gibt.
So hatte ich \textbf{memorizer} zunächst mit \enquote{Merker} übersetzt. Nach fast 2 Jahren habe ich mich dann entschieden, das teilweise negativ belegte Wort \enquote{Merker} (z.B. Blitzmerker) zu ersetzen und \enquote{memorizer} mit \textbf{Auswendiglernender} und \enquote{non-memorizer} mit \textbf{Nichtauswendiglernender} zu übersetzen. Die etwas kürzere Fassung \enquote{Auswendiglerner} wird oft abwertend gebraucht (z.B. Kassenbon-Auswendiglerner), weshalb ich sie nicht benutze.<br>
\label{reversepsychology}
Für den Begriff \textbf{reverse psychology} scheint es außer der für mich unbefriedigenden Eindeutschung \textbf{umgekehrte Psychologie} keinen Ausdruck zu geben.
Damit ist gemeint, jemandem etwas so zu sagen, daß er hinterher das Gegenteil tut, und man genau das erreichen wollte.


\label{HsHt}

Die Abkürzungen wie \textbf{HS} (hands seperated) und \textbf{HT} (hands together) habe ich absichtlich nicht übersetzt.
Wenn Übersetzer aus anderen Ländern das gleiche tun, dann haben Leser aus unterschiedlichen Ländern weniger Probleme, sich untereinander über die in diesem Buch vorgestellten Methoden auszutauschen, weil es z.B. keinen Deutschen gibt, der von GH (getrennte Hände) und keinen Franzosen, der von MS (mains séparées) spricht.


\label{Motherboard}

Können Sie sich etwas unter einem \textit{Mutterbrett} vorstellen? Wie wäre es mit \textit{Hauptplatine}?
Schon besser oder?
Oder unter einer \textit{Klangkarte}? Da ist es leichter.<br>
In den Abschnitten über \hyperref[c1iii13MIDI]{MIDI, Digitalpianos usw.} hat es mich ja förmlich in den Fingern gejuckt aber ich habe die Begriffe \textbf{Motherboard, Soundkarte usw.} kommentarlos dringelassen, da jemand, der sich mit PCs auskennt, weiß was gemeint ist.
Trotzdem fände ich es gut, wenn die schleichende Anglisierung der deutschen Sprache ein wenig gebremst würde.
Dann bekomme ich vielleicht auch mal wieder wohlschmeckende Haferflocken statt crispiger Cerealien zum Frühstück.


\label{Noten}

Die \textbf{Noten} werden im Text gemäß der Konvention der Klavierstimmergilde bezeichnet, d.h. die 88 Tasten des Klaviers tragen die Bezeichnungen \textit{A0} bis \textit{C8}.
\textit{C4} steht somit für das \textit{mittlere C = c'}.
Die meisten Sequenzer-Programme verwenden jedoch einen anderen Wertebereich.
Der MIDI-Wert 0 entspricht dabei \textit{C-2}, was zu \textit{C3} für das mittlere C führt.
Um die Verwirrung zu komplettieren, beginnen manche Programme mit \textit{MIDI 0 = C0}, was zu \textit{C5} für das mittlere C führt.


\label{ueb-canonic}
Chuan C. Chang benutzt im Original ein Wortspiel mit dem aus der Thermodynamik stammenden Begriff \hyperref[canonic]{\textbf{canonical ensemble}} und der musikalischen Bedeutung der Begriffe \textit{canonical} und \textit{ensemble}, das mit der korrekten deutschen Übersetzung \textit{kanonische Gesamtheit} leider nicht mehr funktioniert.


\label{ueb-KV525}
Der Text des Wortwechsels der männlichen und weiblichen Stimme am Anfang von Mozarts \hyperref[KV525]{\enquote{Eine Kleine Nachtmusik} (KV525)} stammt nicht aus der Sekundärliteratur.
Er ist dem Originaltext von Chuan C. Chang angelehnt (\enquote{Hey, are you coming?} und \enquote{OK, OK, I'm coming!}) und berücksichtigt das 9-notige Schema sowie die Sprachgewohnheiten in Opern der damaligen Zeit.
Wenn Sie eine Quelle mit dem genauen Wortlaut kennen, lassen Sie es mich bitte wissen.


\label{upright}
Im allgemeinen ist aus dem Zusammenhang ersichtlich, ob es sich bei dem Wort \textbf{Klavier} um den Sammelbegriff (piano) handelt oder um ein Klavier im engeren Sinne, d.h. um ein aufrecht stehendes Klavier (upright) im Gegensatz zum Flügel (grand).
In der Übersetzung der Datei über \hyperref[c2_1]{das Stimmen} war aber zunächst nicht immer sofort zu erkennen, welche der beiden Bedeutungen jeweils gemeint war.
Da diese Eindeutigkeit jedoch beim Stimmen von besonderer Bedeutung ist, habe ich \textbf{upright} in dieser Datei ausnahmsweise mit \textbf{das \enquote{Aufrechte}} übersetzt.
In manchen Quellen findet man dafür auch die Bezeichnung \textit{Piano}, was aber als veraltet gilt.
Hin und wieder findet man auch \textit{Pianino}, aber das ist genaugenommen ein \enquote{Aufrechtes} mit verringertem Tonumfang, d.h. weniger als 88 Tasten.


\label{pin}
Ist ein \textbf{tuning pin} nun ein \textit{Stimmwirbel}, weil er zum Stimmen der Saite gedreht wird, oder ein \textit{Stimmnagel}, weil er eine relativ glatte Oberfläche hat und manchmal zur Rettung der Stimmbarkeit in den Stimmstock gehämmert wird?
Im Internet findet man \textit{Stimmwirbel} am meisten, der Begriff \textit{Stimmnagel} kommt allerdings auch recht häufig vor.
Manchmal findet man in einer Quelle sogar beide Begriffe in (un-)harmonischer Eintracht nebeneinander.
In Analogie zu anderen Instrumenten, bei denen zum Stimmen der Saiten ähnliche Teile \enquote{herumgewirbelt} werden, habe ich mich für \textbf{Stimmwirbel} entschieden. 


\label{et}
Der Begriff \textbf{gleichstufige Temperatur} bzw. \textbf{gleichmäßige Temperatur} ist korrekter als die meistens in der Literatur zu findende \textbf{gleichschwebende Temperatur}.
Die Frequenzverhältnisse der Halbtonschritte (kleine Sekunde) haben immer den gleichen Wert (100 Cent, bzw. zwölfte Wurzel aus 2, ungefähr 1,059463) und somit auch alle Intervalle (z.B. Quinte = 700 Cent).
Die exakten Werte der Schwebungsfrequenzen der Intervalle weichen jedoch voneinander ab (s. Reblitz S. 212ff.). Die Schwebungsfrequenz von zwei Saiten ist die Differenz der Grund- bzw. Obertöne der Saiten, deren Frequenzen am dichtesten beieinanderliegen. Dazu ein paar Beispiele, ausgehend von A4 = 440 Hz:

<table border cellpadding=\enquote{7}>
 <tr>
  <td bgcolor=\enquote{\#E0E0E0}>\textbf{Intervall}</td>
  <td bgcolor=\enquote{\#E0E0E0}>\textbf{1. Ton}</td>
  <td bgcolor=\enquote{\#E0E0E0}>\textbf{2. Ton}</td>
  <td bgcolor=\enquote{\#E0E0E0}>\textbf{Schwebung} \\ 
 kl. Sekunde & A4 = 440Hz & A\#4 pprox 466,164Hz & pprox 26,164Hz \\ 
   & A\#4 pprox 466,164Hz & H4 pprox 493,884 Hz & pprox 27,720Hz \\ 
 Quinte & A3: 2. Oberton = 660Hz & E4: 1. Oberton pprox 659,256Hz & pprox 0,744Hz \\ 
   & A\#3: 2. Oberton pprox 699,246Hz & F4: 1. Oberton pprox 698,456Hz & pprox 0,790Hz \\ 
</table>

Diese Werte gelten nur für idealisierte Saiten. Die Obertöne und somit auch die Schwebungsfrequenzen realer Saiten weichen wegen der \hyperref[c2_5_stre]{Inharmonizität} davon ab.

\label{mitschwingung}
Das korrekte Wort für \textbf{sympathetic vibration} ist tatsächlich \textbf{Mitschwingung}, da es hier um das Verhalten zweier oder mehrerer gleichzeitig angeschlagener Saiten geht, die ungefähr mit der gleichen Frequenz schwingen.
Das scheinbar naheliegende Wort \textit{Resonanz} beschreibt dagegen, daß eine nicht angeschlagene Saite, deren Frequenz z.B. das ganzzahlige Vielfache der Frequenz der angeschlagenen Saite ist, zur Schwingung angeregt wird. 


\label{moden}
\textbf{Schwingungsmoden} ist der physikalische Begriff für die verschiedenen Schwingungszustände einer Saite, deren Enden befestigt sind.
Vereinfacht könnte man sagen, sie sind die einzelnen Frequenzen, aus denen die Schwingung der Saite zusammengesetzt ist, d.h. der Grundton und die Obertöne.


\label{transversal}
\textbf{Transversale Wellen} haben ihre Amplitude (\enquote{Ausschlag}) quer zur Fortpflanzungsrichtung, wie z.B. an der Wasseroberfläche oder eben in Klaviersaiten.
Wenn eine Klaviersaite also so fest eingespannt wäre, daß sie sich nicht (oder kaum) quer zu ihrer Längsrichtung bewegen könnte, bliebe das Klavier ziemlich stumm.
Wellen, die ihre Amplitude in Bewegungsrichtung haben, heißen \textbf{longitudinale Wellen}.
Man sieht sie z.B. an der Oberseite eines Kornfelds, durch das der Wind weht.
Schallwellen sind ebenfalls longitudinale Wellen.


\label{johndoe}
Vielleicht habe ich es bei der Übersetzung des \hyperref[assoziativ]{assoziativen Prozesses zum Abrufen der Informationen aus dem Gedächtnis} ein wenig übertrieben, aber ich wollte \textbf{John} nicht einfach nur mit \enquote{Johann} oder \enquote{Johannes} übersetzen, weil \enquote{\textbf{Otto} Normalverbraucher} die deutsche Entsprechung von \enquote{John Doe, the average American} ist.




\label{AbkFarben}

<h2 align=\enquote{center}>Im Text verwendete Abkürzungen und Farben</h2>
LH = Left Hand = Linke Hand<br>
RH = Right Hand = Rechte Hand<br>
HS = Hands Separate = (im Wechsel) nur mit der LH oder RH spielen<br>
HT = Hands Together = mit beiden Händen zusammen spielen<br>
TO = Thumb Over = Daumenübersatz<br>
TU = Thumb Under = Daumenuntersatz<br>
FI = Fantaisie Impromptu, Op. 66, von Frederic Chopin /
 \hyperref[FI]{(1)},
 \hyperref[c1iii2]{(2)},
 \hyperref[c1iii5wagen]{(3)}<br>
FPD = Fast Play Degradation = \hyperref[fpd]{Abbau von Fähigkeiten durch zu schnelles Spielen}<br>
NG = Nucleation Growth = \hyperref[ng]{Kernbildung-Wachstums-Mechanismus}<br>
PPI = Post Practice Improvement = \hyperref[c1ii15]{automatische Verbesserung der Fähigkeiten nach dem Üben}<br>
ET = Equal Temperament = gleichschwebende, gleichstufige, bzw. gleichmäßige Temperatur
 (\hyperref[et1]{1}),
 (\hyperref[c2_6_et]{2})<br>
WT = Well Temperament(s) = \hyperref[c2_2_wtk2]{Wohltemperierte Stimmung(en)}<br>
HT = Historical Temperament(s) = \hyperref[c2_2_hist]{Historische Temperatur(en)}<br>
K-II = Kirnberger II
 (\hyperref[c2_2_wtk2]{1}),
 (\hyperref[c2_6_kirn]{2})<br>
<br>



<table>
 \footnote{Blauer kursiver Text in eckigen Klammern} & Vom Übersetzer eingefügte Anmerkungen, die im Original nicht enthalten sind \\ 
 Grüner normaler Text & Zukünftige Links auf Abschnitte, die noch übersetzt werden \\ 
 \textbf{\textit{Grüner fetter kursiver Text}} & Hinweise auf Änderungen \\ 
  (extern)  & Kennzeichnung von Links auf Seiten außerhalb von FOPPDE \\ 
</table>


<!-- danke.html -->



\label{Danke}

<h2 align=\enquote{center}>Danke!</h2>
\textbf{Vielen Dank an alle, die an der Erstellung dieser Übersetzung in irgendeiner Weise beteiligt waren bzw. immer noch sind, insbesondere an:}


\begin{itemize} 
 \item \textbf{meine Frau} für all die Liebe und Unterstützung und die Begeisterung für das Projekt.
 \item \textbf{Chuan C. Chang} dafür, daß er dieses Buch geschrieben hat und für den regen Gedankenaustausch.
 \item \textbf{Christof Pflumm} von der Uni Karlsruhe für die Übersetzung der Passagen mit physikalischen Fachbegriffen.
 \item \textbf{das Team von www.planet-school.de} für das fast 5 Jahre kostenlose Hosten der Site.
 \item \textbf{meine beiden Englischlehrer, Frau Storandt} (ladies first, obwohl sie in der Chronologie an zweiter Stelle kommt) \textbf{und Herr Fues}.
Sie haben den Grundstein dafür gelegt, daß ich all die Jahre stets Spaß an der englischen Sprache hatte und nun in der Lage war bzw. bin, die vorliegende Übersetzung zu erstellen.
 \end{itemize}


%: Ende des Buchs




\label{kontakt}

<table>
 <tr valign=\enquote{top}>
  <td>\textbf{Anregungen, Kritiken, Verbesserungsvorschläge usw. \underline{zum rein fachlichen Inhalt} bitte an den Autor (wenn möglich in Englisch) per}
   \\ 
</table>

<table>
 <tr valign=\enquote{top}>
  <td>\textbf{E-Mail:} & \hyperref[mailto:cc88m@aol.com?subject=foppde]{cc88m@aol.com} \\ 
 <tr valign=\enquote{top}>
  <td>\textbf{oder Post:} & Chuan C. Chang<br>16212 Turnbury Oak Dr.<br>Odessa<br>FL 33556<br>USA \\ 
</table>

\label{mailel}

<table>
 <tr valign=\enquote{top}>
  <td>\textbf{\underline{Alles weitere} bitte an den Übersetzer per} \\ 
</table>

<table>
 <tr valign=\enquote{top}>
  <td>\textbf{E-Mail:} & \hyperref[mailto:musik@uteedgar-lins.de?subject=foppde]{musik@uteedgar-lins.de} \\ 
 <tr valign=\enquote{top}>
  <td>\textbf{oder Post:} & Edgar Lins<br>Usinger Straße 46<br>61231 Bad Nauheim<br>
Deutschland \\ 
</table>



<p align=\enquote{center}>
\hyperref[http://www.uteedgar-lins.de/index.html]{Homepage von www.uteedgar-lins.de} 
\hyperref[./index.html]{Homepage von FOPPDE} 
\hyperref[Inhalt]{Inhaltsverzeichnis} 
</p>

<p align=center>http://foppde.uteedgar-lins.de/foppde.html</p>

</body>
</html>
