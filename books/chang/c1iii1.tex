% File: c1iii1

\chapter{Ausgewählte Themen des Klavierübens}
\label{c1iii1}

% zuletzt geändert 14.02.2010

\section{Klang, Rhythmus, Legato, Staccato}

\subsection{Was ist ein \enquote{Guter Klang}?}
\label{c1iii1a}

\subsubsection{Der Basisanschlag}
\label{c1iii1a1}

Der Basisanschlag muss von jedem Klavierspieler gelernt werden.
Ohne ihn macht alles andere keinen bedeutenden Unterschied - man kann aus Lehmziegeln und Stroh kein indisches Grabmal bauen.
\textbf{Der Anschlag besteht aus drei Hauptkomponenten: dem Abschlag, dem Halten und dem Anheben.}
Das mag als trivial einfach zu lernen erscheinen, ist es aber nicht, und für die meisten Klavierlehrer ist es ein zähes Ringen, ihren Schülern den korrekten Anschlag beizubringen.
Die Schwierigkeiten entstehen hauptsächlich daraus, dass die Mechanismen des Anschlags noch nicht angemessen erklärt wurden; deshalb werden diese Erläuterungen das Hauptthema dieser Abschnitte sein.

Zunächst wird der Klavierklang vom Abschlag erzeugt; bei der korrekten Bewegung muss dieser so schnell wie möglich, die Lautstärke aber kontrolliert sein.
Diese Kontrolle ist nicht so einfach, da wir im Abschnitt über den \hyperref[c1ii10]{Freien Fall} herausgefunden haben, dass ein schnellerer Abschlag im Allgemeinen zu einem lauteren Ton führt.
Die Schnelligkeit verleiht der Note ihr präzises Timing; ohne diese Schnelligkeit fängt das Timing der Note an, unordentlich zu werden.
Deshalb muss der Abschlag - unabhängig davon, ob die Musik langsam oder schnell ist - im Grunde schnell sein.
\textbf{Die Erfordernis eines schnellen Abschlags, der Kontrolle der Lautstärke und vieler anderer Faktoren, die wir in Kürze kennenlernen werden, bringen uns zu einem sehr wichtigen Prinzip des Klavierspielenlernens: der Sensibilität der Finger.
Die Finger müssen in der Lage sein, viele Voraussetzungen zu erfühlen und zu erfüllen, bevor man den Basisanschlag meistern kann.}
Um die Lautstärke zu kontrollieren, sollte der Abschlag aus zwei Teilen bestehen: einer anfänglichen starken Komponente, um die Reibung bzw. Trägheit der Taste zu überwinden und die Bewegung zu beginnen, und einer zweiten Komponente mit einer der gewünschten Lautstärke entsprechenden Kraft.
Die Empfehlung, \enquote{tief in die Tasten zu spielen}, ist in dem Sinne gut, dass der Abschlag nicht langsamer werden darf; er muss zum unteren Punkt hin beschleunigt werden, sodass man nie die Kontrolle über den Hammer verliert.

\textbf{Diese zweiteilige Bewegung ist besonders wichtig, wenn man pianissimo spielt.}
Bei einem gut eingestellten Konzertflügel ist die Reibung fast null und die Trägheit des Systems gering.
Bei allen anderen Klavieren (was 99 Prozent aller Klaviere umfasst) gibt es eine zu überwindende Reibung, besonders wenn man mit dem Abschlag beginnt (die Reibung ist am höchsten, wenn die Bewegung null ist), und es gibt zahlreiche Ungleichgewichte im System, die Trägheit hervorrufen.
Vorausgesetzt, das Klavier ist richtig \hyperref[c2_7_hamm]{intoniert}, können Sie sehr leise pianissimo spielen, indem Sie zunächst die Reibung bzw. die Trägheit überwinden und dann den sanften Abschlag machen.
Diese beiden Komponenten müssen fließend ineinander übergehen, sodass es für einen Beobachter wie eine einzige Bewegung aussieht, bei der das Fleisch der Finger wie ein Stoßdämpfer wirkt.
Der erforderliche schnelle Abschlag bedeutet, dass die Fingermuskeln zu einem großen Teil aus schnellen Muskelfasern bestehen müssen (siehe unten \hyperref[c1iii7a]{Abschnitt 7a}).
Das wird durch das Üben schneller Bewegungen über einen längeren Zeitraum (ungefähr ein Jahr) und das Vermeiden von Kraftübungen erreicht; deshalb ist die Behauptung, dass Klaviertechnik Fingerkraft erfordert, absolut falsch.
Man muss die Geschwindigkeit und Empfindlichkeit der Finger entwickeln.

Die Haltekomponente des Anschlags ist notwendig, um den Hammer mit Hilfe des Fängers zu halten und die Notendauer genau zu kontrollieren.
Ohne das Halten kann der Hammer umherspringen, was zusätzliche Töne erzeugt, Probleme mit Trillern und wiederholten Noten verursacht usw.
Anfänger werden Schwierigkeiten mit dem Übergang vom Abschlag zum Halten haben.
\textbf{Drücken Sie die Taste während des Haltens nicht nach unten, um zu versuchen, \enquote{tief in das Klavier zu drücken}; die Schwerkraft reicht aus, um die Taste unten zu halten.}
Die Länge des Haltens kontrolliert die Farbe und den Ausdruck; deshalb ist es ein wichtiger Teil der Musik.

Das Anheben lässt den Dämpfer auf die Saiten fallen und beendet den Ton.
Zusammen mit dem Halten bestimmt es die Notendauer.
Wie der Abschlag muss auch das Anheben schnell geschehen, um die Notendauer genau zu kontrollieren.
Deshalb muss sich der Klavierspieler bewusst bemühen, so wie für den Abschlag in den Beugemuskeln, auch in den Streckmuskeln schnelle Muskelfasern zu bilden.
Besonders beim schnellen Spielen werden viele Schüler das Anheben völlig vergessen, was zu unsauberem Spielen führt.
Ein Lauf könnte somit aus Staccato, Legato und sich überlappenden Noten bestehen.
Schnelle parallele Sets könnten so klingen, als ob sie mit Pedal gespielt würden.

\textbf{Indem Sie alle drei Komponenten des Basisanschlags genau kontrollieren, behalten Sie die völlige Kontrolle über das Klavier; insbesondere über den Hammer und den Dämpfer, und diese Kontrolle ist für ein selbstsicheres Spielen notwendig}.
Diese Komponenten bestimmen die Natur jeder Note.
Sie können nun sehen, warum ein schneller Abschlag und ein genauso schnelles Anheben so wichtig sind - besonders während des langsamen Spielens.
Beim normalen Spielen fällt das Anheben der Note mit dem Abschlag der nächsten Note zusammen.
Beim \hyperref[c1iii1c]{Staccato} und Legato (siehe Abschnitt c) und schnellen Spielen (7i) müssen wir alle diese Komponenten verändern, und wir werden das gesondert behandeln.
Wenn Sie diese Komponenten nie zuvor geübt haben, beginnen Sie das Üben mit allen fünf Fingern, C bis G, wie beim Spielen einer Tonleiter, und wenden Sie alle Komponenten auf jeden einzelnen Finger an (\hyperref[c1ii7]{mit getrennten Händen}).
Wenn Sie die Streckmuskeln trainieren möchten, können Sie die schnellen Bewegungen zum Anheben übertreiben.
Versuchen Sie, alle nicht spielenden Finger leicht auf den Tasten liegen zu lassen.
Wenn Sie versuchen, die Ab- und Aufwärtsbewegungen zu beschleunigen, und ungefähr eine Note pro Sekunde spielen, werden Sie eventuell Stress aufbauen.
Üben Sie in diesem Fall so lange, bis Sie den Stress eliminieren können.
Denken Sie bei der Haltekomponente immer daran, dass Sie sofort nach dem schnellen Abschlag während des Haltens entspannen müssen.
Mit anderen Worten: \textbf{Sie müssen sowohl die Bewegungsgeschwindigkeit als auch die Entspannungsgeschwindigkeit trainieren.}
Steigern Sie dann schrittweise die Spielgeschwindigkeit; es ist jetzt noch nicht notwendig, schnell zu spielen.
Kommen Sie nur zu einer bequem handhabbaren Geschwindigkeit.
Machen Sie nun dasselbe mit getrennten Händen mit einem langsamen Musikstück, das Sie spielen können, wie dem \hyperref[c1ii25b]{ersten Satz von Beethovens Mondschein-Sonate}.
Wenn Sie es vorher noch nie getan haben, wird das \hyperref[c1ii25]{beidhändig} zunächst sehr merkwürdig sein, weil Sie so viele Komponenten mit beiden Händen koordinieren müssen.
Mit zunehmender Übung wird die Musik jedoch besser werden, Sie gewinnen eine größere Kontrolle über den Ausdruck und sollten das Gefühl bekommen, dass Sie nun musikalischer spielen können.
Es sollte keine fehlenden oder falschen Noten geben, alle Noten sollten gleichmäßiger sein, und Sie können alle Ausdruckszeichen präziser ausführen.
Der Vortrag wird von einem Tag zum anderen beständig sein, und die Technik wird sich schneller weiterentwickeln.
Ohne einen guten Basisanschlag können Sie in Schwierigkeiten geraten, wenn Sie auf verschiedenen Klavieren oder auf nicht gut eingestellten Klavieren spielen, und die Musik kann nach häufigerem Üben oft schlechter sein, da Sie sich \hyperref[c1ii22]{schlechte Angewohnheiten} aneignen können, zum Beispiel ein ungenaues Timing.
Natürlich mag es Wochen oder sogar Monate dauern, bis der ganze in diesem einen Absatz beschriebene Prozess abgeschlossen ist.


\subsubsection{Klang: Einzelne gegenüber mehreren Noten, Pianissimo, Fortissimo}
\label{c1iii1a2}

\textbf{Klang ist die Qualität des Tons; sie ist ein Urteil darüber, ob die Summe aller Eigenschaften des Klangs der Musik angemessen ist}.
Es wird kontrovers diskutiert, ob ein Klavierspieler den \enquote{Klang} einer einzelnen Note auf dem Klavier steuern kann.
Wenn man sich an ein Klavier setzen und eine Note spielen sollte, scheint es fast unmöglich zu sein, den Klang - außer hinsichtlich solcher Eigenschaften wie staccato, legato, laut, leise usw. - zu ändern.
Auf der anderen Seite steht außer Frage, dass verschiedene Klavierspieler unterschiedliche Klänge hervorbringen.
Zwei Klavierspieler können dasselbe Stück auf demselben Klavier spielen und Musik von sehr unterschiedlicher Klangqualität erzeugen.
Das meiste dieses scheinbaren Widerspruchs kann aufgelöst werden, indem man sorgfältig definiert, was \enquote{Klang} bedeutet.
Ein großer Teil der Klangunterschiede zwischen Klavierspielern kann zum Beispiel auf das jeweilige Klavier zurückgeführt werden, das sie benutzen, und darauf, wie diese Klaviere eingestellt oder gestimmt sind.
Den Klang einer einzelnen Note zu steuern, ist wahrscheinlich nur ein Aspekt eines facettenreichen, komplexen Themas.
Deshalb ist die wichtigste Unterscheidung, die wir zunächst treffen müssen, ob wir über eine einzelne Note oder eine Gruppe von Noten sprechen.
Wenn wir verschiedene Töne hören, dann hören wir uns meistens eine Gruppe von Noten an.
In diesem Fall sind Klangunterschiede leichter zu erklären.
Der Klang wird größtenteils durch die Kontrolle der Noten relativ zueinander erzeugt.
Das bedeutet fast immer durch Präzision, Kontrolle und musikalischen Inhalt.
\textbf{Deshalb ist Klang hauptsächlich eine Eigenschaft einer Gruppe von Noten und hängt vom musikalischen Gespür des Spielers ab.}

\textbf{Es ist jedoch auch klar, dass wir den Klang einer einzelnen Note auf mehrere Arten steuern können.}
Wir können ihn durch den Gebrauch des Halte- und des Dämpferpedals steuern\footnote{Anmerkungen zu den Bezeichnungen der Pedale finden Sie \hyperref[Pedale]{hier}}.
Wir können auch den harmonischen Gehalt (die Zahl der Obertöne) ändern, indem wir lauter oder leiser spielen.
Das \hyperref[c1ii24]{Dämpferpedal ändert den Klang, oder das Timbre}, durch die Reduzierung des Anschlagklangs relativ zum Nachklang.
Wenn eine Saite mit einer stärkeren Kraft angeschlagen wird, werden mehr harmonische Schwingungen erzeugt.
Somit erzeugen wir, wenn wir leiser spielen, oft einen Klang mit stärkeren Grundtönen.
Unterhalb einer bestimmten Lautstärke kann die Energie jedoch zu gering sein, um den Grundton zu erzeugen, und es kann sein, dass nur einige wandernde Wellen mit höheren Frequenzen erregt werden - ähnlich dem Flautando bei der Geige (die Trägheit der Saiten wirkt wie die Finger beim Flautando).
Deshalb gibt es irgendwo zwischen \textit{pp} und \textit{ff} eine optimale Anschlagskraft, die den Grundton maximiert.
Das Haltepedal ändert ebenfalls das Timbre, indem es Schwingungen von den nicht angeschlagenen Saiten hinzufügt.

\textbf{Der Klang oder das Timbre können vom Klavierstimmer durch das \hyperref[c2_7_hamm]{Intonieren der Hämmer} oder durch eine andere Stimmung gesteuert werden.}
Ein härterer Hammer erzeugt einen brillanteren Klang (größerer harmonischer Gehalt), und ein Hammer mit einer flachen  Aufschlagsfläche erzeugt einen schrilleren Klang (mehr hochfrequente Obertöne).
Der Klavierstimmer kann die \hyperref[c2_5_stre]{Streckung} ändern oder den Grad der Verstimmung in den Unisoni steuern.
Bis zu einem bestimmten Punkt führt eine größere Streckung zu einem helleren Klang, und ungenügende Streckung kann ein Klavier mit einem wenig aufregenden Klang erzeugen.
Wenn alle Saiten einer Note innerhalb des \hyperref[c2_5_mits]{Mitschwingungsbereichs} verstimmt werden, sind sie in perfekter Stimmung (schwingen mit der gleichen Frequenz), reagieren aber unterschiedlich miteinander.
So kann zum Beispiel die Note zum \enquote{Singen} gebracht werden, das heißt die Lautstärke des Nachklangs schwankt.
Es gibt keine zwei Saiten, die wirklich identisch sind, sodass es einfach keine Möglichkeit gibt, identisch zu stimmen.

Zum Schluss kommen wir zu einer schwierigen Frage: \textbf{Kann man den Klang einer einzelnen Note durch die Steuerung des Abschlags variieren?}
Die meisten Argumente für die Klangsteuerung konzentrieren sich auf den Freien Fall des Hammers, bevor er die Saiten anschlägt.
Gegner (der Klangkontrolle einer einzelnen Note) argumentieren, dass, weil der Hammer im Freien Fall ist, nur seine Geschwindigkeit eine Rolle spielt und deshalb der Klang einer Note, die mit einer bestimmten Lautstärke gespielt wird, nicht steuerbar sei.
Aber die Annahme des Freien Falls wurde nie bewiesen, wie wir nun sehen werden.
\textbf{Ein Faktor, der den Klang beeinflusst, ist die Biegung des Hammerstiels.}
Bei einer lauten Note kann der Stiel deutlich gebogen werden, wenn der Hammer in den Freien Fall übergeht.
In diesem Fall kann der Hammer eine größere effektive Masse als seine wirkliche Masse haben, wenn er die Saiten trifft.
Das kommt daher, dass die Kraft ($F$), mit der der Hammer auf die Saiten wirkt, durch $F = M*\textbf{a}$ gegeben ist, wobei $M$ die Masse des Hammers und $\textbf{a}$ seine Verzögerung beim Auftreffen auf der Saite ist.
Positive Biegung fügt eine zusätzliche Kraft hinzu, weil diese, wenn die Biegung nach dem Lösen der Stoßzunge aufgehoben wird, den Hammer vorwärts schiebt; wenn $F$ zunimmt, ist es egal, ob $M$ oder $\textbf{a}$ zunimmt, der Effekt ist derselbe.
$\textbf{a}$ ist jedoch schwieriger zu messen als $M$ (zum Beispiel kann man leicht ein größeres $M$ simulieren, indem man einen schwereren Hammer benutzt), weshalb wir in diesem Fall üblicherweise sagen, dass die \enquote{effektive Masse} zugenommen hat, um es leichter zu machen, sich den Effekt der größeren $F$ darauf vorzustellen, wie die Saiten reagieren.
In Wirklichkeit erhöht die positive Biegung jedoch \textbf{a}.
Für eine staccato gespielte Note kann die Biegung negativ sein, wenn der Hammer die Saiten anschlägt, sodass der Klangunterschied zwischen \enquote{tiefem} Spielen und staccato erheblich sein kann.
Diese Veränderungen der effektiven Masse werden sicherlich die Verteilung der Obertöne verändern und den Ton, den wir hören, beeinflussen.
\textbf{Da der Stiel niemals hundertprozentig starr ist, wissen wir, dass es immer eine begrenzte Biegung gibt.
Die einzige Frage ist, ob sie ausreichend ist, den Klang, so wie wir ihn hören, zu beeinflussen.}
Sie ist es fast mit Sicherheit, da der Hammerstiel ein relativ biegsames Stück Holz ist.
Wenn das wahr ist, dann sollte der Klang der tieferen Noten mit den schwereren Hämmern kontrollierbarer sein, weil die schwereren Hämmer eine stärkere Biegung verursachen.
Obwohl man erwarten könnte, dass die Biegung vernachlässigbar ist, weil der Hammer so leicht ist, ist die Hammernuss sehr nah am Drehpunkt des Hammerstiels und erzeugt eine enorme Hebelwirkung.
Das Argument, dass der Hammer zu leicht sei, um eine Biegung zu erzeugen, zieht nicht, weil der Hammer genügend massiv ist, um die ganze kinetische Energie aufzunehmen, die erforderlich ist, sogar die lautesten Töne zu erzeugen.
Das ist eine Menge Energie!

Beachten Sie, dass der Hammerabgang nur ein paar Millimeter beträgt, und dass dieser Abstand extrem entscheidend für den Klang ist.
Solch ein kleiner Abgang suggeriert, dass der Hammer so gestaltet wurde, dass er in Beschleunigung ist, wenn er die Saite trifft.
Der Hammer ist nachdem die Stoßzunge auslöst nicht im Freien Fall, weil der Hammer auf den ersten wenigen Millimetern durch die Rückbildung der Stielbiegung beschleunigt wird.
Der Abgang ist die kleinste kontrollierbare Entfernung, welche die Beschleunigung aufrechterhalten kann, ohne dass der Hammer an den Saiten festhängen kann, weil die Stoßzunge nicht auslösen konnte.
Diese Biegung erklärt vier, ansonsten mysteriöse, Tatsachen:

\begin{enumerate}[label={\roman*.}] 
 \item die gewaltige Energie, die solch ein leichter Hammer auf die Saiten übertragen kann,
 \item die Abnahme der Klangqualität (oder -kontrolle), wenn der Abgang zu groß ist,
 \item die entscheidende Abhängigkeit der Tonstärke und Klangsteuerung vom Hammergewicht und der Hammergröße
 \item und den klickenden Ton, den das Klavier von sich gibt, wenn die Buchse des Hammerstiels ausleiert (ein klassisches Beispiel ist die klickende Teflonbuchse).
\end{enumerate}

Das Klicken ist der Ton der Buchse, die zurückspringt, wenn die Stoßzunge loslässt und die Stielbiegung übernimmt - ohne eine zurückgehende Biegung existiert keine Kraft für das Zurückschnappen der Buchse; deshalb gibt es ohne die Biegung kein Klicken.
Da das Klicken sogar bei einigermaßen leisen Tönen zu hören ist, ist der Stiel außer bei den leisesten Tönen bei allen gebogen.

Dieses Szenario hat auch wichtige Auswirkungen für den Klavierspieler (nicht nur für den Klavierstimmer).
Es bedeutet, dass der Klang einer einzelnen Note kontrolliert werden kann.
Es sagt uns auch, wie er kontrolliert werden kann.
Erstens ist bei \textit{ppp}-Tönen die Biegung vernachlässigbar, und wir kümmern uns um den unterschiedlichen Klang der lauteren Töne.
Pianisten wissen, dass man zum \textit{pp}-Spielen\footnote{die Tasten} mit einer konstanten Geschwindigkeit niederdrücken muss - beachten Sie, dass das die Biegung minimiert, weil es keine Beschleunigung beim Auslösen gibt.
Wenn man pianissimo spielt, möchte man die Biegung minimieren, um die effektive Masse des Hammers zu verringern.
Zweitens sollte der Abschlag für eine maximale Biegung am Ende am schnellsten sein.
Das macht Sinn: Ein \enquote{tiefer Ton} wird durch das Hineinlehnen in das Klavier und festes Niederdrücken erzeugt, auch bei leisen Tönen.
Genau so maximieren Sie die Biegung, es kommt dem Gebrauch eines größeren Hammers gleich.
Diese Information ist auch für den Klaviertechniker entscheidend.
Sie bedeutet, dass die optimale Hammergröße genügend klein ist, sodass die Biegung irgendwo um \textit{pp} null ist, aber groß genug ist, dass die Biegung um \textit{mf} deutlich anfängt.
Das ist eine sehr clevere mechanische Anordnung, die das Benutzen von relativ kleinen Hämmern erlaubt, die schnelle Wiederholungen gestatten und trotzdem eine maximale Energiemenge auf die Saiten übertragen können.
Es bedeutet, dass es ein Fehler ist, größere Hämmer zu benutzen, um mehr Klang zu erzeugen, weil die Repetiergeschwindigkeit und die Klangkontrolle verlorengehen.
Die Existenz der Biegung des Hammerstiels ist nun wohlbekannt (\hyperref[Lectures]{Five Lectures on the Acoustics of the Piano}).

Kann man den Unterschied im Klang einer einzelnen Note auf einem Klavier hören, indem man nur eine Note spielt?
Normalerweise nicht; die meisten Menschen sind nicht empfindlich genug, um diesen Unterschied bei den meisten Klavieren zu hören.
Sie werden ein Steinway B oder ein besseres Klavier benötigen, und Sie werden vielleicht anfangen, diesen Unterschied bei den tieferen Noten zu hören (wenn Sie das mit mehreren Klavieren mit stetig höherer Qualität testen).
Wenn jedoch wirklich Musik gespielt wird, ist das menschliche Ohr erstaunlich empfindlich dafür, wie der Hammer auf die Saiten trifft, und dieser Klangunterschied kann leicht gehört werden.
Das ist dem Stimmen ähnlich: Die meisten Menschen (einschließlich der meisten Klavierspieler) werden große Schwierigkeiten haben, den Unterschied zwischen einer hervorragenden Stimmung und einer gewöhnlichen Stimmung zu hören, indem sie einzelne Noten oder Intervalle probieren.
Praktisch jeder Klavierspieler kann jedoch den Unterschied in der Qualität der Stimmung hören, indem er eines seiner Lieblingsstücke spielt.
Sie können das selber demonstrieren.
Spielen Sie ein leichtes Stück zweimal auf die gleiche Weise, außer hinsichtlich des Anschlags.
Spielen Sie zunächst mit Armgewicht und \enquote{pressen Sie tief} in das Klavier, und stellen Sie sicher, dass der Tastenfall den ganzen Weg nach unten beschleunigt wird (korrekter \hyperref[c1iii1a1]{Basisanschlag}).
Vergleichen Sie das mit der Musik, die entsteht, wenn Sie nur leicht drücken, sodass die Taste zwar ganz nach unten geht, es aber keine Beschleunigung am Ende gibt.
Sie müssen wahrscheinlich ein wenig üben, um sicherzustellen, dass es beim ersten Mal nicht lauter ist als beim zweiten Mal.
Sie sollten bei der zweiten Spielweise eine mindere Klangqualität hören.
In den Händen eines großen Pianisten kann dieser Unterschied ziemlich groß sein.
Natürlich haben wir oben besprochen, dass der Klang am stärksten dadurch kontrolliert wird, wie man aufeinanderfolgende Noten spielt, sodass Musik zu spielen nicht der beste Weg ist, um den Effekt einzelner Noten zu testen.
Es ist jedoch der empfindlichste Test.

\textbf{Pianissimo}:
Wir haben gesehen, dass man für \textit{ppp} einen genauen \hyperref[c1iii1a1]{Basisanschlag} und eine schnelle \hyperref[c1ii14]{Entspannung} benötigt.
Die Tasten mit den Fingerpolstern zu erfühlen ist wichtig.
Im Allgemeinen sollten Sie immer mit einem leichten Anschlag üben, bis Sie die Passage gemeistert haben.
Fügen Sie dann das \textit{mf}, \textit{ff} oder was notwendig ist hinzu, weil mit einem leichten Anschlag zu spielen die am schwersten zu entwickelnde Fertigkeit ist.
Es gibt keine Beschleunigung des Abschlags und keine Biegung des Hammerstiels, aber der Fänger muss kontrolliert (die Taste unten gehalten) werden.
\textbf{Der wichtigste Faktor für \textit{ppp} ist das richtige Einstellen des Klaviers (besonders ein minimaler Abgang, das \hyperref[c2_7_hamm]{Intonieren der Hämmer} und das richtige Hammergewicht).
Zu versuchen, die Technik des \textit{ppp} ohne die richtige Wartung des Klaviers zu erreichen und zu erhalten ist sinnlos.}
Im Notfall (während einer \hyperref[c1iii14]{Aufführung} mit einem ungenügenden Instrument), können Sie es bei einem Klavier mit dem \hyperref[c1ii24]{Dämpferpedal} und bei einem Flügel mit teilweise getretenem Dämpferpedal versuchen.
\textit{ppp} ist auf den meisten Digitalpianos schwierig, weil die Mechanik der Tastatur qualitativ schlechter ist und zunehmend verschleißt, wenn das Klavier ungefähr fünf Jahre benutzt wurde.
Aber ein akustisches Klavier, das nicht gewartet wird, kann viel schlechter sein.

Das \textbf{Fortissimo} ist eine Frage der Übertragung von Gewicht auf das Klavier.
Das bedeutet, dass man sich nach vorne lehnt, damit der Schwerpunkt näher an der Tastatur liegt und man \textbf{aus den Schultern spielt}.
Benutzen Sie nicht nur die Hände oder Arme für das \textit{ff}.
Die \hyperref[c1ii14]{Entspannung} ist wieder wichtig, sodass Sie keine Energie verschwenden, für eine maximale Geschwindigkeit des Abschlags sorgen und die richtige Kraft nur dorthin richten, wo Sie benötigt wird.
\textbf{Üben Sie bei einer Passage, die \textit{ff} gespielt werden soll, ohne das \textit{ff}, bis die Passage gemeistert ist, und fügen Sie dann das \textit{ff} hinzu.}

Zusammengefasst ist Klang in erster Linie ein Ergebnis der Einheitlichkeit und der Kontrolle des Spielens und hängt von dem musikalischen Gefühl des Spielers ab.
\textbf{Klangkontrolle ist ein komplexes Thema, das jeden Faktor einbezieht, der die Natur des Tons verändert, und wir haben gesehen, dass es viele Wege gibt, den Klang des Klaviers zu ändern.}
Alles fängt damit an, wie das Klavier eingestellt ist.
Jeder Klavierspieler kann den Klang mit zahlreichen Mitteln steuern, wie laut oder leise zu spielen oder durch das Variieren der Geschwindigkeit.
Indem wir zum Beispiel lauter und schneller spielen, können wir Musik erzeugen, die hauptsächlich aus dem Anschlagston besteht;
ein langsameres und leiseres Spielen wird einen schwächeren Effekt erzeugen und benutzt mehr den Nachklang.
Und es gibt zahllose Arten, das Pedal in Ihr Spiel einzubeziehen.
Wir haben gesehen, dass der Klang einer einzelnen Note gesteuert werden kann, weil der Hammerstiel biegsam ist.
Die große Zahl der Variablen sorgt dafür, dass jeder Klavierspieler einen anderen Klang erzeugt.
 

\subsection{Was ist Rhythmus? (Beethovens Sturm-Sonate und Appassionata)}
\label{c1iii1b}

\textbf{Rhythmus ist der (sich wiederholende) zeitliche Rahmen der Musik}.
Wenn man etwas über Rhythmus liest (siehe \hyperref[Whiteside]{Whiteside}), erscheint er oft wie ein mysteriöser Aspekt der Musik, den man nur mit \enquote{angeborenem Talent} zum Ausdruck bringen kann.
Oder vielleicht muss man ihn das ganze Leben lang üben, wie Schlagzeuger.
\textbf{Meistens ist der korrekte Rhythmus jedoch einfach eine Frage des genauen Zählens und des korrekten Lesens des Notats, insbesondere der Taktart.
Das ist nicht so einfach wie es klingt; Schwierigkeiten treten oft auf, weil die meisten Rhythmuszeichen nicht überall ausdrücklich auf dem Notenblatt angegeben sind, da sie Teil von Merkmalen wie der Taktart sind, die nur einmal am Anfang angegeben wird} (es gibt zu viele solcher \enquote{Merkmale}, um sie hier aufzulisten, wie zum Beispiel den Unterschied zwischen einem Walzer und einer Mazurka.
Ein weiteres Beispiel: Ohne auf die Noten zu sehen wird mancher denken, dass bei dem Lied \enquote{Happy Birthday} der Schlag auf \enquote{Happy} liegt, er ist aber auf \enquote{Birth-}; dieses Lied ist ein Walzer).
In vielen Fällen wird die Musik hauptsächlich durch eine Manipulation dieser rhythmischen Variationen erzeugt, sodass der Rhythmus eines der wichtigsten Elemente der Musik ist.
Kurz gesagt: Die meisten Schwierigkeiten mit dem Rhythmus resultieren daraus, dass man die Noten nicht richtig liest.
Das geschieht oft, wenn man versucht, die Noten für beide Hände gleichzeitig zu lesen; das Gehirn hat einfach zu viele Informationen zu verarbeiten und kann sich nicht um den Rhythmus kümmern, besonders wenn die Musik neue technische Fertigkeiten einschließt.
Dieser anfängliche Fehler beim Notenlesen wird dann beim wiederholten Üben in die entstehende Musik eingebaut.

\textbf{Die Definition des Rhythmus:}
Der Rhythmus besteht aus zwei Teilen - der zeitlichen Abfolge und der Betonung -, die in zwei Formen auftreten: formal und logisch.
Das Geheimnisvolle am Rhythmus und die Schwierigkeiten bei seiner Definition resultieren aus dem \enquote{logischen} Teil, der gleichzeitig das Schlüsselelement und das am schwersten zu fassende Element ist.
Fangen wir also zunächst mit den einfacheren formalen Rhythmen an.
Nur weil sie einfacher sind, bedeutet das nicht, dass sie nicht wichtig sind; zu viele Schüler machen mit diesen Elementen Fehler, was dazu führen kann, dass die Musik nicht mehr wiederzuerkennen ist.

\textbf{Formale zeitliche Abfolge:
Der formale zeitliche Rhythmus ist durch die Taktart bestimmt}; sie wird einmal am Anfang der ersten Zeile des ersten Notenblatts angegeben\footnote{sowie bei einem Wechsel der Taktart an der entsprechenden Stelle}.
Die wichtigsten Taktarten sind Dreivierteltakt - zum Beispiel Walzer - (3/4), Viervierteltakt (4/4), Zweihalbetakt (2/2 oder alla breve) und Zweivierteltakt (2/4).
Der Walzer hat 3 Schläge je Takt, usw.; die Zahl der Schläge je Takt wird durch den Zähler des Bruchs angezeigt. 
4/4 ist der verbreitetste und wird oft nicht angegeben, obwohl er durch ein \enquote{C} am Anfang angezeigt werden sollte.
Der Zweihalbetakt wird durch das gleiche \enquote{C} angezeigt, das durch eine vertikale Linie in der Mitte in zwei Hälften geteilt wird.
Die Bezugsnote wird durch den Nenner des Bruchs angegeben, sodass der 3/4-Walzer 3 Viertelnoten je Takt umfasst und 2/4 im Prinzip doppelt so schnell ist wie 2/2.
Fast jede Taktart wird aus Vielfachen von 2 oder 3 gebildet, obwohl es Ausnahmen gibt, um besondere Effekte zu erzielen (zum Beispiel 5 oder 7 Schläge).

\textbf{Formale Betonung:}
Jede Taktart hat ihre eigene formale Betonung (lautere und leisere Schläge).
Wenn wir festlegen, dass 3 am lautesten ist, 2 leiser usw., dann hat der (Wiener) Walzer die formale Betonung 311 - das berühmte \enquote{um-ta-ta}; die Betonung liegt auf dem ersten Schlag.
Die Mazurka kann 131 oder 113 haben. 
Der Viervierteltakt hat die formale Betonung 3121, bei 2/2 und 2/4 ist die Betonung 21.
Eine Synkopierung ist ein Rhythmus, bei dem die Betonung an einer anderen Stelle als der formale Akzent liegt; ein synkopierter 4/4 könnte zum Beispiel 2131 oder 2113 sein.
Beachten Sie, dass der 2113-Rhythmus die ganze Komposition hindurch fest ist, die 3 aber an einer unkonventionellen Stelle liegt.

\textbf{Logische Abfolge und Betonung:}
Hier bringt der Komponist seine musikalischen Ideen ein.
Es ist eine Abweichung in der Abfolge und Lautstärke vom formalen Rhythmus.
Obwohl die rhythmische Logik nicht notwendig ist, so ist sie doch fast immer vorhanden.
Häufige Beispiele der zeitlichen rhythmischen Logik sind \enquote{accel.} (um die Dinge ein wenig aufregender zu gestalten), \enquote{decel.} (um vielleicht ein Ende anzuzeigen) oder \enquote{rubato}.
Beispiele der dynamischen rhythmischen Logik sind das Ansteigen oder Abfallen der Lautstärke, \enquote{forte}, \enquote{pp} usw.

\textbf{Beethovens Sonate \enquote{Der Sturm}} (Op. 31, \#2) verdeutlicht die formalen und logischen Rhythmen.
So sind zum Beispiel die ersten drei Takte des dritten Satzes drei Wiederholungen derselben Struktur, und sie folgen einfach dem formalen Rhythmus.
In den Takten 43-46 gibt es jedoch sechs Wiederholungen derselben Struktur in der rechten Hand, aber sie müssen in vier formale rhythmische Takte gepresst werden!
Wenn Sie in der rechten Hand sechs identische Wiederholungen spielen, ist das falsch!
Zusätzlich ist in Takt 47 ein unerwartetes \textit{sf}, das nichts mit dem formalen Rhythmus zu tun hat aber ein sehr wichtiges Element des logischen Rhythmus ist.

Wenn der Rhythmus so wichtig ist, welche Richtlinie kann man dann benutzen, um ihn zu entwickeln?
Offensichtlich \textbf{muss man Rhythmus als ein separates Thema des Übens behandeln, für das man einen besonderen Plan benötigt}.
Reservieren Sie deshalb während des anfänglichen Lernens eines neuen Stückes ein wenig Zeit, um am Rhythmus zu arbeiten.
Ein Metronom, besonders eines mit fortgeschrittenen Funktionen, kann hier hilfreich sein.
Zunächst müssen Sie noch einmal prüfen, ob Ihr Rhythmus mit der Taktart übereinstimmt.
Das kann man nicht in Gedanken tun, auch wenn man das Stück bereits spielen kann - man muss sich die Notenblätter noch einmal ansehen und jede Note überprüfen.
Zu viele Schüler spielen ein Stück einfach in einer bestimmten Weise, \enquote{weil es sich richtig anhört}; das darf man nicht tun.
Sie müssen anhand der Notenblätter überprüfen, ob die richtigen Noten die richtige Betonung gemäß der Taktart tragen.
Nur dann können Sie entscheiden, welche rhythmische Interpretation die beste Art zum Spielen ist und wo der Komponist Verstöße gegen die Grundregeln eingefügt hat (kommt sehr selten vor); viel öfter ist der von der Taktart vorgegebene Rhythmus sehr wohl richtig, klingt aber kontraintuitiv.
Ein Beispiel dafür ist das mysteriöse \enquote{Arpeggio} am Anfang von Beethovens Appassionata (Op. 57).
Ein normales Arpeggio (wie CEG) würde mit der ersten Note (C) beginnen, welche die Betonung (Abschlag) tragen sollte.
Beethoven beginnt jedoch jeden Takt bei der dritten Note des Arpeggios (der erste Takt ist unvollständig und trägt die ersten beiden Noten des \enquote{Arpeggios}); das zwingt Sie dazu, die dritte Note (G) zu betonen, nicht die erste, wenn Sie der Taktart korrekt folgen möchten.
Man findet den Grund für dieses ungewöhnliche \enquote{Arpeggio}, wenn das Hauptthema in Takt 35 eingeführt wird.
Beachten Sie, dass dieses \enquote{Arpeggio} einfach eine invertierte, schematisierte (vereinfachte) Form des Hauptthemas ist.
Beethoven hat uns psychologisch auf das Hauptthema vorbereitet, indem er uns nur den Rhythmus gegeben hat!
Deshalb wiederholt er es, nachdem er es um ein seltsames Intervall erhöht hat - er wollte bloß sichergehen, dass wir den ungewöhnlichen Rhythmus erkannt haben (er benutzte am Anfang seiner Fünften Symphonie dasselbe Mittel, indem er das viernotige Motiv mit einer niedrigeren Tonhöhe wiederholte).
Ein weiteres Beispiel ist Chopins Fantaisie-Impromptu.
Die erste Note der rechten Hand (Takt 5) muss leiser sein als die zweite.
Können Sie mindestens einen Grund dafür finden?
Obwohl das Stück im 2/2-Takt steht, kann es lehrreich sein, die rechte Hand im 4/4-Takt zu üben, um sicherzustellen, dass nicht die falschen Noten betont werden.

\textbf{Prüfen Sie den Rhythmus sorgfältig, wenn Sie mit \hyperref[c1ii7]{getrennten Händen} beginnen.
Prüfen Sie ihn noch einmal, wenn Sie mit dem \hyperref[c1ii25]{beidhändigen Üben} anfangen.
Wenn der Rhythmus falsch ist, wird es üblicherweise unmöglich, die Musik  mit der vorgegebenen Geschwindigkeit zu spielen.
Deshalb ist es eine gute Idee, den Rhythmus zu überprüfen, wenn man Schwierigkeiten damit hat, auf Geschwindigkeit zu kommen.
Tatsächlich ist eine falsche rhythmische Interpretation eine der häufigsten Ursachen für Geschwindigkeitsbarrieren und Probleme beim beidhändigen Spielen.
Wenn Sie einen rhythmischen Fehler begehen, wird kein Aufwand an Übung Sie in die Lage versetzen, auf Geschwindigkeit zu kommen!}
Das ist einer der Gründe, warum das \hyperref[c1iii8]{Konturieren} funktioniert: Es kann das korrekte Lesen des Rhythmus vereinfachen.
Konzentrieren Sie sich deshalb beim Konturieren auf den Rhythmus.
Auch werden Sie, wenn Sie das erste Mal mit dem beidhändigen Spielen beginnen, mehr Erfolg haben, wenn Sie den Rhythmus betonen.
Der Rhythmus ist ein weiterer Grund, warum Sie keine Stücke versuchen sollten, die zu schwierig für Sie sind.
Wenn Sie nicht genügend Technik haben, werden Sie nicht in der Lage sein, den Rhythmus zu kontrollieren.
Es kann passieren, dass der Mangel an Technik Ihrem Spielen einen falschen Rhythmus aufzwingt und so eine Geschwindigkeitsbarriere erzeugt.

Suchen Sie als nächstes nach besonderen Rhythmuszeichen, zum Beispiel \textit{\textbf{sf}} oder Akzentzeichen.
Schließlich gibt es auch Situationen, in denen keine Zeichen auf dem Notenblatt stehen und man einfach wissen muss, was zu tun ist, oder sich eine Aufnahme anhören muss, um besondere rhythmische Variationen zu erkennen.
Deshalb sollten Sie als Teil des Übungsplans mit dem Rhythmus experimentieren, unerwartete Noten betonen usw., um zu sehen, was passieren könnte.

Rhythmus ist auch eng mit der Geschwindigkeit verbunden.
Deshalb muss man die meisten Kompositionen von Beethoven oberhalb bestimmter Geschwindigkeiten spielen; ansonsten können die Gefühle, die mit dem Rhythmus und sogar mit der Melodieführung verbunden sind, verloren gehen.
Beethoven war ein Meister des Rhythmus; deshalb kann man Beethoven nicht mit Erfolg spielen, ohne dem Rhythmus besondere Aufmerksamkeit zu schenken.
Er gibt Ihnen üblicherweise mindestens zwei Dinge gleichzeitig:

\begin{enumerate}[label={\roman*.}] 
 \item eine leicht zu verfolgende Melodie, die das Publikum hört,
 \item und ein rhythmisches und harmonisches Mittel, das kontrolliert, was das Publikum \textit{fühlt.}
\end{enumerate}

Deshalb kontrolliert das erregende Tremolo der linken Hand im ersten Satz seiner Pathétique (Op. 13) die Gefühle, während das Publikum damit beschäftigt ist, der merkwürdigen rechten Hand zuzuhören.
Deshalb ist eine bloße technische Fähigkeit, das Tremolo der linken Hand zu bewältigen, ungenügend - man muss in der Lage sein, den emotionalen Gehalt durch dieses Tremolo zu kontrollieren.
Wenn Sie dieses rhythmische Konzept verstehen und ausführen können, wird es viel leichter, den musikalischen Gehalt des ganzen Satzes herauszubringen, und der starke Kontrast mit dem \textit{Grave}-Abschnitt wird offensichtlich.

Es gibt eine Klasse rhythmischer Schwierigkeiten, die mit einem einfachen Trick überwunden werden können: die Klasse der komplexen Rhythmen mit fehlenden Noten.
Ein gutes Beispiel dafür kann man im zweiten Satz von Beethovens Pathétique finden.
Der 2/4-Takt ist in den Takten 17 bis 21 wegen der wiederholten Akkorde der linken Hand, die den Rhythmus beibehalten, leicht zu spielen.
In Takt 22 fehlen jedoch die wichtigsten betonten Noten, was es schwierig macht, das etwas komplexe Spielen in der rechten Hand aufzunehmen.
Die Lösung für dieses Problem ist, einfach die fehlenden Noten der linken Hand aufzufüllen!
Auf diese Art können Sie mit der rechten Hand leicht den richtigen Rhythmus üben.

Zusammengefasst ist das \enquote{Geheimnis} eines großartigen Rhythmus kein Geheimnis - er muss mit dem richtigen Zählen beginnen (was, ich muss es noch einmal betonen, nicht einfach ist).
Für fortgeschrittene Klavierspieler ist er natürlich viel mehr; er ist Magie.
Er ist das, was das Große vom Gewöhnlichen unterscheidet.
Er ist nicht nur das Zählen der Betonungen in jedem Takt, sondern die Art und Weise wie die Takte zusammengefügt sind, um die sich entwickelnde musikalische Idee zu erzeugen - die logische Komponente des Rhythmus.
So ist zum Beispiel bei Beethovens Mondschein-Sonate (Op. 27) der Anfang des dritten Satzes im Grunde der erste Satz, der mit einer verrückten Geschwindigkeit gespielt wird.
Dieses Wissen sagt uns, wie man den ersten Satz spielt, weil es bedeutet, dass die Reihe der Triolen im ersten Satz so verbunden werden muss, dass sie zu einer Kulmination mit den drei wiederholten Noten führt.
Würde man die wiederholten Noten einfach unabhängig von den vorangegangenen Triolen spielen, so würden alle diese Noten ihre Bedeutung und Wirkung verlieren.
Rhythmus ist auch der seltsame oder unerwartete Akzent, den unser Gehirn irgendwie als besonders erkennt.
Klar ist der Rhythmus ein entscheidendes Element der Musik, dem man besondere Aufmerksamkeit schenken muss.
 

\subsection{Legato, Staccato}
\label{c1iii1c}

Legato bedeutet nahtloses Spielen.
Das wird durch das Verbinden von aufeinander folgenden Noten erreicht - heben Sie nicht die Taste der ersten Note, bis die zweite gespielt wird.
Fraser empfiehlt ein weitgehendes Überlappen der beiden Noten.
Die ersten Momente einer Note enthalten viel \enquote{Rauschen}, sodass überlappende Noten nicht so sehr auffallen.
Das Legato ist eine Gewohnheit, die Sie in Ihr Spielen aufnehmen müssen.
Experimentieren Sie deshalb mit verschiedenen Graden des Überlappens, um zu sehen, welches Ausmaß \textit{bei Ihnen} das beste Legato erzeugt.
Üben Sie dieses dann solange, bis es zur Gewohnheit wird, sodass Sie stets denselben Effekt reproduzieren können.
Chopin hielt das Legato für die wichtigste Fertigkeit, die ein Anfänger entwickeln muss.
Chopins Musik erfordert spezielle Arten des Legatos und Staccatos (Ballade Op. 23), weshalb es wichtig ist, auf diese Elemente zu achten, wenn man seine Musik spielt.
\textbf{Wenn Sie das Legato-Spielen üben möchten, spielen Sie etwas von Chopin.}
Der \hyperref[c1iii1a1]{Basisanschlag} ist eine Voraussetzung für das Legato.

\textbf{Beim Staccato prallt der Finger von den Tasten zurück, um so einen kurzen Ton ohne Nachklang zu erzeugen.}
Es ist irgendwie erstaunlich, dass die meisten Bücher über das Klavierlernen das Staccato behandeln aber nie definieren was es ist!
Der Fänger hakt beim Staccato nicht ein, und der Dämpfer unterbricht den Ton sofort nachdem die Note gespielt wird.
Deshalb ist die Halte-Komponente des \hyperref[c1iii1a1]{Basisanschlags} nicht vorhanden.
Es gibt zwei Notationen für das Staccato, die normale (Punkt) und das Staccatissimo (gefülltes Dreieck).
Bei beiden wird die Stoßzunge nicht freigegeben; beim Staccatissimo bewegt sich der Finger viel schneller ab- und aufwärts.
Deshalb kann der Tastenweg beim normalen Staccato ungefähr die Hälfte nach unten sein, aber beim Staccatissimo kann er weniger als die Hälfte sein.
Auf diese Art wird der Dämpfer schneller zu den Tasten zurückgeführt, was zu einer kürzeren Notendauer führt.
Weil der Fänger nicht eingehakt ist, kann der Hammer \enquote{herumspringen}, was Wiederholungen bei bestimmten Geschwindigkeiten trickreich werden lässt.
Geben Sie sich deshalb nicht sofort selbst die Schuld, wenn Sie Probleme mit schnell wiederholten Staccatos haben - es kann die falsche Frequenz sein, bei der der Hammer in die falsche Richtung springt.
Indem Sie die Geschwindigkeit, den Tastenweg usw. ändern, können Sie das Problem eventuell eliminieren.
Beim normalen Staccato kehrt der Dämpfer wegen der Schwerkraft schnell auf die Saiten zurück.
Beim Staccatissimo springt der Dämpfer sogar von der oberen Dämpferstange zurück, sodass er noch schneller zurückkehrt.
Die Biegung des Hammerstiels kann beim Staccato negativ sein, was die effektive Masse des Hammers verringert;
deshalb gibt es eine große Vielfalt an Tönen, die man mit dem Staccato erzeugen kann.
Darum ändern sich die Bewegungen des Fängers, der Stoßzunge und des Dämpfers beim Staccato.
\textbf{Ganz klar: Um Staccatos gut zu spielen ist es hilfreich, die Funktionsweise des Klaviers zu verstehen.}

Staccato wird, abhängig davon wie es gespielt wird, generell in drei Gruppen eingeteilt:

\begin{enumerate}[label={\roman*.}] 
 \item Fingerstaccato,
 \item Handgelenksstaccato,
 \item Armstaccato, was sowohl die Auf- und Abbewegung als auch die Drehung des Arms einschließt.
\end{enumerate}

(i) wird hauptsächlich mit den Fingern gespielt, wobei die Hand und der Arm stillgehalten werden, (ii) wird hauptsächlich mit Bewegung des Handgelenks gespielt, und (iii) wird üblicherweise mit \hyperref[c1iii4SchubZug]{Schub} (siehe \ref*{c1iii4SchubZug}) gespielt, wobei die Spielbewegung aus dem Oberarm kommt.
Wenn man von (i) nach (iii) geht, steht mehr Masse hinter den Fingern; deshalb erzeugt (i) das leichteste und schnellste Staccato und ist für einzelne, leise Noten nützlich, und (iii) erzeugt das stärkste Gefühl, ist für laute Passagen und Akkorde mit vielen Noten nützlich, ist aber auch das langsamste.
(ii) liegt dazwischen.
In der Praxis kombinieren die meisten von uns wahrscheinlich alle drei; da das Handgelenk und der Arm langsamer sind, müssen ihre Amplituden entsprechend reduziert werden, um ein schnelles Staccato zu spielen.
Manche Lehrer rümpfen über das Handgelenksstaccato die Nase und bevorzugen hauptsächlich das Armstaccato; es ist jedoch wahrscheinlich besser, eine Wahl zwischen allen dreien zu haben (oder sie zu kombinieren).
So könnten Sie zum Beispiel in der Lage sein, die Ermüdung zu reduzieren, indem Sie vom einen zum anderen wechseln, obwohl die Standardmethode zum Reduzieren der Ermüdung das Wechseln der Finger ist.
Wenn Sie das Staccato üben, dann üben Sie zunächst alle drei Formen (Finger, Hand, Arm), bevor Sie entscheiden, welches Sie benutzen oder wie Sie sie kombinieren.

Da man das Armgewicht nicht für das Staccato benutzen kann, ist Ihr ruhiger Körper der beste Bezugspunkt.
Deshalb spielt der Körper beim Staccato-Spielen eine Hauptrolle.
Die Geschwindigkeit der Staccato-Wiederholung wird durch das Maß der Auf- und Abwärtsbewegung kontrolliert: je kleiner die Bewegung, desto größer die Wiederholrate, genau wie beim Dribbeln eines Basketballs.
  

% zuletzt geändert 21.03.2010

