% File: c1ii15

\subsection{Automatische Verbesserung nach dem Üben (PPI)}
\label{c1ii15}

\textbf{Während einer Sitzung kann man nur ein bestimmtes Maß an Verbesserung erwarten}, weil es hauptsächlich zwei Arten gibt, sich zu verbessern.
Die erste ist die offensichtliche Verbesserung, die vom Lernen der Noten und Bewegungen kommt und in sofortiger Verbesserung resultiert.
Das tritt bei Passagen auf, für die Sie bereits die Technik zum Spielen haben.
\textbf{Die zweite wird \enquote{Automatische Verbesserung nach dem Üben (PPI)} genannt\footnote{PPI = post practice improvement} und resultiert aus physiologischen Veränderungen beim Erwerben einer neuen Technik.}
Dies ist ein langsamer Veränderungsprozess, der über Wochen oder Monaten abläuft, weil er das Wachstum von Nerven- und Muskelzellen erfordert.

Deshalb sollten Sie beim Üben versuchen, Ihren Fortschritt zu bewerten, sodass Sie aufhören und mit etwas anderem weitermachen können, sobald der Punkt abnehmender Ertragszuwächse erreicht ist, also üblicherweise nach weniger als zehn Minuten.
\textbf{Wie von Zauberhand wird sich Ihre Technik nach einer guten Übung für mindestens einige Tage von selbst weiter verbessern.}
\textit{Wenn Sie alles richtig gemacht haben}, sollten Sie deshalb am nächsten Tag feststellen, dass Sie besser spielen können.
Wenn das nur an einem Tag geschieht, ist der Effekt nicht so groß. Wenn das jedoch über Monate oder Jahre geschieht, kann der kumulative Effekt enorm sein.

Es ist normalerweise profitabler, verschiedene Dinge während einer Sitzung zu üben und sie simultan verbessern zu lassen (während Sie nicht üben!), als zu hart an einer Sache zu arbeiten.
Zu viel zu üben kann sogar Ihrer Technik schaden, wenn es zu Stress, schlechten Angewohnheiten oder Verletzungen führt.
Sie müssen eine bestimmte Minimalanzahl üben, ungefähr einhundert Wiederholungen, damit die PPI eintritt.
Da wir aber über ein paar Takte reden, die mit hoher Geschwindigkeit gespielt werden, sollte das Üben von dutzenden oder hunderten Malen nur ein paar Minuten benötigen.
Seien Sie deshalb unbesorgt, wenn Sie hart üben aber keine große sofortige Verbesserung sehen.
Das könnte für diese bestimmte Passage normal sein.
Wenn Sie nichts finden, das Sie falsch machen, ist es Zeit aufzuhören und die Sache der PPI zu überlassen, nachdem Sie eine für die PPI genügende Anzahl von Wiederholungen ausgeführt haben.
Achten Sie auch darauf, dass Sie entspannt üben, weil Sie keine PPI einer stressbeladenen Bewegung möchten.

In Abhängigkeit davon, was Sie aufhält, gibt es verschiedene Typen von PPI.
Eine der Arten, in denen sich diese Typen offenbaren, ist die Zeitspanne, während der die PPI wirkt.
Sie variiert von einem Tag bis zu vielen Monaten.
Die kürzesten Zeiten können mit der Konditionierung verbunden sein, wie dem Gebrauch von Bewegungen oder Muskeln, die Sie vorher nicht benutzt haben oder Gedächtnisfragen.
Mittlere Zeiten von mehreren Wochen können mit dem Bilden von Nervenverbindungen, wie für das beidhändige Spielen, verbunden sein.
Längere Zeiten können mit dem tatsächlichen Wachstum von Hirn-, Nerven- oder Muskelzellen verbunden sein, sowie der Umwandlung von langsamen in schnelle Muskelzelltypen (siehe \hyperref[c1iii7aMuskeln]{Abschnitt III.7a}).

Sie müssen alles richtig machen, um die PPI zu maximieren.
Viele Schüler kennen die Regeln nicht und können die PPI \textit{umkehren}, mit dem Ergebnis, dass sie am nächsten Tag \textit{schlechter} spielen.
Die meisten dieser Fehler haben ihren Ursprung im falschen Gebrauch des schnellen und langsamen Übens; deshalb werden wir die Regeln für die richtige Wahl der Übungsgeschwindigkeiten in den folgenden Abschnitten behandeln.
Jeder Stress oder unnötige Bewegung während des Übens wird ebenfalls der PPI unterzogen und kann sich in eine schlechte Angewohnheit verwandeln.
Der am weitesten verbreitete Fehler, den Schüler begehen, wenn sie die PPI umkehren, ist, unmittelbar bevor sie mit dem Üben aufhören schnell zu spielen.
Das Letzte, was Sie vor dem Aufhören tun, sollte das korrekteste und beste Beispiel dessen sein, was Sie erreichen wollen; dies geht mit einer moderaten bis langsamen Geschwindigkeit am besten.
\textbf{Der jeweils letzte Durchlauf hat anscheinend einen außerordentlich starken PPI-Effekt.}
Die Methoden dieses Buchs sind ideal für die PPI, hauptsächlich weil sie es betonen, nur die Abschnitte zu üben, die man nicht spielen kann.
Wenn man langsam beidhändig spielt und die Geschwindigkeit für einen großen Abschnitt langsam steigert, wird die PPI ungenügend konditioniert, weil man nicht genügend Zeit hat, die erforderliche Zahl der Wiederholungen auszuführen.
Außerdem gerät der PPI-Prozess durcheinander, weil man eine große Menge an leichtem Material mit dem kleinen Anteil an schwierigem Material vermischt und die Geschwindigkeit, Bewegungen usw. ebenfalls nicht korrekt sind.

PPI ist nichts Neues; sehen wir uns drei bekannte Beispiele an: den Bodybuilder, den Marathonläufer und den Golfer.
Während der Bodybuilder Gewichte stemmt, wachsen seine Muskeln nicht; er verliert sogar Gewicht.
Aber während der folgenden Wochen reagiert der Körper auf die Stimulanz und baut die Muskeln auf.
Das ganze Muskelwachstum erfolgt \textit{nach} dem Üben.
So misst der Bodybuilder nach dem Üben nicht, wie viel Muskeln er gewonnen hat oder wie viel Gewicht er mehr heben kann, sondern er konzentriert sich darauf, dass die Übung die erforderliche Konditionierung hervorruft.
Der Unterschied ist hier, dass wir für das Klavierspielen Koordination und Ausdauer anstelle von starken und großen Muskeln entwickeln.
Der Bodybuilder möchte die langsamen Muskeln wachsen lassen, während der Klavierspieler die langsamen Muskeln in schnelle umwandeln möchte.
Ein weiteres Beispiel ist der Marathonläufer.
Wenn man noch nie im Leben einen Kilometer gelaufen ist und es das erste Mal versucht, ist man vielleicht in der Lage, einen halben Kilometer zu laufen, bevor man langsamer werden muss, um eine Pause zu machen.
Wenn man nach einer Pause versucht, wieder weiterzulaufen, wird man immer noch nach einem halben Kilometer oder weniger müde.
So ergibt der erste Lauf keine erkennbare Verbesserung.
Wenn man jedoch einen Tag wartet und es wieder versucht, wird man vielleicht in der Lage sein, einen Kilometer zu laufen, bevor man ermüdet - man hat gerade die PPI kennengelernt.
Auf diese Art konditionieren sich Marathonläufer, sodass sie schließlich 42 Kilometer laufen können.
Golfer sind mit dem Phänomen vertraut, dass sie den Ball an einem Tag gut treffen aber schlecht am nächsten, weil sie eine schlechte Angewohnheit angenommen haben.
So führt zu oft mit dem Driver (dem schwierigsten Schläger) zu schlagen dazu, dass man das Schwingen ruiniert, während das Üben mit dem \#5-Holz (ein viel einfacherer Schläger) es wieder herstellen kann; deshalb ist es wichtig, mit einem einfacheren Schläger zu üben, bevor man aufhört.
Die Analogie beim Klavierspielen ist, dass es oft die PPI zunichte macht, wenn man schnell mit voller Wucht spielt, während das Üben einfacher Abschnitte (kurzer Abschnitte einhändig) dazu führt, sie zu verbessern.

\textbf{Die PPI geschieht hauptsächlich während des Schlafs.}
Sie können Ihr Auto nicht reparieren, solange Sie auf der Autobahn fahren; genauso kann der größte Teil des Wachstums und der Reparatur des Körpers nicht während der wachen Zeit geschehen.
Der Schlaf ist nicht nur zum Ausruhen da, sondern auch für das Wachstums und die Pflege des Körpers.
Dieser Schlaf muss der normale Nachtschlaf einschließlich aller Hauptbestandteile sein, insbesondere des REM-Schlafs\footnote{REM = Rapid Eye Movement; schnelle Augenbewegungen}.
Babys brauchen so viel Schlaf, weil sie schnell wachsen.
Sie werden vielleicht keine gute PPI erreichen, wenn Sie nachts nicht gut schlafen.
Am besten wird sein, wenn Sie abends für die Konditionierung üben und morgens die PPI überprüfen.
Die PPI wird durch den Zelltod angestoßen; hartes Üben ruft einen vorzeitigen Zelltod hervor, und der Körper überkompensiert es, wenn übermäßig viele Zellen sterben.
Man könnte meinen, dass 100 Wiederholungen keinen Zelltod verursachen können, aber es werden ständig Zellen  erneuert, und jede zusätzliche Arbeit beschleunigt diese Erneuerungsrate.


\subsection{Gefahren des langsamen Spielens - Fallstricke der \enquote{Intuitiven Methode}}
\label{c1ii16}

\textbf{Warum ist wiederholtes langsames Spielen (intuitive Methode) schädlich, wenn man ein neues Stück beginnt?}
Wenn man beginnt, gibt es keine Möglichkeit zu wissen, ob die Bewegung, die man für das langsame Spielen benutzt, richtig oder falsch ist.
Die Wahrscheinlichkeit falsch zu spielen liegt nahe 100\%, weil es fast unendlich viele Möglichkeiten gibt, falsch zu spielen, aber nur eine beste Art.
Wenn diese falsche Bewegung beschleunigt wird, dann wird der Schüler auf eine Geschwindigkeitsbarriere treffen.
Angenommen, dieser Schüler hat die Geschwindigkeitsbarriere erfolgreich überwunden, indem er neue Arten zu spielen gefunden hat, so musste er jeweils die alte Art zu spielen vergessen, die neue Art erneut lernen und diese Zyklen für jede einzelne Geschwindigkeitssteigerung wiederholen, bis er die endgültige Geschwindigkeit erreicht hat.
So kann die Methode, die Geschwindigkeit langsam zu steigern, viel Zeit verschwenden.

Sehen wir uns ein Beispiel dafür an, wie unterschiedliche Geschwindigkeiten verschiedene Bewegungen erfordern.
Denken Sie an die Gangarten des Pferdes.
Wenn die Geschwindigkeit gesteigert wird, geht die Gangart vom Gehen über den Trott und Kanter (leichter Galopp) zum Galopp.
Jede dieser vier Gangarten hat normalerweise mindestens eine langsame und eine schnelle Art.
Auch unterscheidet sich eine Linksdrehung von einer Rechtsdrehung (der führende Huf ist unterschiedlich).
Das macht ein Minimum von 16 Bewegungen.
Das sind die sogenannten natürlichen Gangarten; die meisten Pferde haben sie automatisch; man kann ihnen drei weitere Gangarten beibringen: Schritt, Foxtrott und Rack, bei denen es ebenfalls langsam, schnell, links und rechts gibt.
All das mit nur vier Beinen von relativ einfacher Struktur und einem vergleichsweise eingeschränkten Gehirn.
Wir haben zehn komplexe Finger, vielseitigere Schultern, Arme und Hände und ein viel fähigeres Gehirn!
Unsere Hände sind deshalb fähig, viel mehr \enquote{Gangarten} auszuführen als ein Pferd.
Die meisten Schüler haben eine geringe Vorstellung davon, wie viele Bewegungen möglich sind, wenn der Lehrer sie nicht auf diese hinweist.
Zwei Schüler, die sich selbst überlassen werden und die man bittet, dasselbe Stück zu spielen, werden garantiert bei verschiedenen Handbewegungen landen.
Das ist ein weiterer Grund, warum es so wichtig ist, Stunden bei einem guten Lehrer zu nehmen, wenn man mit dem Klavier anfängt; solch ein Lehrer kann schnell die schlechten Bewegungen aussieben.

Ein langsames Klavierspielen schrittweise zu steigern ist so, als ob man ein Pferd dazu zwingen wollte, so schnell wie im Galopp zu rennen, indem man bloß das Gehen beschleunigt - es geht einfach nicht, denn wenn die Geschwindigkeit steigt, dann ändert sich der Impuls der Beine, des Körpers usw., was die verschiedenen Gangarten notwendig macht.
Deshalb müsste der Schüler, wenn er die Geschwindigkeit schrittweise steigert und die Musik einen \enquote{Galopp} erfordert, all die dazwischenliegenden \enquote{Gangarten} lernen.
Ein Pferd dazu zu bringen, so schnell wie im Galopp zu gehen, würde Geschwindigkeitsbarrieren aufbauen, Stress erzeugen und Verletzungen verursachen.

Ein verbreiteter Fehler beim langsamen Spielen ist die Angewohnheit, die Hand zu stützen oder zu heben.
Beim langsamen Spielen kann die Hand während der Zeit zwischen den Noten, wenn das Abwärtsdrücken nicht notwendig ist, angehoben werden.
Wenn man schneller wird, fällt diese Angewohnheit des \enquote{Hebens} mit dem nächsten Anschlag zusammen; diese Handlungen heben sich auf und resultieren in einer verpassten Note.
Ein anderer häufiger Fehler ist das Wedeln mit den freien Fingern - während er mit Finger 1 und 2 spielt, wedelt der Schüler eventuell mit den Fingern 4 und 5 mehrere Male durch die Luft.
Das stellt keine Schwierigkeit dar, bis die Bewegung so beschleunigt wird, dass keine Zeit bleibt, mit den Fingern zu wedeln.
In dieser Situation hören die freien Finger bei höheren Geschwindigkeiten nicht automatisch mit dem Wedeln auf, weil die Bewegung durch hunderte oder tausende Wiederholungen eingefahren wurde.
Stattdessen müssen die Finger mehrere Male mit Geschwindigkeiten wedeln, die sie nicht erreichen können - das erzeugt die Geschwindigkeitsbarriere.
Die Schwierigkeit ist hier, dass die meisten Schüler, die langsames Üben benutzen, sich dieser schlechten Angewohnheiten nicht bewusst sind.
\textbf{Wenn Sie wissen, wie man schnell spielen muss, ist es sicher, langsam zu spielen, aber wenn Sie nicht wissen, wie man schnell spielen muss, müssen Sie aufpassen, dass Sie nicht die falschen Angewohnheiten für langsames Spielen lernen oder enorm viel Zeit verschwenden.}
Langsames Spielen kann große Zeiträume verschwenden, weil jeder Durchgang so lange dauert.
Wenn Sie schneller werden, müssen Sie den abwärts gerichteten Druck erhöhen, weil Sie innerhalb derselben Zeitspanne mehr Tasten drücken.
\textbf{So funktioniert das \enquote{Fühlen der Schwerkraft} die meiste Zeit nicht, weil Sie beim Spielen mit unterschiedlich starker Kraft nach unten drücken müssen.}

Ein weiteres Problem im Zusammenhang mit dem intuitiven langsamen Üben sind unnötige Körperbewegungen.
Diese Bewegungen erzeugen bei höheren Geschwindigkeiten weitere Schwierigkeiten.
\textbf{Wenn sie nicht ihr Spielen \hyperref[c1iii13]{auf Video aufnehmen} und sorgfältig nach merkwürdigen Körperbewegungen Ausschau halten, sind den meisten Klavierspielern nicht alle Bewegungen, die sie machen, bewusst. 
Das kann unvorhersehbare Fehler zu nicht vorhersehbaren Zeiten verursachen, was zu psychologischen Problemen mit Unsicherheit und \hyperref[c1iii15]{Nervosität} führt.}
Ein Bewusstsein für die Körperbewegungen zu entwickeln, kann dieses Problem eliminieren.
Wir sehen, dass die Intuition zu einer Vielzahl von Schwierigkeiten führen kann; statt der Intuition benötigen wir ein auf Wissen basierendes System.


\label{c1ii17}

% zuletzt geändert 04.10.2009

\subsection{Die Wichtigkeit des langsamen Spielens}

Nachdem wir die Gefahren des langsamen Spielens herausgestellt haben, besprechen wir nun, warum langsames Spielen \textit{unentbehrlich} ist.
\textbf{Beenden Sie eine Übungssitzung immer damit, dass Sie mindestens einmal langsam spielen.
Das ist die wichtigste Regel für eine gute \hyperref[c1ii15]{PPI}}.
Sie sollten sich auch angewöhnen, das zu tun, wenn Sie beim \hyperref[c1ii7]{Üben mit getrennten Händen} die Hände wechseln; spielen Sie vor dem Wechseln mindestens einmal langsam.
\textbf{Das ist vielleicht eine der wichtigsten Regeln dieses Kapitels, weil sie einen solch ungeheuer großen Effekt auf die Verbesserung der Technik hat.}
Sie ist sowohl für die sofortige Verbesserung als auch für die PPI von Nutzen.
Ein Grund, warum es funktioniert, ist eventuell, dass man vollständig entspannen kann (siehe \hyperref[c1ii14]{Abschnitt II.14}).
Ein weiterer Grund kann sein, dass man dazu neigt, sich beim schnellen Spielen mehr schlechte Angewohnheiten anzueignen als man merkt, und man kann diese Angewohnheiten mit langsamem Spiel \enquote{löschen}.
Entgegen der Intuition ist langsames Spielen ohne Fehler schwierig (bis man die Passage komplett gemeistert hat).
So ist das langsame Spielen eine gute Möglichkeit, zu überprüfen, ob Sie dieses Musikstück wirklich gelernt haben.

Der Effekt des langsamen Spielens am Ende auf die PPI ist so dramatisch, dass Sie ihn sich leicht selbst demonstrieren können.
Versuchen Sie, in einer Übungssitzung nur schnell zu spielen, und schauen Sie, was am nächsten Tag geschieht.
Spielen Sie in der nächsten Sitzung langsam, bevor Sie aufhören, und schauen Sie wieder, was am nächsten Tag geschieht.
Oder Sie üben eine Passage nur schnell und eine andere Passage (derselben Schwierigkeit) am Ende langsam und vergleichen sie am nächsten Tag miteinander.
Dieser Effekt ist kumulativ, sodass Sie, wenn Sie dieses Experiment mit diesen beiden Passsagen längere Zeit wiederholen würden, schließlich einen riesigen Unterschied darin feststellen würden, wie gut Sie diese Passagen spielen können.

Wie langsam ist langsam?
Das ist eine Ermessensfrage, die von Ihrer Fertigkeitsstufe abhängt.
Unterhalb einer bestimmten Geschwindigkeit geht der nützliche Effekt des langsamen Spielens verloren.
Es ist wichtig, beim langsamen Spielen dieselbe Bewegung wie beim schnellen Spielen beizubehalten.
Wenn Sie zu langsam spielen, kann das unmöglich sein.
Auch braucht zu langsames Spielen zu viel Zeit und verschwendet somit Zeit.
Die beste Geschwindigkeit, die Sie zuerst ausprobieren sollten, ist eine, in der Sie so genau spielen können wie Sie möchten, ungefähr 1/2 bis 3/4 der endgültigen Geschwindigkeit.
Langsames Spielen wird auch für das Auswendiglernen benötigt (siehe \hyperref[c1iii6h]{Abschnitt III.6h}).
Die optimale langsame Geschwindigkeit für das Auswendiglernen, ungefähr unterhalb 1/2 der endgültigen Geschwindigkeit, ist niedriger als die für die Konditionierung der PPI benötigte.
Wenn die Technik besser wird, kann diese langsame Geschwindigkeit schneller werden.
Einige berühmte Pianisten üben \textit{sehr langsam}!
Einige Quellen sprechen vom Üben mit einer Note pro Sekunde, was fast irrational klingt aber eventuell für das Gedächtnis und die Musikalität nützlich sein mag.

\textbf{Eine wichtige Fertigkeit, die beim langsamen Spielen geübt werden muss, ist, den Noten voraus zu denken.}
Wenn man ein neues Stück schnell übt, gibt es eine Tendenz, gedanklich hinter die Noten zurückzufallen, und das kann zur Gewohnheit werden.
Das ist schlecht, weil man so die Kontrolle verliert.
Denken Sie voraus, wenn Sie langsam spielen, und versuchen Sie dann, diesen Vorsprung zu bewahren, wenn Sie zur höheren Geschwindigkeit zurückkehren.
Durch das Vorausdenken können Sie gewöhnlich Spielfehler oder Schwierigkeiten vorher kommen sehen und haben die Zeit, entsprechende Maßnahmen zu ergreifen.



