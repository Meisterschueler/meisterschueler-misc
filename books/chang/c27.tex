% File: c27

\subsection{Kleinere Reparaturen durchführen}
\label{c2_7}

Wenn man mit dem Stimmen angefangen hat, muß man zwangsläufig kleinere Reparaturen und ein paar Wartungsarbeiten durchführen.
 

\label{c2_7a}
\subsubsection{Intonieren der Hämmer}
\label{c2_7_hamm}

\textbf{Ein verbreitetes Problem, das man bei vielen Klavieren findet, sind verdichtete Hämmer.
Ich bringe diesen Punkt zur Sprache, weil der Zustand der Hämmer für die richtige Entwicklung der Klaviertechnik und der Fertigkeiten für das Auftreten viel wichtiger ist, als vielen Menschen bewußt ist.}
Zahlreiche Stellen in diesem Buch weisen auf die Wichtigkeit des musikalischen Übens für das Erwerben der Technik hin.
Man kann aber nicht musikalisch spielen, wenn die Hämmer ihre Aufgabe nicht erfüllen können -- ein entscheidender Punkt, der sogar von vielen Stimmern übersehen wird (oftmals weil sie fürchten, daß die zusätzlichen Kosten die Kunden vergraulen würden).
Bei einem Flügel ist, daß man es für notwendig hält, den Deckel zumindest teilweise zu schließen, um leise Passagen zu spielen, ein sicheres Zeichen für verdichtete Hämmer.
Ein weiteres sicheres Zeichen ist, daß man dazu neigt, das Dämpferpedal zu Hilfe zu nehmen, um leise zu spielen.
Verdichtete Hämmer erzeugen entweder einen lauten Ton oder überhaupt keinen.
Jede Note neigt dazu, mit einem lästigen, perkussiven Schlag zu beginnen, der zu stark ist, und der Klang ist übermäßig hell.
Es sind diese perkussiven Schläge, die für das Gehör des Stimmers so schädlich sind.
Ein richtig intoniertes Klavier erlaubt die Kontrolle über den ganzen Dynamikbereich und erzeugt einen gefälligeren Klang.

Lassen Sie uns zunächst sehen, wie ein verdichteter Hammer zu so extremen Resultaten führen kann.
Wie können kleine, leichte Hämmer laute Töne erzeugen, wenn sie mit relativ geringer Kraft auf eine Saite treffen, die unter einer solch hohen Spannung steht?
Wenn man versuchen würde, die Saite herunterzudrücken oder zu zupfen, müßte man eine ziemlich große Kraft aufwenden, um nur einen kleinen Ton zu erzeugen.
Die Antwort liegt in einem unglaublichen Phänomen, das auftritt, wenn straff gespannte Saiten im rechten Winkel angeschlagen werden.
\textbf{Es stellt sich heraus, daß die vom Hammer erzeugte Kraft im Moment des Aufpralls theoretisch unendlich ist!}
Diese fast unendliche Kraft ist es, was den leichten Hammer in die Lage versetzt, praktisch jede erreichbare Spannung der Saite zu überwinden und sie zum Schwingen zu bringen.

Hier ist die Berechnung dieser Kraft.
Stellen Sie sich vor, daß der Hammer an seinem höchsten Punkt ist, nachdem er die Saite angeschlagen hat (Flügel).
Die Saite bildet zu diesem Zeitpunkt mit ihrer ursprünglichen horizontalen Position ein Dreieck (das ist nur eine idealisierte Näherung, s.u.).
Die kürzeste Seite dieses Dreiecks ist der Abstand zwischen der Agraffe und dem Aufschlagspunkt des Hammers.
Die zweitkürzeste Seite ist die vom Hammer bis zum Steg.
Die längste ist die ursprüngliche horizontale Lage der Saite, eine gerade Linie vom Steg zur Agraffe.
Wenn wir nun eine vertikale Linie vom Aufschlagspunkt des Hammers nach unten zur ursprünglichen Saitenposition ziehen, erhalten wir zwei aneinanderliegende rechtwinklige Dreiecke.
Das sind zwei extrem spitze rechtwinklige Dreiecke, die sehr kleine Winkel an der Agraffe und dem Steg haben; wir werden diese kleinen Winkel \enquote{theta} nennen.

Das einzige, das wir zu dieser Zeit kennen, ist die Kraft des Hammers, aber das ist nicht die Kraft, die die Saite bewegt, weil der Hammer die Saitenspannung überwinden muß, bevor die Saite nachgibt.
D.h. die Saite kann sich nicht aufwärts bewegen, solange sie nicht länger werden kann.
Das ist verständlich, wenn man sich die beiden oben beschriebenen rechtwinkligen Dreiecke ansieht.
Die Saite hatte, bevor der Hammer auftraf, die Länge der langen Katheten der rechtwinkligen Dreiecke, aber nach dem Auftreffen bildet die Saite die Hypotenusen, welche länger sind.
D.h., wenn die Saite absolut unelastisch wäre und die Enden der Saiten wären fest fixiert, könnte keine noch so große Hammerkraft die Saite dazu bringen, sich zu bewegen.

Es ist eine einfache Angelegenheit, mit Vektordiagrammen zu zeigen, daß die \textit{zusätzliche} Spannungskraft F (zusätzlich zu der ursprünglichen Saitenspannung), die vom Hammeraufschlag erzeugt wird, durch f = F * sin(theta) gegeben ist, wobei f die Kraft des Hammers ist.
Es ist egal, welches rechtwinklige Dreieck wir für diese Berechnung verwenden (das auf der Seite des Stegs oder das auf der Seite der Agraffe).
Deshalb ist die Saitenspannung F = f / sin(theta).
Im ersten Moment des Auftreffens ist theta = 0, und deshalb F = unendlich!
Das geschieht, weil sin(0) = 0.
Selbstverständlich kann F nur unendlich werden, wenn die Saite sich nicht strecken kann und sich nichts anderes bewegt.
In der Realität geschieht folgendes: F steigt in Richtung unendlich an, irgend etwas gibt nach (die Saite streckt sich, der Steg bewegt sich, usw.), so daß der Hammer anfängt, die Saite zu bewegen, und theta größer als Null wird, was F endlich werden läßt.

Diese Vervielfachung der Kraft erklärt, warum ein kleines Kind auf einem Klavier trotz der mehreren hundert Pfund Spannung auf den Saiten einen ziemlich lauten Ton erzeugen kann.
Es erklärt auch, warum eine normale Person eine Saite beim Klavierspielen zerbrechen kann, besonders wenn die Saite alt ist und ihre Elastizität verloren hat.
Der Mangel an Elastizität führt dazu, daß F weitaus mehr ansteigt, als wenn die Saite elastischer ist, die Saite kann sich nicht strecken, und theta bleibt nahe Null.
Diese Situation wird außerordentlich verschärft, wenn der Hammer ebenfalls verdichtet ist, so daß er eine große, flache, harte Kerbe hat, die die Saite berührt.
In diesem Fall gibt die Oberfläche des Hammers nicht nach, und die anfängliche Kraft \enquote{f} in der obigen Gleichung wird sehr groß.
Da das bei einem verdichteten Hammer alles nahe theta = 0 geschieht, wird der Vervielfachungsfaktor der Kraft ebenfalls vergrößert.
Das Resultat ist eine gebrochene Saite.

Die obige Berechnung ist eine starke Vereinfachung und nur qualitativ richtig.
In Wirklichkeit sendet ein Hammerschlag zunächst eine wandernde Welle in Richtung des Stegs, ähnlich dem was geschieht, wenn man das Ende eines Seils nimmt und es schnalzen läßt.
Um solche Wellenformen zu berechnen, muß man bestimmte wohlbekannte Differentialgleichungen lösen.
Der Computer hat die Lösung solcher Differentialgleichungen zu einer einfachen Angelegenheit werden lassen, und realistische Berechnungen dieser Wellenformen können nun routinemäßig erfolgen.
Deshalb führen die obigen Ergebnisse, obwohl sie nicht genau sind, zu einem qualitativen Verständnis dafür, was geschieht und was die wichtigen Mechanismen und kontrollierenden Faktoren sind.

Zum Beispiel zeigt die obige Berechnung, daß es nicht die Energie der Transversalschwingung der Saite ist, sondern die Zugspannung der Saite, die für den Klang des Klaviers verantwortlich ist.
Die Energie, die durch den Hammer abgegeben wird, wird im gesamten Klavier gespeichert, nicht nur in den Saiten.
Das ist weitgehend analog zu Pfeil und Bogen: Wenn die Sehne gezogen wird, dann wird die gesamte Energie im Bogen gespeichert, nicht in der Sehne.
Und die gesamte Energie wird durch die Spannung in den Saiten übertragen.
In diesem Beispiel ist der mechanische Vorteil und die oben berechnete Vervielfachung der Kraft (nahe theta = 0) leicht zu sehen.
Es ist das gleiche Prinzip, auf dem die Harfe basiert.

Warum verdichtete Hämmer höhere harmonische Obertöne erzeugen, ist am einfachsten zu verstehen, wenn man erkennt, daß das Auftreffen in kürzerer Zeit stattfindet.
Wenn es schneller geschieht, generiert die Saite als Antwort auf das schnellere Ereignis Komponenten mit höherer Frequenz.

Die obigen Abschnitte machen deutlich, daß ein verdichteter Hammer zunächst einen großen Aufschlag auf den Saiten erzeugt, während ein richtig intonierter Hammer sanfter auf die Saite trifft und somit mehr seiner Energie an die niedrigeren Frequenzen als an die harmonischen Obertöne abgibt.
Da die gleiche Menge an Energie bei einem verdichteten Hammer in einem kürzeren Zeitraum verteilt wird, kann der anfängliche Lautstärkegrad viel höher als bei einem richtig intonierten Hammer sein, besonders bei den höheren Frequenzen.
Solche kurzen Tonspitzen können das Gehör schädigen, ohne Schmerzen zu verursachen.
Verbreitete Symptome solcher Schäden sind Tinnitus (Klingeln im Ohr) und Hörverlust bei hohen Frequenzen.
Klavierstimmer, die ein Klavier mit solchen abgenutzten Hämmern stimmen müssen, tun gut daran, Ohrenstöpsel zu tragen.
Es ist klar, daß das Intonieren der Hämmer mindestens genauso wichtig ist wie das Stimmen des Klaviers, besonders weil wir über potentielle Gehörschäden sprechen.
Ein verstimmtes Klavier mit guten Hämmern schädigt das Ohr nicht.
Trotzdem lassen viele Klavierbesitzer ihr Klavier zwar stimmen, vernachlässigen aber das Intonieren.

\textbf{Die beiden wichtigsten Prozeduren beim Intonieren sind das Wiederherstellen der Form und das Nadeln.}

Wenn der verflachte Auftreffpunkt des Hammers größer als ungefähr 1 cm ist, ist es Zeit, die Form des Hammers wieder herzustellen.
Beachten Sie, daß Sie zwischen der Länge der Saitenkerbe und dem flachgedrückten Bereich unterscheiden müssen; sogar bei gut intonierten Hämmern können die Kerben mehr als 5 mm lang sein.
Bei der endgültigen Beurteilung werden sie anhand des Klangs entscheiden müssen.
Das Formen wird durch das Schleifen der \enquote{Schultern} des Hammers erreicht, so daß er seine ursprüngliche, gerundete Form am Auftreffpunkt wiedergewinnt.
Das wird üblicherweise mit 1 Zoll\footnote{ca. 2,5 cm} breiten Streifen Sandpapier ausgeführt, die mit Leim oder doppelseitigem Klebeband auf Holz- oder Metallstreifen befestigt sind.
Sie könnten mit Papier der Körnung 80 beginnen und zum Schluß Papier der Körnung 150 verwenden.
Die Schleifbewegung muß in der Ebene des Hammers ausgeführt werden; schleifen Sie niemals quer zur Ebene.
Es besteht fast nie die Notwendigkeit den Auftreffpunkt zu schleifen.
Lassen Sie deshalb ungefähr 2 mm vom Zentrum des Auftreffpunkts unberührt.

\footnote{Eine detaillierte Beschreibung findet man z.B. in der amerikanischen Ausgabe von Reblitz auf den Seiten 137 bis 140:</font></i>

\begin{itemize} 
\item \textit{Schleifen Sie nur an der schmalen umlaufenden Fläche, die durch den Auftreffpunkt (und die Saitenkerben) hindurchgeht.}
\item \textit{Führen Sie dabei das Schleifpapier immer mit einer bogenförmigen Bewegung vom Stiel zum Auftreffpunkt hin.}
\item \textit{Der Filz muß auf beiden Seiten des Auftreffpunkts symmetrisch geformt sein, damit die Spitze des Hammers beim Auftreffen auf die Saiten nicht schrittweise in die Richtung der geringeren Unterstützung hin verformt wird und sich der Auftreffpunkt verschiebt.}
\item \textit{Die Fläche darf nicht nach der Seite abgerundet werden und muß auch rechtwinklig zu den beiden großen Seitenflächen sein, damit die Saiten einer Note gleichzeitig angeschlagen werden.}
\item \textit{Es muß genügend Filz stehenbleiben, so daß die Saite beim Anschlag nicht den Filz durchschlägt und auf das Holz des Hammers trifft.
Deshalb soll der Filz der schmalen Hämmer für die hohen Töne überhaupt nicht geschliffen werden.}
 \end{itemize}
\textit{\textbf{Also alles andere als einfach und somit nur jemandem mit wirklicher handwerklicher Begabung zu empfehlen, der stets größte Sorgfalt walten läßt!}}

Nadeln ist nicht einfach, weil die richtige Stelle zum Nadeln und die richtige Tiefe vom jeweiligen Hammer bzw. Hammerhersteller abhängen und davon, wie der Hammer ursprünglich intoniert war.
Besonders im Diskant werden beim Intonieren der Hämmer in der Fabrik oft Härter wie Lack, usw. benutzt.
Fehler beim Nadeln sind im allgemeinen nicht rückgängig zu machen.
Tiefes Nadeln ist üblicherweise an den Schultern unmittelbar außerhalb des Auftreffpunkt erforderlich.
Sehr sorgfältiges und flaches Nadeln kann im Bereich des Auftreffpunkts nötig werden.
Der Klang des Klaviers reagiert auf das flache Nadeln am Auftreffpunkt sehr empfindlich, so daß man sehr genau wissen muß, was man tut.
Wenn er richtig genadelt ist, sollte der Hammer Ihnen erlauben, sowohl sehr leise Töne zu kontrollieren als auch laute Töne zu produzieren, die nicht schrill sind.
Sie bekommen das Gefühl der totalen klanglichen Kontrolle.
Sie können nun Ihren Flügel ganz öffnen und ohne das Dämpferpedal sehr leise spielen!
Sie können auch diese lauten, reichen, respekteinflößenden Töne erzeugen.
 

\label{c2_7b}
\subsubsection{Polieren der Piloten}
\label{c2_7_pilo} 

\textbf{<i><font size=\enquote{+1} color=\enquote{navy}>[Die Beschreibung des  Aus- und Einbaus der Mechanik und der Tastatur ist z.Zt. (14.2.2005) im Originaltext relativ knapp gehalten.
Ich erinnere deshalb an dieser Stelle noch einmal an das \enquote{\hyperref[c2_1]{Achtung: ...}\index{Achtung: ...}} am Anfang dieses Kapitels!]}}

Das Polieren der Piloten kann eine lohnende Pflegearbeit sein.
Sie müssen eventuell poliert werden, wenn Sie mehr als 10 Jahre nicht gereinigt wurden, manchmal auch früher.
Drücken Sie die Tasten langsam herunter und stellen Sie fest, ob Sie eine Reibung in der Mechanik fühlen können.
Eine reibungslose Mechanik wird sich anfühlen, als ob man mit einem geölten Finger über ein glattes Glas fährt.
Wenn Reibung vorhanden ist, fühlt es sich wie die Bewegung eines sauberen Fingers über quietschendes sauberes Glas an.
Um an die Piloten zu kommen, muß man die Mechanik von den Tasten abheben, indem man bei einem Flügel die Schrauben löst, die die Mechanik unten halten.
Bei \enquote{\hyperref[upright]{Aufrechten}\index{Aufrechten}} muß man im allgemeinen die Knöpfe losschrauben, die die Mechanik an ihrem Ort halten; stellen Sie sicher, daß die Pedalstangen usw. losgelöst sind.

Wenn die Mechanik entfernt wurde, können die Tasten herausgehoben werden, nachdem man die Tastendeckleiste entfernt hat.
Stellen Sie zuerst sicher, daß alle Tasten numeriert sind, so daß Sie sie wieder in der richtigen Reihenfolge einsetzen können.
Das ist ein guter Zeitpunkt, um alle Tasten zu entfernen und alle vorher unzugänglichen Bereiche sowie die Seiten der Tasten zu reinigen.
Sie können ein mildes Reinigungsmittel wie ein mit Xxxxxx befeuchtetes Tuch für das Reinigen der Seiten der Tasten benutzen.

Stellen Sie fest, ob die oberen, kugelförmigen Kontaktflächen der Piloten stumpf sind.
Wenn sie keine glänzende Politur haben, sind sie stumpf.
Benutzen Sie eine gute Messing-, Bronze- bzw. Kupferpolitur (wie z.B. Xxxxx), um die Kontaktflächen zu polieren und blank zu putzen.
Bauen Sie alles wieder zusammen, und die Mechanik sollte nun viel leichtgängiger sein.
 




