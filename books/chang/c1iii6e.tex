% File: c1iii6e

\subsubsection{Wie fängt man an?}
\label{c1iii6e}

\textbf{Es steht außer Frage, daß es der einzig wirklich effektive Weg zum Auswendiglernen ist, die Musiktheorie zu kennen und beim Auswendiglernen eine detaillierte musikalische Analyse und ein tiefes Verständnis der Musik zu benutzen.}
Mit dieser Art von Gedächtnis werden Sie in der Lage sein, den ganzen Notensatz aus dem Gedächtnis aufzuschreiben, d.h. Sie sollten das Stück in Gedanken - ohne Klavier - spielen können.
Die meisten Schüler haben jedoch keine derart fortgeschrittene Ausbildung.
Deshalb beschreiben wir hier einige allgemeine Verfahren für das Auswendiglernen, die nicht von einer umfangreichen Ausbildung in Musiktheorie abhängig sind, und mit denen wir trotzdem das \enquote{mentale Spielen} lernen können.

Lassen Sie mich die Wichtigkeit verdeutlichen, das zu verstehen, was man auswendig lernen möchte.
Wenn wir 50 in einer ausgewählten Reihenfolge angeordnete Buchstaben des Alphabets auswendig lernen sollten, würden die meisten von uns nach einer Weile aufgeben.
Wenn es uns gelingen würde, dann könnten sich die meisten nach 20 Jahren nicht mehr daran erinnern.
Aber überraschenderweise tun wir das alle unser ganzes Leben lang!
Die meisten von uns kennen das erste der Zehn Gebote (oder längere Sätze) - wir haben tatsächlich 60 Buchstaben in der richtigen Reihenfolge gelernt (die Satzzeichen und Leerstellen nicht mitgerechnet)\footnote{wobei ich von der Version \enquote{Ich bin der Herr, Dein Gott. Du sollst keine anderen Götter haben neben mir.} ausgegangen bin}.
Und wir werden sie wahrscheinlich für den Rest unseres Lebens nicht mehr vergessen.
Sie mögen nun sagen: \enquote{Ja, aber die Buchstaben in den Zehn Geboten sind nicht in einer zufälligen Reihenfolge angeordnet!}
Das sind die Noten in einem Musikstück aber auch nicht.
Deshalb ist es für jeden von uns einfach, große Mengen Material auswendig zu lernen, wenn wir lernen können, dieses Material mit Dingen zu assoziieren, die wir verstehen oder mögen, wie z.B. Musik.
Das Publikum ist oft völlig erstaunt darüber, welche unglaublichen Mengen an Musik die Musiker auswendig lernen können, da ihnen die Leistungsfähigkeit des assoziativen Gedächtnisprozesses nicht bewußt ist.
\textbf{Somit kann das Wissen über die Musik und die Musiktheorie einen großen Unterschied darin ausmachen, wie schnell und wie gut jemand auswendig lernen kann.}
Die Musiktheorie ist nützlich aber nicht notwendig, da die Musik uns auf ihre eigene Weise ansprechen kann - jeder, der Musik mag, \enquote{spricht bereits die Sprache der Musik}.

Beginnen Sie mit dem Auswendiglernen, indem Sie einfach die Anweisungen in den \hyperref[c1i1]{Abschnitten I} und \hyperref[c1ii1]{II} befolgen und dabei jeden Abschnitt auswendig lernen, bevor Sie anfangen ihn zu üben.
\textbf{Der beste Test Ihres Gedächtnisses ist, diesen Abschnitt in Gedanken - ohne das Klavier - zu spielen.}
Wir werden dieses entscheidend wichtige Konzept das ganze Buch hindurch wiederholt aufgreifen.
Vergewissern Sie sich deshalb nach jedem Schritt des Auswendiglernens, daß sie ihn in Gedanken spielen können.
Ein gutes Gedächtnis können Sie nur erreichen, wenn Sie von Anfang an auswendig lernen; ebenso werden Sie nicht in der Lage sein, das Stück in Gedanken zu spielen, wenn Sie nicht beim ersten Auswendiglernen damit beginnen.

Wie gut Sie ein Stück verstehen und sich daran erinnern, hängt von der Geschwindigkeit ab.
Wenn man schneller spielt, tendiert man dazu, sich auf höheren Abstraktionsstufen an die Musik zu erinnern.
Beim sehr langsamen Spielen muß man sich Note für Note daran erinnern; in unserem Beispiel mit den Zehn Geboten muß man sich bei \enquote{niedriger Geschwindigkeit} Buchstaben für Buchstaben daran erinnern.
Bei höheren Geschwindigkeiten werden Sie in musikalischen Phrasen denken (Worten bei den Zehn Geboten).
Bei noch höheren Geschwindigkeiten denken Sie vielleicht in Beziehungen zwischen Phrasen oder ganzen musikalischen Konzepten (allmächtiger Gott und falsche Götter).
Diese Konzepte einer höheren Ebene kann man sich stets leichter merken.
Deshalb werden Sie, während Sie die Geschwindigkeit ändern, sehr unterschiedliche Zustände des Gedächtnisses durchlaufen und völlig neue Assoziationen erzeugen.

Während des HS-Übens kann man zu höheren Geschwindigkeiten gelangen als mit HT, was den Verstand dazu zwingt, die Musik in einem anderen Licht zu betrachten.
Die Musik aus vielen Blickwinkeln auswendig zu lernen ist notwendig, um gut auswendig zu lernen; deshalb hilft das Üben mit verschiedenen Geschwindigkeiten dem Gedächtnis enorm.
Es ist im allgemeinen einfacher, schnell auswendig zu lernen als langsam, da auf höheren Abstraktionsstufen weniger Konzepte bzw. Assoziationen notwendig sind.
Deshalb sollten Sie, wenn Sie ein neues Stück anfangen und es nur langsam spielen können, nicht befürchten, daß Sie Schwierigkeiten damit haben werden, es auswendig zu lernen.
Wenn Sie schneller werden, wird es einfacher, es auswendig zu lernen.
Das erklärt, warum das schnelle Hochschrauben der Geschwindigkeit mittels HS-Üben der schnellste Weg zum Auswendiglernen ist.
Viele Schüler werden instinktiv langsamer, wenn sie beim Auswendiglernen auf Schwierigkeiten stoßen; in Wahrheit ist es einfacher, auswendig zu lernen, wenn man schneller wird.
Sie müssen sich nur daran erinnern, daß Sie auch die Stufe des assoziativen Gedächtnisses ändern müssen, wenn Sie schneller werden.
Schüler, die niemals die Stufe des assoziativen Gedächtnisses bewußt geändert haben, werden es zunächst vielleicht schwierig finden, aber das ist etwas, das alle Klavierspieler lernen müssen.

\textbf{Sogar wenn Sie einen bestimmten Abschnitt leicht HT spielen können, sollten Sie ihn HS auswendig lernen}, da wir dieses später brauchen.
Das ist einer der wenigen Fälle, in denen die Prozeduren für das Auswendiglernen und das Lernen voneinander abweichen.
Wenn Sie einen Abschnitt leicht HT spielen können, müssen Sie ihn für die Technik nicht HS üben.
Wenn Sie das Stück aufführen möchten, müssen Sie es sich jedoch HS einprägen, weil Sie das für das Weiterspielen nach einem Hänger, für die Pflege usw. brauchen werden.
Diese Regel ist z.B. auf viele Stücke von Bach und Mozart anwendbar, welche oft technisch einfach aber schwer auswendig zu lernen sind.
Kompositionen dieser Komponisten sind gelegentlich schwieriger mit HS auswendig zu lernen, weil die Noten häufig keinen Sinn ergeben, wenn die Hände voneinander getrennt sind.
Genau deshalb ist das HS-Gedächtnis notwendig - es zeigt, wie tückisch die Musik sein kann, wenn man das Stück nicht vorher mit HS durchgearbeitet hat.
Wenn Sie das Gedächtnis prüfen (z.B. indem Sie versuchen, irgendwo in der Mitte mit dem Spielen anzufangen), werden Sie oft feststellen, daß Sie es nicht können, solange Sie das Stück nicht HS auswendig gelernt haben.
Wir beschreiben weiter unten, wie man die Musik in Gedanken, ohne Klavier, als Teil des Prozesses auswendig zu lernen \enquote{spielt}; das ist ebenfalls mit HS viel einfacher als mit HT, weil der Geist sich nicht auf zwei Dinge gleichzeitig konzentrieren kann.

\textbf{Das Gedächtnis ist ein assoziativer Vorgang; deshalb gibt es nichts hilfreicheres als Ihren eigenen Einfallsreichtum beim Erzeugen von Assoziationen.}
Bis hierhin haben wir gesehen, daß HS, HT und das Spielen mit unterschiedlichen Geschwindigkeiten Elemente sind, die Sie in diesem assoziativen Vorgang kombinieren können.
Jedes Musikstück, das Sie auswendig lernen, wird Ihnen zukünftig dabei helfen, Musikstücke auswendig zu lernen.
Die Funktion des Gedächtnisses ist außerordentlich komplex; seine komplexe Natur ist der Grund, warum intelligente Leute oft auch gute Auswendiglernende sind.
Ihnen fallen schnell nützliche Assoziationen ein.
\textbf{Durch das Auswendiglernen mit HS fügen Sie zwei weitere assoziative Verfahren (RH und LH) mit einem viel einfacheren Aufbau als HT hinzu.}
Haben Sie erst einmal eine Seite oder mehr auswendig gelernt, teilen Sie diese in logische kleinere musikalische Phrasen von ungefähr 10 Takten auf und beginnen Sie, diese Phrasen in zufälliger Reihenfolge zu spielen; d.h. üben Sie die Kunst, mit dem Spielen an einer beliebigen Stelle im Stück anzufangen.
Wenn Sie die Methoden dieses Buchs benutzt haben, um dieses Stück zu lernen, dann sollte es leicht sein, irgendwo anzufangen, weil Sie es in kleinen Abschnitten gelernt haben.
\textbf{Es ist wirklich ein tolles Gefühl, in der Lage zu sein, ein Stück ab einer beliebigen Stelle zu spielen, und diese Fertigkeit hört nie auf, das Publikum zu verblüffen.}
Ein weiterer nützlicher Trick beim Auswendiglernen ist, mit einer Hand zu spielen und sich gleichzeitig die andere Hand in Gedanken vorzustellen.
Wenn Sie das können, dann haben Sie das Stück sehr gut auswendig gelernt!
Es gibt aber noch mehr.
Diese Übungsmethode gestattet Ihnen nur, mit dem Anfang der Abschnitte, die Sie geübt haben, zu beginnen - wir werden im folgenden das mentale Spielen benutzen, um an einer beliebigen Stelle einer Phrase mit dem Spielen beginnen zu können.

Wenn man etwas auswendig lernt, wird es zunächst im temporären oder Kurzzeitgedächtnis gespeichert.
Es dauert ungefähr 2 bis 5 Minuten, bis diese Erinnerungen in das Langzeitgedächtnis übertragen werden (wenn sie überhaupt übertragen werden).
Das wurde unzählige Male durch Tests mit Traumaopfern bestätigt: sie können sich nur an das erinnern, was mindestens 2 bis 5 Minuten vor dem traumatischen Ereignis geschah.
Nach der Übertragung in das Langzeitgedächtnis verringert sich die Fähigkeit, die Erinnerung abzurufen, schrittweise, es sei denn, es erfolgt eine Wiederauffrischung.
Wenn man eine Passage viele Male innerhalb einer Minute wiederholt, erwirbt man Hand-Gedächtnis und Technik, aber das ganze Gedächtnis wird nicht proportional zur Anzahl der Wiederholungen aufgefrischt.
\textbf{Für das Auswendiglernen ist es besser, 2 bis 5 Minuten zu warten und dann erneut auswendig zu lernen.
Das ist ein Grund, warum man während einer Übungseinheit mehrere Dinge auf einmal auswendig lernen sollte.}
Konzentrieren Sie sich deshalb nicht nur lange Zeit auf eine Sache, in dem Glauben, daß mehr Wiederholungen zu einem besseren Gedächtnis führen.
 
Lernen Sie Phrasen oder Gruppen von Noten auswendig; versuchen Sie nie, sich jede Note zu merken.
Je schneller Sie spielen, desto leichter ist das Auswendiglernen, weil Sie die Phrasen und die Struktur bei höherer Geschwindigkeit leichter sehen können.
Deshalb ist das Auswendiglernen mit HS so effektiv.
Viele schlechte Auswendiglernende werden instinktiv langsamer und versuchen am Ende einzelne Noten auswendig zu lernen, wenn sie auf Schwierigkeiten stoßen.
Das ist genau das Falsche.
Schlechte Auswendiglernende können nicht deshalb nicht auswendig lernen, weil ihr Gedächtnis nicht gut wäre, sondern weil sie nicht wissen, wie man auswendig lernt.
\textbf{\textit{Ein Grund für schlechtes Auswendiglernen ist, durcheinander zu geraten.}}
Deshalb ist auswendig lernen mit HT keine gute Idee; man kann nicht so schnell spielen wie mit HS, und es gibt mehr Material, das Verwirrung stiften kann.
Gute Auswendiglernende verfügen über Methoden, ihr Material zu organisieren, so daß es nicht verwirrend ist.
Merken Sie sich die musikalischen Themen und wie diese sich entwickeln oder die Grundstruktur, die ausgebaut wird, um die endgültige Musik zu erzeugen.
Langsames Üben ist gut für das Auswendiglernen - nicht weil das Auswendiglernen beim langsamen Spielen einfacher wäre, sondern weil es ein schwieriger Test dafür ist, wie gut man auswendig gelernt hat.


\subsubsection{Auffrischung des Gedächtnisses}
\label{c1iii6f}

Eines der nützlichsten Mittel für das Gedächtnis ist das Wiederauffrischen.
\textbf{Eine vergessene Erinnerung, die wiedererlangt wird, wird stets besser erinnert.}
Viele Menschen sind darüber beunruhigt, daß sie vergessen.
Der Trick ist, aus der Not eine Tugend zu machen, d.h. wenden Sie auf sich selbst \hyperref[reversepsychology]{umgekehrte Psychologie} an!
Die meisten Menschen müssen etwas vergessen und es drei- oder viermal erneut auswendig lernen, bevor es dauerhaft erinnert wird.
Um die Frustrationen durch das Vergessen und Wiederauffrischen des Gedächtnisses zu eliminieren, versuchen Sie mit Absicht zu vergessen, z.B. indem Sie ein Stück für eine Woche oder länger nicht spielen und es dann erneut lernen.
Oder hören Sie auf, bevor Sie es komplett auswendig gelernt haben, so daß Sie das nächste Mal wieder von vorne anfangen müssen.
Oder anstatt kurze Abschnitte zu wiederholen (die Methode, die Sie anfänglich benutzt haben, um das Stück auswendig zu lernen), spielen Sie das ganze Stück, aber nur einmal am Tag oder mehrere Male am Tag aber mit mehreren Stunden dazwischen.
Finden Sie Wege um zu vergessen (wie mehrere Dinge gleichzeitig auswendig zu lernen); versuchen Sie, künstliche Hänger zu erzeugen - halten Sie mitten in einer Phrase an und versuchen Sie weiterzumachen.

\textbf{Neues Material auswendig zu lernen führt oft dazu, daß Sie vergessen, was Sie sich vorher eingeprägt haben.}
Deshalb ist es nicht effizient, viel Zeit auf das Auswendiglernen eines kleinen Abschnitts zu verwenden.
Wenn man die richtige Anzahl Dinge zum Auswendiglernen wählt, kann man das eine benutzen, um das \enquote{Vergessen} des anderen damit zu steuern, so daß man es für ein besseres Behalten erneut auswendig lernen kann.
Das ist ein Beispiel dafür, wie erfahrene Auswendiglernende ihre Abläufe zum Auswendiglernen feinabstimmen können.

Die Frustration und die Furcht zu vergessen können wie die Furcht vor dem Ertrinken behandelt werden.
Menschen, die nicht schwimmen können, fürchten sich davor, zu sinken und zu ertrinken.
Man kann diese Furcht oftmals mittels Psychologie kurieren.
Sagen Sie ihnen zunächst, sie sollen einen tiefen Atemzug machen und die Luft anhalten.
Halten Sie sie dann waagerecht mit dem Gesicht nach unten aufs Wasser, mit ihrem Gesicht und den Füßen im Wasser.
Bleiben Sie nahe bei ihnen, und halten Sie sie fest, so daß sie sich sicher fühlen (einen Schnorchel zu benutzen ist hilfreich, weil sie dann nicht den Atem anhalten müssen).
Sagen Sie ihnen dann, sie sollen untertauchen, und lassen Sie los. Sie werden erkennen, daß sie nicht tauchen können, weil der Körper dazu neigt zu treiben!
Das funktioniert in Salzwasser am besten, weil das Tauchen in einem Süßwasserbecken leichter ist.
Das Wissen, daß sie nicht sinken können, wird sie auf den langen Weg zur Abschwächung ihrer Angst zu ertrinken führen.
Genauso werden Sie beim Versuch zu vergessen entdecken, daß es gar nicht so einfach ist zu vergessen, und eigentlich glücklich sein, wenn Sie wirklich vergessen, so daß Sie den Prozeß des erneuten Lernens öfter durchlaufen können, um das Gedächtnis wieder aufzufrischen.
\textbf{Die Frustration, die durch den natürlichen Prozeß des Vergessens verursacht wird, zu eliminieren, kann Sie beruhigen und dem Auswendiglernen förderlich sein.}
Wir beschreiben nun weitere Methoden für das Wiederauffrischen und des Einspeicherns in das Gedächtnis.
 

\subsubsection{Kaltstart}
\label{c1iii6g}

\textbf{Üben Sie, auswendig gelernte Stücke \enquote{kalt} (ohne Ihre Hände aufzuwärmen) zu spielen;} das ist offensichtlich schwieriger als mit aufgewärmten Händen, aber unter ungünstigen Bedingungen zu üben ist eine Möglichkeit, Ihre Fähigkeit zum Vorspielen vor Publikum zu stärken.
Diese Fähigkeit, sich einfach hinzusetzen und kalt zu spielen, mit einem ungewohnten Klavier oder in einer ungewohnten Umgebung oder nur mehrmals am Tag, wenn Sie ein paar Minuten übrig haben, ist einer der nützlichsten Vorteile davon, Stücke auswendig zu lernen.
Und Sie können dies überall tun, außerhalb Ihres Zuhauses, wenn Ihre Noten nicht zur Verfügung stehen.
Kalt zu spielen bereitet Sie darauf vor, in einer Gruppe zu spielen usw., ohne 15 Minuten lang Hanon spielen zu müssen, bevor Sie auftreten können.
Kalt zu spielen ist eine Fähigkeit, die erstaunlich leicht zu entwickeln ist, obwohl das zunächst fast unmöglich erscheinen mag.
Das ist ein guter Zeitpunkt, die Passagen zu finden, die zu schwierig sind, um sie mit kalten Händen zu spielen, und zu üben, wie man schwierige Abschnitte verlangsamt oder vereinfacht.
Wenn Sie einen Fehler machen oder hängen bleiben, hören Sie nicht auf und gehen wieder zurück, sondern versuchen Sie, zumindest den Rhythmus oder die Melodie durchzuhalten, und spielen Sie geradewegs durch den Fehler hindurch.

\textbf{Die ersten paar Takte, sogar der einfachsten Stücke, sind oftmals kalt schwer anzufangen} und werden wahrscheinlich zusätzliche Übung erfordern, sogar wenn sie gut auswendig gelernt wurden.
Oftmals ist es bei technisch schwierigen Anfängen leichter zu beginnen.
Lassen Sie sich also von scheinbar leichter Musik nicht aufs Glatteis führen.
Es ist eindeutig wichtig, die Anfänge aller Stücke kalt zu üben.
Natürlich sollten Sie nicht immer mit dem Anfang beginnen; ein weiterer Vorteil des Auswendiglernens ist, daß Sie kleine Auszüge, die irgendwo aus dem Stück stammen, spielen können, und Sie sollten immer üben, Auszüge zu spielen (s. Abschnitt III.14 zur \enquote{\hyperref[c1iii14]{Vorbereitung auf Auftritte und Konzerte}}).
Es ist natürlich ratsam, den Anfang gut auswendig zu lernen.
Welche Tonart und welche Taktart hat das Stück?
Welches ist die erste Note, und welche absolute Tonhöhe hat sie?
 

\subsubsection{Langsam spielen}
\label{c1iii6h}

\textbf{Der allerwichtigste Weg zum Wiederauffrischen des Gedächtnisses ist langsames Spielen, \textit{sehr} langsames Spielen, mit weniger als der halben Geschwindigkeit.}
Die langsame Geschwindigkeit wird auch benutzt, um die Abhängigkeit vom \hyperref[c1iii6d]{Hand-Gedächtnis} zu reduzieren und es durch ein \enquote{\hyperref[c1iii6tastatur]{echtes Gedächtnis}} (s.u.) zu ersetzen, weil der Reiz für den Abruf des Hand-Gedächtnisses verändert und reduziert wird, wenn man langsam spielt.
Die Stimulation durch den Klavierklang ist ebenfalls wesentlich verändert.
Der größte Nachteil des langsamen Spielens ist, daß es so viel Ihrer Übungszeit einnimmt; wenn Sie doppelt so schnell spielen können, üben Sie das Stück in der gleichen Zeit zweimal so oft.
Warum also langsam spielen?
Außerdem kann es schrecklich langweilig werden.
Warum etwas üben, das man nicht braucht, wenn man mit voller Geschwindigkeit spielt?
Man muß wirklich gute Gründe haben, um das sehr langsame Üben zu rechtfertigen.
Damit sich das langsame Spielen auszahlt, versuchen Sie so viele Dinge wie möglich mit Ihrem langsamen Spielen zu kombinieren, so daß es keine Zeit verschwendet.
Einfach langsam zu spielen, ohne wohldefinierte Ziele, \textit{ist} Zeitverschwendung; Sie müssen mehrere Vorzüge gleichzeitig suchen, und dazu wissen, welche es sind. Lassen Sie uns deshalb einige davon auflisten:

\begin{enumerate}[label={\arabic*.}] 
\item Langsames Spielen ist überraschend nützlich für gute Technik, besonders für das Üben der Entspannung.
\item 
Langsames Spielen frischt Ihr Gedächtnis wieder auf, weil Zeit dafür vorhanden ist, daß die Spielsignale mehrere Male von Ihren Fingern zum Gehirn und zurück wandern, bevor nachfolgende Noten gespielt werden.
Wenn Sie nur mit der vorgegebenen Geschwindigkeit üben würden, könnten Sie das Hand-Gedächtnis wieder auffrischen und das wahre Gedächtnis verlieren.
\item Langsames Spielen gestattet es Ihnen, zu üben, der Musik, die Sie gerade spielen, in Gedanken vorauszugehen (nächster Abschnitt).
Das verleiht Ihnen mehr Kontrolle über das Stück und kann Ihnen sogar gestatten, drohende Spielfehler vorauszusehen.
Das ist der Zeitpunkt, an Ihren \hyperref[c1iii7f]{Sprüngen} und \hyperref[c1iii7e]{Akkorden} zu arbeiten (Abschnitte III.7e, f).
Seien Sie immer mindestens einen Sekundenbruchteil voraus, und üben Sie, die Noten vor dem Spielen zu fühlen, um eine hundertprozentige Genauigkeit zu garantieren.
Als generelle Regel gilt: Seien Sie ungefähr einen Takt voraus - mehr dazu später.
\item Langsames Spielen ist einer der besten Wege, um Ihre Hand von schlechten Angewohnheiten zu befreien, besonders von jenen, die Sie unbewußt während des schnellen Übens angenommen haben (\hyperref[fpd]{FPD}).
FPD ist größtenteils ein reflexartiges Hand-Gedächtnis, das das Gehirn umgeht; deshalb ist man sich der schlechten Angewohnheiten im allgemeinen nicht bewußt.
\item Sie haben nun während des Spielens Zeit, die Details der Struktur des Stückes zu analysieren und Ihre Aufmerksamkeit auf alle Ausdrucksbezeichnungen zu richten.
Konzentrieren Sie sich vor allem auf das Erzeugen der Musik.
\item Eine der Hauptursachen von Gedächtnisblockaden und Spielfehlern während des Vorspielens ist, daß das Gehirn viel schneller als gewöhnlich arbeitet, und man kann während einer Aufführung in der gleichen Zeit zwischen den Noten mehr \enquote{denken} als während des Übens.
Dieses zusätzliche Denken führt neue Variablen ein, die das Gehirn durcheinander bringen, was Sie in unbekanntes Gebiet führt und Ihren Rhythmus unterbrechen kann - das ist während einer Aufführung besonders lästig.
Üben Sie deshalb während des langsamen Übens, zwischen den Noten zusätzliche Gedanken einzufügen.
Was sind die vorangegangenen und folgenden Noten?
Sind diese genau richtig oder kann ich sie verbessern?
Was mache ich an dieser Stelle, wenn ich einen Fehler mache?
Usw., usw.
Denken Sie sich Gedanken aus, die während einer Aufführung typisch sind.
Sie können die Fähigkeit entwickeln, sich geistig von den einzelnen Noten zu lösen, die Sie gerade spielen, und in Gedanken an einer anderen Stelle durch die Musik zu wandern, während Sie einen bestimmten Abschnitt spielen.
 \end{enumerate}
Wenn Sie alle obigen Ziele kombinieren, lohnt sich die Zeit wirklich, die mit dem langsamen Spielen verbracht wird, und alle diese Ziele gleichzeitig zu verwirklichen wird eine Herausforderung sein, die keinen Raum für Langeweile läßt.
 

\subsubsection{Vorausschauend spielen}
\label{c1iii6i}

\textbf{Wenn man aus dem Gedächtnis spielt, muß man in Gedanken dem was man spielt stets voraus sein, so daß man vorausplanen, die völlige Kontrolle haben, Schwierigkeiten vorausahnen und sich veränderten Bedingungen anpassen kann.}
Man kann z.B. einen Spielfehler oft kommen sehen und einen der Tricks benutzen, die in diesem Buch besprochen werden (sehen Sie dazu in Abschnitt III.9 \hyperref[c1iii9]{wie man ein Stück auf Hochglanz bringt}), um darüber hinweg zu kommen.
Sie werden diesen Spielfehler nicht kommen sehen, sofern Sie nicht vorausdenken.
Ein Weg, das Vorausdenken zu üben, ist, schnell zu spielen und dann langsamer zu werden.
Durch das schnelle Spielen zwingen Sie Ihr Gehirn, schneller zu denken, so daß Sie, wenn Sie langsamer werden, automatisch der Musik voraus sind.
Sie können nicht vorausdenken, solange die Musik nicht gut auswendig gelernt ist; somit testet und verbessert das Vorausdenken wirklich das Gedächtnis.

Sie können auf mehreren Ebenen der Komplexität vorausdenken.
Sie können eine Note vorausdenken, wenn Sie sehr langsam spielen.
Bei höheren Geschwindigkeiten müssen Sie eventuell in Takten oder Phrasen denken.
Sie können auch in Themen, musikalischen Ideen, verschiedenen Stimmen oder Akkordübergängen denken.
Das sind verschiedene Assoziationen, die Ihnen beim Auswendiglernen helfen werden.

Der beste Weg, sehr schnell zu spielen, ist natürlich HS.
Das ist ein weiteres wertvolles Nebenprodukt des HS-Übens; Sie werden zunächst überrascht sein, welche Auswirkungen wirklich schnelles Spielen auf Ihr Gehirn hat.
Es ist eine völlig neue Erfahrung.
Da man richtig schnell werden muß, um gegen das Gehirn zu gewinnen, sind solche Geschwindigkeiten mit HT nicht leicht zu erreichen.
Ein solch schnelles Spielen ist eine gute Möglichkeit, das Gehirn so zu beschleunigen, daß es vorausdenken kann.



