% File: c1ii18

\subsection{Fingersatz}
\label{c1ii18}

Außer in Büchern für Anfänger werden die grundlegenden Fingersätze in den Notationen nicht angegeben.
Diese grundlegenden Fingersätze finden Sie in den Abschnitten über die Tonleitern (\hyperref[c1iii5d]{\autoref{c1iii5d}} und \hyperref[c1iii5h]{\autoref{c1iii5h}}) und Arpeggios (\hyperref[c1iii5e]{\autoref{c1iii5e}}).
\textbf{Beachten Sie, dass die Tonleitern die Fingersätze für praktisch alle Läufe bestimmen.
Deshalb ist es wichtig, sich die \hyperref[table]{Fingersätze aller Tonleitern} zu merken}.
Das ist nicht schwierig, weil die meisten Tonleitern einem Standardfingersatz und die Ausnahmen einfachen Regeln folgen, wie einen Daumen auf den schwarzen Tasten zu vermeiden.
Eine schwarze Taste mit dem Daumen zu spielen, bringt die Hand zu nah an die Klappe heran und macht es schwierig, die weißen Tasten mit den anderen Fingern zu spielen.
Die meisten Notenblätter geben die Fingersätze für ungewöhnliche Situationen an, in denen besondere Fingersätze notwendig sind.
Befolgen Sie diese Fingersätze, solange Sie keinen besseren haben; wenn Sie dem angegebenen Fingersatz nicht folgen, werden Sie sich vermutlich Ärger einhandeln.
Ein angegebener Fingersatz mag Ihnen zunächst unhandlich erscheinen, aber er steht dort aus guten Gründen.
Diese Gründe werden oft erst offensichtlich, wenn man zur endgültigen Geschwindigkeit kommt oder \hyperref[c1ii25]{beidhändig spielt}.
\textbf{Es ist sehr wichtig, sich einen festen Fingersatz zu suchen und ihn nicht zu ändern, solange es keinen guten Grund dafür gibt.}
Keinen festen Fingersatz zu haben, wird den Lernprozess verlangsamen und Ihnen später Ärger machen, besonders während des \hyperref[c1iii14]{Vorspielens}, wenn eine Unschlüssigkeit beim Fingersatz zu einem Fehler führen kann.
Wenn Sie den Fingersatz ändern, dann bleiben Sie immer bei dem neuen.
Vermerken Sie die Änderung auf dem Notenblatt, sodass Sie den Fingersatz während des Übens nicht versehentlich ändern; auch kann es sehr ärgerlich sein, Monate später zu dieser Musik zurückzukommen und sich nicht mehr an diesen tollen Fingersatz erinnern zu können, den man sich vorher herausgearbeitet hat.

Nicht alle in der Notation vorgeschlagenen Fingersätze sind für jeden angemessen.
Sie haben vielleicht große oder kleine Hände.
Sie haben sich vielleicht aufgrund der Art, wie sie gelernt haben, einen anderen Fingersatz angewöhnt.
Sie könnten einen anderen Satz an Fertigkeiten haben; zum Beispiel könnten Sie \hyperref[c1iii3]{Triller} besser mit 1,3 als mit 2,3 spielen.
Noten von verschiedenen Herausgebern können unterschiedliche Fingersätze haben.
Für fortgeschrittene Spieler kann der Fingersatz einen profunden Einfluss auf den zu erzielenden musikalischen Effekt haben.	
Glücklicherweise sind die in diesem Buch beschriebenen Methoden gut geeignet, um den Fingersatz schnell zu ändern.
Wenn Sie erst einmal mit den Methoden dieses Buchs vertraut sind, werden Sie in der Lage sein, den Fingersatz sehr schnell zu ändern.
Führen Sie alle diese Änderungen durch, bevor Sie mit dem beidhändigen Üben anfangen, weil die Fingersätze sehr schwer zu ändern sind, wenn sie erst einmal in das beidhändige Spiel aufgenommen sind.
Auf der anderen Seite sind einige Fingersätze zwar \hyperref[c1ii7]{einhändig} leicht, werden aber beidhändig schwierig, sodass es sich auszahlt, sie beidhändig zu überprüfen, bevor man irgendeine Änderung dauerhaft akzeptiert.

Zusammengefasst ist der Fingersatz von entscheidender Bedeutung. 
\textbf{Anfänger sollten nicht mit dem Üben beginnen, wenn sie nicht die richtigen Fingersätze kennen.}
Wenn Sie sich über den Fingersatz unsicher sind, versuchen Sie, Noten mit zahlreichen Fingersatzangaben zu finden, oder gehen Sie in ein Internetforum, und bitten Sie um Hilfe.
Wenn Sie sich die Fingersätze für Tonleitern und Arpeggios ansehen, werden Sie ein paar dem \enquote{gesunden Menschenverstand} entsprechende Regeln finden; diese sollten für den Anfang genügen.


\subsection{Akkurates Tempo und das Metronom}
\label{c1ii19}

\textbf{Beginnen Sie alle Stücke mit sorgfältigem Zählen; das gilt insbesondere für Anfänger und Jugendliche.}
Kindern sollte beigebracht werden, laut zu zählen, weil das der einzige Weg ist, herauszufinden, was \textit{ihre}  Vorstellung des Zählens ist.
Sie kann völlig von der beabsichtigten abweichen!
Man sollte die Taktbezeichnung am Anfang jeder Komposition verstehen.
Diese sieht wie ein Bruch, bestehend aus Zähler und Nenner, aus.
Der Zähler gibt die Anzahl der Schläge je Takt an und der Nenner die Note je Schlag.
Zum Beispiel bedeutet 3/4, dass jeder Takt drei Schläge hat, und dass jeder Schlag eine Viertelnote ist.
Die Taktbezeichnung zu kennen ist beim Begleiten entscheidend, weil der Moment, in dem der Begleiter beginnt, durch den Anfangsschlag bestimmt ist, den der Dirigent mit dem Taktstock anzeigt.

Ein Vorteil des \hyperref[c1ii7]{Übens mit getrennten Händen} ist, dass man dazu neigt, genauer zu zählen als beim \hyperref[c1ii25]{beidhändigen Üben}.
Schüler, die beidhändig anfangen, können am Ende unerkannte Fehler beim Zählen haben.
Interessanterweise machen es diese Fehler im Allgemeinen unmöglich, die Musik auf Geschwindigkeit zu bringen.
Es gibt etwas beim falschen Zählen, das seine eigene Geschwindigkeitsbarriere erzeugt.
Es bringt wahrscheinlich den \hyperref[c1iii1b]{Rhythmus} durcheinander.
Deshalb sollten Sie das Zählen überprüfen, wenn Sie Probleme beim Erreichen der Geschwindigkeit bekommen.
Ein Metronom ist dafür sehr nützlich.

\textbf{Benutzen Sie das Metronom, um Ihre Geschwindigkeits- und Schlaggenauigkeit zu überprüfen.}
Ich wurde wiederholt von Fehlern überrascht, die ich auf diese Art beim Prüfen entdeckt habe.
Zum Beispiel neige ich dazu, bei schwierigen Abschnitten langsamer zu werden und schneller bei leichteren, obwohl es mir so vorkommt, als wäre es genau umgekehrt, wenn ich ohne das Metronom spiele.
Die meisten Lehrer prüfen das Tempo ihrer Schüler damit.
Wenn der Schüler das richtige Timing hat, sollte das Metronom abgeschaltet werden.
Das Metronom ist einer Ihrer verlässlichsten Lehrer - wenn Sie erst einmal angefangen haben, es zu benutzen, werden Sie froh sein, es getan zu haben.
Entwickeln Sie die Angewohnheit, das Metronom zu benutzen, und Ihr Spiel wird sich ohne Zweifel verbessern.
Alle ernsthaften Schüler müssen ein Metronom haben.

Metronome sollten nicht übermäßig benutzt werden.
\textbf{Lange Übungssitzungen, bei denen das Metronom Sie begleitet, sind schädlich für das Erwerben der Technik und führen zu einer unmusikalischen Spielweise.}
Wenn das Metronom kontinuierlich länger als ungefähr zehn Minuten benutzt wird, wird Ihr Gehirn anfangen, Ihnen mentale Streiche zu spielen, sodass Sie eventuell die Genauigkeit des Timings verlieren.
Wenn das Metronom Klicks abgibt, erzeugt das Gehirn zum Beispiel nach einiger Zeit Antiklicks in Ihrem Kopf, die den Metronomklick aufheben können, sodass Sie entweder das Metronom nicht mehr hören oder es zur falschen Zeit hören.
Deshalb haben die meisten modernen elektronischen Metronome einen Modus mit pulsierender Leuchtanzeige.
Das visuelle Zeichen ist für mentale Tricks weniger anfällig und stört die Musik nicht akustisch.
Der häufigste Missbrauch des Metronoms ist, es zum Steigern der Geschwindigkeit zu benutzen; das missbraucht das Metronom, den Schüler, die Musik und die Technik.
Wenn Sie die Geschwindigkeit schrittweise steigern müssen, benutzen Sie das Metronom, um das Tempo festzulegen.
Schalten Sie es dann aus, wenn Sie mit dem Üben fortfahren.
Benutzen Sie es dann wieder kurz, wenn Sie die Geschwindigkeit erhöhen.
\textbf{Das Metronom ist dazu da, das Tempo festzulegen und Ihre Genauigkeit zu prüfen.
Es ist kein Ersatz für Ihr eigenes internes Timing.}

Der Vorgang des Schnellerwerdens ist ein Prozess des Herausfindens der geeigneten neuen Bewegungen.
Wenn Sie die richtige neue Bewegung finden, können Sie einen Quantensprung zu einer höheren Geschwindigkeit machen, bei der die Hand komfortabel spielt; in Wahrheit ist bei mittlerer Geschwindigkeit weder die langsame noch die schnelle Bewegung anwendbar, und es ist oft schwieriger zu spielen als mit der schnellen Geschwindigkeit.
Wenn Sie das Metronom zufällig auf diese mittlere Geschwindigkeit gesetzt haben, müssen Sie eventuell längere Zeit damit kämpfen und bauen eine Geschwindigkeitsbarriere auf.
Einer der Gründe, warum die neue Bewegung funktioniert, ist, dass die menschliche Hand ein mechanisches Gerät ist und Resonanzen hat, bei denen bestimmte Kombinationen von Bewegungen auf natürliche Art gut funktionieren.
Es besteht kaum Zweifel darüber, dass manche Musik für bestimmte Geschwindigkeiten komponiert wurde, weil der Komponist diese Resonanzgeschwindigkeit herausgefunden hat.
Auf der anderen Seite hat jeder Einzelne eine andere Hand mit anderen Resonanzgeschwindigkeiten, und das erklärt teilweise, warum verschiedene Pianisten verschiedene Geschwindigkeiten wählen.
Ohne das Metronom können Sie von einer Resonanz zur nächsten wechseln, weil die Hände sich bei diesen Geschwindigkeiten wohl fühlen, während die Chancen sehr gering sind, dass Sie das Metronom genau auf diese Geschwindigkeiten setzen.
Deshalb übt man mit dem Metronom fast immer mit der falschen Geschwindigkeit, solange man die Resonanzen nicht kennt (niemand kennt sie) und das Metronom entsprechend einstellt.

\textbf{Elektronische Metronome sind mechanischen in jeder Hinsicht überlegen}, es sei denn, Sie bevorzugen das Aussehen der alten Modelle.
Elektronische sind genauer, können verschiedene Töne oder Blinkzeichen erzeugen, haben eine variable Lautstärke, sind billiger, sind weniger unhandlich, haben Speicherfunktionen usw., während die mechanischen anscheinend immer im falschen Moment aufgezogen werden müssen.


\subsection{Die schwache linke Hand - Eine Hand unterrichtet die andere}
\label{c1ii20}

\textbf{Schüler, die nicht \hyperref[c1ii7]{mit getrennten Händen üben}, werden immer eine stärkere rechte als linke Hand haben}.\footnote{Das scheint gemäß der eigenen Erfahrung auch für Linkshänder zu gelten.}
Das geschieht, weil die Passagen der rechten Hand im Allgemeinen technisch schwieriger sind.
Die Passagen der linken Hand erfordern meistens mehr Kraft, die linke Hand bleibt aber hinsichtlich Geschwindigkeit und Technik oft zurück.
Deshalb bedeutet \enquote{schwächer} hier technisch schwächer, nicht in Bezug auf die Kraft.
\textbf{Die Methode mit getrennten Händen balanciert die Hände aus, weil man automatisch der schwächeren Hand mehr zu arbeiten gibt.}
Für Passagen, die eine Hand besser spielen kann als die andere, ist die bessere Hand oftmals Ihr bester Lehrer.
Um eine Hand die andere unterrichten zu lassen, wählen Sie einen kurzen Ausschnitt, und spielen Sie ihn schnell mit der besseren Hand.
Wiederholen Sie ihn dann sofort mit der schwächeren Hand und zwar um eine Oktave versetzt, um Kollisionen zu vermeiden.
Sie werden entdecken, dass die schwächere Hand oftmals \enquote{aufschließen} kann oder \enquote{eine Vorstellung davon bekommt}, wie es die bessere Hand macht.
Der Fingersatz sollte ähnlich sein, muss aber nicht identisch sein.
Wenn die schwächere Hand erst einmal \enquote{eine Vorstellung bekommt}, dann machen Sie sie schrittweise stärker, indem Sie mit der schwächeren Hand zweimal und der stärkeren Hand einmal spielen, dann dreimal gegen einmal, usw.

Diese Fähigkeit, mit einer Hand die andere zu unterrichten, ist wichtiger als vielen bewusst ist.
Das obige Beispiel, in dem ein bestimmtes technisches Problem gelöst wurde, ist nur ein Beispiel - wichtiger ist, dass dieses Konzept auf praktisch jede Übungssitzung anwendbar ist.
Der Hauptgrund für diese breite Anwendbarkeit ist, dass eine Hand \textit{immer} etwas besser spielt als die andere, zum Beispiel hinsichtlich der Entspannung, der Geschwindigkeit, den \hyperref[ruhig]{ruhigen Händen} und der unzähligen Finger- bzw. Handbewegungen (\hyperref[c1iii5a]{Daumenübersatz}, \hyperref[c1iii4b]{flache Finger} usw. - siehe folgende Abschnitte), also allem Neuen, das Sie versuchen zu lernen.
Wenn Sie das Prinzip, eine Hand zu benutzen um die andere zu unterrichten, erst einmal gelernt haben, werden Sie es deshalb immer verwenden.



