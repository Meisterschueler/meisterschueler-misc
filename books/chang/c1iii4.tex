% File: c1iii4

\subsection{Bewegungen der Hand, der Finger und des Körpers}
\label{c1iii4}

\subsubsection{Bewegungen der Hand}

Für das Erwerben der Technik sind bestimmte Handbewegungen erforderlich.
Wir haben zum Beispiel oben die \hyperref[c1ii11]{parallelen Sets}\index{parallelen Sets} besprochen aber nicht aufgeführt, welche Arten von Handbewegungen notwendig sind, um sie zu spielen.
\textbf{Es ist wichtig, von Anfang an zu betonen, dass die erforderlichen Handbewegungen extrem klein sein können, fast nicht wahrnehmbar.}
Nachdem Sie ein Experte geworden sind, können Sie sie so weit übertreiben wie Sie möchten.
Deshalb sind während des Konzerts eines berühmten Künstlers die meisten Handbewegungen nicht zu erkennen (sie geschehen meistens zu schnell, sodass das Publikum sie nicht wahrnimmt), sodass die meisten sichtbaren Bewegungen Übertreibungen oder irrelevant sind.
Deshalb kann es sein, dass zwei Künstler, einer mit scheinbar ruhigen Händen und einer mit Flair und Aplomb, in Wirklichkeit die gleichen Handbewegungen des Typs benutzen, den wir hier besprechen.
\textbf{Die hauptsächlichen Handbewegungen sind Pronation und Supination, Schub und Zug, Krallen und Schnellen, Rollung und Bewegungen des Handgelenks.
Sie sind fast immer zu komplexeren Bewegungen kombiniert.}
Beachten Sie, dass sie immer paarweise auftreten (es gibt eine rechte und linke Rollung, und ähnlich für die Handgelenksbewegungen).
Sie sind auch die hauptsächlichen natürlichen Bewegungen der Hände und Finger.

Alle Fingerbewegungen müssen von den Hauptmuskeln der Arme, der Schulterblätter im Rücken und den Brustmuskeln vorne, die in der Mitte der Brust verankert sind, unterstützt werden.
Das kleinste Zucken des Fingers bezieht deshalb alle diese Muskeln mit ein.
\textbf{Es gibt es nicht, dass sich nur ein Finger bewegt -- jede Fingerbewegung bezieht den ganzen Körper mit ein}.
Stressreduzierung ist wichtig für die \hyperref[c1ii14]{Entspannung}\index{Entspannung} dieser Muskeln, sodass sie auf die Bewegung der Fingerspitzen reagieren und diese unterstützen können.
Die hauptsächlichen Handbewegungen werden hier nur kurz besprochen; mehr Details dazu finden Sie in den Quellen (\hyperref[Fink]{Fink} oder \hyperref[Sandor]{Sandor} und Mark für die Anatomie).


\paragraph{Pronation und Supination}
\label{c1iii4ProSup}

Die Hand kann um die Achse des Unterarms gedreht werden.
Die Einwärtsdrehung (Daumen nach unten) wird \textbf{Pronation} und die Auswärtsdrehung (Daumen nach oben) \textbf{Supination} genannt.
Diese Bewegungen kommen zum  Beispiel bei \hyperref[c1iii3b]{Oktavtremolos}\index{Oktavtremolos} ins Spiel.
Es gibt zwei Knochen in Ihrem Unterarm: der innere Knochen (Speiche, verbunden mit dem Daumen) und der äußere Knochen (Elle, verbunden mit dem kleinen Finger).
Die Drehung der Hand geschieht durch die Drehung des inneren Knochens gegen den äußeren (Handposition bezogen auf die des Klavierspielers, dessen Handflächen nach unten zeigen).
Der äußere Knochen wird vom Oberarm in Position gehalten.
Wenn die Hand gedreht wird, bewegt sich deshalb der Daumen viel mehr als der kleine Finger.
Eine schnelle Pronation ist eine gute Art, mit dem Daumen zu spielen.
Beim Spielen eines Oktavtremolos ist es leicht, den Daumen zu bewegen, aber der kleine Finger kann nur schnell bewegt werden, wenn man eine Kombination der Bewegungen benutzt.
\textbf{Somit reduziert sich das Problem, schnelle Oktavtremolos zu spielen, auf das Lösen des Problems, wie man den kleinen Finger\footnote{schnell} bewegt.}
Das Oktavtremolo wird gespielt, indem man den kleinen Finger zusammen mit dem Oberarm und den Daumen zusammen mit dem Unterarm bewegt (kombiniert mit den Fingerbewegungen).
 

\paragraph{Schub und Zug}
\label{c1iii4SchubZug}

Schub ist eine schiebende Bewegung in Richtung der Klappe\footnote{also vom Körper weg}, die üblicherweise von einem leichten Anheben des Handgelenks begleitet wird.
Mit \hyperref[c1ii2]{gebogenen Fingern}\index{gebogenen Fingern} bewirkt die Schubbewegung, dass die vorwärts gerichtete Vektorkraft der Hand entlang der Knochen der Finger geführt wird.
Das fügt Kontrolle und Kraft hinzu.
Er ist deshalb für das Spielen von Akkorden nützlich.
Der Zug ist eine ähnliche Bewegung weg von der Klappe\footnote{also zum Körper hin}.
Bei diesen Bewegungen kann die gesamte Bewegung größer oder kleiner als die abwärts gerichtete Vektorkomponente (der Anschlag) sein, was eine größere Kontrolle erlaubt.
Schub ist einer der Hauptgründe, warum die Grundhaltung der Finger gekrümmt ist.
Versuchen Sie, einen großen Akkord mit vielen Noten zu spielen, zuerst indem Sie die Hand gerade herunter bewegen wie im Freien Fall und dann mit der Schubbewegung.
Beachten Sie die überlegenen Resultate mit dem Schub.
Zug ist für einige Legato- und leise Passagen nützlich.
Experimentieren Sie also immer mit dem Hinzufügen von ein wenig Schub oder Zug, wenn Sie Akkorde üben.


\paragraph{Krallen und Schnellen}

Krallen ist das Bewegen der Finger zur Handfläche hin und Schnellen das Öffnen der Finger in ihre gestreckte Position.
Viele Schüler erkennen nicht, dass die Fingerspitzen zum Spielen zusätzlich zur Auf- und Abwärtsbewegung auch nach innen und außen bewegt werden können.
Das sind nützliche zusätzliche Bewegungen.
Sie fügen eine größere Kontrolle hinzu, besonders bei Legato- und leisen Passagen und ebenso beim Staccato-Spiel.
Genau wie bei \hyperref[c1iii4SchubZug]{Schub und Zug}\index{Schub und Zug} erlauben diese Bewegungen eine größere Bewegung mit einem viel kleineren Tastenweg.
Versuchen Sie deshalb, anstatt die Finger für den Anschlag immer möglichst gerade nach unten zu führen, mit etwas Krallen oder Schnellen zu experimentieren, um zu sehen, ob es etwas bringt.
Beachten Sie, dass die Krallbewegung viel natürlicher und leichter auszuführen ist als eine Bewegung gerade nach unten.
Die gerade Abwärtsbewegung der Fingerspitze ist in Wirklichkeit eine komplexe Kombination eines Krallens und eines Schnellens.
Die Anschlagsbewegung kann manchmal vereinfacht werden, indem man die Finger flach herausstellt und nur mit kleinen Krallbewegungen spielt.
Das ist der Grund, warum man manchmal mit flachen Fingern besser als mit gekrümmten Fingern spielen kann.


\paragraph{Rollung}
\label{Rollung}

Die Rollung ist eine der nützlichsten Bewegungen.
Sie ist eine schnelle Drehung und Gegendrehung der Hand: eine schnelle Kombination von \hyperref[c1iii4ProSup]{Pronation und Supination}\index{Pronation und Supination} oder umgekehrt.
Wir haben gesehen, dass \hyperref[c1ii11]{parallele Sets}\index{parallele Sets} fast mit jeder Geschwindigkeit gespielt werden können.
\textbf{Beim Spielen schneller Passagen tritt das Problem der Geschwindigkeit auf, wenn wir parallele Sets verbinden müssen.}
Es gibt nicht nur eine Lösung für dieses Verbindungsproblem.
\textbf{Die Bewegung, die einer generellen Lösung am nächsten kommt, ist die Rollung, besonders wenn der Daumen beteiligt ist, wie bei \hyperref[c1iii5a]{Tonleitern}\index{Tonleitern} und \hyperref[Arpeggios]{Arpeggios}\index{Arpeggios}.}
Einmalige Rollungen können extrem schnell ohne Stress ausgeführt werden und somit dem Spielen Geschwindigkeit hinzufügen; mehrere schnelle Rollungen müssen jedoch \enquote{aufgeladen} werden; das heißt fortlaufendes schnelles Rollen ist schwierig.
Es ist aber für das Verbinden von parallelen Sets ziemlich praktisch, weil die Rollung benutzt werden kann, um die Verbindung zu spielen, und während des parallelen Sets wieder aufgeladen wird.
Um es noch einmal zu betonen, was am Anfang des Abschnitts herausgestellt wurde: Diese Rollungen und andere Bewegungen müssen nicht groß sein und sind im Allgemeinen kaum wahrnehmbar klein; somit kann die Rollung eher als Rollungsimpuls als eine tatsächliche Bewegung angesehen werden.


\paragraph{Bewegung des Handgelenks}

Wir haben bereits gesehen, dass die Bewegung des Handgelenks nützlich ist, wenn mit dem Daumen oder kleinen Finger gespielt wird; die allgemeine Regel ist, das Handgelenk für den kleinen Finger anzuheben und für den Daumen zu senken.
Natürlich ist dies keine strenge Regel; es gibt viele Ausnahmen.
Die Bewegung des Handgelenks ist auch in Kombination mit anderen Bewegungen nützlich.
Durch das Kombinieren der Handgelenksbewegung mit der \hyperref[c1iii4ProSup]{Pronation und Supination}\index{Pronation und Supination} kann man Drehbewegungen für das Spielen von sich wiederholenden Passagen erzeugen, wie bei Begleitungen durch die linke Hand oder im ersten Satz von Beethovens Mondscheinsonate.
Das Handgelenk kann sowohl auf- und abwärts als auch von einer Seite zur anderen bewegt werden.
Es sollte jede Anstrengung unternommen werden, damit der spielende Finger parallel zum Unterarm ist; das wird durch die seitliche Bewegung des Handgelenks erreicht.
Diese Anordnung bewirkt die geringste Menge von seitlicher Anspannung in den Sehnen beim Bewegen der Finger und vermindert die Wahrscheinlichkeit von Verletzungen wie dem Karpaltunnel-Syndrom.
Wenn Sie feststellen, dass sie die Angewohnheit haben, mit seitwärts abgewinkelten Handgelenken zu spielen (oder zu tippen), kann das ein Warnsignal dafür sein, dass Sie Probleme bekommen werden.
Ein lockeres Handgelenk ist auch eine Voraussetzung für eine völlige \hyperref[c1ii14]{Entspannung}\index{Entspannung}.


\paragraph{Zusammenfassung}

Die obigen Ausführungen sind eine kurze Übersicht der Handbewegungen.
Ein ganzes Buch kann über dieses Thema geschrieben werden.
Und wir haben noch nicht einmal die Themen über das Hinzufügen anderer Bewegungen des Ellbogens, Oberarms, der Schultern, Füße usw. berührt.
Der Schüler wird ermutigt, dieses Gebiet so weit wie möglich zu erforschen, da dies nur hilfreich sein kann.
Die gerade besprochenen Bewegungen werden selten alleine benutzt.
Parallele Sets können mit jeder Kombination der meisten oben angeführten Bewegungen gespielt werden, ohne dass man einen Finger bewegt (relativ zur Hand).
Das meinte ich in dem Abschnitt über das \hyperref[c1ii7]{Üben mit getrennten Händen}\index{Üben mit getrennten Händen} mit der Empfehlung, mit den Handbewegungen zu experimentieren und sie zu ökonomisieren.
Das Wissen um jede Art der Bewegung wird dem Schüler gestatten, jede einzelne auszuprobieren, um zu sehen, welche gebraucht wird.
Es ist in der Tat der Schlüssel zum Gipfel der Technik.



