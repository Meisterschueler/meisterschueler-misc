% File: c1iii2

\subsection{Zyklisch spielen (Chopins Fantaisie Impromptu, Op. 66)}
\label{c1iii2}

\textbf{Zirkulieren ist die beste Technik aufbauende Prozedur für neue oder schnelle Passagen, die Sie nicht beherrschen.
Zirkulieren (auch \enquote{schleifen} genannt) bedeutet, einen Abschnitt zu nehmen und diesen wiederholt, üblicherweise fortlaufend und ohne Pausen, zu spielen.}
Wenn die Verbindung, die für das fortlaufende Zirkulieren notwendig ist, die gleiche ist wie die erste Note des Abschnitts, dann zirkuliert dieser Abschnitt \enquote{natürlich}; er wird ein selbst-zirkulierender Abschnitt genannt.
Ein Beispiel ist das CGEG-Quadrupel.
Wenn die Verbindung abweicht, müssen Sie eine erfinden, die zur ersten Note hinführt, sodass Sie ohne Pausen zirkulieren können.

\textbf{Zirkulieren ist im Grunde reine Wiederholung, aber es ist wichtig, es fast als eine Anti-Wiederholungs-Prozedur zu benutzen, als einen Weg, stupides Wiederholen zu vermeiden.
Die Idee hinter dem Zirkulieren ist, dass man die Technik so schnell erwirbt, dass es unnötiges stupides Wiederholen ausschließt.}
Ändern Sie die Geschwindigkeit und experimentieren Sie mit verschiedenen Hand-, Arm- bzw. Fingerpositionen für ein optimales Spielen, um zu vermeiden, dass Sie schlechte Angewohnheiten annehmen, und achten Sie immer auf die \hyperref[c1ii14]{Entspannung}; versuchen Sie, das exakt gleiche nicht zu oft zu wiederholen.
Spielen Sie leise (auch laute Abschnitte), bis Sie die Technik erlangt haben, gehen Sie bis zu Geschwindigkeiten von mindestens 20\% über der vorgegebenen Geschwindigkeit und wenn möglich bis zur doppelten Geschwindigkeit.
Mehr als 90\% der Zirkulierzeit sollten Sie mit Geschwindigkeiten spielen, die Sie bequem und genau handhaben können.
Zirkulieren Sie dann schrittweise langsamer bis zu sehr langsamen Geschwindigkeiten.
Sie sind fertig, wenn Sie bei jeder Geschwindigkeit, für beliebig lange Zeit, ohne auf die Hand zu sehen, völlig entspannt und mit voller Kontrolle spielen können.
Es könnte sein, dass Ihnen bestimmte Geschwindigkeiten Schwierigkeiten bereiten.
Üben Sie diese Geschwindigkeiten, weil diese eventuell gebraucht werden, wenn Sie mit dem \hyperref[c1ii25]{beidhändigen Spielen} anfangen.
Üben Sie ohne das Pedal (teilweise um die schlechte Angewohnheit zu vermeiden, die Taste während des Anschlags nicht ganz herunterzudrücken), bis die Technik erworben ist.
\textbf{\hyperref[c1ii7]{Wechseln Sie oft die Hände}, um Verletzungen zu vermeiden.}

Wenn eine Technik 10.000 Wiederholungen erfordert (eine typische Erfordernis für wirklich schwieriges Material), erlaubt Ihnen das Zirkulieren, diese in der kürzest möglichen Zeit auszuführen.
Typische Zykluszeiten liegen bei einer Sekunde, sodass man für 10.000 Zyklen weniger als vier Stunden benötigt.
Wenn Sie diesen Abschnitt täglich zehn Minuten, an fünf Tagen die Woche, zirkulieren, werden 10.000 Zyklen fast einen Monat dauern.
Natürlich dauert es Monate, sehr schwieriges Material zu lernen, wenn man die besten Methoden benutzt, und \textit{viel} länger, wenn man weniger effiziente Methoden benutzt.

\textbf{Zirkulieren ist potenziell die verletzungsgefährdendste Prozedur beim Klavierüben}; seien Sie deshalb bitte vorsichtig.
Übertreiben Sie es nicht am ersten Tag, und schauen Sie, was am nächsten Tag geschieht.
Wenn Ihnen am nächsten Tag nichts weh tut, können Sie mit dem Zirkuliertraining weitermachen bzw. es steigern.
Arbeiten Sie beim Zirkulieren vor allem immer an zwei Sachen gleichzeitig, einer für die rechte Hand und einer anderen für die linke Hand, sodass Sie die Hände oft abwechseln können.
Bei jungen Menschen kann zu viel zu zirkulieren zu Schmerzen führen; hören Sie in diesem Fall mit dem Zirkulieren auf, und die Hand sollte sich innerhalb weniger Tage erholen.
Bei älteren Menschen kann zu viel zu zirkulieren Ausbrüche von Arthrose verursachen, bei denen es Monate dauern kann, bis sie abklingen.
 

\label{c1iii2fi}

Lassen Sie uns das Zirkulieren auf Chopins Fantaisie Impromptu anwenden: das Arpeggio in der linken Hand, Takt 5.
Die ersten sechs Noten zirkulieren in sich selbst, sie können es also mit diesen versuchen.
Als ich es das erste Mal versucht habe, war die Streckung für meine kleinen Hände zu groß, sodass ich zu schnell müde wurde.
Ich zirkulierte deshalb die ersten 12 Noten.
Die leichteren zweiten sechs Noten erlaubten es meinen Händen, sich ein wenig zu erholen, und ich konnte so den Abschnitt aus 12 Noten länger und mit höherer Geschwindigkeit spielen.
Wenn Sie natürlich die Geschwindigkeit wirklich steigern möchten (für die linke Hand nicht notwendig, könnte aber in diesem Stück für die rechte Hand nützlich sein), zirkulieren Sie nur das erste parallele Set (die ersten drei oder vier Noten für die linke Hand).

Dass man den ersten Abschnitt spielen kann, bedeutet nicht, dass man nun all die anderen Arpeggios spielen kann.
Sie werden sogar für die gleichen Noten eine Oktave tiefer praktisch bei Null anfangen müssen.
Natürlich wird das zweite Arpeggio einfacher sein, wenn man das erste gemeistert hat, aber Sie werden überrascht darüber sein, wie viel Arbeit es bei den Wiederholungen erfordert, wenn sich nur ein klein wenig in dem Abschnitt ändert.
Das geschieht, weil es so viele Muskeln im Körper gibt, dass das Gehirn verschiedene Gruppen auswählen kann, um Bewegungen zu erzeugen, die nur ganz leicht anders sind (und es macht es üblicherweise).
Anders als ein Roboter haben Sie wenig Einfluss darauf, welche Muskeln sich Ihr Gehirn aussucht.
Nur wenn Sie eine sehr große Zahl von solchen Arpeggios gespielt haben, fällt Ihnen das nächste leicht.
Deshalb sollten Sie davon ausgehen, dass Sie einige Arpeggios zirkulieren müssen.

Damit man versteht, wie dieses Stück von Chopin zu spielen ist, ist es hilfreich, die mathematische Grundlage des Teils der Komposition mit dem \enquote{3 gegen 4}-Timing zu analysieren.
Die rechte Hand spielt sehr schnell, sagen wir (ungefähr) vier Noten je halber Sekunde.
Gleichzeitig spielt die linke Hand mit einer langsameren Geschwindigkeit, drei Noten je halber Sekunde.
Wenn alle Noten sehr genau gespielt werden, hört das Publikum eine Notenfrequenz von zwölf Noten je halber Sekunde, weil diese Frequenz dem kleinsten Zeitintervall zwischen Noten entspricht.
\textbf{Das heißt, wenn Ihre rechte Hand so schnell spielt wie sie kann, dann hat Chopin es erreicht, dieses Stück durch das Hinzufügen des \textit{langsameren} Spielens mit der linken Hand auf Ihre dreifache Maximalgeschwindigkeit zu bringen!}

Aber warten Sie, nicht alle der zwölf Noten sind vorhanden; es sind in Wirklichkeit nur sieben, fünf Noten fehlen also.
Diese fehlenden Noten erzeugen was man ein Moiré-Muster nennt, welches ein drittes Muster ist, das auftaucht, wenn zwei nicht vergleichbare Muster überlagert werden.
Dieses Muster erzeugt einen wellenartigen Effekt innerhalb jedes Takts und Chopin verstärkte diesen Effekt, indem er in der linken Hand ein Arpeggio benutzte, das synchron mit dem Moiré-Muster wie eine Welle aufsteigt und fällt.
Die Beschleunigung um einen Faktor von drei und das Moiré-Muster sind rätselhafte Effekte, die sich beim Publikum einschleichen, weil dieses keine Ahnung hat, was sie erzeugt hat oder dass sie überhaupt existieren.
Mechanismen, die das Publikum ohne sein Wissen beeinflussen, erzeugen oft dramatischere Effekte als jene, die offensichtlich sind (wie laut, legato oder rubato).
Die großen Komponisten haben eine unglaubliche Anzahl dieser versteckten Mechanismen erfunden, und eine mathematische Analyse ist oftmals der leichteste Weg, sie hervorzukitzeln.
Chopin dachte wahrscheinlich nie in Begriffen wie nicht vergleichbaren Gruppen und Moiré-Mustern; er hat diese Konzepte allein auf Grund seiner Genialität intuitiv verstanden.

Es ist aufschlussreich, über den Grund für die fehlende erste Note des Taktes (5) für die rechte Hand zu spekulieren, denn wenn wir den Grund ermitteln können, werden wir genau wissen, wie man ihn spielen muss.
Beachten Sie, dass dies direkt am Anfang der Melodie der rechten Hand auftritt.
Am Anfang einer Melodie oder musikalischen Phrase stoßen Komponisten immer auf zwei gegensätzliche Erfordernisse: Eines ist, dass die Phrase (im Allgemeinen) leise anfangen sollte, und das zweite ist, dass die erste Note des Takts ein Abschlag ist und betont sein sollte.
Der Komponist kann geschickt beiden Erfordernissen genügen, indem er die erste Note eliminiert und so den \hyperref[c1iii1b]{Rhythmus} bewahrt und doch leise anfängt (in diesem Fall kein Ton)!
Sie werden keine Schwierigkeiten haben, zahlreiche Beispiele dieses Mittels zu finden -- sehen Sie dazu \hyperref[c1iii20]{Bachs Inventionen}.
Ein weiteres Mittel ist, die Phrase am Ende eines unvollständigen Takts beginnen zu lassen, sodass der erste Abschlag des ersten vollständigen Takts kommt, nachdem ein paar Noten gespielt sind (ein klassisches Beispiel dafür ist der Anfang des ersten Satzes von Beethovens Appassionata).
Das bedeutet, dass die erste Note der rechten Hand in diesem Takt von Chopins Fantaisie-Impromptu leise sein muss und die zweite Note lauter als die erste, um den Rhythmus streng aufrechtzuerhalten (ein weiteres Beispiel der Wichtigkeit des Rhythmus!).
Wir sind nicht gewohnt, auf diese Art zu spielen; normalerweise spielen wir so, dass wir mit der ersten Note als Abschlag beginnen.
Es ist in diesem Fall wegen der Geschwindigkeit besonders schwierig; deshalb benötigt dieser Anfang eventuell zusätzliches Üben.

Diese Komposition beginnt damit, dass sie das Publikum schrittweise wie eine unwiderstehliche Einladung mit der lauten Oktave im ersten Takt, gefolgt von dem rhythmischen Arpeggio im unteren Notensystem, in ihren Rhythmus zieht.
Die fehlende Note im fünften Takt wird nach einigen Wiederholungen wiederhergestellt und somit die Moiré-Wiederholungsfrequenz und der effektive Rhythmus verdoppelt.
Im zweiten Thema (Takt 13) wird die fließende Melodie der rechten Hand durch zwei gebrochene Akkorde ersetzt und somit der Eindruck einer Vervierfachung des Rhythmus erzeugt.
Diese \enquote{rhythmische Beschleunigung} gipfelt in dem Forte-Höhepunkt der Takte 19 und 20.
Das Publikum kann dann wegen der \enquote{Besänftigung} des Rhythmus durch die verzögerte melodische Note (des kleinen Fingers) der rechten Hand und durch das schrittweise Leiserwerden der rechten Hand, das durch das \textit{diminuendo} bis zum \textit{pp} verwirklicht wird, Atem holen.
Der ganze Zyklus wird dann wiederholt, dieses Mal mit zusätzlichen Elementen, die den Höhepunkt verstärken, bis er in den absteigenden donnernden gebrochenen Akkorden endet.
Um diesen Teil zu üben, kann jeder gebrochene Akkord einzeln zirkuliert werden.
Diesen Akkorden fehlt das \enquote{3,4}-Konstrukt. Sie bringen Sie aus der rätselhaften \enquote{3,4}-Unterwelt zurück und bereiten Sie auf den langsamen Abschnitt vor.

Wie bei den meisten Stücken von Chopin, gibt es für dieses Stück kein \enquote{korrektes} Tempo.
Wenn man jedoch schneller als ungefähr zwei Sekunden je Takt spielt, neigt der \enquote{3x4}-Multiplikationseffekt dazu, zu verschwinden, und man hat üblicherweise nur noch hauptsächlich das Moiré und andere Effekte.
Das ist teilweise wegen der abnehmenden Genauigkeit mit zunehmender Geschwindigkeit so, aber wichtiger noch, weil die zwölffache Geschwindigkeit zu schnell für das Ohr wird, um ihr zu folgen.
Oberhalb von ungefähr 20 Hz beginnen Wiederholungen für das menschliche Ohr eher die Eigenschaften von Klang anzunehmen.
Deshalb funktioniert das Multiplikationsmittel nur bis ungefähr 20 Hz; oberhalb davon bekommt man einen neuen Effekt, der sogar noch mehr als eine unglaubliche Geschwindigkeit etwas besonderes sein kann -- die \enquote{schnellen Noten} verwandeln sich in einen \enquote{niederfrequenten Klang}.
Somit ist bei 20 Hz eine Klanggrenze.
Deshalb ist die tiefste Note des Klaviers ein A mit ungefähr 27 Hz.
Hier ist die große Überraschung: Es gibt Hinweise, dass Chopin diesen Effekt gehört hat!
Beachten Sie, dass der schnelle Teil am Anfang die Bezeichnung \enquote{Allegro agitato} hat; das bedeutet, dass jede Note deutlich hörbar sein muss.
Das Allegro auf dem Metronom entspricht einer zwölffachen Geschwindigkeit bei 10 bis 20 Hz, der richtigen Frequenz, um die Vervielfachung zu hören -- direkt unterhalb der Klanggrenze.
Das \enquote{Agitato} stellt sicher, dass diese Frequenz hörbar wird.
Wenn dieser schnelle Abschnitt nach dem Moderato-Abschnitt noch einmal kommt, ist er mit Presto bezeichnet, was 20 bis 40 Hz entspricht -- Chopin wollte, dass wir ihn unterhalb und oberhalb der Klanggrenze spielen!
Es gibt also mathematische Indizien dafür, dass Chopin diese Klanggrenze kannte.

Der langsame mittlere Abschnitt wurde kurz in \hyperref[c1ii25]{\autoref{c1ii25}} beschrieben.
Der schnellste Weg ihn zu lernen, ist, wie bei vielen Stücken von Chopin, mit dem Auswendiglernen der linken Hand anzufangen.
Das deshalb, weil der Verlauf der Akkorde oftmals der gleiche bleibt, selbst wenn Chopin die rechte Hand durch eine neue Melodie ersetzt, da die linke Hand hauptsächlich die Begleitakkorde beisteuert.
Beachten Sie, dass das \enquote{4,3}-Timing nun durch ein \enquote{2,3}-Timing ersetzt wird, das viel langsamer gespielt wird.
Es wird für einen anderen Effekt benutzt, um die Musik sanfter zu machen und ein freieres Rubato zuzulassen.

Der dritte Teil ist dem ersten ähnlich, außer dass er schneller gespielt wird, was zu einem ganz anderen Effekt führt, und der Schluss ist anders.
Dieser Schluss ist für kleine Hände schwierig und erfordert eventuell zusätzliche Zirkulierarbeit mit der rechten Hand.
In diesem Abschnitt trägt der kleine Finger der rechten Hand die Melodie, aber die antwortende Oktavnote des Daumens bereichert die melodische Linie.
Das Stück endet mit einer nostalgischen Wiederaufnahme des langsamen Satzthemas in der linken Hand.
Unterscheiden Sie die oberste Note dieser Melodie der linken Hand (Gis -- im siebten Takt von hinten) deutlich von der gleichen Note, die von der rechten Hand gespielt wird, indem Sie sie ein wenig länger halten und sie dann mit dem Pedal aushalten.

Das Gis ist die wichtigste Note in diesem Stück.
So ist der \textit{\textbf{sf}}-Anfang mit der Gis-Oktave nicht nur eine Fanfare, die das Stück einleitet, sondern eine geschickte Art, wie Chopin das Gis in den Kopf der Zuhörer einpflanzt.
Deshalb sollten Sie diese Note nicht zu eilig spielen; nehmen Sie sich Zeit, und lassen Sie sie einwirken.
Wenn Sie das Stück durchsehen, werden Sie feststellen, dass das Gis alle wichtigen Positionen besetzt.
Im langsamen Abschnitt ist das Gis ein As, was\footnote{bei \hyperref[et1]{gleichmäßig temperierter Stimmung}} dieselbe Note ist.
Dieses Gis ist ein weiteres dieser Mittel, mit denen ein großer Komponist dem Publikum wiederholt \enquote{eins überziehen} kann, ohne dass das Publikum merkt, was ihm geschieht.
Dem Klavierspieler hilft das Wissen um das Gis beim Interpretieren und Auswendiglernen des Stücks.
So kommt der konzeptionelle Höhepunkt des Stücks am Ende (wie er sollte), wenn beide Hände dasselbe Gis spielen müssen (8. und 7. Takt vom Ende her); deshalb muss das beidhändige Gis mit äußerster Sorgfalt ausgeführt werden, während man die kontinuierlich ausklingende Gis-Oktave der rechten Hand beibehält.

Unsere Analyse führt uns zum Brennpunkt, das heißt zur Frage, wie schnell man dieses Stück spielt.
Eine hohe Genauigkeit ist erforderlich, um den Zwölf-Noten-Effekt zum Vorschein zu bringen, sowie ein unmenschlich genaues Spielen oberhalb der Klanggrenze.
Wenn man dieses Stück zum ersten Mal lernt, wird die Frequenz von zwölf Noten wegen des Mangels an Genauigkeit zunächst nicht zu hören sein.
Wenn man es am Ende \enquote{packt}, hört sich die Musik urplötzlich sehr \enquote{rege} an.
Wenn man zu schnell spielt und die Genauigkeit verliert, dann kann man die Verdreifachung verlieren --  es verwascht, und das Publikum hört nur die vier Noten.
Anfänger können erreichen, dass sich das Stück schneller anhört, indem sie langsamer werden und die Genauigkeit erhöhen.
Obwohl die rechte Hand die Melodie trägt, muss die linke deutlich zu hören sein, da sonst sowohl der Zwölf-Noten-Effekt als auch das Moiré-Muster verschwinden.
Da dies ein Stück von Chopin ist, ist es nicht erforderlich, dass der Zwölf-Noten-Effekt hörbar ist; diese Komposition ist einer unendlichen Zahl von Interpretationen zugänglich, und manche von Ihnen möchten vielleicht die linke Hand außen vor lassen und sich nur auf die rechte konzentrieren und können trotzdem etwas magisches erzeugen.

Ein Vorteil des Zirkulierens ist, dass die Hand fortlaufend spielt, was das fortlaufende Spielen besser simuliert, als wenn man isolierte Abschnitte übt.
Es erlaubt Ihnen auch, mit kleinen Änderungen in den Fingerpositionen usw. zu experimentieren, um die optimalen Bedingungen für das Spielen herauszufinden.
Der Nachteil ist, dass die Handbewegungen beim Zirkulieren von denen abweichen können, die man beim Spielen des Stücks braucht.
Die Arme sind während des Zirkulierens meistens unbeweglich, während im richtigen Stück die Hände üblicherweise bewegt werden müssen.
Deshalb müssen Sie eventuell in den Fällen, in denen der Abschnitt nicht natürlich zirkuliert, das abschnittsweise Üben benutzen, ohne zu zirkulieren.
Ohne das Zirkulieren haben sie den Vorteil, dass Sie nun die Verbindung einschließen können.



