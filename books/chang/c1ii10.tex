% File: c1ii10

\section{Freier Fall, Akkord-Übung und Entspannung}\hypertarget{c1ii10}{}

\textbf{Das Spielen von exakten Akkorden zu lernen, ist der erste Schritt in der Anwendung des \hyperlink{c1ii9}{Akkord-Anschlags}.}
Lassen Sie uns den obigen CEG-Akkord mit der linken Hand üben.
Die Armgewichtsmethode ist der beste Weg, Genauigkeit und Entspannung zu erreichen; dieser Ansatz wurde ausreichend in den angegebenen Quellen behandelt (\hyperlink{Fink}{Fink}, \hyperlink{Sandor}{Sandor}) und wird deshalb hier nur kurz angesprochen.
Setzen Sie Ihre Finger auf die Tasten, um CEG zu spielen.
Entspannen sie Ihren Arm (eigentlich den ganzen Körper), halten Sie Ihr Handgelenk flexibel, heben Sie die Hand 5 bis 20 cm über die Tasten (am Anfang die kürzere Entfernung), und lassen Sie Ihre Hand einfach frei fallen.
Lassen Sie die Hand und die Finger als eine Einheit fallen, bewegen Sie nicht die Finger.
Entspannen Sie die Hände völlig während des Fallens, dann \enquote{platzieren} Sie die Finger und das Handgelenk im Moment des Aufpralls auf die Tasten und beugen Sie das Handgelenk ein wenig, um den Stoß des Aufpralls zu mindern und die Tasten niederzudrücken.
\textbf{Indem Sie die Schwerkraft Ihre Hand absenken lassen, überantworten Sie Ihre Stärke oder Empfindlichkeit einer konstanten Kraft.}

Es mag zunächst unglaublich erscheinen, aber ein untergewichtiger Sechsjähriger und ein gigantischer Sumoringer, die ihre Hände aus derselben Höhe fallen lassen, werden einen Ton mit der gleichen Lautstärke erzeugen, wenn sie beide den Freien Fall korrekt ausführen (was nicht einfach ist, insbesondere für den Sumoringer).
Dies geschieht, weil die Geschwindigkeit des Freien Falls unabhängig von der Masse ist und der Hammer in freien Flug übergeht, sobald die Hammernuss die Stoßzunge verlässt.
Physikstudenten werden bemerken, dass bei einem elastischen Stoß (Kollision von Billardkugeln) die kinetische Energie erhalten bleibt und das oben Gesagte nicht gilt.
Bei einem solchen elastischen Stoß würde sich die Taste mit hoher Geschwindigkeit von der Fingerkuppe wegbewegen, wie ein Golfball, der von einem Betonboden abspringt.
Hier wird aber, weil die Finger entspannt und die Fingerkuppen weich sind (inelastischer Stoß), die kinetische Energie nicht erhalten und die kleine Masse (Klaviertaste) kann bei der größeren Masse (Finger, Hand und Arm) bleiben, was zu einem kontrollierten Anschlag führt.
Deshalb gilt das oben Gesagte, vorausgesetzt, dass die Klaviermechanik korrekt eingestellt ist und die effektive Masse für den Anschlag viel kleiner ist als die Masse der Hand des Sechsjährigen.
Eine Versteifung der Hand nach dem Aufprall gewährleistet eine Übertragung des gesamten Armgewichts beim Anschlag.
Versteifen Sie die Hand nicht, bevor Sie den unteren Punkt des Tastenwegs erreicht haben, weil das eine zusätzliche Kraft hinzufügen würde - wir möchten die Tasten nur mit der Schwerkraft spielen.

Genaugenommen wird der Sumoringer wegen der Impulserhaltung einen etwas lauteren Ton erzeugen, aber der Unterschied wird trotz der Tatsache, dass sein Arm vielleicht zwanzigmal schwerer ist, gering sein.
Eine weitere Überraschung ist, dass der Anschlag mit dem Freien Fall, wenn er erst einmal richtig gelernt ist, den lautesten Ton erzeugt, den dieses Kind jemals gespielt hat (bei einem hohen Fall), und dass er eine hervorragende Art ist, junge Schüler zu lehren, wie man fest spielt.
Fangen Sie bei jungen Schülern mit kurzen Fällen an, weil am Anfang ein wirklicher Freier Fall schmerzhaft sein kann, wenn die Höhe zu groß ist.
Für einen erfolgreichen Freien Fall ist es wichtig, insbesondere bei jungen Schülern, ihnen beizubringen, dass sie so tun, als ob kein Klavier da sei und die Hand durch die Tastatur hindurch fallen soll (aber durch das Klavier gestoppt wird).
Sonst werden die meisten jungen Schüler unbewusst die Hand anheben, wenn sie auf dem Klavier landet.
Mit anderen Worten: Der Freie Fall ist eine konstante Beschleunigung, und die Hand beschleunigt sogar noch während die Tasten gedrückt werden.
Am Ende ruht die Hand mit ihrem eigenen Gewicht auf den Tasten - diese Spielweise erzeugt einen angenehmen, tiefen Klang.
Beachten Sie, dass es für diesen Anschlag wichtig ist, den ganzen Weg abwärts zu beschleunigen - s. \hyperlink{c1iii1}{Abschnitt III.1} über das Erzeugen eines guten Klangs.

Die bekannte \enquote{Beschleunigte Mechanik} von Steinway funktioniert, weil sie der Hammerbewegung durch eine abgerundete Stütze unter der Buchse in der Tastenmitte eine Beschleunigung hinzufügt.
Das verschiebt den Drehpunkt mit dem Niederdrücken der Taste nach vorne, verkürzt somit die vordere und verlängert die hintere Seite der Taste und bewirkt dadurch eine Beschleunigung der Pilote bei einem konstanten Niederdrücken.
Dies veranschaulicht die Bedeutung, die Klavierdesigner der Beschleunigung des Tastendrucks beimessen, und die Armgewichtsmethode stellt sicher, dass wir den vollen Nutzen aus der Gravitationsbeschleunigung ziehen, um einen guten Klang zu erzeugen.
Die Wirksamkeit der Beschleunigten Mechanik wird kontrovers diskutiert, weil es exzellente Klaviere ohne dieses Merkmal gibt.
Offensichtlich ist für den Klavierspieler wichtiger, die Beschleunigung zu kontrollieren als vom Klavier abhängig zu sein.

Die Finger müssen \enquote{gesetzt} werden, nachdem die Tasten den unteren Punkt erreichen, um die Abwärtsbewegung der Hand zu stoppen.
Dies erfordert eine kurze Kraftanwendung auf die Finger.
Lassen Sie diese Kraft weg und entspannen Sie völlig, sobald die Hand stoppt, sodass Sie fühlen können, wie die Schwerkraft Ihren Arm nach unten zieht.
Lassen Sie die Hand auf den Tasten ruhen, sodass nur die Schwerkraft die Tasten unten hält.
Sie haben soeben erreicht, dass Sie die Tasten mit der geringstmöglichen Anstrengung niederdrücken; das ist das Wesentliche der Entspannung.
\textbf{Beachten Sie, dass es ein wichtiges Element der Entspannung ist, alle Muskeln sofort zu entspannen, sobald der Freie Fall vorüber ist.}

Anfänger werden die Akkorde mit zu vielen unnötigen Kräften spielen, die nicht genau kontrolliert werden können.
\textbf{Die Benutzung der Schwerkraft kann alle unnötigen Kräfte oder Spannungen eliminieren.}
Es mag wie ein merkwürdiger Zufall erscheinen, dass die Schwerkraft die richtige Kraft ist, um Klavier zu spielen.
\textit{Das ist kein Zufall.}
\textbf{Die Menschen haben sich unter dem Einfluss der Schwerkraft entwickelt.
Unsere Kräfte zum Gehen, Heben usw. entwickelten sich, um \textit{genau} zur Schwerkraft zu passen.}
Das Klavier wurde natürlich so \textit{entworfen}, dass es zu diesen Kräften passt.
Wenn Sie wirklich entspannt sind, können Sie die Wirkung der Schwerkraft auf Ihre Hände richtig fühlen, wenn Sie spielen.
Einige Lehrer werden die Entspannung bis zu dem Punkt, an dem alles andere vernachlässigt wird, betonen, bis die \enquote{völlige} Entspannung erreicht ist; das könnte zu weit gehen - in der Lage zu sein, die Schwerkraft zu fühlen, ist ein notwendiges und ausreichendes Kriterium für die Entspannung.
\textbf{Der Freie Fall ist eine Methode, um Entspannung zu üben.
Wenn dieser entspannte Zustand erst einmal erreicht ist, muss er ein permanenter, integraler Bestandteil Ihres Klavierspiels werden.}
Eine völlige Entspannung bedeutet nicht, dass Sie zum Klavierspielen immer nur die Schwerkraft benutzen sollen.
Die meiste Zeit werden Sie Ihre eigene Kraft anwenden; \enquote{die Schwerkraft zu fühlen} ist nur eine Möglichkeit, den Grad Ihrer Entspannung zu messen.


\section{Parallele Sets}\hypertarget{c1ii11}{}

Nun, da der CEG-Akkord mit der linken Hand zufriedenstellend ist, versuchen Sie, plötzlich vom Akkord zum Quadrupel zu wechseln.
Sie werden nun die Finger bewegen müssen, beschränken Sie die Fingerbewegungen aber auf ein Minimum.
Damit das Wechseln klappt, bauen Sie die richtigen Hand- und Armbewegungen ein (siehe \hyperlink{Fink}{Fink}, \hyperlink{Sandor}{Sandor}), die wir später besprechen, aber das ist ein Thema für Fortgeschrittene, lassen Sie uns deshalb einen Schritt kürzer treten und annehmen, dass Ihnen das Wechseln nicht gelingt, damit wir eine mächtige Methode zum Lösen dieses Problemtyps zeigen können.

\textbf{Die grundlegendste Art zu lernen, wie man eine schwierige Passage spielt, ist, sie mit jeweils zwei Noten aufzubauen und dabei den \hyperlink{c1ii9}{Akkord-Anschlag} zu benutzen.} In unserem CGEG-Beispiel (der linken Hand) fangen wir mit den ersten beiden Noten an.
Ein zweinotiger Akkord-Anschlag (strenggenommen ein Intervall-Anschlag)!
Spielen Sie diese zwei Noten als perfektes Intervall; lassen Sie Ihre Hand und Finger (5 und 1) auf und ab springen, wie Sie es bereits beim CEG-Akkord getan haben.
Um diese zwei Noten schnell nacheinander zu spielen, senken Sie beide Finger zusammen, aber halten Sie Finger 1 etwas oberhalb von 5, sodass die 5 zuerst landet.
Es ist nur ein schnelles zweinotiges rollendes Intervall.
Da Sie beide Finger gleichzeitig nach unten bringen und nur einen leicht verzögern, können Sie sie so kurz hintereinander spielen wie Sie möchten, indem Sie die Verzögerung verringern.
\textbf{So verlangsamt man von unendlicher Geschwindigkeit!}

Ist es auf diese Art möglich, jede Kombination von Noten unendlich schnell zu spielen? Natürlich nicht.
Wie wissen wir, welche wir unendlich schnell spielen können und welche nicht?
Um diese Frage zu beantworten, müssen wir das Konzept des parallelen Spielens einführen.
Die obige Methode, die Finger zusammen zu senken, wird paralleles Spielen genannt, weil die Finger gleichzeitig gesenkt werden, also parallel.
\textbf{Ein paralleles Set ist eine Gruppe von Noten, die gleichzeitig mit einer Hand gespielt werden können.
Alle parallelen Sets können unendlich schnell gespielt werden - beim \hyperlink{c1ii9}{Akkord-Anschlag} werden parallele Sets benutzt.
Die Verzögerung zwischen aufeinander folgenden Fingern wird Phasenwinkel genannt.}
In einem Akkord ist der Phasenwinkel für alle Finger Null; eine detaillierte Behandlung der parallelen Sets finden Sie in \hyperlink{c1iii7b2}{Übung 2 in Abschnitt III.7b}.
Das ist einfach ein Akkord-Anschlag, aber der Begriff \enquote{parallele Sets} ist notwendig, um Irrtümer zu vermeiden, die aus der Tatsache resultieren, dass in der Musiktheorie \enquote{Akkord} und \enquote{Intervall} bestimmte Bedeutungen haben, die nicht auf alle parallelen Sets anwendbar sind.
Die höchste Geschwindigkeit der parallelen Sets wird durch die Reduzierung der Phase auf den kleinsten kontrollierbaren Wert erreicht.
Dieser Wert ist ungefähr gleich dem Fehler in Ihrem Akkordspiel.
Mit anderen Worten: Je genauer Ihre Akkorde sind, desto schneller wird Ihre maximal erreichbare Geschwindigkeit sein.
Deshalb wurde dem Üben perfekter Akkorde oben so viel Platz gewidmet.

Haben Sie erst einmal das CG gemeistert, können Sie mit dem nächsten - GE (1 3) - fortfahren, dann EG und schließlich das GC, um das Quadrupel und die Verbindung zu vervollständigen.
Verbinden Sie sie dann paarweise, CGE usw., um das Quadrupel zu vervollständigen.
Beachten Sie, dass CGE (5 1 3) ebenfalls ein paralleles Set ist.
Deshalb kann das Quadrupel plus die Verbindung aus den parallelen Sets (5 1 3) und (3 1 5) gebildet werden.
In diesem Schema ist 3 die Verbindung.
Das ist schneller, als zweinotige parallele Sets zu benutzen, aber schwieriger.
Die allgemeine Regel für die Anwendung der parallelen Sets ist: \textbf{Konstruieren Sie das Übungssegment, indem Sie die größtmöglichen parallelen Sets benutzen, die zum Fingersatz passen.}
Unterteilen Sie diese nur in kleinere parallele Sets, wenn das große parallele Set zu schwierig ist.
\hyperlink{c1iii7}{Abschnitt III.7} behandelt die Einzelheiten zum Gebrauch der parallelen Sets.

Nachdem Sie ein Quadrupel gut spielen können, üben Sie, zwei hintereinander zu spielen, dann drei usw.
Schließlich werden Sie in der Lage sein, so viele hintereinander zu spielen wie Sie möchten!
Wenn Sie am Anfang den Akkord \enquote{gesprungen} haben, hat sich die Hand auf und ab bewegt.
Aber am Schluss, wenn Sie die Quadrupel in schneller Folge spielen, ist die Hand ziemlich stationär.
Sie werden auch Handbewegungen usw. hinzufügen müssen - dazu später mehr.

Der zweite schwierige Abschnitt in \enquote{Für Elise} endet mit einem Arpeggio, das drei parallele Sets enthält: 123, 135 und 432.
Üben Sie zunächst jedes parallele Set einzeln (zum Beispiel 123), fügen Sie dann die Verbindung (1231) hinzu, verbinden Sie sie dann paarweise (123135) usw., um das Arpeggio aufzubauen.

Damit jeder geübte Ausschnitt flüssig und musikalisch klingt, \textbf{müssen wir zwei Dinge vollbringen:}

\begin{enumerate}[label={\arabic*.}] 
\item \textbf{die Phasenwinkel genau kontrollieren (Unabhängigkeit der Finger), und}
\item \textbf{die parallelen Sets flüssig verbinden.}
 \end{enumerate}

Die meisten der in den Quellen beschriebenen Finger-, Hand- und Armbewegungen zielen darauf ab, diese beiden Aufgaben auf geschickte Art zu bewältigen.
Wir werden viele dieser Themen in \hyperlink{c1iii1}{Abschnitt III} behandeln.
Die Quellen sind nützliche Begleiter dieses Buchs.
Um Ihnen bei der Entscheidung zu helfen, welche der Quellen Sie benutzen sollten, habe ich im \hyperlink{reference}{Quellenverzeichnis} viele davon (sehr kurz) beschrieben.

Sie werden den größten Teil von \hyperlink{c1iii1}{Abschnitt III} lesen müssen, damit Sie wissen, wie man die parallelen Sets am effektivsten benutzt.
Das oben beschriebene parallele Spielen nennt man \enquote{phasengekoppeltes} paralleles Spielen und ist der einfachste Weg anzufangen aber nicht das endgültige Ziel.
Um sich die Technik anzueignen, brauchen sie eine vollständige Unabhängigkeit der Finger, die mit dem Üben kommt, keine phasengekoppelten Finger.
\textbf{Parallele Sets bewirken zweierlei: Sie lehren Ihr Gehirn das Konzept des extrem schnellen Spielens und geben den Händen eine Vorstellung davon, wie sich das schnelle Spielen anfühlt.}
Für diejenigen, die noch nicht derart schnell gespielt haben, sind das völlig neue und erstaunliche Erfahrungen.
Mit dem parallelen Spielen erreichen Sie die vorgesehene Geschwindigkeit, sodass Sie mit verschiedenen Bewegungen experimentieren können, um herauszufinden, welche funktionieren.
Weil diese Methoden Ihnen hunderte von Versuchen innerhalb von Minuten gestatten, können diese Experimente rasch ausgeführt werden.
 


