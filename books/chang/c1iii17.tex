% File: c1iii17

\subsection{Klaviere, Flügel und elektronische Klaviere; Kauf und Wartung}
\label{c1iii17}
 
\subsubsection{Flügel, akustisches oder elektronisches Klavier?}
\label{c1iii17a}

Flügel haben bestimmte Vorteile gegenüber \hyperref[upright]{Klavieren}\index{Klavieren}.
Diese Vorteile sind jedoch im Vergleich zur Wichtigkeit der Fertigkeitsstufe des Pianisten gering.
\textbf{Es gibt große Pianisten, die technisch fortgeschritten wurden und hauptsächlich auf Klavieren übten.
Es gibt keinen Beweis dafür, daß man für die anfängliche technische Entwicklung einen Flügel braucht}, obwohl es ein paar Klavierlehrer gibt, die darauf bestehen, daß jeder ernsthafte Schüler auf einem Flügel üben muß.
Ein Argument kann, zumindest für Anfänger, zur Bevorzugung der Klaviere angeführt werden, weil Klaviere ein festeres Spielen erfordern und vielleicht für die frühe Fingerentwicklung besser sind (man muß die Tasten fester herunterdrücken, um einen lauteren Ton zu erzeugen).
Sie sind eventuell sogar für Mittelstufenschüler überlegen, weil Klaviere weniger verzeihen und eine größere technische Fertigkeit erfordern.
Diese Argumente sind umstritten, aber sie zeigen, daß für Schüler bis zur Mittelstufe die Unterschiede zwischen Klavieren und Flügeln gegenüber anderen Faktoren wie die Motivation des Schülers, die Qualität der Lehrer, den Übungsmethoden und der richtigen Wartung des Klaviers von geringer Bedeutung sind.
Ein weiterer Faktor ist die Qualität: Gute Klaviere sind Flügeln von geringer Qualität (was die meisten Flügel einschließt, die kleiner als 5,2 ft = 1,58 m sind) überlegen.
Die Regel hinsichtlich aufrechter Klaviere ist einfach: Wenn Sie bereits eines besitzen, gibt es keinen Grund, sich davon zu trennen, bis Sie sich ein elektronisches Klavier oder einen Flügel kaufen; wenn Sie noch kein Klavier besitzen, gibt es keinen zwingenden Grund, ein aufrechtes Klavier zu kaufen.
Schüler oberhalb der Mittelstufe benötigen wahrscheinlich einen Flügel, weil technisch schwierige Musik auf den meisten aufrechten und elektronischen Klavieren viel schwerer (wenn nicht gar unmöglich) zu spielen ist.

Elektronische Klaviere unterscheiden sich grundlegend von akustischen (Flügel und Aufrechte).
Die Konstruktion ihrer Mechanik ist nicht so gut (nicht so teuer) und bei den meisten elektronischen Klavieren sind die Lautsprecher nicht gut genug, um mit den akustischen Klavieren zu konkurrieren.\footnote{Das war ein Grund, warum ich mich damals für ein Stage-Piano ohne Lautsprecher entschieden habe.
Es kostet weniger, beim Üben benutze ich fast immer einen Kopfhörer und zum Vorspielen die vorhandene Stereoanlage.}
Akustische Klaviere erzeugen den Klang deshalb auf grundlegend andere Weise, was dazu führt, daß viele Kritiker akustische Klaviere wegen der besseren Kontrolle des \enquote{Tons} bevorzugen.
Deshalb ist die Frage, welches Instrument das beste ist, komplex und hängt von den Umständen und den Anforderungen des Einzelnen ab.
Wir werden im folgenden jeden einzelnen Typ besprechen, so daß wir eine durchdachte Entscheidung darüber fällen können, welches Klavier für welchen Schüler das beste ist.

 
\subsubsection{Elektronische Klaviere}
\label{c1iii17b}

Heutige elektronische Klaviere sind guten Flügeln hinsichtlich der Entwicklung der Spieltechnik immer noch 
unterlegen, aber sie verbessern sich rapide.
Auch die besten elektronischen Klaviere sind für fortgeschrittene Klavierspieler ungeeignet; ihr mechanisches Ansprechverhalten ist schlechter, das musikalische Ergebnis und ihr Dynamikumfang sind unterlegen, und es wird schwierig, schnelles, technisch fortgeschrittenes Material auszuführen.
Die meisten preisgünstigeren Lautsprecher können nicht mit dem Resonanzboden eines Flügels konkurrieren.
Die elektronischen Klaviere gestatten nicht die Kontrolle des Klangs, der Farbe, des Pianissimo, des Staccato und der besonderen Manipulationsmöglichkeiten des Halte- und Dämpferpedals, die ein guter Flügel bietet.
Deshalb steht außer Frage, daß ein fortgeschrittener Klavierspieler einem Flügel den Vorzug vor einem elektronischen Klavier gibt; das stimmt jedoch nur unter der Voraussetzung, daß der Flügel mindestens zweimal im Jahr gestimmt und, wann immer es notwendig ist, eingestellt und \hyperref[c2_7_hamm]{intoniert}\index{intoniert} wird.
Die meisten aufrechten Klaviere bieten keinen genügenden Vorteil für die technische Entwicklung, um ihren Gebrauch gegenüber qualitativ guten elektronischen Klavieren, die ohne weiteres verfügbar und vergleichsweise preisgünstig sind und wenig in der Unterhaltung kosten, zu rechtfertigen.

Die elektronischen Klaviere haben einige besondere Vorteile, die wir im folgenden besprechen.
Wegen dieser Vorteile werden die meisten ernsthaften Klavierspieler sowohl ein akustisches als auch ein elektronisches Klavier besitzen.

\begin{enumerate}[label={\arabic*.}] 
%: 1
\item Für weniger als die Hälfte des Preises eines durchschnittlichen akustischen Aufrechten können Sie ein neues elektronisches Klavier mit allen notwendigen Eigenschaften kaufen, z.B. Kopfhöreranschluß, Lautstärkeregler, Anschlagsdynamik, Klänge für Orgel, Saiteninstrumente, Cembalo usw., Metronom, Aufnehmen, Midi-Anschlüsse, Analog-Ausgänge, Transposition, verschiedene Stimmungen und Begleitrhythmen.
Die meisten elektronischen Klaviere bieten viel mehr, aber das sind die minimalen Eigenschaften, die Sie erwarten können.
Das Argument, daß ein akustisches Klavier eine bessere Investition als ein elektronisches sei, ist falsch, weil ein akustisches Klavier keine gute Investition ist, besonders wenn es wesentlich mehr kostet und der anfängliche Wertverlust hoch ist.
Das elektronische Klavier erfordert keine Wartung, während die Wartungskosten eines akustischen beträchtlich sind, da es ungefähr zweimal jährlich gestimmt, intoniert und eingestellt sowie hin und wieder repariert werden muß.

%: 2
\item Elektronische Klaviere sind stets perfekt gestimmt.
Sehr junge Kinder, die genügend oft perfekt gestimmte Klaviere hören, erwerben automatisch ein \hyperref[c1iii12]{absolutes Gehör}\index{absolutes Gehör}, obwohl die meisten Eltern das nicht bemerken, weil es, wenn es nicht erkannt und gepflegt wird, in der Jugendzeit wieder verlorengeht.
Das akustische Klavier fängt an zu verstimmen, sobald der Stimmer Ihr Haus verläßt, und einige Noten werden die meiste Zeit aus der Stimmung sein (tatsächlich werden die meisten Noten die meiste Zeit aus der Stimmung sein).
Diese kleinen Abweichungen von der Stimmung werden den Erwerb des absoluten Gehörs jedoch nicht beeinflussen, solange das Klavier nicht stark verstimmt ist.
Da zu viele akustische Klaviere unzureichend gewartet sind, kann die Tatsache, daß die elektronischen Klaviere immer richtig gestimmt sind, ein großer Vorteil sein.
\textbf{Die Wichtigkeit eines gut gestimmten Klaviers für die musikalische und technische Entwicklung kann nicht überbetont werden, denn ohne die musikalische Entwicklung wird man nie lernen, \hyperref[c1iii14]{vorzuspielen und aufzutreten}\index{vorzuspielen und aufzutreten}.}
Der Klang eines elektronischen Klaviers kann in hohem Maß verbessert werden, indem man es an einen guten Verstärker mit guten Lautsprechern anschließt.

%: 3
\item Sie können Kopfhörer benutzen oder die Lautstärke so einstellen, daß Sie beim Üben niemand anderen stören.
Die Möglichkeit, die Lautstärke herunterzudrehen, ist auch zum Vermindern von \hyperref[c1iii10gehoer]{Gehörschäden}\index{Gehörschäden} beim Üben lauter Passagen nützlich: ein wichtiger Faktor für jeden, der älter als 60 Jahre ist; ein Alter, in dem viele unter einsetzendem Hörverlust oder Tinnitus leiden.
Wenn man ein fortgeschrittener Spieler ist, erzeugt auch ein elektronisches Klavier (trotz abgeschalteter Lautsprecher) ein erhebliches \enquote{Spielgeräusch}, das in unmittelbarer Nähe ziemlich laut sein kann, und diese Vibrationen können durch den Boden in die unter dem Klavier liegenden Räume übertragen werden.
Deshalb ist es ein Fehler zu glauben, daß die Geräusche eines elektronischen Klaviers (oder eines akustischen \enquote{Silent}-Klaviers) völlig eliminiert werden können.

%: 4
\item Sie sind viel leichter zu transportieren als akustische Klaviere.
Obwohl es leichte Keyboards mit ähnlichen Eigenschaften gibt, ist es für das Klavierüben am besten, ein schwereres elektronisches Klavier zu benutzen, damit es sich beim Spielen von schneller lauter Musik nicht bewegt.
Auch diese schwereren elektronischen Klaviere können leicht von zwei Personen getragen werden und passen in viele Autos.

%: 5
\item \textbf{Ein variables Spielgewicht ist wichtiger als vielen bewußt ist.}
Man muß jedoch wissen, was \enquote{Spielgewicht} bedeutet, bevor man es vorteilhaft einsetzen kann; \hyperref[touchweight]{Details}\index{Details} finden Sie weiter unten.
Im allgemeinen ist das Spielgewicht eines elektronischen Klaviers etwas geringer als das eines akustischen.
Das leichtere Gewicht wurde aus zwei Gründen gewählt: um es Keyboard-Spielern einfacher zu machen, diese elektronischen Klaviere zu spielen (das Spielgewicht von Keyboards ist noch geringer), und um es im Vergleich zu akustischen Klavieren einfacher zu machen, sie zu spielen.
Der Nachteil des leichteren Gewichts ist, daß man es eventuell etwas schwieriger findet, auf einem akustischen Klavier zu spielen, nachdem man auf einem elektronischen geübt hat.
Das Spielgewicht eines akustischen Klaviers muß höher sein, um einen volleren Klang zu erzeugen.
Ein Vorteil des höheren Gewichts ist, daß man die Tasten eines akustischen Klaviers während des Spielens erfühlen kann, ohne aus Versehen falsche Tasten zu drücken.
Das kann jedoch auch zu nachlässigem Spielen mit einigen ungewollten Fingerbewegungen führen, weil man die Tasten eines akustischen Klaviers leicht anschlagen kann, ohne einen Ton zu erzeugen.
Man kann üben, diese unkontrollierten Bewegungen loszuwerden, indem man ein elektronisches Klavier benutzt und ein leichtes Spielgewicht auswählt, so daß ein ungewollter Anschlag einen Ton erzeugt.
Viele Menschen, die nur auf akustischen Klavieren üben, wissen nicht einmal, daß sie solche unkontrollierten Bewegungen haben, bis sie versuchen, auf einem elektronischen Klavier zu spielen, und feststellen, daß sie ziemlich viele zusätzliche Tasten anschlagen.
Der leichte Anschlag ist auch für das schnelle Erwerben schwieriger Technik nützlich.
Wenn man später auf einem akustischen Klavier spielen muß, kann man mit erhöhtem Gewicht üben, nachdem man die Technik bereits erworben hat.
Dieser zweistufige Prozeß ist gewöhnlich schneller, als wenn man versucht, sich die Technik bei einem hohen Spielgewicht anzueignen.

%: 6
\item Klaviermusik \hyperref[c1iii13]{aufzunehmen}\index{aufzunehmen} ist mit einer konventionellen Ausrüstung eine der schwierigsten Aufgaben.
Mit einem elektronischen Klavier geht das \enquote{auf Knopfdruck}!
Man kann leicht ein Album mit allen gelernten Stücken aufbauen.
Aufzunehmen ist nicht nur eine der besten Möglichkeiten, Ihre Stücke wirklich zu vollenden und auf Hochglanz zu polieren, sondern auch um zu lernen, \hyperref[c1iii14]{wie man für ein Publikum spielt}\index{wie man für ein Publikum spielt}.
Jeder sollte es sich vom ersten Tag des Unterrichts an zur Gewohnheit machen, jedes fertige Stück aufzunehmen.
Selbstverständlich werden die ersten Vorträge nicht perfekt sein, so daß Sie die Stücke wahrscheinlich noch einmal aufnehmen, wenn Sie besser geworden sind.
Zu viele Schüler nehmen ihre Stücke niemals auf, was der Hauptgrund für übermäßige \hyperref[c1iii15]{Nervosität}\index{Nervosität} und Schwierigkeiten während des Vorspielens ist.

%: 7
\item Die meisten Klavierspieler, die gute Übungsmethoden befolgen und das Klavierspielen in jungen Jahren beherrschen, komponieren irgendwann ihre eigene Musik.
Elektronische Klaviere sind beim Aufnehmen dieser Kompositionen hilfreich, so daß man sie nicht aufschreiben muß, und dafür, sie mit verschiedenen für die jeweilige Komposition geeigneten Instrumenten zu spielen.
Mit etwas zusätzlicher Software oder Hardware kann man ganze Symphonien komponieren und jedes Instrument selbst spielen.
Es gibt sogar Software, die Ihre Musik (wenn auch nicht perfekt) in ein Notat umwandelt.
Es gibt jedoch nichts hilfreicheres für das Komponieren als ein qualitativ hochwertiger Flügel -- der Klang eines guten Klaviers ist eine Inspiration für den Kompositionsprozeß; wenn Sie ernsthaft komponieren, werden deshalb die meisten elektronischen Klaviere ungenügend sein.

%: 8
\item Wenn Sie sich die Technik schnell aneignen können, hält Sie nichts davon ab, Ihren Horizont jenseits der klassischen Musik zu erweitern und Pop, Jazz, Blues usw. zu spielen.
Sie werden ein breiteres Publikum ansprechen, wenn Sie die Musikgenres mischen können, und es wird Ihnen mehr Spaß machen.
Die im elektronischen Klavier verfügbaren Begleitrhythmen, Schlagzeuge usw. können bei diesen Arten der Musik hilfreich sein.
Deshalb können diese zusätzlichen Fähigkeiten der elektronischen Klaviere sehr nützlich sein und sollten nicht ignoriert werden.
Elektronische Klaviere sind auch für Auftritte leichter zu transportieren.

%: 9
\item Ein elektronisches Klavier zu kaufen ist ziemlich einfach, besonders wenn man es mit dem \hyperref[c1iii17e]{Kauf eines akustischen Klaviers}\index{Kauf eines akustischen Klaviers} vergleicht (s.u.).
Alles, was Sie wissen müssen, ist Ihre Preisspanne, die benötigte Ausstattung und den Hersteller.
Sie brauchen keinen erfahrenen Klaviertechniker, der Ihnen bei der Bewertung des Klaviers hilft.
Es stellt sich nicht die Frage, ob der Klavierhändler alle vorbereitenden Arbeiten am Klavier ausgeführt hat bzw. ausführen ließ, ob der Händler die Vereinbarung einhält, dafür zu sorgen, daß das Klavier nach der Lieferung einwandfrei funktioniert, ob das Klavier während des ersten Jahres richtig \enquote{stabilisiert} wurde oder ob Sie eines mit gutem oder minderwertigem Klang und Anschlag bekommen haben.
Viele renommierte Hersteller, wie Yamaha, Roland, Korg, Technics, Kawai und Kurzweil, produzieren elektronische Klaviere exzellenter Qualität.

%: 10
\item Und das ist nur der Anfang; die elektronischen Klaviere verbessern sich im Laufe der Zeit sprunghaft.
Eine neue Entwicklung ist die Modellierung des Klaviers (z.B. von Pianoteq, siehe www.pianoteq.com), anstelle des Samplens, das zuvor benutzt wurde.
Ein gutes Sampling erfordert eine enorme Menge an Speicher und Rechenleistung, was das Ansprechverhalten des Klaviers verlangsamen kann.
Die Modellierung ist vielseitiger und gestattet Dinge, die man nicht einmal auf einem Flügel machen kann, wie z.B. teilweise getretenes Dämpferpedal, Kontrolle der Biegung des Hammerschafts oder auf Chopins Pleyel zu spielen.

%: 11
\item Wir sollten von der \hyperref[c2_6_et]{gleichschwebenden Stimmung}\index{gleichschwebenden Stimmung} zu den \hyperref[c2_2_wtk2]{wohltemperierten Stimmungen}\index{wohltemperierten Stimmungen} (siehe im \hyperref[c2_1]{Kapitel über das Klavierstimmen}) übergehen.
Wenn Sie sich dazu entschlossen haben, die wohltemperierten Stimmungen zu verwenden, benötigen Sie mehrere davon.
Zu lernen, die Farbe einer Tonart zu erkennen und hervorzubringen ist eine sehr wertvolle Fertigkeit.
Die gleichschwebende Stimmung ist dafür am wenigsten geeignet.
Bei elektronischen Klavieren können Sie die meisten der verbreiteten wohltemperierten Stimmungen erhalten.
\end{enumerate}

\label{touchweight}

Das dynamische Spielgewicht (touch weight) eines Klaviers wird nicht einfach dadurch verändert, daß man Bleigewichte von den Tasten entfernt oder ihnen hinzufügt, um die zum Niederdrücken der Tasten notwendige Kraft zu verändern.
Das dynamische Spielgewicht ist eine Kombination aus dem statischen Spielgewicht (down weight) zur Überwindung der Trägheit der Taste und des Hammers, und der Kraft, die notwendig ist, um einen Ton mit einer bestimmten Lautstärke zu erzeugen.
Das statische Spielgewicht ist das maximale Gewicht, dem die Taste widersteht, bevor sie anfängt, sich abwärts zu bewegen.
Das ist das Gewicht, das mit Bleigewichten justiert wird.
Das statische Spielgewicht aller Klaviere, einschließlich der elektronischen mit \enquote{gewichteter Tastatur}, beträgt in der Regel ungefähr 50 Gramm und variiert geringfügig von Klavier zu Klavier, unabhängig vom dynamischen Spielgewicht.
Wenn man ein Klavier spielt, sind diese 50 Gramm ein kleiner Teil der Kraft, die zum Spielen erforderlich ist -- der größte Teil der Kraft wird zum Erzeugen des Tons benutzt.
Bei akustischen Klavieren ist das die Kraft, die notwendig ist, um den Hammer auf Geschwindigkeit zu bringen.
In elektronischen Klaviere ist es die elektronische Reaktion auf die Tastenbewegung und ein fester mechanischer Widerstand.
In beiden Fällen muß man, zusätzlich zum Aufbringen der für die Erzeugung des Tons notwendigen Kraft, auch die Trägheit des Mechanismus überwinden.
Wenn man z.B. staccato spielt, wird der größte Teil der Kraft zur Überwindung der Trägheit benötigt, während beim Legatospiel die Trägheitskomponente klein ist.
Elektronische Klaviere haben eine kleinere Trägheitskomponente, weil sie nur die Trägheit der Tasten haben, während die akustischen Klaviere zusätzlich die Trägheit der Hämmer haben; das macht die akustischen weniger empfindlich für das versehentliche Drücken von Tasten.
Deshalb werden Sie den größten Unterschied zwischen akustischen und elektronischen Klavieren fühlen, wenn Sie schnell oder staccato spielen und wenig Unterschied, wenn Sie legato spielen.
Für den Klavierspieler ist das dynamische Spielgewicht nur die Kraft, die erforderlich ist, um eine bestimmte Lautstärke des Tons zu erzeugen und hat wenig mit dem statischen Spielgewicht zu tun.
Bei akustischen Klavieren wird das dynamische Spielgewicht hauptsächlich von der Masse und dem \hyperref[c2_7_hamm]{Intonieren der Hämmer}\index{Intonieren der Hämmer} (Härte) bestimmt.
Es gibt nur einen schmalen Bereich der Hammermassen, der ideal ist, weil man schwerere Hämmer für einen stärkeren Klang aber leichtere Hämmer für eine schnelle Mechanik möchte.
Deshalb kann ein großer Teil des dynamischen Spielgewichts vom Klaviertechniker eher durch das Intonieren der Hämmer als durch das Ändern der Gewichte justiert werden.
Bei elektronischen Klavieren wird das dynamische Spielgewicht auf folgende Weise durch die Software kontrolliert, um zu simulieren, was in einem Flügel geschieht.
Für ein höheres dynamisches Spielgewicht wird der Klang auf den eines weicheren Hammers umgeschaltet und umgekehrt.
Es erfolgt keine mechanische Veränderung des statischen Spielgewichts der Tasten oder der Trägheitskomponente.
Wenn Sie auf das höchste Spielgewicht umschalten, werden Sie deshalb den Klang eventuell als gedämpft empfinden, und wenn Sie auf das geringste Gewicht umschalten, könnte der Klang zu schrill sein.
Bei elektronischen Klavieren ist es einfacher, das dynamische Spielgewicht zu vermindern, ohne den Klang nachteilig zu beeinflussen, weil keine Hämmer bewegt werden müssen.
Auf der anderen Seite wird der maximale Dynamikumfang der meisten elektronischen Klaviere durch die Elektronik und die Lautsprecher begrenzt, so daß der Flügel für die lautesten Töne ein geringeres dynamisches Spielgewicht haben kann.
\textbf{Zusammengefaßt ist das dynamische Spielgewicht ein subjektives Urteil des Klavierspielers darüber, wieviel Kraft notwendig ist, um eine bestimmte Lautstärke zu erzeugen; es ist nicht das feste Gewicht oder der Widerstand der Tasten gegenüber dem Anschlag.}

Man kann diese subjektive Beurteilung demonstrieren, indem man die Lautstärke eines elektronischen Klaviers hoch- oder herunterdreht.
Wenn man längere Zeit auf einem elektronischen Klavier mit heruntergedrehter Lautstärke übt und dann auf einem  akustischen Klavier spielt, kann sich das akustische geradezu leicht anfühlen.
Leider sind die Dinge etwas komplizierter, denn wenn man ein elektronisches Klavier auf ein höheres Spielgewicht umschaltet, erzeugt es den Klang eines weicheren Hammers.
Um den Klang eines richtig intonierten Hammers zu erzeugen, muß man härter anschlagen.
Das ergibt zusammen die Wahrnehmung des höheren Spielgewichts, und dieser Effekt kann nicht durch ein Drehen am Lautstärkeregler simuliert werden.
Anhand dieser Überlegungen können wir folgende Schlüsse ziehen:
Es gibt geringe Unterschiede im Spielgewicht zwischen Flügeln und elektronischen Klavieren, wobei das der Flügel meistens höher ist, aber diese Unterschiede reichen nicht aus, um größere Probleme zu bereiten, wenn man vom einen zum anderen wechselt.
Deshalb ist die Befürchtung, daß das Üben auf einem elektronischen Klavier es erschwert, auf einem Flügel zu spielen, unbegründet; in Wahrheit ist es wahrscheinlich sogar einfacher, obwohl es eventuell ein paar Minuten dauern kann, bis man sich an das Spielen auf dem Flügel gewöhnt hat.

\textbf{Wenn Sie ein Anfänger sind und Ihr erstes Klavier kaufen möchten, ist ein\footnote{qualitativ gutes} elektronisches Klavier die offensichtliche Wahl, es sei denn, Sie können sich einen qualitativ guten Flügel leisten und haben den Platz dafür.}
Sogar in diesem Fall möchten Sie wahrscheinlich ein elektronisches Klavier, weil die Kosten im Vergleich zu einem Flügel gering sind und es Ihnen Ausstattungsmerkmale bietet, die ein Flügel nicht hat.
Die meisten akustischen aufrecht stehenden Klaviere sind nun obsolet, außer Sie sind bereit, Preise zu zahlen, die mit denen eines guten Flügels vergleichbar sind.


\subsubsection{Klaviere}
\label{c1iii17c}

Akustische Klaviere haben ihre eigenen Vorteile.
Sie sind weniger teuer als Flügel.
Sie benötigen weniger Platz, und für kleine Räume erzeugen große Flügel unter Umständen zuviel Schall, so daß sie bei ganz geöffnetem Deckel nicht mit voller Lautstärke gespielt werden können, ohne daß die \hyperref[c1iii10gehoer]{Ohren schmerzen oder geschädigt}\index{Ohren schmerzen oder geschädigt} werden.
Die elektronischen Klaviere haben jedoch dieselben, sowie viele weitere, Vorteile.
Besitzer eines Klaviers vernachlässigen zu häufig das \hyperref[c2_7_hamm]{Intonieren der Hämmer}\index{Intonieren der Hämmer} völlig, da dieses Vernachlässigen zu einem stärkeren Ton führt.
Da Klaviere im Grunde geschlossene Instrumente sind, ist das Vernachlässigen des Intonierens weniger wahrnehmbar.
Klaviere sind meistens weniger teuer in der Wartung, hauptsächlich weil teure Reparaturen sich nicht lohnen und deshalb nicht durchgeführt werden.
Selbstverständlich gibt es qualitativ hochwertige Klaviere, die im Spielgefühl und in der Klangqualität mit Flügeln vergleichbar sind; aber ihre Zahl ist klein.

Unter den Klavieren sind die Kleinklaviere die mit der geringsten Höhe und im allgemeinen die billigsten; die meisten erzeugen keinen zufriedenstellenden Klang, auch für Schüler.
Die geringe Höhe der Kleinklaviere begrenzt die Saitenlänge, was die hauptsächliche Begrenzung der Schallerzeugung ist.
Theoretisch sollte der Diskantbereich einen ausreichenden Schall erzeugen (es gibt auch bei Kleinklavieren keine Einschränkung der Saitenlänge), aber die meisten Kleinklaviere sind im Diskant wegen der schlechten Qualität der Konstruktion schwach; testen Sie deshalb unbedingt die höheren Noten, wenn Sie ein Kleinklavier beurteilen -- vergleichen Sie es einfach mit einem größeren Klavier.
Konsolenklaviere oder größere Klaviere können sehr gute Schülerklaviere sein.
Alte Klaviere mit schlechtem Klang sind im allgemeinen nicht zu retten, egal wie groß sie sind.
In einem solchen Alter ist der Wert des Klaviers geringer als die Kosten für das Restaurieren; es ist billiger, ein neueres Klavier mit einem zufriedenstellenden Klang zu kaufen.
\textbf{Die meisten Klaviere wurden von den elektronischen Klavieren überholt.
Deshalb gibt es keinen Grund, ein neues Klavier zu kaufen, obwohl einige Klavierlehrer und die meisten Verkäufer in den Klaviergeschäften etwas anderes vorschlagen.}
Viele Klavierlehrer haben nicht genug Erfahrung mit elektronischen Klavieren, sind mit dem Gefühl und dem Klang der akustischen Klaviere vertrauter und neigen dazu, die akustischen als \enquote{richtige Klaviere} zu empfehlen, was im allgemeinen ein Fehler ist.
Der geringe Unterschied im \enquote{Klang} (wenn es ihn gibt) wiegt die Schwierigkeit des Kaufs eines qualitativ guten Klaviers, die Probleme, die man oft damit hat, es vor und nach der Auslieferung richtig \enquote{herzurichten}, und die Notwendigkeit, es reguliert und in der Stimmung zu halten, nicht auf.


\subsubsection{Flügel}
\label{c1iii17d}

Die Vorteile der meisten Flügel sind: größerer Dynamikbereich (laut / leise), offene Struktur, die dem Schall gestattet, frei zu entweichen (was mehr Kontrolle und Ausdruck bietet), vollerer Klang, schnellere Repetierung, weichere Mechanik (Benutzung der Schwerkraft statt Federn), ein \enquote{wahres} Dämpferpedal (\hyperref[c1ii24]{s. \autoref{c1ii24}}), klarerer Klang (leichter exakt zu stimmen) und eine eindrucksvollere Erscheinung.
Eine Ausnahme bildet die Klasse der Stutzflügel (kleiner als ca. 5'-2``\footnote{ca. 1,57m}), deren erzeugter Klang üblicherweise nicht zufriedenstellend ist und die hauptsächlich als dekorative Möbelstücke gesehen werden sollten.
Ein paar Firmen (Yamaha, Kawai) beginnen damit, Stutzflügel mit akzeptablem Klang zu produzieren.\footnote{Ich habe die hier und im folgenden genannten Firmen jeweils ohne Prüfung der aktuellen Gegebenheiten aus dem Originaltext übernommen.
Falls sich eine Firma dadurch nicht angemessen berücksichtigt sieht, kann sie gerne mit dem Autor oder mir \hyperref[kontakt]{Kontakt}\index{Kontakt} aufnehmen.}
Sie sollten also diese sehr neuen Flügel nicht abschreiben, ohne sie getestet zu haben.
Größere Flügel können in zwei Hauptklassen unterteilt werden: die \enquote{Schülerflügel}  (kleiner als ca. 6 bis 7 ft\footnote{ca. 1,83m -- 2,13m}) und die Konzertflügel.
Die Konzertflügel bieten einen größeren Dynamikbereich, bessere Klangqualität und mehr Tonkontrolle.

Nehmen wir die Steinway-Flügel als ein Beispiel für dieses Thema \enquote{Qualität gegen Größe}:

\begin{itemize} 
\item Das Stutzflügelmodell, Modell S (5'-2``\footnote{ca. 1,57m}), ist im Grunde ein dekoratives Möbelstück, und sehr wenige erzeugen einen qualitativ genügenden Klang, um als spielbar angesehen zu werden, und sind vielen Klavieren unterlegen.
\item Die nächste Größengruppe besteht aus den Modellen M, O und L (5'-7`` bis 5'-11``\footnote{ca. 1,57m -- 1,80m}).
Diese Modelle sind einander ziemlich ähnlich und exzellente Schülerklaviere.
Fortgeschrittene Klavierspieler würden sie jedoch wegen des geringeren Sustains, des zu stark perkussiven Klangs und den Noten mit zu hohem harmonischen Gehalt nicht als wahre Flügel betrachten.
\item Das nächste Modell, A (6'-2``\footnote{ca. 1,88m}), ist ein Grenzfall.
\item B (6'-10``\footnote{ca. 2,08m}), C (7'-5``\footnote{ca. 2,26m}) und D (9'\footnote{ca. 2,74m}) sind richtige Flügel.
\end{itemize}

Ein Problem beim Beurteilen von Steinways ist, daß die Qualität innerhalb eines Modells sehr unterschiedlich ist; im Durchschnitt gibt es jedoch mit jeder Steigerung in der Größe eine deutliche Verbesserung der Klangqualität und -stärke.

Flügel erfordern ein häufigeres \hyperref[c2_7_hamm]{Intonieren der Hämmer}\index{Intonieren der Hämmer} als Klaviere; sonst werden sie zu \enquote{brillant} oder \enquote{schrill}, und an diesem Punkt spielen die meisten Besitzer schließlich nur noch bei geschlossenem Deckel.
Viele Besitzer vernachlässigen das Intonieren völlig.
Das Ergebnis ist, daß solche Flügel zu viel und einen zu schrillen Klang erzeugen und deshalb mit geschlossenem Deckel gespielt werden.
Es ist aus technischer Sicht nichts falsch daran, einen Flügel mit geschlossenem Deckel zu spielen.
Einige Puristen werden jedoch über eine solche Praxis bestürzt sein, und man verschenkt sicherlich etwas wundervolles, für das man eine bedeutende Investition getätigt hat.
Vorführungen bei Konzerten erfordern fast immer, daß der Deckel offen ist, was dazu führt, daß der Flügel empfindlicher reagiert.
Deshalb sollten Sie vor einem Auftritt immer mit offenem Deckel üben, auch wenn Sie normalerweise bei geschlossenem Deckel üben.
In einem größeren Raum oder in einer Konzerthalle gibt es jedoch viel weniger Mehrfachreflexionen der Töne, so daß man nicht den ohrenbetäubenden Lärm hört, der in einem kleinen Raum daraus resultieren kann.
Eine Konzerthalle wird den Schall des Klaviers absorbieren, so daß man, wenn man gewohnt ist, in einem kleinen Raum zu üben, in einer Konzerthalle Schwierigkeiten haben wird, sein eigenes Spielen zu hören.

Einer der größten Vorteile von Flügeln ist die Ausnutzung der Schwerkraft als Kraft für das Zurückstellen der Hämmer.
Bei Klavieren wird die Rückstellkraft von Federn zur Verfügung gestellt.
Die Schwerkraft ist immer konstant und über die ganze Tastatur hinweg gleichförmig, während Ungleichmäßigkeiten in den Federn und Reibung Ungleichmäßigkeiten in dem Gefühl für die Tasten eines Klaviers erzeugen können.
Gleichmäßigkeit im Gefühl ist eine der wichtigsten Eigenschaften von gut eingestellten Qualitätsklavieren.
Viele Schüler sind von der Erscheinung großer Flügel bei Konzerten und Wettbewerben eingeschüchtert, aber diese Flügel sind in Wahrheit leichter zu spielen als Klaviere.
Eine Furcht, die diese Schüler in bezug auf jene Flügel haben, ist, daß deren Mechanik schwerer sei.
Das Spielgewicht wird jedoch durch den Techniker, der das Piano einstellt, reguliert und kann sowohl bei einem Flügel als auch bei einem Klavier auf jeden Wert eingestellt werden.
Fortgeschrittene Schüler werden es natürlich leichter finden, anspruchsvolle Stücke auf einem Flügel als auf einem Klavier zu spielen; hauptsächlich wegen der schnelleren Mechanik und der Gleichmäßigkeit.
Folglich können Flügel eine Menge Zeit sparen, wenn man versucht, fortgeschrittene Fertigkeiten zu erwerben.
Der Hauptgrund dafür ist, daß es leicht ist, schlechte Angewohnheiten zu entwickeln, wenn man auf Klavieren mit schwierigem Material kämpft.
Anspruchsvolles Material ist auf elektronischen Klavieren sogar noch schwieriger (und bei Modellen ohne richtiges Spielgewicht unmöglich), weil sie nicht die Robustheit und das Ansprechverhalten auf den Anschlag haben, die bei höheren Geschwindigkeiten erforderlich sind.

Einige Menschen mit kleinen Räumen zermartern sich den Kopf darüber, ob ein großer Flügel an so einem Platz zu laut wäre.
Lautstärke ist üblicherweise nicht das wichtigste Thema, und Sie haben immer die Option, den Deckel in unterschiedlichem Maß zu schließen.
Die maximale Lautstärke von mittleren und großen Flügeln ist nicht so unterschiedlich, und man kann mit den größeren Flügeln leiser spielen.
Es sind die Mehrfachreflexionen, die am lästigsten sind.
Mehrfachreflexionen können leicht durch einen Teppich auf dem Boden und durch die Schalldämmung einer oder zweier Wände eliminiert werden.
Wenn das Klavier von der Größe her ohne offensichtliche Schwierigkeiten in den Raum paßt, dann kann es deshalb hinsichtlich des Schalls akzeptabel sein.



