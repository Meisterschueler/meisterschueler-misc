% File: c1iii3

\subsection{Triller und Tremolos (Beethovens Pathétique, 1. Satz)}
\label{c1iii3}

\subsubsection{Triller}

\textbf{Es gibt nichts besseres, um die Wirksamkeit der \hyperref[c1iii7b]{Übungen für parallele Sets} (siehe III.7b) zu demonstrieren, als sie zu benutzen, um den Triller zu lernen.
Es gibt hauptsächlich zwei Probleme zu lösen, um zu trillern: Geschwindigkeit (mit Kontrolle) und so lange weiter zu machen, wie man möchte.}
Die Übungen für parallele Sets wurden entwickelt, um genau diese Art von Problemen zu lösen, und funktionieren deshalb beim Üben von Trillern sehr gut.
\hyperref[Whiteside]{Whiteside} beschreibt eine Methode, um den Triller zu üben, die eine Art von \hyperref[c1ii9]{Akkord-Anschlag} ist.
Somit ist es nichts Neues, den Akkord-Anschlag für das Üben des Trillers zu benutzen.
Da wir nun jedoch den Lernmechanismus detaillierter verstehen, können wir die direkteste und effektivste Vorgehensweise entwickeln, indem wir parallele Sets benutzen.

Das erste zu lösende Problem sind die ersten beiden Noten.
\textbf{Wenn man die ersten beiden Noten nicht richtig anfängt, wird das Lernen des Trillers zu einer schwierigen Aufgabe.
Die Wichtigkeit der ersten beiden Noten betrifft auch Läufe, Arpeggios usw.}
Aber die Lösung ist fast trivial - wenden Sie die \hyperref[c1iii7b2]{Übung für 2-notige parallele Sets} an.
Nehmen Sie deshalb für einen mit 2323 usw. gespielten Triller die erste 3 als die Verbindung und üben Sie 23.
Üben Sie danach die 32, dann 232 usw.
So einfach ist es! Versuchen Sie es! Es funktioniert zauberhaft!
Sie sollten eventuell die Abschnitte \hyperref[c1iii7b]{\autoref{c1iii7b}} und \hyperref[c1iii7c]{\autoref{c1iii7c}} lesen, bevor Sie die parallelen Sets auf den Triller anwenden.

\textbf{Der Triller besteht aus zwei Bewegungen: Einer Bewegung der Finger und aus der Drehung des Unterarms.
Üben Sie deshalb die beiden Fertigkeiten getrennt.}
Benutzen Sie zunächst nur die Finger für das Trillern, halten Sie die Hand und den Arm völlig ruhig.
Halten Sie dann die Finger still, und trillern Sie nur mit einer Drehung des Arms.
So finden Sie heraus, ob Sie von den Fingern oder der Armdrehung gebremst werden.
Viele Schüler haben nie die schnelle Armdrehung (Wiegen des Arms) geübt, und das wird oft die langsamere Bewegung sein.
Bei schnellen Trillern ist dieses Vor- und Zurückdrehen verschwindend gering aber notwendig.
Wenden Sie die \hyperref[c1iii7b]{Übungen für parallele Sets} sowohl auf die Fingerbewegungen als auch die Armdrehungen an.
Übertreiben Sie die Bewegungen bei langsamen Trillern, und steigern Sie die Geschwindigkeit, indem Sie das Ausmaß der Bewegung reduzieren.
Das endgültige Maß der beiden Bewegungen muss nicht identisch sein, da Sie für die langsamere Bewegung (Armdrehung) ein kleineres Ausmaß benutzen werden, um ihre Langsamkeit zu kompensieren.
Wenn Sie diese Bewegungen üben, experimentieren Sie mit verschiedenen Fingerhaltungen.
Sehen Sie dazu auch den Abschnitt über \hyperref[c1iii3b]{Tremolos}, bei denen ähnliche Methoden angewandt werden - der Triller ist nur ein verkürztes Tremolo.

Wegen der Notwendigkeit des schnellen Impulsausgleichs \textbf{ist \hyperref[c1ii14]{Entspannung} für den Triller sogar noch wichtiger als fast jede andere Technik}, das heißt da die parallelen Sets nur aus zwei Noten bestehen, gibt es zu viele Verbindungen, als dass wir uns nur auf die Parallelität verlassen könnten, um auf Geschwindigkeit zu kommen.
Deshalb müssen wir in der Lage sein, den Impuls der Finger schnell zu ändern.
Bei Trillern muss die Armdrehung dem Impuls der Finger entgegenwirken.
Stress bindet die Finger an die größeren Glieder wie Handfläche und Hand und vergrößert somit die effektive Masse der Finger.
Größere Masse bedeutet langsamere Bewegung: Denken Sie an die Tatsache, dass der Kolibri schneller mit seinen Flügeln schlagen kann als der Kondor und kleinere Insekten sogar schneller als der Kolibri.
Das ist sogar dann wahr, wenn der Luftwiderstand ignoriert wird; tatsächlich ist die Luft für den Kolibri effektiv viskoser als für den Kondor, und für ein kleines Insekt ist die Luft fast so viskos, wie es Wasser für einen großen Fisch ist; trotzdem können Insekten ihre Flügel rasch bewegen, weil die Flügelmasse so klein ist.
Es ist deshalb wichtig, von Anfang an die völlige Entspannung in den Triller einzubeziehen und somit die Finger von der Hand zu befreien.
\textbf{Trillern ist eine Fertigkeit, die dauernde Pflege erfordert.
Wenn man ein guter \enquote{Trillerer} sein möchte, dann muss man das Trillerspielen täglich üben.}
Die \hyperref[c1iii7b2]{Übung \#2 für parallele Sets} (zwei Noten) ist die beste Prozedur, um den Triller in guter Verfassung zu halten, besonders wenn man ihn eine Weile nicht benutzt hat oder ihn weiter verbessern möchte.

Der Triller ist keine Reihe von Staccatos.
Die Fingerspitzen müssen so lange wie möglich am Ende des Anschlags sein, das heißt die Fänger müssen bei jeder Note eingegriffen haben.
Beachten Sie sorgfältig das minimale Anheben, das notwendig ist, damit die Repetierung funktioniert\footnote{das heißt damit die Mechanik wieder in die Ausgangsstellung geht}.
Wer üblicherweise auf einem Flügel übt, sollte sich darüber im Klaren sein, dass die Strecke für das Anheben bei einem Klavier fast doppelt so hoch sein kein.
Schnellere Triller erfordern ein geringeres Anheben; deshalb muss man auf einem Klavier den Triller eventuell verlangsamen.
Schnelle Triller sind auf elektronischen Klavieren schwierig, weil deren Mechanik schlechter ist.


\subsubsection{Tremolos (Beethovens Pathétique, 1. Satz)}
\label{c1iii3b}

Tremolos werden genau auf die gleiche Art geübt wie Triller.
Lassen Sie uns dies auf die manchmal gefürchteten, langen Oktavtremolos von Beethovens Pathétique-Sonate (Opus 13) anwenden.
Für einige Schüler scheinen diese Tremolos unmöglich zu sein, und viele haben sich beim Üben die Hände verletzt, manche davon durch zu viel Üben dauerhaft.
Andere haben wenig Schwierigkeiten.
Wenn man weiß, wie man sie üben muss, sind sie in Wahrheit ziemlich einfach.
\textbf{Das letzte, was man tun möchte, ist, stundenlang diese Tremolos in der Hoffnung zu üben, Ausdauer aufzubauen - das ist der sicherste Weg, \hyperref[c1ii22]{schlechte Angewohnheiten} zu erwerben und \hyperref[c1iii10hand]{Verletzungen} zu erleiden.}

Da Sie die Oktavtremolos bei beiden Händen benötigen, werden wir mit der linken und der rechten Hand abwechselnd üben; wenn die rechte es schneller begreift, können Sie sie \hyperref[c1ii20]{benutzen, um die linke zu unterrichten}.
Ich werde Ihnen eine Folge von Übungsmethoden vorschlagen; mit ein wenig Phantasie sollten Sie in der Lage sein, sich Ihre eigene Folge zu erstellen, die vielleicht besser für Sie ist - mein Vorschlag dient nur der Illustration.
Aus Gründen der Vollständigkeit ist er zu detailliert und zu lang.
Sie sollten die Übungsfolge je nach Ihren spezifischen Bedürfnissen und Schwächen kürzen können.

Um das \hyperref[Noten]{C2-C3}-Tremolo zu üben, üben Sie zunächst die C2-C3-Oktave (linke Hand).
Lassen Sie die Hand leicht \hyperref[c1ii9]{hoch- und runterspringen}, wiederholen Sie die Oktave mit Betonung der \hyperref[c1ii14]{Entspannung} - können Sie die Oktave ohne Ermüdung oder Stress weiterspielen, besonders wenn Sie schneller werden?
Wenn Sie müde werden, finden Sie Möglichkeiten, die Oktave zu wiederholen, ohne Ermüdung zu entwickeln, indem Sie Ihre Handposition, -bewegung usw. ändern.
Sie könnten zum Beispiel das Handgelenk schrittweise anheben und dann wieder senken - so können Sie für jedes Quadrupel vier verschiedene Handpositionen benutzen.
Wenn Sie immer noch müde werden, hören Sie auf, und wechseln Sie die Hand; üben Sie mit der rechten Hand die Ab4-Ab5-Oktave, die Sie später benötigen werden.
Wenn Sie die Oktave viermal je Schlag (einschließlich des korrekten \hyperref[c1iii1b]{Rhythmus}) ohne Ermüdung wiederholen können, versuchen Sie, sie zu beschleunigen.
Bei der maximalen Geschwindigkeit werden Sie wieder ermüden; werden Sie dann entweder langsamer oder versuchen Sie, andere Möglichkeiten zu finden, die Ermüdung zu reduzieren.
Wechseln Sie die Hände, sobald Sie sich müde fühlen.
\textbf{Spielen Sie nicht laut; ein Trick, die Ermüdung zu reduzieren, ist, leise zu spielen.
Sie können die Dynamikbezeichnungen später hinzufügen, wenn Sie die Technik erworben haben.}
Es ist extrem wichtig, leise zu üben, sodass Sie sich auf die Technik und die Entspannung konzentrieren können.
Am Anfang, wenn Sie sich anstrengen um schneller zu spielen, wird sich Ermüdung einstellen.
Wenn Sie aber die richtigen Bewegungen, Handstellungen usw. finden, werden Sie fühlen, wie die Müdigkeit die Hand verlässt, und Sie sollten die Hand ausruhen und sogar neu beleben können \textit{während Sie schnell spielen}.
Sie haben gelernt zu entspannen!

Wie der Triller besteht das Tremolo aus der Fingerbewegung und der Armdrehung.
Üben Sie zuerst das Fingertremolo, übertreiben Sie die Fingerbewegungen, spielen Sie ein sehr langsames Tremolo, heben Sie die Finger so hoch Sie können, und senken Sie die Finger mit Kraft in die Tasten.
Genauso mit der Armdrehung: Halten Sie die Finger still, und spielen Sie das Tremolo nur mit der Armdrehung und auf übertriebene Weise.
Alle Auf- und Abwärtsbewegungen müssen schnell sein; um langsam zu spielen, warten Sie einfach zwischen den Bewegungen, und üben Sie während des Wartens ein schnelles und völliges Entspannen.
Steigern Sie nun die Geschwindigkeit schrittweise, indem Sie die Bewegungen reduzieren.
Nachdem beide zufriedenstellend sind, kombinieren Sie sie; da beide Bewegungen ihren Beitrag zum Tremolo leisten, benötigen Sie von jeder einzelnen nur wenig, weshalb Sie sehr schnell spielen können.

Sie können die Geschwindigkeit sogar noch mehr steigern, wenn Sie die \hyperref[c1iii7b]{Übungen für parallele Sets} sowohl zu den Übungen der Fingerbewegung, der Armdrehung oder einer Kombination aus beiden hinzufügen.
Zuerst das 5,1-Set.
Beginnen Sie mit den wiederholten Oktaven, und ersetzen Sie dann schrittweise jede Oktave mit einem parallelen Set.
Wenn Sie zum Beispiel Gruppen von vier Oktaven spielen (4/4-Takt), fangen Sie damit an, die vierte Oktave durch ein paralleles Set zu ersetzen, dann die dritte und vierte, usw.
Bald sollten Sie alles als parallele Sets üben.
Wenn die parallelen Sets ungleichmäßig werden oder die Hand anfängt müde zu werden, gehen Sie zur Oktave zurück, um zu entspannen, oder wechseln Sie die Hand.
Arbeiten Sie an den parallelen Sets, bis Sie die zwei Noten des parallelen Sets fast \enquote{unendlich schnell} und reproduzierbar und schließlich mit guter Kontrolle und völliger Entspannung spielen können.
Bei den schnellsten Geschwindigkeiten der parallelen Sets sollten Sie Schwierigkeiten haben, zwischen parallelen Sets und Oktaven zu unterscheiden.
Verlangsamen Sie dann die parallelen Sets, sodass Sie bei allen Geschwindigkeiten mit Kontrolle spielen können.
Beachten Sie, dass in diesem Fall die 5-Note etwas lauter als die 1 sein sollte.
Sie sollten es jedoch auf beide Arten üben - mit dem Schlag auf der 5 und mit dem Schlag auf der 1 -, damit Sie eine ausgeglichene, kontrollierbare Technik entwickeln.
Wiederholen Sie das ganze Verfahren mit dem 1,5-Set.
Dieses parallele Set ist, obwohl es nicht zwingend erforderlich ist, um dieses Tremolo zu spielen (nur das vorhergehende ist notwendig), für die Entwicklung einer ausgeglichenen Kontrolle nützlich.
Sobald das 5,1- und das 1,5-Set zufriedenstellend sind, gehen Sie zu 5,1,5 oder 5,1,5,1 über (gespielt wie ein kurzer Oktavtriller).
Wenn Sie das 5,1,5,1 sofort können, besteht keine Notwendigkeit, das 5,1,5 zu üben.
Das Ziel ist hier sowohl Geschwindigkeit als auch Ausdauer, Sie sollten deshalb Geschwindigkeiten üben, die \textit{viel} schneller als die endgültige Tremolo-Geschwindigkeit sind, zumindest für diese kurzen Tremolos.
Arbeiten Sie dann an dem 1,5,1,5.

Sind die parallelen Sets erst einmal zufriedenstellend, beginnen Sie, Gruppen von zwei Tremolos zu spielen, eventuell mit einer kurzen Pause zwischen den Gruppen.
Steigern Sie dann zu Gruppen von drei und dann zu vier Tremolos.
Der beste Weg, die Tremolos zu beschleunigen, ist, zwischen Tremolos und Oktaven zu wechseln.
Beschleunigen Sie die Oktave und versuchen Sie, bei dieser schnelleren Geschwindigkeit zum Tremolo zu wechseln.
Alles, was Sie jetzt noch tun müssen, ist, die Hände abzuwechseln und Ausdauer aufzubauen.
Auch hier bedeutet Ausdauer aufzubauen nicht so sehr den Aufbau von Muskeln, sondern zu wissen, wie man entspannt und wie man die richtigen Bewegungen benutzt.
Entkoppeln Sie die Hände von Ihrem Körper; binden Sie nicht das Hand-Arm-Körper-System zu einem festen Knoten, sondern lassen Sie die Hände und Finger unabhängig vom Körper operieren.
Sie sollten frei atmen, unbeeinflusst von dem, was die Finger machen.
\textbf{Langsames Üben mit übertriebenen Bewegungen ist überraschend effektiv, kehren Sie deshalb dazu zurück, sobald Sie Probleme bekommen.}

Bei der rechten Hand (B-Oktave in Takt 149) sollte die 1 lauter als die 5 sein, aber bei beiden Händen sollten die leiseren Noten klar hörbar sein, und ihr offensichtlicher Zweck ist, die Geschwindigkeit verglichen mit der beim Spielen von Oktaven zu verdoppeln.
Erinnern Sie sich daran, leise zu üben; Sie können lauter spielen, wann immer Sie es später möchten, wenn Sie erst die Technik und Ausdauer erworben haben.
Es ist wichtig, in der Lage zu sein, bei den höchsten Geschwindigkeiten leise zu spielen und trotzdem jede Note hören zu können.
Üben Sie, bis Sie bei der endgültigen Geschwindigkeit die Tremolos länger spielen können als Sie es im Stück benötigen.
Der endgültige Effekt der linken Hand ist ein konstantes Getöse, das Sie in der Lautstärke auf und ab modulieren können.
Die untere Note trägt den Rhythmus bei, und die obere Note verdoppelt die Geschwindigkeit.
Üben Sie dann die aufsteigenden Tremolos wie sie in den Noten stehen.

Das Grave, das diesen ersten Satz beginnt, ist wegen seines ungewöhnlichen \hyperref[c1iii1b]{Rhythmus} und den schnellen Läufen in den Takten 4 und 10 nicht einfach, obwohl das Tempo langsam ist.
Der Rhythmus des ersten Takts ist nicht einfach, weil die erste Note des zweiten Schlags fehlt.
Um den korrekten Rhythmus zu lernen, benutzen Sie ein \hyperref[c1ii19]{Metronom} oder spielen Sie mit der linken Hand einzelne Rhythmusnoten, während Sie mit der rechten üben.
Obwohl der Rhythmus 4/4 ist, ist es einfacher, wenn Sie die Noten der linken Hand verdoppeln und wie bei einem 8/8-Rhythmus üben.
Der Lauf in Takt 4 ist sehr schnell; es gibt neun Noten in der letzten Gruppe von 1/128-Noten.
Sie müssen deshalb als Triolen gespielt werden und mit der doppelten Geschwindigkeit wie die vorangegangenen zehn Noten.
Das entspricht 32 Noten je Schlag, was für die meisten Klavierspieler unmöglich ist, Sie benötigen deshalb vielleicht ein wenig Rubato; \textbf{die richtige Geschwindigkeit wird, nach dem Originalmanuskript, ungefähr die Hälfte der angegebenen Geschwindigkeit sein}.
Der zehnte Takt enthält so viele Noten, dass er in der Dover-Edition zwei Zeilen umfasst!
Die letzte Gruppe von 16 1/128-Noten wird wieder mit der doppelten Geschwindigkeit der vorangegangenen Noten gespielt, für die meisten Klavierspieler unmöglich schnell.
Der viernotige chromatische Fingersatz (\hyperref[c1iii5h]{\autoref{c1iii5h}}) kann bei solchen Geschwindigkeiten nützlich sein.
Jeder Schüler, der dieses Grave zum ersten Mal lernt, muss die Noten und Schläge sorgfältig zählen, damit er eine klare Vorstellung davon bekommt, worum es geht.
Diese verrückten Geschwindigkeiten sind vielleicht ein Fehler eines Herausgebers.

Der erste (und dritte) Satz ist eine Variation des Themas im Grave-Abschnitt. 
Dieses berühmte \enquote{Dracula}-Thema wurde der linken Hand des ersten Takts entnommen; die linke Hand trägt den emotionalen Inhalt, obwohl die rechte die Melodie trägt.
Achten Sie auf das Staccatissimo und das \textit{sf} in den Takten 3 und 4.
In den Takten 7 und 8 müssen die letzten Noten der drei ansteigenden chromatischen Oktaven als 1/16-, 1/8- und 1/4-Noten gespielt werden, die zusammen mit der ansteigenden Tonhöhe und dem Crescendo den dramatischen Effekt erzeugen.
Das ist der wahre Beethoven mit maximalem Kontrast: leise - laut, langsam - schnell, einzelne Noten - komplexe Akkorde.
In Beethovens Manuskript gibt es keine Pedalzeichen.


% zuletzt geändert 24.10.2010

