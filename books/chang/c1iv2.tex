% File: c1iv2

\section{Die Theorie der Fingerbewegung}\hypertarget{c1iv2}{}

\subsection{Serielles und paralleles Spielen}\hypertarget{c1iv2a}{}

Die Fingerbewegungen für das Klavierspielen können auf unterster Stufe in serielle und parallele Bewegungen unterteilt werden.
Beim seriellen Spielen werden die Finger nacheinander gesenkt um zu spielen.
Eine Tonleiter ist ein Beispiel für etwas, das seriell gespielt werden kann.
Beim parallelen Spielen bewegen sich alle Finger gleichzeitig.
Ein Akkord ist ein Beispiel für paralleles Spielen.
Wie wir später sehen werden, kann eine Tonleiter auch parallel gespielt werden.

Serielles Spielen kann durch eine oszillierende Funktion, wie z.B. eine trigonometrische Funktion, beschrieben werden.
Es ist im Grunde durch eine Amplitude (die Entfernung, die der Finger sich auf und ab bewegt) und eine Frequenz (wie schnell man spielt) charakterisiert.
Außer bei Akkorden und schnellen Rollungen können die meisten langsamen Stücke seriell gespielt werden, und Anfänger neigen dazu, mit seriellem Spielen zu beginnen.
Beim parallelen Spielen gibt es eine wohldefinierte Phasenbeziehung zwischen den verschiedenen Fingern.
Deshalb müssen wir nun die Phase etwas ausführlicher behandeln.

Die Phase ist ein Maß dafür, wo sich der Finger relativ zu den anderen Fingern befindet.
Angenommen, wir benutzen die trigonometrischen Funktionen (Sinus, Cosinus, usw.), um die Fingerbewegung zu beschreiben.
Dann befindet sich der Finger in seiner Ruheposition z. B. bei 0 Grad im Phasenraum.
Da wir wissen, wie das Klavier gespielt werden sollte, werden wir etwas von diesem Wissen in unsere Definition der Phase einbauen.
Weil das Abheben der Finger von den Tasten im allgemeinen nicht die richtige Art zu spielen ist, werden wir den Nullpunkt der Phase als die obere Ruheposition der Tasten definieren.
Somit liegt der Phasennullpunkt der schwarzen Tasten um die zusätzliche Höhe der schwarzen Tasten höher als der Phasennullpunkt der weißen Tasten.
Weiterhin nehmen wir an, daß es bezüglich der Phase nicht berücksichtigt wird, wenn man die Finger von den Tasten abhebt.
Diese Konventionen befinden sich in Übereinstimmung mit einer guten Technik und vereinfachen auch die Mathematik.
Dann können die Phasen dieser Bewegung folgendermaßen definiert werden:

\begin{itemize} 
 \item der Finger drückt den halben Weg nach unten = 90 Grad
 \item niederdrücken bis zur unteren Position = 180 Grad
 \item den halben Weg anheben = 270 Grad
 \item zurück in die Ausgangsposition anheben = 360 Grad, was wieder 0 Grad ist.
 \end{itemize}
Wenn nun beim parallelen Spielen der zweite Finger seine Bewegung beginnt, wenn der erste Finger bei 90 Grad ist, der dritte Finger beginnt, wenn der erste Finger bei 180 Grad ist, usw., dann werden bei diesem parallelen Spielen die Noten 4 mal so schnell gespielt wie beim seriellen Spielen mit derselben Fingergeschwindigkeit.
In diesem Fall beträgt die Phasendifferenz zwischen den Fingern 90 Grad.
Wenn man die Phasendifferenz auf 9 Grad senken würde, könnte man die Noten 40 mal so schnell spielen - dieses Beispiel zeigt die Stärke des parallelen Spielens für das Steigern der Spielgeschwindigkeit.
Bei einem Akkord ist die Phasendifferenz 0.

Serielles Spielen kann als paralleles Spielen definiert werden, bei dem die Phasendifferenz zwischen aufeinanderfolgenden Fingern ungefähr 360 Grad oder mehr ist oder bei dem die Phasen in keinem Zusammenhang stehen.
Eine Bewegung der Hand nützt sowohl dem seriellen als auch dem parallelen Spielen aber auf unterschiedliche Weise.
Sie hilft dem seriellen Spielen durch das Vergrößern der Amplitude.
Sie beeinflußt jedoch das parallele Spielen in erheblicherem Maß, indem sie Ihnen bei der Kontrolle der Phase hilft.
Mit diesen einfachen Definitionen können wir damit beginnen, einige nützliche Resultate zu erzeugen.


\subsection{Geschwindigkeitsbarrieren}\hypertarget{c1iv2b}{}

Nehmen wir an, jemand beginnt ein Musikstück zu üben, indem er zunächst langsam spielt und überwiegend serielles Spielen benutzt, weil das der einfachste Weg ist (ignorieren wir erst einmal die Akkorde).
Wenn die Geschwindigkeit der Finger schrittweise gesteigert wird, wird er natürlich auf eine Geschwindigkeitsbarriere stoßen, weil menschliche Finger sich nicht unendlich schnell bewegen können.
Somit haben wir eine Geschwindigkeitsbarriere mathematisch entdeckt und zwar die Geschwindigkeitsbarriere des seriellen Spielens.
Wie überwindet man diese Geschwindigkeitsbarriere?
Wir müssen eine Spielmethode finden, die keine Geschwindigkeitsbegrenzung hat.
Das ist das parallele Spielen.
Beim parallelen Spielen steigert man die Geschwindigkeit, indem man die Phasendifferenz verringert.
D.h. die Geschwindigkeit ist der Phasendifferenz umgekehrt proportional.
Da wir wissen, daß die Phasendifferenz bis auf Null vermindert werden kann (was einen Akkord bedeutet), wissen wir, daß paralleles Spielen das Potential für unendliche Geschwindigkeit und deshalb keine theoretische Geschwindigkeitsbeschränkung hat.
Wir sind bei einer mathematischen Grundlage des \hyperlink{c1ii9}{Akkord-Anschlags} angekommen!

Die Unterscheidung zwischen seriellem und parallelem Spielen ist in gewisser Weise künstlich und stark vereinfacht.
In Wahrheit wird praktisch alles parallel gespielt.
Deshalb diente uns die obige Diskussion nur zur Illustration, wie man eine Geschwindigkeitsbarriere definiert oder erkennt.
Die tatsächliche Situation jedes einzelnen ist zu komplex, um sie zu beschreiben (weil Geschwindigkeitsbarrieren durch schlechte Angewohnheiten, Streß und HT-Spielen verursacht werden), aber es ist klar, daß falsche Spielmethoden Geschwindigkeitsbarrieren erzeugen und jeder seine eigenen Fehler hat, die zu Geschwindigkeitsbarrieren führen.
Das wird durch die Benutzung der \hyperlink{c1iii7b}{Übungen für Parallele Sets} gezeigt, mit denen man die Geschwindigkeitsbarrieren überwindet.
Das bedeutet, daß Geschwindigkeitsbarrieren nicht immer von sich aus vorhanden sind, sondern von jedem einzelnen \textit{erzeugt} werden.
Deshalb gibt es für jeden eine beliebige Zahl möglicher Geschwindigkeitsbarrieren und jeder hat einen unterschiedlichen Satz Geschwindigkeitsbarrieren.
Es gibt selbstverständlich allgemeine Klassen von Geschwindigkeitsbarrieren, wie z.B. jene, die durch Streß, falsche Fingersätze, Mangel an HS-Technik, Mangel an HT-Koordination, usw. erzeugt werden.
Meiner Meinung nach wäre es sehr kontraproduktiv, zu sagen, daß solche komplexen Konzepte nicht irgendwann wissenschaftlich oder mathematisch behandelt werden.
Wir müssen es tun.
So spielt z.B. beim parallelen Spielen die Phase eine sehr wichtige Rolle.
Indem man die Phase auf Null vermindert, können wir im Prinzip unendlich schnell spielen.

Können wir wirklich unendlich schnell spielen? Natürlich nicht.
Was ist also dann die Höchstgeschwindigkeit beim parallelen Spielen und welcher Mechanismus erzeugt diese Grenze?
Wir wissen, daß verschiedene Menschen verschiedene Geschwindigkeitsbeschränkungen haben, deshalb muß die Antwort einen Parameter einschließen, der von diesem Menschen abhängt.
Wenn wir diesen Parameter kennen, können wir erklären, wie man schneller spielt!
Sicherlich wird die schnellste Geschwindigkeit durch die kleinste Phasendifferenz bestimmt, die der einzelne kontrollieren kann.
Wenn die Phasendifferenz so klein ist, daß sie nicht kontrolliert werden kann, dann verliert die \enquote{parallele Spielgeschwindigkeit} ihre Bedeutung.
Wie mißt man diese winzige Phasendifferenz beim einzelnen Menschen?
Das kann durch das Anhören seiner Akkorde erreicht werden.
Die Genauigkeit des Akkord-Spiels, d.h. wie genau alle Noten des Akkords gleichzeitig gespielt werden können, ist ein gutes Maß für die Fähigkeit des einzelnen, die kleinsten Phasendifferenzen zu kontrollieren.
Deshalb muß man, um schnell parallel zu spielen, in der Lage sein, genaue Akkorde zu spielen.
Das bedeutet, daß Sie bei der Anwendung des \hyperlink{c1ii9}{Akkord-Anschlags} zuerst in der Lage sein müssen, genaue Akkorde zu spielen, bevor Sie zum nächsten Schritt übergehen.

Es ist klar, daß es viele weitere Geschwindigkeitsbarrieren gibt, und die bestimmte Geschwindigkeitsbarriere sowie die Methoden für das Überwinden jeder Barriere hängen von der Art der Finger- oder Handbewegung ab.
So kann man unendliche Geschwindigkeit mit parallelem Spielen nur erreichen, wenn man eine unendliche Zahl von Fingern hat (z.B. für einen langen Lauf).
Leider haben wir nur zehn Finger, und oft stehen nur fünf für eine bestimmte Passage zur Verfügung, weil die anderen fünf benötigt werden, um andere Teile der Musik zu spielen.
Als eine grobe Näherung kann gelten, daß wenn serielles Spielen es gestattet, mit einer maximalen Geschwindigkeit von M zu spielen, dann kann man mit zwei Fingern mit einer Geschwindigkeit von 2M spielen, mit drei Fingern 3M, usw.
Die maximale Geschwindigkeit wird dadurch begrenzt, wie schnell man die Finger zirkulieren kann.
In Wahrheit stimmt das wegen des Impulsausgleichs nicht ganz (er erlaubt es, schneller zu spielen), was weiter unten gesondert behandelt wird.
Somit führt jede Zahl zur Verfügung stehender Finger zu einer anderen neuen Geschwindigkeitsbarriere.
Deshalb kommen wir zu zwei weiteren nützlichen Ergebnissen: 

\begin{enumerate} 
 \item Es existieren beliebig viele Geschwindigkeitsbarrieren.
 \item Man kann seine Geschwindigkeitsbarriere dadurch ändern, daß man den Fingersatz ändert.
 \end{enumerate}
Allgemein gesagt: Je mehr Finger Sie beim parallelen Spielen benutzen können, bevor Sie zirkulieren müssen, desto schneller können Sie spielen.
Anders gesagt: Die meisten Verbindungen führen zu ihrer eigenen Geschwindigkeitsbarriere.


\subsection{Die Geschwindigkeit steigern}\hypertarget{c1iv2c}{}

Diese Ergebnisse bieten uns auch die mathematische Grundlage, um den wohlbekannten Trick zu erklären, die Finger abzuwechseln, wenn man dieselbe Note mehrmals spielt.
Man mag zunächst denken, daß nur einen Finger zu benutzen einfacher wäre und mehr Kontrolle bieten würde, aber diese Note kann schneller wiederholt gespielt werden, indem man parallel spielt und so viele Finger benutzt wie man in dieser Situation kann, als wenn man seriell spielen würde.

Die Notwendigkeit für paralleles Spielen läßt schnell gespielte Triller ebenfalls zu einer besonders großen Herausforderung werden, weil Triller im allgemeinen mit nur zwei Fingern ausgeführt werden müssen.
Wenn man versuchen würde, mit einem Finger zu trillern, würde man bei einer Geschwindigkeit von sagen wir M auf eine Geschwindigkeitsbarriere treffen; wenn man mit zwei Fingern trillert, wird die Geschwindigkeitsbarriere bei 2M liegen (wobei wir wieder den Impulsausgleich vernachlässigen).
Schlägt die Mathematik einen anderen Weg vor, um noch höhere Geschwindigkeiten zu erreichen?
Ja: Phasenkürzung.

Sie können den Finger für das Spielen der Note senken aber ihn nur so weit heben, wie es notwendig ist, um den Repetiermechanismus zurückzusetzen, bevor Sie die nächste Note spielen.
Sie müssen den Finger vielleicht nur um 90 Grad statt der normalen 180 Grad anheben.
Das ist es, was ich mit Phasenkürzung meine; der unnötige Teil der kompletten Phase wird abgeschnitten.
Wenn die ursprüngliche Amplitude des Fingerwegs für die Bewegung von 360 Grad 2 cm betragen hat, dann muß der Finger bei einer Verkürzung um 180 Grad nur 1 cm bewegt werden.
Dieser eine Zentimeter kann weiter bis zu der Grenze reduziert werden, an der der Repetiermechanismus nicht mehr funktioniert, d.h. bei ungefähr 5 mm.
Phasenkürzung ist die mathematische Basis für die schnelle Repetierung des Flügels und erklärt, warum die schnelle Repetierung so konstruiert ist, daß sie mit einer kurzen Strecke bis zum Umkehrpunkt funktioniert.

Eine gute Analogie dafür, auf diese Art auf Geschwindigkeit zu kommen, ist die Bewegung eines Basketballs beim Dribbeln im Gegensatz zur schwingenden Bewegung eines Pendels.
Ein Pendel hat eine feste Schwingungsfrequenz unabhängig von der Schwingungsamplitude.
Ein Basketball \enquote{schwingt} jedoch schneller, wenn man näher am Boden dribbelt (wenn man die Dribbelamplitude reduziert).
Ein Basketballspieler wird es im allgemeinen schwer haben zu dribbeln, bis er diese Veränderung der Dribbelfrequenz in Abhängigkeit von der Dribbelhöhe gelernt hat.
Ein Klavier verhält sich (zum Glück!) mehr wie ein Basketball als ein Pendel, und die Trillerfrequenz steigt mit sinkender Amplitude, bis man die Grenze des Repetiermechanismus erreicht.
Beachten Sie, daß der Fänger auch beim schnellsten Triller eingehakt sein muß, d.h. die Taste muß immer vollständig heruntergedrückt sein.
Der Triller wird möglich, weil die mechanische Antwort des Fängers schneller ist als die schnellste Geschwindigkeit, die der Finger erreichen kann.

Die Trillergeschwindigkeit wird, außer durch die Höhe, bei der die Repetierung aufhört zu funktionieren, nicht durch den Klaviermechanismus begrenzt.
Deshalb ist es bei den meisten Klavieren \textit{[im Gegensatz zu den Flügeln]} schwieriger, schnell zu trillern, weil die Phasenkürzung hier keine so große Auswirkung hat.
Diese mathematische Schlußfolgerungen sind mit der Tatsache konsistent, daß wir um schnell zu trillern die Finger auf den Tasten halten und die Bewegungen auf das für das Funktionieren des Repetiermechanismus notwendige Minimum reduzieren müssen.
Die Finger müssen \enquote{tief in das Klavier} drücken und dürfen nur gerade genügend angehoben werden, um den Repetiermechanismus zu aktivieren.
Außerdem hilft es, die Saiten zu benutzen, um den Hammer zurückspringen zu lassen, genauso wie man einen Basketball vom Boden wegspringen läßt.
Beachten Sie, daß man einen Basketball bei einer gegebenen Amplitude schneller dribbeln kann, wenn man ihn fester herunterdrückt.
Auf dem Klavier wird dies dadurch erreicht, daß man die Finger fest auf die Tasten drückt und sie nicht \enquote{hochschweben} läßt während man trillert.

Ein weiterer wichtiger Faktor ist die Abhängigkeit der Kontrolle des Klangs, Staccatos und anderer Eigenschaften des Klavierklangs im Zusammenhang mit dem Ausdruck von der Funktion der Fingerbewegung (rein trigonometrisch oder hyperbolisch usw.).
Mit einfachen elektronischen Instrumenten ist es eine leichte Aufgabe, die exakte Fingerbewegung, zusammen mit der Tastengeschwindigkeit, Beschleunigung, usw., zu messen.
Diese Spieleigenschaften eines jeden Klavierspielers können mathematisch analysiert werden, um die charakteristischen elektronischen Signaturen zu ermitteln, die damit verbunden werden können, wie wir das Gehörte wahrnehmen, z.B. als wütend, gefällig, angeberisch, tief, flach, usw.
Die Bewegung der Taste kann z.B. unter Benutzung der schnellen Fourier-Transformation analysiert werden, und es sollte möglich sein, anhand der Ergebnisse jene Elemente der Bewegung zu identifizieren, die die entsprechenden hörbaren Merkmale erzeugen.
Indem man sich von diesen Merkmalen aus rückwärts arbeitet, sollte es dann möglich sein, zu ermitteln wie man spielen muß, um diese Effekte zu erzeugen.
Das ist ein völlig neues Gebiet des Klavierspielens, das bisher noch nicht erforscht worden ist.
Diese Art der Analyse ist nicht möglich, indem man sich nur eine Aufnahme eines berühmten Pianisten anhört, und mag das wichtigste Thema für die zukünftige Forschung sein.



