% File: c1iii6m

\subsection{Funktion des menschlichen Gedächtnisses}
\label{c1iii6m}

Die Gedächtnisfunktion des Gehirns wird bisher nur unvollständig verstanden.
\textbf{Es gibt keinen Beweis für die Existenz eines \enquote{fotografischen Gedächtnisses} im engeren Sinne des Wortes}, obwohl ich diesen Ausdruck in diesem Buch benutzt habe.
\textbf{Das ganze Gedächtnis ist assoziativ.}
Somit assoziieren wir, wenn wir uns ein Gemälde von Monet visuell einprägen, in Wirklichkeit die Motive des Gemäldes mit etwas tief im Inneren unseres Gedächtnisses und merken uns nicht bloß ein zweidimensionales Bild, das aus so vielen Bildpunkten besteht.
Deshalb ist es leichter, sich an bedeutende Gemälde oder ungewöhnliche Fotografien zu erinnern als an ähnliche Bilder mit weniger charakteristischen Merkmalen, obwohl beide vielleicht dieselbe Bandbreite (Anzahl Bildpunkte) haben.
Ein weiteres Beispiel: Wenn Sie einen Kreis auf einem Blatt Papier fotografieren, wird das Foto genau sein; der Durchmesser und die Lage des Kreises wird übereinstimmen.
Wenn Sie in Gedanken ein \enquote{fotografisches Engramm} desselben Kreises anfertigen und dann versuchen, ihn auf einem anderen Blatt Papier zu zeichnen, wird der Durchmesser und die Lage abweichen.
Das bedeutet, daß Sie ihn sich begrifflich gemerkt (und mit dem bereits vorhandenen Wissen über Kreise, ungefähre Größen und Orte assoziiert) haben.
Wie ist es nun mit dem fotografischen Gedächtnis bei Notenblättern?
Ich kann sie tatsächlich in Gedanken sehen!
Ist das nicht fotografisch?
Man kann leicht beweisen, daß das ebenfalls assoziativ ist - in diesem Fall assoziiert mit Musik.
Wenn Sie einen Musiker mit \enquote{fotografischem} Gedächtnis bitten, sich eine ganze Seite zufälliger Noten einzuprägen, wird er große Schwierigkeiten haben, obwohl er vielleicht keine Probleme haben wird, sich innerhalb kurzer Zeit eine ganze Sonate zu merken\footnote{dasselbe gilt analog für Schachspieler für zufällige und sinnvolle Stellungen von Figuren auf einem Schachbrett}.
Deshalb gibt es keinen besseren Weg, sich Musik einzuprägen (fotografisch oder auf eine andere Weise), als vom Standpunkt der Musiktheorie aus.
Sie müssen nur die Musik mit der Theorie assoziieren, und Sie haben sie sich eingeprägt.
Mit anderen Worten: Wenn Menschen sich etwas einprägen, speichern sie nicht die Datenbits im Gehirn wie ein Computer, sondern sie assoziieren die Daten mit einem Grundgerüst oder einem \enquote{Algorithmus}, bestehend aus vertrauten Dingen im Gehirn.
In diesem Beispiel ist die Musiktheorie das Grundgerüst.
Natürlich kann ein sehr guter Auswendiglernender (der kein Musiker sein muß) Methoden dafür entwickeln, sich sogar eine zufällige Reihenfolge von Noten zu merken, indem er einen angemessenen Algorithmus ausarbeitet, wie wir im folgenden erklären.

Der beste Beweis für die assoziative Natur des menschlichen Gedächtnisses stammt aus Tests mit guten Auswendiglernenden, die unglaubliche Meisterleistungen ausführen können, wie Hunderte von Telefonnummern aus einem Telefonbuch auswendig lernen usw.
Es gibt zahlreiche Gedächtniswettbewerbe, in denen gute Auswendiglernende miteinander wetteifern.
Diese guten Auswendiglernenden wurden intensiv befragt, und es stellt sich heraus, daß keiner von Ihnen fotografisch auswendig lernt, obwohl das Endresultat fast nicht von einem fotografischen Gedächtnis zu unterscheiden ist.
Wenn sie gefragt werden, wie sie sich etwas einprägen, stellt sich heraus, daß sie alle assoziative Algorithmen benutzen.
Der Algorithmus ist bei jedem einzelnen verschieden (auch bei der gleichen Aufgabe), aber alle Algorithmen sind Mittel dafür, die Objekte mit etwas zu assoziieren, das ein Muster hat, an das man sich erinnern kann.
Zum Erinnern von hunderten von Zahlen ist z.B. ein Algorithmus, jede Zahl mit einem Klang zu assoziieren.
Die Klänge werden so gewählt, daß sie \enquote{Worte} bilden, wenn man sie aneinanderreiht - nicht in Deutsch, sondern in einer anderen Sprache, die für diesen Zweck geeignet ist.
Japanisch ist eine Sprache mit einer solchen Eigenschaft.
Ein Beispiel: Die Quadratwurzel von 2 ist 1,41421356, was man als Satz lesen kann, der übersetzt ungefähr \enquote{Gute Menschen, gute Menschen sind das Ansehen wert.} lautet, und die Japaner benutzen ständig solche Algorithmen, um sich an Nummern, wie z.B. Telefonnummern, zu erinnern.
Auf 7 Stellen ist die Quadratwurzel von 3 \enquote{Behandele die ganze Welt!}, und die Wurzel von 5 ist \enquote{An der sechsten Station des Fudschijama schreit eine Eule.}
Das Erstaunliche ist die Geschwindigkeit, mit der gute Auswendiglernende das auswendig zu lernende Objekt auf ihren Algorithmus abbilden können.
Es stellt sich auch heraus, daß diese guten Auswendiglernenden nicht so geboren wurden - obwohl sie vielleicht mit mentalen Fähigkeiten geboren wurden, die zu einem guten Gedächtnis führen können.
\textbf{Auswendiglernende entwickeln sich nach harter Arbeit zur Perfektionierung ihrer Algorithmen und täglichem Üben, genau wie Pianisten.}
Diese \enquote{harte Arbeit} leisten sie aber ohne Anstrengung, weil sie es genießen.

Ein einfacher, aber weniger effizienter Algorithmus ist, die Nummern in eine Geschichte zu verpacken.
Angenommen, Sie möchten sich an die 14 Ziffern 53031791389634 erinnern.
Sie könnten z.B. folgende Geschichte verwenden: \enquote{Ich wachte morgens um 5:30 Uhr mit meinen 3 Brüdern und 1 Großmutter auf; das Alter meiner Brüder ist 7, 9 und 13, und meine Großmutter ist 89 Jahre alt, und wir sind abends um 6:34 Uhr zu Bett gegangen.}
Das ist ein Algorithmus, der auf alltäglichen Erfahrungen basiert, was die Zufallszahlen \enquote{bedeutungsvoll} macht.
Das Faszinierende daran ist, daß der Algorithmus 43 Worte enthält und trotzdem viel einfacher zu behalten ist als die 14 Ziffern\footnote{wobei ich \enquote{5:30 Uhr} und \enquote{6:34 Uhr} als \enquote{5 Uhr 30} bzw. \enquote{6 Uhr 34} mit jeweils 3 Worten gerechnet habe}.
Tatsächlich haben Sie sich 203 Zeichen und Ziffern\footnote{inkl. Leerstellen und Satzzeichen} leichter gemerkt als die 14 Ziffern!
Sie können das leicht selbst testen.
Merken Sie sich zunächst sowohl die 14 Ziffern (wenn Sie können - für mich ist es nicht einfach) als auch den obigen Algorithmus.
Versuchen Sie 24 Stunden später, die Ziffern aus dem Gedächtnis und anhand des Algorithmus aufzuschreiben; Sie werden den Algorithmus viel leichter und genauer finden.
Alle guten Auswendiglernenden haben unglaublich effektive Algorithmen entwickelt und die Kunst gepflegt, jede Gedächtnisaufgabe sofort in ihre Algorithmen zu übersetzen.

Können Klavierspieler einen Vorteil aus diesem Gebrauch der effizienten Algorithmen ziehen?
Natürlich können wir das!
Wie hätte Liszt sonst innerhalb kurzer Zeit mehr als 80 Kompositionen auswendig lernen und aufführen können?
Es gibt keinen guten Grund, anzunehmen, daß Liszt in bezug auf das Gedächtnis spezielle Fähigkeiten hatte, er muß also einen Algorithmus benutzt haben.
Dieser Algorithmus ist überall - er heißt Musik!
Musik ist einer der effizientesten Algorithmen für das Auswendiglernen großer Datenmengen.
Praktisch alle Pianisten können leicht mehrere Beethoven-Sonaten auswendig lernen.
Von der Datenmenge her entspricht jede Sonate mindestens den Telefonnummern von vier Seiten eines Telefonbuchs.
Wir können also das Äquivalent von mehr als 20 Seiten Telefonnummern auswendig lernen - das würde als Wunder angesehen, wenn es tatsächlich Telefonnummern wären.
Und wir könnten uns wahrscheinlich noch mehr merken, wenn wir nicht soviel Zeit für das Üben der Technik und der Musikalität aufwenden müßten.
Deshalb unterscheidet sich, was Klavierspieler routinemäßig erreichen, nicht so sehr davon, wofür die \enquote{Gedächtniskünstler} berühmt sind.
Musik ist ein besonders effizienter Algorithmus, weil sie einigen strengen Regeln folgt.
Komponisten wie Liszt sind mit diesen Regeln vertraut und können schneller auswendig lernen (siehe dazu \hyperref[c1iv4]{Mozarts Formel} in Kapitel IV.4).
Zudem ist uns allen die musikalische Logik angeboren; diesen Teil des Musik-Algorithmus müssen wir nicht lernen.
Deshalb haben Musiker hinsichtlich des Auswendiglernens praktisch gegenüber jedem anderen Beruf einen Vorteil, und die meisten von uns sollten eine Stufe des Gedächtnisses erreichen, die der von guten Auswendiglernenden eines Gedächtniswettbewerbs nahekommt, da wir eine Menge darüber wissen, wie man es macht.

Es ist nun möglich, zu verstehen, wie Auswendiglernende viele Seiten von Telefonnummern auswendig lernen können.
Sie haben am Ende einfach eine \enquote{Geschichte} anstelle einer Reihe von Zahlen.
Beachten Sie, daß ein 90-jähriger Mann sich eventuell nicht mehr an Ihren Namen erinnern kann, aber er kann sich hinsetzen und Ihnen stunden- oder sogar tagelang Geschichten aus dem Gedächtnis erzählen.
Und er muß keine Art von Gedächtnisspezialist sein, um das zu tun.
Wenn man weiß, wie man sein Gehirn benutzen muß, dann kann man Dinge, die zunächst vollkommen unmöglich erschienen.

\textbf{Was an den Assoziationen befähigt uns also tatsächlich, etwas zu tun, was wir ansonsten nicht können?
Die vielleicht einfachste Art, dieses zu beschreiben, ist zu sagen, daß Assoziationen uns befähigen, das einzuprägende Objekt zu \textit{verstehen}.}
Das ist eine sehr nützliche Definition, weil sie jedem dabei helfen kann, in der Schule oder bei jedem Bemühen, etwas zu lernen, besser zu sein.
Wenn man Physik, Mathematik oder Chemie wirklich versteht, dann muß man sie nicht auswendig lernen, weil man sie nicht vergessen kann.
Das mag sinnlos erscheinen, weil wir bloß die Frage \enquote{Was ist Gedächtnis?} zunächst in \enquote{Was ist Assoziation?} und dann in \enquote{Was ist Verstehen?} umgewandelt haben.
Es ist nicht sinnlos, wenn wir \textit{Verstehen} definieren können: Es ist der geistige Prozeß, ein neues Objekt mit anderen Objekten (je mehr, desto besser!) zu assoziieren, die Ihnen bereits vertraut sind.
D.h., das neue Objekt wird nun \enquote{bedeutungsvoll}.

Was bedeutet \enquote{verstehen} und was \enquote{bedeutungsvoll}?
Das menschliche Gedächtnis besteht aus zahlreichen Komponenten, wie der visuellen, auditiven, taktilen, emotionalen, bewußten, automatischen, dem Kurz- und Langzeitgedächtnis usw.
Deshalb kann jede Eingabe in das Gehirn zu einer fast unendlichen Zahl von Assoziationen führen.
Die meisten Menschen stellen jedoch nur einige wenige her.
Das Gehirn von guten Auswendiglernenden erzeugt - beinahe automatisch und regelmäßig - aus jeder Eingabe zahlreiche Assoziationen.
Die große Zahl der Assoziationen gewährleistet, daß auch wenn man einige davon vergißt, eine genügende Anzahl übrig bleibt, um die Erinnerung aufrecht zu erhalten.
Das reicht jedoch nicht.
Wir haben gesehen, daß wir zum Auswendiglernen etwas verstehen müssen, was bedeutet, daß diese Assoziationen verknüpft und auf eine logische Weise geordnet sein müssen.
Ein guter Auswendiglernender kann also diese Assoziationen auch gut organisieren, so daß er, wenn er eine Eingabe erhält (z.B. den Namen einer Person), sofort das gewünschte (z.B. die Telefonnummer) finden kann, indem er diese Zusammenhänge verfolgt.
Wenn die Assoziationen nicht gut geordnet und untereinander verbunden sind, dann ist man eventuell nicht in der Lage, sich an die Nummer zu erinnern, obwohl sie irgendwo im Gedächtnis gespeichert ist.
Gute Auswendiglernende erzeugen ständig eine große Anzahl Assoziationen, sie verstärken sie laufend und sind in der Lage, diese Assoziationen in logischen Strukturen zu ordnen, so daß sie abgerufen werden können.
Die Gehirne guter Auswendiglernender suchen ständig nach \enquote{interessanten}, \enquote{erstaunlichen}, \enquote{unerklärlichen}, \enquote{außergewöhnlichen} usw. Assoziationen, die das Abrufen vereinfachen.
Dieselben Prinzipien gelten für das Auswendiglernen von Musik.

Die assoziative Natur des Gedächtnisses erklärt, warum das \hyperref[c1iii6tastatur]{Tastatur-Gedächtnis} funktioniert: Sie assoziieren die Musik mit den einzelnen Bewegungen und den Tasten, die zum Erzeugen der Musik gespielt werden müssen.
Das sagt uns auch, wie man das Tastatur-Gedächtnis optimiert.
Es ist sicher ein Fehler, zu versuchen, sich an jeden Tastendruck zu erinnern; wir sollten in Begriffen wie \enquote{RH-Arpeggio, das mit C anfängt und mit der LH eine Oktave tiefer wiederholt wird, Staccato, mit fröhlichem Gefühl} usw. denken und diese Bewegungen mit der daraus resultierenden Musik und ihrer Struktur assoziieren; merken Sie sich Notengruppen, Notenfamilien und abstrakte Konzepte.
Sie sollten so viele Assoziationen wie möglich herstellen: Bachs Musik kann bestimmte Eigenschaften haben, wie spezielle Verzierungen, kollidierende Hände und \hyperref[c1ii11]{parallele Sets}.
Sie machen damit die Aktion des Spielens \enquote{bedeutungsvoll}, und zwar in Begriffen, wie die Musik erzeugt wird und wie die Musik in Ihr geistiges Universum paßt.
Deshalb ist das Üben von Tonleitern und Arpeggios so wichtig.
Wenn Sie auf einen Lauf aus 30 Noten treffen, können Sie ihn sich einfach als Teil einer Tonleiterfolge statt als 30 Noten merken.
Sich ein \hyperref[c1iii12]{absolutes Gehör oder zumindest ein relatives Gehör} anzutrainieren, ist für das Gedächtnis ebenfalls hilfreich, weil das weitere Assoziationen zu den einzelnen Noten ermöglicht.
Musiker erzeugen am häufigsten Assoziationen mit den von der Musik hervorgerufenen Emotionen.
Einige benutzen Farben oder Landschaften.
\enquote{Geborene Auswendiglernende} ist ein Begriff ohne Definition, da jeder gute Auswendiglernende ein System hat, und alle Systeme scheinen einigen sehr ähnlichen Grundprinzipien zu folgen.


\subsection{Ein guter Auswendiglernender werden}
\label{c1iii6n}

Niemand wird ohne Übung ein guter Auswendiglernender, so wie niemand ohne Übung ein guter Klavierspieler wird.
Die gute Nachricht ist, daß mit dem richtigen Training praktisch jeder ein guter Auswendiglernender werden kann, so wie jeder mit den richtigen Übungsmethoden ein guter Klavierspieler werden kann.
Bei den meisten Schülern ist der Wunsch auswendig zu lernen stark genug, und sie sind deshalb bereit zu üben; trotzdem scheitern viele.
Wissen wir, warum sie scheitern, und gibt es eine einfache Lösung für das Problem?
Die Antwort ist: \enquote{Ja!}

\textbf{Schlechte Auswendiglernende versagen beim Auswendiglernen, weil sie aufhören, bevor sie angefangen haben.}
Man hat sie nie in effektive Gedächtnismethoden eingeführt, und sie haben genügend Fehlschläge erlebt, um zu dem Schluß zu kommen, daß es nutzlos sei, das Auswendiglernen zu versuchen.
Um ein guter Auswendiglernender zu werden, ist die Erkenntnis sehr hilfreich, daß unser Gehirn unabhängig davon, ob wir das möchten oder nicht, alles aufzeichnet.
Unser einziges Problem beim Auswendiglernen ist, daß wir diese Daten nicht so leicht abrufen können.

Wir haben gesehen, daß das endgültige Ziel aller besprochenen Gedächtnisprozeduren das gute und solide \hyperref[c1ii12mental]{mentale Spielen} war.
\textbf{Bevor ich das mentale Spielen untersuchte, dachte ich, daß es nur von wirklich begabten Künstlern durchgeführt werden könnte.
Das hat sich als völlig falsch erwiesen.}
Wir alle führen das mentale Spielen in unserem täglichen Leben aus!
Das mentale Spielen ist nur ein Prozeß, bei dem wir Informationen aus unserem Gedächtnis abrufen und sie für das Planen unserer Aktionen, Lösen unserer Probleme usw. ordnen oder benutzen.
Wir tun das praktisch in jedem wachen Moment und wahrscheinlich sogar während des Schlafens.
Wenn eine Mutter mit drei Kindern am Morgen aufsteht, die täglichen Aktivitäten für ihre Familie plant, überlegt, was es zu essen geben soll, wie jedes Gericht für das Frühstück, Mittagessen und Abendessen zubereitet wird, usw., führt sie eine mentale Prozedur aus, die genauso komplex ist wie bei Mozart, wenn dieser eine Bach-Invention in Gedanken spielte.
Wir halten diese Mutter nur deshalb nicht für ein Genie vom Range Mozarts, weil wir mit diesen mentalen Prozessen, die wir jeden Tag ohne Anstrengung ausführen, so vertraut sind.
Obwohl Mozarts Fähigkeit, Musik zu komponieren, in der Tat außerordentlich war, ist mentales Spielen nichts ungewöhnliches - mit ein wenig Übung können wir es alle.
\textbf{In der heutigen Lehr- und Übungspraxis wurde das mentale Durchgehen des Ablaufs in den meisten Disziplinen, die den Einsatz des Gedächtnisses erfordern, zum Standard (z.B. beim Golf, Eiskunstlauf, Tanzen, Abfahrtsskilauf usw.).
Es sollte auch Klavierschülern von Anfang an gelehrt werden.}

Eine weitere Möglichkeit, das Gedächtnis zu verbessern, ist die Anwendung der \enquote{Vergiß-es-dreimal-Regel}: Wenn man dasselbe dreimal vergessen und erneut auswendig lernen kann, wird man sich gewöhnlich ewig daran erinnern.
Diese Regel funktioniert, weil sie die Frustration über das Vergessen eliminiert und Ihnen dreimal die Gelegenheit bietet, verschiedene Methoden zum Auswendiglernen und Abrufen zu üben.
Die Frustration über das Vergessen und die Furcht vor dem Vergessen sind die größten Feinde schlechter Auswendiglernender.
Sie müssen etwas nicht wirklich völlig vergessen, aber lassen Sie sich genügend Zeit (ein paar Tage oder mehr), so daß die Wahrscheinlichkeit groß ist, daß sie etwas vergessen, und lernen Sie es dann erneut auswendig.

Nachdem Sie angefangen haben, das Auswendiglernen und dessen \hyperref[c1iii6k]{Pflege} zu üben, können Sie nach und nach die oben besprochenen Konzepte hinzufügen (Assoziationen, Verstehen, Informationen ordnen usw.).
Ein junger Mensch, der am Anfang des Lebens diese Techniken wie selbstverständlich anwendet, wird auf fast allen Gebieten ein guter Auswendiglernender werden.
Mit anderen Worten: \textbf{Das Gehirn beschäftigt sich ständig mit dem Auswendiglernen, es wird zur mühelosen, automatischen Routine.
Das Gehirn sucht automatisch nach interessanten Assoziationen und pflegt das Gedächtnis fortlaufend ohne bewußten Aufwand.}
Bei älteren Menschen ist dieser \enquote{Automatismus} viel schwieriger und wird länger dauern.
Wenn es Ihnen gelingt, sich die ersten Informationen einzuprägen (z.B. ein Repertoire von Klavierstücken), dann werden Sie gleichzeitig anfangen, dieselben Prinzipien auf alles andere anzuwenden, und Ihr allgemeines Gedächtnis wird sich verbessern.
Deshalb müssen Sie, um ein guter Auswendiglernender zu werden, zusätzlich zur Anwendung der hier besprochenen Gedächtnismethoden, die Art und Weise, wie Sie Ihr Gehirn benutzen, ändern.
Das Gehirn muß darauf trainiert werden, ständig Assoziationen zu suchen, besonders nach stimulierenden (lustigen, fremdartigen, furchteinflößenden usw.), die Ihnen dabei helfen, das Auswendiggelernte abzurufen.
Das ist der schwierigste Teil: zu ändern, wie das Gehirn arbeitet.


\subsection{Zusammenfassung}
\label{c1iii6o} 

\textbf{Benutzen Sie zum Auswendiglernen von Klaviermusik einfach die Regeln für das Lernen mit dem zusätzlichen Vorbehalt, daß Sie alles auswendig lernen, \textit{bevor} Sie anfangen, das Stück zu üben.}
Es ist diese Wiederholung während des Übens aus dem Gedächtnis, die das Gedächtnis automatisch mit wenig zusätzlichem Aufwand implantiert, d.h. zusätzlich zu dem Aufwand, der für das Lernen des Stücks notwendig ist.
Der wichtigste erste Schritt ist das Auswendiglernen mit HS.
Wenn man etwas über einen bestimmten Punkt hinaus auswendig lernt, wird man es fast niemals vergessen.
HS-Spielen ist auch das Hauptelement der \hyperref[c1iii6k]{Gedächtnispflege}.
Für das Auswendiglernen können Sie das \hyperref[c1iii6hand]{Hand-Gedächtnis}, \hyperref[c1iii6foto]{fotografische Gedächtnis}, \hyperref[c1iii6tastatur]{Tastatur-Gedächtnis und mentale Spielen}, \hyperref[c1iii6musik]{Musik-Gedächtnis} und die \hyperref[c1iii6theorie]{Musiktheorie} benutzen.
Das menschliche Gedächtnis ist assoziativ, und ein guter Auswendiglernender kann gut Assoziationen finden und sie so ordnen, daß sie zu einem \enquote{Verständnis} des Themas führen.
Erinnern Sie sich daran, daß die Musik einer der effizientesten Algorithmen für das Gedächtnis ist; ein \hyperref[c1iii12]{absolutes Gehör} ist ebenfalls hilfreich.
Alle diese Gedächtnismethoden sollten in das mentale Spielen münden - Sie können die Musik in Gedanken spielen und hören, so als ob Sie ein Klavier im Kopf hätten.
Das mentale Spielen ist praktisch für alles, was man am Klavier tut, unentbehrlich; es versetzt Sie z.B. in die Lage, das Auswendiglernen und Abrufen jederzeit zu üben.
Wir haben gesehen, daß gute Auswendiglernende deshalb gut sind, weil ihr Gehirn sich stets automatisch etwas einprägt; man kann sein Gehirn nur dazu trainieren, wenn man mental spielen kann.
Sie sollten zwei Repertoires haben: ein auswendig gelerntes und ein weiteres für das Spielen vom Blatt.
Auswendiglernen ist notwendig, um ein Stück schnell und gut zu lernen, \hyperref[c1iii14d]{musikalisch zu spielen}, sich schwierige Technik anzueignen usw.
Das mentale Spielen bringt eine ganz neue wunderbare Welt musikalischer Fähigkeiten mit sich, wie z.B. ein  Stück ab einer Stelle irgendwo in der Mitte zu spielen, das absolute Gehör zu erlernen, Komponieren, fehlerfrei vorzuspielen usw.
Viele dieser unglaublichen Meisterleistungen, die den musikalischen Genies nachgesagt werden, sind für uns alle tatsächlich in Reichweite!



