% File: c1iii8

\subsection{Konturieren (Beethovens Sonate \#1)}
\label{c1iii8} 

\textbf{Konturieren ist eine Methode, den Lernprozeß durch die Vereinfachung der Musik zu beschleunigen.}
Es gestattet Ihnen, den musikalischen Fluß oder Rhythmus beizubehalten, und das fast sofort mit der endgültigen Geschwindigkeit.
Das versetzt Sie in die Lage, den musikalischen Gehalt eines Abschnitts, lange bevor dieser befriedigend oder mit der richtigen Geschwindigkeit gespielt werden kann, zu üben.
\textbf{Es hilft Ihnen auch dabei, sich schwierige Techniken schnell anzueignen, da man zunächst den größeren Spielgliedern (Arme, Schultern) lehrt, wie sie sich richtig bewegen müssen; wenn das erreicht ist, finden die kleineren Glieder oftmals leichter ihren Platz.}
Es eliminiert auch viele Fallen für Timing- und für musikalische Interpretationsfehler.
Die Vereinfachungen werden durch verschiedene Mittel erreicht, wie z.B. \enquote{weniger wichtige Noten} zu löschen oder Notenfolgen zu einem Akkord zusammenzufassen.
Sie gehen dann schrittweise zur Originalmusik zurück, indem Sie nach und nach die vereinfachten Noten wieder herstellen.
\hyperref[Whiteside]{Whiteside} hat eine gute Beschreibung des Konturierens auf Seite 141 des ersten Buchs und den Seiten 54-61, 105-107 und 191-196 des zweiten Buchs, in dem verschiedene Beispiele analysiert werden.

Für einen bestimmten Abschnitt gibt es üblicherweise viele Möglichkeiten, den Notensatz zu vereinfachen, und wenn jemand das Konturieren das erste Mal benutzt, braucht es einige Übung, bis er den vollen Nutzen aus der Methode ziehen kann.
Es ist offensichtlich am leichtesten, das Konturieren unter der Anleitung eines Lehrers zu lernen.
Die Idee hinter dem Konturieren ist, daß man zunächst den Zugang zur Musik findet und dadurch die Technik schneller folgt, weil Musik und Technik untrennbar sind.
In der Praxis erfordert es einige Arbeit, bevor das Konturieren nützlich werden kann.
Anders als das HS-Üben usw. kann es nicht so leicht gelernt werden.
Benutzen Sie es nur, wenn es absolut notwendig ist (wenn andere Methoden versagt haben).
Es kann hilfreich sein, wenn Sie es, nachdem Sie Ihre Arbeit mit HS beendet haben, schwierig finden, HT zu spielen.
Das Konturieren kann auch benutzt werden, um die Genauigkeit zu erhöhen und das \hyperref[c1iii6]{Auswendiglernen}\index{Auswendiglernen} zu verbessern.

Ich werde das Konturieren anhand von zwei sehr einfachen Beispielen verdeutlichen.
Allgemeine Methoden zur Vereinfachung sind:
\begin{enumerate}[label={\arabic*.}] 
\item Noten löschen
\item Läufe usw. in Akkorde verwandeln
\item Komplexe Passagen in einfachere umwandeln
 \end{enumerate}
Eine wichtige Regel ist, daß Sie, obwohl die Musik vereinfacht ist, im allgemeinen denselben Fingersatz beibehalten sollten, der vor der Vereinfachung erforderlich war.

Chopin benutzte in seiner Musik oft ein Rubato und andere Mittel, die eine ausgezeichnete Kontrolle und Koordination der beiden Hände erfordern.
In seiner \hyperref[c1iii2fi]{Fantaisie Impromptu (Op. 66)}\index{Fantaisie Impromptu (Op. 66)} können die sechs Noten jedes LH-Arpeggios (z.B. \textit{\hyperref[Noten]{C\#3}\index{C\#3}, G\#3, C\#4, E4, C\#4, G\#3}) zu zwei Noten vereinfacht werden (\textit{C\#3.E4} gespielt mit 51).
Es sollte nicht notwendig sein, die RH zu vereinfachen.
Das ist eine gute Möglichkeit, um sicherzustellen, daß alle Noten der beiden Hände, die auf denselben Schlag fallen, genau zusammen gespielt werden.
Auch wird dies Schülern, die Schwierigkeiten mit dem 3-4-Timing haben, erlauben, mit jeder Geschwindigkeit ohne diese Schwierigkeiten zu spielen.
Wenn Sie die Geschwindigkeit zunächst auf diese Art steigern, wird es später einfacher sein, sich das 3-4-Timing anzueignen, besonders wenn Sie nur einen halben Takt \hyperref[c1iii2]{zirkulieren}\index{zirkulieren}.

Die zweite Anwendung ist Beethovens Sonate \#1 (Op. 2, No. 1).
Ich habe im Quellenverzeichnis angemerkt, daß \hyperref[Gieseking]{Gieseking} nachlässig war, als er den vierten Satz trotz des schwierigen und sehr schnellen LH-Arpeggios mit \enquote{bringt keine weiteren neuen Probleme}\footnote{S. 38} bewertete.
Lassen Sie uns versuchen, die wunderbare Arbeit zu vervollständigen, die Gieseking mit der Einführung in diese Sonate geleistet hat, indem wir sicherstellen, daß wir diesen aufregenden Schlußsatz spielen können.

Die ersten vier Triolen der LH kann man lernen, indem man die \hyperref[c1iii7b]{Übungen für parallele Sets}\index{Übungen für parallele Sets} auf jede Triole anwendet und danach \hyperref[c1iii2]{zirkuliert}\index{zirkuliert}.
Die \hyperref[c1iii7b1]{Übung \#1 für parallele Sets}\index{Übung \#1 für parallele Sets} ist dabei nützlich (spielen Sie die Triolen als Akkorde), und üben Sie das \hyperref[c1ii14]{Entspannen}\index{Entspannen}.
Die erste Triole im dritten Takt kann auf die gleiche Art geübt werden, mit dem Fingersatz 524524.
Hier habe ich eine falsche Verbindung eingefügt, um ein leichtes, fortlaufendes Zirkulieren zu ermöglichen, damit man in der Lage ist, am schwachen vierten Finger zu arbeiten.
Wenn der vierte Finger stark und unter Kontrolle ist, können Sie die richtige Verbindung 5241 hinzufügen.
Hierbei ist der \hyperref[c1iii5]{Daumenübersatz}\index{Daumenübersatz} erforderlich.
Danach können Sie das absteigende \hyperref[Arpeggios]{Arpeggio}\index{Arpeggio} 5241235 üben.
Üben Sie das darauffolgende aufsteigende Arpeggio mit den gleichen Methoden, aber seien Sie darauf bedacht, beim aufsteigenden Arpeggio nicht den Daumenuntersatz zu benutzen, da dies sehr leicht geschehen kann.
Erinnern Sie sich an die Notwendigkeit eines geschmeidigen Handgelenks bei allen Arpeggios.
Für die RH können Sie die Regeln für das Üben von \hyperref[c1iii7e]{Akkorden}\index{Akkorden} und \hyperref[c1iii7f]{Sprüngen}\index{Sprüngen} benutzen (Abschnitte 7e und 7f weiter oben).
Bis jetzt ist alles ein Arbeiten mit HS.

Benutzen Sie für das HT-Spielen das Konturieren.
Vereinfachen Sie die LH, so daß Sie nur die Schlagnoten spielen (beginnend mit dem 2. Takt): \textit{F3, F3, F3, F3, F2, E2, F2, F3}, mit dem Fingersatz 55\textbf{5}155\textbf{5}1, der fortlaufend zirkuliert werden kann.
Das sind nur die ersten Noten jeder Triole.
Haben Sie das HS gemeistert, können Sie mit dem HT beginnen.
Wenn das mit HT zufriedenstellend gelingt, wird das Hinzufügen der Triolen einfacher, und es besteht die viel geringere Wahrscheinlichkeit, daß Sie dabei Fehler in sich aufnehmen.
Da diese Arpeggios die herausforderndsten Teile dieses Satzes sind, können Sie nun durch deren Konturieren den ganzen Satz mit jeder Geschwindigkeit üben.

Bei der RH sind die ersten drei Akkorde \textit{piano} und die zweiten drei \textit{forte}.
Üben Sie am Anfang hauptsächlich die Genauigkeit und die Geschwindigkeit, d.h. üben Sie alle sechs Akkorde \textit{piano}, bis dieser Abschnitt gemeistert ist.
Fügen Sie dann das \textit{forte} hinzu.
Um zu vermeiden, daß Sie die falschen Noten treffen, gewöhnen Sie sich an, die Tasten der Akkorde zu erfühlen, bevor Sie sie niederdrücken.
Stellen Sie bei der RH-Oktavmelodie der Takte 34-36 sicher, daß Sie ohne jegliches \textit{crescendo} spielen, besonders das letzte G.
Und die ganze Sonate wird natürlich ohne Pedal gespielt.
Um jede Möglichkeit eines katastrophalen Endes zu eliminieren, achten Sie darauf, daß Sie die letzten vier Noten dieses Satzes mit der LH spielen und diese immer ein gutes Stück früher als notwendig in Position bringen.

\label{pausen}\footnote{Anfänger (ich hoffe, nur diese) neigen oft dazu, an den Tasten zu \enquote{kleben}, die sie gerade zum Spielen benutzt haben, und die Hände erst im letzten Moment zu den als nächstes zu spielenden Tasten zu bewegen.
In vielen Stücken, genau wie in dieser Sonate, hat der Komponist an solchen Stellen freundlicherweise eine Pause, Staccato-Noten oder das Ende einer 
Phrase eingefügt, um Ihnen Zeit zu geben, die Position der Hand zu verändern.
\textbf{\enquote{Spielen} Sie die Bewegung der Hand während dieser Pausen ganz bewußt, und lernen Sie diese Bewegung so wie alle anderen Bewegungen auswendig.}}

\textbf{Für das Erwerben der Technik sind die anderen Methoden dieses Buchs üblicherweise effektiver als das Konturieren, das, auch wenn es funktioniert, zeitaufwendig sein kann.}
Wie bei dem obigen Beispiel der Sonate kann ein einfaches Konturieren Sie jedoch in die Lage versetzen, einen ganzen Satz mit der vorgegebenen Geschwindigkeit zu üben und die meisten musikalischen Gesichtspunkte einzubeziehen.
Währenddessen können Sie die anderen Methoden dieses Buchs benutzen, um sich die Technik anzueignen, die notwendig ist, um die Konturen zu füllen.



