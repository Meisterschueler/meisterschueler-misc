% File: c25e

\hypertarget{c2_5e}{}
\section{Mitschwingung}
\label{c2_5_mits} 

Die Genauigkeit, die erforderlich ist, um zwei Saiten in perfekte Stimmung zu bringen, ist so hoch, daß es eine fast unmögliche Aufgabe ist.
Es stellt sich heraus, daß es in der Praxis einfacher ist: \textbf{Wenn die Frequenzen sich in einem Bereich einander annähern, der \enquote{Mitschwingungsbereich} genannt wird, dann ändern die beiden Saiten ihre Frequenzen aufeinander zu, so daß Sie mit der gleichen Frequenz schwingen.}
Das geschieht, weil die beiden Saiten nicht unabhängig sind, sondern am Steg miteinander gekoppelt.
Wenn sie gekoppelt sind, dann bringt die Saite, die mit einer höheren Frequenz schwingt, die langsamere Saite dazu, mit einer etwas höheren Frequenz zu schwingen und umgekehrt.
Der Nettoeffekt ist, daß beide Frequenzen zur Durchschnittsfrequenz der beiden hin getrieben werden.
Somit wissen Sie, wenn Sie die Saiten 1 und 2 unisono stimmen, überhaupt nicht, ob sie perfekt gestimmt sind oder nur im Mitschwingungsbereich (außer wenn Sie ein erfahrener Stimmer sind).
Am Anfang werden sie wahrscheinlich nicht perfekt gestimmt sein.

Wenn Sie nun versuchen müßten, die dritte Saite nach den beiden Saiten zu stimmen, die in Mitschwingung sind, würde die dritte Saite die Saite, die ihr in der Frequenz am nächsten ist, in Mitschwingung versetzen.
Die andere Saite kann aber in bezug auf die Frequenz zu weit entfernt sein.
Sie wird aus der Mitschwingung ausbrechen und dissonant klingen.
Das Resultat ist, daß Sie, egal wo Sie sind, immer Schwebungen hören werden - der Stimmpunkt verschwindet!
Man könnte meinen, daß wenn die dritte Saite in der Durchschnittsfrequenz der beiden Saiten, die in Mitschwingung sind, gestimmt wäre, alle drei zur Mitschwingung übergehen sollten.
Es stellt sich heraus, daß das nicht geschieht, außer wenn alle drei Frequenzen perfekt gestimmt sind.
Wenn die ersten beiden Saiten genügend weit auseinander sind, erfolgt ein komplexer Energietransfer zwischen allen drei Saiten.
Sogar wenn die ersten beiden nah beieinander sind, gibt es höhere harmonische Schwingungen, die verhindern, daß alle Schwebungen verschwinden, wenn eine dritte Saite hinzukommt.
Zusätzlich gibt es häufig Fälle, in denen man nicht alle Schwebungen völlig eliminieren kann, weil die beiden Saiten nicht identisch sind.
Deshalb würde sich ein Anfänger völlig verirren, wenn er eine dritte Saite nach einem Paar Saiten stimmen sollte.
\textbf{Bis Sie es beherrschen, den Mitschwingungsbereich herauszufinden, stimmen Sie immer eine Saite nach einer, niemals eine nach zwei.}
Außerdem bedeutet, daß Sie 1 nach 2 und 3 nach 2 gestimmt haben, nicht, daß die drei Saiten \enquote{sauber} zusammen klingen werden.
Prüfen Sie es immer; wenn die Saiten nicht völlig \enquote{sauber} sind, müssen Sie die störende Saite finden und es erneut versuchen.

Beachten Sie den Gebrauch des Ausdrucks \enquote{sauber}.
Mit genügender Übung werden Sie bald aufhören, auf die Schwebungen zu hören; statt dessen werden Sie nach einem reinen Klang suchen, der sich irgendwo innerhalb des Mitschwingungsbereichs ergibt.
Dieser Punkt hängt davon ab, welche Arten von Obertönen jede Saite erzeugt.
Im Prinzip versuchen wir, wenn wir unisono stimmen, die Grundschwingungen zur Deckung zu bringen.
In der Praxis ist ein kleiner Fehler in den Grundschwingungen verglichen mit demselben Fehler in einer hohen Oberschwingung unhörbar.
Leider sind diese hohen Obertöne im allgemeinen keine exakten harmonischen Obertöne, sondern sind von Saite zu Saite unterschiedlich.
Wenn die Grundtöne übereinstimmen, erzeugen deshalb diese hohen Obertöne hochfrequente Schwebungen, die die Note \enquote{schmutzig} oder \enquote{blechern} machen.
Wenn die Grundtöne gerade so verstimmt sind, daß die Obertöne keine Schwebungen erzeugen, \enquote{versäubert} sich die Note.
\textbf{Die Realität ist sogar noch komplizierter, weil einige Saiten, besonders bei Klavieren niedrigerer Qualität, eine zusätzliche Eigenresonanz haben, was es unmöglich macht, bestimmte Schwebungen völlig zu eliminieren.}
Diese Schwebungen werden sehr ärgerlich, wenn man diese Note benutzen muß, um eine andere zu stimmen.
 

\hypertarget{c2_5f}{}
\section{Diese letze infinitesimale Bewegung ausführen}
\label{c2_5_infi}

Wir kommen nun zur nächsten Schwierigkeitsstufe.
Finden Sie eine Note nahe G5, die leicht außerhalb der Stimmung ist, und wiederholen Sie das oben für G3 angegebene Verfahren.
Die Stimmbewegungen für diese höheren Noten sind viel kleiner, was sie schwieriger macht.
Sie werden vielleicht nicht in der Lage sein, durch das Drehen des Wirbels eine ausreichende Genauigkeit zu erreichen.
Wir müssen eine neue Fertigkeit erlernen.
\textbf{Diese Fertigkeit erfordert, daß Sie auf die Tasten \enquote{hämmern}, benutzen Sie deshalb Ihre Ohrenschützer oder Ohrstöpsel.}

Typischerweise werden Sie bei Bewegung (4) erfolgreich sein, aber bei Bewegung (5) wird sich der Wirbel entweder nicht bewegen oder über den Stimmpunkt hinwegspringen.
\textbf{Damit die Saite sich in kleineren Schritten vorwärts bewegt, müssen Sie einen Druck auf den Stimmhammer ausüben, der knapp unter dem Punkt liegt, an dem der Wirbel springt.
Schlagen Sie nun die Note fest an, während Sie den gleichen Druck auf den Stimmhammer aufrechterhalten.}
Die zusätzliche Saitenspannung durch den harten Hammerschlag\footnote{Hammer der Klaviermechanik, nicht der Stimmhammer!} läßt die Saite ein kleines Stück vorwärtsgehen.
Wiederholen Sie das, bis sie perfekt gestimmt ist.
Es ist wichtig, niemals den Druck auf den Stimmhammer nachzulassen und den Druck während dieser wiederholten Vorwärtssprünge konstant zu halten, oder Sie werden\footnote{in bezug auf die Saitenfrequenz} schnell die Orientierung verlieren.
Wenn die Saite perfekt gestimmt ist und Sie den Hammer loslassen, könnte der Wirbel zurückspringen und die Saite leicht \textit{tiefer} werden lassen. Sie werden aus der Erfahrung heraus lernen müssen, wie weit er zurückspringt, und es während des Stimmvorgangs entsprechend kompensieren müssen.

Die Notwendigkeit, auf die Saite zu hämmern, damit sie sich vorwärts bewegt, ist ein Grund, warum man Stimmer oft auf die Tasten hämmern hört.
Es ist eine gute Idee, sich anzugewöhnen, die meisten Noten zu hämmern, weil das die Stimmung stabilisiert.
Der daraus resultierende Ton kann so laut sein, daß das Ohr geschädigt wird, und eines der Berufsrisiken von Stimmern ist ein Gehörschaden wegen des Hämmerns.
Die Lösung ist die Benutzung von Ohrenstöpseln.
Beim Hämmern werden Sie auch mit Ohrstöpseln die Schwebungen problemlos hören.
Das verbreitetste anfängliche Symptom eines Gehörschadens ist der Tinnitus (Klingeln im Ohr).
Sie können die zum Hämmern notwendige Kraft minimieren, indem Sie den Druck auf den Stimmhammer erhöhen.
Ein geringeres Hämmern ist auch erforderlich, wenn der Stimmhammer parallel zu den Saiten statt rechtwinklig dazu steht, und ein noch geringeres, wenn Sie ihn nach links zeigen lassen.
Das ist ein weiterer Grund, warum viele Stimmer ihren Stimmhammer eher parallel zu den Saiten benutzen als rechtwinklig dazu.
Beachten Sie, daß es zwei Möglichkeiten gibt, ihn parallel zu halten: zu den Saiten hin (12 Uhr) und von den Saiten weg (6 Uhr).
Experimentieren Sie mit unterschiedlichen Hammerpositionen, wenn Sie an Erfahrung gewonnen haben, da Ihnen das viele Möglichkeiten für das Lösen verschiedener Probleme eröffnet.
Mit dem beliebten 5-Grad-Kopf auf dem Hammer sind Sie z.B. nicht in der Lage, bei der höchsten Oktave den Griff nach rechts zeigen zu lassen, weil er auf den hölzernen Klavierrahmen treffen kann.
 

\hypertarget{c2_5g}{}
\section{Ausgleich der Saitenspannung}
\label{c2_5_span} 

\textbf{Das Hämmern hilft auch dabei, die Saitenspannung gleichmäßiger auf die ganzen nicht klingenden Abschnitte der Saite zu verteilen, wie z.B. den Bereich im Duplex, aber besonders den Abschnitt zwischen dem Capotasto (Druckstab) und der Agraffe.}
Es gibt eine Kontroverse darüber, ob der Ausgleich der Saitenspannung den Klang verbessert.
Es steht außer Frage, daß eine gleichmäßige Spannung die Stimmung stabiler macht.
Es kann jedoch fraglich sein, ob sie einen \textit{wesentlichen} Unterschied in der Stabilität ausmacht, besonders wenn die Wirbel während des Stimmens korrekt eingestellt wurden.
Bei vielen Klavieren sind die Duplex-Abschnitte fast völlig mit Filz gedämpft, weil sie unerwünschte Schwingungen erzeugen könnten.
Tatsächlich sind die \enquote{nicht klingenden} Abschnitte bei fast jedem Klavier gedämpft.
Anfänger müssen sich über die Spannung in diesen Abschnitten der Saiten keine Gedanken machen.
Deshalb ist das schwere Hämmern für den Anfänger nicht notwendig, obwohl es nützlich ist, diese Fertigkeit zu erlernen.

\textbf{\textit{Meiner persönlichen Meinung nach trägt der Klang des Duplex-Abschnitts nichts zum Klavierklang bei.}}
In Wahrheit ist dieser Klang unhörbar und wird im Baß, wo er hörbar würde, völlig abgedämpft.
Deshalb ist die \enquote{Kunst des Stimmens des Duplex-Abschnitts} ein Mythos, obwohl den meisten Klavierstimmern (einschließlich Reblitz!) von den Herstellern beigebracht wurde, daran zu glauben, weil es ein gutes Verkaufsargument abgibt.
Der einzige Grund, warum man den Duplex-Abschnitt stimmen sollte, ist, daß der Steg sowohl im Knoten des klingenden als auch des nicht klingenden Bereichs sein sollte; ansonsten wird das Stimmen schwierig, der Sustain wird eventuell verkürzt, und man verliert die Gleichmäßigkeit.
Wenn man Begriffe der Mechanik benutzt, kann man sagen, daß den Duplex-Abschnitt zu stimmen die Schwingungsimpedanz des Stegs optimiert.
Mit anderen Worten: Der Mythos ändert nichts an der Fähigkeit der Stimmer, ihren Job zu machen.
Nichtsdestoweniger ist ein gutes Verständnis sicher förderlich.
Der Duplex-Abschnitt wird benötigt, damit der Steg sich freier bewegen kann, nicht für die Tonerzeugung.
Offensichtlich verbessert der Duplex-Abschnitt die Klangqualität (des klingenden Bereichs), weil er die Impedanz des Stegs optimiert, aber nicht, weil er einen Ton erzeugt.
Die Tatsache, daß der Duplex-Abschnitt im Baß gedämpft und im Diskant völlig unhörbar ist, beweist, daß der Klang des Duplex-Abschnitts nicht benötigt wird.
Sogar im unhörbaren Diskant ist der Duplex-Abschnitt - um die Impedanz zu optimieren - in gewissem Sinne \enquote{gestimmt}, d.h. die Aliquotleiste ist so angebracht, daß die Länge des Duplex-Abschnitts der Saite eine harmonische Länge des klingenden Abschnitts der Saite ist (\enquote{aliquot} bedeutet \enquote{ohne Rest teilend}).
Wenn der Ton des Duplex-Abschnitts hörbar wäre, dann müßte der Duplex-Abschnitt genauso sorgfältig gestimmt werden wie der klingende Abschnitt der Saite.
Für das Anpassen der Impedanz muß das Stimmen jedoch nur annähernd genau sein, was in der Praxis auch der Fall ist.
Manche Hersteller haben diesen Mythos des Duplex-Abschnitts ins Lächerliche gesteigert, indem sie auf der Seite des Stimmwirbels einen zweiten Duplex-Abschnitt vorsehen.
Da der Hammer auf diesen Bereich (wegen des festen Capotasto) nur Zugkräfte übertragen kann, kann dieser Bereich der Saite nicht schwingen, um einen Klang zu erzeugen.
Folglich gibt praktisch kein Hersteller ausdrücklich an, daß der nicht klingende Abschnitt auf der Seite der Stimmwirbel gestimmt werden soll.


\hypertarget{c2_5h}{}
\section{Wiegen im Diskant}
\label{c2_5_disk}

\textbf{Die am schwierigsten zu stimmenden Noten sind die höchsten.}
Hier brauchen Sie beim Bewegen der Saiten eine unglaubliche Genauigkeit, und die Schwebungen sind schwer zu hören.
Anfänger können leicht den Bezugspunkt verlieren und es schwer haben, den Weg zurück zu finden.
Ein Vorteil der Notwendigkeit für solch kleine Bewegungen ist, daß Sie nun die wiegende Bewegung des Wirbels für das Stimmen benutzen können.
Da die Bewegung so klein ist, schädigt das Wiegen des Wirbels nicht den Stimmstock.
\textbf{Um den Wirbel zu wiegen, plazieren Sie den Stimmhammer parallel zu den Saiten, und lassen Sie ihn auf die Saiten zeigen (weg von Ihnen selbst).
Um \textit{höher} zu stimmen, ziehen Sie am Hammer nach oben, und um \textit{tiefer} zu stimmen, drücken Sie nach unten.}
Stellen Sie zuerst sicher, daß der Stimmpunkt nahe dem Mittelpunkt der wiegenden Bewegung ist.
Wenn er es nicht ist, dann drehen Sie den Wirbel so, daß er es ist.
Da diese Drehung viel größer ist als jene, die für das endgültige Stimmen benötigt wird, ist es nicht schwierig, aber denken Sie daran, den Wirbel richtig einzustellen.
Es ist besser, wenn der Stimmpunkt vor der Mitte ist (nach der Saite zu), aber ihn zu weit nach vorne zu bringen, würde das Risiko bedeuten, den Stimmstock zu beschädigen, wenn man versucht \textit{tiefer} zu stimmen.
Beachten Sie, daß \textit{höher} zu stimmen für den Stimmstock nicht so schädlich ist wie \textit{tiefer} zu stimmen, weil der Wirbel bereits gegen die Vorderseite des Lochs gedrückt ist.
 

\hypertarget{c2_5i}{}
\section{Grollen im Baß}
\label{c2_5_bass}

\textbf{Die tiefsten Baßsaiten sind (nach den höchsten Noten) jene, die am zweitschwierigsten zu stimmen sind.}
Diese Saiten erzeugen einen Ton, der zum größten Teil aus höheren Obertönen besteht.
Nahe dem Stimmpunkt sind die Schwebungen so langsam und leise, daß sie nur schwer zu hören sind.
Manchmal kann man sie besser \enquote{hören}, indem man sein Knie gegen das Klavier drückt, um die Vibrationen zu fühlen, als zu versuchen, sie mit den Ohren zu hören, besonders im einsaitigen Abschnitt.
Sie können das Unisono-Stimmen nur bis zum letzten zweisaitigen Abschnitt hinunter üben.
\textbf{Stellen Sie fest, ob sie die hochtönenden, metallischen, klingelnden Schwebungen erkennen können, die in diesem Abschnitt vorherrschend sind.}
Versuchen Sie, diese zu eliminieren, und sehen Sie, ob Sie ein wenig verstimmen müssen, um sie zu eliminieren.
Wenn Sie diese hohen, klingelnden Schwebungen hören können, bedeutet das, daß Sie auf dem richtigen Weg sind.
Machen Sie sich keine Gedanken, wenn Sie sie zunächst nicht einmal erkennen können - von Anfängern wird das nicht erwartet.
 

\hypertarget{c2_5j}{}
\section{Harmonisches Stimmen}
\label{c2_5_harm}

Wenn Sie mit Ihrer Fähigkeit unisono zu stimmen zufrieden sind, fangen Sie an, das Stimmen von Oktaven zu üben.
Nehmen Sie eine Oktave nahe des mittleren C und dämpfen Sie die beiden oberen Saiten jeder Note, indem Sie einen Keil zwischen ihnen einfügen.
Stimmen Sie die obere Note nach der Note eine Oktave unterhalb davon und umgekehrt.
Beginnen Sie wie beim Unisono nahe dem mittleren C, arbeiten Sie sich dann bis zu den höchsten Noten im Diskant vor, und üben Sie dann im Baß.
Wiederholen Sie die gleiche Übung mit den Quinten, Quarten und den großen Terzen.

\textbf{Nachdem Sie perfekte Harmonien stimmen können, versuchen Sie sie zu verstimmen, um festzustellen, ob Sie die zunehmenden Schwebungsfrequenzen hören können, wenn Sie ganz leicht von der perfekten Stimmung abweichen.}
Versuchen Sie, verschiedene Schwebungsfrequenzen zu identifizieren, insbesondere 1 bps (beats per second = Schwebungen je Sekunde) und 10 bps, indem Sie Quinten benutzen.
Diese Fertigkeiten werden sich später als nützlich erweisen.
 

\hypertarget{c2_5k}{}
\section{Was ist Streckung?}
\label{c2_5_stre}

Harmonisches Stimmen ist immer mit einem Phänomen verbunden, das Streckung genannt wird.
Harmonische Obertöne in Klaviersaiten sind niemals exakt, weil reale Saiten, die an realen Enden befestigt sind, sich nicht wie ideale mathematische Saiten verhalten.
Diese Eigenschaft der nicht exakten Obertöne nennt man Inharmonizität.
Die Differenz zwischen den tatsächlichen und den theoretischen harmonischen Frequenzen nennt man Streckung.
Experimentell findet man, daß die meisten harmonischen Obertöne im Vergleich zu ihren idealen theoretischen Werten \textit{höher} sind, obwohl es ein paar geben kann, die \textit{tiefer} sind.

Gemäß eines Untersuchungsergebnisses (Young, 1952) wird Streckung durch Inharmonizität verursacht, die aus der Steifheit der Saiten resultiert.
Ideale mathematische Saiten haben eine Steifheit von Null.
Steifheit ist eine extrinsische Eigenschaft - sie hängt von den Abmessungen des Drahtes ab.
Wenn diese Erklärung richtig ist, dann muß Streckung ebenfalls extrinsisch sein.
Wenn eine bestimmte Art Stahl vorgegeben ist, dann ist der Draht steifer, wenn er dicker oder kürzer ist.
Eine Konsequenz aus dieser Abhängigkeit von der Steifheit ist eine Steigerung der Frequenz mit der Zahl der \hyperref[moden]{Schwingungsmoden}; d.h. der Draht erscheint bei harmonischen Obertönen mit kürzeren Wellenlängen steifer.
Steifere Drähte vibrieren schneller, weil sie zusätzlich zur Saitenspannung eine weitere Rückstellkraft haben.
Diese Inharmonizität wurde mit einer Genauigkeit von einigen Prozent berechnet, so daß die Theorie richtig erscheint, und dieser einzelne Mechanismus scheint für den größten Teil der beobachteten Streckung verantwortlich zu sein.

Diese Berechnungen zeigen, daß die Streckung für die zweite Schwingungsmode bei C4 ungefähr 1,2 Cent beträgt und sich ungefähr alle 8 Halbtöne bei höheren Frequenzen verdoppelt (C4 = mittleres C, die erste Mode ist die tiefste oder Grundfrequenz, ein Cent ist ein hundertstel Halbton, und es gibt 12 Halbtöne in einer Oktave).
Die Streckung wird für tiefere Noten kleiner, besonders unterhalb von C3, weil die drahtumwickelten Saiten ziemlich flexibel sind.
Die Streckung nimmt schnell mit steigender Modenzahl zu und nimmt mit steigender Saitenlänge noch schneller ab.
Prinzipiell ist die Streckung bei größeren Klavieren kleiner und bei Klavieren mit geringerer Spannung größer, wenn Saiten mit dem gleichen Durchmesser benutzt werden.
Streckung führt zu Problemen beim Entwerfen von Tonleitern, weil abrupte Veränderungen des Saitentyps, Saitendurchmessers, der Länge, usw. eine diskontinuierliche Veränderung in der Streckung erzeugen.
Obertöne sehr hoher Moden bereiten, wenn Sie ungewöhnlich laut sind, wegen ihrer großen Streckung Probleme beim Stimmen - ihre Schwebungen herauszustimmen könnte die unteren, wichtigeren Obertöne hörbar aus der Stimmung bringen.

Da größere Klaviere oft eine geringere Streckung haben, aber auch dazu neigen, besser zu klingen, könnte man daraus schließen, daß eine kleinere Streckung besser ist.
Die Differenz der Streckung ist jedoch im allgemeinen gering, und die Klangqualität eines Klaviers wird zu einem großen Teil von anderen Eigenschaften als der Streckung kontrolliert.

Beim harmonischen Stimmen stimmt man z.B. die Grundfrequenz oder einen Oberton der oberen Note nach einem höheren Oberton der tieferen Note.
Die resultierende neue Note ist kein genaues Vielfaches der tieferen Note, sondern ist um den Betrag der Streckung \textit{höher}.
Das interessante an der Streckung ist, daß eine Tonleiter mit Streckung \enquote{lebhaftere} Musik erzeugt als eine ohne!
Das hat einige Stimmer veranlaßt, mit doppelten Oktaven statt mit einzelnen Oktaven zu stimmen, was die Streckung vergrößert.

Der Betrag der Streckung ist für jedes Klavier einzigartig und, in Wahrheit, einzigartig für jede Note des Klaviers.
Moderne elektronische Stimmhilfen sind genügend mächtig, um die Streckung für alle gewünschten Noten eines bestimmten Klaviers aufzuzeichnen.
Stimmer mit elektronischen Stimmhilfen können auch die durchschnittliche Streckung oder die Streckungsfunktion für jedes Klavier berechnen und das Klavier entsprechend stimmen.
Tatsächlich gibt es anekdotenhafte Berichte über Pianisten, die eine Streckung weit über der natürlichen Streckung des Klaviers wünschen.
Beim auralen Stimmen wird die Streckung natürlich und genau berücksichtigt.
Deshalb muß der Stimmer, obwohl die Streckung ein wichtiger Aspekt des Stimmens ist, nichts besonderes tun, um die Streckung einzubeziehen, wenn man nur die natürliche Streckung des Klaviers möchte.
 

\hypertarget{c2_5l}{}
\section{Präzision, Präzision, Präzision}
\label{c2_5_prae} 

\textbf{Das, worum es beim Stimmen geht, ist Präzision.}
Alle Stimmverfahren sind so angeordnet, daß man nacheinander die erste Note nach einer Stimmgabel stimmt, die zweite nach der ersten, usw.
Deshalb werden sich eventuelle Fehler schnell aufaddieren.
Tatsächlich wird ein Fehler an einem Punkt oft einige nachfolgende Schritte unmöglich machen.
Das geschieht, weil man auf den kleinsten Hinweis auf eine Schwebung hört, und wenn die Schwebungen in einer Note nicht vollständig eliminiert wurden, kann man sie nicht benutzen, um eine andere zu stimmen, weil diese Schwebungen klar zu hören sein werden.
Das wird bei Anfängern, bevor sie gelernt haben, wie präzise man sein muß, tatsächlich oft geschehen.
Wenn das geschieht, hört man Schwebungen, die man nicht eliminieren kann.
Gehen Sie in diesem Fall zu Ihrer Referenznote zurück, und stellen Sie fest, ob sie die gleiche Schwebung hören; wenn das so ist, ist dort der Ursprung Ihres Problems - beseitigen Sie es.

\textbf{Der beste Weg, die Präzision sicherzustellen, ist, die Stimmung zu prüfen.}
Fehler treten auf, weil jede Saite anders ist und Sie nie sicher sind, daß die Schwebung, die Sie hören, jene ist, nach der Sie suchen; das gilt besonders für Anfänger.
Ein weiterer Faktor ist, daß Sie die Schwebungen pro Sekunde (bps) zählen müssen, und Ihre Vorstellung von sagen wir 2 bps wird an verschiedenen Tagen oder zu verschiedenen Zeiten desselben Tags unterschiedlich sein, bis Sie sich diese \enquote{Schwebungsgeschwindigkeiten} gut gemerkt haben.
Wegen der entscheidenden Wichtigkeit der Präzision zahlt es sich aus, jede gestimmte Note zu prüfen.
Das gilt besonders, wenn Sie \enquote{\hyperref[c2_6]{die Bezugsnoten einstellen}}, was unten erklärt wird.
Unglücklicherweise ist die Note genauso schwierig zu prüfen, wie sie richtig zu stimmen ist; d.h. ein Person, die nicht hinreichend genau stimmen kann, ist üblicherweise unfähig, eine sinnvolle Prüfung durchzuführen.
Außerdem funktioniert das Prüfen nicht, wenn die Stimmung weit genug daneben ist.
Deshalb \textbf{habe ich Methoden gewählt, die ein Minimum an Prüfungen benutzen.}
Die resultierende Stimmung wird für die Gleichschwebende Temperatur zunächst nicht sehr gut sein.
Die \hyperref[c2_6_kirn]{Kirnberger-Stimmung} (s.u.) ist einfacher akkurat zu stimmen.
Auf der anderen Seite können Anfänger ohnehin keine guten Stimmungen erzeugen, unabhängig davon, welche Methoden Sie benutzen.
Zumindest werden die Verfahren, die unten vorgestellt werden, eine Stimmung bieten, die keine Katastrophe sein sollte und die besser wird, sobald sich Ihre Fertigkeiten verbessern.
\textbf{Tatsächlich ist das wahrscheinlich der schnellste Weg zum Lernen.}
Nachdem Sie sich genug verbessert haben, können Sie die Prüfungsverfahren untersuchen, wie jene, die bei Reblitz oder in \enquote{Tuning} von Jorgensen angegeben sind.
 


