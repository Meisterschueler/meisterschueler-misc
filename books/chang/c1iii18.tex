% File: c1iii18

\subsection{Wie man das Klavierspielenlernen beginnt - vom jüngsten Kind bis zum ältesten Erwachsenen}
\label{c1iii18}

\subsubsection{Benötigt man einen Lehrer?}
\label{c1iii18a}

Viele Anfänger möchten sich das Klavierspielen gerne selbst beibringen, und es gibt viele stichhaltige Gründe dafür.
Es steht jedoch völlig außer Frage, daß es für die ersten sechs Monate (und wahrscheinlich viel länger) keinen schnelleren Weg zum Anfangen gibt, als Stunden bei einem Lehrer zu nehmen, sogar bei einem, der die intuitive Methode lehrt.
Die einzigen Lehrer, die man gänzlich meiden sollte, sind diejenigen, die nicht das lehren können, was man spielen möchte (z.B. wenn Sie Pop, Jazz oder Blues spielen möchten, während der Lehrer nur klassische Musik unterrichtet), oder diejenigen, die strenge, inflexible Methoden lehren, die für den Schüler nicht angemessen sind (eine Methode kann für sehr junge Kinder entwickelt worden sein, aber Sie sind ein älterer Anfänger).
Warum sind Lehrer am Anfang so hilfreich?
Erstens sind die grundlegendsten Dinge, die Sie jedesmal beim Spielen benutzen, wie \hyperref[c1ii2]{Haltung der Hand}, \hyperref[c1ii3]{Sitzposition}, \hyperref[c1iii4]{Handbewegungen} usw., in einem Lehrbuch schwer zu erklären, während Ihnen ein Lehrer sofort zeigen kann, was richtig und was falsch ist.
Sie möchten sich keine dieser falschen Angewohnheiten aneignen und Ihr ganzes Leben damit zurechtkommen müssen.
Zweitens macht ein Anfänger, der sich ans Klavier setzt und zum ersten Mal spielt, mindestens 20 Fehler gleichzeitig (\hyperref[c1ii25]{Koordination der rechten und linken Hand}, Kontrolle der \hyperref[c1iii14d]{Lautstärke}, \hyperref[c1iii1b]{Rhythmus}, Arm- und \hyperref[c1iii4c]{Körperbewegungen}, \hyperref[c1ii13]{Geschwindigkeit}, Timing, \hyperref[c1ii18]{Fingersatz}, der Versuch das Falsche zuerst zu lernen, völliges Vernachlässigen der \hyperref[c1iii14d]{Musikalität} usw.).
Es ist die Aufgabe des Lehrers, alle Fehler zu erkennen und eine gedankliche Prioritätenliste derer zu erstellen, die als erste korrigiert werden müssen, so daß die schlimmsten schnell beseitigt werden können.
Die meisten Lehrer wissen auch, welche grundlegenden Fertigkeiten man benötigt, und lehren sie in der richtigen Reihenfolge.
Lehrer sind auch dabei hilfreich, das richtige Lehrmaterial zu finden.
Lehrer sorgen für eine strukturierte Lernumgebung, ohne die ein Student am Ende die falschen Dinge tut und nicht merkt, daß er keinen Fortschritt macht.
Kurz gesagt: Lehrer sind für Anfänger definitiv ihr Geld wert.


\subsubsection{Bücher für Anfänger; Keyboards}
\label{c1iii18b}

Wenn man anfängt, ist die Auswahl der Lehrbücher der erste Tagesordnungspunkt.
Diejenigen, die mit dem Erlernen der allgemeinen Technik anfangen möchten (kein Spezialgebiet wie Jazz oder Gospel), können jedes der zahlreichen Bücher für Anfänger, wie Michael Aaron, Alfred, Bastien, Faber und Faber, Schaum oder Thompson, benutzen.
Von diesen bevorzugen viele Faber und Faber.
Die meisten haben Bücher für Anfänger, die für Kinder oder Erwachsene entwickelt wurden.\footnote{Erfahrungsberichte und Vorschläge für den deutschen Sprachraum lasse ich gerne hier einfließen.}
Es gibt eine exzellente Website für Klavier (www.amsinternational.org/piano_pedagogy.htm), welche die meisten dieser Lehrbücher auflistet und viele davon bespricht.
In Abhängigkeit von Ihrem Alter und Ihrer bisherigen musikalischen Ausbildung können Sie diese Bücher in Ihrem eigenen Tempo durchgehen und Ihre Lernrate optimieren.

Diese Bücher für den Anfang werden Ihnen die Grundlagen beibringen: \hyperref[c1iii11]{Notenlesen}, verschiedene allgemeine \hyperref[c1ii18]{Fingersätze} wie \hyperref[c1iii5a]{Tonleitern}, \hyperref[Arpeggios]{Arpeggios} und Begleitungen usw.
Sobald Sie mit den meisten Grundlagen vertraut sind, können Sie damit beginnen, Stücke zu lernen, die Sie spielen möchten.
Hierbei sind Lehrer wieder äußerst wertvoll, weil sie die meisten Stücke kennen, die Sie vielleicht spielen möchten und 
Ihnen sagen können, ob sie zu der Schwierigkeitsstufe gehören, die Sie bewältigen können.
Sie können Ihnen die schwierigen Abschnitte herausstellen und Ihnen zeigen, wie Sie über diese Schwierigkeiten hinwegkommen.
Sie können Ihnen die Unterrichtsstücke vorspielen, um Ihnen zu zeigen, was sie versuchen müssen zu erreichen; meiden Sie Lehrer, die Ihnen nicht vorspielen können oder möchten.
Nach ein paar Monaten bis zu einem Jahr Unterricht werden Sie soweit sein, daß Sie mit dem Material im Buch weitermachen können.
Um die zahlreichen Fallen zu vermeiden, die auf sie lauern, sollten Sie das Buch zumindest einmal kurz durchlesen, bevor Sie mit der ersten Lektion beginnen.

Ganz am Anfang, vielleicht bis zu einem Jahr lang, ist es möglich, mit einem Keyboard zu lernen, auch mit einem kleineren mit weniger als den 88 Tasten des Standardklaviers.
Wenn Sie beabsichtigen, Ihr ganzes Leben lang elektronische Keyboards zu spielen, ist es sicherlich in Ordnung, wenn Sie nur auf Keyboards üben.
Im Grunde haben jedoch alle Keyboards eine Mechanik, die zu leicht ist, um wirklich ein \hyperref[c1iii17c]{akustisches Klavier} zu simulieren\footnote{\hyperref[c1iii17b]{elektronische Klaviere} (Digitalpianos) sind in dieser Hinsicht bedeutend besser als Keyboards}.
Sie werden so bald wie möglich auf ein Digitalpiano mit 88 gewichteten Tasten (oder ein akustisches Klavier) umsteigen wollen - siehe oben in \hyperref[c1iii17]{Abschnitt 17}.


\subsubsection{Anfänger im Alter von 0 bis über 65}
\label{c1iii18c}

Viele Eltern fragen: \enquote{In welchem Alter können unsere Kinder mit dem Klavierspielen beginnen?},
während ältere Anfänger fragen: \enquote{Bin ich zu alt, um das Klavierspielen zu lernen?
Wie gut werde ich spielen können?
Wie lange wird es dauern?}
Wir beginnen zunehmend zu erkennen, daß das, was wir dem \enquote{Talent} zugeschrieben haben, in Wahrheit das Ergebnis unserer Ausbildung war.
Diese relativ neue \enquote{Entdeckung} verändert die Landschaft der Klavierpädagogik radikal.
\textbf{Deshalb ist es legitim, wenn wir in Frage stellen, daß das Talent solch ein wichtiger Faktor dafür sei, wie schnell man das Spielen lernen könne.}
Was ist also ein wichtiger Faktor?
Das Alter ist einer, weil das Klavierspielenlernen ein Prozeß ist, bei dem - insbesondere im Gehirn - Nervenzellen gebildet werden.
Der Prozeß des Wachstums von Nervenzellen verlangsamt sich mit zunehmendem Alter.
Betrachten wir also die Kategorien der Anfänger und die Auswirkungen des sich verlangsamenden Zellwachstums in Abhängigkeit vom Alter.


\label{c1iii18c0}

\textbf{Alter von 0 bis 6:} Babys können hören, sobald sie geboren sind, und auf den meisten Geburtsstationen wir das Gehör der Babys gleich nach der Geburt getestet.
Die Gehirne tauber Babys entwickeln sich aufgrund des Mangels an akustischen Reizen langsam, und bei solchen Babys muß die Fähigkeit zur Verarbeitung akustischer Reize (wenn möglich) wieder hergestellt oder müssen andere Verfahren angewandt werden, um eine normale Entwicklung des Gehirns zu fördern.
Deshalb wird ein frühes musikalisches Stimulieren die Gehirnentwicklung normaler Babys beschleunigen, nicht nur hinsichtlich der Musik, sondern auch allgemein.
Im Alter von 6 bis 10 Monaten haben die meisten Babys eine genügende Menge Töne und Sprache gehört, die eine für den Beginn des Sprechens ausreichende Gehirnentwicklung stimulierte.
Sie können innerhalb weniger Minuten nach der Geburt schreien und mit uns kommunizieren.
Musik kann eine zusätzliche Stimulation bieten, um Babys innerhalb eines Jahres nach der Geburt einen enormen Vorsprung in der Gehirnentwicklung zu verschaffen.
Alle Eltern sollten eine gute Sammlung von Klaviermusik, Orchestermusik, Klavier- und Violinkonzerten, Opern usw. haben und sie im Zimmer des Babys oder irgendwo im Haus, wo das Baby die Musik immer noch hören kann, abspielen.
Viele Eltern flüstern und gehen leise, während das Baby schläft, aber das ist ein schlechtes Training.
Babys kann man darauf trainieren, in einer (normal) lauten Umgebung zu schlafen, und das ist die gesunde Alternative.

Bis zu einem Alter von ungefähr 6 Jahren erwerben sie neue Fertigkeiten schrittweise; d.h., sie erwerben plötzlich eine neue Fertigkeit wie das Laufen und werden schnell gut darin.
Aber jeder einzelne erwirbt diese Fertigkeiten zu unterschiedlichen Zeiten und in einer unterschiedlichen Reihenfolge.
Die meisten Eltern machen den Fehler, dem Baby nur Babymusik vorzuspielen.
Denken Sie daran: Kein Baby hat jemals Babymusik komponiert; Erwachsene haben es getan - Babymusik verlangsamt nur die Entwicklung des Gehirns.
Es ist keine gute Idee, sie lauten Trompeten und Schlagzeugen auszusetzen, die das Baby erschrecken können, aber Babys können Bach, Beethoven, Chopin usw. verstehen.
Musik ist ein erworbener Geschmack; deshalb hängt, wie sich das Gehirn der Babys musikalisch entwickelt, von der Art der Musik ab, die sie hören.
Ältere klassische Musik enthält mehr grundlegende Akkordstrukturen und Harmonien, die vom Gehirn auf natürliche Weise erkannt werden.
Später wurden komplexere Akkorde und Dissonanzen hinzugefügt, als wir im Laufe der Jahre vertrauter mit ihnen wurden.
Deshalb ist die ältere klassische Musik geeigneter für Babys, weil sie mehr stimulierende Logik und weniger Dissonanzen und Betonungen enthält, die später eingeführt wurden, um die \enquote{moderne Zivilisation} widerzuspiegeln.
Klaviermusik ist besonders geeignet, denn wenn sie schließlich Klavierunterricht nehmen, werden sie ein höheres Verständnis der Musik haben, die sie als Baby gehört haben.


\label{c1iii18c3}

\textbf{Alter von 3 bis 12:} Im Alter von weniger als drei Jahren sind die Hände der meisten Kinder für das Klavierspielen zu klein, die Finger können sich nicht unabhängig voneinander krümmen oder bewegen, und das Gehirn und der Körper (Stimmbänder, Muskel usw.) sind eventuell noch nicht genügend entwickelt, um die musikalischen Konzepte zu bewältigen.
Im Alter von mehr als vier Jahren können die meisten Kinder eine bestimmte Art von Musikerziehung erhalten, besonders wenn sie Musik seit der Geburt gehört haben; deshalb \textbf{sollten sie laufend auf ihr Verständnis der Tonhöhe (\hyperref[c1iii12]{relatives und absolutes Gehör}; können sie \enquote{eine Melodie halten}?), des \hyperref[c1iii1b]{Rhythmus}, von laut und leise, von langsam  und schnell, sowie des Notenlesens, das leichter zu lernen ist als jedes Alphabet, getestet werden}.
Diese Gruppe kann aus dem enormen Gehirnwachstum, das in dieser Altersspanne stattfindet, den vollen Nutzen ziehen;
das Lernen geschieht ohne Aufwand und wird eher von der Fähigkeit des Lehrers begrenzt, das entsprechende Material zu bieten, als durch die Fähigkeit des Schülers, das Material in sich aufzunehmen.
Ein bemerkenswerter Aspekt (es gibt viele!) dieser Altersgruppe ist ihre \enquote{Formbarkeit}; ihre \enquote{Talente} können geformt werden.
Deshalb können sie, auch wenn sie von alleine nie zu Musikern geworden wären, durch das richtige Training zu Musikern gemacht werden.
Das ist das ideale Alter, um mit dem Klavierspielen zu beginnen.
\hyperref[c1ii12mental]{Mentales Spielen} ist nichts besonderes - bei dieser Altersgruppe kommt es wie von selbst.
Viele Erwachsene halten das mentale Spielen für eine seltene Fertigkeit, weil sie es - wie das \hyperref[c1iii12]{absolute Gehör} - während ihrer Jugendzeit zu wenig benutzt und deswegen verloren haben.
Achten Sie deshalb darauf, daß die Schüler das mentale Spielen ausführen und die Musik in Gedanken spielen.


\label{c1iii18c13}

\textbf{Alter von 13 bis 19:} Die Jugendzeit.
Diese Gruppe hat immer noch eine ausgezeichnete Chance, die Stufe eines Konzertpianisten zu erreichen.
Sie haben aber eventuell die Chance verpaßt, zu diesen Superstars zu werden, zu denen die jüngeren Anfänger werden können.
Obwohl die Entwicklung des Gehirns sich verlangsamt hat, wächst der Körper ungefähr bis zum Alter von 16 Jahren weiterhin schnell und danach langsamer.
Die wichtigsten Faktoren sind hier die Liebe zur Musik und zum Klavier.
Mitglieder dieser Altersgruppe können praktisch alles erreichen, was sie möchten, solange sie ein intensives Interesse an der Musik haben.
Sie sind jedoch nicht mehr so formbar; sie anzuregen, das Klavierspielen zu lernen, funktioniert  nicht, wenn sie am Cello oder Saxophon mehr interessiert sind, und die Rolle der Eltern wechselt von der Vorgabe der Richtung hin zur Unterstützung dessen, was der Jugendliche tun möchte.
Das ist die Altersstufe, in der Jugendliche lernen, was es bedeutet, Verantwortung zu übernehmen und was es bedeutet, ein Erwachsener zu werden - alles Lektionen, die durch die Erfahrungen mit dem Klavier gelernt werden können.
Um sie zu beeinflussen, muß man fortgeschrittenere Methoden anwenden, wie z.B. Logik, Wissen und Psychologie.
Sie werden wahrscheinlich niemals etwas vergessen, das sie in diesem oder einem jüngeren Alter \hyperref[c1iii6]{auswendig gelernt} haben.
Oberhalb dieser Altersstufe wird die Einteilung nach dem Alter schwierig, weil es zwischen den einzelnen Menschen so viele Unterschiede gibt.


\label{c1iii18c20}

\textbf{Alter von 20 bis 35:}
Einige Mitglieder dieser Altersgruppe haben immer noch die Chance, die Stufe eines Konzertpianisten zu erreichen.
Sie können die Erfahrungen, die sie im Leben gemacht haben, nutzen, um sich die Fertigkeiten für das Klavierspielen effektiver als jüngere Schüler anzueignen.
Diejenigen, die sich in diesem Alter dafür entscheiden, das Klavierspielen zu lernen, haben im allgemeinen eine größere Motivation und eine klarere Vorstellung davon, was sie wollen.
Aber sie werden sehr hart arbeiten müssen, weil der Fortschritt sich nur mit einem ausreichenden Maß an Arbeit einstellt.
In dieser Altersgruppe kann die \hyperref[c1iii15]{Nervosität} für einige zu einem großen Problem werden.
Obwohl jüngere Schüler nervös werden können, scheint die Nervosität im Laufe der Jahre zuzunehmen.
Das geschieht, weil eine starke Nervosität aus der Angst zu versagen resultiert, und die Angst erwächst aus Assoziationen mit Erinnerungen an schreckliche Erlebnisse, egal ob diese real sind oder nur in der Vorstellung existieren.
Diese schrecklichen Erinnerungen oder Vorstellungen sammeln sich im Laufe der Zeit an.
Wenn Sie \hyperref[c1iii14]{vorspielen} möchten, sollten Sie sich deshalb mit der Kontrolle der Nervosität auseinandersetzen, z.B. indem Sie selbstsicherer werden oder indem Sie bei jeder Gelegenheit das öffentliche Vorspielen üben usw.
Die Nervosität kann sowohl aus dem Bewußtsein als auch dem Unterbewußtsein kommen; deshalb werden Sie sich mit beiden befassen müssen, um zu lernen, die Nervosität zu kontrollieren.
Diejenigen, die nur technisch versiert genug werden möchten, um das Spielen der großen Klavierwerke zu genießen, sollten, wenn sie in dieser Altersstufe beginnen, keine Probleme haben.
Obwohl einige \hyperref[c1iii6c]{Pflege} notwendig sein wird, können Sie alles ein Leben lang behalten, was Sie in dieser Altersstufe auswendig gelernt haben.


\label{c1iii18c35}

\textbf{Alter von 35 bis 45:}
Mitglieder dieser Altersgruppe können sich nicht zu Konzertpianisten entwickeln, können aber für einfacheres Material, wie leichte Klassik und Cocktail-Musik (\enquote{Fake Books}, Jazz), gut genug \hyperref[c1iii14]{vorspielen}.
Sie können genug Fertigkeiten erwerben, um die meisten berühmten Kompositionen zur eigenen Freude oder bei informellen Auftritten zu spielen.
Das anspruchsvollste Material wird wahrscheinlich außer Reichweite sein.
Die \hyperref[c1iii15]{Nervosität} erreicht irgendwo im Alter von 40 bis 60 Jahren ein Maximum und nimmt danach oft langsam ab. 
Das mag erklären, warum viele berühmte Pianisten irgendwann in dieser Altersstufe mit dem Auftreten aufgehört haben.
Das \hyperref[c1iii6]{Auswendiglernen} beginnt in dem Sinne zu einem Problem zu werden, daß es zwar möglich ist, praktisch alles auswendig zu lernen, man aber dazu neigt, es fast völlig zu vergessen, wenn man es nicht richtig \hyperref[c1iii6c]{pflegt}.
Das Notenlesen kann für einige zum Problem werden, die stark korrigierende Gläser benötigen, weil sich der Abstand der Augen zur Tastatur oder dem Notenständer zwischen dem Abstand zum Lesen und dem Fernblick befindet.
Deshalb benötigen Sie vielleicht eine Brille für den Zwischenbereich.
Gleitsichtgläser könnten das Problem lösen, aber manche finden sie wegen ihres kleinen Fokussierbereichs lästig.


\label{c1iii18c45}

\textbf{Alter von 45 bis 65:}
Das ist das Alter, in dem es je nach der Person zunehmende Einschränkungen dafür gibt, was man spielen lernen kann.
Man kann wahrscheinlich bis zur Stufe der Beethoven-Sonaten kommen, obwohl die schwierigsten eine große Herausforderung sein werden und es mehrere Jahre dauern wird, sie zu lernen.
Sich ein genügend großes Repertoire anzueignen wird schwierig sein, und man wird immer nur ein paar Stücke \hyperref[c1iii14]{vorspielen} können.
Aber es gibt unzählige Kompositionen, die man zur eigenen Freude spielen kann.
Da es mehr wunderbare Kompositionen zu lernen gibt, als man Zeit zum Lernen hat, muß man nicht unbedingt das Gefühl haben, daß man bei dem, was man spielen möchte, eingeschränkt ist.
Es gibt noch immer keine großen Probleme beim Lernen neuer Stücke, aber man muß sie ständig \hyperref[c1iii6c]{pflegen}, wenn man sie in seinem Repertoire behalten möchte.
Das wird Ihr spielbares Repertoire einschränken, da Sie beim Lernen neuer Stücke die alten völlig vergessen, wenn Sie diese nicht in viel jüngerem Alter gelernt haben.
Außerdem wird Ihre Lernrate definitiv anfangen abzunehmen.
Durch das mehrfache Vergessen und erneute \hyperref[c1iii6]{Auswendiglernen} können Sie trotzdem eine bedeutsame Menge Material auswendig lernen.
Es ist am besten, wenn Sie sich auf ein paar Stücke konzentrieren und lernen, diese gut zu spielen.
Es ist wenig Zeit für Bücher und Übungen für Anfänger - diese sind nicht schädlich, aber Sie sollten innerhalb weniger Monate nach dem Beginn des Unterrichts damit anfangen, Stücke zu lernen, die Sie spielen möchten.


\label{c1iii18c65}

\textbf{Alter von mehr als 65:}
Es gibt keinen Grund, warum man in irgendeinem Alter nicht mit dem Klavierspielenlernen beginnen können soll.
Diejenigen, die in diesem Alter anfangen, sehen es realistisch, was sie spielen lernen können und haben im allgemeinen keine unerfüllbaren Erwartungen.
Es gibt jede Menge einfache aber wundervolle Musik zum Spielen, und die Freude am Spielen bleibt genauso hoch wie sie in jungen Jahren war.
Solange Sie leben und nicht stark behindert sind, können Sie in jedem Alter das Klavierspielen lernen und zufriedenstellende Fortschritte machen.
Eine Komposition \hyperref[c1iii6]{auswendig zu lernen}, die man gerade übt, ist für die meisten kein Problem.
Die größte Schwierigkeit beim Auswendiglernen resultiert aus der Tatsache, daß man längere Zeit braucht, bis man bei schwierigem Material zur endgültigen Geschwindigkeit gekommen ist, und mit langsamem Spielen auswendig zu lernen ist die schwierigste Arbeit beim Auswendiglernen.
Deshalb werden Sie, wenn Sie leichte Stücke auswählen, die man leicht auf die endgültige Geschwindigkeit bringen kann, diese schneller auswendig lernen.
Die Hände zu dehnen, um weite Akkorde oder Arpeggios zu greifen, sowie schnelle Läufe werden schwieriger, und das \hyperref[c1ii14]{Entspannen} wird ebenfalls schwerer.
Wenn Sie sich jeweils nur auf eine Komposition konzentrieren, können Sie immer eine oder zwei Kompositionen haben, die Sie \hyperref[c1iii14]{vorspielen} können.
Es gibt keinen Grund, die Übungsmethoden zu ändern - es sind dieselben, die auch für jüngere Schüler benutzt werden.
Und Sie werden nicht so \hyperref[c1iii15]{nervös} sein, wie Sie eventuell in den mittleren Altersstufen waren.
Das Klavierspielenlernen, insbesondere die Arbeit am Gedächtnis, ist eine der besten Übungen für das Gehirn; deshalb sollten die ernsthaften Bemühungen beim Klavierspielenlernen den Alterungsprozeß verzögern, so wie das richtige körperliche Training notwendig ist, um die Gesundheit zu erhalten.
Nehmen Sie sich keinen Lehrer, der Sie wie einen jungen Anfänger behandelt und Ihnen nur Übungen gibt - dafür haben Sie keine Zeit.
Fangen Sie sofort damit an, Musik zu spielen.



