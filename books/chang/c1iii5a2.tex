% File: c1iii5a2

\label{c1iii5a2}
\subsubsection{Daumenübersatz üben: Geschwindigkeit, Glissandobewegung}
\label{c1iii5c}

\textbf{Wir besprechen nun Verfahren für das Üben schneller Tonleitern mit Daumenübersatz.}
Die aufsteigende C-Dur-Tonleiter der rechten Hand besteht aus den \hyperref[c1iv2]{parallelen Sets} 123 und 1234.
Benutzen Sie zunächst die \hyperref[c1iii7b]{Übungen für parallele Sets} (Abschnitt III.7b), um ein schnelles 123 zu erreichen, wobei die 1 auf dem \hyperref[Noten]{C4} ist.
Üben Sie dann 1231, wobei der Daumen hochgeht und dann hinter der 3 herunterkommt.
Bewegen Sie dabei die 3 schnell aus dem Weg, wenn der Daumen herunterkommt.
Der größte Teil der Seitwärtsbewegung des Daumens wird von der Bewegung der Hand beigetragen.
Die letzte 1 in 1231 ist die Verbindung, die aufgrund der \hyperref[c1ii8]{Kontinuitätsregel} (siehe Abschnitt II.8) erforderlich ist.
Wiederholen Sie es mit 1234, wobei die 1 auf F4 ist, und dann 12341, wobei die letzte 1 über die 4 rollt und auf C5 landet.
Spielen Sie mit den Fingern 2 bis 4 nahe an den schwarzen Tasten, um dem Daumen mehr Platz zum Landen zu geben.
Drehen Sie den Unterarm und das Handgelenk, sodass die Spitzen der Finger 2 bis 5 eine gerade Linie parallel zur Tastatur bilden; beim Spielen des mittleren C sollte der Unterarm dadurch einen Winkel von ungefähr 45 Grad zur Tastatur bilden.
Verbinden Sie dann die beiden parallelen Sets, um die Oktave zu vervollständigen.
Nachdem Sie eine Oktave können, spielen Sie zwei, usw.

\textbf{Wenn man schnelle Tonleitern spielt, sind die Hand- und Armbewegungen denen eines Glissandos ähnlich.}
Die glissandoartige Bewegung gestattet es, den Daumen sogar noch näher an die übergangenen Finger heranzubringen, weil die Finger 2 bis 5 leicht rückwärts zeigen.
Sie sollten auf diese Weise nach einigen Minuten Übung in der Lage sein (Machen Sie sich jetzt noch keine Gedanken über die Gleichmäßigkeit!), eine schnelle Oktave zu spielen (ungefähr 1 Oktave/Sekunde).
Üben Sie bis zu dem Punkt zu \hyperref[c1ii14]{entspannen}, an dem Sie das Gewicht Ihres Arms spüren können.
Wenn Sie den Daumenübersatz beherrschen, sollten Sie finden, dass lange Tonleitern nicht schwieriger sind als kurze, und dass \hyperref[c1ii25]{beidhändiges Spielen} mit dem Übersatz nicht so schwierig ist wie mit dem Untersatz.
Das geschieht, weil die Drehungen des Ellbogens usw. beim Untersatz schwierig werden, insbesondere am oberen und unteren Ende der Tonleitern (es gibt viele weitere Gründe).
An dieser Stelle muss betont werden, \textbf{dass es nicht notwendig ist, Tonleitern beidhändig zu üben, und dass sie beidhändig zu üben - bis man es ziemlich gut beherrscht - mehr Schaden anrichtet als es nützt}.
Es gibt soviel wichtiges Material, das \hyperref[c1ii7]{einhändig} geübt werden muss, so dass man durch das beidhändige Üben - außer bei kurzen Experimenten - wenig erreicht.
Die meisten fortgeschrittenen Lehrer (\hyperref[Gieseking]{Gieseking}) halten das beidhändige Üben von schnellen Tonleitern für eine Zeitverschwendung.

\textbf{Um die \hyperref[c1iv2a]{Phasenwinkel} (zeitliche Folge der einzelnen Finger) innerhalb des parallelen Sets exakt zu kontrollieren, heben Sie Ihr Handgelenk (ein ganz klein wenig), während Sie die parallelen Sets 123 oder 1234 spielen.
Machen Sie dann den Übergang zum nächsten parallelen Set, indem Sie das Handgelenk senken, um den Übersatz zu spielen.
Diese Bewegungen des Handgelenks sind extrem klein, für das untrainierte Auge fast nicht wahrnehmbar, und sie werden sogar kleiner, wenn Sie schneller werden.}
Sie können das gleiche auch erreichen, wenn Sie das Handgelenk im Uhrzeigersinn drehen, um die parallelen Sets zu spielen, und gegen den Uhrzeigersinn drehen, um den Daumen beim Zurückspielen zu senken.
Die Auf- und Abwärtsbewegung des Handgelenks ist jedoch gegenüber der Drehung zu bevorzugen, weil sie einfacher ist, und die Drehung kann für andere Zwecke reserviert werden (\hyperref[Sandor]{Sandor}).
Wenn Sie nun versuchen, mehrere Oktaven zu spielen, kann es zunächst wie ein Waschbrett klingen.

Der schnellste Weg, das Spielen von Tonleitern zu beschleunigen, ist, nur eine Oktave zu üben.
Wenn Sie die höheren Geschwindigkeiten erreicht haben, \hyperref[c1iii2]{zirkulieren} Sie zwei Oktaven auf und ab.
Bei hohen Geschwindigkeiten sind diese kürzeren Oktaven nützlicher, weil es schwieriger ist, die Richtung am Anfang und Ende umzukehren, wenn die Geschwindigkeit erhöht wird.
Die kurzen Oktaven geben Ihnen die Gelegenheit, die Richtungsänderungen öfter zu üben.
Bei längeren Läufen kommt man nicht so oft dazu, die Enden zu üben, und die zusätzliche Streckung des Arms, um die höheren und die tieferen Oktaven zu erreichen, ist eine unnötige Ablenkung von der Konzentration auf den Daumen.
\textbf{Der Weg zum Spielen schneller Umkehrungen am Anfang und am Ende ist, sie mit einem einzigen Abwärtsdrücken der Hand zu spielen.}
Um zum Beispiel am oberen Ende umzukehren, spielen Sie das letzte aufsteigende parallele Set, die Verbindung und das erste abwärts laufende parallele Set alle in einer Abwärtsbewegung.
In diesem Schema wird die Verbindung dadurch eliminiert, dass sie in eines der parallelen Sets eingebaut wird.
Das ist eine der effektivsten Arten, eine schnelle Verbindung zu spielen - indem man sie verschwinden lässt!

\textbf{Drehen Sie beim Glissando die Hände so ein- oder auswärts (\hyperref[c1iii4ProSup]{Pronation oder Supination}), dass die Finger von der Bewegungsrichtung der Hand weg zeigen.}
Nun sind die Anschlagsbewegungen der Finger nicht gerade nach unten gerichtet, sondern haben eine horizontale Rückwärtskomponente, die die Fingerspitzen in die Lage versetzt, ein wenig länger auf den Tasten zu verweilen, während die Hand an der Tastatur entlang bewegt wird.
Das ist besonders für das Legatospielen hilfreich.
Drehen Sie zum Beispiel bei der aufsteigenden Tonleiter der rechten Hand den Unterarm ein wenig im Uhrzeigersinn, sodass die Finger nach links zeigen.
Mit anderen Worten: Wenn die Finger (relativ zur Hand) gerade herunterkämen und die Hand sich bewegen würde, dann würden die Finger nicht gerade auf die Tasten herunterkommen.
Durch das leichte Drehen der Hand in die Glissandorichtung kann dieser Fehler kompensiert werden.
Somit gestattet die Glissandobewegung der Hand, sanft zu gleiten.
Sie können diese Bewegung durch das Auf- und Abwärtszirkulieren von einer Oktave üben; die Handbewegung sollte der eines Schlittschuhläufers gleichen, der mit den Füßen abwechselnd seitwärts tritt und dessen Körper sich abwechselnd nach beiden Seiten neigt, während er geradeaus gleitet.
Die Hand sollte mit jeder Richtungsänderung der Oktave ein- oder auswärts drehen.
So wie man sich beim Schlittschuhlaufen in die entgegengesetzte Richtung lehnen muss, bevor man die Bewegungsrichtung ändern kann, muss die Drehung der Hand (Umkehrung der Glissando-Handposition) dem Wechsel der Richtung der Tonleiter vorausgehen.
Diese Bewegung übt man am besten mit einer einzelnen Oktave.

Üben Sie für die absteigende Tonleiter der rechten Hand mit Daumenübersatz das parallele Set 54321 und die anderen relevanten Sets mit und ohne ihre Verbindungen.
\textbf{Sie müssen nur eine kleine Veränderung vornehmen, um zu vermeiden, dass der Daumen sich komplett unter die Hand falten kann, während das nächste parallele Set über den Daumen rollt.}
Heben Sie, während Sie die Tonleiter gleichmäßig halten, den Daumen so früh wie möglich, indem Sie das Handgelenk anheben und/oder drehen, um den Daumen hoch zu ziehen - fast das Gegenteil von dem, was Sie bei der aufsteigenden Tonleiter getan haben.
Wenn Sie den Daumen komplett unter die Hand falten, wird er gelähmt und ist schwer auf die nächste Position zu bewegen.
Das ist die \enquote{leichte Änderung}, die oben angesprochen wurde, und ist der Daumenbewegung für die aufsteigende Tonleiter ziemlich ähnlich.
Beim Spielen mit Untersatz darf sich der Daumen komplett unter die Handfläche falten.
\textbf{Weil diese Bewegung beim Daumenübersatz und -untersatz ziemlich ähnlich ist und sich nur graduell unterscheidet, kann sie leicht unkorrekt gespielt werden.}
Obwohl die Unterschiede in der Bewegung sichtbar gering sind, sollte der Unterschied im Gefühl für den Klavierspieler - besonders bei schnellen Passagen - wie Tag und Nacht sein.

Denken Sie bei superschnellen Tonleitern (mehr als eine Oktave je Sekunde) nicht in Begriffen von einzelnen Noten, sondern in Einheiten von parallelen Sets.
Benennen Sie bei der rechten Hand 123=A und 1234=B, und spielen Sie AB anstelle von 1231234, das heißt zwei Elemente anstelle von sieben.
Denken Sie bei noch schnellerem Spielen in Einheiten von Paaren paralleler Sets: AB, AB usw.
Wenn Sie in der Geschwindigkeit voranschreiten und anfangen, in größeren Einheiten zu denken, sollte die Kontinuitätsregel von A1 über AB1 zu ABA geändert werden (wobei das letzte A die Verbindung ist).
Es ist eine schlechte Idee, zu viel schnell zu üben, mit Geschwindigkeiten, die man nicht bequem handhaben kann.
\textbf{Die Ausflüge in sehr schnelles Spielen sind nur nützlich, um das genaue Üben mit einer geringeren Geschwindigkeit zu vereinfachen.
Üben Sie deshalb die meiste Zeit mit einer geringeren als der Maximalgeschwindigkeit; Sie werden auf diese Art schneller an Geschwindigkeit gewinnen.}

Versuchen Sie das folgende Experiment, um ein Gefühl für wirklich schnelle Tonleitern zu bekommen.
Zirkulieren Sie das fünffingrige parallele Set 54321 für die absteigende Tonleiter der rechten Hand nach dem Schema, wie es in den \hyperref[c1iii7b]{Übungen für parallele Sets} beschrieben wird (beginnen Sie mit \hyperref[c1iii7b1]{Übung \#1}).
Beachten Sie, dass Sie beim Steigern der Wiederholungsgeschwindigkeit die Hand ausrichten und ein gewisses Maß an Schub oder Drehung benutzen müssen, um das schnellste, flüssige und gleichmäßige parallele Spielen zu erreichen.
Sie müssen eventuell den Abschnitt über \hyperref[c1iii5SchubZug]{Schub und Zug bei Arpeggios} (Abschnitt f) weiter unten durchgehen, bevor Sie es korrekt ausführen können.
Ein Mittelstufenschüler sollte in der Lage sein, schneller als zwei Zyklen pro Sekunde zu werden.
Wenn Sie das erst einmal schnell, zufriedenstellend und entspannt können, spielen Sie einfach eine weitere Oktave mit derselben hohen Geschwindigkeit nach unten, und stellen Sie sicher, dass Sie alles mit Daumenübersatz spielen.
Sie haben gerade entdeckt, wie man einen sehr schnellen Lauf spielt!
Wie schnell Sie spielen können, hängt von Ihrer technischen Stufe ab, und wenn Sie besser werden, wird Ihnen diese Methode gestatten, sogar noch schnellere Tonleitern zu spielen.
Üben Sie diese schnellen Läufe nicht zu viel, wenn sie anfangen ungleichmäßig zu werden, weil Sie sonst am Ende eventuell die Angewohnheit haben, unmusikalisch zu spielen.
\textbf{Diese Experimente sind hauptsächlich für das Entdecken der Bewegungen wertvoll, die bei solchen Geschwindigkeiten benötigt werden, und das Gehirn zu trainieren, solche Geschwindigkeiten zu bewältigen.}
Gewöhnen Sie sich nicht an, schnell zu spielen und zuzuhören; stattdessen muss das Gehirn zuerst eine klare Vorstellung von dem haben, was erwartet wird, bevor Sie es spielen.

Am besten fängt man nicht an, Tonleitern beidhändig zu spielen, bis man einhändig sehr zufrieden ist.
Falls Sie der Meinung sind, dass Sie Tonleitern beidhändig üben müssen (manche benutzen sie zum Aufwärmen), beginnen Sie das beidhändige Üben mit einer Oktave oder einem Teil einer Oktave, zum Beispiel einem parallelen Set.
Die C-Dur-Tonleiter ist für das Üben mit parallelen Sets nicht ideal, weil die Daumen nicht synchron sind - benutzen Sie sie \hyperref[hdur]{H-Dur-Tonleiter}, bei der die Daumen der beiden Hände synchron sind (siehe nächster Absatz).
Pflegen Sie die Angewohnheit, mit einer hohen Geschwindigkeit zum beidhändigen Spielen überzugehen (obwohl es viel leichter erscheint, langsam zu starten und dann die Geschwindigkeit schrittweise zu steigern).
Spielen Sie dazu eine Oktave mehrere Male mit der linken Hand mit einer handhabbaren hohen Geschwindigkeit, wiederholen Sie mehrere Male mit der rechten Hand mit der gleichen Geschwindigkeit, und kombinieren Sie die Hände dann mit der gleichen Geschwindigkeit.
Machen Sie sich nichts daraus, wenn die Finger zunächst nicht perfekt zusammenpassen.
Bringen Sie zuerst die ersten Noten zur Deckung, danach die ersten und letzten Noten.
Zirkulieren Sie anschließend die Oktave fortlaufend.
Arbeiten Sie dann daran, jede Note zur Deckung zu bringen.
Üben Sie \textit{dann} entspannt mit langsamer Geschwindigkeit und behalten dieselben Bewegungen bei, bis die Tonleitern sehr genau und völlig kontrolliert sind.


\label{hdur}

\textbf{Bevor Sie mit der C-Dur-Tonleiter zu weit gehen, überlegen Sie sich, die H-Dur-Tonleiter zu üben.}
Sehen Sie dazu in der \hyperref[table]{Tabelle} weiter unten die Fingersätze der Tonleitern.
Bei dieser Tonleiter spielen nur der Daumen und der kleine Finger auf den weißen Tasten, außer beim tiefsten Ton der linken Hand (Finger 4).
Alle anderen Finger spielen auf den schwarzen Tasten.
Diese Tonleiter hat folgende Vorteile:

\begin{enumerate}[label={\arabic*.}] 
\item Sie ist zunächst einfacher zu spielen, besonders für jemanden mit größeren Händen oder langen Fingern.
Jede Taste kommt ganz natürlich unter die einzelnen Finger, und jeder Finger hat genug Platz.
Aus diesem Grund lehrte Chopin diese Tonleiter den Anfängern vor der C-Dur-Tonleiter.
\item Sie gestattet es Ihnen, das Spielen der schwarzen Tasten zu üben.
Die schwarzen Tasten sind schwieriger zu spielen (leichter zu verfehlen), weil sie schmaler sind, und erfordern eine größere Genauigkeit.
\item Sie erlaubt es, mit \hyperref[c1iii4b]{flacheren Fingern} (weniger gekrümmt) zu spielen, was zum Üben des Legatos und für die Klangkontrolle besser ist.
\item Das Spielen mit Daumenübersatz ist viel einfacher.
Das ist der Grund, warum ich die C-Dur-Tonleiter benutzte, um den Übersatz zu veranschaulichen.
Bei H-Dur ist es schwieriger, den Unterschied zwischen den Unter- und den Übersatzbewegungen zu sehen.
Um jedoch die richtigen Bewegungen zu üben, ist H-Dur eventuell überlegen, wenn Sie bereits den Unterschied zwischen Untersatz und Übersatz verstehen, weil es einfacher ist, zu den höheren Geschwindigkeiten zu kommen, ohne schlechte Angewohnheiten zu erwerben.
\item Die Daumen sind bei der H-Dur-Tonleiter synchron, was es ermöglicht, das beidhändige Spielen paralleles Set für paralleles Set zu üben.
Somit ist das beidhändige Spielen leichter als bei der C-Dur-Tonleiter.
Wenn Sie diese Tonleiter erst beidhändig beherrschen, wird das beidhändige Lernen der C-Dur-Tonleiter einfacher, was Ihnen Zeit spart.
Sie werden auch genau verstehen, warum die C-Dur-Tonleiter schwieriger ist.

 \end{enumerate}
\textbf{Dieser Abschnitt ist für diejenigen gedacht, die nur den Daumenuntersatz gelernt haben und nun den Übersatz lernen müssen.
Zunächst wird es Ihnen eventuell so vorkommen, als ob die Finger sich alle verknoten und es schwierig sei, eine klare Vorstellung davon zu bekommen, was Daumenübersatz ist.}
Der Hauptgrund für diese Schwierigkeit ist die Angewohnheit, die man beim Spielen mit Untersatz erworben hat und die nun verlernt werden muss.
Der Daumenübersatz ist eine neue Fertigkeit, die Sie lernen müssen, und er ist nicht schwieriger zu lernen als eine Bach-Invention.
\textbf{Die beste Nachricht von allen ist aber, dass Sie wahrscheinlich bereits wissen, wie man mit Daumenübersatz spielt!
Versuchen Sie, eine sehr \hyperref[c1iii5h]{schnelle chromatische Tonleiter} zu spielen.}
Beginnen Sie mit C, und spielen Sie 13131231313 usw.
Die \hyperref[c1iii4b]{flache Fingerhaltung} wird hierbei nützlich sein.
Wenn Sie eine sehr schnelle chromatische Tonleiter spielen können, dann ist die Daumenbewegung genau die, die Sie für den Übersatz benötigen, weil es unmöglich ist, eine sehr schnelle chromatische Tonleiter mit dem Untersatz zu spielen.
Verlangsamen Sie nun die Daumenbewegung der schnellen chromatischen Tonleiter und übertragen Sie diese auf die H-Dur-Tonleiter; betrachten Sie diese H-Dur-Tonleiter als eine chromatische Tonleiter, in der nur ein paar der weißen Tasten gespielt werden.
Wenn Sie die H-Dur-Tonleiter mit Übersatz spielen können, übertragen Sie diese Bewegung auf C-Dur.

Natürlich ist das Lernen von Tonleitern und \hyperref[Arpeggios]{Arpeggios} (siehe unten) mit Daumenübersatz nur der Anfang.
Dieselben Prinzipien sind auf jede Situation anwendbar, in die der Daumen einbezogen ist - in jedem Musikstück, an jeder Stelle, die ziemlich schnell ist.
Wenn die Tonleitern und Arpeggios erst einmal gemeistert sind, dann sollte in diesen anderen Situationen mit Daumenübersatz alles fast wie von selbst gehen.
Damit sich dies gewissermaßen automatisch entwickelt, müssen Sie gleichbleibende und optimierte Fingersätze für die Tonleitern benutzen; diese sind in den \hyperref[table]{Tabellen} weiter unten aufgelistet.

Diejenigen, für die der Daumenübersatz neu ist und die viele Stücke mit dem Daumenuntersatz gelernt haben, werden zurückgehen und alle alten Stücke überarbeiten müssen, die schnelle Läufe und gebrochene Akkorde enthalten.
Ideal wäre es, all die alten Stücke, die mit Untersatz gelernt wurden, zu wiederholen, um an den Stellen, an denen der Übersatz besser geeignet ist, die Angewohnheit des Untersatzes völlig loszuwerden.
Es ist eine schlechte Idee, einige Stücke mit Untersatz und andere mit Übersatz zu spielen, wenn sie ähnliche Fingersätze haben.
Eine Möglichkeit, die Umstellung zum Übersatz zu verwirklichen, ist, zunächst Tonleitern und Arpeggios zu üben, sodass Sie sich an den Übersatz gewöhnen.
Lernen Sie dann einige \textit{neue} Kompositionen, und benutzen Sie dabei den Übersatz.
Nach ungefähr sechs Monaten, wenn Sie sich an den Übersatz gewöhnt haben, können Sie damit beginnen, Ihre ganzen alten Stücke umzustellen.

Daumenübersatz und Daumenuntersatz sollten als die Extreme von zwei verschiedenen Arten, den Daumen zu benutzen, angesehen werden.
Das heißt, es gibt viele weitere Bewegungen dazwischen.
\textbf{Ein unerwarteter Nutzen des Lernens des Daumenübersatzes ist, dass man beim Untersatzspielen besser wird.
Das geschieht, weil Ihr Daumen technisch fähiger und geschickter wird: Er wird \underline{frei}.}
Und Sie gewinnen die Fähigkeit, all die Bewegungen zwischen Über- und Untersatz zu benutzen, die eventuell erforderlich sind, je nachdem welche weiteren Noten gespielt werden oder welche Art von Ausdruck Sie erzeugen möchten.
\textbf{Der Daumen hat nun die Freiheit, alle ihm zur Verfügung stehenden Bewegungen zu benutzen und für die Steuerung des Klangs.
Diese Freiheit sowie die Fähigkeit, nun technisch viel schwierigeres Material korrekt zu spielen, verwandeln den Daumen in einen sehr vielseitigen Finger.}


\subsubsection{Tonleitern: Herkunft, Namensgebung, Fingersätze}
\label{c1iii5d}

\textbf{Es wird in diesem Buch davon abgeraten, Tonleitern und Übungen stupide zu wiederholen.
Es ist jedoch von entscheidender Bedeutung, die Fertigkeit zu entwickeln, perfekte Tonleitern und Arpeggios zu spielen, um einige grundlegende Techniken und Standard-Fingersätze für das routinemäßige Spielen und das \hyperref[c1iii11]{Spielen vom Blatt} zu erwerben.}
Tonleitern und Arpeggios sollten in allen Dur- und Molltonarten geübt werden, bis Sie mit den Fingersätzen vertraut sind.
Sie sollten frisch und respekteinflößend klingen, nicht laut aber überzeugend; sie anzuhören sollte einem die Stimmung anheben.
Das wichtigste Ziel ist dabei, solange zu üben, bis die Fingersätze der einzelnen Tonleitern automatisiert sind. 

Lassen Sie uns, bevor wir mit den Fingersätzen fortfahren, einige grundlegende Eigenschaften von Tonleitern besprechen: die Namensgebung der Tonarten und die Frage, was eine Tonleiter ist.
\textbf{Es ist nichts magisches oder gar musikalisches an der C-Dur-Tonleiter; sie erwächst einfach aus dem Wunsch, so viele Intervalle wie möglich in eine Oktave zu fassen, die mit einer Hand gespielt werden kann.}
Das ist nur ein auf die Bequemlichkeit ausgerichtetes Designelement (so wie die modernsten Elemente in das Design eines jeden neuen Autos einfließen), das sowohl das Lernen des Klavierspielens als auch das Spielen vereinfacht.
Anhand der Größe der menschlichen Finger und Hände können wir annehmen, dass das größte Intervall acht Tasten umfassen sollte.
Wie viele Intervalle kann man darin unterbringen?
Wir benötigen die Oktave, sowie Terzen, Quarten, Quinten und Sexten.
Wenn wir mit C4 beginnen, haben wir nun E4, F4, G4, A4 und C5, also insgesamt sechs Noten, was nur noch Platz für zwei Noten mit einem Intervall von einem Ganztonschritt und einem Halbtonschritt lässt.
Beachten Sie, dass sogar die kleine Terz bereits als A4-C5 vorhanden ist.
Fügt man den Halbtonschritt nach C4 ein, benötigt man ein Vorzeichen (schwarze Taste) bei C4 und vier Vorzeichen bei C5, um die chromatische Tonleiter zu vervollständigen\footnote{das heißt nach den derzeitigen Notenbezeichnungen kämen auf die weißen Tasten die Noten C4-C\#4-E4-F4-G4-A4-C5 und auf die schwarzen Tasten die Noten D4-F\#4-G\#4-A\#4-H4}.
Es ist deshalb besser, den Halbtonschritt vor C5 einzufügen, sodass die Oktave mit zwei Vorzeichen bei C4 und drei bei C5 ausgewogener ist.
Das vervollständigt die Konstruktion der C-Dur-Tonleiter, einschließlich der Vorzeichen (Sabatella, Mathiew).

Für die Namensgebung ist es unglücklich, dass die C-Dur-Tonleiter auf der Tastatur mit dem C und nicht mit dem A beginnt.
Somit wechseln die Oktavnummern beim C, nicht beim A; deshalb tragen die Noten um C4 die Bezeichnungen ...A3-H3-C4-D4-E4...\footnote{die internationale Reihenfolge ist ...A3-B3-C4-D4-E4...}.
Bei jeder Tonleiter wird die erste Note als \textbf{Tonika} bezeichnet, das heißt C ist die Tonika der C-Dur-Tonleiter.
Die tiefste Note auf einer Tastatur mit 88 Tasten ist A0, und die höchste Note ist C8.


\label{table}

Die \textbf{Standard-Fingersätze für aufsteigende Durtonleitern} sind 12312345 (rechte Hand, eine Oktave) und 54321321 (linke Hand) für die C-G-D-A-E Dur-Tonleitern (mit jeweils 0-1-2-3-4 Kreuzen); diese Fingersätze werden im Folgenden mit S1 und S2 abgekürzt, wobei \textbf{S für \enquote{Standard} steht}.
Die Kreuze kommen in der Reihenfolge der Noten F-C-G-D-A hinzu (G-Dur hat F\#; D-Dur hat F\# und C\#; A-Dur hat F\#, C\# und G\#; usw.), und bei den F-B-Es-As-Des-Ges Dur-Tonleitern kommen die Be's in der Reihenfolge der Noten H-E-A-D-G-C hinzu; \textbf{jedes Intervall zwischen zwei aufeinander folgenden Buchstaben ist eine Quinte}.
Sie sind deshalb leicht zu merken, besonders wenn Sie ein Geigenspieler sind (die freien Saiten der Geige sind G-D-A-E).
Die Buchstaben erscheinen immer in der Reihenfolge G-D-A-E-H-F-C, das heißt im kompletten Quintenzirkel; diese Reihenfolge sollten Sie sich merken.
Schauen Sie sich die H-Dur- oder Ges-Dur-Tonleiter in einem Notenheft an, und Sie werden sehen, wie die fünf Kreuze oder sechs Be's in derselben Folge aufgereiht sind.
Somit stehen zwei Kreuze bei F-C, drei Kreuze stehen bei F-C-G, usw.
Die Be's nehmen in umgekehrter Reihenfolge wie die Kreuze zu.
Jede Tonleiter wird durch ihre \textbf{Tonartenvorzeichnung} identifiziert; so hat zum Beispiel die Vorzeichnung der G-Dur-Tonleiter ein Kreuz (F\#).
\textbf{Wenn Sie gelernt haben, eine Quinte zu erkennen, können Sie alle Tonleitern in der ansteigenden Reihenfolge der Kreuze erzeugen (indem Sie vom C aus in Quinten aufwärts gehen) oder in der absteigenden Reihenfolge der Be's (indem Sie in Quinten abwärts gehen)}; das ist nützlich, wenn Sie alle Tonleitern nacheinander üben möchten, ohne auf einen Ausdruck sehen zu müssen.
In der folgenden Tabelle sehen Sie die Fingersätze für die aufsteigenden Durtonleitern (kehren Sie die Fingersätze für die absteigenden Tonleitern um).

\label{enharmonisch}\footnote{In der Literatur sind manchmal auch die weiteren Tonarten mit mehr als fünf Kreuzen bzw. sechs Be's zu finden.
Diese können aber - zumindest auf dem Klavier - durch enharmonische Verwechslung aus den hier angegebenen erzeugt werden, zum Beispiel wird Cis-Dur mit 7 Kreuzen zu Des-Dur mit 12-7=5 Be's.}

<table border cellpadding=\enquote{7}>
 <tr>
  <td bgcolor=\enquote{\#E0E0E0}>Linke Hand</td>
  <td bgcolor=\enquote{\#E0E0E0}>Rechte Hand</td>
  <td bgcolor=\enquote{\#E0E0E0}>Tonarten</td>
  <td bgcolor=\enquote{\#E0E0E0}>Kreuze / Be's \\ 
 S2=54321321 & S1=12312341 & CGDAE & 0-4 Kreuze \\ 
 43214321321 & S1 & H & 5 Kreuze \\ 
 S2 & 12341231 & F & 1 Be \\ 
 32143213 & 41231234 & B & 2 Be's \\ 
 32143213 & 31234123 & Es & 3 Be's \\ 
 32143213 & 34123123 & As & 4 Be's \\ 
 32143213 & 23123412 & Des & 5 Be's \\ 
 43213214 & 23412312 & Ges & 6 Be's \\ 
</table>

\label{tablemoll}

Die Molltonleitern sind komplex, weil es drei davon gibt.
Es kann verwirrend sein, dass sie oftmals nur Molltonleitern genannt werden, ohne genau anzugeben, welche der drei jeweils gemeint ist.
Es werden auch verschiedene Bezeichnungen benutzt.
Die Molltonleitern wurden geschaffen, weil sie eine von den anderen Tonleitern abweichende Atmosphäre erzeugen.
Die einfachste Molltonleiter ist die \textbf{reine Molltonleiter} (auch \textbf{natürliche Molltonleiter} genannt); sie ist einfach, weil sie dieselbe Tonartenvorzeichnung wie die Durtonart hat aber die Tonika sechs Noten höher als die der Durtonleiter ist.
Ich finde es leichter, sich das als kleine Terz abwärts statt als eine Sexte aufwärts zu merken. 
Somit hat die reine Molltonleiter zu G-Dur ihre Tonika bei E, die Vorzeichnung ist F\#, und sie wird e-Moll genannt.\footnote{Gemeint ist die \textbf{Paralleltonart}, die aus denselben Noten wie die Durtonart besteht aber mit der sechsten Stufe der Durtonleiter beginnt. Wenn man die Noten der Durtonleiter und der parallelen reinen Molltonleiter durchnummeriert, haben die Noten mit derselben Nummer jeweils denselben Abstand voneinander: drei Halbtöne, das heißt eine kleine Terz. Davon zu unterscheiden ist die \textbf{Varianttonart}. Die Molltonleiter der Varianttonart beginnt mit derselben Note wie die Durtonleiter, das heißt die Varianttonart von G-Dur ist g-Moll mit den Noten G-A-B-C-D-Eb-F-G und der Vorzeichnung B-Eb.}
Die \textbf{harmonische  Molltonleiter} wird am häufigsten benutzt und entsteht, wenn man die siebte Note der reinen Molltonleiter um einen Halbton anhebt.
Die \textbf{melodische  Molltonleiter} entsteht, wenn man die sechste und die siebte Note der reinen Molltonleiter um einen Halbton anhebt.
Die sechste und die siebte Note werden meistens nur beim Aufsteigen angehoben und bleiben beim Absteigen unverändert.

\textbf{Die Fingersätze für die harmonischen Molltonleitern} (die letzte Spalte bezeichnet die geänderte Note\footnote{das heißt auf welchen Ton die 7. Stufe der reinen Molltonleiter jeweils um einen Halbtonschritt angehoben wird, um die harmonische Molltonleiter zu bilden.}; die harmonische a-Moll-Tonleiter ist A-H-C-D-E-F-G\#-A, und die parallele Dur-Tonart ist C):


<table border cellpadding=\enquote{7}>
 <tr>
  <td bgcolor=\enquote{\#E0E0E0}>LH</td>
  <td bgcolor=\enquote{\#E0E0E0}>RH</td>
  <td bgcolor=\enquote{\#E0E0E0}>Tonarten</td>
  <td bgcolor=\enquote{\#E0E0E0}>Kreuze / Be's</td>
  <td bgcolor=\enquote{\#E0E0E0}>Note \\ 
 S2 & S1 & A &   & Gis \\ 
 S2 & S1 & E & 1 Kreuz & Dis \\ 
 43214321 & S1 & H & 2 Kreuze & Ais \\ 
 43213214 & 34123123 & Fis & 3 Kreuze & Eis \\ 
 32143213 & 34123123 & Cis & 4 Kreuze & His \\ 
 32143213 & 34123123 & Gis & 5 Kreuze & Fisis \\ 
 S2 & S1 & D & 1 Be & Cis \\ 
 S2 & S1 & G & 2 Be's & Fis \\ 
 S2 & S1 & C & 3 Be's & H \\ 
 S2 & 12341231 & F & 4 Be's & E \\ 
 21321432 & 21231234 & B & 5 Be's & A \\ 
 21432132 & 31234123 & Es & 6 Be's & D \\ 
</table>
Wie bereits gesagt, ist an den Tonleitern nichts Magisches; sie sind einfach menschliche Erzeugnisse, die aus Bequemlichkeit erdacht wurden - nur ein Rahmen, in den wir die Musik einspannen.
Deshalb kann man eine beliebige Anzahl davon erzeugen, und die hier behandelten sind, obwohl sie häufig benutzt werden, nicht die einzigen.
\textit{[Informationen über weitere Tonleitern finden Sie unter anderem in Marc Sabatellas \enquote{A Jazz Improvisation Primer}: \hyperref[http://www.outsideshore.com/primer/primer/index.html]{das Original in Englisch} <font color=\enquote{blue} size=\enquote{-1}>(extern), \hyperref[http://msjipde.uteedgar-lins.de/index.html]{als deutsche Übersetzung} (extern).]}</font>

Man kann Tonleitern nicht zu gut spielen.
Wenn Sie Tonleitern üben, versuchen Sie immer, etwas Bestimmtes zu erreichen - weicher, leiser, deutlicher, schneller.
Bringen Sie die Hände zum Gleiten, die Tonleiter zum Singen; fügen Sie Farbe hinzu\footnote{gemeint ist der Gesamteindruck, der aus der Kombination von Dynamik, Rhythmus, Phrasierung usw. entsteht, zum Beispiel Gefühle oder die Beschreibung einer Landschaft}, Ausdrucksstärke oder das Gefühl von  Begeisterung.
Hören Sie auf, sobald Ihre Konzentration nachzulassen beginnt.
Es gibt keine maximale Geschwindigkeit beim parallelen Spielen.
Deshalb können Sie im Prinzip Ihr ganzes Leben lang die Geschwindigkeit und die Genauigkeit steigern - was ordentlich Spaß machen kann und sicherlich auch süchtig.
Wenn Sie Ihre Geschwindigkeit einem Publikum demonstrieren möchten, können Sie das wahrscheinlich mit Tonleitern und Arpeggios mindestens genauso gut wie mit jedem Musikstück.



