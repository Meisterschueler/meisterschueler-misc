% File: c1iii7f

\subsubsection{Sprünge}
\label{c1iii7f}

Viele Schüler beobachten, wie berühmte Pianisten diese schnellen, weiten Sprünge machen, und fragen sich, warum sie selbst es nicht können, egal wie viele Male sie es üben.
Es erscheint so, als ob diese großen Pianisten ohne Anstrengung von Position zu Position springen und Noten oder Akkorde flüssig spielen könnten, egal wo sie sich gerade befinden.
In Wirklichkeit machen die Pianisten mehrere Bewegungen, die zu schnell und zu fein sind, um vom Auge wahrgenommen zu werden, solange man nicht weiß, wonach man sehen muß.
\textbf{Sprünge bestehen im wesentlichen aus zwei Bewegungen:}

\begin{enumerate}[label={\arabic*.}] 
 \item die horizontale Verschiebung der Hand zur richtigen Position
 \item die wirkliche Abwärtsbewegung zum Spielen
\end{enumerate}

Es gibt noch zwei optionale Bewegungen: das Fühlen der Tasten und die Bewegung des Abhebens; diese werden unten erklärt.
Die kombinierte Bewegung sollte mehr wie ein umgekehrtes \enquote{U} als ein umgekehrtes \enquote{V} aussehen.

Schüler ohne Ausbildung für die Sprünge neigen dazu, die Hand in einer umgekehrten V-Bewegung zu führen.
Mit dieser Art von Bewegung (keine horizontale Beschleunigung) ist es schrecklich schwierig, eine Note genau zu treffen, weil man in einem beliebigen Winkel herunterkommt.
Dieser Winkel ist niemals der gleiche (sogar wenn derselbe Abschnitt ein anderes Mal erneut gespielt wird), weil er von der Entfernung des Sprungs, dem Tempo, wie hoch man die Hand anhebt usw. abhängt.
Zu üben, gerade herunterzukommen, ist schwer genug; kein Wunder, daß manche Schüler Sprünge als unmöglich ansehen, wenn sie alle diese Winkel üben müssen.
Deshalb ist es wichtig, am Ende des Sprungs gerade herunterzukommen (oder die Tasten unmittelbar bevor man sie spielt zu fühlen).

Schüler ohne Ausbildung für die Sprünge erkennen im allgemeinen auch nicht, daß die horizontale Bewegung in hohem Maß beschleunigt werden kann; \textbf{deshalb ist die erste Fertigkeit, die trainiert werden muß, die horizontale Bewegung so schnell wie möglich auszuführen, damit genug Zeit für das genaue Lokalisieren der Tasten übrigbleibt, wenn man am Ziel angekommen ist.}
Lokalisieren Sie die Tasten, indem Sie sie erfühlen, bevor Sie sie tatsächlich spielen.
\textbf{Die Tasten zu erfühlen ist die 3. Komponente eines Sprungs.}
Diese 3. Komponente ist optional, weil sie nicht immer notwendig ist und manchmal nicht genug Zeit dafür bleibt.
Wenn diese Kombination der Bewegungen perfekt ist, sieht es so aus, als würde sie in einer Bewegung ausgeführt.
Das kommt daher, daß Sie nur den Bruchteil einer Sekunde bevor Sie die Note spielen dort sein müssen.
Wenn man nicht übt, die horizontale Bewegung zu beschleunigen, neigt man dazu, den Bruchteil einer Sekunde später anzukommen als man müßte.
Dieser kaum wahrnehmbare Unterschied macht den ganzen Unterschied zwischen 100\% Genauigkeit und schlechter Genauigkeit aus.
Stellen Sie sicher, daß Sie auch bei langsamen Sprüngen die schnellen horizontalen Bewegungen üben.

Obwohl das Erfühlen der Tasten vor dem Spielen optional ist, werden Sie überrascht sein, wie schnell man es tun kann.
In den meisten Fällen hat man die Zeit dazu.
Deshalb ist es ein guter Grundsatz, \textit{immer} die Tasten zu erfühlen, wenn man Sprünge langsam übt.
Wenn Sie alle Fertigkeiten lernen, die hier aufgeführt sind, werden Sie jede Menge Zeit haben, um die Tasten zu erfühlen, sogar bei der endgültigen Geschwindigkeit.
Es gibt ein paar Fälle, in denen keine Zeit bleibt, die Tasten zu erfühlen, und in diesen Fällen können Sie genau spielen, wenn Sie die meisten der anderen Sprünge durch das Erfühlen genau lokalisiert haben.

\textbf{Die vierte Komponente des Sprungs ist das Abheben.}
Gewöhnen Sie sich an, immer schnell abzuheben, unabhängig von der Geschwindigkeit des Sprungs.
Es ist nichts falsch daran, weit vor der Zeit anzukommen.
Sie sollten das schnelle Abheben auch bei langsamen Sprüngen üben, damit Sie diese Fertigkeit bereits haben, wenn Sie schneller werden.
Beginnen Sie das Abheben mit einem kleinen, abwärts und seitwärts gerichteten Ausschlag des Handgelenks.
Obwohl es für das Spielen der Noten notwendig ist, daß Sie gerade herunterkommen, gibt es keine Notwendigkeit dafür, beim Abheben gerade nach oben zu gehen.
Offensichtlich ist der ganze Sprungvorgang so gestaltet, daß die Hand schnell, genau und reproduzierbar am Ziel ankommt, so daß viel Zeit übrig bleibt, um gerade nach unten zu spielen und die Tasten zu erfühlen.

\textbf{Das wichtigste Element, das Sie üben müssen, wenn Sie die Komponenten eines Sprungs erst einmal kennen, ist das Beschleunigen der horizontalen Bewegung.}
Sie werden überrascht sein, wie schnell man die Hand horizontal bewegen kann, wenn man sich nur auf diese Bewegung konzentriert.
Sie werden auch darüber verwundert sein, wieviel schneller Sie sie nach ein paar Tagen Übung bewegen können -- etwas, das einige Schüler im ganzen Leben nicht erreichen, weil man ihnen nie beigebracht hat es zu üben.
Diese Geschwindigkeit ist das, was die notwendige zusätzliche Zeit dazu beiträgt, eine 100\%-ige Genauigkeit zu sichern und alle anderen Komponenten des Sprungs ohne Anstrengung in sich aufzunehmen -- besonders das Erfühlen der Tasten.
Üben Sie das Erfühlen der Tasten wann immer es möglich ist, so daß es zu einer zweiten Natur wird und Sie nicht auf Ihre Hände sehen müssen.
Haben Sie es erst einmal einwandfrei in Ihr Spielen aufgenommen, werden die meisten Menschen, die Ihnen beim Spielen zusehen, es nicht einmal merken, daß Sie die Tasten erfühlen, weil Sie es im Bruchteil einer Sekunde tun können.
Wie ein vollendeter Magier werden Sie Ihre Hände schneller bewegen als das Auge sehen kann.
Die \hyperref[c1iii4b]{flachen Fingerhaltungen} sind dafür wichtig, weil Sie den empfindlichsten Teil der Finger für das Erfühlen der Tasten benutzen können und diese Haltungen die Genauigkeit beim Treffen der Tasten, insbesondere der schwarzen, erhöhen.

Da Sie nun die Komponenten eines Sprungs kennen, können Sie danach Ausschau halten, wenn Sie Konzertpianisten beim Auftritt zusehen.
Sie sollten nun in der Lage sein, jede Komponente zu erkennen, und Sie werden verblüfft sein, wie oft die Pianisten die Tasten erfühlen, bevor sie sie anschlagen und wie sie diese Komponenten in Windeseile ausführen.
Diese Fertigkeiten versetzen auch Sie in die Lage, ohne auf die Hände zu sehen zu spielen und weite Sprünge auszuführen.

Die beste Art, schnelle horizontale Bewegungen zu üben, ist, es ohne Klavier zu tun.
Setzen Sie sich so, daß der Ellbogen gerade nach unten und der Unterarm nach vorne zeigt.
\textbf{Bewegen Sie die Hand schnell seitwärts, indem Sie den Unterarm um den Ellbogen schwingen, wobei der Ellbogen am Ort bleibt.
Denken Sie wieder daran, am Ende der Bewegung völlig zu entspannen.
Lassen Sie nun den Unterarm gerade nach vorne zeigen und bewegen Sie ihn horizontal seitwärts (nicht in einem Bogen aufwärts), indem Sie den Oberarm um das Schultergelenk drehen und gleichzeitig die Schulter nach unten bewegen.}\footnote{Die Beschreibung der Bewegungen ist optisch zu verstehen.
Anatomisch geschieht folgendes:
Beim \enquote{Drehen des Unterarms um den Ellbogen} dreht sich nur der Oberarm im Schultergelenk und bleibt seitlich am Körper.
Der Unterarm wird nicht aktiv bewegt, d.h. Elle und Speiche behalten ihre relative Position zum Ellbogen.
Beim \enquote{horizontalen Seitwärtsbewegen des Unterarms} soll der Unterarm immer auf derselben Höhe und gerade nach vorne gerichtet bleiben.
Er wird entlang der gedachten Klavierkante seitwärts bewegt.
Dazu wird der Oberarm im Schultergelenk seitlich vom Körper weg rotiert.
Zum Ausgleich der daraus resultierenden Aufwärtsbewegung des Ellbogens muß die Schulter gesenkt bzw. der Oberkörper zur Seite geneigt werden.
Zusätzlich rotiert der Unterarm, d.h. Elle und Speiche, im Ellbogengelenk gegen die Drehrichtung des Oberarms, damit die Handfläche waagerecht bleibt.}
Beim wirklichen Sprung werden diese Bewegungen auf eine komplexe Art kombiniert.
Üben Sie die Bewegungen nach rechts und nach links so schnell Sie können, mit jeder der beiden Bewegungsarten einzeln und mit der Kombination der beiden.
\textit{Versuchen Sie nicht}, diese Bewegungen innerhalb eines Tages zu lernen.
Es ist möglich, sich dabei selbst zu verletzen, und bedeutende Verbesserungen lassen sich nur mit der \hyperref[c1ii15]{Automatischen Verbesserung nach dem Üben (PPI)} erzielen.

Wenn Sie gelernt haben, die horizontale Bewegung zu beschleunigen, werden die Sprünge sofort einfacher.
\textbf{Um den Streß zu reduzieren, entspannen Sie alle Muskeln, sobald die horizontale Bewegung vorbei ist.
Dasselbe ist auf die nachfolgende Abwärtsbewegung anwendbar -- entspannen Sie alle Muskeln, sobald die Noten gespielt sind}, und lassen Sie das Gewicht der Hand auf dem Klavier ruhen (heben Sie nicht die Hand bzw. die Finger von den Tasten).
Ein gutes Stück für das Üben dieser Sprünge mit der LH ist die 4. Variation in Mozarts berühmter A-Dur Sonate \#11 (KV331 bzw. K300i).
Diese Variation hat große Sprünge, in denen die LH über die RH kreuzt.
Ein beliebtes Stück, das Sie benutzen können, um Sprünge mit der RH zu üben, ist der 1. Satz von Beethovens Pathétique (Opus 13), direkt nach den Oktavtremolos der LH, in denen die RH Sprünge macht, die die LH kreuzen.

Üben Sie, die horizontale Bewegung zu beschleunigen, indem Sie mit langsamem Tempo spielen aber sich so schnell Sie können horizontal bewegen und dann über der richtigen Position anhalten und warten bevor Sie spielen.
Sie werden nun die Zeit haben, das Erfühlen der Noten vor dem Spielen zu üben, um 100\% Genauigkeit zu garantieren.
Die Idee ist hier, sich anzugewöhnen, immer vorzeitig an der Position anzukommen.
Steigern Sie das Tempo erst, wenn Sie überzeugt sind, eine schnelle horizontale Bewegung zu haben.
Alles was Sie tun müssen, um schneller zu werden, ist, einfach die Wartezeit vor dem Spielen der Noten zu reduzieren, wenn das Tempo steigt.
Wenn Sie fortgeschrittener werden, werden Sie immer \enquote{mindestens den Bruchteil einer Sekunde vorher ankommen}.
Kombinieren Sie dann schrittweise alle vier Sprungbewegungen zu einer gleichmäßigen Bewegung.
Nun sieht Ihre Bewegung genauso aus, wie diese der großen Pianisten, die Sie beneidet haben!
Besser sogar, Sprünge sind schließlich gar nicht so schwer, und man muß keine Angst vor ihnen haben.
 

\subsubsection{Weitere Übungen}
\label{c1iii7g}

Die meisten Dehnungsübungen für die großen Muskeln des Körpers sind hilfreich (s. \hyperref[Bruser]{Bruser}).
Eine Dehnungsübung für die Beugemuskeln (der Finger) kann folgendermaßen ausgeführt werden.
Drücken Sie mit der Handfläche der einen Hand die Finger der anderen Hand rückwärts zur Oberseite des Unterarms.
Menschen haben eine sehr unterschiedliche Gelenkigkeit, und einige werden in der Lage sein, die Finger ganz zurück zu drücken, so daß die Fingernägel den Arm berühren (180 Grad zur gestreckten Position!), während andere vielleicht in der Lage sind, nur ungefähr 90 Grad zurück zu drücken (die Finger zeigen bei horizontalem Arm aufwärts).
Die Fähigkeit, die Beugemuskeln zu strecken, nimmt mit zunehmendem Alter ab; deshalb ist es eine gute Idee, sie im Laufe des Lebens oft zu dehnen, um ihre Flexibilität zu erhalten.
Um die Streckmuskeln zu dehnen, drücken Sie die Finger zur Unterseite des Unterarms herunter.
Sie könnten diese Dehnungsübungen unmittelbar vor dem \enquote{\hyperref[c1iii6g]{kalt Spielen}} ausführen.

 Es gibt zahlreiche Übungen bei \hyperref[Sandor]{Sandor} und \hyperref[Fink]{Fink} (s. Quellenverzeichnis).
Diese sind interessant, weil jede Übung ausgewählt wurde, um eine bestimmte Handbewegung zu demonstrieren.
Zusätzlich werden die Bewegungen oft mit Passagen aus klassischen Kompositionen berühmter Komponisten veranschaulicht.



