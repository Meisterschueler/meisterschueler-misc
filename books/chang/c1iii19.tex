% File: c1iii19

\section{Der \enquote{ideale} Übungsablauf (Bachs Invention \#4)}
\label{c1iii19}

\textbf{Gibt es einen idealen, universellen Übungsablauf?
Nein,} weil jeder bei jeder Übungseinheit seinen eigenen Übungsablauf entwickeln muß.
Mit anderen Worten: \textbf{Dieses Buch handelt davon, wie Sie Ihre eigenen Übungsabläufe entwickeln können.}
Einige Unterschiede zwischen einem durchdachten und dem in \hyperref[c1ii1]{\ref*{c1ii1}} gezeigten intuitiven Ablauf werden im letzten Absatz dieses Abschnitts besprochen.
Ein guter Klavierlehrer wird während des Unterrichts die richtigen Übungsabläufe für die Übungsstücke mit Ihnen besprechen.
Diejenigen, die bereits wissen, wie man Übungsabläufe erstellt, werden diesen Abschnitt trotzdem interessant finden, weil wir zusätzlich zu den Übungsabläufen viele nützliche Punkte (wie Bachs Lehren und Details über das Üben der Invention \#4) besprechen.


\subsection{Die Regeln lernen}
\label{c1iii19a}

Deshalb ist der erste \enquote{Übungsablauf}, den Sie benutzen sollten, Kapitel 1 zu verfolgen.
Fangen Sie vorne an und wenden Sie die Konzepte auf eine Komposition an, die Sie spielen möchten.
Das Ziel ist, mit allen verfügbaren Übungsmethoden vertraut zu werden.
Wenn Sie erst ein wenig mit den meisten Übungsmethoden vertraut sind, sind wir bereit, Übungsabläufe zu entwickeln.
Um allgemein nützliche Abläufe zu entwickeln, nehmen wir an, daß Sie das Klavierspielen mindestens ein Jahr ernsthaft geübt haben.
Unser Ziel ist es, Bachs Invention \#4 zu lernen.


\subsection{Ein neues Stück lernen (Invention \#4)}
\label{c1iii19b}

\enquote{Ein neues Stück lernen} bedeutet, es auswendig zu lernen.
Fangen Sie deshalb ohne Aufwärmen usw. direkt mit dem \hyperref[c1iii6]{Auswendiglernen} von Bachs Invention \#4 an; zuerst mit der RH, beginnend mit Abschnitten von einem bis zu drei Takten, die eine Phrase bilden, dann mit der LH; weitere Details zu den einzelnen Schritten finden Sie in \hyperref[c1iii6l]{\ref*{c1iii6l}}.
Fahren Sie mit dem Vorgang fort, bis Sie das ganze Stück - nur HS - auswendig gelernt haben.
Diejenigen, die bereits gut darin sind, die Methoden dieses Buchs zu benutzen, sollten in der Lage sein, die ganze Invention (nicht perfekt) am ersten Tag innerhalb von einer bis zwei Stunden üben HS auswendig zu lernen (das gilt für eine durchschnittliche Person mit einem IQ von ungefähr 100).
Konzentrieren Sie sich nur auf das Auswendiglernen, machen Sie sich keine Gedanken, daß Sie etwas \enquote{nicht zufriedenstellend spielen können} (wie z.B. den 1,3-Triller in der LH), und spielen Sie mit einer beliebigen Geschwindigkeit, mit der Sie gut zurechtkommen.
Wenn Sie dieses Stück so schnell wie möglich auswendig lernen möchten, ist es am besten, wenn Sie sich nur auf dieses Stück konzentrieren und keine anderen Stücke spielen.
Anstelle einer langen Sitzung von 2 Stunden, könnten Sie zweimal am Tag eine Stunde üben.
Beginnen Sie am zweiten Tag das HT langsam und immer noch in Abschnitten von wenigen Takten; verbinden Sie diese dann.
Wenn Sie dieses Stück so schnell wie möglich auswendig lernen möchten, üben Sie wieder nichts anderes; sogar Fingerübungen zum Aufwärmen zu spielen wird dazu führen, daß Sie etwas von dem vergessen, was Sie gerade auswendig gelernt haben.


\subsection{\enquote{Normale} Übungsabläufe und Bachs Lehren}
\label{c1iii19c}  

Nach 3 oder 4 Tagen können Sie zu Ihrem \enquote{normalen} Übungsablauf zurückkehren.
Beim Ablauf für das Auswendiglernen haben wir im Grunde nichts anderes getan als auswendig zu lernen, weil der Prozeß des Auswendiglernens verlangsamt wird, wenn man das Auswendiglernen mit anderen Übungen mischt.
Beim \enquote{normalen} Ablauf können wir einen Vorteil aus dem Anfang ziehen, wenn die Hände noch kalt sind, und einige fertige Stücke \hyperref[c1iii6g]{\enquote{kalt} spielen}.
Natürlich können Sie schwierige, schnelle Stücke nicht kalt spielen.
Spielen Sie entweder leichtere Stücke oder spielen Sie die schwierigen langsam.
Ein gutes Verfahren ist, mit leichteren Stücken zu beginnen und schrittweise schwierigere zu spielen.
Wenn Sie im \hyperref[c1iii14]{Aufführen} stark genug geworden sind, so daß Sie keine Probleme haben kalt zu spielen (das mag ein Jahr dauern), wird dieser Schritt, besonders wenn Sie täglich Klavier spielen, optional.
Wenn Sie nicht täglich spielen, verlieren Sie eventuell die Fähigkeit kalt zu spielen, wenn Sie aufhören es zu üben.
Sie können während dieser Aufwärmphase auch Tonleitern und Arpeggios üben; Sie finden dazu in den Abschnitten \hyperref[c1iii4b]{\ref*{c1iii4b} (Mit flachen Fingern spielen)} und \hyperref[c1iii5]{\ref*{c1iii5} (Schnelle Tonleitern und Arpeggios)} nähere Details.
Sie könnten auch die Übungen zur \hyperref[c1iii7d]{Unabhängigkeit der Finger und dem Anheben der Finger} in \ref*{c1iii7d} versuchen; einige Klavierspieler führen diese Übungen regelmäßig ein- oder zweimal täglich aus.
Beginnen Sie damit, zusätzlich zu dem Bach-Stück andere Kompositionen zu lernen.

Zu diesem Zeitpunkt sollten Sie in der Lage sein, die gesamte Bach-Invention ohne Probleme in Gedanken HS zu spielen.
Das ist der richtige Zeitpunkt, um Stücke, die Sie bereits auswendig gelernt haben, zu überarbeiten (s. Abschnitte \hyperref[c1iii6c]{\ref*{c1iii6c}} und \hyperref[c1iii6f]{\ref*{c1iii6f}}), weil ein neues Stück zu lernen oft dazu führt, daß man Teile von zuvor gelernten Stücken vergißt.
Wechseln Sie beim Üben zwischen der Bach-Invention und Ihren alten Stücken.
Sie sollten die Invention die meiste Zeit HS üben, bis Sie sich die gesamte notwendige Technik angeeignet haben.
Steigern Sie die Geschwindigkeit - indem Sie kurze Abschnitte spielen - so schnell Sie es können auf Geschwindigkeiten, die schneller als die endgültige Geschwindigkeit sind.
Üben sie hauptsächlich die Abschnitte, die Ihnen Schwierigkeiten bereiten; es besteht keine Notwendigkeit, Abschnitte zu üben, die Ihnen leichtfallen.
Wenn Sie mit HS eine bestimmte Geschwindigkeit erreicht haben, fangen Sie damit an, HT mit einer niedrigeren Geschwindigkeit zu üben.
Sobald Sie mit dem HT bei niedrigeren Geschwindigkeiten zurechtkommen, können Sie es - wieder mit kurzen Abschnitten - auf höhere Geschwindigkeit bringen.
\textbf{Um die Geschwindigkeit zu steigern (HS oder HT), benutzen Sie nicht das Metronom oder zwingen Ihre Finger schneller zu spielen. Warten Sie, bis Sie das Gefühl bekommen, daß Ihre Finger schneller spielen \textit{wollen}, und erhöhen Sie dann die Geschwindigkeit um einen leicht zu bewältigenden Betrag.}
Das gestattet Ihnen, \hyperref[c1ii14]{entspannt} zu üben und alle Geschwindigkeitsbarrieren zu vermeiden.

\textbf{Entwickeln Sie für den Übergang vom HS- zum HT-Üben das Gefühl, daß die beiden Hände einander brauchen um zu spielen.}
Das wird Ihnen dabei helfen, die Bewegungen zu finden, die für das HT-Spielen hilfreich sind.
Das HS-Spielen ist sogar während des HT-Spielens nützlich; wenn Sie z.B. beim HT-Spielen mit der einen Hand einen Fehler machen, können Sie mit der anderen Hand weiterspielen und das Spielen mit der Hand, die den Fehler begangen hat, wieder aufnehmen, wann immer es möglich ist.
Ohne ausgedehntes HS-Üben wäre eine solche Leistung unmöglich.
Sie können solch ein Manöver als Teil des Auswendiglernens üben - warten Sie nicht bis zum Auftritt damit, zu versuchen es auszuführen!

Um die besonderen Techniken zu erwerben, die Bach im Sinn hatte, müssen wir die Invention detaillierter analysieren.
Bachs Inventionen wurden hauptsächlich als Übungsstücke für die Technik komponiert, und jede Invention lehrt uns bestimmte Techniken.
Deshalb müssen wir wissen, welche Arten von Techniken uns diese Invention lehren soll.
\textbf{Bach lehrt uns nicht nur besondere Fertigkeiten, sondern auch \textit{wie man sie übt}! Indem wir die Inventionen analysieren, können wir deshalb viele Übungsmethoden dieses Buchs lernen!}
Spielen Sie vor allem das ganze Stück mit \hyperref[c1iii5b]{Daumenübersatz}.
Beachten Sie, daß Bach ein Maximum an Kreuzungen des Daumens eingefügt hat, so daß wir viele Gelegenheiten haben, sie zu üben - offenbar ein absichtliches Konstrukt.
Üben Sie beim 212345 der RH in Takt 1 das Drehen um die 2 mit der Hand in der \hyperref[c1iii5c]{Glissando-Position}, um den Daumenübersatz zu erleichtern.

Das Hauptthema dieser Invention wird in den ersten 4 Takten der RH eingeführt.
Es wird dann von der LH wiederholt.
\textbf{Bach sagt uns, wir sollen HS üben!}
Beide Hände spielen im Grunde dasselbe, was uns die Gelegenheit gibt, die technische Fertigkeit der beiden Hände einander anzugleichen; das kann nur dadurch erreicht werden, daß man HS übt und der schwächeren Hand mehr Arbeit gibt.
Es gibt keine bessere Möglichkeit, die Unabhängigkeit der Hände zu üben - die wichtigste Lektion der Inventionen -, als die Hände getrennt zu üben.
Der Abschnitt, in dem eine Hand trillert, wäre unheimlich schwierig, wenn man ihn von Anfang an HT üben würde, während er HS ziemlich einfach ist.
Einige Schüler, die das HS-Üben nicht kennen, werden versuchen, die beiden Hände zur Deckung zu bringen, indem Sie die Noten des Trillers vorher ermitteln und sie dann für das HT-Üben verlangsamen.
Das mag für Anfänger oder Kinder angemessen sein, die das Trillern noch nicht gelernt haben.
Die meisten Schüler sollten von Anfang an (HS) trillern und daran arbeiten, die Triller so rasch wie möglich zu beschleunigen.
Es ist nicht notwendig, die beiden Hände mathematisch zur Deckung zu bringen; das ist Kunst, keine Mechanik!
Bach möchte, daß Sie mit jeder Hand unabhängig von der anderen trillern.
Das wird Ihnen gestatten, diese Invention mit jeder Geschwindigkeit zu spielen, ohne daß Sie die Trillergeschwindigkeit spürbar ändern müssen.
Der Grund, warum man die Noten nicht zur Deckung bringen muß, ist, daß diese Triller nur ein Mittel sind, die Noten längere Zeit auszuhalten und die einzelnen Noten keine rhythmische Bedeutung haben.
Was tun Sie, wenn Sie am Ende des Trillers mit der falschen Note aufhören?
Sie sollten in der Lage sein, das zu kompensieren, indem Sie entweder kurz warten oder die Geschwindigkeit des Trillers am Ende ändern - das ist die Art von Fertigkeit, die diese Invention lehrt.
Deshalb würde es die Lektion aus dieser Invention zunichte machen, wenn man üben würde, den Triller mit der anderen Hand zur Deckung zu bringen.
Das Staccato in den Takten 3 und 4 der RH ist ein weiteres Mittel, die Unabhängigkeit der Hände zu üben; Staccato in der einen Hand gegen Legato in der anderen erfordert mehr Kontrolle als Legato in beiden Händen.
Das Staccato sollte im ganzen Stück benutzt werden, obwohl es in vielen Ausgaben nur am Anfang angegeben ist.

Die meisten Unterrichtsstücke von Bach lehren nicht nur die Unabhängigkeit der Hände, sondern auch die Unabhängigkeit der Finger einer Hand, besonders des vierten Fingers.
So sind in den Takten 11 und 13 der RH sechs Noten, die als zwei Triolen gespielt werden könnten aber in Wirklichkeit wegen des 3/8-Taktes drei Zweiergruppen sind.
Diese Takte können für den Anfänger schwierig sein, weil sie die Koordination von drei verschiedenen Bewegungen erfordern:

\begin{enumerate}[label={\roman*.}] 
 \item Der Aufbau des Fingersatzes der RH ist der von zwei Triolen (\textbf{3}45\textbf{3}45 Rhythmus), muß aber als drei Zweiergruppen gespielt werden (\textbf{3}4\textbf{5}3\textbf{4}5).
 \item Gleichzeitig muß die LH etwas völlig anderes spielen.
 \item All das muß hauptsächlich mit den drei schwächsten Fingern - 3, 4 und 5 - durchgeführt werden.
\end{enumerate}

Bach benutzte dieses Mittel häufig, um uns zu zwingen, einen Rhythmus zu spielen, der sich vom Aufbau des Fingersatzes unterscheidet, um die Unabhängigkeit der Finger zu entwickeln.
Er versucht auch, dem vierten Finger soviel Arbeit wie möglich zu geben, wie z.B. im \textbf{4}5 am Ende.

Die Triolen sind mit dem Fingersatz 234 leichter zu spielen als mit 345, besonders mit großen Händen, und die meisten Ausgaben empfehlen den Fingersatz 234, weil die meisten Herausgeber das Konzept der parallelen Sets nicht kannten.
Die Kenntnis der \hyperref[c1iii7b]{Übungen für parallele Sets} zeigt jedoch, daß Bachs ursprüngliche Absicht 345 war (für einen maximalen Wert für die technische Entwicklung), und es ist eine \enquote{musikalische Freiheit}, den Fingersatz in 234 zu ändern, um die Musikalität zu vereinfachen.
In jeder anderen Komposition als dieser Invention wäre 234 der korrekte Fingersatz.
Der Gebrauch von 234 kann hier gerechtfertigt werden, weil es den Schüler das Prinzip lehrt, den Fingersatz mit der größten Kontrolle auszuwählen.
Deshalb kann der Schüler beide Fingersätze wählen.
Eine ähnliche Situation tritt in Takt 38 auf, in dem Bachs ursprüngliche Absicht für die LH wahrscheinlich 154321 war (ein vollständigeres paralleles Set), während die musikalische Freiheit 143212 anzeigt, was technisch weniger herausfordernd ist.
Ohne die Hilfe der Übungen für parallele Sets wäre die offensichtliche Wahl die musikalische Freiheit.
Durch die Anwendung der Übungen für parallele Sets kann der Schüler lernen, beide Fingersätze mit gleicher Leichtigkeit zu benutzen.

Die \enquote{Triolen im 3/8-Takt} sind ein gutes Beispiel, wie das fehlerhafte Lesen der Noten es schwierig macht, auf Geschwindigkeit zu kommen, und zu Geschwindigkeitsbarrieren führt.
Wenn man HT spielt, trifft man auf Probleme, wenn man die RH-Triolen auf zwei Schläge spielt (falsch) und die LH in drei (richtig).
Sogar wenn man einen zweiten Fehler begeht und die LH in zwei Schlägen spielt, um sie an die RH anzupassen, gibt es wegen der rhythmischen Änderung ein Problem mit den nachfolgenden Takten.
Man ist vielleicht bei niedriger Geschwindigkeit in der Lage, durch diese Fehler hindurchzuspielen, aber wenn man schneller wird, dann wird es unmöglich es zu spielen, und man baut eine Geschwindigkeitsbarriere auf.
Das ist ein Beispiel für die \hyperref[c1iii1b]{Wichtigkeit des Rhythmus}.
Es ist erstaunlich, wie viele Lektionen Bach in etwas hineinstecken konnte, das so einfach aussieht, und diese Komplexität erklärt teilweise, warum \textbf{viele Schüler es ohne die richtigen Übungsmethoden oder die Anleitung eines erfahrenen Lehrers unmöglich finden, Bach \hyperref[c1iii6]{auswendig zu lernen} oder seine Kompositionen jenseits einer bestimmten Geschwindigkeit zu spielen.
Der Mangel an richtigen Übungsmethoden ist der Hauptgrund, warum so viele Schüler so wenige Stücke von Bach spielen.}

Die Inventionen sind ausgezeichnete technische Unterrichtsstücke.
\hyperref[c1iii7h]{Hanon}, Czerny usw. versuchten dasselbe zu erreichen, indem sie das benutzten, was sie für einen einfacheren, systematischeren Ansatz hielten, aber sie versagten, weil sie versuchten, etwas zu vereinfachen, was außerordentlich komplex ist.
Im Gegensatz dazu packte Bach, wie oben gezeigt, so viele Lektionen in jeden Takt wie er konnte.
Hanon, Czerny usw. muß die Schwierigkeit Bach zu lernen bewußt gewesen sein, aber sie kannten die guten Übungsmethoden nicht und versuchten, indem sie ihren intuitiven Instinkten folgten, einfachere Methoden zum Erwerb der Technik zu finden.
Das ist eines der besten historischen Beispiele für die Fallen der intuitiven Vorgehensweise.

Weil die Inventionen für das Lehren bestimmter Fertigkeiten komponiert wurden, können sie etwas gezwungen klingen. 
Trotz dieser Gezwungenheit enthalten alle von Bachs Unterrichtsstücken mehr Musik als praktisch alles, was jemals komponiert wurde, und es gibt genug davon, um die Bedürfnisse von Schülern aller Stufen zu befriedigen, einschließlich von Anfängern.
Wenn die Inventionen zu schwierig sind, denken Sie darüber nach, die große Zahl wunderbarer (und vorzüglich aufführbarer) einfacheren Unterrichtsstücke, die Bach komponiert hat, zu studieren.
Die meisten davon finden Sie im \enquote{Notenbüchlein für Anna Magdalena Bach} (seiner zweiten Frau).
Da es so viele davon gibt, enthalten die meisten Bücher nur eine kleine Auswahl davon.
\textbf{Da die Inventionen Unterrichtsstücke sind, werden in fast jeder Ausgabe die kritischen Stellen der Fingersätze angegeben.}
Deshalb sollte es kein Problem sein, die Fingersätze herauszufinden, was extrem wichtig ist.
\enquote{J. S. Bach, Inventions and Sinfonias} von Willard A. Palmer, Alfred, CA, (www.alfredpub.com) zeigt alle nicht offensichtlichen Fingersätze und enthält auch einen Abschnitt über das Spielen der Verzierungen.

Die Inventionen wurden komponiert, indem wohldefinierte Abschnitte aneinandergefügt wurden, die üblicherweise nur ein paar Takte lang sind.
Das macht sie für das abschnittsweise HS-Üben ideal, einem weiteren Schlüsselelement der Methoden dieses Buchs.
Diese und viele andere Eigenschaften von Bachs Kompositionen machen sie zu einer idealen Musik, um die Methoden dieses Buchs zu lernen, und es ist ziemlich wahrscheinlich, daß sie mit dem Gedanken an diese Übungsmethoden komponiert wurden.
Bach war wohl das meiste Material in diesem Buch bekannt!

\textbf{Eine weitere wichtige Lektion von Bachs Inventionen sind die parallelen Sets.
Die hauptsächliche technische Lektion dieser Invention \#4 ist das parallele Set 12345, das Basis-Set, das benötigt wird, um die \hyperref[c1iii5]{Tonleitern} und Läufe zu spielen.}
Bach wußte jedoch, daß ein einziges paralleles Set von einem technischen Standpunkt aus zu gefährlich ist, weil man durch \hyperref[c1iii7b2]{Phasenkopplung} schummeln kann ohne Technik zu erwerben.
Um eine Phasenkopplung zu verhindern, fügte er dem parallelen Set eine oder zwei Noten hinzu.
Wenn man nun versucht zu schummeln, wird man sofort erwischt, weil die Musik nicht gleichmäßig herauskommt: Bach hat uns keine andere Wahl gelassen, als die erforderliche Technik zu erwerben, wenn man das musikalisch spielen will!
Hier ist ein weiteres Beispiel, in dem Bach uns lehrt, warum Musik und Technik untrennbar sind (indem er Musik als Kriterium für den Erwerb der Technik benutzt).
Deshalb ist der schnellste Weg, diese Invention spielen zu lernen, die parallelen Sets 12345 und 54321 zu üben und den \hyperref[c1iii5b]{Daumenübersatz} zu lernen.
\textbf{Sobald sie Ihre Finger mit Hilfe dieser parallelen Sets testen, werden Sie verstehen, warum Bach diese Invention komponiert hat.}
Wenn Sie diese Übung für parallele Sets zufriedenstellend ausführen können, wird dieses Stück ziemlich einfach sein, aber Sie werden finden, daß die parallelen Sets überhaupt nicht einfach sind und wahrscheinlich jede Menge Arbeit erfordern, auch wenn Sie zur Mittelstufe gehören.
Arbeiten Sie zunächst an diesen Sets, indem Sie nur die weißen Tasten benutzen; arbeiten Sie dann an den anderen, die schwarze Tasten beinhalten, wie von Bach vorgeschlagen.
Ein gutes Beispiel ist das parallele Set 12345 in der LH in den Takten 39-40 mit dem schwierigen vierten Finger auf einer weißen Taste, der auf 3 auf einer schwarzen Taste folgt.
Bach zieht den schwierigsten Teil dieses parallelen Sets, 2345, heraus und wiederholt ihn in Takt 49.

Bach hat klar den Wert davon gesehen, für die Entwicklung der Technik (Geschwindigkeit) eine kleine Anzahl Noten, wie Verzierungen und Triller, sehr schnell zu spielen.
Somit sind seine Verzierungen ein weiteres wichtiges Mittel für den Erwerb der Technik, und sie sind im Grunde eine kleine Ansammlung paralleler Sets.
Es gibt zahlreiche Diskussionen darüber, wie man Bachs Verzierungen spielen sollte (siehe Palmer, 3 Absätze zuvor); diese Diskussionen sind vom Standpunkt des korrekten musikalischen Ausdrucks wichtig, aber \textbf{wir dürfen nicht vergessen, daß die Verzierungen in Unterrichtsstücken technisch gesehen ein wesentliches Mittel zum Erwerb der Geschwindigkeit und nicht bloß musikalische Verzierungen sind.}
Spielen Sie sowohl die RH- als auch die LH-Triller mit den Fingern 1 und 3, was das Lernen des LH-Trillers vereinfacht.
Die meisten Schüler werden den RH-Triller zunächst besser spielen als den LH-Triller; \hyperref[c1ii20]{benutzen Sie in diesem Fall die RH, um die LH zu unterrichten}.
Dieser \enquote{Techniktransfer} von einer Hand zur anderen ist einfacher, wenn beide Hände einen ähnlichen Fingersatz benutzen.
Da der Zweck des Trillers einfach ein Aushalten der Noten ist, ist für den Triller keine bestimmte Geschwindigkeit erforderlich; versuchen Sie jedoch, die Triller mit beiden Händen mit der gleichen Geschwindigkeit auszuführen.
Wenn Sie sehr schnell trillern möchten, benutzen Sie die parallelen Sets, um die Triller wie in \hyperref[c1iii3]{\ref*{c1iii3}} beschrieben zu üben.
Es ist wichtig, daß Sie die ersten beiden Noten schnell beginnen, wenn Sie schnell trillern möchten.
Beachten Sie die Haltung der Finger 2, 4 und 5 während Sie trillern.
Diese sollten stillstehen, nahe an den Tasten und leicht gebogen sein.

Die meisten Schüler finden es schwierig, diese Inventionen jenseits einer bestimmten Geschwindigkeit zu spielen.
Lassen Sie uns deshalb einen Übungsablauf für das Steigern der Geschwindigkeit ansehen.
Wenn Sie diese Art von Ablauf benutzen, sollten Sie irgendwann in der Lage sein, praktisch mit jeder vernünftigen Geschwindigkeit zu spielen, einschließlich der hohen Geschwindigkeiten von Glenn Gould und anderen berühmten Pianisten.
Wir werden lernen, wie man die Takte 1 und 2 schnell spielt, und danach sollten Sie in der Lage sein, selbst herauszufinden, wie man den Rest beschleunigt.
Beachten Sie, daß diese beiden Takte \hyperref[c1iii2]{selbst-zirkulierend} sind (s. \ref*{c1iii2}).
Versuchen Sie, diese schnell zu zirkulieren.
Es besteht die Wahrscheinlichkeit, daß Sie es nicht schaffen, weil sich mit steigender Geschwindigkeit sehr schnell Streß aufbaut.
Üben Sie dann nur 212345 von Takt 1, bis es gleichmäßig und schnell ist.
Üben Sie dann 154, dann 54321 des zweiten Takts.
Verbinden Sie sie nun und zirkulieren Sie am Ende die beiden Takte.
Sie sind am ersten Tag vielleicht noch nicht in der Lage alles zu vollenden, aber die \hyperref[c1ii15]{PPI} wird es am zweiten Tag einfacher machen.
Benutzen Sie ähnliche Methoden, um Ihre technischen Probleme im ganzen Stück zu lösen.
Die Hauptschwierigkeit der LH ist das 521 in Takt 4, üben Sie deshalb das parallele Set 521, bis Sie es mit jeder Geschwindigkeit völlig \hyperref[c1ii14]{entspannt} spielen können.
Beachten Sie, daß das 212345 der RH und das 543212 der LH Übungen für das Vorbeigehen des Daumens sind.
Bach erkannte sicherlich, daß das \hyperref[c1iii5a]{Über- und Untersetzen des Daumens} bei hohen Geschwindigkeiten kritische Elemente der Technik sind, und er entwickelte zahlreiche geniale Gelegenheiten für uns, es zu üben.
Bevor man schnell HT spielen kann, muß man mit HS zu Geschwindigkeiten kommen, die viel schneller sind als die gewünschte HT-Geschwindigkeit.
\enquote{Auf Geschwindigkeit kommen} bedeutet nicht nur, in der Lage zu sein, die Geschwindigkeit zu erreichen, sondern man muß die \hyperref[ruhig]{ruhigen Hände} fühlen und die volle Kontrolle über jeden einzelnen Finger haben.
Anfänger müssen eventuell monatelang HS üben, um höhere Geschwindigkeiten zu erreichen.
Vielen Schülern gelingt es schneller zu spielen, wenn sie laut spielen.
Das ist ebenfalls nicht die wahre Geschwindigkeit; spielen Sie deshalb während der Übungseinheiten alles leise.
Wenn Sie damit beginnen, HT schnell zu spielen, übertreiben Sie den Rhythmus - das macht es wahrscheinlich einfacher.
Obwohl die meisten Kompositionen von Bach mit verschiedenen Geschwindigkeiten gespielt werden können, ist die minimale Geschwindigkeit für die Inventionen jene, bei der man die ruhigen Hände fühlen kann, sobald man die notwendige Technik erworben hat, denn wenn man nicht bis zu dieser Geschwindigkeit kommt, dann hat man eine von Bachs wichtigsten Lektionen versäumt.

Ein Schüler der Mittelstufe sollte die technischen Schwierigkeiten dieser Invention innerhalb einer Woche meistern.
\textbf{Nun sind wir bereit, es als Musikstück zu üben!}
Hören Sie sich verschiedene Aufnahmen an, um eine Vorstellung davon zu bekommen, was man tun kann und was Sie tun möchten.
Probieren Sie verschiedene Geschwindigkeiten aus, und entscheiden Sie sich für Ihre endgültige Geschwindigkeit.
\hyperref[c1iii13]{Nehmen Sie sich auf Video auf}, und prüfen Sie, ob das Ergebnis optisch und musikalisch zufriedenstellend ist; üblicherweise ist es das nicht, und Sie werden vieles finden, das Sie verbessern möchten. Sie werden vielleicht nie ganz zufrieden sein, sogar wenn Sie dieses Stück Ihr ganzes Leben lang geübt haben.

Um musikalisch zu spielen, müssen Sie jede Note mit den Fingern fühlen, bevor Sie sie spielen, auch wenn es nur den Bruchteil einer Sekunde früher ist.
Das wird Ihnen nicht nur mehr Kontrolle verleihen und Fehler eliminieren, sondern Ihnen auch gestatten, über den ganzen Tastenweg zu beschleunigen, so daß der Hammerstiel genau im richtigen Maß gebogen wird, wenn der Hammer die Saiten anschlägt.
Tun Sie so, als ob es keinen unteren Punkt für den Tastenweg gäbe, und lassen Sie Ihren Finger durch den unteren Punkt stoppen.
Sie können das tun und trotzdem leise spielen.
Das wird \enquote{in die Tasten hineingehen} genannt.
Man kann nicht \enquote{die Finger gut anheben und anschlagen} wie \hyperref[c1iii7h]{Hanon} empfahl und erwarten Musik zu machen.
Solch eine Bewegung kann den Hammerstiel schwingen statt sich biegen lassen und einen nicht vorhersagbaren und schrillen Klang erzeugen.
Üben Sie deshalb, wenn Sie HS üben, auch die Musikalität.
Benutzen Sie die \enquote{\hyperref[c1iii4b]{flachen Fingerhaltungen}} aus \ref*{c1iii4b}.
Kombinieren Sie diese mit einem geschmeidigen Handgelenk.
Spielen Sie soviel wie möglich mit dem fleischigen Teil der Finger (gegenüber dem Fingernagel), nicht mit der knochigen Fingerspitze.
Wenn Sie Ihr Spielen auf Video aufnehmen, dann wird die gebogene Fingerhaltung kindlich und amateurhaft aussehen.
Sie können nicht entspannt spielen, bis Sie die Streckmuskeln der ersten 2 oder 3 Glieder der Finger 2 bis 5 völlig entspannen können.
Diese Entspannung ist das Wesentliche der flachen Fingerhaltungen.
Zunächst werden Sie nur bei niedrigeren Geschwindigkeiten in der Lage sein, diese Überlegungen zu berücksichtigen.
Sobald Sie jedoch die ruhigen Hände entwickeln, werden Sie die Fähigkeit erlangen, sie bei höheren Geschwindigkeiten einzuschließen.
Tatsächlich werden Sie, weil diese Fingerhaltungen Ihnen die völlige Entspannung und Kontrolle gestatten, in der Lage sein, mit viel höherer Geschwindigkeit zu spielen.
Das ist einer der (vielen) Gründe, warum ruhige Hände so wichtig sind.


<h3><br>Klang und Farbe</h3>

Der verbesserte Klang wird am deutlichsten, wenn man leise spielt; das leisere Spielen ist auch für die \hyperref[c1ii14]{Entspannung} und Kontrolle hilfreich.
Es ist die \hyperref[c1iii4b]{flache Fingerhaltung}, die ein leiseres Spielen mit Kontrolle ermöglicht.
Wie leise ist leise?
Das hängt von der Musik, der Geschwindigkeit usw. ab, aber für Übungszwecke ist immer leiser zu spielen, bis Sie anfangen, einige Noten auszulassen, ein nützliches Kriterium; diese Lautstärke (oder ein wenig lauter) ist meistens die beste um leise zu üben.
Wenn Sie die Kontrolle über den Klang erlangt haben (Klang jeder einzelnen Note), versuchen Sie, Ihrer Musik Farbe hinzuzufügen (Effekt von Notengruppen).
Die Farbe ist bei jedem Komponisten anders.
Chopin erfordert Legato, besonderes Staccato, Rubato usw.
Mozart erfordert die äußerste Aufmerksamkeit auf die Ausdrucksbezeichnungen.
Beethoven erfordert ununterbrochene \hyperref[c1iii1b]{Rhythmen}, die über viele, viele Takte gehen; deshalb müssen Sie die Fertigkeit entwickeln, aufeinanderfolgende Takte zu \enquote{verbinden}.
Bachs Inventionen sind etwas künstlich und \enquote{eingeengt}, weil sie hauptsächlich auf einfache \hyperref[c1iii7b]{parallele Sets} beschränkt sind.
Man kann dieses Handicap leicht überwinden, indem man die Vielzahl der musikalischen Konzepte betont, die seiner Musik fast unendliche Tiefe verleihen.
Die offensichtliche Musikalität kommt von der Harmonie bzw. Konversation zwischen beiden Händen.
Der Schluß jedes Stücks muß etwas besonderes sein, und Bachs Schlüsse sind immer überzeugend.
Lassen Sie deshalb den Schluß nicht einfach so daherkommen; stellen Sie sicher, daß der Schluß einen Zweck hat.
Seien Sie bei dieser Invention in Takt 50, bei dem die beiden Hände in Gegenbewegung sind, besonders aufmerksam, wenn Sie zu dem respekteinflößenden Schluß kommen.
Wenn Sie die Musik auf Geschwindigkeit bringen und \hyperref[ruhig]{ruhige Hände} entwickeln, sollten die 6-notigen Läufe (z.B. 212345 usw.) wie aufsteigende und fallende Wellen klingen.
Der RH-Triller ist einer Glocke ähnlich, weil die Noten einen Ganzton umfassen, während der LH-Triller düsterer ist, weil er ein Halbtonschritt ist.
Wenn Sie HS üben, beachten Sie, daß der RH-Triller nicht nur ein Triller ist, sondern an Lautstärke zunimmt.\footnote{Das gilt zumindest für Ausgaben mit Dynamikzeichen. Im \enquote{Urtext} findet man davon nichts, was zeigt, wie sich die Bach-Ausgaben über die Jahrhunderte hinweg geändert haben und was man alles mit den Inventionen anstellen kann.}
Ähnlich ist der LH-Triller eine Einführung zu dem darauffolgenden Kontrapunkt zur RH.
\textbf{Man kann die Farbe nicht herausbringen, wenn man nicht jeden Finger präzise im richtigen Moment anhebt.}
Die meisten Unterrichtsstücke Bachs enthalten Übungen zum richtigen Anheben der Finger.
Selbstverständlich sollte die Färbung zunächst HS untersucht werden.
Ruhige Hände werden am leichtesten HS erworben; deshalb ist eine angemessene Vorbereitung mit HS vor dem HT-Üben für den Klang und die Farbe von entscheidender Wichtigkeit.
Wenn die Vorbereitungen abgeschlossen sind, können Sie mit HT beginnen und den unglaublichen Reichtum von Bachs Musik hervorbringen!

Klang und Farbe haben in dem Sinn keine Grenze, daß es, sobald es einmal gelungen ist, leichter wird, mehr davon hinzuzufügen, und die Musik sogar leichter spielbar wird.
Sie entdecken vielleicht plötzlich, daß Sie die ganze Komposition ohne einen einzigen hörbaren Fehler spielen können.
Das ist wahrscheinlich das deutlichste Beispiel für die Aussage, daß man die Musik nicht von der Technik trennen kann.
Das Erzeugen guter Musik macht Sie tatsächlich zu einem besseren Klavierspieler.
Das bietet eine der Erklärungen dafür, warum man gute und schlechte Tage haben kann - wenn Ihre geistige Verfassung und die Konditionierung Ihrer Finger genau so sind, daß Sie den Klang und die Farbe kontrollieren können, werden Sie einen guten Tag haben.
Das lehrt uns, daß man an schlechten Tagen in der Lage sein kann \enquote{darüber hinwegzukommen}, indem man versucht, sich an die Grundlagen zu erinnern, wie man den \hyperref[c1iii1]{Klang und die Farbe kontrolliert}.
Damit beenden wir die Besprechung der Invention \#4.
Wir wenden uns nun wieder dem Übungsablauf zu.

Sie haben nun über eine Stunde geübt, und die Finger fliegen.
Das ist der Zeitpunkt, wenn Sie wirklich Musik machen!
Sie müssen jede Anstrengung unternehmen, um mindestens während der Hälfte der gesamten Übungszeit das Musikmachen zu üben.
Wenn Sie ein genügend großes Repertoire aufgebaut haben, dann sollten Sie versuchen, diese \enquote{Musikzeit} von 50\% auf 90\% zu steigern.
Deshalb müssen Sie bewußt diesen Anteil Ihres Übungsablaufs für die Musik reservieren.
Spielen Sie sich die Seele aus dem Leib, mit allen Emotionen und dem Ausdruck, den Sie zustande bringen.
\textbf{Den musikalischen Ausdruck zu finden ist sehr erschöpfend; deshalb wird es zunächst viel mehr Konditionierung und Aufwand erfordern als alles, was man mit \hyperref[c1iii7h]{Hanon} machen kann.}
Wenn Sie keinen Lehrer haben, dann ist der einzige Weg, Musikalität zu lernen, sich Aufnahmen anzuhören und Konzerte zu besuchen.
Wenn Sie in nächster Zukunft einen Auftritt mit einer bestimmten Komposition planen, spielen Sie sie einmal langsam oder zumindest mit einer bequemen und völlig kontrollierbaren Geschwindigkeit, bevor Sie an etwas anderes herangehen.
Wenn man langsam spielt, ist der Ausdruck nicht wichtig.
Es mag sogar nützlich sein, absichtlich mit wenig Ausdruck zu spielen, wenn man langsam spielt, bevor man zu etwas anderem übergeht.

Bach zu lernen wird in diesem Buch sehr betont.
Warum?
\textbf{Weil die Musik, die Bach für die technische Entwicklung schrieb, in der Klavierpädagogik in ihrem gesunden, vollständigen, effizienten und korrekten Herangehen an den Erwerb der Technik einmalig ist - es gibt nichts vergleichbares.}
Jeder erfahrene Lehrer wird einige Stücke von Bach zu Studienzwecken zuweisen.
Wie oben erwähnt, ist der einzige Grund, warum Schüler nicht mehr Stücke von Bach lernen, daß sie ohne die richtigen Übungsmethoden so schwierig erscheinen.
Sie können sich selbst den Nutzen von Bachs Lektionen beweisen, indem Sie fünf seiner technischen Kompositionen lernen und sie ein halbes Jahr oder länger üben.
Kehren Sie dann zurück und spielen Sie die schwierigsten Stücke, die Sie zuvor gelernt haben, und Sie werden über die größere Leichtigkeit und die Kontrolle, die Sie erworben haben, überrascht sein.
Bachs Kompositionen wurden entworfen, um Konzertpianisten mit einer gesunden Grundlage der Technik hervorzubringen.
Chopins Etüden wurden nicht für eine schrittweise, vollständige technische Entwicklung entworfen, und viele von Beethovens Kompositionen können \hyperref[c1iii10hand]{Handverletzungen} und \hyperref[c1iii10gehoer]{Gehörschäden} verursachen, wenn man nicht die richtige Anleitung bekommt (anscheinend haben sie Beethovens Gehör geschädigt).
Keine davon lehrt, wie man übt.
Deshalb ragen Bachs Kompositionen gegenüber allen anderen hinsichtlich der technischen Entwicklung heraus.
Mit den Übungsmethoden dieses Buchs können wir nun den vollen Vorteil aus Bachs Quellen für die technische Entwicklung ziehen, die in der Vergangenheit sträflich vernachlässigt wurden.

Natürlich sind \hyperref[c1iii7h]{Hanon} usw. (es gibt viele andere, wie z.B. Cramer-Bülow) es nicht Wert, hier besprochen zu werden, weil sie den wichtigsten Punkt außer Acht gelassen haben: Ohne Musik ist Technik nicht möglich.
Aber \hyperref[c1iii7d]{Tonleitern und Arpeggios} sind notwendig, weil sie die Grundlage von praktisch allem bilden, das wir spielen.
Die Erfordernis der Musikalität bedeutet, daß man sie auf eine Weise üben muß, daß andere, die einem beim Spielen von Tonleitern zuhören, sagen: \enquote{Enorm!}
Warum soll man Hanon nicht genauso üben können?
Man kann, aber es ist nicht notwendig; es gibt viel besseres Material, mit dem man die Kunst des Musikers üben kann.

\textbf{Alles in allem gibt es keinen Standard-Übungsablauf.
Das Konzept eines festen Übungsablaufs kam auf, weil diejenigen, die die intuitiven Methoden praktizierten und nicht wußten, wie man Übungsmethoden lehrt, ihn benutzten, weil sie nicht wußten, was sie sonst lehren sollten.}
Für diejenigen, die die Übungsmethoden kennen, wird das Konzept eines Standard-Übungsablaufs irgendwie zu einer dummen Idee.
Ein typischer Standardablauf mag z.B. mit \hyperref[c1iii7h]{Hanon-Übungen} anfangen; man kann die Hanon-Übungen jedoch mit den Methoden dieses Buchs leicht auf lächerliche Geschwindigkeiten bringen.
Und wenn man das vollbracht hat, fragt man sich, warum man das ständig wiederholen soll.
Was wird man erreichen, wenn man diese lächerlich schnellen Hanon-Stücke jeden Tag spielt?
Statt eines Standard-Übungsablaufs müssen Sie jeweils das Ziel Ihrer Übungseinheit definieren und die Übungsmethoden auswählen, die Sie benötigen, um dieses Ziel zu erreichen.
Ihr Übungsablauf wird sich im Laufe der Übungseinheiten kontinuierlich weiterentwickeln.
Deshalb ist der Schlüssel zum Entwickeln eines guten Übungsablaufs eine genaue Kenntnis aller Übungsmethoden.
Wie sich das von dem intuitiven Ablauf, der in \hyperref[c1ii1]{\ref*{c1ii1}} beschrieben wurde, unterscheidet!



