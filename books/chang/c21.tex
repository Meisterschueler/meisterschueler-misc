% File: c21

\chapter{Stimmen des Klaviers}
\label{c2}

\subsection{Einleitung}
\label{c2_1} 

\textbf{\textit{[Achtung: Beim Stimmen, Intonieren und anderen Wartungsarbeiten kann ein Klavier durch Mangel an Vorsicht, unsachgemäßes Vorgehen, usw. beschädigt werden!
Wenn jemand zum ersten Mal die Mechanik ausbaut und daran arbeitet, stehen die Chancen ungefähr 1:1, \underline{daß}  er etwas beschädigt.
Sie sollten sich deshalb eingehend mit dem Thema beschäftigen, bevor (!) Sie Hand an Ihr eigenes oder ein fremdes Klavier legen.
Lassen Sie sich ggf. von jemandem beraten, der sich mit der Materie auskennt, und beschaffen Sie sich gute weiterführende Literatur zu dem Thema, wie z.B. das von Chuan C. Chang angeführte Buch \enquote{Piano Servicing, Tuning, and Rebuilding} von Arthur Reblitz.]}}

\textbf{Dieses Kapitel ist für diejenigen bestimmt, die ihr Klavier noch nie selbst gestimmt haben und sehen möchten, ob sie der Aufgabe gewachsen sind.}
Das Buch \textit{Piano Servicing, Tuning, and Rebuilding} von Arthur Reblitz ist dabei ein sehr hilfreiches Nachschlagewerk.\footnote{Hinweise für deutsche Bücher gleicher Qualität nehme ich gerne hier auf.}
Der schwierigste Teil beim Lernen des Stimmens ist das Anfangen.
Wer in der glücklichen Lage ist, jemanden zu haben, der ihn unterrichtet, ist natürlich am besten dran.
Unglücklicherweise sind Lehrer für das Klavierstimmen nicht ohne weiteres verfügbar.
Probieren Sie die Vorschläge in diesem Kapitel aus, und sehen Sie, wie weit sie kommen.
Nachdem Sie sich darüber im klaren sind, was Ihnen Probleme bereitet, können Sie mit Ihrem Stimmer über 30-minütige Lektionen mit einer vereinbarten Vergütung verhandeln oder ihn bitten zu erklären, was er tut, wenn er stimmt.
Seien Sie darauf bedacht, Ihrem Stimmer nicht zuviel aufzubürden; Stimmen und Unterrichten kann mehr als viermal so lang dauern wie nur zu stimmen.
Seien Sie auch vorgewarnt, daß Klavierstimmer keine ausgebildeten Lehrer sind, und einige von ihnen mögen unberechtigte Befürchtungen hegen, daß sie einen Kunden verlieren könnten.
Diese Befürchtungen sind unbegründet, weil die tatsächliche Zahl der Menschen, die professionelle Stimmer erfolgreich ersetzt haben, vernachlässigbar klein ist.
Am wahrscheinlichsten bekommen Sie am Ende ein besseres Verständnis dafür, was es bedeutet, ein Klavier zu stimmen.

Für Klavierspieler bietet das Vertrautwerden mit der Kunst des Stimmens eine Ausbildung, die für ihre Fähigkeit Musik zu erzeugen und ihre Instrumente zu warten sehr wichtig ist.
Es versetzt sie auch in die Lage, vernünftig mit ihren Stimmern zu kommunizieren.
So kannte z.B. die Mehrzahl der Klavierlehrer, denen ich die Frage stellte, noch nicht einmal den Unterschied zwischen \hyperref[et1]{gleichschwebender Temperatur}\index{gleichschwebender Temperatur} und historischen Stimmungen.
Der Hauptgrund, warum die meisten Menschen versuchen das Stimmen zu lernen, ist aus Neugier -- für die meisten ist das Klavierstimmen ein rätselhaftes Geheimnis.
Sind sie erst einmal über die Vorteile eines gestimmten (gewarteten) Klaviers unterrichtet, ist es wahrscheinlicher, daß sie ihren Stimmer regelmäßig rufen.
Klavierstimmer können bestimmte Töne, die vom Klavier kommen, hören, die die meisten Menschen, sogar Pianisten, nicht wahrnehmen.
Diejenigen, die das Stimmen üben, werden für die Töne von verstimmten Klavieren sensibilisiert.
Unter der Annahme, daß Sie die Zeit haben, mindestens einmal alle ein oder zwei Monate für mehrere Stunden zu üben, wird es wahrscheinlich ungefähr ein Jahr dauern, bis Sie anfangen mit dem Stimmen zurechtzukommen.

Lassen Sie mich hier ein wenig abschweifen, um zu besprechen, wie wichtig es unter dem Gesichtspunkt, von dem Stimmer einen Gegenwert für Ihr Geld zu bekommen, so daß Ihr Klavier richtig gewartet werden kann, ist, die Lage des Klavierstimmers zu verstehen und richtig mit ihm zu kommunizieren.
Diese Überlegungen haben sowohl eine direkte Auswirkung auf Ihre Fähigkeit, sich Klaviertechnik anzueignen als auch auf Ihre Entscheidungen darüber, was oder wie Sie bei einem Auftritt vorspielen, wenn Sie ein bestimmtes Klavier zur Verfügung haben.
Eine der verbreitetsten Schwierigkeiten, die ich z.B. bei Schülern festgestellt habe, ist ihre Unfähigkeit, pianissimo zu spielen.
Aus meinem Verständnis des Klavierstimmens heraus gibt es dafür eine einfache Erklärung -- die meisten Klaviere dieser Schüler werden zu wenig gewartet.
Die Hämmer sind zu abgenutzt bzw. verdichtet und die Mechanik so sehr verstellt, daß pianissimo zu spielen unmöglich ist.
Diese Schüler werden nicht einmal in der Lage sein, pianissimo zu üben!
Das gilt auch für den musikalischen Ausdruck und die Tonkontrolle.
Diese zu wenig gewarteten Klaviere sind wahrscheinlich eine der Ursachen für die Ansicht, daß Klavierüben eine Qual für die Ohren ist, aber das sollte es nicht sein.

Ein weiterer Faktor ist, daß man sich im allgemeinen das Klavier nicht aussuchen kann, wenn man gebeten wird, etwas vorzuspielen.
Man kann auf alles treffen: von einem wundervollen Konzertflügel über ein Kleinklavier bis zu (Schock!) einem billigen Stutzflügel, der völlig vernachlässigt wurde, seit er vor 40 Jahren gekauft wurde.
Ihr Verständnis dafür, was man mit jedem dieser Klaviere tun kann bzw. nicht tun kann, sollte der erste Punkt bei der Entscheidung sein, was und wie man spielt.

Wenn Sie erst einmal angefangen haben, das Stimmen zu üben, werden Sie schnell verstehen, warum es einem genauen und qualitativen Stimmen nicht förderlich ist, wenn jemand dabei Staub saugt, Kinder herumspringen, der Fernseher oder die Stereoanlage plärrt oder in der Küche die Töpfe klappern, und warum ein Stimmen auf die Schnelle für 70 Euro kein Schnäppchen ist im Vergleich zu einem Stimmen für 150 Euro, bei dem der Stimmer die Hämmer neu formt und nadelt.
Wenn man aber die Besitzer fragt, was der Stimmer an ihrem Klavier getan hat, haben sie im allgemeinen keine Vorstellung davon.
Eine Beschwerde, die ich oft von Besitzern höre, ist, daß das Klavier nach dem Stimmen tot oder schrecklich klingen würde.
Das geschieht oft, wenn der Besitzer keinen festen Bezugspunkt hat, von dem aus er den Klavierklang beurteilen kann -- das Urteil basiert darauf, ob der Besitzer den Klang mag oder nicht.
Solche Wahrnehmungen sind allzuoft durch die Vorgeschichte des Besitzers falsch beeinflußt.
Der Besitzer kann sich tatsächlich an den Klang eines verstimmten Klaviers mit verdichteten Hämmern gewöhnen, so daß wenn der Stimmer den Klang wiederherstellt, der Besitzer diesen nicht mag, weil er sich nun zu sehr von dem gewohnten Klang oder Gefühl unterscheidet.
Der Stimmer könnte sicherlich Schuld daran haben; der Besitzer braucht jedoch ein minimales Wissen über technische Details des Stimmens, um solch ein Urteil hieb- und stichfest zu machen.
Der Nutzen eines Verständnisses für das Stimmen und der richtigen Wartung des Klaviers wird offensichtlich von der Allgemeinheit unterschätzt.
Vielleicht ist das wichtigste Ziel dieses Kapitels, dieses Bewußtsein zu vergrößern.

\textbf{Klavierstimmen erfordert -- im Gegensatz zum \hyperref[c1iii12]{absoluten Gehör}\index{absoluten Gehör} -- keine guten Ohren, weil das ganze Stimmen durch den Vergleich mit einer Referenz ausgeführt wird, bei dem Schwebungen benutzt werden und man mit der Bezugsfrequenz einer Stimmgabel beginnt.}
Tatsächlich kann die Fähigkeit des absoluten Gehörs bei einigen Menschen mit dem Stimmen in Konflikt geraten.
Deshalb ist die \enquote{einzige} notwendige Hörfertigkeit die Fähigkeit, die verschiedenen Schwebungen zu hören und zwischen ihnen zu unterscheiden, wenn zwei Saiten angeschlagen werden.
Diese Fähigkeit entwickelt sich durch Übung und ist nicht mit dem Wissen über Musiktheorie oder mit Musikalität verknüpft.
Größere Flügel sind leichter zu Stimmen als \enquote{\hyperref[upright]{Aufrechte}\index{Aufrechte}}; die meisten Stutzflügel sind jedoch schwieriger zu stimmen als gute \enquote{Aufrechte}.
Obwohl man seine Übungen logischerweise an einem qualitativ schlechteren Klavier beginnen sollte, wird dieses deshalb schwieriger zu stimmen sein.
 


