% File: c1iii6l

\subsection{Blattspieler und Auswendiglernende (Bachs Inventionen)}
\label{c1iii6l}

\textbf{Viele gute Blattspieler sind schlechte Auswendiglernende, und viele gute Auswendiglernende sind schlechte Blattspieler.}
Dieses Problem tritt auf, weil gute Blattspieler es am Anfang kaum als notwendig erachten, auswendig zu lernen und Ihnen das Blattspiel Spaß macht, so daß Sie am Ende das Blattspiel auf Kosten des Auswendiglernens üben.
Je mehr sie vom Blatt spielen, desto weniger Gedächtnis brauchen sie, und je weniger sie auswendig lernen, desto schlechtere Auswendiglernende werden sie, mit dem Ergebnis, daß sie eines Tages wach werden und zu dem Schluß kommen, sie seien unfähig, auswendig zu lernen.
Selbstverständlich gibt es von Natur aus begabte Blattspieler, die echte Gedächtnisprobleme haben, aber diese bilden eine vernachlässigbar kleine Minderheit.
Deshalb tritt die Schwierigkeit auswendig zu lernen prinzipiell aufgrund einer psychologischen Denkblockade auf, die sich über einen langen Zeitraum aufgebaut hat.
Gute Auswendiglernende können das umgekehrte Problem erfahren: Sie können nicht vom Blatt spielen, weil sie automatisch alles auswendig lernen und selten die Chance haben, das Blattspiel zu üben.
Das ist jedoch kein symmetrisches Problem, weil praktisch alle fortgeschrittenen Klavierspieler wissen, wie man auswendig lernt.
\textbf{Schlechte Auswendiglernende hatten zusätzlich das Pech, daß sie nie eine fortgeschrittene Technik erworben haben, d.h. die technische Stufe von schlechten Auswendiglernenden ist im allgemeinen niedriger als jene von guten Auswendiglernenden.}

\enquote{Blattspiel} wird in diesem Abschnitt in einem weiteren Sinne gebraucht, d.h. es meint sowohl das wahre Vom-Blatt-Spielen (\hyperref[c030530]{Prima-Vista-Spiel}) als auch das Üben der Musik mit der Hilfe der Noten.
Die Unterscheidung zwischen dem Vom-Blatt-Spielen eines Stücks, das man vorher noch nie gesehen hat, und eines Stücks, das zuvor schon gespielt wurde, ist hier nicht wichtig.
Im Interesse der Kürze wird diese Unterscheidung dem Zusammenhang des Satzes überlassen.

\textbf{Es ist wichtiger, auswendig lernen zu können, als vom Blatt spielen zu können, weil man als Pianist ohne die Fähigkeit, gut vom Blatt zu spielen, existieren kann, aber man kann ohne die Fähigkeit, auswendig zu lernen, kein guter Pianist werden.}
Auswendiglernen ist für den durchschnittlichen Klavierspieler, dessen Gedächtnis nicht geschult wurde, nicht einfach.
Blattspieler, die nicht auswendig lernen können, sehen sich einem noch schwierigeren Problem gegenüber.
\textbf{Deshalb müssen schlechte Auswendiglernende, die ein auswendig gelerntes Repertoire erwerben möchten, es mit der Einstellung anfangen, daß dies ein Langzeitprojekt mit zahlreichen zu überwindenden Hindernissen sein wird.}
Wie oben gezeigt, ist die Lösung im Prinzip einfach: Machen Sie es sich zur Gewohnheit, alles auswendig zu lernen, \textit{bevor} Sie das Stück lernen.
In der Praxis ist die Versuchung, durch das Vom-Blatt-Spielen schnell zu lernen, oftmals zu unwiderstehlich.
Sie müssen die Art, wie sie neue Stücke üben, grundlegend ändern.

Das schwierigste Problem, auf das Blattspieler stoßen, ist das psychologische Problem der Motivation.
\textbf{Für diese guten Blattspieler erscheint Auswendiglernen als eine Zeitverschwendung, weil sie schnell lernen können, viele Stücke hinreichend gut vom Blatt zu spielen.}
Sie könnten sogar in der Lage sein, schwierige Stücke unter Benutzung des \hyperref[c1iii6d]{Hand-Gedächtnisses} zu spielen, und wenn sie hängenbleiben, können sie immer in den Noten vor sich nachsehen.
Deshalb kommen sie ohne Auswendiglernen zurecht.
Wenn man jahrelang auf diese Art Klavier geübt hat, wird es sehr schwierig, zu lernen wie man auswendig lernt, weil das Gehirn von den Noten abhängig geworden ist.
Schwierige Stücke sind bei diesem System unmöglich und werden deshalb zugunsten einer großen Zahl leichterer Stücke gemieden.
Lassen Sie uns im Bewußtsein dieser potentiellen Schwierigkeiten versuchen, ein typisches Programm zum Lernen wie man auswendig lernt durchzuarbeiten.

\textbf{Die beste Art anzufangen ist, ein paar kurze, neue Stücke auswendig zu lernen.}
Haben Sie erst einmal ein paar Stücke ohne allzuviel Aufwand erfolgreich auswendig gelernt, können Sie anfangen, Ihr Selbstvertrauen aufzubauen und Ihre Fertigkeiten zum Auswendiglernen zu verbessern.
Wenn diese Fertigkeiten ausreichend entwickelt sind, können Sie auch daran denken, alte Stücke auswendig zu lernen, die Sie durch Vom-Blatt-Spielen gelernt haben.

Meine Klaviersitzungen sind entweder Sitzungen zum Auswendiglernen oder technische Übungssitzungen, denn wenn ich zwischen den Sitzungen zum Auswendiglernen anderes Material spiele, vergesse ich, was auch immer ich zuletzt auswendig gelernt habe.
Während der technischen Übungssitzungen brauche ich fast nie die Noten.
Sogar während der Sitzungen zum Auswendiglernen benutze ich die Noten nur zu Anfang und lege sie weg, sobald ich kann.

Lassen Sie uns als ein Beispiel zum Auswendiglernen kurzer Stücke drei von Bachs zweistimmigen Inventionen lernen: \#1, \#8 und \#13.
Ich werde mit Ihnen \#8 durchgehen.
Nachdem Sie \#8 gelernt haben, versuchen Sie \#1 selbst, und fangen Sie dann mit \#13 an.
Die Idee ist, alle drei gleichzeitig zu lernen, aber wenn sich das als zu schwierig erweist, versuchen Sie nur zwei (\#8 und \#1) oder sogar nur \#8.
Es ist wichtig, daß Sie nur das versuchen, von dem Sie glauben, daß Sie gut damit zurechtkommen, weil hier gezeigt werden soll, wie einfach es ist.
Der unten angegebene Plan ist dafür gedacht, alle drei auf einmal zu lernen.
Wir nehmen an, daß Sie das Material der Abschnitte I bis III gelernt haben, und daß Sie technisch soweit sind, die Bach-Inventionen in Angriff zu nehmen.
Das Pedal wird bei keiner der Bach-Inventionen benutzt.



