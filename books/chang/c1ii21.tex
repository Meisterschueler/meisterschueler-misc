% File: c1ii21

\subsection{Ausdauer aufbauen, Atmung}
\label{c1ii21}

\enquote{Ausdauer} ist ein umstrittener Begriff beim Klavierüben.
Diese Auseinandersetzung ist in der Tatsache begründet, dass \textbf{Klavierspielen Kontrolle und nicht Muskelkraft erfordert}, und viele Schüler haben den falschen Eindruck, dass sie keine Technik erwerben werden, bevor sie genug Muskeln entwickelt haben.
Auf der anderen Seite ist ein gewisses Maß an Ausdauer notwendig.
Dieser offensichtliche Widerspruch kann beseitigt werden, wenn man genau versteht, was benötigt wird und wie man es bekommt.
Offensichtlich kann man laute, grandiose Passagen nicht ohne Energieaufwand spielen.
Große, starke Pianisten, die ansonsten dieselben Fertigkeiten haben, können sicherlich mehr Klang erzeugen als kleine, schwache Pianisten.
Und die stärkeren Pianisten können leichter \enquote{anstrengende} Stücke spielen.
Jeder Pianist hat genug körperliche Ausdauer, um Stücke seiner Stufe zu spielen, einfach wegen der Menge an Übung, die erforderlich war, um auf diese Stufe zu kommen.
Doch wissen wir, dass Ausdauer ein Problem \textit{ist}.
Die Antwort liegt in der \hyperref[c1ii14]{Entspannung}.
Wenn die Ausdauer ein Thema wird, wird es fast immer durch übermäßige Spannung verursacht.

Ein Beispiel dafür ist das Oktavtremolo der linken Hand im ersten Satz von Beethovens Pathétique.
Das \textit{einzige}, was über 90\% der Schüler tun müssen, ist, den Stress zu eliminieren; doch viele Schüler üben es für Monate mit geringem Fortschritt.
Ihr erster Fehler ist, dass sie es zu laut üben.
Das fügt gerade dann zusätzlichen Stress und Ermüdung hinzu, wenn man es sich am wenigsten leisten kann.
Üben Sie es leise, und konzentrieren Sie sich nur darauf, den Stress zu eliminieren, wie in \hyperref[c1iii3b]{\autoref{c1iii3b}} beschrieben.
In einer Woche oder zwei werden Sie so viele Tremolos so schnell spielen wie Sie möchten.
Fangen Sie nun an, Lautstärke und Ausdruck hinzuzufügen. Fertig!
An diesem Punkt unterscheidet sich Ihre körperliche Stärke und Ausdauer nicht von der, die Sie hatten, als Sie vor wenigen Wochen angefangen hatten - Sie haben sich hauptsächlich damit beschäftigt, die beste Art zu finden, den Stress zu eliminieren.

Anspruchsvolle Stücke zu spielen, erfordert ungefähr so viel Energie wie ein langsames Joggen mit ungefähr vier Meilen pro Stunde, wobei das Gehirn fast die Hälfte der gesamten Energie benötigt.
Viele Jugendliche können nicht mehr als eine Meile ununterbrochen joggen.
Deshalb würde es die Ausdauer überbeanspruchen, wenn man einen jungen Menschen bitten würde, schwierige Passagen 20 Minuten lang ununterbrochen zu üben, weil es ungefähr dem Joggen von einer Meile entspräche.
Lehrer und Eltern müssen aufpassen, wenn Jugendliche ihre Klavierstunden beginnen, dass die Übungszeiten am Anfang auf weniger als 15 Minuten begrenzt sind, bis der Schüler genügend Ausdauer erlangt hat.
Marathonläufer haben Ausdauer, aber sie sind nicht muskulös.
Man muss den Körper für die für das Klavierspielen notwendige Ausdauer konditionieren, aber man braucht keine zusätzlichen Muskeln.

Nun \textit{gibt es} einen Unterschied zwischen dem Klavierspielen und dem Marathonlaufen wegen der Notwendigkeit, zusätzlich zur Muskelkonditionierung das Gehirn für die Ausdauer zu konditionieren.
Deshalb kann man mit stupidem Üben keine Ausdauer erreichen.
Die effizienteste Art, Ausdauer zu erlangen, ist, entweder fertig gelernte Stücke zu spielen und Musik zu machen oder schwierige Abschnitte kontinuierlich \hyperref[c1ii7]{mit getrennten Händen zu üben}.
Benutzen wir wieder den Vergleich mit dem Joggen.
Es wäre für die meisten Schüler sehr schwer, schwieriges Material ununterbrochen länger als einige Stunden zu üben, weil zwei Stunden zu üben, sechs Meilen zu joggen entsprechen würde, was ein \enquote{Wahnsinnstraining} ist.
Deshalb werden Sie zwischen den schweren Übungsteilen ein paar leichte Stücke spielen müssen.
Konzentrierte Übungssitzungen von mehr als ein paar Stunden sind nicht so hilfreich, bevor Sie nicht auf einer fortgeschrittenen Stufe sind.
Es ist wahrscheinlich besser, zu unterbrechen und nach einer Pause erneut mit dem Üben zu beginnen.
\textbf{Klar, hartes Üben ist anstrengende Arbeit, und ernsthaftes Üben kann den Schüler in eine gute körperliche Verfassung bringen.}
Mit getrennten Händen zu üben ist in dieser Hinsicht am wertvollsten, weil es einer Hand gestattet sich zu erholen, während die andere hart arbeitet, was dem Klavierspieler erlaubt, 100\% der Zeit ohne Verletzung oder Ermüdung so hart zu arbeiten wie er möchte.
Natürlich ist es von der Ausdauer her gesehen nicht schwierig (wenn man die Zeit hat), sechs oder acht Stunden an täglicher Übungszeit aufzuwenden, indem man jede Menge stupider Fingerübungen einschließt.
Das ist ein Prozess der Selbsttäuschung, in welchem der Schüler glaubt, dass der bloße Zeitaufwand ihn ans Ziel bringt - wird er aber nicht.
Wenn überhaupt, ist es wichtiger, das Gehirn zu konditionieren als die Muskeln, denn bei den meisten Schülern ist es das Gehirn, das mehr konditioniert werden muss.
Die Konditionierung des Gehirns ist für das \hyperref[c1iii14]{Vorspielen} besonders wichtig.
Eine anstrengende Konditionierung der Muskeln wird dazu führen, dass der Körper schnelle Muskeln in langsame umwandelt (diese sind ausdauernder) - genau das Gegenteil von dem, was man möchte.
Entgegen der verbreiteten Meinung benötigen Klavierspieler deshalb nicht mehr Muskeln; sie benötigen eine größere Nervenkontrolle und die Umwandlung langsamer Muskeln in schnelle - siehe \hyperref[c1iii7aMuskeln]{\autoref{c1iii7aMuskeln}}.

Was ist Ausdauer? Sie ist etwas, das uns befähigt, weiter zu spielen ohne müde zu werden.
Bei langen Übungssitzungen über mehrere Stunden bekommen Klavierspieler genauso wie Athleten (insbesondere Marathonläufer) ihren neuen Energieschub.
Wenn Sie sich generell müde fühlen, warten Sie deshalb darauf, dass Sie den toten Punkt überwinden - dieses bewusste Wissen um den neuen Energieschub kann bedeuten, dass er zuverlässiger einsetzt, insbesondere, wenn Sie es schon mehrmals erlebt haben und wissen, wie es sich anfühlt.
Gewöhnen Sie sich deshalb nicht an, jedes Mal auszuruhen, wenn Sie müde werden, wenn die Möglichkeit besteht, dass Sie den neuen Energieschub bekommen.

Können wir irgendwelche biologischen Faktoren bestimmen, die die Ausdauer kontrollieren?
Die biologische Basis zu kennen ist der beste Weg, Ausdauer zu verstehen.
Da keine spezifischen biophysikalischen Studien für Klavierspieler vorliegen, können wir nur spekulieren.
\textbf{Klar ist, dass wir eine genügende Sauerstoffaufnahme und einen adäquaten Blutfluss zu den Muskeln, bestimmten Organen und dem Gehirn brauchen.}
Der größte Faktor, der die Sauerstoffaufnahme beeinflusst, ist die Leistungsfähigkeit der Lunge, und wichtige Komponenten davon sind Atmung und Körperhaltung.
Das ist wahrscheinlich ein Grund, warum Meditation mit einer Betonung der richtigen Atmung unter Benutzung des Zwerchfells so hilfreich ist.
\textbf{Wenn nur die Rippenmuskulatur zum Atmen benutzt wird, dann wird der Atmungsapparat zu viel und das Zwerchfell zu wenig benutzt.}
Das daraus resultierende schnelle Pumpen des Brustkorbs oder die übertriebene Ausdehnung der Brust kann mit dem Klavierspielen in Konflikt geraten.
Der Gebrauch des Zwerchfells gerät mit den Spielbewegungen weniger in Konflikt.
Wenn beim Spielen Stress aufgebaut wird, werden außerdem diejenigen, die das Zwerchfell nicht bewusst benutzen, eventuell das Zwerchfell anspannen und es nicht einmal merken.
Indem sowohl die Rippen als auch das Zwerchfell benutzt werden und eine gute Haltung aufrechterhalten wird, können die Lungen mit geringstem Aufwand bis zu ihrem maximalen Volumen ausgedehnt werden und somit die maximale Menge an Sauerstoff aufnehmen.

Die folgende Atemübung kann sehr hilfreich sein, nicht nur für das Klavierspielen, sondern auch für das allgemeine Wohlbefinden.
Atmen Sie tief ein, und dehnen Sie dabei Ihren Brustkorb, schieben Sie Ihr Zwerchfell nach unten (Ihr Unterbauch wölbt sich nach außen), heben Sie Ihre Schultern an, und ziehen Sie sie nach hinten; atmen Sie dann vollständig aus, und kehren Sie dabei alle vorigen Bewegungen um.
Wenn Sie tief einatmen, ist ein vollständiges Ausatmen wichtiger als ein vollständiges Einatmen.
Atmen Sie durch die Nase (Sie können dabei den Mund offen lassen oder schließen). Achten Sie aber darauf, dass die Muskeln in der Nase entspannt sind und die Nasenflügel nicht eingezogen werden, und dass der Rachenraum nicht verengt wird, weil das leicht geschehen kann, wenn man angestrengt durch die Nase einatmet. Am besten geht es wahrscheinlich, wenn Sie sich auf das Einatmen durch den Rachen - in der Nähe der Stimmbänder - konzentrieren und die Luft einfach durch die Nase einströmen lassen. Das entspannt die Nasenmuskeln, und der Luftstrom durch die Nase wird größer.
Wenn Sie lange Zeit nicht tief eingeatmet haben, werden Sie wahrscheinlich nach einer oder zwei solcher Übungen hyperventilieren, und es wird Ihnen schwindlig. \textbf{Hören Sie sofort auf, falls Sie hyperventilieren!}
Wiederholen Sie diese Übung dann zu einem späteren Zeitpunkt; Sie sollten dann mehr Atemzüge nehmen können ohne zu hyperventilieren.
Wiederholen Sie diese Übung, bis Sie mindestens fünf Atemzüge hintereinander nehmen können, ohne zu hyperventilieren.
Wenn Sie dann zu Ihrem Arzt gehen und er Sie beim Abhören mit dem Stethoskop bittet, tief einzuatmen, können Sie das tun, ohne dass es Ihnen schwindlig wird!
Normal zu atmen während man etwas Schwieriges spielt, ist ein wichtiges Element der \hyperref[c1ii14]{Entspannung}.
Führen Sie diese Übung mindestens einmal alle paar Monate durch, und bauen Sie sie in Ihre normalen Atemgewohnheiten sowohl am Klavier als auch sonst ein.

\label{c1ii21uebung}\footnote{Achten Sie auch im Alltag hin und wieder auf Ihre Atmung. Atmen Sie dann ein paarmal \enquote{mit dem Bauch} ein und aus - möglichst durch die Nase, nicht extra langsam (Sie sollen ja schließlich keine Atemnot bekommen) aber auch nicht zu schnell, das heißt Sie sollten die Luft nicht mit Kraft durch die Nase strömen, sondern eher wie von selbst ein- und ausfließen lassen. Den Brustkorb, das heißt die Brustmuskulatur, sollten Sie nicht mehr als notwendig einbeziehen. Es kommt nicht darauf an, die Lungen so weit wie möglich zu füllen, sondern darauf, das normale Atemvolumen so entspannt wie möglich aufzunehmen. Diese Übung eignet sich auch hervorragend für die schnelle Entspannung zwischendurch, wenn es mal wieder \enquote{hoch hergeht}. Und wenn Sie schon dabei sind, können Sie auch gleich nachprüfen, ob Ihre Muskulatur angespannt ist. Gute Kandidaten sind zum Beispiel übereinandergeschlagene oder unter dem Bürostuhl versteckte Beine, die Schulter- bzw. Nackenmuskulatur und die Kiefermuskeln.}

Klavierspielenüben kann gesund oder ungesund sein, je nachdem wie man übt.
Viele Schüler vergessen zu atmen, während sie schwieriges Material üben; diese schlechte Angewohnheit ist ungesund.
Sie reduziert den Sauerstoffzufluss zum Gehirn, wenn es ihn am meisten benötigt, was zu Sauerstoffmangel und zu Symptomen führt, die einer Schlafapnoe ähneln (Organschäden, hoher Blutdruck usw.)
Der Sauerstoffmangel erschwert das musikalische und das \hyperref[c1ii12mental]{mentale Spielen} und macht es unmöglich, eine mentale Ausdauer zu entwickeln.

Weitere Methoden zum Erhöhen der Ausdauer sind die Steigerung des Blutflusses und die Vergrößerung der Blutmenge im Körper.
Beim Klavierspielen wird zusätzliches Blut sowohl im Gehirn als auch im Spielmechanismus benötigt; \textbf{deshalb sollten Sie sowohl das Gehirn als auch die Muskeln während des Übens völlig und gleichzeitig trainieren.
Das veranlasst den Körper, als Reaktion auf den erhöhten Blutbedarf, mehr Blut herzustellen.}
Stupide Wiederholungen von Übungen usw. sind in dieser Beziehung nicht hilfreich, weil sie das Gehirn ausschließen und so die Notwendigkeit für mehr Blut reduzieren können.
Nach einer großen Mahlzeit zu üben, erhöht ebenfalls die Blutversorgung, und umgekehrt wird es die Ausdauer reduzieren, wenn man sich nach jeder Mahlzeit ausruht - ein bekanntes japanisches Sprichwort sagt, dass man sich in eine Kuh verwandelt, wenn man nach einem Essen schläft.
Da die meisten Menschen nicht genügend Blut haben, um anstrengende Tätigkeiten mit einem vollen Magen auszuführen, wird der Körper zunächst rebellieren, und Sie werden sich schlecht fühlen, aber das ist eine zu erwartende Reaktion.
Solche Aktivitäten müssen innerhalb sicherer medizinischer Grenzen durchgeführt werden; so können Sie zum Beispiel vorübergehend Probleme mit der Verdauung bekommen, oder Sie sind ein wenig benommen (was wahrscheinlich der Grund für die falsche Auffassung ist, dass man nach einer großen Mahlzeit nicht üben soll).
Wenn der Körper erst das notwendige zusätzliche Blut erzeugt, werden diese Probleme verschwinden.
Sie sollten deshalb nach einer Mahlzeit so aktiv wie möglich bleiben, um einem Blutmangel vorzubeugen.
Das Üben direkt nach einer Mahlzeit führt dazu, dass Blut für die Verdauung, für die Spielmuskeln und für das Gehirn benötigt wird und so die größten Anforderungen an die Blutversorgung gestellt werden.
Klar ist, dass die Teilnahme am Sport, eine gute Gesundheit und ein körperliches Training hilfreich sind, um Ausdauer beim Klavierspielen zu bekommen.

Zusammengefasst: Anfänger, die noch nie zuvor ein Klavier angerührt haben, müssen ihre Ausdauer schrittweise entwickeln, weil Klavierüben eine anstrengende Arbeit \textit{ist}.
Eltern müssen auf die Übungsdauer von sehr jungen Anfängern achten und ihnen erlauben, aufzuhören oder eine Pause einzulegen, wenn sie müde werden (nach ungefähr 10 bis 15 Minuten).
\textbf{Erlauben Sie niemals einem kranken Kind, Klavier zu üben, selbst einfache Stücke, wegen des Risikos die Krankheit zu verschlimmern und von Hirnschädigungen}\footnote{falls durch das Klavierspielen das Fieber stark ansteigt}.
Auf jeder Fertigkeitsstufe haben wir alle mehr Muskeln als wir brauchen, um die Klavierstücke unserer Stufe zu spielen.
Sogar professionelle Pianisten, die jeden Tag mehr als sechs Stunden üben, sehen am Ende nicht aus wie Popeye.
Franz Liszt war dünn, nicht muskulös.
So ist das Aneignen von Technik und Ausdauer keine Frage des Muskelaufbaus, sondern des Lernens wie man entspannt und seine Energie sinnvoll einsetzt.
 

% zuletzt geändert 31.10.2009

\subsection{Schlechte Angewohnheiten: Der größte Feind des Pianisten}
\label{c1ii22}

\textbf{Schlechte Angewohnheiten sind die schlimmsten Zeitverschwender beim Klavierüben.
Die meisten schlechten Angewohnheiten werden durch Stress beim beidhändigen Üben von Stücken, die zu schwierig sind, verursacht.}
Viele der aus dem beidhändigen Üben resultierenden schlechten Angewohnheiten sind schwierig zu diagnostizieren, was sie um einiges schlimmer macht.
Das beste Mittel gegen schlechte Angewohnheiten ist sicherlich das \hyperref[c1ii7]{Üben mit getrennten Händen}.
Unmusikalisches Spielen ist eine der schlechten Angewohnheiten; vergessen Sie deshalb nicht, dass \hyperref[c1iii14d]{musikalisches Üben} mit dem Üben mit getrennten Händen beginnt.

\textbf{Eine weitere schlechte Angewohnheit ist der übermäßige Gebrauch des Halte- oder Dämpferpedals}, wie weiter unten besprochen\footnote{Anmerkungen zu den Bezeichnungen der Pedale finden Sie \hyperref[Pedale]{hier}}.
Das ist das sicherste Zeichen eines Amateurschülers, der Unterricht bei einem unqualifizierten Lehrer nimmt.
Zu häufiger Gebrauch dieser Pedale kann nur einem Schüler mit ernsthaften technischen Defiziten \enquote{helfen}.

\textbf{Eine weitere schlechte Angewohnheit ist, ohne Rücksicht auf die Musikalität auf das Klavier einzuhämmern.}
Der Schüler setzt laut mit aufregend gleich.
Dazu kommt es oft, wenn der Schüler so ins Üben vertieft ist, dass er vergisst, auf die Töne zu hören, die aus dem Klavier kommen.
Das kann vermieden werden, indem man die Angewohnheit entwickelt, sich stets selbst beim Spielen zuzuhören.
Sich selbst zuhören ist viel schwerer als vielen Menschen bewusst ist, weil viele Schüler (besonders diejenigen, die mit Stress spielen) ihre ganze Mühe für das Spielen aufwenden und nichts für das Zuhören übrig bleibt.
Eine Möglichkeit, dieses Problem zu verringern, ist, das \hyperref[c1iii13]{eigene Spielen aufzunehmen}, sodass man es sich mit einem gewissen geistigen Abstand anhören kann.
Aufregende Passagen sind oft laut, aber sie sind dann am aufregendsten, wenn der Rest der Musik leise ist.
Zu viel lautes Üben kann die technische Entwicklung, und dass man auf Geschwindigkeit kommt, verhindern und den Sinn für die Musik ruinieren.
Diejenigen, die laut spielen, haben am Ende oft einen schrillen Klang.

\textbf{Dann sind da noch diejenigen mit schwachen Fingern.}
Dieses Problem ist unter Anfängern weit verbreitet und kann einfacher korrigiert werden als das zu laute Draufhämmern.
Schwache Finger werden dadurch verursacht, dass man die Arme nicht entspannt und der Schwerkraft nicht die Führung überlässt.
Der Schüler hebt unbewusst die Arme, und diese Angewohnheit ist eine Form von Stress.
Diesen Schülern muss man den vollen Dynamikumfang des Klaviers zeigen und wie man ihn benutzt.

Ebenfalls eine schlechte Angewohnheit ist, mit der falschen Geschwindigkeit zu spielen, also entweder zu langsam oder zu schnell - besonders wenn Sie während eines \hyperref[c1iii14]{Auftritts} zu aufgeregt sind und das Gefühl für das Tempo verlieren.
Die richtige Geschwindigkeit wird von mehreren Faktoren bestimmt, einschließlich der Schwierigkeit des Stückes in Bezug auf Ihre technischen Fähigkeiten, was das Publikum erwartet, der Zustand des Klaviers, welches Stück vorausging oder welches diesem folgt usw.
Einige Schüler könnten dazu neigen, Stücke gemäß ihrer Fertigkeitsstufe zu schnell vorzuführen und viele Fehler zu machen, während andere schüchtern sind, zu langsam spielen und so nicht den vollen Gehalt der Musik hervorbringen.
Langsam zu spielen kann schwieriger sein, als mit der richtigen Geschwindigkeit zu spielen, was die Probleme eines schüchternen Spielers verschlimmert.
Diejenigen, die zu schnell vorspielen, können entmutigt werden, weil sie zu viele Fehler begehen, und zu der Überzeugung kommen, dass sie schlechte Klavierspieler sind.
Diese Probleme treffen nicht nur auf Vorführungen zu, sondern auch auf das Üben;
diejenigen, die zu schnell üben, glauben eventuell am Ende, dass sie schlechte Klavierspieler sind, weil sie so viele Fehler machen.
Nur etwas langsamer zu spielen kann dazu führen, dass sie genau und schön spielen und auf lange Sicht die Technik für das schnelle Spielen beherrschen.

Eine schlechte Klangqualität ist ein weiteres verbreitetes Problem.
Während der meisten Zeit hört beim Üben niemand zu, sodass der Klang keine Rolle zu spielen scheint.
Wenn der Klang ein wenig schlechter wird, stört es den Schüler nicht, mit dem Ergebnis, dass der Klang ignoriert wird.
Schüler müssen sich immer um den Klang bemühen, weil er der wichtigste Teil der Musik ist.
Auf einem schlechten oder nicht gut eingestellten Klavier kann man keinen guten Klang erzeugen;
das ist der Hauptgrund, warum man einen guten Flügel statt eines qualitativ schlechten Klaviers möchte und warum das \hyperref[c2_1]{Stimmen}, das Einstellen und das \hyperref[c2_7_hamm]{Intonieren der Hämmer} wichtiger sind als den meisten Schülern bewusst ist.
Gute Aufnahmen anzuhören ist der beste Weg, in dem Schüler das Bewusstsein für die Existenz des guten Klangs zu erwecken.
Wenn sie nur ihr eigenes Spiel anhören, haben sie eventuell keine Ahnung, was guter Klang bedeutet.
Achtet man jedoch erst einmal auf den Klang und fängt an, Resultate zu erzielen, verstärkt sich das selbst, und man kann ohne weiteres die Kunst lernen, Klänge zu produzieren, die ein Publikum anziehen.
Was noch wichtiger ist: Ohne einen guten Klang ist eine fortgeschrittene technische Verbesserung nicht möglich, weil ein guter Klang Kontrolle erfordert und die technische Entwicklung von der Kontrolle abhängt.

\textbf{Stottern} wird durch Üben im \enquote{Stop and Go} verursacht, wenn der Schüler bei jedem Fehler anhält und den Abschnitt noch einmal spielt.
\textbf{Wenn Sie einen Fehler machen, spielen Sie immer durch den Fehler hindurch; halten Sie nicht an, um ihn zu korrigieren.}
Machen Sie im Geiste einen Vermerk an der fehlerhaften Stelle, und spielen Sie den Abschnitt später noch einmal, um zu sehen, ob sich der Fehler wiederholt.
Wenn ja, fischen Sie ein kurzes Stück heraus, das den Fehler enthält, und arbeiten Sie damit.
Haben Sie erst einmal die Angewohnheit entwickelt, durch Fehler hindurchzuspielen, können Sie zur nächsten Stufe aufsteigen, in der Sie Fehler vorhersehen (ihr Kommen fühlen können, bevor sie auftreten) und Ausweichmanöver durchführen können, wie langsamer werden, den Abschnitt vereinfachen oder bloß den \hyperref[c1iii1b]{Rhythmus} beibehalten.
Meistens macht dem Publikum ein Fehler nichts aus, solange der Rhythmus nicht unterbrochen wird, oder es hört den Fehler nicht einmal.

\textbf{Das Schlimmste an den schlechten Angewohnheiten ist, dass es so lange dauert, sie zu eliminieren, besonders wenn sie das beidhändige Spielen betreffen.}
Deshalb beschleunigt nichts Ihre Lernrate mehr als die Kenntnis aller schlechten Angewohnheiten und deren Vermeidung, bevor sie verfestigt sind.
Zum Beispiel \textbf{ist die richtige Zeit, das Stottern zu verhindern, wenn der Schüler das erste Mal mit dem Unterricht beginnt.
Am Anfang stottern die meisten Schüler nicht;
man muss ihnen aber sofort beibringen, durch Fehler hindurchzuspielen.}
Wenn das Hindurchspielen durch Fehler in diesem Stadium gelehrt wird, wird es zur zweiten Natur und ist einfach;
es ist kein zusätzlicher Aufwand nötig, um diesen \enquote{Trick} zu lernen.
Einem Stotterer beizubringen, durch Fehler hindurchzuspielen, ist eine sehr schwierige Aufgabe.

Die Zahl der möglichen schlechten Angewohnheiten ist so groß, dass sie hier nicht alle angesprochen werden können.
Nur so viel sei gesagt: Eine rigorose Einstellung zu schlechten Angewohnheiten ist eine Voraussetzung für rasche Verbesserung.



