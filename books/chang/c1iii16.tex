% File: c1iii16

\subsection{Unterrichten}
\label{c1iii16}

\subsubsection{Lehrer}
\label{c1iii16a}

Klavierspielen zu unterrichten ist ein schwieriger Beruf, weil praktisch alles, was man zu tun versucht, im Widerspruch zu etwas anderem steht, das getan werden sollte.
Wenn man das \hyperref[c1iii11]{Blattspiel} lehrt, ist der Schüler am Ende vielleicht unfähig \hyperref[c1iii6]{auswendig zu lernen}.
Wenn man \hyperref[c1ii17]{langsames, genaues Spielen} lehrt, erwirbt der Schüler eventuell innerhalb eines vernünftigen Zeitraums nicht genügend Technik.
Wenn man sie zu schnell antreibt, vergessen sie vielleicht alles über die \hyperref[c1ii14]{Entspannung}.
Wenn man sich auf die Technik konzentriert, könnte der Schüler das \hyperref[c1iii14d]{musikalische Spielen} aus den Augen verlieren.
Man muß ein System entwickeln, das erfolgreich durch all diese gegensätzlichen Arten von Anforderungen navigiert und immer noch die individuellen Wünsche und Bedürfnisse jedes einzelnen Schülers befriedigt.
Bevor dieses Buch geschrieben wurde, gab es kein Standardlehrbuch, die Lehrer mußten, wenn Sie ihre Laufbahn begannen, erst ihr eigenes Lehrsystem entwickeln, und sie hatten dafür nur sehr wenige Anhaltspunkte.
\textbf{Klavierspielen zu unterrichten ist eine herkulische Aufgabe, die nichts für Hasenfüße ist.}

\textbf{Historisch gesehen lassen sich Lehrer in mindestens drei Kategorien einteilen: Lehrer für Anfänger, Mittelstufenschüler und Fortgeschrittene.}
Der erfolgreichste Ansatz bezieht eine Gruppe von Lehrern ein, die aus allen drei Kategorien zusammengesetzt ist; die Lehrer sind so koordiniert, daß ihre Art zu lehren zueinander paßt und der passende Schüler zum passenden Lehrer geleitet wird.
Ohne eine solche Zusammenarbeit weigerten sich viele Lehrer für fortgeschrittene Schüler, Schüler von bestimmten Lehrern zu nehmen, weil letztere \enquote{nicht die richtigen Grundlagen lehren}.
Das sollte nicht passieren, wenn die Grundlagen standardisiert sind.
Das letzte, was ein Lehrer für Fortgeschrittene möchte, ist ein Schüler, dem am Anfang lauter \enquote{falsche} Methoden beigebracht wurden.
Somit würde eine Standardisierung mittels eines Lehrbuchs, wie diesem hier, solche Probleme lösen.


\subsubsection{Kinder unterrichten, Eltern einbeziehen}
\label{c1iii16b}

\textbf{Kinder sollten im Alter von zwei bis acht Jahren darauf getestet werden, ob sie bereit für den Klavierunterricht sind.
Die ersten Unterrichtsstunden für Anfänger, besonders für junge Kinder, die weniger als sieben Jahre alt sind, sollten kurz sein, höchstens 10 oder 15 Minuten.}
Verlängern Sie die Unterrichtszeit nur, wenn sich ihre Aufmerksamkeitsspanne und Ausdauer steigert.
Wenn mehr Zeit notwendig ist, teilen Sie den Unterricht in mehrere Einheiten mit Pausen dazwischen (\enquote{Keks-Zeit} o.ä.) auf.
Dieselben Regeln gelten für die Übungszeiten zu Hause.
Man kann in 10 Minuten eine Menge unterrichten; es ist besser, wenn man jeden zweiten Tag für 15 Minuten unterrichtet wird (d.h. dreimal pro Woche), als wenn man nur einmal in der Woche für eine Stunde oder länger Unterricht erhält.
Das gilt für alle Altersstufen, obwohl die Zeit zwischen den Unterrichtsstunden mit zunehmenden Alter und zunehmender Fertigkeitsstufe zunimmt.

Es ist für Kinder wichtig, sich Aufnahmen anzuhören.
Sie können sich in jedem Alter Chopin anhören und spielen.
Sie sollten sich auch \hyperref[c1iii13]{Aufnahmen ihres eigenen Spielens} anhören; sonst verstehen sie vielleicht nicht, warum Sie ihre Fehler kritisieren.
Geben Sie ihnen keine Musik, nur weil sie klassisch ist oder von Bach geschrieben wurde.
Spielen Sie, was Ihnen und den Kindern gefällt.

Kinder entwickeln sich sowohl körperlich als auch geistig in Schüben, und sie können nur das lernen, wofür sie geistig reif genug sind es zu lernen.
Mit anderen Worten: Man kann ihnen nicht etwas beibringen, solange sie nicht dafür bereit sind.
\textbf{Deshalb muß ein Teil des Unterrichtens aus einem ständigen Testen des Grades ihrer Reife bestehen:
Tonhöhe, \hyperref[c1iii1b]{Rhythmus}, \hyperref[c1iii12]{absolutes Gehör}, \hyperref[c1iii11]{vom Blatt spielen}, Kontrolle der Finger, Aufmerksamkeitsspanne, das Interesse an der Musik, welches Instrument das beste ist usw.}
Auf der anderen Seite sind die meisten Kinder für viel mehr Dinge bereit als den meisten Erwachsenen bewußt ist, und wenn sie bereit sind, ist der Himmel die Grenze.
Deshalb ist es auch ein Fehler, anzunehmen, daß alle Kinder ständig als Kinder zu behandeln sind.
Sie können in vielerlei Hinsicht erstaunlich entwickelt sein, und sie als Kinder zu behandeln (z.B. indem man sie nur \enquote{Kinderlieder} hören läßt), hält sie nur zurück und beraubt sie der Möglichkeit, ihr volles Potential auszuschöpfen.
Kinderlieder existieren nur in der Vorstellung der Erwachsenen und richten im allgemeinen mehr Schaden an als sie nutzen.

Die Entwicklung des Gehirns und des Körpers kann mit sehr unterschiedlichen Raten erfolgen.
\textbf{Das Gehirn ist dem Körper im allgemeinen weit voraus.
Wegen dieses körperlichen Rückstands nehmen zu viele Eltern an, daß die Entwicklung des Gehirns ebenfalls langsam ist.}
Es ist wichtig, das Gehirn zu testen, seine Entwicklung zu fördern und die körperliche Entwicklung nicht die Entwicklung des Gehirns verlangsamen zu lassen.
Das ist besonders deshalb wichtig, weil das Gehirn die Entwicklung des Körpers beschleunigen kann.
Sprache, Logik und Musik, sowie optische Reize, sind für die Entwicklung des Gehirns am wichtigsten.

\textbf{Mindestens während der ersten zwei Jahre des Unterrichts (bei Kindern länger) müssen Lehrer darauf bestehen, daß die Eltern am Lehr- bzw. Lernprozeß teilhaben.}
Die erste Aufgabe der Eltern ist, die Methoden zu verstehen, die der Lehrer lehrt.\textbf{Da so viele Übungsmethoden und Abläufe zur \hyperref[c1iii14]{Vorbereitung auf Konzerte} kontraintuitiv sind, müssen die Eltern mit ihnen vertraut sein, so daß sie nicht nur dabei helfen können, die Schüler zu leiten, sondern es auch vermeiden, den Anweisungen des Lehrers zu widersprechen.}
Wenn die Eltern nicht am Unterricht teilnehmen, werden sie nach ein paar Lektionen zurückfallen und können sogar zum Hindernis für die Entwicklung des Kindes werden.
Die Eltern müssen an der Entscheidung beteiligt sein, wie lange der Schüler täglich übt, da sie am besten mit all den Zeitanforderungen des Schülers vertraut sind.
Die Eltern kennen auch die endgültigen Ziele des Schülers am besten - ist der Unterricht nur für das Spielen in der Freizeit gedacht oder um zu viel höheren Stufen zu gelangen?
Welche Arten von Musik möchte der Schüler am Ende spielen?
Anfänger benötigen zu Hause immer Hilfe, beim Herausarbeiten des optimalen Ablaufs für das tägliche Üben genauso wie beim Einhalten des wöchentlichen Pensums.
Wenn der Unterricht angefangen hat, ist es erstaunlich, wie oft die Lehrer die Hilfe der Eltern benötigen: wo und wie die Noten gekauft werden, wie oft das Klavier gestimmt wird oder wann man auf ein besseres Klavier umsteigen soll usw.
Die Lehrer und Eltern müssen darin übereinstimmen, wie schnell die Schüler lernen sollen und daran arbeiten, diese Lernrate zu erreichen.
Die Eltern müssen über die Stärken und Schwächen des Schülers informiert sein, damit sie in der Lage sind, ihre Erwartungen und Pläne damit in Einklang zu bringen, was erreichbar ist und was nicht.
\textbf{Am wichtigsten ist, daß es die Aufgabe der Eltern ist, den Lehrer auszuwählen und die richtige Entscheidung darüber zu fällen, wann der richtige Zeitpunkt ist, den Lehrer zu wechseln.}

Dieses Buch sollte sowohl dem Schüler als auch den Eltern als Lehrbuch dienen.
Das spart dem Lehrer sehr viel Zeit, und der Lehrer kann sich dann darauf konzentrieren, die Fertigkeiten zu demonstrieren und Musik zu lehren.
Eltern müssen dieses Buch lesen, damit sie nicht mit den Lehrmethoden des Lehrers in Konflikt geraten.

Schüler brauchen sehr viel Hilfe von ihren Eltern, und die Art der Hilfe ändert sich mit dem Alter.
Wenn sie jung sind, brauchen die Schüler ständige Hilfe bei den täglichen Übungsabläufen: Die Eltern müssen überwachen, daß sie korrekt üben und den Anweisungen des Lehrers folgen.
Es ist in dieser Phase am wichtigsten, korrekte Übungsgewohnheiten zu etablieren.
\textbf{Die Eltern müssen sicherstellen, daß die Schüler es sich während des Übens zur Gewohnheit machen, durch Fehler hindurchzuspielen statt zurückzugehen, was eine Gewohnheit zu stottern erzeugen und den Schüler anfällig für Fehler während der Auftritte machen würde.}
Die meisten Kinder werden die Anweisungen des Lehrers, die während ihrer Unterrichtsstunden eilig gegeben wurden, nicht verstehen; die Eltern können diese Anweisungen eher verstehen.
Wenn die Schüler Fortschritte machen, brauchen Sie eine Rückmeldung, ob sie musikalisch spielen, ob ihr Tempo und ihr Rhythmus genau sind oder ob sie ein Metronom benutzen müssen und ob sie aufhören sollten zu üben und anfangen, sich Aufnahmen anzuhören.

Die geistige Entwicklung ist der Hauptgrund, warum man Kinder klassische Musik hören lassen sollte - der \enquote{Mozart-Effekt}.
Die Argumentation ist ungefähr folgende:
Nehmen Sie an, der durchschnittliche Elternteil hat eine durchschnittliche Intelligenz; dann gibt es eine 50-prozentige Wahrscheinlichkeit, daß das Kind intelligenter als die Eltern ist.
D.h., daß die Eltern nicht auf derselben intellektuellen Stufe wie ihr Baby konkurrieren können!
Wie sollen nun Eltern einem Baby Musik lehren, dessen musikalisches Gehirn sich schnell auf eine Stufe entwickeln kann, die viel höher ist als die seiner Eltern?
Indem man es die großen Klassiker hören läßt!
Lassen Sie es direkt mit Mozart, Chopin usw. reden und von ihnen lernen.
Musik ist eine universelle Sprache; anders als diese verrückten Erwachsenensprachen, die wir sprechen, ist Musik angeboren, so daß Babys mit Musik kommunizieren können, lange bevor sie \enquote{dada} sagen können.
Deshalb kann klassische Musik das Gehirn eines Babys lange bevor die Eltern mit ihm auf niedrigster Ebene kommunizieren können stimulieren.
Und diese Kommunikationen werden auf den Stufen von genialen Komponisten geführt, etwas, von dem wenige Eltern hoffen können, daß sie dazu in der Lage sind!

\textbf{Wie unterrichten Sie ihr Kind?}
Wir befassen uns hier mit der musikalischen Entwicklung und der des Gehirns.
Die Entwicklung des Gehirns ist bereits lange vor der Geburt wichtig.
Deshalb muß die Mutter für eine möglichst streßfreie Umgebung sorgen und auf eine ausgewogene Ernährung achten, darf nicht rauchen, nicht exzessiv Alkohol trinken usw.
Nach der Geburt ist anerkanntermaßen Muttermilch die beste Ernährung.\footnote{Das ist ein schwieriges Thema. Stichworte sind z.B.: Aufbau des Immunsystems, soziale Bindung, gesamte Dauer des Stillens, Belastung mit Schadstoffen.}
Einige Frauen mit kleinen Brüsten fürchten, daß sie nicht genug Milch produzieren könnten, aber diese Furcht ist unbegründet.
Alle Frauen haben dieselbe Zahl von Milchdrüsen; der Unterschied in der Brustgröße wird nur durch die Abweichung der in der Brust gespeicherten Fettmenge\footnote{bzw. implantierten Silikonmenge} verursacht.
Der wichtige Faktor beim Stillen ist, daß regelmäßig und mit beiden Brüsten zu gleichen Anteilen gestillt wird - jede Unterbrechung kann die Milchproduktion in der einen Brust stoppen.
Babys gedeihen in einer \enquote{normalen} Umgebung am besten; der Raum muß nicht besonders ruhig sein, während das Baby schläft (das erzeugt unruhige Schläfer, die nicht genug Schlaf bekommen, wenn irgendwelche Geräusche vorhanden sind); es gibt sogar einige Argumente dafür, einen gewissen Geräuschpegel im Zimmer des Babys zu haben, um bessere Schlafgewohnheiten zu fördern.
Babys sollten an normale Temperaturschwankungen gewöhnt werden - man muß sie nicht mit zusätzlichen Decken zudecken oder ihnen mehr Kleidungsstücke anziehen als einem Erwachsenen.
Babys können jede Stimulation gebrauchen, die Sie ihnen geben können; die hauptsächlichen sind akustisch, visuell, Geschmack, Geruch, sowie der Druck und die Temperatur bei einer Berührung.
So ist das Herumtragen eines Babys sehr gut für die Reizung der Sinne zur Entwicklung des Gehirns; berühren Sie das Baby überall, und sorgen Sie für viele visuelle und akustische Reize.
Füttern Sie sie so früh wie möglich mit Essen mit so vielen unterschiedlichen Gerüchen und soviel unterschiedlichem Geschmack wie möglich.
Berichten zufolge hat ein Baby bei der Geburt mehr Hirnzellen als ein Erwachsener, obwohl das Gehirnvolumen nur ein Viertel dessen von Erwachsenen beträgt.
Die Stimulation läßt einige Zellen wachsen, und der Mangel an Reizen läßt andere verkümmern und verschwinden.

Zum Unterrichten von Babys ist der wichtigste Schritt, laufend zu testen, was sie zu lernen bereit sind.
Nicht alle Babys werden zu Pianisten, obwohl sie in diesem Stadium praktisch zu jedem Talent hingeführt werden können, und Eltern sind am besten dafür gerüstet, ihre Kinder in eine Karriere zu leiten, für die sie selbst über das Fachwissen verfügen.
\textbf{Babys können sofort nach der Geburt hören.}
Viele Krankenhäuser untersuchen Babys gleich nach der Geburt, um hörbehinderte Babys zu erkennen, die sofort eine spezielle Behandlung benötigen.
Da hörbehinderte Babys keine Klangreize erhalten, wird sich die Entwicklung ihres Gehirns verlangsamen; das ist ein weiterer Beweis dafür, daß Musik der Entwicklung des Gehirns förderlich sein kann.
Bei Babys ist das Gedächtnis für Geräusche von außen fast leer.
Deshalb ist jedes Geräusch, das in diesem Stadium gehört wird, etwas besonderes, und alle nachfolgenden Geräusche werden mit diesen anfänglichen Geräuschen verglichen.
Zusätzlich benutzen Babys (der meisten Spezies, nicht nur menschliche) Geräusche, um die Eltern zu erkennen und sich an sie zu binden (üblicherweise an die Mutter).
Von allen Klangeigenschaften, die das Baby für dieses Erkennen benutzt, ist die absolute Tonhöhe wahrscheinlich eine Haupteigenschaft.
Diese Überlegungen erklären, warum fast jedes Kind sich so leicht ein \hyperref[c1iii12]{absolutes Gehör} aneignen kann.
Einige Eltern setzen ihre Babys bereits vor der Geburt Musik aus, um die Entwicklung des Babys zu beschleunigen, aber ich frage mich, ob das für ein absolutes Gehör hilfreich ist, weil die Schallgeschwindigkeit im Fruchtwasser anders ist als in der Luft, was zu einer scheinbar unterschiedlichen Frequenz führt.
Deshalb kann diese Praxis eventuell das absolute Gehör verwirren, wenn sie denn überhaupt funktioniert.
Um ein absolutes Gehör aufzubauen ist ein elektronisches Klavier besser als ein akustisches, weil es immer richtig gestimmt ist.

\textbf{Praktisch jeder Musiker, Athlet usw. von Weltklasse hatte Eltern\footnote{oder andere Förderer}, die ihn bereits in frühen Jahren unterrichteten}; d.h. \enquote{Wunderkinder} werden erzeugt und nicht geboren, und Eltern\footnote{s.o.} üben einen größeren Einfluß auf das Erzeugen von \enquote{Wunderkindern} aus als Lehrer.
Testen Sie das Kind hinsichtlich Gehör, \hyperref[c1iii1b]{Rhythmus} (in die Hände klatschen), Tonhöhe (singen), Kontrolle der Bewegungsabläufe, Aufmerksamskeitsspanne, was sie interessiert usw.
Sobald sie bereit sind (laufen, sprechen, Musik usw.), muß man sie unterrichten.
Babys zu unterrichten, ist etwas anderes, als Erwachsene zu unterrichten.
Erwachsene müssen unterrichtet werden; bei jungen Kindern muß man nur dem Gehirn das Konzept vermitteln und dann eine unterstützende Umgebung zur Verfügung stellen, wenn das Gehirn diese Richtung einschlägt.
Kinder können schnell so weit voranschreiten, daß Sie sie nicht weiter unterrichten können.
Gute Beispiele sind das \hyperref[c1ii12mental]{mentale Spielen} und die \hyperref[c1iii12]{absolute Tonhöhe}.
Erwecken Sie das mentale Spielen, indem Sie sie Musik hören lassen und sie bitten, Ihnen das Stück vorzusingen.
Vermitteln Sie ihnen die Vorstellung, daß sie Musik im Kopf haben, und daß die Musik nicht bloß durch die Ohren hereinkommt.
Stellen Sie sicher, daß sie sich Musik anhören, die in der richtigen Tonhöhe gespielt wird.
Bringen Sie ihnen dann die Tonleiter bei (benutzen Sie C-D-E usw., nicht do-re-mi, das sollte später kommen), und testen Sie sie danach in der C4-Oktave.
In diesem Alter erfolgt das Lernen der absoluten Tonhöhen automatisch und fast sofort; wenn man ihnen das C4 lehrt, werden sie erkennen, daß keine andere Note ein C4 ist, weil sie keine andere Erinnerung haben, die sie durcheinanderbringen kann.
Deshalb ist es so entscheidend, sie zu unterrichten, sobald sie dazu bereit sind.
Lehren Sie ihnen anschließend die höheren und tieferen Noten, das Konzept der relativen Tonhöhen, wie z.B. Oktaven, dann Intervalle aus zwei Noten (das Kind muß beide Noten identifizieren), dann Akkorde aus drei Noten oder drei beliebige Noten, die gleichzeitig gespielt werden, und immer so weiter, wenn möglich bis zu zehn Noten.
Diese musikalischen Lektionen können im Alter zwischen zwei und acht Jahren gegeben werden.
Unterstützen Sie das mentale Spielen, indem Sie ihnen viel gute Musik zum Zuhören anbieten und sie darauf trainieren, die Namen und den Komponisten der Kompositionen zu kennen.
Singen oder ein einfaches (richtig gestimmtes) musikalisches Spielzeug ist eine gute Möglichkeit, die Tonhöhe, den Rhythmus und die Kontrolle der Bewegungen zu lehren.
Verankern Sie die Vorstellung, daß die Musik ständig in ihrem Kopf ablaufen kann.
Sobald sie mit dem Klavierunterricht beginnen, wird das mentale Spielen durch das \hyperref[c1iii6]{Auswendiglernen} und den Aufbau eines auswendig gelernten Repertoires weiterentwickelt.
Seien Sie darauf vorbereitet, sie zu unterstützen, wenn sie sofort mit dem Komponieren beginnen - bieten Sie ihnen Möglichkeiten, ihre Stücke \hyperref[c1iii13]{aufzuzeichnen} oder lehren Sie ihnen Diktate.
Lange vor ihrer ersten Klavierstunde können Sie ihnen Bilder von vergrößerten Noten zeigen und sie mit dem Notensystem, wo die Noten stehen und wo sie auf dem Klavier zu finden sind vertraut machen.
Das wird die Aufgabe des Lehrers vereinfachen, ihnen das Notenlesen beizubringen.
Wenn Sie kein Klavierspieler sind, können Sie zur selben Zeit Klavierunterricht nehmen, wie Ihr Kind; das ist eine der besten Möglichkeiten, sie zum Anfangen zu bewegen.

Denken Sie vor allem daran, daß jedes Kind Stärken und Schwächen hat.
Es ist die Aufgabe der Eltern, die Stärken herauszufinden und zu unterstützen, und die Stärken werden nicht immer in die Richtung einer Karriere als Pianist weisen.
Sie müssen auch in Sport, Literatur, Wissenschaft, Kunst usw. getestet werden, weil jedes Kind ein Individuum ist.
Seien Sie nicht enttäuscht, wenn die Tests zeigen, daß das Kind die meiste Zeit noch nicht bereit ist - das ist normal.
Eine grundlegende Ausbildung am Klavier, die einer auf Wissen basierenden Methode, ähnlich einer Methode zur Projektsteuerung, wie sie in diesem Buch benutzt wird, folgt, wird dem Kind jedoch, unabhängig davon, welche Karriere es wählt, von Nutzen sein.

Eltern müssen die körperliche und geistige Entwicklung ihrer Kinder in der Balance halten.
Da das Klavierspielenlernen so schnell gehen kann, sind diese alten Zeiten, in denen fleißige Klavierspieler nicht genügend Zeit für Sport und andere Aktivitäten hatten, vorbei.
Techniker und Künstler müssen nicht zwangsläufig zu Weichlingen werden.
Es gibt diese verstörende Neigung, jedes Kind als hirn- oder körperlastig einzuordnen und eine Wand zwischen oder sogar einen gegenseitigen Ausschluß von Kunst und körperlichen Aktivitäten, Wissenschaft usw. zu erzeugen.
In Wahrheit folgen alle ähnlichen Prinzipien.
So sind z.B. die Regeln für das Lernen des Golfspielens und des Klavierspielens so ähnlich, daß dieses Buch mit nur ein paar Änderungen in ein Golfhandbuch umgewandelt werden kann.
Die Griechen hatten bereits vor langer Zeit recht - die geistige und körperliche Entwicklung müssen parallel verlaufen -, heute können wir sogar noch mehr tun.
Wenn die Eltern nicht die richtige Anleitung geben, werden einige Kinder ihre ganze Zeit einer Sache widmen und alles andere vernachlässigen, psychologische Probleme entwickeln und kostbare Zeit verschwenden.
\hyperref[c1iii10krank]{Gesundheit und Verletzungen} sind ein weiteres Thema.
Die Musikgeräte mit Kopfhörern können \hyperref[c1iii10gehoer]{das Gehör schädigen}, so daß man bereits bevor man 40 Jahre alt wird beginnt, das Gehör zu verlieren und unter einem Tinnitus leidet.
Eltern müssen ihren Kindern beibringen, die Lautstärke der Kopfhörer herunterzudrehen, besonders wenn sie eine Art von Musik hören, die oft extrem laut gespielt wird.
 

\subsubsection{Auswendiglernen, Blattspiel, Theorie, mentales Spielen, absolutes Gehör}
\label{c1iii16c}

Der Lehrer muß zu einem frühen Zeitpunkt wählen, ob dem Schüler das \hyperref[c1iii6]{Spielen aus dem Gedächtnis} gelehrt werden sollte oder das \hyperref[c1iii11]{Spielen vom Blatt}.
Diese Wahl ist notwendig, weil die Details des Unterrichtsprogramms und wie der Lehrer mit dem Schüler zusammenarbeitet davon abhängen.
\textbf{Die Suzuki-Violin-Methode betont das Spielen aus dem Gedächtnis zu Lasten des Notenlesens, besonders für Kinder, und das ist auch für das Klavier der beste Ansatz.}
Es ist einfacher, das Blattspiel zu üben, \textit{nachdem} man ziemlich gut spielen kann,
so wie man erst das Sprechen lernt und dann das Lesen.
Die Fähigkeit zu sprechen und zu musizieren sind natürliche evolutionäre Eigenschaften, über die wir alle verfügen; das Lesen kam erst später als Konsequenz unserer Zivilisation hinzu.
Das Sprechenlernen ist einfach ein Prozeß, alle Klänge und logischen Konstrukte jeder einzelnen Sprache auswendig zu lernen.
Deshalb ist das Lesen \enquote{fortgeschrittener} und weniger \enquote{natürlich} und kann dem Auswendiglernen logischerweise nicht vorausgehen.
Man hat sich z.B. (durch das Hören von Aufnahmen) viele musikalische Konzepte gemerkt, die niemals niedergeschrieben werden können, wie Farbe, das Spielen mit Autorität und Überzeugung usw.

Das Notenlesen sollte jedoch am Anfang nicht völlig vernachlässigt werden.
Es ist nur eine Frage der Priorität.
Da die Musiknotation einfacher als jedes Alphabet ist, können Kinder sogar das Notenlesen lernen, bevor sie das Bücherlesen lernen können.
Deshalb sollte das Notenlesen von Anfang an gelehrt werden, aber nur soweit, daß das Kind in der Lage ist, die Noten zu lesen, um ein Stück zu üben und auswendig zu lernen.
\textbf{Das Blattspiel sollte ermutigt werden, solange es nicht das Spielen aus dem Gedächtnis stört.}
Das bedeutet, daß bei einem Stück, das bereits auswendig gelernt wurde, die Notenblätter nicht mehr zum täglichen Üben benutzt werden sollten.
Der Lehrer muß jedoch sicherstellen, daß diese geringe Betonung des Blattspiels nicht zu einem schlechten Blattspieler führt, der automatisch alles auswendig lernt aber keine Noten lesen kann.
Die meisten Anfänger neigen dazu, entweder gute Blattspieler und schlechte Auswendiglernende oder gute Auswendiglernende und schlechte Blattspieler zu werden, denn wenn man bei dem einen gut wird, benötigt man das andere weniger.
Durch die sorgfältige Überwachung des Schülers kann ein Elternteil oder ein Lehrer verhindern, daß der Schüler zu einem schlechten Blattspieler oder einem schlechten Auswendiglernenden wird.
Die Hilfe der Eltern ist oft notwendig, damit diese Überwachung erfolgreich ist, da der Lehrer nicht immer dabei ist, wenn der Schüler übt.
Viele Eltern erzeugen sogar unabsichtlich schlechte Auswendiglernende oder schlechte Blattspieler, weil sie ihren Kindern aushelfen, anstatt sie ihre schwächeren Fertigkeiten trainieren zu lassen.
Da es längere Zeit dauert, bis man zu einem schlechten Blattspieler oder Auswendiglernenden wird, üblicherweise viele Jahre, ist genügend Zeit vorhanden, den Trend zu erkennen und zu korrigieren.
Genau wie Talente, Wunderkinder oder Genies werden Blattspieler und Auswendiglernende nicht geboren sondern erzeugt.

Notenlesen ist für Lehrer ein unverzichtbares Unterrichtsmittel; die Aufgabe des Lehrers kann vereinfacht werden, wenn dem Schüler das Notenlesen beigebracht werden kann.
Lehrer, die das Notenlesen betonen, haben sicherlich wegen der enormen Informationsmenge, die bereits in der einfachsten gedruckten Musik enthalten ist, recht, und praktisch jeder Anfänger wird einen großen Teil dieser Information verpassen.
Sogar fortgeschrittene Klavierspieler kehren oft zu den Notenblättern zurück, um sicherzustellen, daß sie nichts vergessen haben.
Klar basiert das beste Programm auf dem Auswendiglernen, es muß aber ein ausreichendes Training des Notenlesens beinhalten, damit der Schüler kein schlechter Blattspieler wird.

Der normale zum Lernen der neuen Stücke notwendige Aufwand des Notenlesens ist im allgemeinen ausreichend.
Besonders für Anfänger zahlt es sich nicht aus, das Notenlesen zu vertiefen, nur damit man vom Blatt spielen kann (da die Finger die Stücke ohnehin nicht spielen können), obwohl die anfängliche langsame Lesegeschwindigkeit sowohl für den Lehrer als auch den Schüler schrecklich frustrierend sein kann.
Ein wichtiger Lerntrick in der Klavierpädagogik ist, mehrere Fertigkeiten gleichzeitig zu lernen, besonders weil es bei vielen Fertigkeiten so lange dauert sie zu lernen.
So können das Auswendiglernen, das Blattspiel, die Theorie usw. alle gleichzeitig gelernt werden, was auf die Dauer viel Zeit spart.
Zu versuchen, eine dieser Fertigkeiten zu Lasten der anderen schnell zu lernen, kann nur zu Frustrationen führen.

\textbf{Man kann nicht zuviel Musiktheorie (Solfège), Notation, Diktate, \hyperref[c1iii12]{absolute Tonhöhenerkennung}, \hyperref[c1iii1b]{Rhythmus} usw. unterrichten.}
Theorie zu lernen hilft den Schülern beim Erwerb der Technik, Auswendiglernen, Verstehen der Struktur der Komposition und beim richtigen Aufführen.
Es wird auch beim Improvisieren und beim Komponieren hilfreich sein.
Statistisch gesehen komponiert die Mehrheit der erfolgreichen Klavierschüler am Ende selbst Musik.
Das einzige Problem mit Solfège-Stunden ist, daß viele Lehrer ineffizient unterrichten und viel Zeit der Schüler verschwenden.
Moderne Musik (Pop, Jazz usw.) benutzt heutzutage sehr fortgeschrittene musikalische Konzepte, und die Theorie ist für das Verständnis von Akkordprogressionen, Musikstruktur und Improvisation hilfreich.
Deshalb \textbf{ist es vorteilhaft, sowohl klassische als auch moderne Musik zu lernen.
Moderne Musik trägt zeitgenössische Theorie bei, hilft bei der Entwicklung von Rhythmus und den Fertigkeiten zum \hyperref[c1iii14]{Auftreten} und erreicht auch ein breiteres Publikum.}
\label{c040119}
\textit{[Weitere Informationen über das Improvisieren finden Sie u.a. in Marc Sabatellas \enquote{A Jazz Improvisation Primer}: \hyperref[http://www.outsideshore.com/primer/primer/index.html]{das Original in Englisch} <font color=\enquote{blue} size=\enquote{-1}>(extern), \hyperref[http://msjipde.uteedgar-lins.de/index.html]{als deutsche Übersetzung} (extern).]}</font>

\textbf{Das \hyperref[c1ii12mental]{mentale Spielen} sollte von Anfang an gelehrt werden, damit die Schüler stets üben, in Gedanken Musik zu spielen.
Wenn das mit der richtigen Tonhöhe geschieht, dann werden junge Schüler, die oft genug Musik gehört haben, nach nur wenigen Lektionen ohne Aufwand ein \hyperref[c1iii12]{absolutes Gehör} erwerben.}
Das ist ein guter Zeitpunkt, um diejenigen Schüler zu ermitteln, die nur eine geringe Vorstellung von Tonhöhen haben, und ein Verfahren zu entwickeln, um ihnen zu helfen.
Fortgeschrittene Schüler entwickeln automatisch Fähigkeiten zum mentalen Spielen, weil diese so notwendig sind; wenn man es ihnen jedoch von Anfang an lehrt, wird das ihre Lernrate für alles andere beschleunigen.
Wenn das mentale Spielen nicht gelehrt wird, werden die Schüler nicht einmal merken, daß sie es benutzen, und es nicht richtig entwickeln.
Außerdem werden sie, da ihnen nicht bewußt ist was sie tun, dazu neigen, das mentale Spielen zu vernachlässigen, wenn sie älter werden und ihr Gehirn mit anderen wichtigen Dingen beschäftigt ist.
Wenn sie das mentale Spielen vernachlässigen, werden sie ihr absolutes Gehör und die Fähigkeit, mit Leichtigkeit vorzuspielen, verlieren.
Ältere Schüler und Erwachsene, die das mentale Spielen und ein absolutes Gehör erlernen möchten, sollten sich \hyperref[c1iii12]{\autoref{c1iii12}} ansehen.



