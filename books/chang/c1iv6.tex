% File: c1iv6

\section{Noch zu erforschende Themen}
\label{c1iv6}

Die einzelnen Punkte dieses Abschnitts sind unvollständig; ich habe nur einige erste Ideen niedergeschrieben.

Dieses Buch basiert auf einem wissenschaftlichen Ansatz, was sicherstellt, daß Fehler so schnell wie möglich korrigiert werden, alle bekannten Fakten erklärt, dokumentiert und in einer nützlichen Weise geordnet werden und wir insgesamt nur voranschreiten.
In der Vergangenheit war es oft so, daß ein Klavierlehrer eine sehr nützliche Methode lehrte und ein anderer Lehrer nichts davon wußte oder zwei Lehrer völlig entgegengesetzte Methoden lehrten.
Das sollte nicht geschehen.
Ein wichtiger Teil des wissenschaftlichen Ansatzes ist eine Diskussion darüber, was immer noch unbekannt ist und was noch erforscht werden muß.
Es folgt eine Sammlung solcher Themen.


\subsection{Impulstheorie des Klavierspielens}
\label{c1iv6a}

Langsames Klavierspielen wird \enquote{Spielen im statischen Gleichgewicht} genannt.
Das bedeutet, daß die Kraft des herunterkommenden Fingers beim Herunterdrücken der Taste die hauptsächliche zum Spielen benutzte Kraft ist.
Wenn wir schneller werden, gehen wir vom statischen Gleichgewicht zum dynamischen Gleichgewicht über.
Das bedeutet, daß der Impuls der Hände, Arme, Finger, usw. eine viel wichtigere Rolle spielt als die Kraft, mit der die Tasten heruntergedrückt werden.
Selbstverständlich wird eine Kraft benötigt, um die Taste herunterzudrücken, aber im dynamischen Gleichgewicht sind die Phasen der Kraft und der Bewegung üblicherweise um 180  Grad zueinander versetzt, d.h. ihr Finger bewegt sich nach oben, wenn Ihre Fingermuskeln versuchen ihn herunterzudrücken!
Das geschieht bei hoher Geschwindigkeit, weil Sie den Finger vorher so schnell angehoben haben, daß Sie auf seinem Weg nach oben anfangen müssen herunterzudrücken, so daß Sie seine Bewegung für den nächsten Anschlag umkehren können.
Die wahren Bewegungen sind komplex, weil Sie die Hände, die Arme und den Körper benutzen, um die Impulse abzugeben und abzufangen.
Das ist einer der Gründe, warum der ganze Körper am Spielen beteiligt wird, besonders wenn man schnell spielt.
Beachten Sie, daß das Schwingen des Pendels und das Dribbeln des Basketballs im dynamischen Gleichgewicht stattfinden.
Beim Klavierspielen befinden Sie sich im allgemeinen irgendwo zwischen dem statischen und dem dynamischen Gleichgewicht - mit einer zunehmenden Tendenz zum dynamischen Gleichgewicht bei steigender Geschwindigkeit.

Beim statischen Spielen sind der Kraftvektor und die Bewegung des Fingers in Phase.
Wenn wir zum dynamischen Spielen übergehen, baut sich eine Phasendifferenz auf, bis sie im reinen dynamischen Gleichgewicht 180 Grad beträgt, wie es beim Pendel der Fall ist.

Die Wichtigkeit des dynamischen Spielens ist offensichtlich; es bezieht viele neue Finger- bzw. Handbewegungen ein, die im statischen Spielen nicht möglich sind.
Deshalb trägt das Wissen, welche Bewegungen der statischen oder dynamischen Art sind, viel zu dem Verständnis dafür bei, wie sie auszuführen sind und wann man sie benutzen muß.
Da das dynamische Spielen bis jetzt nie in der Literatur besprochen wurde, gibt es im Klavierspielen ein großes Gebiet, von dem wir sehr wenig verstehen.


\subsection{Die Physiologie der Technik}
\label{c1iv6b}

Wir haben immer noch ein sehr primitives Verständnis der biomechanischen Prozesse, die der Technik zugrunde liegen.
Sie hat ihren Ursprung sicherlich im Gehirn und ist wahrscheinlich damit verbunden, wie die Nerven mit den Muskeln kommunizieren, besonders mit den schnellen Muskeln.
Was sind die biologischen Veränderungen, die mit der Technik einhergehen?
Wann sind Finger \enquote{aufgewärmt}?


\subsection{Gerhirnforschung (HS- und HT-Spielen, usw.)}
\label{c1iv6c}

Die Gehirnforschung wird in naher Zukunft eines der wichtigsten Gebiete der medizinischen Forschung sein.
Diese Forschung wird sich anfänglich auf die Verhinderung des geistigen Verfalls durch das Altern (z.B. die Heilung von Alzheimer) konzentrieren.
Es werden sicherlich gleichzeitig Anstrengungen zur tatsächlichen Kontrolle des Wachstums der geistigen Fähigkeiten unternommen.
Die Musik sollte bei solchen Entwicklungen eine wichtige Rolle spielen, weil wir mit Kindern aural \footnote{d.h. über die Ohren} kommunizieren können, lange bevor wir es mit einer anderen Methode können, und es ist bereits klar, daß die Resultate um so besser sind, je früher man den Kontrollprozeß beginnt.

Wir sind alle mit der Tatsache vertraut, daß sogar wenn wir ziemlich gut HS spielen können, HT trotzdem sehr schwierig sein kann.
Warum ist HT soviel schwieriger?
Einer der Gründe mag sein, daß die beiden Hände von den verschiedenen Hälften des Gehirns gesteuert werden.
Wenn das so ist, dann erfordert das Lernen von HT, daß das Gehirn Wege entwickelt, die beiden Hälften zu koordinieren.
Das würde bedeuten, daß das HS- und das HT-Üben völlig verschiedene Arten von Gehirnfunktionen benutzen und stützt die Behauptung, daß diese Fertigkeiten separat entwickelt werden sollten, so daß wir jeweils an einer Fertigkeit arbeiten können.
Eine faszinierende Möglichkeit wäre, wenn wir parallele Sets für HT entwickeln könnten, die dieses Problem lösen.


\subsection{Was verursacht Nervosität?}
\label{c1iv6d}

In der Klavierpädagogik wurde die Nervosität zu lange \enquote{unter den Teppich gekehrt} (ignoriert).
Wir müssen sie von einem medizinischen und psychologischen Standpunkt aus untersuchen.
Wir müssen wissen, ob einzelne Menschen von einer angemessenen Medikation profitieren können.
Gibt es darüber hinaus eine medizinische oder psychologische Behandlung, mit deren Hilfe die Nervosität schließlich überwunden werden kann?
Von einem formalen psychologischen Standpunkt aus müssen wir eine Lehrprozedur entwickeln, die die Nervosität reduziert.
Nervosität ist sicherlich das Resultat einer mentalen Haltung, Reaktion und Wahrnehmung und ist deshalb der aktiven Kontrolle sehr zugänglich.
Klavierspieler, die Pop- oder Jazz-Musik spielen, scheinen im allgemeinen viel weniger nervös zu sein als diejenigen, die klassische Musik spielen.
Es gibt keinen Grund, warum wir nicht untersuchen sollten, warum das so ist, und einen Vorteil aus diesem Phänomen ziehen.


\subsection{Ursachen von und Mittel gegen Tinnitus}
\label{c1iv6e}

Struktur der Cochlea, hoch- und niederfrequenter Tinnitus.

Es gibt Anzeichen dafür, daß die moderate Einnahme von Aspirin den altersbedingten Gehörverlust verlangsamen kann.
Es gibt jedoch auch Anzeichen dafür, daß Aspirin einen Tinnitus unter bestimmten Bedingungen verschlimmern kann \footnote{auch kann es z.B. dem Magen und den Nieren schaden}.
Es gibt anscheinend keinen Beweis dafür, daß Tinnitus nur durch das Altern verursacht wird; statt dessen gibt es zahlreiche Beweise, daß er durch Infektionen, Krankheiten und Mißbrauch des Gehörs verursacht wird.
Deshalb können in den meisten dieser Fälle die Ursachen und die Arten der Schäden direkt studiert werden.


\subsection{Was ist Musik?}
\label{c1iv6f}

Struktur der Cochlea und die Beziehung zu Tonleitern und Akkorden.
Parameter: Timing (Rhythmus), Tonhöhe, Muster (Sprache, Gefühle), Lautstärke, Geschwindigkeit.
Musikalische Informationsverarbeitung im Gehirn.


\subsection{In welchem Alter soll bzw. darf man mit dem Klavierspielen anfangen?}
\label{c1iv6g}

Wir brauchen medizinische, psychologische und soziologische Studien darüber, wie bzw. wann Kinder anfangen sollten.
Einzelne Sportorganisationen haben, zumindest informell, bereits diese Art der Forschung für den Bereich des Sports begonnen und Methoden zum Unterricht von Kindern bis herunter auf ein Alter von ungefähr zwei Jahren entwickelt.
In der Musik können wir beginnen, sobald ein Baby geboren ist, indem wir es die angemessenen Arten von Musik hören lassen.
Bei der Musik sind wir wahrscheinlich mehr an der Entwicklung des Gehirns als am Erwerb von Fertigkeiten der Bewegung interessiert.
Da wir erwarten, daß die Gehirnforschung in naher Zukunft rapide zunimmt, ist das ein günstiger Zeitpunkt, um einen Vorteil aus dieser Forschung zu ziehen und die Resultate für das Klavierspielenlernen zu benutzen.


\subsection{Die Zukunft des Klavierspielens}
\label{c1iv6h}

Zum Schluß ein Ausblick in die Zukunft.
Der Abschnitt mit den \hyperref[testimonials]{Leserkommentaren} ist ein deutlicher Hinweis darauf, daß unser neuer Ansatz zum Klavierüben praktisch jeden befähigen wird, das Klavierspielen zu seiner Zufriedenheit zu lernen.
Er wird sicherlich die Zahl der Klavierspieler erhöhen.
Deshalb sind die folgenden Fragen sehr wichtig:

\begin{enumerate} 
 \item Können wir die zu erwartende Zunahme an Klavierspielern berechnen?
 \item Was bedeutet dieser Anstieg für die wirtschaftliche Seite des Klavierspielens: Künstler, Lehrer, Techniker und Hersteller?
 \item Wenn die Popularität des Klaviers rapide zunimmt, was wird die hauptsächliche Motivation für so viele Menschen sein, das Klavierspielen zu lernen?
 \end{enumerate}
Klavierlehrer werden zustimmen, daß 90\% der Klavierschüler das Klavierspielen in dem Sinn niemals richtig lernen, daß sie nicht in der Lage sind, zu ihrer Zufriedenheit zu spielen und es im Grunde aufgeben, vollendete Klavierspieler zu werden.
Da das ein wohlbekanntes Phänomen ist, hält es Kinder und ihre Eltern davon ab, sich dafür zu entscheiden, mit dem Klavierunterricht zu beginnen.
Da ernsthaftes Befassen mit dem Klavier das Verdienen des Lebensunterhalts wesentlich beeinträchtigt, hält der wirtschaftliche Faktor ebenfalls vom Einstieg in das Klavierspielen ab.
Es gibt viele weitere negative Faktoren, die die Beliebtheit des Klaviers begrenzen (Mangel an guten Lehrern, hohe Kosten für gute Klaviere und ihre Wartung, usw.),
die letztendlich fast alle mit der Tatsache zusammenhängen, daß es so schwierig war, Klavierspielen zu lernen.
Wahrscheinlich haben nur 10\% von denen, die versucht haben könnten, das Klavierspielen zu lernen, sich auch dazu entschieden.
Deshalb können wir ernsthaft erwarten, daß die Beliebtheit des Klaviers hundertfach gesteigert wird, wenn sich die Erwartungen dieses Buchs erfüllen.

Eine solche Steigerung würde bedeuten, daß ein großer Teil der Bevölkerung in den entwickelten Ländern das Klavierspielen lernt.
Da dieses ein bedeutender Teil ist, brauchen wir keine exakte Zahl, nehmen wir also eine vernünftige Zahl, z.B. 30\%.
Das würde mindestens eine zehnfache Zunahme der Zahl der Klavierlehrer erfordern.
Das wäre großartig für die Schüler, weil es heutzutage eines der großen Probleme ist, gute Lehrer zu finden.
In jeder Region gibt es zur Zeit nur ein paar Lehrer, und die Schüler haben wenig Auswahl.
Die Zahl der verkauften Klaviere würde ebenfalls zunehmen, wahrscheinlich um mehr als 300\%.
Viele Haushalte besitzen zwar bereits ein Klavier, viele davon sind jedoch nicht spielbar.
Da die meisten der neuen Klavierspieler auf einem fortgeschrittenen Niveau sein werden, wird die Zahl der notwendigen guten Flügel um einen noch größeren Prozentsatz steigen.

Indem sie dieses Buch als Grundlage für die Übungsmethoden benutzen, können sich die Klavierlehrer auf das konzentrieren, was sie am besten tun: lehren Musik zu machen.
Da Lehrer das die ganze Zeit getan haben, werden nur geringfügige neue Änderungen in der Art wie die Lehrer unterrichten notwendig sein.
Das einzige neue Element ist das Hinzufügen der Übungsmethoden, die innerhalb kurzer Zeit zu lernen sind.
Die größte Veränderung ist natürlich, daß Lehrer von dem alten langsamen Prozeß befreit werden, Technik zu lehren.
Es wird für Lehrer viel leichter sein, zu entscheiden was sie lehren, weil technische Schwierigkeiten ein viel geringeres Hindernis sein werden.
Innerhalb weniger Generationen von Lehrern und Schülern wird sich die Qualität der Lehrer dramatisch verbessern, was die Lernraten zukünftiger Schüler weiter beschleunigen wird.

Ist eine hundertfache Zunahme der Anzahl der Klavierspieler realistisch?
Was würden sie tun?
Sie können mit Sicherheit nicht alle Konzertpianisten und Klavierlehrer sein.
Die ganze Art, wie wir das Klavierspielen sehen, wird sich verändern.
Vor allem wird das Klavier bis dahin zu einem Standard-Zweitinstrument für alle Musiker werden, weil es so einfach sein wird es zu lernen und es überall Klaviere geben wird.
Die Freude am Klavierspielen wird für viele Belohnung genug sein.
Die Vielzahl der Musikliebhaber, die bisher nur Aufnahmen hören konnte, kann nun ihre eigene Musik spielen - eine viel befriedigendere Erfahrung.
Jeder, der ein vollendeter Klavierspieler geworden ist, wird Ihnen bestätigen, daß man, wenn man dieses Niveau erreicht hat, gar nicht anders kann, als anzufangen Musik zu komponieren.
Somit wird die Klavierrevolution auch eine Revolution in der Komposition in Gang setzen, und es wird eine große Nachfrage nach neuen Kompositionen bestehen, weil viele Klavierspieler nicht damit zufrieden sind, immer \enquote{dieselben alten Sachen} zu spielen.
Klavierspieler werden wegen der Entwicklung der Keyboards mit leistungsstarker Software Musik für jedes Instrument komponieren, und jeder Klavierspieler wird ein akustisches Klavier und ein elektronisches Keyboard oder ein Doppelinstrument besitzen (s.u.).
Die große Versorgung mit guten Keyboardspielern würde bedeuten, daß ganze Orchester aus Keyboardspielern bestehen werden.
Ein weiterer Grund, warum das Klavier allgemein beliebt werden würde, ist, daß es als eine Methode zur Steigerung des IQ von heranwachsenden Kindern benutzt wird.
Die Gehirnforschung wird sicherlich offenbaren, daß die Intelligenz durch die richtige Stimulation des Gehirns während der frühen Entwicklungsstadien verbessert werden kann.
Da es nur zwei Eingangskanäle zum Gehirn kleiner Kinder gibt, akustisch und visuell, und der akustische Teil zu Beginn weiter entwickelt ist als der visuelle, ist Musik das logischste Mittel, um das Gehirn während der frühen Entwicklung zu beeinflussen.

Wenn derart starke Kräfte am Werk sind, wird sich das Klavier selbst rasch weiterentwickeln.
Zunächst wird das elektronische Keyboard in zunehmendem Maß in den Klaviersektor vorstoßen.
Die Unzulänglichkeiten der elektronischen Klaviere werden weiter abnehmen, bis die elektronischen von den akustischen nicht mehr zu unterscheiden sind.
Unabhängig davon, welches Instrument benutzt wird, werden die technischen Erfordernisse dieselben sein.
Bis dahin werden die akustischen Klaviere viele Merkmale der elektronischen haben: Sie werden jederzeit gestimmt sein (statt 99\% der Zeit nicht richtig gestimmt zu sein, wie sie es heute sind), man wird die Temperaturen durch Umlegen eines Schalters ändern können, und sie werden MIDI-fähig sein.
Die akustischen Klaviere werden nie völlig verschwinden, weil die Kunst, Musik mit mechanischen Geräten zu machen, so faszinierend ist.
Um auf diesem neuen Gebiet erfolgreich zu sein, müssen Klavierhersteller viel flexibler und innovativer werden.

Die Klavierstimmer werden sich ebenfalls an diese Veränderungen anpassen müssen.
Alle Klaviere werden selbststimmend sein, so daß die Einkünfte aus dem Stimmen abnehmen werden.
Klaviere, die immer hundertprozentig gestimmt sind, müssen öfter intoniert werden, und
wie Hämmer gemacht sind und intoniert werden, wird sich ändern müssen.
Es ist nicht so, daß die heutigen Klaviere nicht genauso viel intoniert werden müßten, aber wenn die Saiten perfekt gestimmt sind, wird jede Abnutzung der Hämmer zu einem begrenzenden Faktor der Klangqualität.
Klavierstimmer werden schließlich in der Lage sein, Klaviere richtig einzustellen und zu intonieren, statt sie nur zu stimmen; sie können sich auf die Qualität des Klavierklangs konzentrieren, statt nur die Dissonanzen zu beseitigen.
Da die neue Generation der vollkommeneren Klavierspieler aural anspruchsvoller sein wird, werden sie nach einem besseren Klang verlangen.
Die stark gestiegene Zahl an Klavieren und ihr ständiger Gebrauch wird eine Vielzahl neuer Klaviertechniker erfordern, um sie einzustellen und zu reparieren.
Klavierstimmer werden auch mehr daran beteiligt sein, akustischen Klavieren elektronische Fähigkeiten (MIDI, usw.) hinzuzufügen und diese zu warten.
Deshalb wird sich das Geschäft der Klavierstimmer in Richtung Wartung und Ausbau der elektronischen Klaviere erweitern.
Somit werden die meisten Menschen entweder ein Hybridklavier oder sowohl ein akustisches als auch ein elektronisches Klavier besitzen.
 

\subsection{Die Zukunft des Unterrichts}
\label{c1iv6i}

Das Internet verändert offensichtlich die Natur der Ausbildung.
Eines meiner Ziele beim Schreiben dieses Buchs im WWW ist, Möglichkeiten für eine Steigerung der Kosteneffizienz der Ausbildung zu erforschen.
Wenn ich auf meine erste Ausbildung und meine Tage auf dem College zurückblicke, wundere ich mich über die Effizienz des Ausbildungsprozesses, den ich durchlaufen habe.
Die Aussicht auf eine viel größere Effizienz durch das Internet ist jedoch im Vergleich dazu schwindelerregend.
Meine bisherige Erfahrung war sehr lehrreich.
Hier sind einige der Vorteile internetbasierter Ausbildung:

\begin{enumerate}[label={\roman*.}] 
 \item Kein Warten auf Schulbusse oder Laufen von Klassenzimmer zu Klassenzimmer mehr; keine Kosten mehr für Schulgebäude und zugehörige Einrichtungen.
 \item Keine teuren Lehrbücher.
Alle Bücher sind auf dem neuesten Stand, im Gegensatz zu vielen Lehrbüchern, die in Universitäten benutzt werden, die über 10 Jahre alt sind.
Querverweise, Inhaltsverzeichnisse, Stichwortsuche, usw. können elektronisch vorgenommen werden.
Jedes Buch ist überall verfügbar, solange man einen Computer und eine Internetverbindung hat.
 \item Viele Menschen können gemeinsam an einem Buch arbeiten, und die Aufgabe, es in andere Sprachen zu übersetzen, wird sehr effizient, insbesondere wenn die Übersetzer von einer guten Übersetzungssoftware unterstützt werden.
 \item Fragen und Vorschläge können per E-Mail gesandt werden, und der Lehrer hat reichlich Zeit, sich eine detaillierte Antwort zu überlegen.
Dieser Austausch kann an jeden versandt werden, der sich dafür interessiert, und für den späteren Gebrauch gespeichert werden.
 \item Der Lehrberuf wird sich drastisch verändern.
Auf der einen Seite wird es eine regere direkte Kommunikation per E-Mail, Videokonferenzen und Datenaustausch (wie z.B. Audiodateien vom Schüler zum Lehrer) geben,
aber auf der anderen Seite wird es weniger Gruppenkontakte geben, bei denen die Gruppe der Studenten in einem Klassenzimmer zusammenkommt.
Jeder Lehrer kann mit dem \enquote{Hauptlehrbuchzentrum} zusammenarbeiten, um Verbesserungen vorzuschlagen, die in das System eingebunden werden können.
Und die Schüler können auf viele verschiedene Lehrer zurückgreifen, sogar für das gleiche Thema.
 \item Ein solches System würde bedeuten, daß ein Experte auf dem Gebiet nicht durch das Schreiben des besten Lehrbuchs der Welt reich werden kann.
Das ist aber so wie es sein sollte - Ausbildung muß für jeden zu den niedrigsten Kosten verfügbar sein.
Wenn die Ausbildungskosten sinken, müssen deshalb Institutionen, die auf die alte Art Geld verdienten, etwas ändern und sich an die neue Lage anpassen.
Würde das nicht die Experten davon abhalten, Lehrbücher zu schreiben?
Ja, aber man braucht nur einen solchen \enquote{Freiwilligen} für die ganze Welt; außerdem hat das Internet bereits genügend solcher freien Systeme wie z.B. Linux, Browser, usw. hervorgebracht, so daß dieser Trend nicht nur unumkehrbar sondern auch bewährt ist.
Mit anderen Worten: Der Wunsch, der Gesellschaft einen Dienst zu erweisen, wird zu einem großen Faktor beim Beitragen zur Ausbildung.
Für Projekte, die der Gesellschaft einen erheblichen Nutzen bringen, werden sich sicherlich Geldgeber finden (Regierungen, Philanthropen, Sponsoren).
 \item Dieses neue Paradigma, etwas zur Gesellschaft beizutragen, kann sogar noch tiefgreifendere Veränderungen für die Gesellschaft bringen.
Eine Art, das heutige Geschäftsleben zu sehen, ist die der Wegelagerei.
Man verlangt soviel wie möglich, ungeachtet wie viel oder wie wenig Gutes das Produkt dem Käufer bringt.
In einem akkuraten Rechnungsparadigma sollte der Käufer immer den Gegenwert seines Geldes bekommen.
Das ist die einzige Situation, in der die Geschäftswelt auf lange Sicht überleben kann.
Das funktioniert in beide Richtungen; gut funktionierende Geschäfte sollten nicht einfach nur wegen übermäßigen Wettbewerbs Bankrott gehen.
In einer offenen Gesellschaft, in der alle relevanten Informationen sofort verfügbar sind, können wir ein Rechnungswesen haben, daß die Preise dem Service angemessen gestalten kann.
Die Philosophie ist hier, daß eine Gesellschaft, die aus Mitgliedern besteht, die sich verpflichtet fühlen, sich gegenseitig zu helfen, besser funktionieren wird als eine, die aus Räubern besteht, die sich gegenseitig bestehlen.
Insbesondere sollte in der Zukunft alle Grundausbildung im Prinzip kostenlos sein.
Das bedeutet nicht, daß die Lehrer ihre Arbeit verlieren, weil Lehrer die Lernrate in hohem Maß steigern können und entsprechend bezahlt werden sollten.
\end{enumerate}

Anhand der obigen Überlegungen ist klar, daß ein freier Informationsaustausch das Feld der Ausbildung (so wie jedes andere auch) verwandeln wird.
Dieses Buch ist einer der Versuche, den vollen Vorteil aus diesen neuen Möglichkeiten zu ziehen.





