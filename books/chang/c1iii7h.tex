% File: c1iii7h

\subsection{Probleme mit Hanons Übungen}\hypertarget{c1iii7h}{}

Ungefähr seit 1900 wurden die Übungen von Charles Louis Hanon (1820-1900) von zahlreichen Klavierspielern in der Hoffnung benutzt, ihre Technik zu verbessern.
Es gibt nun zwei Lehrmeinungen: einmal die, daß Hanons Übungen hilfreich sind und die, daß sie es nicht sind.
Viele Lehrer empfehlen Hanon, während andere meinen, daß die Übungen kontraproduktiv sind.
Es gibt einen \enquote{Grund}, den viele Menschen dafür angeben, daß sie Hanon benutzen: um die Hände vom einen auf den anderen Tag in guter Verfassung zum Spielen zu halten.
Dieser Grund wird am meisten von Personen zitiert, die ihre Finger mit abgeschaltetem Gehirn aufwärmen möchten.
Ich habe den Verdacht, daß diese Angewohnheit daraus resultiert, daß die Person Hanon in der frühen Klavierkarriere gelernt hat, und daß dieselbe Person Hanon nicht benutzen würde, wenn sie nicht so daran gewöhnt wäre.

Ich habe Hanons Übungen während meiner Jugend ausgiebig benutzt, gehöre aber nun fest zur Anti-Hanon-Schule.
Im folgenden liste ich einige Gründe dafür auf.
Czerny, Cramer-Bülow und verwandte Übungsstücke teilen viele dieser Nachteile.
\textbf{Hanon ist möglicherweise das beste Beispiel dafür, wie intuitive Methoden ganze Scharen von Klavierspielern  dazu verleiten können, Methoden zu benutzen, die im Grunde nutzlos oder sogar schädlich sind.}


\begin{enumerate}[label={\roman*.}]  

\item \hypertarget{c1iii7h1}{}\textbf{Hanon stellt in seiner Einführung einige überraschende Behauptungen, ohne eine rationale Erklärung oder einen experimentellen Nachweis, auf.
Das wird in seinem Titel deutlich: \enquote{Der Klaviervirtuose, in 60 Übungen}.}
Bei sorgfältigem Lesen seiner Einführung stellt man fest, daß er einfach fühlte, daß diese nützliche Übungen seien und er sie deshalb niedergeschrieben hat.
Es ist ein weiteres sehr gutes Beispiel für den \enquote{intuitiven Ansatz}.
\textbf{Die meisten fortgeschrittenen Lehrer, die diese Einführung lesen, würden zu dem Schluß kommen, daß dieser Ansatz für das Aneignen der Technik amateurhaft ist und nicht funktionieren wird.}
Hanon unterstellt, daß die Fähigkeit, diese Übungen zu spielen, sicherstellt, daß man alles spielen kann - das ist nicht nur völlig falsch, sondern enthüllt auch einen überraschenden Mangel an Verständnis dafür, was Technik ist.
\textbf{Technik kann nur durch das Lernen von vielen Kompositionen vieler Komponisten erworben werden.}

Es steht außer Frage, daß es viele vollendete Pianisten gibt, die die Hanon-Übungen benutzen.
\textbf{Die fortgeschrittenen Pianisten stimmen jedoch alle darin überein, daß Hanon nicht für das Erwerben der Technik ist}, aber dafür nützlich sein könnte, sich aufzuwärmen oder die Hände in guter Verfassung zum Spielen zu halten.
Ich glaube, es gibt viele bessere Stücke zum Aufwärmen als Hanon, wie z.B. Etüden, zahlreiche Kompositionen von Bach und andere leichte Stücke.
Die Fertigkeiten, die für das Spielen jedes bedeutenden Musikstücks notwendig sind, sind unglaublich vielfältig - fast unendlich in der Zahl.
\textbf{Zu denken, daß Technik auf 60 Übungen reduziert werden könnte, offenbart die Naivität Hanons, und jeder Schüler, der das glaubt, wurde in die Irre geführt.}



\item \hypertarget{c1iii7h2}{}\textbf{Alle 60 sind fast nur beidhändige Übungen}, bei denen die beiden Hände die gleichen Noten um eine Oktave versetzt spielen, zuzüglich ein paar Übungen mit Gegenbewegungen, bei denen die Hände in entgegengesetzte Richtungen bewegt werden.
\textbf{Diese gekoppelte HT-Bewegung ist eine der größten Einschränkungen dieser Übungen, weil die bessere Hand keine fortgeschritteneren Fertigkeiten üben kann als die schwächere Hand.}
Bei langsamer Geschwindigkeit wird keine der Hände stark trainiert.
Bei maximaler Geschwindigkeit wird die langsamere Hand gestreßt, während die bessere entspannt spielt.
\textbf{Weil Technik hauptsächlich dann erworben wird, wenn man entspannt spielt, entwickelt die schwächere Hand schlechte Angewohnheiten, und die stärkere Hand wird stärker.}
Der beste Weg, die schwächere Hand zu stärken, ist, nur mit dieser Hand zu üben, \textit{nicht} HT zu spielen.
Tatsächlich ist die beste Art mit Hanon zu lernen, die Hände, wie hier in diesem Buch empfohlen, zu trennen, aber Hanon hat anscheinend noch nicht einmal daran gedacht.
Zu glauben, daß durch das HT-Spielen die schwächere Hand die stärkere Hand einholt, offenbart eine für jemanden mit soviel Lehrerfahrung erstaunliche Unwissenheit.
Das ist ein Teil dessen, was ich oben mit \enquote{amateurhaft} meinte; weitere Beispiele folgen.

Die Hände zu koppeln hilft dabei zu lernen, wie man die Hände koordiniert, tut aber nichts dafür, die unabhängige Kontrolle jeder Hand zu lehren.
Praktisch in der ganzen Musik spielen die beiden Hände unterschiedliche Teile.
Hanon gibt uns keine Gelegenheit, das zu üben.
Bachs Inventionen sind viel besser und stärken (wenn Sie HS üben) wirklich die schwächere Hand.
\textbf{Der Punkt ist hier, daß Hanon sehr begrenzt ist; er lehrt nur einen kleinen Bruchteil der gesamten Technik, die man benötigt.}



\item \hypertarget{c1iii7h3}{}\textbf{Es gibt keine Vorkehrung für das Ausruhen einer ermüdeten Hand.
Das führt im allgemeinen zu Streß und Verletzungen.}
Ein eifriger Schüler, der die Schmerzen und Ermüdung bei dem Bemühen, den Anweisungen Hanons zu folgen, bekämpft, wird fast mit Sicherheit Streß aufbauen, schlechte Angewohnheiten annehmen und Verletzungen riskieren.
\textbf{Das Konzept der Entspannung wird noch nicht einmal erwähnt.}
Klavierspielen ist eine Kunst zur Erzeugung von Schönheit und Eleganz; es ist keine Demonstration von Machos, wieviel Bestrafung ihre Hände, Ohren und Gehirne aushalten können.

\textbf{Hingebungsvolle Schüler benutzen Hanon am Ende oft als eine Möglichkeit, intensive Übungen auszuführen, in dem falschen Glauben, daß Klavierspielen wie Gewichtheben ist, und daß \enquote{ohne Schmerzen kein Erfolg} auch für das Klavier gilt.}
Solche Übungen sollen angeblich bis zur Grenze der menschlichen Belastbarkeit ausgeführt werden können, sogar bis einige Schmerzen spürbar sind.
Das offenbart einen Mangel an richtiger Ausbildung darüber, was für den Erwerb der Technik notwendig ist.
Die verschwendeten Ressourcen aufgrund solcher falschen Vorstellungen können den Unterschied zwischen Erfolg und Versagen für eine große Zahl von Schülern ausmachen, sogar wenn sie keine Verletzungen erleiden.
Natürlich sind viele Schüler erfolgreich, die routinemäßig Hanon üben; in diesem Fall arbeiten sie so hart, daß sie \textit{trotz} Hanon Erfolg haben.

 

\item \hypertarget{c1iii7h4}{}\textbf{Die Hanon-Übungen sind frei von Musik, so daß Schüler am Ende möglicherweise wie Roboter üben.}
Es erfordert keine musikalische Genialität, um eine Serie von Übungen der Hanon-Art zusammenzustellen.
Die Freude am Klavier kommt von der direkten Auseinandersetzung mit den größten Genies, die jemals gelebt haben, wenn Sie deren Kompositionen spielen.
Für zu viele Jahre hat Hanon die falsche Botschaft verbreitet, daß Technik und Musik getrennt gelernt werden können.
Bach ist in dieser Hinsicht überlegen; seine Musik trainiert sowohl die Hände als auch das Gehirn.
\textbf{Hanon hat wahrscheinlich das meiste seines Material aus Bachs berühmter Toccata und Fuge entnommen und so geändert, daß jede Einheit selbstzyklisch ist.
Der Rest wurde wahrscheinlich ebenfalls aus Bachs Werken entnommen, besonders aus den Inventionen und Sinfonien.}

Einer der größten Schäden, die Hanon anrichtet, ist, daß er soviel Zeit verschwendet.
Der Schüler hat am Ende nicht genügend Zeit, um sein Repertoire zu entwickeln oder wirkliche Technik zu erwerben.
\textbf{Hanon kann schädlich für die Technik und das Aufführen sein!}



\item \hypertarget{c1iii7h5}{}\textbf{Viele Klavierspieler benutzen Hanon routinemäßig als Übung zum Aufwärmen.
Das konditioniert die Hände so, daß Sie unfähig werden, sich einfach hinzusetzen und \enquote{\hyperlink{c1iii6g}{kalt}} zu spielen; etwas, das jeder vollendete Pianist innerhalb vernünftiger Grenzen können sollte.}
Da die Hände für mindestens 10 bis 20 Minuten kalt sind, raubt das \enquote{Aufwärmen} dem Schüler dieses kostbare kleine Fenster der Gelegenheit, das kalte Spielen zu üben.
Diese Angewohnheit, Hanon zum Aufwärmen zu benutzen, ist heimtückischer als viele erkennen.
Diejenigen, die Hanon zum Aufwärmen benutzen, können zu dem Glauben verleitet werden, daß Hanon ihre Finger zum Fliegen bringt, während die Finger in Wahrheit nach jeder guten Übungseinheit fliegen, egal ob mit oder ohne Hanon.
Es ist tückisch, weil die hauptsächliche Konsequenz aus diesem Mißverständnis ist, daß die Person weniger in der Lage ist vorzuspielen, unabhängig davon, ob die Finger aufgewärmt sind oder nicht.
Es ist wirklich unglücklich, daß die Hanon-Art zu denken einen großen Bestand an Schülern hervorgebracht hat, die glauben, daß man ein Mozart sein muß, um in der Lage zu sein, sich einfach hinzusetzen und zu spielen, und daß man von gewöhnlichen Sterblichen nicht erwarten kann, solche zauberhaften Meisterleistungen zu vollbringen.
Wenn Sie in der Lage sein möchten, \enquote{auf Kommando zu spielen}, fangen Sie am besten damit an, daß Sie aufhören Hanon zu üben.

 

\item \hypertarget{c1iii7h6}{}Es gibt kaum einen Zweifel daran, daß ein gewisses Maß an Technik erforderlich ist, um diese Übungen zu spielen, besonders ungefähr die letzten 10.\textbf{Das Problem ist, daß Hanon keine Anleitung dafür liefert, wie man diese Technik erwirbt.}
Es ist exakt analog dazu, einem armen Menschen zu sagen, er solle etwas Geld verdienen, wenn er reich werden möchte.
Es hilft ihm nicht.
Wenn ein Schüler die mit Hanon verbrachte Zeit dazu benutzt hätte, eine Beethoven-Sonate zu üben, hätte er viel mehr Technik erworben.
\textbf{Wer würde nicht lieber Mozart, Bach, Chopin usw. statt Hanon-Übungen mit besseren Ergebnissen spielen und am Ende ein aufführbares Repertoire haben?}

Sogar wenn Sie alle Hanon-Übungen gut spielen können, wird Ihnen Hanon nicht helfen, wenn Sie bei einer schwierigen Passage einer anderen Komposition festhängen.
Hanon stellt keine Diagnosen zur Verfügung, die Ihnen sagen, warum Sie eine bestimmte Passage nicht spielen können.
Die \hyperlink{c1iii7b}{Übungen für parallele Sets} bieten Ihnen sowohl die Diagnosewerkzeuge als auch die Lösungen für praktisch jede Situation einschließlich der Verzierungen usw., die Hanon noch nicht einmal berücksichtigt.



\item \hypertarget{c1iii7h7}{}\textbf{Die wenigen Ratschläge, die er erteilt, erweisen sich \textit{alle} als falsch!}
Schauen wir sie uns an:
(a) Er empfiehlt, \enquote{die Finger weit anzuheben}, was für schnelles Spielen offensichtlich nicht in Frage kommt, da es die größte Quelle von Streß ist.
Ich habe nie einen berühmten Pianisten gesehen, der im Konzert die Finger weit anhebt, um einen schnellen Lauf zu spielen; tatsächlich habe ich dies nie \textit{jemanden} tun sehen!
Dieser Rat von Hanon hat eine enorme Zahl von Schülern zu dem Glauben verleitet, daß das Klavier gespielt werden sollte, indem man den Finger anhebt und auf die Taste herunterknallt.
Es ist eine der unmusikalischsten und technisch unkorrektesten Arten zu spielen.
Es ist wahr, daß die Streckmuskeln oft vernachlässigt werden, aber es gibt \hyperlink{c1iii7finger}{Übungen}, um dieses Problem direkt zu behandeln.

(b) Er empfiehlt fortlaufendes Üben beider Hände, als ob Klaviertechnik eine Art Training für das Gewichtheben wäre.
Schüler dürfen niemals mit ermüdeten Händen üben.
Deshalb funktioniert die HS-Methode dieses Buchs so gut - sie erlaubt Ihnen, 100\% der Zeit ohne Ermüdung hart zu trainieren, weil eine Hand ruht, während die andere arbeitet.
Ausdauer wird nicht durch Üben mit Ermüdung und Streß erzielt, sondern durch die richtige Konditionierung.
Außerdem brauchen die meisten von uns geistige Ausdauer und keine Ausdauer der Finger.
Und seine Empfehlung ignoriert völlig die \hyperlink{c1ii14}{Entspannung}.

(c) Er empfiehlt, unabhängig von Ihrer Fertigkeitsstufe, Ihr ganzes Leben lang jeden Tag zu spielen.
Aber wenn man erst einmal eine Fertigkeit erworben hat, muß man sie nicht immer und immer wieder neu erwerben; man muß nur an der Technik arbeiten, die man noch nicht hat.
Somit gibt es, wenn man alle 60 Stücke gut spielen kann, keine Notwendigkeit, sie weiter zu spielen - was wird man dabei gewinnen?

(d) Er kennt offenbar nur den Daumenuntersatz, während der \hyperlink{c1iii5a}{Daumenübersatz} wichtiger ist.

(e) Bei den meisten Übungen empfiehlt er ein festes Handgelenk, was nur teilweise korrekt ist.
Seine Empfehlung offenbart einen Mangel an Verständnis, was \enquote{\hyperlink{ruhig}{ruhige Hände}} sind.
(f) Es gibt keine Möglichkeit, einen Großteil der wichtigen Handbewegungen zu üben, obwohl es ein paar Handgelenksübungen für Wiederholungen gibt.



\item \hypertarget{c1iii7h8}{}\textbf{Die Hanon-Übungen erlauben es nicht, mit den Geschwindigkeiten zu üben, die mit den oben beschriebenen \hyperlink{c1iii7b}{Übungen für parallele Sets} möglich sind.}
Ohne solche Geschwindigkeiten zu benutzen, können bestimmte hohe Geschwindigkeiten nicht geübt werden, können Sie nicht die \enquote{Über-Technik} trainieren (d.h. mehr Technik als für das Spielen dieser Passage notwendig ist - eine notwendige Sicherheitsreserve für \hyperlink{c1iii14}{Auftritte}), und die Hanon-Übungen bieten keine Möglichkeit, ein bestimmtes technisches Problem zu lösen.



\item \hypertarget{c1iii7h9}{}Die ganze Übung ist ein Üben von Verschwendung.
Alle Ausgaben, die ich gesehen habe, drucken die ganzen Läufe, während alles was man braucht, höchstens die ersten 2 aufsteigenden Takte, die ersten 2 absteigenden Takte und der Schlußtakt sind.
Obwohl die Zahl der Bäume, die gefällt wurden, um Hanon zu drucken, bei näherem Hinsehen vernachlässigbar ist, offenbart das die Mentalität hinter diesen Übungen, einfach das intuitiv \enquote{offensichtliche} zu wiederholen, ohne wirklich zu verstehen was man tut oder die wichtigen Elemente jeder Übung aufzuzeigen.
\textbf{\enquote{Wiederholung ist wichtiger als die zugrunde liegenden technischen Konzepte} - das ist wahrscheinlich die schlechteste Einstellung, die die Schüler in der Geschichte des Klaviers am meisten behindert hat.}
Eine Person, die 2 Stunden täglich übt und dabei wie empfohlen eine Stunde Hanon spielt, verschwendet die Hälfte ihrer Klavierzeit!
Eine Person, die 8 Stunden zum Üben zur Verfügung hat, \textit{braucht} keinen Hanon.



\item \hypertarget{c1iii7h10}{}Ich habe festgestellt, daß sich Lehrer ebenfalls in Abhängigkeit davon, ob sie Hanon lehren oder nicht, in zwei Schulen aufteilen.
Diejenigen, die nicht Hanon lehren, neigen dazu sachkundiger zu sein, weil sie die wahren Methoden für den Erwerb der Technik kennen und damit beschäftigt sind, diese zu lehren - für Hanon bleibt dann keine Zeit.
Wenn sie nach einem Klavierlehrer Ausschau halten, erhöhen Sie deshalb die Chancen einen überlegenen zu finden, wenn Sie ihn nur aus denjenigen auswählen, die nicht Hanon lehren.

 
 \end{enumerate} 


