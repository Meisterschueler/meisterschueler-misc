% File: c1iii7d

\subsubsection{Tonleitern, Arpeggios, Unabhängigkeit der Finger und Anheben der Finger}
\label{c1iii7d}

\textbf{\hyperref[c1iii5a]{Tonleitern}\index{Tonleitern} und \hyperref[Arpeggios]{Arpeggios}\index{Arpeggios} müssen gewissenhaft geübt werden.}
Sie gehören wegen der zahlreichen notwendigen Techniken, die man unter ihrer Verwendung am schnellsten lernen kann (wie z.B. \hyperref[c1iii5a]{Daumenübersatz}\index{Daumenübersatz}, \hyperref[c1iii4b]{flache Fingerhaltungen}\index{flache Fingerhaltungen}, Tasten fühlen, Geschwindigkeit, parallele Sets, Glissandobewegung, Klang und Farbe, wie man Umkehrungen spielt, geschmeidiges Handgelenk usw.), nicht zur Klasse der stupide wiederholenden Übungen.
Tonleitern und Arpeggios müssen HS geübt werden; sie stets HT zu üben stellt sie in die gleiche Kategorie wie \hyperref[c1iii7h]{Hanon}\index{Hanon}.
Es gibt zwei Ausnahmen von dieser \enquote{Kein-HT-Regel}:

\begin{enumerate} 
 \item Wenn Sie sie (z.B. vor Konzerten usw.) zum Aufwärmen benutzen.
 \item Wenn Sie üben, um sicherzustellen, daß die Hände exakt synchronisiert werden können.
\end{enumerate}
Zu lernen, sie gut zu spielen, ist sehr schwierig, und Sie werden dafür sicherlich parallele Sets benötigen; weitere Einzelheiten finden Sie in den Abschnitten \hyperref[c1iii4b]{\autoref{c1iii4b}} und \hyperref[c1iii5a]{III.5}.


\label{c1iii7finger}

\textbf{Die Übungen zum \hyperref[c1iii7anheben]{Anheben}\index{Anheben} (s.u.) und für die Unabhängigkeit der Finger werden ausgeführt, indem man zunächst alle fünf Finger herunterdrückt, z.B. von C bis G mit der RH.}
Spielen Sie dann mit jedem Finger drei- bis fünfmal, z.B. CCCCDDDDEEEEFFFFGGGG.
Während ein Finger spielt, müssen die anderen vier unten gehalten werden.
Drücken Sie nicht fest herunter, da dies eine Form von Streß ist und sehr schnell Ermüdung verursachen wird.
Auch möchten Sie die langsamen Muskeln nicht mehr als notwendig wachsen lassen.
Alle heruntergedrückten Tasten müssen völlig unten sein, aber die Finger ruhen nur mit gerade soviel abwärts gerichteter Kraft auf ihnen, wie notwendig ist, um die Tasten unten zu halten.
Das durch die Schwerkraft verursachte Gewicht Ihrer Hand sollte genügen.
Anfänger werden diese Übung zu Beginn schwierig finden, weil die nicht spielenden Finger dazu neigen, aus ihrer optimalen Position einzuknicken oder sich ungewollt anzuheben, besonders wenn sie anfangen müde zu werden.
Wenn sie dazu neigen einzuknicken, versuchen Sie es ein paarmal, und wechseln Sie die Hände, oder hören Sie auf.
Versuchen Sie es nach einer Pause erneut.
Eine Variation dieser Übung ist, die Noten auf eine Oktave auszudehnen.
Diese Art der Übungen wurde bereits in der Zeit von F. Liszt benutzt (Moscheles).
Sie sollten sowohl mit der \hyperref[c1ii2]{gebogenen}\index{gebogenen} als auch mit allen \hyperref[c1iii4b]{flachen Fingerhaltungen}\index{flachen Fingerhaltungen} ausgeführt werden.

Versuchen Sie bei der \textbf{Übung für die Unabhängigkeit der Finger}, die Geschwindigkeit zu steigern.
Beachten Sie die Ähnlichkeit zu \hyperref[c1iii7b1]{Übung \#1}\index{Übung \#1} für parallele Sets in Abschnitt (b); für die allgemeine technische Entwicklung ist Übung \#1 dieser jedoch überlegen.
Das Hauptziel von Übung \#1 war die Geschwindigkeit; die Betonung liegt hier auf etwas anderem -- auf der Unabhängigkeit der Finger.
Einige Klavierlehrer empfehlen, diese Übung einmal während jeder Übungssitzung durchzuführen, wenn Sie sie zufriedenstellend spielen können.
Bis Sie sie zufriedenstellend spielen können, könnten Sie sie während jeder Übungssitzung mehrmals ausführen.
Sie während einer Sitzung viele Male zu üben und in den folgenden Sitzungen wegzulassen funktioniert nicht so gut.

Alle Übungsmethoden und Übungen, die in diesem Buch besprochen werden, behandeln hauptsächlich die Muskeln, die benutzt werden, um die Taste nach unten zu drücken (Flexoren = Beugemuskeln).
Es ist für diese Muskeln möglich, weitaus stärker entwickelt zu werden als jene, die benutzt werden, um die Finger anzuheben (Extensoren = Streckmuskeln).
Das gilt besonders, wenn man stets laut übt und nie die Kunst schnell zu spielen entwickelt und somit Probleme bei der Kontrolle verursacht, ganz besonders, wenn es dazu führt, daß man sehr viele langsame Muskeln entwickelt.
Wenn man älter wird, können die Beugemuskeln die Streckmuskeln schließlich an Kraft übertreffen.
Deshalb ist es eine gute Idee, die relevanten Streckmuskeln durch Übungen zum Anheben zu trainieren.
Die \hyperref[c1iii4b]{flachen Fingerhaltungen}\index{flachen Fingerhaltungen} sind beim Trainieren der Streckmuskeln für das Anheben der Finger und das gleichzeitige Entspannen der Streckmuskeln der Fingerspitzen wertvoll.
Diese beiden sind verschiedene Streckmuskeln.


\label{c1iii7anheben}

Wiederholen Sie zum \textbf{Üben des Anhebens} die \hyperref[c1iii7finger]{obige Übung}\index{obige Übung}, aber heben Sie jeden Finger schnell so hoch Sie können und senken ihn sofort wieder.
Die Bewegung sollte so schnell wie möglich sein, aber langsam genug, daß Sie die völlige Kontrolle haben; das ist kein Geschwindigkeitswettbewerb, Sie müssen nur vermeiden, daß die langsamen Muskeln wachsen.
Behalten Sie wieder alle anderen Finger mit minimalem Druck unten.
Wie üblich ist es wichtig, die Anspannung in den Fingern zu reduzieren, die nicht angehoben werden.

Jeder hat Probleme damit, den 4. Finger anzuheben.
Viele glauben fälschlicherweise, man müßte den 4. Finger so hoch wie alle anderen Finger anheben können, und sie wenden deshalb ungeheuer viel Energie bei dem Versuch auf, das zu erreichen.
Solch ein Aufwand ist erwiesenermaßen vergeblich und kontraproduktiv.
Das kommt daher, daß die Anatomie des 4. Fingers es nicht erlaubt, ihn über einen bestimmten Punkt hinaus anzuheben.
Der 4. Finger darf nur nicht versehentlich eine Taste niederdrücken; dies erfordert ein viel geringeres Anheben.
Deshalb können Sie zu jeder Zeit mit dem 4. Finger knapp über den Tasten spielen oder ihn sogar darauf ruhen lassen.
Schwierige Passagen mit übertriebenem Aufwand zum höheren Heben dieses Fingers zu üben, kann Streß im 3. und 5. Finger verursachen.
Es ist produktiver, zu lernen mit weniger Streß zu spielen, solange der 4. Finger nicht in irgendeiner Weise stört.
\label{c1iii7finger4}
Die \textbf{Übung für das unabhängige Anheben des 4. Fingers} wird folgendermaßen ausgeführt:

\begin{itemize} 
 \item Drücken Sie alle Finger auf CDEFG nach unten wie zuvor.
 \item Spielen Sie 1,4,1,4,1,4 usw., mit Betonung auf der 1, und heben Sie den Finger 4 so schnell und so hoch Sie können.
 Finger 1 sollte so nah an der Taste bleiben wie möglich.
 \item Wiederholen Sie es mit 2,4,2,4,2,4 usw.
 \item Nun mit Finger 3 und 4.
 \item Und zum Schluß mit 5 und 4.
 \end{itemize}
Sie können diese Übung auch mit der 4 auf einer schwarzen Taste durchführen.

Sowohl die Übung für die Unabhängigkeit der Finger als auch die zum Anheben können ohne ein Klavier, auf jeder glatten Oberfläche, durchgeführt werden.
Das ist der beste Zeitpunkt, um das Entspannen der Streckmuskeln der beiden letzten Fingerglieder (Nagelglied und mittleres Glied) der Finger 2 bis 5 zu üben; Details s. \hyperref[c1iii4b]{\autoref{c1iii4b}}.
Während der gesamten Übung sollten diese beiden Glieder bei allen Fingern völlig entspannt sein, sogar bei dem angehobenen Finger.
 

\subsubsection{(Große) Akkorde spielen, Dehnung der Handflächen}
\label{c1iii7e}

Wir behandeln zunächst das Problem, genaue Akkorde zu spielen, in denen alle Noten so simultan wie möglich gespielt werden müssen.
Dann gehen wir das Problem an, große Akkorde zu spielen.
Wenn Sie kleine Hände haben, müssen Sie sowohl die Handflächen so weit wie möglich dehnen als auch die Finger seitwärts ausstrecken.

Wir haben bereits gesehen, daß der \hyperref[c1ii10]{Freie Fall}\index{Freie Fall} (Abschnitt II.10) benutzt werden kann, um die Genauigkeit der Akkorde zu verbessern.
Wenn jedoch nach dem Benutzen des Freien Falls immer noch eine Ungleichmäßigkeit vorhanden ist, dann gibt es ein fundamentales Problem, das mit den \hyperref[c1iii7b]{Übungen für parallele Sets}\index{Übungen für parallele Sets} diagnostiziert und behandelt werden muß.
Akkorde werden ungleichmäßig, wenn die Kontrolle über einzelne Finger ungleichmäßig ist.
Welche Finger schwach sind oder langsam usw. kann mit den Übungen für parallele Sets diagnostiziert und korrigiert werden.
Lassen Sie uns ein Beispiel betrachten.
Angenommen, Sie spielen einen C.E-Akkord gegen ein G (alles mit der LH) in der dritten Oktave.
Dann werden C3.E3 und G3 mit den Fingern 5.3 und 1 gespielt.
Sie spielen eine Serie von 5.3,1,5.3,1,5.3,1 usw., wie ein Tremolo.
Lassen Sie uns weiter annehmen, daß es ein Akkord-Problem mit dem 5.3 gibt.
Diese beiden Finger landen nicht simultan und ruinieren so das Tremolo.
Versuchen Sie zur Diagnose dieses Problems das parallele Set 5,3, um zu sehen, ob Sie es spielen können.
Testen Sie nun das umgekehrte Set 3,5.
Wenn Sie ein Problem mit dem Akkord haben, ist es möglich, daß Sie eher ein Problem mit einem der beiden parallelen Sets haben als mit dem anderen, oder daß Sie Probleme mit beiden parallelen Sets haben.
Meistens ist 3,5 schwieriger als 5,3.
Arbeiten Sie an dem/n problematischen parallelen Set/s.
Wenn Sie beide parallele Sets gut spielen können, sollte der Akkord viel besser klingen.
Es besteht auch die -- geringere -- Möglichkeit, daß Ihr Problem am parallelen Set 5,1 oder 3,1 liegt.
Wenn also die Arbeit an 5,3 nichts bringt, versuchen Sie es mit diesen.

\textbf{In der Hand gibt es zwei Muskelgruppen, mit denen man die Finger und die Handfläche spreizen kann, um große Akkorde zu erreichen.
Eine Gruppe öffnet hauptsächlich die Handfläche, und die andere spreizt hauptsächlich die Finger auseinander.}
Benutzen Sie hauptsächlich die Muskelgruppe, die die Handfläche öffnet, wenn Sie die Hand strecken, um große Akkorde zu spielen.
Das Gefühl ist das gleiche wie beim Strecken der Handfläche aber mit freien Fingern; d.h. spreizen Sie die Knöchel statt der Fingerspitzen.
Die zweite Muskelgruppe spreizt einfach die Finger auseinander.
Dieses Spreizen hilft, die Handfläche zu verbreitern, beeinträchtigt aber die Bewegung der Finger, weil es dazu führt, daß die Finger an die Handfläche gefesselt werden.
Gewöhnen Sie sich an, die Muskeln der Handflächen getrennt von den Fingermuskeln zu benutzen.
Das wird sowohl den Streß als auch die Ermüdung reduzieren, wenn Sie Akkorde spielen, und mehr zur Kontrolle beitragen.
Natürlich ist es am einfachsten, beide Muskelgruppen gleichzeitig zu benutzen, aber es ist nützlich zu wissen, daß es zwei Muskelgruppen gibt, wenn man seine Übungen plant und wenn man entscheidet, wie man Akkorde spielt.

\textbf{Spreizen der Finger: Um zu testen, ob die Finger völlig gespreizt sind, öffnen Sie Ihre Handfläche bis zum Maximum und spreizen Sie die Finger für eine maximale Reichweite -- wenn der kleine Finger und der Daumen fast eine gerade Linie bilden, dann werden Sie nicht in der Lage sein, sie mehr zu spreizen.}
Wenn sie ein \enquote{V} bilden, dann sind Sie eventuell in der Lage weiter zu reichen, indem Sie Dehnungsübungen durchführen.
Eine andere Möglichkeit, diese Ausrichtung zu testen, ist, Ihre Hand so auf eine Tischplatte zu legen, daß die Finger 2 bis 4 und die Handfläche so weit wie möglich auf der Tischplatte liegen und Sie mit dem Daumen und dem kleinen Finger horizontal gegen die Tischkante drücken.
Wenn der Daumen und der kleine Finger ein Dreieck mit der Tischkante bilden, sind Sie eventuell in der Lage, sie mehr zu strecken.
Sie können eine Dehnungsübung durchführen, indem Sie die Hand zur Tischkante hin schieben, um so den Daumen und den kleinen Finger weiter auseinander zu spreizen.
Sie können viel Zeit sparen, wenn Sie eine Hand an der oberen Kante des Klaviers dehnen, während Sie mit der anderen Hand HS üben.

\textbf{Spreizen der Handfläche:}
Es ist wichtiger, aber schwieriger, die Handfläche anstatt der Finger zu dehnen.
Eine Möglichkeit ist, die Innenseiten der Handflächen so vor der Brust übereinander zu legen, daß jeweils der Daumen der einen Hand den kleinen Finger der anderen Hand berührt und die Ellbogen nach außen zeigen.
Verschränken Sie die Daumen mit den kleinen Fingern, so daß die Finger 2 bis 4 auf der Innenseite der Handflächen sind und die Finger 1 und 5 an der Rückseite der Handflächen hervorstehen.
Schieben Sie dann die Hände aufeinander zu, so daß die Daumen und die kleinen Finger sich gegenseitig zurückdrücken und somit die Handflächen dehnen.
Sehen Sie dazu ein \hyperref[http://www.pianopractice.org/palmstretch.jpg]{Foto} (extern).
Die zum Spreizen notwendige Kraft wird erzeugt, indem Sie die Hände aufeinander zu bewegen (RH nach links und die LH nach rechts).
Sie können auch die Muskeln für das Spreizen der Finger und Handflächen gleichzeitig trainieren, während Sie die Hände aufeinander zu bewegen.
Das ist keine isometrische Übung, d.h. die Spreizbewegungen sollten schnell und kurz sein.
Diese Fähigkeit, schnell zu spreizen und sofort zu entspannen ist für das Entspannen wichtig.
Regelmäßiges Dehnen in jungen Jahren kann einen beträchtlichen Unterschied bei der Reichweite ausmachen wenn man älter wird, und regelmäßiges Üben verhindert, daß die Reichweite mit zunehmendem Alter nachläßt.
Die Haut zwischen den Fingern können Sie auch dehnen, indem Sie jeweils einen Fingerzwischenraum der einen Hand gegen den gleichen Fingerzwischenraum der anderen drücken, wobei die Hände um 90 Grad zueinander gedreht sind.
Benutzen Sie für eine maximale Wirkung bei jeder Drückbewegung sowohl die Hand- als auch die Fingermuskeln, um die Handfläche zu spreizen.
Führen Sie auch das nicht wie eine isometrische Übung sondern mit schnellen Bewegungen aus.
Die meisten Menschen haben eine etwas größere linke als rechte Hand, und einige können mit 1.4 weiter reichen als mit 1.5.

Wenn Sie große Akkorde spielen, sollte der Daumen leicht einwärts gekrümmt sein, nicht völlig ausgestreckt.
Es ist kontraintuitiv, daß man mit eingezogenem Daumen weiter reichen kann; das geschieht wegen der besonderen Krümmung der Fingerspitze des Daumens.
Beim Spielen von Akkorden müssen Sie im allgemeinen die Hand bewegen, und diese Bewegung muß sehr genau sein, wenn die Akkorde richtig erklingen sollen.
Das ist die \hyperref[c1iii7f]{Sprungbewegung}\index{Sprungbewegung}, die unten besprochen wird.
Für das Spielen von Akkorden ist es notwendig, daß Sie sowohl saubere Sprungbewegungen entwickeln als auch die Angewohnheit, die Tasten zu fühlen.
Sie können nicht bloß Ihre Hand hoch über die Tasten heben, alle Ihre Finger an der richtigen Stelle positionieren, sie herunterdonnern lassen und erwarten, daß Sie alle richtigen Noten genau zum gleichen Zeitpunkt treffen.
Bei großen Pianisten erscheint es oft so, als ob sie dies tun würden, aber wie wir unten sehen werden, tun sie es nicht.
Bis Sie die Sprungbewegung perfektioniert haben und in der Lage sind, die Tasten zu erfühlen, werden deshalb irgendwelche Probleme mit dem Spielen von Akkorden (fehlende oder falsche Noten) eventuell nicht durch einen Mangel an Reichweite oder Kontrolle der Finger verursacht.
Wenn Sie Schwierigkeiten haben, die Akkorde zu treffen \textit{und} keine Sicherheit bei Ihren Sprüngen haben, ist das ein sicheres Zeichen, daß Sie die Sprünge lernen müssen, bevor Sie daran denken können, Akkorde zu treffen.
 


