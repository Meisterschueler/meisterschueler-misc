% File: c1ii1

\section{Grundlegende Verfahren des Klavierübens}
\label{c1ii1}

% zuletzt geändert 22.08.2009

Dieser Abschnitt enthält die minimalen Anleitungen, die Sie benötigen, bevor Sie mit dem Üben anfangen.
 

\subsection{Der Übungsablauf} 

Viele Schüler benutzen folgenden Übungsablauf:

\begin{enumerate}[label={\arabic*.}] 
\item Zunächst Tonleitern oder Fingerübungen spielen, bis die Finger aufgewärmt sind.
Zum Verbessern der Technik wird dies, insbesondere unter Verwendung von Übungen wie der Hanon-Serie, 30 Minuten durchgeführt - wenn man Zeit hat auch länger.
\item Dann nimmt man ein neues Musikstück und liest langsam eine oder zwei Seiten, während man das Stück sorgfältig mit beiden Händen vom Anfang ab spielt.
Dieses langsame Spielen wird so lange wiederholt, bis man das Stück ziemlich gut vorspielen kann, und nun wird die Geschwindigkeit schrittweise gesteigert, bis die endgültige Geschwindigkeit erreicht ist.
Für dieses schrittweise Steigern könnte ein Metronom benutzt werden.
\item Am Ende einer zweistündigen Übungseinheit fliegen die Finger, sodass die Schüler so schnell spielen können wie sie möchten und die Erfahrung genießen können, bevor sie mit dem Üben aufhören.
Nach all dem sind sie des Übens müde, entspannen sich und spielen mit Leib und Seele mit voller Geschwindigkeit.
Dies ist der Moment, in dem sie Spaß an der Musik haben!\item Wenn das Stück zufriedenstellend gespielt werden kann, wird es auswendig gelernt und dann geübt, \enquote{bis die Musik in den Händen ist}.

\item Am Tag des Konzerts oder des Unterrichts üben sie das Stück in der richtigen Geschwindigkeit (oder schneller!) so oft wie möglich, um sicherzustellen, dass es in bestem Zustand ist.
Das ist die letzte Gelegenheit, und offensichtlich gilt: je mehr Übung desto besser.
 \end{enumerate}
\textbf{Jeder Schritt dieses Ablaufs ist falsch!}
Dieser Ablauf wird mit ziemlicher Sicherheit dazu führen, dass die Schüler nicht über die Mittelstufe hinauskommen, auch wenn sie täglich mehrere Stunden üben.
Zum Beispiel gibt dieser Ablauf den Schülern keinen Hinweis, was sie tun müssen, wenn sie auf eine nicht spielbare Passage treffen, außer dass sie diese ständig - manchmal ein Leben lang - wiederholen sollen, ohne eine klare Vorstellung darüber, wann und wie die dafür notwendige Technik erworben wird.
Diese Methode überlässt die Aufgabe, das Klavierspielen zu lernen, völlig dem Schüler.
Zudem wird die Musik während des Vorspielens flach klingen und unerwartete Fehler werden fast unausweichlich sein.
Sie werden das alles verstehen, sobald sie die weiter unten beschriebenen, effizienteren Methoden kennenlernen.

\textbf{Mangel an Fortschritt ist der Hauptgrund, warum so viele Schüler mit dem Klavier aufhören}.
Schüler, insbesondere jüngere, sind clever; warum wie ein Sklave schuften und nichts dabei lernen?
Belohnen Sie die Schüler, und sie werden mehr Hingabe erzielen als jeder Lehrer erwarten kann.
Man kann Arzt sein, Wissenschaftler, Rechtsanwalt, Athlet oder was auch immer man möchte und trotzdem ein guter Pianist werden, weil es Methoden gibt, mit denen Sie die Technik rasch erwerben können, wie Sie gleich sehen werden.

\textbf{Beachten Sie, dass der obige Übungsablauf eine \enquote{intuitive} (oder \enquote{instinktive}) Methode ist.}
Wenn jemand, der durchschnittlich intelligent ist, mit nichts außer einem Klavier auf einer einsamen Insel ausgesetzt worden wäre und sich entscheiden würde zu üben, würde diese Person wahrscheinlich eine Übungsmethode wie die obige entwerfen.
Das heißt, ein Lehrer, der diese Art von Übungsmethoden lehrt, lehrt im Grunde nichts - die Methode ist intuitiv.
\textbf{Als ich zum ersten Mal damit anfing, die \enquote{richtigen Lernverfahren} zusammenzutragen, war ich am meisten davon überrascht, wie viele davon kontraintuitiv waren.}
Ich werde später erklären, warum sie so kontraintuitiv sind, aber dies bietet die beste Erklärung, warum so viele Lehrer den intuitiven Ansatz verwenden.
Diese Lehrer haben die richtigen Methoden niemals gelernt und wurden deshalb zu den intuitiven Methoden hingezogen.
Die Schwierigkeit mit kontraintuitiven Methoden ist, dass sie schwerer anzunehmen sind als intuitive; Ihr Gehirn sagt Ihnen ständig, sie seien falsch und Sie sollten zu den intuitiven zurückkehren.
Diese Botschaft des Gehirns kann vor der Unterrichtsstunde oder dem Konzert unwiderstehlich werden - versuchen Sie, (nicht informierten) Schülern zu sagen, sie sollen keinen Spaß damit haben, ihre fertigen Stücke zu spielen, bevor sie mit dem Üben aufhören, oder am Tag eines Konzerts nicht zu viel zu üben!
Es geht nicht nur um die Schüler oder Lehrer.
Es sind auch Eltern oder Freunde mit guten Absichten, die die Übungsgewohnheiten junger Schüler beeinflussen.
\textbf{Nicht informierte Eltern werden ihre Kinder stets dazu zwingen, die intuitiven Methoden zu benutzen.}
Dies ist ein Grund, warum gute Lehrer immer die Eltern bitten, ihre Kinder zu den Unterrichtsstunden zu begleiten.
Wenn die Eltern nicht informiert sind, gibt es praktisch eine Garantie dafür, dass sie die Schüler dazu zwingen, Methoden zu benutzen, die im Widerspruch zu den Anweisungen des Lehrers stehen.

Schüler, die von Anfang an mit den richtigen Methoden begannen, sind \textit{scheinbar} die Glücklichen.
Sie müssen jedoch später aufpassen, falls man ihnen nicht beigebracht hat, was die falschen Methoden sind.
Wenn sie ihren Lehrer verlassen, dann können sie plötzlich in die intuitiven Methoden verfallen und haben keine Ahnung, warum ihnen alles entgleitet.
Es ist wie ein Bär, der noch nie eine Bärenfalle gesehen hat - er wird jedes Mal gefangen.
Diese \enquote{Glücklichen} können oftmals auch nicht unterrichten, weil sie vielleicht nicht erkennen, dass viele intuitive Methoden zur Katastrophe führen können.
Die scheinbar unglücklichen Schüler, die zuerst die intuitiven Methoden gelernt haben und dann zu den besseren übergegangen sind, haben hingegen einige unerwartete Vorteile.
Sie kennen sowohl die richtigen als auch die falschen Methoden und sind oft die viel besseren Lehrer.
\textbf{Obwohl dieses Buch die richtigen Methoden lehrt, ist es deshalb genauso wichtig, zu wissen, was man \textit{nicht} tun darf und warum.}
Deshalb werden die am häufigsten benutzten falschen Methoden hier ausgiebig besprochen.

Wir beschreiben die Komponenten eines angemessenen Übungsablaufs in den folgenden Abschnitten.
Sie werden ungefähr in der Reihenfolge dargeboten, in der sie ein Schüler vom Anfang bis zum Ende eines neuen Musikstücks benutzen könnte.
\textbf{Anfänger sollten zunächst \hyperref[c1iii18]{Abschnitt III.18} lesen.}


\subsection{Position der Finger}
\label{c1ii2}

Entspannen Sie die Finger, und setzen Sie die Hand auf eine glatte Fläche, wobei alle Fingerspitzen auf der Oberfläche ruhen und das Handgelenk in gleicher Höhe wie die Knöchel ist.
\textbf{Die Hand und die Finger sollten eine Kuppel formen.
Alle Finger sollten gebogen sein.
Der Daumen sollte leicht nach unten zeigen und leicht zu den Fingern hin gebeugt sein, sodass das Nagelglied des Daumens von oben gesehen parallel zu den anderen Fingern ist.}
Diese leichte Einwärtsbeugung des Daumens ist nützlich, wenn Sie \hyperref[c1iii7e]{Akkorde mit weiter Spanne} spielen.
Dieses bringt die Daumenspitze in eine Position parallel zu den Tasten und macht es unwahrscheinlicher, dass Sie eine benachbarte Taste treffen.
Es richtet den Daumen außerdem so aus, dass die richtigen Muskeln zum Anheben und Senken des Daumens benutzt werden.
\textbf{Die Finger sind leicht gekrümmt, abwärts gebogen und treffen in einem Winkel von ungefähr 45 Grad auf die Oberfläche.}
Diese gekrümmte Haltung erlaubt es den Fingern, zwischen den schwarzen Tasten zu spielen.
Die Daumenspitze und die anderen Fingerspitzen sollten ungefähr einen Halbkreis auf der glatten Fläche bilden.
Wenn Sie dieses mit beiden Händen nebeneinander tun, dann sollten sich die beiden Daumennägel gegenüberliegen.
Benutzen Sie den Teil des Daumens direkt unter dem Daumennagel zum Spielen, nicht das Gelenk zwischen dem Nagelglied und dem mittleren Glied.
Der Daumen ist ohnehin zu kurz; spielen Sie deshalb mit seinem vorderen Teil, damit Sie mit allen Fingern möglichst gleichmäßig spielen.
Bei den anderen Fingern liegt der Knochen an den Fingerspitzen nah an der Haut.
An der Unterseite der Finger (gegenüber dem Nagel) ist das Fleisch dicker.
Dieses Polster sollte die Tasten berühren, nicht die Fingerspitze.

Das ist die Ausgangsposition.
Wenn Sie erst begonnen haben zu spielen, müssen Sie Ihre Finger eventuell fast vollständig strecken oder sie mehr krümmen, je nachdem, was Sie gerade spielen.
\textbf{Obwohl der Anfänger die ideale gekrümmte Haltung lernen muss, ist ein striktes Beibehalten der gekrümmten Haltung deshalb nicht richtig; das werden wir \hyperref[c1iii4b]{später detailliert besprechen}, insbesondere weil die gekrümmte Haltung bedeutende Nachteile hat.}


\subsection{Höhe der Sitzbank und ihr Abstand zum Klavier}
\label{c1ii3}

Die richtige Höhe der Sitzbank und ihr Abstand zum Klavier sind ebenfalls eine Frage des persönlichen Geschmacks.
Setzen Sie sich zunächst so auf die Bank, dass die Ellbogen an Ihrer Seite sind und die Unterarme geradeaus in Richtung Klavier zeigen.
\textbf{Mit den Händen in Spielposition auf den Tasten sollten die Ellbogen ein wenig unterhalb der Hände, ungefähr in Höhe der Tasten sein.}
Setzen Sie nun Ihre Hände auf die weißen Tasten - der Abstand der Sitzbank zum Klavier und Ihre Sitzposition sollten so sein, dass die Ellbogen dicht am Körper vorbeigehen, wenn Sie sie aufeinander zubewegen.
Setzen Sie sich nicht in die Mitte der Bank, sondern sitzen Sie näher zur Vorderkante, sodass Sie Ihre Füße fest auf den Boden oder die Pedale stellen können.
Die Höhe der Sitzbank und die Sitzposition sind beim Spielen lauter Akkorde am wichtigsten.
Deshalb können Sie diese Position testen, indem Sie gleichzeitig zwei Akkorde so laut Sie können auf den schwarzen Tasten spielen.
Die Akkorde sind \textit{C\#2 G\#2 C\#3} (5 2 1) für die linke Hand und \textit{C\#5 G\#5 C\#6} (1 2 5) für die rechte Hand.
Drücken Sie die Tasten mit dem vollen Gewicht Ihrer Arme und Schultern fest nieder, wobei Sie sich leicht nach vorne beugen, um einen donnernden und respekteinflößenden Klang zu erzeugen.
Vergewissern Sie sich, dass die Schultern vollkommen einbezogen sind.
Laute, eindrucksvolle Klänge können nicht durch den Einsatz der Hände und Unterarme allein erzeugt werden; die Kraft muss aus den Schultern und dem Körper kommen.
Wenn dies bequem möglich ist, sollten die Position der Bank und die Sitzposition korrekt sein.
In der Vergangenheit neigten die Lehrer dazu, ihre Schüler zu hoch sitzen zu lassen;
aus diesem Grund ist die Standardhöhe von Sitzbänken mit fester Höhe meistens einen bis zwei Zoll\footnote{2,5 - 5 cm} zu hoch, was den Schüler dazu zwingt, mehr mit den Fingerspitzen als mit den vorderen Fingerpolstern zu spielen.
Es ist deshalb wichtig, eine Bank mit verstellbarer Höhe zu haben.


\subsection{Ein neues Stück - Anhören und analysieren (\enquote{Für Elise})}
\label{c1ii4}

\textbf{Die beste Möglichkeit, mit dem Lernprozess zu beginnen, ist, sich eine Aufführung (Aufnahme) davon anzuhören.}
Der Einwand, dass das Stück als erstes anzuhören eine Art \enquote{Betrug} sei, hat keine vertretbare Grundlage.
Der angebliche Nachteil ist, dass Schüler am Ende nur noch imitieren könnten, anstatt ihre Kreativität zu benutzen.
\textbf{Es ist jedoch unmöglich, das Spiel von jemand anderem zu imitieren, weil die Spielstile so individuell sind.}
Diese Tatsache kann für einige Schüler beruhigend sein, die sich vielleicht selbst dafür die Schuld geben, dass sie es nicht schaffen, einen berühmten Pianisten zu imitieren.
Wenn möglich, hören Sie sich mehrere Aufnahmen an.
Diese können Ihnen alle Arten von neuen Ideen und Möglichkeiten eröffnen, die zu lernen mindestens genauso wichtig ist wie die Fingertechnik.
Sich nichts anzuhören ist wie zu behaupten, man dürfe nicht zur Schule gehen, weil die Schule die Kreativität zerstören wird.
Einige Schüler glauben, dass das Anhören eine Zeitverschwendung sei, weil sie niemals so gut spielen werden.
Denken Sie in diesem Fall noch einmal darüber nach.
Wenn die hier beschriebenen Methoden nicht dazu führen würden, dass Sie \enquote{so gut} spielen werden, würde ich dieses Buch nicht schreiben!
Wenn Schüler sich viele Aufnahmen anhören, geschieht meistens folgendes: Sie entdecken, dass die Vortragsweisen nicht einheitlich gut sind; dass sie sogar \textit{ihr eigenes} Spielen gegenüber dem in einigen Aufnahmen vorziehen.

\textbf{Der nächste Schritt ist, die Struktur der Komposition zu analysieren.}
Diese Struktur wird benutzt, um das Übungsprogramm zu bestimmen und die für das Lernen des Stücks benötigte Zeit zu schätzen.
\textbf{Wie jeder erfahrene Klavierlehrer weiß, ist die Fähigkeit, die zum vollständigen Lernen eines Stücks notwendige Zeit zu schätzen, für den Erfolg des Übungsablaufs von entscheidender Bedeutung.}
Lassen Sie uns Beethovens \enquote{Für Elise} als Beispiel benutzen.
\textbf{Die Analyse beginnt immer mit dem Nummerieren der Takte auf dem Notenblatt.}
Wenn die Takte noch nicht markiert sind, markieren Sie jeden zehnten Takt mit einem Bleistift direkt über der Mitte des Takts.
Ich zähle jeden unvollständigen Takt am Anfang als Takt 1; andere zählen nur die vollständigen Takte, aber das macht es schwierig, den ersten unvollständigen Takt zu identifizieren.\footnote{In der Literatur wird in der Regel ein  unvollständiger erster Takt (= Auftakt) nicht gezählt.
Wenn der erste Takt jedoch mit Hilfe von Pausenzeichen vollständig gedruckt wird, dann wird er gezählt.
Wenn die Wiederholungen mit Voltenklammern notiert werden, dann ist der Auftakt in \enquote{Für Elise} gleichzeitig das dritte Achtel des neunten Takts und wird deshalb wieder nicht gezählt.}
In \enquote{Für Elise} werden die ersten vier vollständigen Takte im Grunde fünfzehnmal wiederholt, das heißt Sie müssen nur vier Takte lernen, um 50\% des Stücks spielen zu können (es hat 124 vollständige Takte).\footnote{Der letzte Takt ist zwei Achtel lang und kann mit dem Auftakt einen vollständigen Takt bilden, sodass man \enquote{Für Elise} wie viele Lieder mit ähnlichem Aufbau ohne die Taktart zu unterbrechen mehrmals hintereinander spielen kann.}
Weitere sechs Takte werden viermal wiederholt, sodass man nur zehn Takte lernen muss, um 70\% des Stücks zu spielen.
Wenn Sie die Methoden dieses Buchs benutzen, können Sie also 70\% dieses Stücks in weniger als 30 Minuten auswendig lernen, weil diese Takte ziemlich einfach sind.
Zwischen diesen wiederholten Takten stehen zwei Unterbrechungen, die nicht einfach sind.
Ein Schüler mit ein bis zwei Jahren Unterricht sollte in der Lage sein, die erforderlichen 50 abweichenden Takte dieses Stücks in zwei bis fünf Tagen zu lernen und fähig sein, das ganze Stück nach ein bis zwei Wochen in der richtigen Geschwindigkeit und auswendig zu spielen.
Danach kann der Lehrer anfangen, mit dem Schüler den musikalischen Inhalt des Stücks zu besprechen; wie lange das dauert, hängt von den musikalischen Kenntnissen des Schülers ab.
Wir werden nun die technischen Details der schwierigen Abschnitte besprechen.

\textbf{Das Geheimnis, die Technik schnell zu erwerben, liegt darin, bestimmte Tricks dafür zu kennen, schwierige Passagen nicht nur zu spielbaren sondern oft zu trivial einfachen zu reduzieren.}
Wir können uns nun auf die wundersame Reise in die Gehirne der Genies begeben, die unglaublich effiziente Arten herausgefunden haben, das Klavierspielen zu üben!



