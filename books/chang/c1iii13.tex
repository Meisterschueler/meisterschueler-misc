% File: c1iii13

%\section{Filmen und Aufnehmen des eigenen Spielens,\footnote{MIDI, Digitalpianos, Keyboards usw.}}\hypertarget{c1iii13}{} 
\section{Filmen und Aufnehmen des eigenen Spielens}\hypertarget{c1iii13}{} 


Eine der besten Möglichkeiten, Ihr musikalisches Spielen zu verbessern und für Auftritte zu üben, ist, sich selbst zu filmen bzw. aufzunehmen und es sich anzusehen oder anzuhören.
Sie werden überrascht sein, wie gut oder wie schlecht die verschiedenen Teile Ihres Spielens sind.
Oftmals unterscheidet es sich sehr von dem, was Sie zu tun glauben.
Haben Sie einen guten Anschlag?
Haben Sie Rhythmus?
Ist Ihr Tempo genau und konstant?
Welche Bewegungen unterbrechen den Rhythmus?
Stellen Sie die Melodielinien klar heraus?
Ist eine Hand zu laut oder zu leise?
Sind die Arme, Hände und Finger in der optimalen Position?
Benutzen Sie den ganzen Körper, d.h. sind die Körperbewegungen mit den Händen synchron oder arbeiten sie gegeneinander?
All das und noch viel mehr wird sofort offensichtlich.
Die gleiche Musik klingt sehr unterschiedlich, wenn Sie sie spielen oder sich Ihre Aufnahme anhören.
Man hört viel mehr, wenn man sich eine Aufnahme anhört, als wenn man spielt.
Eine Videoaufnahme ist die beste Möglichkeit, sich auf ein Konzert vorzubereiten und kann manchmal die \hyperlink{c1iii15}{Nervosität} fast völlig eliminieren, da Sie eine genauere Vorstellung von Ihrem Auftritt haben.

Zuerst stellten die meisten Pianisten nur Audioaufnahmen her, weil sie dachten, daß das musikalische Ergebnis das wichtigste sei; hinzu kommt, daß die älteren Videokameras Musik nicht angemessen aufzeichnen konnten.
Audioaufnahmen haben den Nachteil, daß eine gute Aufzeichnung des Klavierklangs schwieriger ist als den meisten bewußt ist, und solche Versuche führen oft zu einem Fehlschlag und zur Aufgabe der Bemühungen.
Videokameras sind mittlerweile so erschwinglich und vielseitig, daß die Videoaufzeichnung nun zweifellos die bessere Methode ist.
Obwohl der resultierende Klang eventuell nicht der CD-Qualität entspricht (glauben Sie den Behauptungen der Hersteller von digitalen Videokameras nicht), brauchen sie keine solche Qualität, um alle nützlichen pädagogischen Ziele zu erreichen.
Wählen Sie eine Videokamera aus, bei der man die automatische Aussteuerung (AGC = automatic gain control) der Audioaufnahme abschalten kann; ansonsten werden die pianissimo gespielten Passagen verstärkt und verzerrt.
Viele Verkäufer kennen sich mit diesen Eigenschaften nicht aus, da sie meistens Optionen in den Einstellungen der Software sind.
Sie werden auch ein ziemlich stabiles Stativ benötigen; ein leichtes könnte wackeln, wenn Sie auf das Klavier \enquote{einhämmern}.
Nur Konzertpianisten benötigen höherwertige Audioaufnahmesysteme; suchen Sie sich, um die besten Resultate kosteneffizient zu erzielen, ein Aufnahmestudio. 
Hochwertige Audioaufnahmen benötigen Sie eventuell für mehrere Zwecke; die Aufnahmetechnik ändert sich so schnell, daß Sie am besten im Internet nach den aktuellen Geräten und Methoden suchen, ich werde deshalb hier nicht weiter darauf eingehen.

Fangen Sie damit an, daß Sie eine 1:1-Zuordnung zwischen dem, was Sie \textit{denken}, daß Sie spielen, und dem tatsächlichen Ergebnis (Video oder Audio) herstellen.
Auf diese Art können Sie Ihre allgemeinen Spielgewohnheiten so abändern, daß das Ergebnis richtig herauskommt.
Wenn Sie z.B. bei leichteren Abschnitten schneller spielen als Sie denken und langsamer bei schwereren Abschnitten, können Sie die richtigen Anpassungen vornehmen.
Sind Ihre Pausen lange genug?
Sind die Enden überzeugend?

Das Aufnehmen wird offenbaren, wie Sie bei einer richtigen Aufführung reagieren würden, z.B. wenn Sie einen Fehler machen oder hängenbleiben.
Reagieren Sie negativ auf Fehler und werden entmutigt, oder können Sie sich davon lösen und auf die Musik konzentrieren?
Während eines Konzerts neigt man dazu, Gedächtnisblockaden usw. an unerwarteten Stellen zu bekommen, an denen man im allgemeinen während des Übens keine Probleme hatte.
Das Aufnehmen kann die meisten dieser Problemstellen zutage fördern.
Ihre Stücke sind nicht \enquote{fertig}, solange Sie sie nicht zufriedenstellend aufnehmen können.
Videoaufnahmen sind eine sehr gute Simulation für das Spielen in einem Konzert.
Wenn Sie also während der Videoaufzeichnung zufriedenstellend spielen können, sollten Sie wenig Probleme haben, dieses Stück in einem Konzert zu spielen.
Wenn Sie erst einmal mit dem Aufnehmen begonnen haben, möchten Sie die Aufnahmen vielleicht sogar anderen zusenden!

Was sind die Nachteile?
Der Hauptnachteil ist, daß es viel Zeit beansprucht, da man sich die Aufnahmen ansehen und anhören muß.
Sie können vielleicht etwas Zeit sparen, indem Sie sich die Aufnahmen anhören, während Sie etwas anderes erledigen.
Die Aufnahme selbst braucht wenig zusätzliche Zeit, weil das als Teil der Übungszeit zählt.
Jedesmal wenn Sie einen Abschnitt korrigieren, müssen Sie jedoch erneut aufnehmen und wieder abhören.
Deshalb läßt sich die Tatsache nicht leugnen, daß sich selbst anzusehen oder anzuhören eine zeitaufwendige Aufgabe ist.
Es ist jedoch etwas, das jeder Klavierschüler tun muß.
Ein Problem mit Videokameras ist, daß ihr Motorgeräusch durch das eingebaute Mikrofon aufgenommen wird.
Wenn Sie das stört, finden Sie entweder ein Modell, das ein anschließbares Mikrofon von guter Qualität hat oder eines mit einem  Mikrofoneingang und kaufen Sie ein separates Mikrofon guter Qualität.


\hypertarget{c1iii13MIDI}{}

<h3><br>MIDI, Digitalpianos, Keyboards usw.</h3>

\footnote{Die folgenden Abschnitte sind im Original nicht enthalten.
Ich habe sie jedoch wegen der besseren Lesbarkeit in normaler Schrift belassen.
MIDI steht für Musical Instruments Digital Interface. Die Töne werden dabei als Code gespeichert, der in unserer Sprache z.B. \enquote{Ein C3 auf Kanal 2 mit der Lautstärke 90 und der Länge von 192 Ticks.} oder \enquote{Eine Viertelnote der Tonhöhe C3, die forte und legato gespielt wird.} bedeutet.}


\textbf{Besitzer eines PCs und eines MIDI-Keyboards haben eine gute Alternative zu Audioaufnahmen per MD, Tonband usw.}
Dazu muß entweder ein Soundchip auf dem \hyperlink{Motherboard}{Motherboard} des PC integriert sein und der PC einen Joystick/MIDI-Anschluß haben oder Sie brauchen eine extra Soundkarte.
Die preisgünstigeren Karten haben einen kombinierten Joystick/MIDI-Anschluß, für den Sie dann ein zusätzliches Adapterkabel brauchen.
Die teureren Karten haben teilweise richtige 5-polige MIDI-Buchsen.
Leider gibt es ein paar Soundkarten, die zwar MIDI ausgeben aber nicht MIDI aufnehmen können; auf der Verpackung steht davon natürlich nichts.
Am besten lassen Sie sich vor dem Kauf vom Händler oder Vorbesitzer bestätigen, daß die Karte in beiden Richtungen MIDI-fähig ist und bringen die Karte wieder zurück, wenn es nicht stimmt.

Beim Verkabeln werden gerne die Anschlüsse vertauscht.
Haben Sie ein Joystick/MIDI-Adapterkabel, dann stecken Sie den Stecker, der mit \textit{MIDI-Out} beschriftet ist, in die \textit{MIDI-In}-Buchse des Keyboards und den \textit{MIDI-In}-Stecker in die \textit{MIDI-Out}-Buchse des Keyboards.
Bei mehreren Geräten gibt es natürlich noch weitere Möglichkeiten.
\textbf{Es muß aber immer der MIDI-Out-Anschluß eines Geräts mit dem MIDI-In- oder MIDI-Thru-Anschluß eines anderen Geräts verbunden werden.}

Zu guter Letzt brauchen Sie noch ein Programm, mit dem Sie MIDI aufnehmen, bearbeiten und wiedergeben können, d.h. ein Sequenzer-Programm.
Es gibt einige professionelle Programme, die natürlich auch einiges kosten.
Sie bieten jede Menge Funktionen, insbesondere für die nachträgliche Bearbeitung, teilweise auch Notensatz und virtuelle (d.h. vom Programm erzeugte) Geräte, wie z.B. Drum-Computer oder Synthesizer.
Wenn es nur darum geht, das eigene Klavierspielen zu kontrollieren, kann man jedoch erst einmal getrost auf diese ganzen Funktionen verzichten.
Es gibt auch ein paar Free- und Shareware-Programme, die für diesen Zweck völlig ausreichen.

Sie können die aufgenommenen MIDI-Signale direkt über das Keyboard wiedergeben, d.h. Sie erhalten original den gleichen Klang wie bei der Aufnahme.
Dazu müssen Sie im Sequenzer-Programm die Ausgabe auf die MIDI-Out-Schnittstelle anstatt auf den Klangerzeuger-Chip der Soundkarte einstellen.

\hypertarget{midi_check}{}

Wenn Sie weiter ins Detail gehen möchten, können Sie sich im Piano-Roll-Editor ansehen, ob der Einsatz der einzelnen Noten exakt auf die Taktschläge kommt, ob die Notenlängen stimmen, ob Sie wirklich legato spielen usw.
Der Piano-Roll-Editor ist das Fenster, in dem die Noten wie auf der Steuerrolle einer Kirmesorgel als mehr oder weniger lange Striche dargestellt sind.
Im Event-Editor, also in der Einzelanzeige der MIDI-Daten können Sie die Stärke Ihres Anschlags, die meistens als Velocity bezeichnet wird, sehen.
Natürlich dürfen Sie die Werte auch nicht zu ernst nehmen. Schließlich sind Sie kein Roboter und außerdem klingt es tot und langweilig, wenn alle Noten exakt die gleiche Lautstärke und auf die interne Auflösung genau die gleiche Länge haben.
\textbf{Auch dürfen Sie trotz der ganzen technischen Unterstützung das Training Ihres Gehörs nicht vernachlässigen.}
Hören Sie sich bei der Kontrolle per MIDI ebenfalls selber beim Spielen zu, und achten Sie beim Abhören auf die gleichen Dinge, die im \hyperlink{c1iii13}{ersten Abschnitt} beschrieben wurden.

Wenn Ihr Keyboard keine \enquote{Split}-Funktion hat (Aufteilung der Tastatur bei einer bestimmten Note in 2 MIDI-Spuren) können Sie mit dem Programm die Noten der RH und LH voneinander trennen und jeweils separat anhören.
Bei mehrstimmigen Stücken können Sie auch den Aufwand noch ein wenig weiter treiben und die einzelnen Stimmen voneinander trennen.
Das ist vor allem dann sinnvoll, wenn aus der Einspielung ein Notat erstellt werden soll.
Die automatische Umwandlung von eingespielten MIDI-Dateien in einen Notensatz ist nämlich eine Sache für sich.
Hier trennt sich bei den Programmen schnell die Spreu vom Weizen, und in den meisten Fällen ist eine erhebliche Nachbearbeitung der MIDI-Daten erforderlich.
Beim Umsetzen von Live-Einspielungen sind Programme von Vorteil, die die MIDI-Daten so weit wie möglich unverändert lassen und die Notation separat speichern, da der Originalklang der aufgenommenen MIDI-Signale zur Hörkontrolle erhalten bleibt.
Wenn es ohnehin nur darum geht, Noten z.B. für einen Chor oder eine Band zu setzen, ist man mit einer schrittweisen Eingabe der Noten (Step-Recording) oft schneller am Ziel.


\hypertarget{kauf}{}

Zum Schluß noch ein paar Fragen, die man vor der Entscheidung für und dem Kauf eines Keyboards klären sollte:

\begin{enumerate}[label={\arabic*.}] 
\item Stellen Sie sich zunächst die Frage, was Ihr Ziel bei der ganzen Sache ist.<br>
\textbf{Sie möchten richtig Klavier spielen.} Eigentlich geht das ja nur auf einem guten Klavier.
Zum Glück sind die guten Digitalpianos mittlerweile sehr nah dran am Klavier.
Am wichtigsten ist die Qualität der Tastatur und ihre Ähnlichkeit mit einer Klaviertastatur hinsichtlich Spielgefühl und Ansprechverhalten.
Ein gutes Digitalpiano ist auf jeden Fall besser als ein schlechtes oder verstimmtes Klavier!<br>
\textbf{Sie möchten irgendwann als Alleinunterhalter auftreten.}
Vergessen Sie die Sache mit dem Klavier.
Kaufen Sie sich ein Entertainer-Keyboard mit NNN(N) Klängen und Begleitrhythmen.
Achten Sie nicht zu sehr auf die eingebauten Lautsprecher; für Ihren Auftritt brauchen Sie ohnehin noch einen Verstärker und vernünftige Boxen.<br>
\textbf{Sie möchten das Keyboard nur als Eingabe für ein Sequenzer-Programm benutzen.} Hauptsache MIDI -  Live-Einspielungen können selten so benutzt werden wie sie sind und müssen in der Regel im Sequenzer-Programm kräftig nachbearbeitet werden.
Als Alternative sei hier noch einmal an das Step-Recording erinnert.<br>
\textbf{Sie möchten eigene Klänge \enquote{basteln}.} Willkommen im Wunderland der Sampler, Synthesizer und Workstations.
Sie werden viel Zeit mit Tätigkeiten verbringen, für die Sie die in diesem Buch vorgestellten Methoden nicht brauchen.
Wenn Sie allerdings nicht nur Musik im Studio produzieren, sondern auch mal live auf der Bühne spielen möchten, ist es sehr von Vorteil, die Methoden zu kennen.<br>


\item Da die Soundkarten für PCs auch immer besser werden und zunehmend Funktionalität und ladbare Klänge bieten, ist die Frage, auf welche Eigenschaften und Funktionen beim Keyboard verzichtet werden kann.
Beim Keyboard lassen sich so eventuell ein paar Euro einsparen, die für etwas anderes sinnvoller angelegt sind.
Beim Kopfhörer sollte man z.B. nicht sparen; allerdings auch nicht Geld für Meßwertunterschiede ausgeben, die man nicht hört.
Probieren Sie am besten im Geschäft in Ruhe mehrere Kopfhörer an Ihrem favorisierten Keyboard aus.


\item Wohin mit all den Teilen? Unterschätzen Sie nicht den Platzbedarf.
Zuerst ist es nur das Keyboard, das in der Nähe des PCs steht, weil das MIDI-Kabel schließlich nicht unendlich lang ist.
Irgendwann kommt dann meistens ein vernünftiger Verstärker inkl. Boxen hinzu oder die Stereoanlage zieht um.
Wenn die Zahl der \enquote{Eingabegeräte} steigt und in der PC-Software kein Mischpult enthalten ist, dann kommt das auch noch hinzu.
Auch hier stellt sich wieder die Frage \textbf{\enquote{Was will ich...?}}


 \end{enumerate}
\footnote{Ende der Einfügung.}



