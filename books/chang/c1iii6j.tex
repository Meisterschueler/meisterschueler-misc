% File: c1iii6j

\subsubsection{Langzeitgedächtnis aufbauen}
\label{c1iii6j} 

\textbf{Es gibt mindestens fünf grundlegende Arten von Gedächtnis:}

\begin{enumerate}[label={\arabic*.}] 
\item \textbf{Hand-Gedächtnis (hören, fühlen)}
\item \textbf{Musik-Gedächtnis (hören)}
\item \textbf{fotografisches Gedächtnis (sehen)}
\item \textbf{Tastatur-Gedächtnis und mentales Spielen (sehen, fühlen, Gehirn)}
\item \textbf{theoretisches Gedächtnis (Gehirn)}
\end{enumerate}

Praktisch jeder benutzt eine Kombination davon.
Die meisten verlassen sich hauptsächlich auf eine dieser Arten und benutzen die anderen zur Ergänzung.
\label{c1iii6hand}
Wir haben \hyperref[c1iii6d]{oben} bereits das \textbf{Hand-Gedächtnis} besprochen.
Es wird durch bloßes Wiederholen \enquote{bis die Musik in der Hand ist} erworben.
Bei der intuitiven Lehrmethode wurde das als bester Weg zum Auswendiglernen angesehen, da man bessere Methoden nicht kannte.
Wir wollen es nun durch die anderen Gedächtnisverfahren ersetzen, um ein dauerhafteres und verläßlicheres Gedächtnis aufzubauen.


\label{c1iii6musik}

Das \textbf{Musik-Gedächtnis} basiert auf der Musik, d.h. der Melodie, dem Rhythmus, dem Ausdruck, den Emotionen usw.
Dieser Ansatz funktioniert bei emotionalen und musikalischen Menschen, die mit der Musik starke Gefühle assoziieren, am besten.
Diejenigen mit \hyperref[c1iii12]{absolutem Gehör} werden damit ebenfalls Erfolge erzielen, weil sie die Noten auf dem Klavier einfach nach ihrer Erinnerung der Musik finden können.
Menschen, die gerne komponieren, neigen ebenfalls dazu, dieses Gedächtnisverfahren zu benutzen.
Musiker haben meistens nicht automatisch ein gutes musikalisches Gedächtnis.
Es hängt von der Art ihres Gehirns ab, und man kann diese Fähigkeit trainieren, wie es unten in Abschnitt \hyperref[c1iii6m]{\autoref{c1iii6m}} besprochen wird.
Zum Beispiel können sich Menschen mit gutem Musik-Gedächtnis auch an andere Dinge erinnern, wie den Namen des Komponisten und der Komposition.
Bei den meisten Kompositionen, die sie ein paar mal gehört haben, können sie die Melodien gut abrufen, so daß sie das Lied summen können, wenn man ihnen den Titel nennt.


\label{c1iii6foto}

Beim \textbf{fotografischen Gedächtnis} merken Sie sich die ganzen Notenblätter, stellen sie sich bildlich vor und lesen sie in Gedanken.
Selbst diejenigen, die glauben, sie hätten kein fotografisches Gedächtnis, können es sich aneignen, wenn sie das fotografische Gedächtnis \textit{von Anfang an} routinemäßig trainieren, wenn sie das Stück üben.
Wenn sie das Verfahren ab dem ersten Tag (wenn sie mit dem Stück beginnen) gewissenhaft anwenden, werden viele Menschen feststellen, daß es im Durchschnitt je Seite nur ein paar Takte gibt, die noch nicht fotografisch abgespeichert sind, wenn sie das Stück zufriedenstellend spielen können.
Ein Weg, fotografisch auswendig zu lernen, ist, die hier umrissenen Methoden für Technik und Gedächtnis genau zu befolgen, aber auch gleichzeitig das Notenblatt Hand für Hand, Takt für Takt und Abschnitt für Abschnitt fotografisch auswendig zu lernen.

Eine andere Möglichkeit, zu einem fotografischen Gedächtnis zu gelangen, ist, sich zunächst den allgemeinen Aufbau einzuprägen, z.B. wie viele Zeilen es auf der Seite gibt und wie viele Takte je Zeile, danach die Noten jedes Takts, dann die Ausdrucksbezeichnungen usw.
Fangen Sie mit den groben Zügen an, und ergänzen Sie schrittweise die Details.
Beginnen Sie das fotografische Gedächtnis mit dem Einprägen jeweils einer Hand.
Sie müssen wirklich ein genaues Foto der Seite anfertigen, komplett mit ihren Fehlern und zusätzlichen Markierungen.
Wenn Sie Schwierigkeiten haben, sich bestimmte Takte zu merken, zeichnen Sie etwas ungewöhnliches dort hin, wie ein lachendes Gesicht (\enquote{Smiley}) oder Ihre eigenen Markierungen, die Ihr Gedächtnis wachrütteln.
Wenn Sie sich das nächste Mal an diesen Abschnitt erinnern möchten, denken Sie zuerst an das lachende Gesicht.

Ein Vorteil des fotografischen Gedächtnisses ist, daß man ohne das Klavier an dem Stück arbeiten kann, jederzeit, überall.
Tatsächlich muß man sich das Stück, wenn man es sich erst einmal angeeignet hat, abseits des Klaviers in Gedanken so oft wie möglich vorstellen, bis es dauerhaft gespeichert ist.
Ein weiterer Vorteil ist, daß man, wenn man beim Spielen eines Stücks in der Mitte stecken bleibt, leicht wieder anfangen kann, indem man diesen Abschnitt in Gedanken liest.
Das fotografische Gedächtnis gestattet es Ihnen auch, vorauszulesen während Sie spielen, was Ihnen dabei hilft vorauszudenken.
Ein weiterer Vorteil ist, daß es Ihnen beim Spielen vom Blatt helfen wird.

Der Hauptnachteil ist, daß die meisten Menschen die fotografische Erinnerung nicht für lange Zeiträume aufrecht halten können, weil die Pflege dieser Art von Gedächtnis üblicherweise mehr Arbeit erfordert als andere Methoden.
Im Gegensatz zu den meisten anderen Methoden erhält sich das fotografische Gedächtnis ohne einen zusätzlichen Aufwand nicht selbst.
Ein weiterer Nachteil ist, daß es ein vergleichsweise langsamer geistiger Prozeß ist, sich die Noten in Gedanken vorzustellen und zu lesen, der mit dem Spielen in Konflikt geraten kann.
\textbf{Deshalb ist das fotografische Gedächtnis für die meisten Menschen nicht das praktikabelste Gedächtnisverfahren.}
Es eignet sich nur für diejenigen, die bereits ein gutes fotografisches Gedächtnis haben und denen es Spaß macht, es weiterzuentwickeln.

Ich arbeite nicht bewußt für das fotografische Gedächtnis, außer bei den ersten paar Takten, um mir beim Anfangen zu helfen.
Trotzdem habe ich am Anfang, wenn ich ein neues Stück lerne, wegen der Notwendigkeit oft auf die Noten zurückzugreifen, eine erhebliche fotografische Erinnerung.
Selbst für diejenigen, die nicht planen, ein fotografisches Gedächtnis zu erwerben, ist es eine gute Idee, jede fotografische Erinnerung, die man in diesem Stadium bekommen kann, zu behalten; d.h. unterstützen Sie es, weisen Sie es nicht von sich.
Sie werden überrascht sein, wie lange und wie gut es Ihnen erhalten bleibt, besonders wenn Sie es weiter pflegen.
Ich zwinge mich selbst nicht dazu, fotografisch auswendig zu lernen, weil ich weiß, daß ich am Ende meistens ein Tastatur-Gedächtnis wie unten beschrieben und ein Musik-Gedächtnis habe.
Es ist erstaunlich, daß man oftmals etwas viel besser tun kann, wenn kein Druck dahinter ist, und ich eigne mir wie von selbst eine ganze Menge fotografischer Erinnerungen an, die ich ein Leben lang behalte.
Ich wünschte mir sicherlich, daß ich das fotografische Gedächtnis früher mehr praktiziert hätte, da ich vermute, daß ich darin viel besser geworden wäre als ich es jetzt bin.

Diejenigen, die glauben, sie hätten kein fotografisches Gedächtnis, können es mit dem folgenden Trick versuchen.
Lernen Sie zunächst ein kurzes Musikstück mit so viel fotografischem Gedächtnis auswendig, wie Sie ohne weiteres zuwege bringen, und machen Sie sich keine Sorgen, wenn es nur teilweise klappt.
Wenn ein Abschnitt auswendig gelernt ist, bilden Sie ihn jeweils auf die Noten ab, von denen Sie das Stück gelernt haben, d.h. versuchen Sie sich für jede Note, die Sie spielen, die entsprechende Note auf dem Blatt vorzustellen.
Da Sie jeden Teil HS kennen, sollte dieses Abbilden von der Tastatur auf die Notenblätter einfach sein.
Beim Abbilden werden Sie auf das Notenblatt sehen müssen, um sich zu vergewissern, daß jede Note in der korrekten Position auf der richtigen Seite ist.
Sogar die Ausdrucksbezeichnungen sollten abgebildet werden.
Spielen Sie so lange abwechselnd aus dem fotografischen Gedächtnis und bilden die Tastatur auf die Noten ab, bis die Fotografie vollständig ist.
Dann können Sie Ihre Freunde verblüffen, indem Sie die Noten für das ganze Stück aufschreiben und das ab einer beliebigen Stelle!
Beachten Sie, daß Sie in der Lage sind, alle Noten zu schreiben, sowohl vorwärts als auch rückwärts, oder von irgendwo in der Mitte oder sogar jede Hand einzeln.
Und sie dachten, nur Wolfgang\footnote{A. Mozart} könnte das!


\label{c1iii6tastatur}

\textbf{Tastatur-Gedächtnis und mentales Spielen}: Beim Tastatur-Gedächtnis erinnern Sie sich während des Spielens zusammen mit der Musik an die Reihenfolge der Tasten und die Handbewegungen.
Es ist, als ob Sie ein Klavier im Kopf hätten und es spielen könnten.
Beginnen Sie mit dem Tastatur-Gedächtnis, indem Sie HS auswendig lernen, dann HT.
Spielen Sie dann, wenn Sie nicht am Klavier sind, das Stück in Ihrem Kopf, zunächst wieder HS.
\textbf{In Gedanken zu spielen (mentales Spielen), ohne Klavier, ist unser endgültiges Ziel; wir benutzen das Tastatur-Gedächtnis als Zwischenstufe.}
Es ist zunächst nicht notwendig, in Gedanken HT zu spielen, insbesondere wenn Sie es zu schwierig finden, obwohl Sie schließlich mit Leichtigkeit HT spielen werden.
Merken Sie sich, wenn Sie in Gedanken spielen, welche Abschnitte sie vergessen haben.
Nehmen Sie danach die Noten oder gehen Sie ans Klavier, und frischen Sie Ihr Gedächtnis auf.
Sie könnten auch das fotografische Gedächtnis für Teile ausprobieren, die Sie beim Benutzen des Tastatur-Gedächtnisses vergessen, da Sie sich ohnehin die Noten anschauen müssen, um sie erneut auswendig zu lernen.
Das mentale Spielen ist nicht nur deshalb schwierig, weil Sie das Stück auswendig gelernt haben müssen, sondern auch, weil Sie das Hand-Gedächtnis oder den Klavierklang nicht als Hilfe haben; aber genau deshalb ist es so mächtig.

Das Tastatur-Gedächtnis besitzt die meisten Vorteile des fotografischen Gedächtnisses, hat aber zusätzlich den Vorteil, daß die auswendig gelernten Noten Klaviertasten anstelle von dicken ovalen Punkten auf einem Blatt Papier sind; deshalb müssen Sie nicht von den ovalen Punkten zu den Tasten übersetzen.
Das erlaubt Ihnen, im Vergleich zum fotografischen Gedächtnis mit weniger Aufwand zu spielen, da der zusätzliche Prozeß, das Notenbild umzusetzen, entfällt.
Die Ausdrucksbezeichnungen sind keine Markierungen auf dem Papier, sondern gedankliche Vorstellungen der Musik (Musik-Gedächtnis).
Jedesmal, wenn Sie üben, pflegt sich das Tastatur-Gedächtnis -- einschließlich der Handbewegungen -- im Gegensatz zum fotografischen Gedächtnis von selbst.
Sie können das mentale Spielen ohne ein Klavier üben und so die zum Üben verfügbare Zeit mehr als verdoppeln, und Sie können vorausspielen, wie beim fotografischen Gedächtnis.

Als ich begann, das Tastatur-Gedächtnis zu benutzen, war meine seltsamste Beobachtung, daß ich dazu neigte, an den gleichen Stellen die gleichen Fehler zu machen und stecken zu bleiben, wie wenn ich tatsächlich am Klavier saß!
Wenn man darüber nachdenkt, macht das Sinn, weil alle Fehler ihren Ursprung im Gehirn haben, ob man am Klavier sitzt oder nicht.
Das Klavier macht niemals den Fehler, ich mache ihn.
Das läßt darauf schließen, daß wir vielleicht dazu in der Lage sind, bestimmte Aspekte des Klavierspielens zu üben und zu verbessern, indem wir in Gedanken üben, ohne ein Klavier.
Das wäre ein wahrhaft einzigartiger Vorteil des mentalen Spielens!
Die meisten Vorschläge für das Auswendiglernen, die in diesem Buch gemacht wurden, sind am besten auf das Tastatur-Gedächtnis anwendbar.
Das ist ein weiterer seiner Vorteile.
Das mentale Spielen ist der beste Test für das wahre Gedächtnis; wenn Sie das mentale Spielen ausführen, wird Ihnen bewußt werden, wie stark Sie noch vom Hand-Gedächtnis abhängig sind -- auch nachdem Sie dachten, Sie hätten das Tastatur-Gedächtnis erworben.
Erst nachdem Sie sich genügend mentales Spielen angeeignet haben, können Sie im Grunde vom Hand-Gedächtnis befreit sein.
Das Hand-Gedächtnis ist jedoch immer eine gute Reserve -- auch wenn Sie Ihr mentales Gedächtnis verloren haben, können Sie es gewöhnlich wiederherstellen, ohne auf die Notenblätter zu sehen, indem Sie das Stück einfach aus dem Hand-Gedächtnis auf dem Klavier spielen.

\textbf{Bei denjenigen, die das Singen vom Blatt lernen und ein absolutes Gehör erwerben möchten (\hyperref[c1iii12]{s. Abschnitt 12}), entwickeln sich diese Fertigkeiten durch das mentale Spielen automatisch.}
Bei der Tastatur-Methode stellt man sich die Tastatur vor, was dabei hilft, die richtigen Tasten für die absoluten Tonhöhen zu finden -- eine Fertigkeit, die Sie für das Komponieren oder das Improvisieren am Klavier benötigen.
Deshalb sollten diejenigen, die daran interessiert sind, das Tastatur-Gedächtnis zu erlernen, auch das Blattsingen und absolute Tonhöhen üben, weil sie schon teilweise auf dem Wege dorthin sind.
Das ist ein erstklassiges Beispiel dafür, wie Ihnen das Lernen einer Fertigkeit (Auswendiglernen) dabei hilft, viele andere zu lernen.
Zweifellos ist das mentale Spielen eine der Arten, wie die musikalischen Genies das geworden sind, was sie sind oder waren.
So können viele dieser \enquote{genialen Fähigkeiten} praktisch von uns allen erworben werden, wenn wir wissen, wie man sie übt.
Wir sind nun bei einer erstaunlichen Schlußfolgerung angelangt: Gedächtnis führt zu mentalem Spielen, was wiederum zu relativem bzw. absolutem Gehör führt!
Mit anderen Worten: Das sind wesentliche Bausteine der Technik -- wenn Sie alle drei erwerben, wird Ihre Fähigkeit auswendig zu lernen und vorzuspielen einen Quantensprung machen.

Wie bei jedem Gedächtnisverfahren muß das mentale Spielen von Anfang an geübt werden, ansonsten wird es Ihnen \textbf{niemals} gelingen.
Spielen Sie deshalb einen Abschnitt, sobald sie ihn auswendig gelernt haben, sofort in Gedanken, und behalten Sie dieses wie die anderen Gedächtnisverfahren bei.
Sie sollten schließlich in der Lage sein, die ganze Komposition in Gedanken zu spielen.
Sie werden erstaunt zurückblicken und sagen: \enquote{Nanu, das war leichter als ich dachte!}, weil dieses Buch alle für das mentale Spielen notwendigen Voraussetzungen bietet.

Können Sie die ganze Komposition erst in Gedanken spielen, werden Sie feststellen, daß Sie nun mit Leichtigkeit überall im Stück mit dem Spielen beginnen können, sogar mitten in einem Abschnitt oder einer Phrase.
Auch wenn sie abschnittsweise üben, ist das Beginnen in der Mitte eines Abschnitts gewöhnlich ziemlich schwierig; das mentale Spielen wird Sie in die Lage versetzen, irgendwo in einem Abschnitt zu beginnen -- was mit jeder anderen Methode schwer zu erlernen ist.
Sie können auch eine viel deutlichere Vorstellung der Struktur der Komposition und der Folge der Melodien bekommen, weil Sie nun diese ganzen Konstrukte in Ihrem Kopf analysieren können.
Sie können sogar mit Geschwindigkeiten \enquote{üben}, die Ihre Finger nicht bewältigen können.
Die Finger können niemals Geschwindigkeiten erreichen, die das Gehirn nicht erreichen kann; man kann es sicherlich mit teilweisem Erfolg versuchen, aber es wird zu vielen Fehlern führen.
Das mentale Spielen mit hohen Geschwindigkeiten wird das schnelle Spielen der Finger fördern.
Das Spielen in Gedanken braucht nicht unbedingt viel Zeit, da es sehr schnell geht, wenn Sie es beherrschen.
Sie können es auch abkürzen, indem Sie einfache Abschnitte übergehen und sich nur auf Stellen konzentrieren, an denen Sie gewöhnlich auf Schwierigkeiten stoßen.

Das Spielen in Gedanken hat einen weiteren Vorteil: Je mehr Stücke Sie auswendig gelernt haben, desto leichter wird es, mehr auswendig zu lernen!
Das geschieht, weil Sie die Zahl der Assoziationen erhöhen.
Auch wird sich Ihre Fähigkeit zum mentalen Spielen rasch steigern, während Sie es üben und seine zahlreichen Vorteile entdecken.
Beim \hyperref[c1iii6d]{Hand-Gedächtnis} ist es im Gegenteil dazu so, daß es schwieriger wird, mehr auswendig zu lernen, wenn Ihr Repertoire größer wird, weil die Wahrscheinlichkeit der Konfusion steigt.
Praktisch alle Konzertpianisten wenden das mentale Spielen in einem gewissen Maß an.
Auf der Stufe eines Konzertpianisten wird von Ihnen erwartet, daß Sie es kennen, da es absolut notwendig ist.
Es wird aber nicht immer offiziell gelehrt.
Einigen glücklichen Schülern wurde das mentale Spielen gelehrt; für die anderen ist es ein zähes Ringen, diese \enquote{neue} Fertigkeit zu erlernen, die man von ihnen erwartet, wenn sie eine bestimmte Fertigkeitsstufe erreichen.
Zum Glück ist es eine Fertigkeit, die für den ernsthaften Schüler nicht schwierig zu meistern ist, weil der Nutzen so 
unmittelbar und weitreichend ist, daß die Motivation kein Problem darstellt.
Auf der fortgeschrittenen Stufe ist es einfach zu lernen, da solche Schüler einiges an Theorie gelernt haben.
Ein guter Solfège-Kurs sollte diese Fertigkeit lehren, aber Solfège-Lehrer lehren nicht immer Fertigkeiten zum Auswendiglernen oder das mentale Spielen.

Zusammengefaßt: Das Tastatur-Gedächtnis sollte Ihre hauptsächliche Gedächtnismethode sein.
Sie müssen gleichzeitig\footnote{in Gedanken} die Musik hören, so daß das \hyperref[c1iii6musik]{Musik-Gedächtnis} ein Teil dieses Prozesses ist.
Benutzen Sie das \hyperref[c1iii6foto]{fotografische Gedächtnis}, wann immer es einfach ist, und fügen Sie soviel \hyperref[c1iii6theorie]{theoretisches Gedächtnis} (s.u.) wie möglich hinzu.
\textbf{Sie haben das Stück erst dann wirklich auswendig gelernt, wenn Sie es in Gedanken spielen können} -- das ist die einzige Möglichkeit, die Zuversicht zu bekommen, daß Sie ohne hörbare Fehler vorspielen können (alle Konzertpianisten können das).
Sie können damit die \hyperref[c1iii15]{Nervosität} reduzieren, und es ist der schnellste und leichteste Weg, sich ein \hyperref[c1iii12]{relatives und ein absolutes Gehör} anzueignen.
Das mentale Spielen ist in der Tat eine mächtige eigenständige Methode, die praktisch jede Ihrer musikalischen Aktivitäten beeinflußt, egal ob am Klavier oder nicht.
Das überrascht nicht, da alles, was man tut, seinen Ursprung im Gehirn hat.
Es verfestigt nicht nur das \hyperref[c1iii6tastatur]{Tastatur-Gedächtnis}, sondern unterstützt auch das \hyperref[c1iii6musik]{Musik-Gedächtnis}, \hyperref[c1iii6foto]{fotografische Gedächtnis}, \hyperref[c1iii14]{Vorspielen}, die \hyperref[c1iii12]{Genauigkeit der Tonhöhe}, \hyperref[c1iii6g]{kalt spielen} usw., und sollte der erste Schritt beim \hyperref[c1iii14d]{musikalischen Spielen} sein.
Seien Sie nicht passiv, warten Sie nicht, bis die Musik aus dem Klavier kommt, sondern sehen Sie die Musik, die Sie erzeugen möchten, aktiv voraus -- das ist die einzige Möglichkeit, eine überzeugende Vorstellung zu geben.
Mit Hilfe des mentalen Spielens haben die großen Genies viel von dem erreicht, was sie vollbrachten, aber wenige Lehrer haben bisher diese Methode gelehrt: Es verwundert wenig, daß so viele Schüler diese Leistungen der großen Pianisten als unerreichbar ansehen.
Wir haben hier gezeigt, daß das mentale Spielen nicht nur erreichbar ist, sondern ein integraler Bestandteil des Klavierspielenlernens sein muß.

Wenn Sie beim Lernen der anderen Methoden dieses Buchs eine Art Erleuchtung hatten, warten Sie, bis Sie das mentale Spielen beherrschen.
Sie werden sich fragen, wie Sie es jemals wagen konnten, etwas öffentlich aufzuführen, ohne es in Gedanken spielen zu können.
Sie haben eine völlig neue Welt betreten und haben Fähigkeiten erworben, die Ihr Publikum in hohem Maß bewundern wird.


\label{c1iii6theorie}

Wir müssen alle danach streben, soviel \textbf{theoretisches Gedächtnis} wie möglich zu benutzen.
Das schließt Elemente wie Tonart, Taktart, Rhythmus, Akkordstruktur, Ausweichungen in andere Tonarten, Harmonien, melodische Struktur usw. ein.
Ein wahres Gedächtnis kann nicht ohne ein Verständnis der theoretischen Basis des jeweiligen Stücks aufgebaut werden. 
Leider erhalten die meisten Klavierschüler nicht genügend theoretische Ausbildung, um eine solche Analyse durchzuführen.
Beginnen Sie mit dem Lernen der chromatischen Tonleiter und des Quintenzirkels (\hyperref[c2_2]{Kapitel 2, Abschnitt 2}).
Jeder kann eine einfache Strukturanalyse durchführen: \hyperref[c1iv4]{Mozarts Wiederholungen}, \hyperref[c1iii20]{einfache parallele Sets in Bachs technischen Stücken}, Konzepte ähnlich der \hyperref[c1iv4Gruppe]{Gruppentheorie in Beethovens Musik}, wiederholtes Benutzen derselben Akkordprogressionen in Chopins Musik usw.
Die älteren Kompositionen liefern einfachere Beispiele, wie die Theorie angewandt oder dagegen verstoßen wird, um Musik zu erzeugen.
Obwohl man darüber streiten kann, ob sich die Qualität der Musik im Laufe der Zeit verbessert oder verschlechtert hat, steht es außer Frage, daß die Theorie sich weiterentwickelt hat.
Beim Spielen zeitgenössischer Musik und nach \enquote{Fake Books} sowie beim Üben in der Improvisation stehen Sie der Theorie von Angesicht zu Angesicht gegenüber und sind gezwungen, die praktischen Grundlagen zu lernen.
Deshalb sollte zu lernen, moderne Musik zu spielen, stets ein Teil des Prozesses das Klavierspielen zu lernen sein und wird eine gute Grundlage und einen Klangvorrat für das Gedächtnis bieten.


\subsubsection{Pflege}
\label{c1iii6k}

\textbf{Es gibt keine effektivere Pflegeprozedur als das \hyperref[c1iii6tastatur]{Tastatur-Gedächtnis und das mentale Spielen}.}
Gewöhnen Sie sich an, bei jeder sich bietenden Gelegenheit in Gedanken zu spielen.
Der Unterschied zwischen einem guten und einem schlechten Auswendiglernenden ist nicht so sehr die \enquote{Auswendiglernstärke}, sondern die geistige Haltung -- was machen Sie mit Ihrem Gehirn während Sie wach sind und während Sie schlafen?
Gute Auswendiglernende haben die Angewohnheit entwickelt, ihr Gedächtnis ständig zu \hyperref[c1iii2]{zirkulieren}.
Wenn Sie das Auswendiglernen üben, müssen Sie deshalb auch Ihren Geist dazu trainieren, immer wieder mit dem Auswendiggelernten zu arbeiten.
Bei schlechten Auswendiglernenden wird das zunächst einen hohen Aufwand erfordern, aber wenn es über einen größeren Zeitraum (Jahre) geübt wird, ist es nicht so schwierig.
Sobald Sie das mentale Spielen gelernt haben, wird diese Aufgabe viel leichter.
Manche Behinderte haben ein Problem mit sich wiederholenden Bewegungen: Ihr Gehirn zirkuliert immer wieder dasselbe.
Das kann eine Erklärung dafür sein, daß sie viele normale Funktionen nicht ausführen können aber über ein unglaubliches Gedächtnis und erstaunliche musikalische Fähigkeiten verfügen, besonders wenn man diese Behinderten im Licht unserer obigen Diskussion über das Gedächtnis und das Spielen in Gedanken betrachtet.

Während der Pflege sollten Sie sich noch einmal die Noten ansehen und die Genauigkeit prüfen, sowohl für die einzelnen Noten als auch für die Ausdrucksbezeichnungen.
Da Sie beim Lernen des Stücks dieselben Notenblätter benutzt haben, ist die Wahrscheinlichkeit hoch, daß Sie, falls Sie beim ersten Lesen der Noten etwas falsch gemacht haben, diesen Fehler später wiederholen und den Fehler niemals bemerken werden. 
Eine Möglichkeit, dieses Problem zu umgehen, ist, sich Aufnahmen anzuhören.
Jeder größere Unterschied zwischen Ihrem Spielen und der Aufnahme wird sich deutlich abheben und im allgemeinen leicht zu erkennen sein.

\textbf{Eine weitere Aufgabe der Pflege ist es, sicherzustellen, daß Sie das Stück noch HS erinnern.}
Das kann bei größeren Stücken eine wahre Last werden, aber das ist es wert, weil Sie nicht während eines Konzerts herausfinden möchten, daß Sie es brauchen.
Beachten Sie, daß diese HS-Pflege-Sitzungen nicht nur dem Gedächtnis dienen.
Das ist die richtige Zeit, neue Dinge zu versuchen, viel schneller als die endgültige Geschwindigkeit zu spielen und Ihre Technik allgemein zu versäubern.
Ausgedehntes HT-Spielen bringt oft Timing- und andere unerwartete Fehler in das Spielen ein, und das ist die Zeit sie zu korrigieren.
\textbf{Deshalb ist HS-Spielen sowohl für die Gedächtnis- als auch für die Technik-Verbesserung eine lohnende Anstrengung.}
Das ist einer der besten Zeitpunkte, ein Metronom dazu zu benutzen, die Genauigkeit des Rhythmus und des Timings sowohl für das Spielen mit HS als auch mit HT zu überprüfen.
Die beste Vorbereitung darauf, nach einem Fehler während des Vorspielens den Anschluß wiederzufinden, ist das HS-Üben und das Spielen in Gedanken.
Dann haben Sie nach dem Fehler oder nach einer Gedächtnisblockade viele Möglichkeiten, den Faden wiederzufinden, wie z.B. mit einer Hand weiterzuspielen oder zunächst das Spielen nur mit einer Hand wieder aufzunehmen und dann die andere hinzuzufügen.
Diese Methode, sich wieder zu fangen, funktioniert deshalb, weil Fehler und Gedächtnisblockaden selten bei beiden Händen gleichzeitig auftreten -- sie treten gewöhnlich nur in einer auf, in der anderen aber nicht, besonders wenn Sie HS geübt hatten.

Zusammengefaßt besteht die Pflege aus folgenden Komponenten:

\begin{enumerate}[label={\arabic*.}] 
\item Prüfen Sie mit dem Notenblatt oder durch Anhören von Aufnahmen die Genauigkeit jeder Note und Ausdrucksbezeichnung.
\item Stellen Sie sicher, daß Sie das ganze Stück HS spielen können.
Sie könnten sehr schnell HS üben, um die Technik aufzupolieren.
\item Üben Sie, an willkürlichen Stellen im Stück anzufangen.
Das ist eine exzellente Art, das Gedächtnis und Ihr Verständnis der Struktur der Komposition zu testen.
Wenn Ihr \hyperref[c1iii6tastatur]{mentales Spielen} gut ist, sollten Sie mit jeder Note beginnen können, nicht nur mit dem Anfang einer Phrase.
\item Stellen Sie fest, ob sie ohne Fehler und Gedächtnisblockaden sehr \hyperref[c1iii6h]{langsam spielen} können.
\item Spielen Sie \enquote{\hyperref[c1iii6g]{kalt}}. Es wird Ihre Fähigkeit zum \hyperref[c1iii14]{Vorspielen} sehr steigern.
\item \hyperref[c1iii6tastatur]{Spielen Sie \enquote{in Gedanken}}, zumindest HS.
Wenn Sie dies von Anfang an tun, wenn Sie das Stück zuerst lernen, und es beibehalten, ist es überraschend einfach.
\end{enumerate} 


