% File: c1iii10

\subsection{Kalte Hände, rutschende Finger, Krankheiten, Handverletzungen, Gehörschäden}
\label{c1iii10}

\subsubsection{Kalte Hände}

Kalte, steife Hände an einem kalten Tag sind ein verbreitetes Leiden, das hauptsächlich durch die natürliche Reaktion des Körpers auf die Kälte verursacht wird.
Ein paar Menschen haben sicherlich pathologische Probleme, die eventuell medizinische Betreuung erfordern, aber die Mehrzahl der Fälle sind natürliche Reaktionen des Körpers auf Unterkühlung.
In diesem Fall zieht der Körper das Blut -- hauptsächlich aus den Extremitäten -- zum Zentrum des Körpers zurück, um den Wärmeverlust gering zu halten.
Die Finger sind für diese Abkühlung am anfälligsten, gefolgt von den Händen und den Füßen.

In solchen Fällen ist die Lösung im Prinzip einfach.
Man muß nur die Körpertemperatur anheben.
In der Praxis ist das oft nicht so leicht.
In einem kalten Raum wird das Problem auch dadurch, daß die Körpertemperatur (durch zusätzliche Bekleidung) so weit erhöht wird, daß einem zu warm ist, nicht immer eliminiert.
Sicherlich sollte jede Methode zur Vermeidung von Wärmeverlusten helfen.
Natürlich ist es am besten, wenn man die Raumtemperatur erhöhen kann.
Wenn nicht, sind allgemeine Hilfen:
\begin{enumerate}[label={\arabic*.}] 
\item die Hände bzw. Arme in warmes Wasser eintauchen
\item ein Heizgerät, wie z.B. ein tragbarer Heizstrahler (ca. 1 kW), den Sie direkt auf den Körper richten können
\item dicke Socken, Pullover oder thermale Unterwäsche
\item Handschuhe ohne Finger (damit Sie mit den Handschuhen Klavier spielen können)
\item \footnote{Wer autogenes Training o. ä. beherrscht, kann es auch mit den dafür erlernten Techniken zur Steuerung der Durchblutung versuchen.}
 \end{enumerate}
Wenn Sie die Hände nur vor dem Spielen warmhalten möchten, sind Fausthandschuhe wahrscheinlich besser als Fingerhandschuhe.
Die meisten Haartrockner haben nicht genügend Energie, sind nicht dafür entwickelt, länger als ungefähr 10 Minuten benutzt zu werden, ohne gefährlich zu überhitzen, und sind für den Zweck, warme Luft um einen Klavierspieler zu erzeugen, zu laut.

Es ist nicht klar, ob es besser ist, die ganze Zeit warm zu bleiben oder nur, wenn man Klavier übt.
Wenn man die ganze Zeit warm bleibt (wie z.B. durch das Tragen thermaler Wäsche), wird der Körper eine Abkühlung eventuell nicht erkennen und deshalb den gewünschten Blutfluß aufrechterhalten.
Auf der anderen Seite wird der Körper eventuell sensibler gegenüber Kälte und schließlich auch, wenn der Körper warm ist, mit kalten Händen darauf reagieren, daß der Raum kalt ist.
Wenn man z.B. immer die fingerlosen Handschuhe trägt, gewöhnen sich die Hände an diese Wärme und fühlen sich sehr kalt an, wenn man die Handschuhe entfernt.
Und der wärmende Effekt dieser Handschuhe geht eventuell nach und nach verloren, wenn sich die Hände daran gewöhnt haben.
Deshalb ist es wahrscheinlich am besten, sie nur beim Üben oder nur vor dem Üben zu tragen.
Das Gegenargument ist, daß sie immer zu tragen es Ihnen gestattet, zu jeder Zeit Klavier zu spielen, ohne Aufwärmen oder die Hände in warmes Wasser tauchen zu müssen.
Das ist sicherlich ein komplexes Problem, und nur Handschuhe zu tragen löst im allgemeinen das Problem nicht und kann es verschlimmern.

\textbf{Die Spielmuskeln befinden sich in den Armen, wenn Sie also die Klaviermuskeln erwärmen möchten, ist es wichtiger, die Unterarme und Ellbogen zu erwärmen als die Finger.}
Es ist sogar jeder Muskel von den Unterarmen bis zur Körpermitte in das Klavierspielen einbezogen.
Deshalb sollten Sie, wenn Sie warmes Wasser benutzen, um die Hände vor einem Auftritt zu erwärmen, versuchen, die Unterarme einzutauchen, besonders die obere Hälfte (in der Nähe der Ellbogen), wo die Beuge- und Streckmuskeln konzentriert sind.
Wenn das nicht möglich ist, dann müssen Sie Ihre Hände lange genug eintauchen, daß das warme Blut von den Händen in die Arme fließen kann.
Die tiefen Hohlhandmuskeln (die Mm. lumbricales unter und die Mm. interossei zwischen den Mittelhandknochen) befinden sich in den Händen, diese müssen deshalb auch erwärmt werden.

Kalte Finger dieser Art sind klar die Reaktion des Körpers auf niedrige Temperaturen.
\textbf{Die beste Lösung mag sein, die Hände mehrmals am Tag in sehr \textit{kaltes} Wasser zu tauchen, um sie an niedrige Temperaturen zu gewöhnen.
Dann reagieren sie vielleicht überhaupt nicht auf Kälte.
Das könnte eine dauerhafte Lösung bieten.}
Sie könnten die Hände z.B. direkt nach dem Üben auf diese Art kühlen, so daß es das Üben nicht stört.
Das Ziel des Kühlens ist, die Haut an kalte Temperaturen anzupassen.
Tauchen Sie die Hände nicht länger als 5 bis 10 Sekunden in kaltes Wasser; kühlen Sie nicht die ganze Hand bis auf die Knochen ab.
Sie könnten sogar zunächst die Hände in warmem Wasser aufwärmen und dann nur die Haut in eiskaltem Wasser kühlen.
Solch eine Behandlung sollte sich gut anfühlen, ohne jeglichen Kälteschock oder Schmerz.
Das ist genau das Prinzip hinter der nordischen Praxis, nach einer heißen Sauna in eine Öffnung in einem gefrorenen See zu springen.
Diese scheinbar masochistische Praxis ist in Wahrheit völlig schmerzlos und hat nützliche Konsequenzen, wie die Haut an kalte Temperaturen anzupassen und die Schweißbildung zu stoppen, die sonst dazu führen würde, daß die Kleidung durchnässen und in der extremen Kälte gefrieren würde.
Ohne in kaltes Wasser zu springen, könnte jemand mit nach der Sauna naßgeschwitzter Kleidung sogar erfrieren!
Die Poren der Haut können geschlossen werden, indem man die Hände nach dem Erwärmen in kaltes Wasser taucht.
So wird das Schwitzen verhindert, und die Wärme bleibt in den Händen gespeichert.
 

\subsubsection{Rutschende (trockene oder schwitzende) Finger}
\label{c1iii10rutschen}

Wenn die Finger übermäßig trocken oder feucht sind, können sie rutschig werden.
Zu häufiges Waschen mit starken Reinigungsmitteln kann die Hände trocken werden lassen.
Die Anwendung der meisten qualitativen Feuchtigkeitslotionen wie z.B. Xxxxxxx wird das Problem lösen.
Um zu vermeiden, daß Sie die Klaviertasten mit der überschüssigen Lotion beschmieren, tragen Sie jeweils nur eine geringe Menge der Lotion auf, und warten Sie, bis sie vollständig in die Haut eingezogen ist, bevor Sie wieder neue auftragen.
Mehrmals kleine Menge aufzutragen hält länger vor als einmal viel aufzutragen.
Wischen Sie vor dem Klavierspielen alle überschüssigen Reste ab.
Menschen, die zum Schwitzen während des Spielens neigen, müssen auch auf rutschige Finger achten.
Wenn Sie zunächst eine Lotion aufgetragen haben, weil Ihre Hände trocken waren, Sie aber während des Spielens zu schwitzen  anfangen, dann können Sie in große Schwierigkeiten geraten, wenn auf den Finger noch überschüssige Lotion ist.
Seien Sie deshalb, wenn Sie zum Schwitzen neigen, mit jeder Art von Lotion vorsichtig.
Sogar ohne Lotion können feuchte oder trockene Finger rutschen.
Üben Sie in diesem Fall, mit \hyperref[c1iii4SchubZug]{Schub- und Zugbewegungen}\index{Schub- und Zugbewegungen} zu spielen, so daß Sie die Position der Finger genauer kontrollieren können.
Diese Bewegungen erfordern ein gewisses Gleiten der Finger über die Tasten und sind deshalb für rutschige Finger besser geeignet.


\subsubsection{Krankheiten}
\label{c1iii10krank}

\textbf{Einige Menschen könnten glauben, daß eine harmlose Krankheit, wie z.B. eine Erkältung, es ihnen immer noch erlaubt, Klavier zu üben.
Schließlich gibt es, während man wegen der Erkältung zuhause bleibt, nichts zu tun, und Klavierspielen wird nicht als anstrengende Tätigkeit angesehen.
Das ist eine schlechte Idee.}
Es ist besonders wichtig für Eltern, zu verstehen, daß Klavierspielen, insbesondere für das Gehirn, eine enorme Anstrengung darstellt, und daß das Klavierspielen im Falle einer Krankheit nicht als ein entspannender Zeitvertreib behandelt werden soll.
Deshalb sollten Kinder, auch bei leichten Erkältungen, nicht zum Klavierüben gezwungen werden, solange sie nicht spielen möchten.
Das Gehirn ist während des Klavierspielens aktiver als die meisten glauben.
Infektionen wirken sich nicht auf den ganzen Körper gleich aus; sie setzen sich gewöhnlich bevorzugt in gestreßten Organen fest.
Wenn man Fieber hat und Klavier spielt, besteht ein gewisses Risiko für eine Schädigung des Gehirns\footnote{sofern das Fieber durch die Anstrengung stark ansteigt}.
Zum Glück verlieren die meisten Menschen die Lust zum Klavierüben bereits, wenn sie nur leicht krank sind, und das ist ein deutliches Signal, daß man nicht üben sollte.

Ob jemand Klavier spielen kann, wenn er krank ist, ist eine persönliche Angelegenheit.
Ob man spielt oder nicht, ist für den Klavierspieler ziemlich klar; die meisten Menschen fühlen den Streß des Klavierspielens bereits, bevor die Symptome der Krankheit deutlich werden.
Deshalb ist es wahrscheinlich am sichersten, die Entscheidung, zu spielen oder nicht zu spielen, dem Klavierspieler zu überlassen.
\textbf{Es ist nützlich, zu wissen, daß wenn Sie sich plötzlich müde fühlen oder andere Symptome spüren, die das Spielen erschweren, es ein Anzeichen dafür sein kann, daß Sie krank werden.}
Das Problem damit, während einer Krankheit nicht zu spielen, ist, daß die Hände einen beträchtlichen Teil der Technik verlieren, wenn die Krankheit länger als eine Woche dauert.
Vielleicht sind Übungen, die das Gehirn nicht belasten, wie z.B. \hyperref[c1iii5]{Tonleitern}\index{Tonleitern}, \hyperref[Arpeggios]{Arpeggios}\index{Arpeggios} und Übungen der \hyperref[c1iii7h]{Hanon}\index{Hanon}-Art, in einer solchen Situation geeignet.
 

\subsubsection{Gesundes und ungesundes Üben}
\label{c1iii10ungesund}

Es ist wichtig, etwas über die Auswirkungen des Klavierübens auf die Gesundheit zu lernen, da jede Tätigkeit auf gesunde oder ungesunde Weise ausgeübt werden kann.
Ein streßfreies, psychologisch gesundes Herangehen an das Klavierüben kann die Gesundheit einer Person stärken, hingegen kann es ungesund sein, ohne Beachtung des Wohlbefindens zu üben.
Es ist wichtig, das richtige Atmen zu lernen, um einen Sauerstoffmangel zu vermeiden.
Aus der Unfähigkeit zum \hyperref[c1iii6]{Auswendiglernen}\index{Auswendiglernen} oder zum Erwerben bestimmter Fertigkeiten resultierende Frustrationen müssen durch das Lernen effizienter Übungsmethoden verhindert werden.
In diesem Buch werden Methoden zum \hyperref[c1ii14]{Vermeiden von Ermüdung}\index{Vermeiden von Ermüdung} besprochen.
\hyperref[c1iii10hand]{Verletzungen der Hand}\index{Verletzungen der Hand} sind vermeidbar.
Übermäßige \hyperref[c1iii15]{Nervosität}\index{Nervosität} ist nicht nur schlecht für das \hyperref[c1iii14]{Auftreten}\index{Auftreten}, sondern auch für die Gesundheit.
Wir müssen die richtigen Beziehungen zwischen den Schülern, Lehrern, Eltern und dem Publikum bedenken oder durch die Erfahrungen lernen.
Indem man die gesundheitlichen Aspekte beachtet, kann das Klavierüben zu einer nützlichen Aktivität werden, die genauso wirksam ist, wie die richtige Ernährung und das richtige Training.


\subsubsection{Verletzungen der Hand}
\label{c1iii10hand}

Handverletzungen sind ungefähr bis zur Mittelstufe für die Schüler im allgemeinen kein großes Problem.
Für fortgeschrittene Klavierspieler sind sie ein wichtiges Thema, weil die menschliche Hand nicht dafür gedacht ist, solchen extremen Belastungen standzuhalten.
Verletzungsbedingte Probleme sind bei professionellen Pianisten denen von professionellen Sportlern z.B. im Tennis, Golf oder Fußball ähnlich.
\textbf{Deshalb sind nach der zum Üben zur Verfügung stehenden Zeit die Einschränkungen durch mögliche Verletzungen vielleicht die zweitwichtigsten.}
Es mag so erscheinen, als ob keine Verletzungen auftreten sollten, da die \hyperref[c1ii14]{Entspannung}\index{Entspannung} eine wichtige Komponente der Klaviertechnik ist.
Leider sind die körperlichen Anforderungen des Spielens auf fortgeschrittenen Stufen so hoch, daß (wie im Sport) Verletzungen trotz der bekannten Vorsichtsmaßnahmen und anderer Mittel, die professionelle Spieler anwenden, sehr wohl auftreten können.
Verletzungen treten häufig beim Üben zum Erwerb schwieriger Techniken auf.
Schüler, die die Methoden dieses Buchs benutzen, müssen sich der Möglichkeit der Verletzung besonders bewußt sein, weil sie schnell damit anfangen werden, Material zu üben, das hohe technische Fertigkeiten verlangt.
Deshalb ist es wichtig, die verbreiteten Arten von Verletzungen zu kennen, und zu wissen, wie man sie vermeidet.

\textbf{Jede Verletzung hat eine Ursache.}
Obwohl es eine Vielzahl dokumentierter Berichte über Verletzungen und Erfolge bzw. Fehlschläge von Heilanwendungen gibt, sind definitive Informationen über Ursachen und Heilung schwer zu finden.
Die einzigen Heilmittel, die erwähnt werden, sind Ruhe und schrittweises Zurückkehren zum Spielen mit streßfreien Methoden.
Ich verletzte mir z.B. die Beugesehnen meiner linken Hand durch die Benutzung von Golfschlägern mit abgenutzten, harten Griffen, obwohl ich immer Golfhandschuhe trug.
Mein Orthopäde diagnostizierte sofort die Ursache meiner Schmerzen (eine Kerbe in meinen Sehnen), konnte mir aber nicht sagen, wie ich meine Hand verletzt hatte, so daß er mir nicht richtig sagen konnte, wie man es heilt.
Ich fand später heraus, daß der Druck des Golfgriffs die Kerben in meinen Sehnen erzeugt hatte, und diese Kerben bewegten sich in meiner Hand während des Klavierspielens auf und ab; die daraus resultierende Reibung erzeugte nach langen Übungseinheiten am Klavier Entzündungen und Schmerzen.
Der Arzt zeigte mir, wie man diese Kerben fühlen kann, wenn man auf die Sehne drückt und den Finger bewegt.
Nun ersetze ich die Griffe meiner Schläger häufiger und habe in meinen Golfhandschuh Polster eingesetzt (aus Xxxxxxxxs selbstklebenden Fußpolstern geschnitten\footnote{Ich will hier keine Schleichwerbung für eine bestimmte Marke machen und habe auch keine Werbeverträge. Wer es also genau wissen möchte, den verweise ich auf den Originaltext.}), und mein Problem ist gelöst.
Das jahrelange zu feste Greifen des Schlägers (ich wußte damals noch nichts über \hyperref[c1ii14]{Entspannung}\index{Entspannung}) führte jedoch an meinen Händen zu einem dauerhaften Schaden, so daß meine Finger nicht so unabhängig sind, wie ich es gerne hätte.

Man kann sich versehentlich bestimmte Muskeln oder Sehnen zerren, besonders in den Schultern und im Rücken.
Das wird meistens durch ein schlechtes Ausrichten der Hände oder des Körpers und durch nicht ausbalanciertes Spielen verursacht.
Das beste Vorgehen ist hier Vorsicht -- Klavierspieler müssen besonders vorsichtig sein und solche Verletzungen vermeiden, weil es Jahre dauern kann, bis sie geheilt sind.
Hören Sie mit dem Üben auf, sobald Sie einen Schmerz spüren.
Ein paar Tage Pause werden Ihrer Technik nicht schaden und ernste Verletzungen vermeiden.
Natürlich ist es am besten, zu einem Orthopäden zu gehen; viele Orthopäden sind jedoch mit Verletzungen durch Klavierspielen nicht vertraut.

Fingerspitzen können durch zu hartes (lautes) Spielen verletzt werden.
Dieser Zustand kann mit geeigneter Bandage etwas gemildert werden.
\textbf{Die \hyperref[c1ii2]{gebogene Fingerhaltung}\index{gebogene Fingerhaltung} kann Prellungen der Fingerspitzen verursachen, weil das Polster zwischen Knochen und Haut an der Spitze minimal ist.}
Bei der gebogenen Haltung kann es auch passieren, daß sich das Fleisch unter dem Fingernagel von diesem löst, wenn man die Fingernägel zu kurz schneidet.
Sie können beide Arten der Verletzung vermeiden, indem Sie die \hyperref[c1iii4b]{flachen Fingerhaltungen}\index{flachen Fingerhaltungen} benutzen (s. Abschnitt III.4b).

Die meisten Handverletzungen sind vom Typ der Verletzungen durch wiederholten Streß (RSI = repetitive stress injury).
Das Karpaltunnelsyndrom und Sehnenentzündungen sind verbreitete Leiden.
Erlebnisberichte legen nahe, daß chirurgische Eingriffe im allgemeinen das Problem des Karpaltunnelsyndroms nicht lösen und mehr schaden als nutzen können.
Hinzu kommt, daß chirurgische Eingriffe irreversibel sind.
Zum Glück haben Masseure vor kurzem das Problem gelöst, das Karpaltunnelsyndrom zu heilen.
Warum Masseure?
Weil sowohl Pianisten als auch Masseure Ihre Finger als ihr hauptsächliches berufliches Werkzeug benutzen.
Deshalb leiden sie unter den gleichen Verletzungen.
Masseure sind jedoch eher in der Lage, zu experimentieren und Heilmethoden zu entdecken, während Pianisten nicht medizinisch ausgebildet sind und nicht einmal wissen, wie sie ihre Gebrechen diagnostizieren sollen.
Es ist jedoch glücklicherweise so, daß man Schmerzen bereits lange bevor ein irreversibler Schaden auftreten kann spürt, so daß das Syndrom geheilt werden kann, wenn man es behandelt, sobald man Schmerzen fühlt.
Obwohl man die Schmerzen üblicherweise in der Nähe der Handgelenke spürt, liegt die Ursache der Schmerzen nicht in den Handgelenken, sondern hauptsächlich in den Armen und im Nacken, wo große Muskeln und Sehnen schädliche Kräfte auf die Sehnen ausüben können, die durch den querliegenden Sehnenring des Handgelenks verlaufen, der alle zu den Fingern führende Sehnen bündelt.
Deshalb beseitigt eine Behandlung des Handgelenks nicht die Schmerzen und verschlimmert eine Operation des Handgelenks nur das Problem.
Die Gruppe mit den fortgeschrittensten Methoden zur Behandlung des Karpaltunnelsyndroms sind die Spezialisten für SET-Massage (Structural Energetic Therapy	extregistered); Sie beginnen mit dem Schädel und gehen dann zu einer Behandlung der tieferen Gewebeschichten der entsprechenden Bereiche des Kopfes, der Arme und des Körpers über.
Das Einbeziehen des Schädels ist notwendig, weil es am schnellsten Erleichterung verschafft und die Behandlung des Gewebes alleine das Problem nicht beseitigt.
Bevor man eine Behandlung bekommt, ist es schwer vorstellbar, daß die Schädelknochen eine Auswirkung auf das Karpaltunnelsyndrom haben.
Mehr Informationen dazu finden Sie auf der Website von SET (www.structuralenergetictherapy.com).
Obwohl diese Website für Masseure gedacht ist, können Sie erfahren, was in die Behandlung des Karpaltunnelsyndroms einbezogen wird, bis zu welchem Ausmaß es heilbar ist und wie Sie den richtigen Therapeuten finden.
Bis jetzt sind wenige Therapeuten in diesem Verfahren ausgebildet, aber zumindest können Sie mit den Experten Kontakt aufnehmen und Ihr Problem diskutieren.
Es gibt einen einfachen Test für fortgeschrittene Fälle des Karpaltunnelsyndroms.
Stellen Sie sich vor einen Spiegel, und lassen Sie die Arme völlig entspannt in ihrer \enquote{natürlichen} Haltung gerade herunterhängen.
Wenn die Daumen dem Spiegel am nächsten sind, dann ist alles in Ordnung.
Wenn man mehr Knöchel sieht (die Arme sind einwärts gedreht), dann haben Sie ein fortgeschritteneres Karpaltunnelsyndrom.
Auch sollte der Körper aufrecht sein.
Praktisch niemand hat eine perfekt aufrechte Haltung, und es kann notwendig sein, eine eventuelle ungenügende Haltung zu korrigieren, um das Karpaltunnelsyndrom vollständig zu behandeln.
Athleten wie Golfer und Tennisspieler sind eine Ausnahme, weil ihre asymmetrischen Spielbewegungen zu asymmetrischen Veränderungen der Knochendichte und Knochenstruktur führen.
Rechtshändige Golfer haben in ihrer rechten Hüfte eine höhere Knochendichte.

Methoden zur Streßreduzierung beim Klavierüben, wie z.B. die Methoden  von Taubman, Alexander und Feldenkrais können sowohl für die  Vermeidung von Verletzungen als auch für die Erholung von Verletzungen wirksam sein.
Im allgemeinen ist es das beste, den spielenden Finger (außer den Daumen) soviel wie möglich in einer Linie mit dem Unterarm zu halten, um Verletzungen durch wiederholten Streß zu vermeiden.
Natürlich ist die beste Vorbeugungsmaßnahme, nicht zuviel mit  Streß zu üben.
Die HS-Methode ist besonders nützlich, weil der Streß minimiert wird  und jede Hand zur Ruhe kommt, bevor ein Schaden auftreten kann.
Der Ansatz \enquote{ohne Schmerzen kein Erfolg} ist extrem schädlich.
Klavierspielen kann eine enorme Anstrengung erfordern, aber es darf niemals  schmerzhaft sein.
Sehen Sie dazu im \hyperref[Websites]{Quellenverzeichnis}  einige informative Websites über Handverletzungen bei Klavierspielern.


\subsubsection{Gehörschäden}
\label{c1iii10gehoer}

\textbf{Gehörschäden treten im allgemeinen altersbedingt auf; der Gehörverlust kann bereits im Alter von 40 Jahren beginnen, und mit 70 haben die meisten Menschen etwas von ihrer Hörfähigkeit verloren.}
Gehörverlust kann entstehen, wenn man hohen Lautstärken zu häufig ausgesetzt ist oder durch Infektionen und andere pathologische Ursachen.
Man verliert das Gehör meistens zuerst im unteren oder im oberen Frequenzbereich.
Das wird oft von einem Tinnitus (Pfeifen oder Klingeln im Ohr) begleitet.
Diejenigen, die das Gehör bei den niedrigen Frequenzen verlieren, neigen zu einem tiefen, tosenden oder pochenden Tinnitus, und diejenigen, die das Gehör bei den hohen Frequenzen verlieren, neigen dazu, ein hochtönendes Pfeifen zu hören.
Tinnitus kann durch ein ungewolltes Feuern der Hörnerven im beschädigten Abschnitt des Ohres verursacht werden; es gibt jedoch viele weitere Ursachen.
Im \hyperref[Websites]{Quellenverzeichnis} finden Sie Internet-Adressen zum Thema Gehörschäden.

Obwohl schwere Fälle von Gehörverlust von einem HNO-Arzt oder Hörgeräteakustiker leicht diagnostiziert werden können, sind die Ursachen und die Möglichkeiten zur Verhütung von Schäden noch nicht völlig bekannt.
Leichte Fälle von Gehörverlust sind auch für Fachleute schwer zu diagnostizieren, weil das menschliche Gehirn versucht, diese Verluste auszugleichen, indem die internen Mechanismen zur Tonverstärkung heraufgesetzt werden.
Diejenigen mit leichtem Gehörverlust haben z.B. Schwierigkeiten, Unterhaltungen zu verstehen, sind aber gegenüber lauten Geräuschen sehr empfindlich -- sogar etwas laute Geräusche, die andere Menschen nicht stören, können schmerzhaft laut sein -- einfache Hörtests würden zeigen, daß diese Menschen ein empfindliches Gehör haben.
Es gibt keine Methode, einen Tinnitus zu diagnostizieren, außer anhand der Beschreibungen des Patienten.
Für die Tests und die Behandlung muß man sich an einen HNO-Spezialisten wenden.
In nicht durch Krankheit bedingten Fällen werden Schäden im  allgemeinen dadurch verursacht, daß jemand lauten  Geräuschen ausgesetzt ist.
Trotzdem leiden viele Menschen, die sehr lauten Geräuschen ausgesetzt sind, wie z.B. Pianisten, die täglich mehrere Stunden auf Konzertflügeln spielen, Klavierstimmer, die routinemäßig während des Stimmens auf das Klavier \enquote{einhämmern} oder Mitglieder von Rockbands, nicht unter Gehörverlust.
Auf der anderen Seite können einige, die weniger Geräuschen ausgesetzt sind, ihr Gehör verlieren, besonders im Alter.
Deshalb gibt es große Unterschiede in der Anfälligkeit für Gehörverlust.
Es besteht jedoch eine Tendenz, daß Menschen, die lauteren Geräuschen ausgesetzt sind, mehr unter Gehörverlusten leiden.
\textbf{Es ist ziemlich wahrscheinlich, daß Gehörverluste bei Pianisten und Klavierstimmern (sowie bei Mitgliedern von Rockbands usw. und Menschen, die ständig sehr laute Musik hören) viel verbreiteter sind als allgemein bekannt ist, weil die meisten Fälle nicht dokumentiert werden.}

Ein Tinnitus ist im Grunde bei allen Menschen die ganze Zeit vorhanden, ist aber bei den meisten Menschen so leise, daß sie ihn, außer in schalldichten Räumen, nicht hören können.
Er kann durch ein spontanes Feuern der Hörnerven bei Abwesenheit eines genügend großen Reizes ausgelöst werden, d.h. der menschliche Hörapparat \enquote{dreht automatisch die Verstärkung auf}, wenn es kein Geräusch gibt.
Vollständig zerstörte Regionen erzeugen keinen Laut, weil der Schaden so ernsthaft ist, daß sie nicht mehr funktionieren.
Teilweise zerstörte Regionen erzeugen offenbar einen Tinnitus, weil sie geschädigt genug sind, daß sie fast kein Umgebungsgeräusch mehr wahrnehmen; diese Stille führt dazu, daß das Gehirn die Verstärkung aufdreht und die Detektoren abfeuert, oder das System entwickelt eine Fehlerstelle in der Weiterleitung des Schallsignals.
Diese Detektoren sind entweder piezo-elektrisches Material an der Basis von Haaren in der Gehörschnecke (Cochlea) oder Ionenkanäle, die durch an diesen Haaren befindliche Moleküle geöffnet und geschlossen werden -- zu diesem Thema gibt es in der Literatur widersprüchliche Angaben.
Es gibt selbstverständlich viele weitere Ursachen von Tinnitus, und einige könnten ihren Ursprung sogar im Gehirn haben.
Ein Tinnitus ist fast immer ein Zeichen eines einsetzenden Gehörverlusts.

Für diejenigen, die keinen hörbaren Tinnitus haben, besteht wahrscheinlich -- innerhalb vernünftiger Grenzen -- keine Notwendigkeit, laute Musik zu meiden.
Deshalb sollte das Klavierüben bis zu einem Alter von 25 Jahren mit jeder Lautstärke unschädlich sein.
Diejenigen, die bereits einen Tinnitus haben, sollten laute Klaviermusik meiden.
\textbf{Tinnitus \enquote{beschleicht} einen jedoch üblicherweise, so daß das Einsetzen des Tinnitus oftmals unentdeckt bleibt, bis es zu spät ist.
Deshalb sollte jeder über Tinnitus Bescheid wissen und ab 40 während des Klavierübens einen Gehörschutz tragen.
Gehörschutz ist für die meisten Klavierspieler eine abscheuliche Vorstellung, aber wenn man die Konsequenzen bedenkt (s.u.), lohnt es sich absolut.}
Bevor Sie einen Gehörschutz tragen, tun Sie alles, um die Lautstärke zu verringern, wie den Raum schallarm machen (Teppiche auf harten Böden auslegen usw.), den Deckel eines Flügels schließen und im allgemeinen leise üben (auch laute Passagen -- was sowieso eine gute Idee ist, auch ohne die Möglichkeit eines Gehörschadens).

Einen Gehörschutz kann man leicht in Baumärkten kaufen, weil viele Arbeiter, die Baumaschinen oder Gartengeräte benutzen, einen Gehörschutz benötigen.
Für Klavierspieler reicht ein preisgünstiger Schutz, weil sie noch etwas von der Musik hören müssen.
Sie können auch die meisten größeren Audio-Kopfhörer benutzen.
Kommerzielle Schützer umschließen das Ohr völlig und isolieren den Schall besser.
Da die heute verfügbaren Schützer nicht für Klavierspieler entwickelt wurden, haben sie keinen gleichmäßigen Frequenzgang; d.h., der Klang des Klaviers wird verändert.
Das menschliche Ohr kann sich jedoch gut an verschiedene Arten von Klängen anpassen und wird sich sehr schnell an den neuen Klang gewöhnen.
Das Klavier wird auch ziemlich anders klingen, wenn Sie den Gehörschutz entfernen (was Sie hin und wieder tun müssen, damit Sie wissen, wie der wahre Klang ist).
Diese verschiedenen Klänge können uns lehren, wie das Gehirn Einfluß darauf nimmt, welche Klänge man hört und welche nicht und wie unterschiedlich verschiedene Menschen dieselben Klänge interpretieren.
\textbf{Es lohnt sich, einen Gehörschutz auszuprobieren, so daß man diese verschiedenen Klänge erfahren kann.
Sie werden z.B. feststellen, daß das Klavier viele fremdartige Klänge erzeugt, die Sie nie zuvor gehört haben!}
Die Unterschiede im Klang sind so erstaunlich und komplex, daß man sie nicht in Worte fassen kann.
Bei Klavieren geringerer Qualität führt das Benutzen eines Gehörschutzes dazu, daß der Klang eines höherwertigen Instruments simuliert wird, weil die unerwünschten hohen Obertöne und zusätzliche Geräusche herausgefiltert werden.

Das Gehirn verarbeitet automatisch alle eingehenden Informationen, ob Sie es wollen oder nicht.
Das ist natürlich ein Teil dessen, was Musik ist -- sie ist die Interpretation der hereinkommenden Klänge durch das Gehirn, und der größte Teil unserer Reaktion auf die Musik geschieht automatisch.
Wenn man einen Gehörschutz trägt, verschwindet deshalb das meiste dieses Reizes, und ein großer Anteil der Verarbeitungskapazität des Gehirns wird für andere Aufgaben frei.
Insbesondere haben Sie nun mehr Mittel zur Verfügung, die Sie für das HS-Üben verwenden können.
Schließlich üben Sie deshalb HS und nicht HT -- so daß Sie mehr Energie auf die schwierige Aufgabe, mit dieser einen Hand zu spielen, verwenden können.
Deshalb werden Sie eventuell feststellen, daß Sie HS schnellere Fortschritte machen, wenn Sie einen Gehörschutz tragen!
Aus dem gleichen Grund schließen viele Pianisten ihre Augen, wenn sie etwas mit einem hohen emotionalen Gehalt spielen möchten -- sie brauchen alle verfügbaren Kräfte, um das hohe Maß an Emotion zu erzeugen.
Wenn man die Augen schließt, eliminiert man eine enorme Menge an Informationen, die in das Gehirn strömt, weil das Sehen eine zweidimensionale, vielfarbige, bewegte Quelle eines Datenstroms mit hoher Bandbreite ist, der sofort und automatisch auf viele komplexe Arten interpretiert werden muß.
Obwohl das Publikum meistens bewundert, daß ein Pianist mit geschlossenen Augen spielen kann, ist es in Wahrheit einfacher.
\textbf{Darum werden in naher Zukunft wahrscheinlich die meisten Klavierschüler einen Gehörschutz tragen, so wie heutzutage viele Athleten und Bauarbeiter einen Helm tragen.}
Es macht für niemanden von uns einen Sinn, die letzten 10, 30 oder mehr Jahre unseres Lebens ohne Gehör zu verbringen.

Wie schädigt der Klavierklang das Gehör?
Sicherlich sind laute Töne mit vielen Noten am schädlichsten.
Deshalb ist es wahrscheinlich kein Zufall, daß Beethoven vorzeitig taub wurde.
Das ermahnt uns auch, beim Üben seiner Musik an den Schutz des Gehörs zu denken.
Der Typ des Klaviers ist auch wichtig.
\textbf{Die meisten \enquote{\hyperref[upright]{Aufrechten}\index{Aufrechten}}, die keinen ausreichenden Klang erzeugen, sind wahrscheinlich am wenigsten schädlich.
Große Flügel, die die Energie effizient auf die Saiten übertragen und den Ton lange aushalten, verursachen wahrscheinlich nicht so große Schäden wie Klaviere mittlerer Qualität, bei denen im Moment des ersten Schlags beim Auftreffen des Hammers auf den Saiten eine große Energiemenge übertragen wird.}
Obwohl ein großer Teil dieser schädigenden Tonenergie wahrscheinlich nicht im hörbaren Bereich liegt, können wir sie als unangenehmen oder schrillen Klang erkennen.
Deshalb sind Flügel mittlerer Größe (6 bis 7 ft; ca. 1,80 bis 2,10 m) eventuell am schädlichsten.
In dieser Hinsicht ist der Zustand der Hämmer wichtig, da ein abgenutzter Hammer einen viel lauteren Anschlagsklang als ein neuer oder richtig intonierter Hammer erzeugen kann.
Deshalb verursachen abgenutzte Hämmer öfter einen Saitenbruch als neue bzw. gut intonierte Hämmer.
Mit alten, verhärteten Hämmern können wahrscheinlich die meisten Klaviere das Gehör schädigen.
Deshalb ist das richtige, regelmäßige \hyperref[c2_7_hamm]{Intonieren der Hämmer}\index{Intonieren der Hämmer} für das musikalische Spielen, die technische Entwicklung und den Schutz des Gehörs viel wichtiger als vielen Menschen bewußt ist.
Wenn Sie den Deckel eines Flügels schließen müssen, um leise zu spielen oder den Ton auf ein erträgliches Maß zu reduzieren, dann müssen die Hämmer wahrscheinlich intoniert werden.

Einige der lautesten Geräusche werden von jenen kleinen Ohrhörern erzeugt, die man zum Musikhören benutzt.
Eltern sollten ihre Kinder davor warnen, die Lautstärke ständig weit aufzudrehen, besonders wenn sie Fans von Musik sind, die sehr laut gespielt wird.
Einige Kinder schlafen mit lärmenden Ohrhörern ein; das kann sehr schädlich sein, weil die Zeit, die man der Lautstärke ausgesetzt ist, ebenfalls wichtig ist.
\textbf{Es ist eine schlechte Idee, Kindern Geräte mit solchen Ohrhörern zu geben -- zögern Sie es so lange wie möglich hinaus.}
Früher oder später bekommen sie jedoch eins; in diesem Fall sollten Sie sie warnen, \textit{bevor} sie Gehörschäden bekommen.

Außer in einigen besonderen Fällen von Tinnitus (besonders jene, in denen man den Klang durch Bewegen des Kiefers usw. verändern kann), gibt es bisher kein Heilverfahren.
Große Dosen von Aspirin können Tinnitus verursachen; in diesem Fall kann die Einnahme zu beenden den Prozeß manchmal umkehren.
Kleine Mengen von Aspirin, die wegen Problemen mit dem Herzen genommen werden (81mg), verursachen offensichtlich keinen Tinnitus, und es wird in der Literatur manchmal behauptet, daß diese kleinen Mengen vielleicht das Einsetzen des Tinnitus verzögern.
Ein lauter Tinnitus kann sehr anstrengend sein, weil er nicht verändert werden kann, ständig vorhanden ist und mit der Zeit immer schlimmer wird.
Viele, die darunter leiden, haben schon an Selbstmord gedacht.
Obwohl es keine Heilung gibt, ist eine Abhilfe möglich, und alles deutet darauf hin, daß man irgendwann in der Lage sein sollte, Möglichkeiten zur Heilung zu finden.
Es gibt Hörhilfen, die die Wahrnehmung des Tinnitus reduzieren, z.B. indem sie ein Geräusch abgeben, so daß der Tinnitus entweder maskiert oder die Verstärkung im geschädigten Bereich reduziert wird.
Deshalb kann für diejenigen, die unter Tinnitus leiden, absolute Stille schädlich sein.

Eine der ärgerlichsten Eigenschaften des Gehörverlusts ist nicht, daß das Gehör seine Empfindlichkeit verloren hätte (oft zeigen Tests der Empfindlichkeit nur sehr geringe Verluste), sondern die Unfähigkeit der Person, die Geräusche richtig zu verarbeiten, so daß man ein Gespräch verstehen kann.
Menschen mit normalem Gehör können Sprache verstehen, die mit vielen zusätzlichen Geräuschen vermischt ist.
\textbf{Sprache zu verstehen ist im allgemeinen die erste Fähigkeit, die man mit dem Einsetzen des Gehörverlusts verliert.}
Moderne Hörgeräte können sehr hilfreich sein, indem sie sowohl die Frequenzen verstärken, die notwendig sind, um Sprache zu verstehen, als auch Geräusche unterdrücken, die laut genug sind, um Schäden zu verursachen.
Mit anderen Worten: Wenn das Hörgerät lediglich alle Geräusche verstärkt, kann es sogar einen größeren Schaden verursachen.
Ein weiteres Vorgehen gegen den Tinnitus ist, das Gehirn darauf zu trainieren, den Tinnitus zu ignorieren.
Das Gehirn kann erstaunlich gut trainiert werden, und ein Teil des Grunds, warum Tinnitus Leid verursachen kann, ist eine unangemessene Reaktion des Gehirns.
\textbf{Das Gehirn hat die Fähigkeit, sich entweder auf das Geräusch zu konzentrieren und Sie verrückt zu machen, oder das Geräusch zu ignorieren, so daß Sie es nicht hören, solange Sie nicht daran erinnert werden.}
Das beste Beispiel für diesen Effekt ist das Metronom.
Die meisten Klavierspieler wissen nicht, daß ihnen das Gehirn, wenn sie zu lange mit dem Metronom üben, einen Streich spielt und man das Klicken entweder überhaupt nicht mehr hört oder zur falschen Zeit, besonders wenn das Klicken hoch und laut ist.
Das ist ein Grund, warum moderne Metronome blinkende Lichter haben; es versetzt Sie nicht nur in die Lage, ohne Ton den richtigen Rhythmus zu halten, sondern Sie können auch prüfen, ob das, was Sie hören, mit den blinkenden Lichtern übereinstimmt.
Deshalb beginnen moderne Behandlungen des Tinnitus damit, dem Patienten beizubringen, daß andere bereits gut und mit minimalen Beschwerden damit zurechtkommen.
Danach erhält der Patient ein Gehörtraining, so daß er in der Lage ist, den Tinnitus zu ignorieren.
Zum Glück kann das Gehirn besonders leicht lernen, ein konstantes Geräusch zu ignorieren, das immer vorhanden ist.

Wenn Sie genug Berichte über das Leiden an Tinnitus gelesen haben, werden Sie wahrscheinlich dem Rat folgen, ab 40 einen Gehörschutz zu tragen, zumindest wenn Sie längere Zeit laute Passagen üben.
Bei den ersten Anzeichen von Tinnitus ist es dringend erforderlich, daß Sie etwas für den Schutz Ihres Gehörs tun, denn wenn der Tinnitus eingesetzt hat, kann eine Verschlechterung des Gehörs, wenn es lauten Geräuschen ausgesetzt ist, schnell mit einer hohen jährlichen Rate fortschreiten.
Ein Digital-Piano zu benutzen und die Lautstärke herunterzudrehen ist eine sehr gute Lösung.
Gehen Sie sofort zu einem HNO-Arzt, der möglichst auf die Behandlung von Tinnitus spezialisiert ist.
Der Schutz des Gehörs ist auch für die anderen Mitglieder des Haushalts wichtig; wenn es möglich ist, isolieren Sie deshalb den Raum, in dem das Klavier steht, akustisch vom Rest des Hauses.
Die meisten Qualitätstüren (aus Glas) werden genügen.
Es gibt ein paar Kräuter und \enquote{natürliche} Medikamente, die eine Wirkung gegen den Tinnitus versprechen.
Die meisten davon wirken nicht, und jene, die scheinbar einigen Menschen nützen, haben gefährliche Nebenwirkungen.
Obwohl es wahr ist, daß es herzlich wenig Spezialisten gibt, die Tinnitus behandeln, verbessert sich die Situation rasch, und es gibt nun viele Seiten im Internet mit Informationen zum Tinnitus, wie z.B. von der American Tinnitus Association.



