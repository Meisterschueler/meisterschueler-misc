% File: c1iii15

\section{Ursachen und Kontrolle von Nervosität}
\label{c1iii15}
 
\textbf{Nervosität ist ein natürliches menschliches Gefühl wie Glücklichsein, Angst, Trauer usw.}
Nervosität entsteht aus der geistigen Wahrnehmung einer Situation, in der die Leistung entscheidend ist.
\textbf{Deshalb ist die Nervosität, wie alle Gefühle, eine leistungssteigernde Reaktion auf eine Situation.}
Glücklichsein fühlt sich gut an, weshalb wir versuchen, glückliche Situationen zu erzeugen, die uns helfen; Furcht hilft uns, Gefahren zu entfliehen, und Traurigkeit bringt uns dazu, schmerzliche Situationen zu vermeiden, was dazu führt, daß wir unsere Chancen zu überleben verbessern.
Nervosität läßt uns all unsere Energien auf die anstehende kritische Aufgabe konzentrieren und ist deshalb ein weiteres nützliches Überlebenswerkzeug der Evolution.
Die meisten Menschen haben eine Abneigung gegen die Nervosität, weil sie zu häufig von der Furcht zu versagen begleitet oder verursacht wird.
Obwohl die Nervosität für eine große Leistung erforderlich ist, muß man sie deshalb unter Kontrolle halten; man darf nicht zulassen, daß sie den Auftritt stört.
Die Geschichte der großen Künstler ist voller Berichte sowohl von sehr nervösen als auch von überhaupt nicht nervösen Künstlern.
Das zeigt, daß die bisherigen wissenschaftlichen, medizinischen oder psychologischen Untersuchungen des Phänomens  zu keinen praktischen Ergebnissen führten - sogar auf Konservatorien, bei denen das ein wichtiger Bestandteil des Lehrplans sein sollte.

Gefühle sind grundlegende, primitive, animalische Reaktionen, so etwas wie Instinkt, und sind nicht völlig rational.
Unter normalen Umständen leiten die Gefühle unsere täglichen Aktionen recht ordentlich.
\textbf{Unter extremen Bedingungen können die Gefühle jedoch außer Kontrolle geraten, und sie können dann zu einer Belastung werden.}
Klar, Gefühle waren dazu gedacht, nur unter normalen Umständen zu funktionieren.
So läßt z.B. die Furcht den Frosch flüchten, lange bevor ein Raubtier ihn fangen kann.
Wenn er jedoch in die Enge getrieben wird, erstarrt der Frosch vor Angst, und das läßt ihn für die Schlange zu einer leichteren Beute werden, als wenn ihn die überwältigende Furcht nicht gelähmt hätte.
Ebenso \textbf{ist die Nervosität normalerweise gemäßigt und hilft uns, eine wichtige Aufgabe besser zu bewältigen, als wenn wir gleichgültig wären.}
Unter extremen Bedingungen kann sie jedoch schlagartig außer Kontrolle geraten und unsere Leistung behindern.
Die Anforderung, ein schwieriges Pianosolo vor einem großen Publikum fehlerfrei aufzuführen, kann berechtigtermaßen als extreme Situation bezeichnet werden.
Es ist keine Überraschung, daß die Nervosität außer Kontrolle geraten kann, solange unser Name nicht Wolfgang oder Franz ist (für Frederic traf das offensichtlich nicht zu, da er ein Nervenbündel war und öffentliche Aufführungen nicht ausstehen konnte; in einem Salon fühlte er sich jedoch anscheinend wohler).
Obwohl Geigenspieler ebenfalls nervös werden, gerät dies jedoch nicht außer Kontrolle, wenn sie in einem Orchester spielen, weil die Bedingungen nicht so extrem wie bei Soloauftritten sind.
Kinder, die zuviel Angst davor haben, solo aufzutreten, haben fast immer Spaß daran, in einer Gruppe aufzutreten.
Das zeigt die vorrangige Wichtigkeit der mentalen Wahrnehmung der Situation.

\textbf{Klar ist der Weg, Nervosität zu kontrollieren, zunächst ihre Ursachen und ihre Form zu untersuchen und dann Methoden zu ihrer Kontrolle zu entwickeln, die auf diesem Wissen basieren.}
Da sie ein Gefühl ist, wird jede Methode zur Kontrolle von Gefühlen funktionieren.
Einige haben behauptet, daß unter ärztlicher Aufsicht Medikamente wie Inderal und Atenolol oder sogar Zantac zur Beruhigung der Nerven geeignet sind.\footnote{Beim \enquote{Griff in die Medikamentenkiste} ist äußerste Vorsicht geboten. Wenn überhaupt, dann sollten diese Mittel wirklich nur unter ärztlicher Aufsicht eingenommen werden! Inderal und Atenolol sind Beta-Blocker und somit z.B. zur Senkung von viel zu hohem Blutdruck gedacht; Zantac ist ein Histamin-H2-Blocker und wird z.B. zur Behandlung von Magen- und Zwölffingerdarmgeschwüren eingesetzt. Ich halte den Einsatz solcher Mittel zur Dämpfung von Nervosität bzw. Lampenfieber für übertrieben und bedenklich.
Ein gesundes Maß Lampenfieber ist der Leistung beim Auftritt durchaus förderlich, und alles weitere läßt sich - wie im folgenden beschrieben - auch ohne Chemie gut in den Griff bekommen.}
Umgekehrt kann man die Nervosität verschlimmern, indem man Kaffee oder Tee trinkt, nicht genug Schlaf bekommt oder bestimmte Medikamente gegen Erkältung einnimmt.
Gefühle können auch durch Psychologie, Training oder Konditionierung kontrolliert werden.
Wissen ist das effektivste Mittel zur Kontrolle.
Erfahrene Schlangenbeschwörer leiden z.B. aufgrund ihres Wissens über Schlangen nicht unter einem der Gefühle, die uns überkommen würden, wenn wir in die Nähe einer Giftschlange kämen.

\textbf{Zu dem Zeitpunkt, an dem die Nervosität zum Problem wird, ist sie üblicherweise ein zusammengesetztes Gefühl, das schlagartig außer Kontrolle gerät.}
Zusätzlich zur Nervosität kommen noch andere Gefühle wie Furcht und Sorge hinzu.
Ein Mangel an Verständnis der Nervosität erzeugt ebenfalls Furcht wegen der Furcht vor dem Unbekannten.
Deshalb kann das bloße Wissen, was Lampenfieber ist, durch die Reduzierung der Furcht vor dem Unbekannten ein beruhigender Faktor sein.


\hypertarget{ng}{}

Wie gerät die Nervosität außer Kontrolle, und gibt es Wege, dies zu verhindern?
Eine Möglichkeit, diese Frage anzugehen, ist, einige Prinzipien der Grundlagenforschung zu betrachten.
\textbf{Praktisch alles in unserem Universum wächst durch einen Prozeß der als Kernbildung-Wachstum-Mechanismus (nucleation-growth = NG) bekannt ist.
Die NG-Theorie besagt, daß sich ein Objekt in zwei Stufen bildet: Kernbildung und Wachstum.}
Diese Theorie wurde populär und nützlich, weil es tatsächlich die Art ist, in der die meisten Objekte in unserem Universum gebildet werden, von Regentropfen bis zu Städten, Sternen, Menschen usw.
\textbf{Die beiden Schlüsselelemente der NG-Theorie sind:}

\begin{enumerate}[label={\arabic*.}] 
 \item \textbf{Kernbildung}<br>
Es bilden sich ständig Kerne und verschwinden welche.
Es gibt jedoch etwas, das ein kritischer Kern genannt wird, der stabil wird, wenn er sich gebildet hat - er verschwindet nicht.
Im allgemeinen bildet sich der kritische Kern nicht, solange es keine Übersättigung des Materials gibt, das sich verbindet, um ihn zu bilden.

\item \textbf{Wachstum}<br>
Damit das Objekt zu seiner endgültigen Größe anwächst, braucht der kritische Kern einen Wachstumsmechanismus, durch den seine Größe zunimmt.
\end{enumerate}

Im allgemeinen unterscheidet sich der Wachstumsmechanismus völlig von dem Mechanismus der Kernbildung.
Ein interessanter Aspekt der Kernbildung ist, daß es immer eine Schwelle zur Kernbildung gibt - ansonsten hätten sich bereits vor langer Zeit alle Kerne gebildet.
Die Größenänderung verläuft in beiden Richtungen: Sie kann positiv oder negativ sein.

Lassen Sie uns ein Beispiel untersuchen: Regen.
Regen tritt auf, wenn Wassertropfen kritische Kerne in Luft bilden, die mit Wasserdampf übersättigt ist (relative Feuchtigkeit größer als 100\%).
Gegen die oft falsch zitierte \enquote{wissenschaftliche Wahrheit}, daß die relative Luftfeuchtigkeit nie 100\% überschreitet, wird ständig von der Natur verstoßen, weil diese \enquote{Wahrheit} nur unter Gleichgewichtsbedingungen gültig ist, wenn sich alle Kräfte ausgleichen konnten.
Die Natur ist fast immer dynamisch, und sie kann weit vom Gleichgewicht entfernt sein.
Das geschieht z.B., wenn die Luft sich schnell abkühlt und mit Wasserdampf übersättigt wird.
Sogar ohne Übersättigung bildet Wasserdampf dauernd Wassertropfen, aber diese verdunsten, bevor sie kritische Kerne bilden können.
Bei Übersättigung können sich plötzlich kritische Kerne bilden, besonders wenn Wasser anziehende Staubpartikel in der Luft sind oder bei einer Druckstörung wie z.B. ein Donnerschlag, der die Moleküle näher zusammenbringt und so die Übersättigung steigert.
Die Luft, die mit kritischen Kernen gefüllt ist, nennen wir Wolken oder Nebel.
Wenn die Bildung der Wolke die Übersättigung auf Null reduziert, wird eine stabile Wolke gebildet; wenn nicht, wachsen die Kerne weiter, um die Übersättigung zu reduzieren.
Die Kerne können durch andere Mechanismen wachsen.
Das ist die Wachstumsphase des NG-Prozesses.
Die Kerne können aneinanderstoßen und sich zusammenballen, oder sie beginnen zu fallen und treffen andere Wassermoleküle und Kerne, bis sich Regentropfen bilden.

Wenden wir die NG-Theorie auf die Nervosität an.
Im täglichen Leben kommt und geht das Gefühl der Nervosität, ohne etwas Ernstes zu werden.
In einer ungewöhnlichen Situation, wie kurz vor einem Auftritt, gibt es jedoch eine Übersättigung mit Faktoren, die Nervosität verursachen: Sie müssen fehlerfrei vorspielen, Sie haben nicht genügend Zeit gehabt, das Stück zu üben, es wartet ein großes Publikum da draußen auf Sie, usw.
Das mag immer noch keinerlei Probleme bereiten, weil es bei der Nervosität natürliche Barrieren für die Kernbildung gibt, wie den Fluß des Adrenalins, die Selbstsicherheit oder einfach einen Mangel an Erkennen, daß man nervös ist, oder Sie sind vielleicht zu sehr damit beschäftigt, sich endgültig auf den Auftritt vorzubereiten.
Aber dann sagt ein anderer Künstler \enquote{Mann, ich habe vielleicht Schmetterlinge im Bauch!}, und Sie fühlen plötzlich einen Kloß im Hals und merken, daß Sie nervös sind - der kritische Kern hat sich gebildet!
Das mag immer noch nicht so schlimm sein, bis Sie anfangen sich zu sorgen, daß Ihr Stück vielleicht noch nicht bereit zur Aufführung ist oder daß die Nervosität anfängt, das Spielen zu stören - diese Sorgen lassen die Nervosität anwachsen.
Das sind genau die Prozesse, die durch die NG-Theorie beschrieben werden.
Das schöne an jeder wissenschaftlichen Theorie ist, daß sie nicht nur den Prozeß detailliert beschreibt, sondern auch Lösungen für Probleme anbietet.
Wie hilft uns also die NG-Theorie?

Wir können die Nervosität im Kernbildungsstadium angreifen; wenn wir die Kernbildung verhindern können, wird sich nie ein kritischer Kern bilden können.
Ein bloßes Verzögern der Kernbildung wird hilfreich sein, weil dies die zum Wachsen verfügbare Zeit reduziert.
Leichtere Stücke zu spielen, wird die Übersättigung mit Sorge reduzieren.
Simulierte Konzerte verleihen Ihnen mehr Erfahrung und Selbstsicherheit; beides wird die Angst vor dem Unbekannten verringern.
Im allgemeinen muß man ein Stück dreimal oder öfter vorführen, bevor man weiß, ob man es erfolgreich aufführen kann oder nicht; deshalb ist es hilfreich, Stücke zu spielen, die man mehrmals vorgeführt hat.
\textbf{Die Nervosität ist im allgemeinen vor einem Auftritt am schlimmsten; haben Sie erst einmal angefangen zu spielen, sind Sie so mit den bevorstehenden Aufgaben beschäftigt, daß Sie keine Zeit haben, sich länger mit der Nervosität zu befassen, und der Wachstumsfaktor somit reduziert wird.}
Dieses Wissen hilft, weil es die Furcht abschwächt, daß während des Auftritts alles schlimmer wird.
Sich nicht länger mit der Nervosität zu befassen, ist eine weitere Möglichkeit, sowohl die Kernbildung zu verzögern als auch die Wachstumsphase zu verlangsamen.
Deshalb ist es eine gute Idee, sich selbst beschäftigt zu halten, während man auf den Anfang des Konzerts wartet.\footnote{Sie können sich zusätzlich mit einer \hyperref[c1ii21uebung]{einfachen Atemübung} entspannen.}
\textbf{Das \hyperref[c1ii12mental]{mentale Spielen} ist nützlich, weil Sie gleichzeitig Ihr Gedächtnis prüfen und sich selbst beschäftigt halten können; deshalb ist es das wichtigste Werkzeug zur Vermeidung und Verzögerung der Kernbildung und zur Reduzierung des Wachstums.}
Sehen Sie dazu in den Abschnitten \hyperref[c1iii16c]{16c} und \hyperref[c1iii16d]{16d} einige Vorschläge dafür, wie Lehrer ein Auftrittstraining zur Verfügung stellen können.

Bei einem wichtigen Konzert ist das Vermeiden der Kernbildung wahrscheinlich nicht möglich.
Deshalb sollten wir auch über Wege zur Unterbindung des Wachstums nachdenken.
Da die Nervosität im allgemeinen geringer wird, nachdem der Auftritt beginnt, kann dieses Wissen dazu benutzt werden, die Sorge zu reduzieren und somit die Nervosität.
Das kann sich selbst verstärken, und wenn Sie sich sicherer fühlen, kann sich die Nervosität oftmals völlig auflösen, wenn Sie sie unterhalb des kritischen Kerns reduzieren können.
Weitere wichtige Faktoren sind die geistige Haltung und die Vorbereitung.
Ein Auftritt ist immer ein interaktiver Prozeß zwischen Ihnen selbst und dem Publikum.
Musikalisch zu spielen ist, natürlich, immer die Antwort - wenn Sie Ihr komplettes Gehirn in die Aufgabe Musik zu erzeugen einbeziehen können, bleibt nur sehr wenig Kapazität dafür übrig, sich um die Nervosität zu sorgen.
Das sind alles Maßnahmen, die das Anwachsen der Nervosität reduzieren.

\textbf{Es ist - besonders bei Kindern, da sie leichter langfristige psychologische Schäden erleiden können - keine gute Idee, so zu tun, als ob die Nervosität nicht existieren würde.}
Kinder sind clever, und sie können diese Verstellung leicht durchschauen, und die Notwendigkeit, mit der Verstellung zu spielen, kann den Streß nur verstärken.
Deshalb ist ein Auftrittstraining, in dem offen über Nervosität gesprochen wird, so wichtig.
Im Fall von jungen Schülern müssen ihre Eltern und Freunde, die das Konzert besuchen, ebenfalls Bescheid wissen.
Sätze wie \enquote{Ich hoffe, Du bist nicht nervös!}, oder \enquote{Wie kannst Du auftreten, ohne nervös zu sein?}, führen fast mit Sicherheit zu Kernbildung und Wachstum.
Andererseits ist es jedoch auch unverantwortlich, die Nervosität völlig zu ignorieren und Kinder ohne Auftrittstraining in den Auftritt zu schicken, und kann sogar zu irreparablen psychologischen Schäden führen.

\textbf{Die richtige geistige Haltung zu entwickeln, ist die beste Möglichkeit, das Lampenfieber zu kontrollieren.}
Wenn Sie zu der Auffassung gelangen können, daß aufzutreten die wundervolle Erfahrung ist, Musik für andere zu machen, und die richtigen Reaktionen für den Fall entwickeln, daß Sie Fehler machen, dann wird Nervosität kein Problem sein.
Es ist z.B. ein großer Unterschied, einen Fehler mit Humor zu nehmen bzw. leicht darüber hinwegzukommen oder den Fehler wie eine Katastrophe erscheinen zu lassen, die den ganzen Auftritt verdirbt.
Das Auftrittstraining muß Lektionen über die Reaktion auf verschiedene Umstände beinhalten.
Deshalb ist es so wichtig, früh in der Karriere des Schülers sehr leichte Stücke zu spielen, die ohne Nervosität aufgeführt werden können; eine einzige solche Erfahrung kann der Beweis dafür sein, daß es möglich ist, ohne Nervosität aufzutreten.
Diese eine Erfahrung kann Ihr Verhalten bei Auftritten für den Rest Ihres Lebens beeinflussen.
Um einen solch fehlerfreien Auftritt zu garantieren, entwickeln Sie am besten ein sicheres \hyperref[c1ii12mental]{mentales Spielen}, das Sie dazu befähigt, von jeder beliebigen Note des Stücks aus mit dem Spielen zu beginnen, der Musik immer voraus zu sein, die Musikalität in Ihrem Geist zu erzeugen, ein \hyperref[c1iii12]{absolutes Gehör} zu entwickeln, über Fehler hinwegzukommen oder sie zu kaschieren, jeden Tag in Gedanken Klavier zu spielen, d.h. jeden Teil des Stücks jederzeit und überall zu üben, usw.; all das zu erreichen, wird Ihnen die Zuversicht eines vollendeten Musikers geben.
Das Publikum wird sicherlich der Meinung sein, daß es mit einem großen Talent zusammengekommen ist.

Um es zusammenzufassen: Lampenfieber ist eine Form der Nervosität, die in einer Spirale außer Kontrolle geraten ist.
Ein gewisses Maß an Nervosität ist normal und nützlich.
Man kann die Nervosität minimieren, indem man sich beschäftigt hält und somit ihre Kernbildung verzögert und indem man ihr Wachstum durch musikalisches Spielen reduziert; \hyperref[c1ii12mental]{mentales Spielen} ist dafür das nützlichste Mittel.
Deshalb macht es keinen Sinn, und ist ein Fehler, zu fragen: \enquote{Wirst Du nervös, wenn Du auftrittst?}
Jeder wird es und sollte es auch.
Wir müssen die Nervosität nur eindämmen, so daß sie nicht jenseits unserer Kontrolle anwächst.
\textbf{Zu erkennen, daß ein gewisses Maß an Nervosität normal ist, ist die beste Ausgangslage, um zu lernen, wie man sie kontrolliert.}
Natürlich gibt es einen großen Bereich unterschiedlicher Menschen: von denjenigen, die nicht nervös werden, bis zu denjenigen, die schrecklich unter Lampenfieber leiden.
Am besten begegnet man der Nervosität mit Ehrlichkeit - wir müssen ihre Wirkung auf jeden einzelnen zugeben und entsprechend mit ihr umgehen.
Vertrauen in Ihre Fähigkeiten für das Auftreten zu erlangen, kann üblicherweise die Nervosität eliminieren, und die Kunst des mentalen Spielens zu perfektionieren ist der einzige Weg, wirklich ein solches Vertrauen zu erreichen.



