% File: anmerkungen

\label{anmerkungen}

<h2>Anmerkungen</h2>

Diese Website bietet Ihnen einen kostenlosen Klavierunterricht, Lehrmaterial für das Klavierspielen und Anweisungen zum \hyperref[c2_1]{Stimmen des Klaviers}.
Sie können das Klavierspielen im Vergleich zu anderen Methoden\footnote{bis zu} \textbf{1000mal schneller (!)} lernen (s. \hyperref[c1iv5]{Kapitel 1, IV.5}).
Dieses ist das erste Buch, das jemals darüber geschrieben wurde, wie man das Klavierspielen übt.
Hunderte von Jahren lehrten viele Lehrer und andere Bücher, welche Techniken man erwerben muß, aber das nutzt wenig, wenn man im Gegensatz zu Mozart, Liszt usw. nicht weiß, wie man sich diese schnell aneignet.
Sie können sich entweder \hyperref[copy]{hier} das ganze Buch oder mit Hilfe der Links im \hyperref[Inhalt]{Inhaltsverzeichnis} oder in der \hyperref[./dateien.html\#copy]{Übersicht der Dateien} Teile davon kostenlos herunterladen.

Benutzen Sie dieses Buch zum Klavierspielenlernen als ergänzendes Lehrbuch, wenn Sie einen Lehrer haben.
Wenn Sie keinen Lehrer haben, suchen Sie sich ein beliebiges Musikstück aus, das Sie lernen möchten (das innerhalb Ihrer Fertigkeitsstufe liegt), und beginnen Sie, es mit den hier beschriebenen Methoden zu lernen; die Methoden sind grob in der Reihenfolge angeordnet, in der Sie sie benötigen, wenn Sie ein neues Stück lernen.
In beiden Fällen (mit oder ohne Lehrer), sollten Sie das ganze Buch zunächst einmal durchlesen.
Beginnen Sie mit dem \hyperref[preface]{Vorwort}, das Ihnen einen kurzen Überblick vermittelt.
Überspringen Sie alle Abschnitte, von denen Sie glauben, sie seien nicht relevant oder zu ausführlich.
Versuchen Sie nicht, jedes Konzept zu verstehen oder sich alles zu merken - lesen Sie es so, wie Sie einen Roman lesen würden, nur zum Spaß.
Machen Sie sich nur mit dem Buch vertraut, damit Sie eine Vorstellung davon bekommen, wo bestimmte Themen besprochen werden.
Lesen Sie zuletzt die \hyperref[testimonials]{Leserkommentare}, soweit Sie diese interessant finden.
Beginnen Sie dann erneut ab einer Stelle, die Ihnen das benötigte Material bietet; die meisten werden die kompletten Abschnitte \hyperref[c1i1]{I} und \hyperref[c1ii1]{II} des ersten Kapitels benötigen.
Sie können danach zu den einzelnen Themen springen, die die Komposition betreffen, die Sie gerade lernen.
Für den Fall, daß Sie nicht genau wissen, welche Kompositionen Sie lernen sollen, werden in diesem Buch zahlreiche Beispiele vorgestellt: von Material für Anfänger (s. \hyperref[c1iii18]{Kapitel 1, III.18}) bis zur Mittelstufe; merken Sie sich deshalb beim ersten Lesen diese Stellen mit Beispielen und Vorschlägen.

\textbf{Ein Wunsch} an diejenigen, die dieses Material nützlich fanden: Bitte lassen Sie mindestens zwei Menschen von meiner Website wissen, so daß wir eine Kettenreaktion von noch mehr Leuten, die über diese Website informiert werden, starten können.

Ich suche Freiwillige, die das Buch in jede andere Sprache übersetzen.
Senden Sie bitte eine E-Mail an \hyperref[mailto:cc88m@aol.com?subject=foppde:\%20Translation\%20request]{cc88m@aol.com} um die Einzelheiten zu besprechen.


\label{HinUeber}

Übersetzer sollten sich etwas mit HTML auskennen und in der Lage sein, eine eigene Site für die Webseiten zu unterhalten (XXX und Xxxxx sollten gut sein).
Meine Vision ist, daß dieses Buch irgendwann zu einer dauerhaften Site umziehen wird.
Alle Übersetzungen sollten zur gleichen Site umziehen können.
Der Speicherbedarf aller potentiellen Übersetzungen ist bescheiden, da jede Sprache nur etwas mehr als 1 MB erfordert.

Übersetzer sind für ihre eigene Website verantwortlich und sollten nach Möglichkeit mit den Updates des Originals Schritt halten.
Es gibt jede Menge Software, um geänderte Version mit älteren Versionen zu vergleichen, so daß dies kein Problem darstellen sollte.
Sie werden aber eine Kopie der älteren Version auf Ihrem Computer behalten müssen, weil die älteren Versionen von meiner Website verschwinden werden, wenn sie geändert werden\footnote{und falls mal die Platte samt Backup \enquote{abraucht}, gibt es ja noch den netten Kollegen von der anderen Sprache.}

Übersetzer sollten vorzugsweise Klavierspieler oder Klavierlehrer sein und etwas über das Klavier selbst wissen (\footnote{Mechanik,} Stimmen, Einstellen, Aufarbeiten).
Wenn der Übersetzer bei einem bestimmten Thema Lücken hat, können wir immer Helfer für dieses Thema finden, so daß ein Mangel an Fachkenntnissen eines Übersetzers kein Problem ist.

Ich schreibe dieses Buch auf einer Freiwilligenbasis und kann deshalb Übersetzern nichts zahlen, bis irgendein Stifter auftaucht.
Wir haben ein Programm zur Aufteilung der Aufwandsentschädigungen; ich werde es mit Ihnen besprechen, sobald Sie sich anbieten.
Ich würde mich selbstverständlich freuen, soweit wie möglich dabei zu helfen, die Übersetzung zu beschleunigen und werde einen Link zur Übersetzung auf meiner Inhaltsverzeichnisseite zur Verfügung stellen.

Wir leisten hier Pionierarbeit für eine neue Art von Internet-Buch. Das ist ein Wink der Zukunft und wirklich aufregend.
Dieses Buch sollte sich zu dem vollständigsten Lehrbuch für das Klavierspielen entwickeln, das kostenlos ist, immer auf dem neuesten Stand, in dem Fehler eliminiert werden, sobald sie entdeckt werden, und das in allen verbreiteten Sprachen verfügbar sein wird.
Es gibt keinen Grund, warum Schulen und Schüler für elementare Lehrbücher von Arithmetik bis Zoologie zahlen müssen.
In der Zukunft werden sie alle kostenlos zum Download zur Verfügung stehen.
Der Weltwirtschaft wird enorm dadurch geholfen, daß das Ausbildungsmaterial jedem frei zugänglich ist.
Es ist einfach unglaublich, sich die Zukunft der Ausbildung im Internet vorzustellen.
Da alles, was man braucht, ein paar der besten Experten der Welt sind, die Lehrbücher schreiben und weitere Freiwillige, um sie zu übersetzen, sind die dafür notwendigen Mittel vernachlässigbar im Vergleich zum ökonomischen Nutzen.
Deshalb bringen Übersetzer nicht nur ihren Landsleuten einen Nutzen, sondern nehmen auch an einem prächtigen neuen Experiment teil, das alle Klavierspieler, Klavierlehrer, Klavierstimmer und die Klavierindustrie fördert.





