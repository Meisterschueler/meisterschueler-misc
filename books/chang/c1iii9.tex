% File: c1iii9

\section{Ein Stück auf Hochglanz bringen - Fehler beseitigen}
\label{c1iii9}

Beim Ausfeilen eines \enquote{fertigen} Stückes möchte man fünf Ziele erreichen: ein gutes \hyperref[c1iii6]{Gedächtnis} gewährleisten, Fehler beseitigen, \hyperref[c1iii14d]{Musik machen}, die Technik weiterentwickeln und sich \hyperref[c1iii14]{auf Auftritte vorbereiten}.
\textbf{Der erste Schritt ist das Gewährleisten des Gedächtnisses, und wir haben in Abschnitt III.6 gesehen, daß es dazu am besten ist, das ganze Stück in Gedanken - ohne das Klavier - zu spielen.}
\hyperref[c1ii12]{Mentales Spielen} garantiert, daß das Gedächtnis praktisch unfehlbar ist.
Wenn einige Teile etwas unsicher sind, können Sie jederzeit an ihnen arbeiten, auch wenn Sie nicht am Klavier sind.
Das mentale Spielen ist die sicherste Form des Gedächtnisses, weil es ein rein mentales Gedächtnis ist - es ist nicht von akustischen, taktilen oder visuellen Reizen abhängig.
Es beseitigt auch die meisten Fehler, weil diese ihren Ursprung im Gehirn haben.
Sehen wir uns ein paar verbreitete Ursachen von Fehlern an.
Erinnerungsblockaden treten wegen einer zu großen Abhängigkeit vom \hyperref[c1iii6hand]{Hand-Gedächtnis} auf.
\hyperref[c1ii22]{Stottern} ist die Angewohnheit, bei jedem Fehler anzuhalten, während man HT spielt, ohne vorher genügend HS geübt zu haben.
Man trifft falsche Noten, weil die Hände nicht stets die Tasten fühlen und man nicht mehr weiß, wo welche Tasten sind.
Fehlende Noten resultieren aus einem Mangel an \hyperref[c1ii14]{Entspannung} und dem ungewollten Heben der Hände - eine Angewohnheit, die man üblicherweise durch zuviel langsames HT-Üben erwirbt.
Wir haben Lösungen  zum Beseitigen all dieser Fehlerquellen besprochen.
Das musikalische Spielen und das Hervorbringen der \enquote{Farbe} einer Komposition ist die endgültige Aufgabe beim Ausfeilen.
Man kann nicht einfach nur die Noten exakt spielen und erwarten, daß die Musik und die Farbe auf magische Weise zum Vorschein kommen; man muß beides aktiv in Gedanken erzeugen, bevor man die Noten spielt.
Das mentale Spielen gestattet Ihnen das alles.
Wenn die Finger diese geistigen Vorstellungen nicht reproduzieren können, ist vielleicht das Stück zu schwierig.
Sie werden die Technik schneller entwickeln, indem Sie Stücke üben, die Sie bis zur Perfektion ausfeilen können.
Geben Sie aber auch nicht zu leicht auf, da die Ursache der Schwierigkeiten eventuell nicht bei Ihnen liegt, sondern ein anderer Faktor ist, wie z.B. die Qualität oder der Zustand des Klaviers.

Ein großer Teil des Ausfeilens ist die Aufmerksamkeit gegenüber den Details.
Die beste Art, den korrekten Ausdruck sicherzustellen, ist, zu den Noten zurückzugehen und jedes Ausdruckszeichen, jedes Staccato, jede Pause, Tasten, die unten gehalten werden, das Heben des Fingers und des Pedals usw. noch einmal durchzugehen.
Das wird Ihnen das exakteste Bild der logischen Struktur der Musik vermitteln, das notwendig ist, um den richtigen Ausdruck hervorzubringen.
Die Schwächen jedes Einzelnen sind verschieden und sind diesem im allgemeinen nicht bekannt.
Jemand, dessen Timing unsauber ist, kann üblicherweise das falsche Timing nicht hören.
\textbf{Das ist der Punkt, an dem ein Lehrer eine Schlüsselrolle beim Erkennen dieser Schwächen spielt.}\footnote{\enquote{Digital-Pianisten} können diese Probleme teilweise selbst verringern, indem sie ihr Spiel \hyperref[c1iii13MIDI]{mit einem Sequenzer-Programm aufnehmen} und sich die \hyperref[midi_check]{MIDI-Signale genauer ansehen}.}

Musik zu machen ist der wichtigste Teil des Ausfeilens eines Stücks.
Einige Lehrer betonen diesen Punkt, indem sie sagen, man solle 10\% seiner Zeit mit dem Erlernen der Technik und 90\% der Zeit mit dem Lernen, Musik zu machen, verbringen.
Die meisten Schüler ringen mehr als 90\% ihrer Zeit mit der Technik, in dem falschen Glauben, daß zu üben, was man nicht spielen kann, die Technik entwickeln wird.
Dieser Fehler erwächst aus der intuitiven Logik, daß man, wenn man etwas übt, das man nicht spielen kann, irgendwann in der Lage sein sollte, es zu spielen.
Das stimmt aber nur für Material, das im Rahmen Ihrer Fertigkeitsstufe ist.
Bei zu schwierigem Material weiß man nie, was geschieht, und häufig führt solch ein Versuch zu irreversiblen Problemen wie Streß und \hyperref[c1iv2b]{Geschwindigkeitsbarrieren}.
Wenn Sie z.B. die Geschwindigkeit steigern möchten, ist der schnellste Weg dazu, leichte Stücke zu spielen, die Sie bereits ausgefeilt haben, und dieses Spielen zu beschleunigen.
Wenn die Geschwindigkeit der Finger erst einmal steigt, sind Sie bereit, schwierigeres Material mit einer höheren Geschwindigkeit zu spielen.
\textbf{Somit ist die Zeit des Ausfeilens auch die beste Zeit für die technische Entwicklung und kann sehr viel Spaß machen.}

Ihre Fertigkeiten zum Vorspielen zu perfektionieren, ist Teil des Ausfeilens; das wird unten in \hyperref[c1iii14]{Abschnitt 14} besprochen.
Viele Klavierspieler begegnen dem folgenden merkwürdigen Phänomen.
Es gibt Zeiten, in denen sie nichts falsch machen können und sich ohne Fehler oder Probleme die Seele aus dem Leib spielen können.
Zu anderen Zeiten wird jedes Stück schwierig und sie machen Fehler an Stellen, die ihnen normalerweise keine Probleme bereiten.
Was verursacht diese Höhen und Tiefen?
Nicht zu wissen, welcher der beiden Zustände auf einen zukommt, kann ein schrecklicher Gedanke sein, der \hyperref[c1iii15]{Nervosität} hervorrufen kann.
Offensichtlich gibt es viele Faktoren, wie z.B. \hyperref[fpd]{FPD}, der besonnene Gebrauch des \hyperref[c1ii17]{langsamen Spielens} usw.
Der wichtigste Faktor ist jedoch das \hyperref[c1ii12]{mentale Spielen}.
Alle Klavierspieler benutzen, bewußt oder unbewußt, etwas mentales Spielen.
Das Vorspielen hängt oft von der Qualität dieses mentalen Spielens ab.
Solange man das mentale Spielen nicht bewußt benutzt, weiß man nie, in welchem Zustand es ist.
So stört z.B. das Üben eines neuen Stücks das mentale Spielen eines anderen Stücks.
Deshalb ist es so wichtig, zu wissen, was dieses mentale Spielen ist, ein gutes mentales Spielen aufzubauen und zu wissen, wann man es überprüfen und wieder auffrischen muß.
Wenn Ihr mentales Spielen sich aus irgendeinem Grund verschlechtert hat, wird es vor einem Konzert zu überprüfen Sie auf die drohende Gefahr aufmerksam machen und Ihnen die Gelegenheit geben, den Schaden zu reparieren.

Ein verbreitetes Problem ist, daß Schüler dauernd neue Stücke lernen und wenig Zeit für das Ausfeilen der Stücke haben.
Das passiert hauptsächlich Schülern, die die intuitiven Lernmethoden benutzen.
Es dauert so lange, jedes einzelne Stück zu lernen, daß keine Zeit bleibt, sie auszufeilen, bevor man mit einem anderen Stück anfangen muß.
Die Lösung sind natürlich bessere Lernmethoden.
 
Zusammengefaßt \textbf{ist ein solides \hyperref[c1ii12]{mentales Spielen} die wichtigste Voraussetzung dafür, ein Stück auszufeilen und es für einen Auftritt vorzubereiten}.
Fortgeschrittene Technik erlangt man nicht nur durch das Üben neuer Fertigkeiten, sondern auch durch das Spielen fertiger Stücke.
Das ständige Üben neuer Fertigkeiten ist sogar kontraproduktiv und führt zu Geschwindigkeitsbarrieren, Streß und unmusikalischem Spielen.
 


