% File: c22

\subsection{Chromatische Tonleiter und Temperaturen}
\label{c2_2}

\subsubsection{Einleitung}
\label{c2_2a} 

Die meisten von uns sind mit der chromatischen Tonleiter einigermaßen vertraut und wissen, daß sie temperiert sein muß, aber was sind die präzisen Definitionen der beiden Begriffe?
\textbf{Warum ist die chromatische Tonleiter so besonders, und warum ist das Temperieren notwendig?}
Wir erforschen zunächst die mathematische Grundlage der chromatischen Tonleiter und des Temperierens, weil der mathematische Ansatz die knappste, klarste und präziseste Behandlung ist.
Wir besprechen dann die historischen und musikalischen Gesichtspunkte, damit wir die relativen Vorzüge der verschiedenen Stimmungen besser verstehen.
Ein grundlegendes mathematisches Fundament dieser Konzepte ist entscheidend für ein gutes Verständnis dafür, wie Klaviere gestimmt werden.
Informationen über das Stimmen finden Sie bei White, Howell, Fischer, Jorgensen oder Reblitz (s. \hyperref[reference]{Quellenverzeichnis}).
 

\label{c2_2b}

\subsubsection{Mathematische Behandlung}
\label{c2_2_math} 

In Tabelle 2.2a sind drei Oktaven aufgeführt.
Die schwarzen Tasten des Klaviers werden mit Kreuzen dargestellt, z.B. steht das \# rechts von C für ein C\#, und ist nur bei der höchsten Oktave ausgewiesen.
\textbf{Jede der aufeinanderfolgenden Frequenzänderungen in der chromatischen Tonleiter wird ein Halbton genannt, und eine Oktave besteht aus 12 Halbtönen.}
Die Hauptintervalle und die Ganzzahlen, die die Frequenzverhältnisse dieser Intervalle repräsentieren, sind jeweils oberhalb und unterhalb der chromatischen Tonleiter aufgeführt.
Das Wort Intervall wird hierbei im Sinne von zwei Noten benutzt, deren Frequenzverhältnis der Quotient kleiner ganzer Zahlen ist.
Außer für Vielfache dieser Grundintervalle erzeugen Ganzzahlen, die größer als ungefähr 10 sind, Intervalle, die für das Ohr nicht einfach zu erkennen sind.
Gemäß Tabelle 2.2a ist das grundlegendste Intervall die Oktave, bei der die Frequenz der höheren Note das Doppelte der Frequenz der tieferen Note ist.
Das Intervall zwischen C und G ist eine Quinte, und die Frequenzen von C und G stehen in einem Verhältnis von 2 zu 3 zueinander.
Die große Terz hat vier Halbtöne, und die kleine Terz hat drei.
\textbf{Die Zahl, die jedem Intervall zugeordnet ist, z.B. vier in der Quarte, ist bei der C-Dur-Tonleiter die Zahl der weißen Tasten inkl. der beiden Tasten am Anfang und Ende des Intervalls und hat keine weitere mathematische Bedeutung.}
Beachten Sie, daß das Wort \enquote{Tonleiter} bzw. \enquote{Skala} in \enquote{chromatische Tonleiter}, \enquote{C-Dur-Tonleiter} und \enquote{logarithmische oder Frequenz-Skala} (s.u.) eine völlig unterschiedliche Bedeutung hat; die zweite ist eine Untermenge der ersten.

<table border>
 <tr>
  <td bgcolor=\enquote{\#E0E0E0}>\textbf{Oktave}</td>
  <td bgcolor=\enquote{\#E0E0E0}>\textbf{Quinte}</td>
  <td bgcolor=\enquote{\#E0E0E0}>\textbf{Quarte}</td>
  <td bgcolor=\enquote{\#E0E0E0}>\textbf{Gr. Terz}</td>
  <td bgcolor=\enquote{\#E0E0E0}>\textbf{Kl. Terz}</td>
  <td bgcolor=\enquote{\#E0E0E0}> </td>
  <td bgcolor=\enquote{\#E0E0E0}>  \\ 
 CDEFGAH & C D E F & G A H & C \# D \# & E F \# & G \# A \# H & C \\ 
 1 & 2 & 3 & 4 & 5 & 6 & 8 \\ 
</table>
<h5>(Tabelle 2.2a: Frequenzverhältnisse der Intervalle in der chromatischen Tonleiter)</h5>
Wir können oben sehen, daß eine Quarte und eine Quinte sich zu einer Oktave \enquote{aufaddieren} und eine große Terz und eine kleine Terz sich zu einer Quinte \enquote{aufaddieren}.
Beachten Sie, daß diese Addition im logarithmischen Raum erfolgt, wie unten erklärt wird.
Die fehlende Ganzzahl 7 wird ebenfalls unten erklärt.
 

\label{et1}

\textbf{Die \hyperref[et]{gleichschwebend temperierte} (ET) chromatische Tonleiter besteht aus \enquote{gleichen} Halbtonschritten für jede nachfolgende Note.}
Sie sind in dem Sinne gleich, daß das Verhältnis der Frequenzen von zwei aufeinanderfolgenden Noten immer das gleiche ist.
Diese Eigenschaft stellt sicher, daß jede Note (außer in der Tonhöhe) mit allen anderen identisch ist.
Diese Gleichförmigkeit der Noten gestattet es dem Komponisten oder Künstler, jede Tonhöhe und jede Tonart zu benutzen, ohne auf große Dissonanzen zu treffen, wie unten weiter erklärt wird.
In einer Oktave einer ET-Tonleiter gibt es 12 gleiche Halbtöne und jede Oktave ist ein genauer Faktor von 2 im Frequenzverhältnis.
Deshalb beträgt die Frequenzänderung für jeden Halbton:


\label{gleich21}
<table border>&\#9;Halbtonschritt<sup>12</sup>&\#9;= 2&\#9;oder &\#9;Halbtonschritt&\#9;= 2<sup>1/12</sup> pprox 1,05946 \\ </table>
<h5>(Gleichung 2.1)</h5>

Gleichung 2.1 definiert die ET chromatische Tonleiter und erlaubt die Berechnung der Frequenzverhältnisse von \enquote{Intervallen} in dieser Tonleiter.
Wie verhalten sich die \enquote{Intervalle} bei ET zu den Frequenzverhältnissen der reinen Intervalle?
\textbf{Der Vergleich ist in Tabelle 2.2b aufgeführt und zeigt, daß die Intervalle der ET-Tonleiter den reinen Intervallen sehr nah kommen.}

<table border cellpadding=3>
 <tr>
  <td bgcolor=\enquote{\#E0E0E0}>\textbf{Intervall}</td>
  <td bgcolor=\enquote{\#E0E0E0}>\textbf{Freq.-Verh.}</td>
  <td bgcolor=\enquote{\#E0E0E0}>\textbf{ET-Tonleiter}</td>
  <td bgcolor=\enquote{\#E0E0E0}>\textbf{Differenz} \\ 
 <tr>
  <td valign=\enquote{bottom}>Kleine Terz</td>
  <td valign=\enquote{bottom}>6/5 = 1,2000 & Halbtonschritt<sup>3</sup> pprox 1,1892</td>
  <td valign=\enquote{bottom}>+0,0108 \\ 
 <tr>
  <td valign=\enquote{bottom}>Große Terz</td>
  <td valign=\enquote{bottom}>5/4 = 1,2500 & Halbtonschritt<sup>4</sup> pprox 1,2599</td>
  <td valign=\enquote{bottom}>-0,0099 \\ 
 <tr>
  <td valign=\enquote{bottom}>Quarte</td>
  <td valign=\enquote{bottom}>4/3 pprox 1,3333 & Halbtonschritt<sup>5</sup> pprox 1,3348</td>
  <td valign=\enquote{bottom}>-0,0015 \\ 
 <tr>
  <td valign=\enquote{bottom}>Quinte</td>
  <td valign=\enquote{bottom}>3/2 = 1,5000 & Halbtonschritt<sup>7</sup> pprox 1,4983</td>
  <td valign=\enquote{bottom}>+0,0017 \\ 
 <tr>
  <td valign=\enquote{bottom}>Oktave</td>
  <td valign=\enquote{bottom}>2/1 = 2,0000 & Halbtonschritt<sup>12</sup> = 2,0000</td>
  <td valign=\enquote{bottom}>0,0000 \\ 
</table>
<h5>(Tabelle 2.2b: Vergleich der reinen Intervalle mit der gleichschwebend temperierten Tonleiter)</h5>
\textbf{Die Abweichung ist bei den Terzen am größten, mehr als fünfmal so groß wie die Abweichung bei den anderen Intervallen aber trotzdem nur ungefähr 1\%.}
Nichtsdestoweniger sind diese Abweichungen leicht zu hören, und einige Klavierliebhaber haben sie großmütig als \enquote{die rollenden Terzen} tituliert, während sie in Wahrheit inakzeptable Dissonanzen sind.
Es ist ein Mangel, mit dem wir leben müssen, wenn wir die ET-Tonleiter akzeptieren wollen.
Die Abweichungen bei den Quarten und Quinten erzeugen um das mittlere C Schwebungen von ungefähr 1 Hz, was bei den meisten Musikstücken kaum zu hören ist; diese Schwebungsfrequenz verdoppelt sich jedoch mit jeder höheren Oktave.

Wäre die Ganzzahl 7 in Tabelle 2.2a aufgenommen worden, hätte sie ein Intervall mit dem Verhältnis 7/6 repräsentiert und würde dem Quadrat eines Halbtonschritts entsprechen.
Die Abweichung zwischen diesen beiden Zahlen beträgt mehr als 4\% und ist zu groß, um ein musikalisch akzeptables Intervall zu bilden; sie wurde deshalb nicht in Tabelle 2.2a aufgeführt.
Es ist nur ein mathematischer Zufall, daß die aus 12 Tönen bestehende chromatische Tonleiter so viele Verhältnisse nahe an den reinen Intervallen erzeugt.
\textbf{Von den 8 kleinsten Ganzzahlen führt nur die Zahl 7 zu einem völlig inakzeptablen Intervall.
Die chromatische Tonleiter basiert auf einem glücklichen mathematischen Zufall der Natur!
Sie wird durch die kleinste Anzahl von Noten gebildet, die die maximale Anzahl von Intervallen ergeben.}
Kein Wunder, daß frühe Zivilisationen glaubten, es läge etwas mystisches in dieser Tonleiter.
Die Zahl der Noten in einer Oktave zu erhöhen, führt zu keiner großen Verbesserung der Intervalle, bis die Zahlen sehr groß werden, was diesen Ansatz für die meisten Musikinstrumente undurchführbar macht.

Beachten Sie, daß die Frequenzverhältnisse der Quarten und Quinten sich nicht zu dem der Oktave aufaddieren (1,5000 + 1,3333 = 2,8333 statt 2,0000).
Sie addieren sich allerdings im logarithmischen Maßstab, weil (3/2)x(4/3) = 2.
Im logarithmischen Raum wird die Multiplikation zur Addition.
Warum ist das so wichtig?
Weil die Geometrie der Cochlea (Ohrschnecke) anscheinend eine logarithmische Komponente hat.
Akustische Frequenzen auf einer logarithmischen Skala wahrzunehmen erreicht zwei Dinge: man kann bei gegebener Größe der Cochlea einen breiteren Frequenzbereich hören, und das Analysieren der Frequenzverhältnisse wird einfach, weil man, anstatt die zwei Frequenzen zu dividieren oder multiplizieren, nur ihre Logarithmen subtrahieren oder addieren muß.
Wenn z.B. das C3 von der Cochlea an einer Stelle erkannt wird und das C4 an einer anderen Stelle, die 2 mm weiter aufwärts liegt, dann wird das C5 an einer Stelle erkannt, die 4 mm aufwärts liegt, genau wie bei einem Rechenschieber.
Um zu zeigen, wie nützlich das ist: bei einem gegebenen F5 weiß das Gehirn, daß das F4 2 mm weiter unten zu finden ist!
Deshalb sind Intervalle (erinnern Sie sich daran, daß Intervalle Divisionen von Frequenzen sind) von einer logarithmisch aufgebauten Cochlea besonders einfach zu analysieren.
Wenn wir Intervalle spielen, üben wir mathematische Operationen im logarithmischen Raum auf einem mechanischen Computer genannt Klavier aus, ähnlich wie es früher mit dem Rechenschieber getan wurde.
\textbf{Deshalb hat die logarithmische Natur der chromatischen Tonleiter viel mehr Konsequenzen, als nur einen größeren hörbaren Frequenzbereich zur Verfügung zu stellen.}
Die logarithmische Skala stellt sicher, daß die beiden Noten jedes Intervalls, unabhängig davon wo man sich auf dem Klavier befindet, immer denselben Abstand voneinander haben.
Durch die Übernahme einer logarithmischen Skala wird die Tastatur mechanisch auf das menschliche Ohr abgebildet!
Das ist wahrscheinlich der Grund, warum Harmonien für das Ohr so angenehm sind - Harmonien werden durch das menschliche Gehör am leichtesten entschlüsselt und erinnert.

Angenommen, wir würden \hyperref[gleich21]{Gleichung 2.1} nicht kennen; können wir die ET chromatische Tonleiter aus den Beziehungen der Intervalle erzeugen?
Wenn die Antwort ja ist, kann ein Klavierstimmer ein Klavier stimmen, ohne Berechnungen durchführen zu müssen.
Diese Intervallbeziehungen, so stellt sich heraus, bestimmen die Frequenzen aller Noten der zwölfnotigen chromatischen Tonleiter.
Eine Temperatur ist eine Gruppe von Intervallbeziehungen, die diese Bestimmung ermöglicht.
Von einem musikalischen Standpunkt aus gibt es keine chromatische Tonleiter, die besser wäre als alle anderen, obwohl ET die einmalige Eigenschaft hat, daß sie ein freies Transponieren erlaubt.
Unnötig zu sagen, daß \textbf{ET nicht die einzige musikalisch nützliche Temperatur ist, und wir werden unten weitere Temperaturen besprechen.}
Die Temperatur ist keine Option, sondern eine Notwendigkeit; wir \textit{müssen} eine Temperatur wählen, um diese mathematischen Schwierigkeiten zu überwinden.
\textbf{Kein musikalisches Instrument, das auf der chromatischen Tonleiter basiert, ist völlig frei von Temperatur.}
So müssen z.B. die Löcher eines Blasinstruments und die Bünde der Gitarre für eine bestimmte temperierte Tonleiter in einem entsprechenden Abstand angeordnet sein.
Die Geige ist ein ziemlich cleveres Instrument, weil es alle Probleme mit der Temperatur dadurch vermeidet, daß die leeren Saiten zueinander einen Abstand von einer Quinte haben.
Wenn man die A(440)-Saite richtig stimmt und alle anderen Saiten dazu in Quinten, dann sind die anderen rein und nicht temperiert.
Man kann Probleme mit der Temperatur auch vermeiden, indem man alle Noten mit Ausnahme des A(440) greift.
Außerdem ist das Vibrato größer als die Korrekturen der Temperatur, was die Differenzen der Temperatur unhörbar werden läßt.

\textbf{Die Erfordernis für das Temperieren entsteht, weil eine chromatische Tonleiter, die auf eine Tonart gestimmt ist (z.B. C-Dur mit reinen Intervallen), in anderen Tonarten keine akzeptablen Intervalle erzeugt.}
Wenn man eine Komposition in C-Dur, die viele reine Intervalle enthält, transponiert, kann das zu schrecklichen Dissonanzen führen.
Es gibt ein noch grundlegenderes Problem.
Reine Intervalle in einer Tonart erzeugen auch Dissonanzen in anderen Tonarten, die im selben Musikstück benutzt werden.
Die Temperierschemata wurden deshalb dafür entwickelt, diese Dissonanzen zu minimieren, indem man die Verstimmung der reinen Intervalle bei den wichtigsten Intervallen minimierte und die größten Dissonanzen in die weniger benutzten Intervalle verschob.
Die zum schlimmsten Intervall gehörende Dissonanz wurde \enquote{Wolfsquinte} genannt.

Das Hauptproblem ist natürlich die Intervallreinheit; die obige Diskussion macht klar, daß egal was man tut, irgendwo eine Dissonanz auftreten wird.
\textbf{Es mag für manche ein Schock sein, daß das Klavier im Grunde ein unvollkommenes Instrument ist!}
Wir werden deshalb in jeder Tonleiter immer mit einigen Kompromissen bei den Intervallen leben müssen.

Der Name \enquote{chromatische Tonleiter} wird im allgemeinen auf jede zwölfnotige Tonleiter mit beliebiger Temperatur angewandt.
Natürlich erlaubt die chromatische Tonleiter des Klaviers nicht die Benutzung von Frequenzen zwischen den Noten (wie man das bei der Geige tun kann), so daß es eine unendliche Zahl fehlender Noten gibt.
In diesem Sinne ist die chromatische Tonleiter unvollständig.
Nichtsdestoweniger ist die zwölfnotige Tonleiter für die meisten musikalischen Anwendungen genügend vollständig.
Die Situation ist einer digitalen Fotografie analog.
Wenn die Auflösung ausreichend ist, kann man den Unterschied zwischen einem digitalen Foto und einem analogen Foto mit viel höherer Informationsdichte nicht sehen.
Ähnlich \textbf{hat die zwölfnotige Tonleiter offenbar für eine genügend große Anzahl musikalischer Anwendungen eine ausreichende Auflösung in der Tonhöhe.}
Diese zwölfnotige Tonleiter ist für ein bestimmtes Instrument oder musikalisches Notationssystem mit begrenzter Zahl zur Verfügung stehender Noten ein guter Kompromiß zwischen \enquote{mehr Noten je Oktave für eine größere Vollständigkeit haben} und \enquote{genug Frequenzbereich haben, um den Bereich des menschlichen Gehörs abzudecken}.

Es gibt eine fruchtbare Debatte darüber, welche Temperatur musikalisch gesehen am besten ist.
ET war von der frühesten Geschichte des Temperierens an bekannt.
Es hat definitiv Vorteile, auf eine Temperatur zu standardisieren, aber das ist hinsichtlich der Unterschiedlichkeit der Meinungen über Musik und der Tatsache, daß viel der zur Zeit existierenden Musik mit dem Gedanken an eine bestimmte Temperatur geschrieben wurde, wahrscheinlich nicht möglich oder sogar nicht wünschenswert.
Deshalb werden wir nun die verschiedenen Temperaturen erforschen.
 

\label{c2_2c}
\subsubsection{Temperatur und Musik}
\label{c2_2_temp} 

Der obige \hyperref[c2_2b]{mathematische Ansatz} ist nicht die Art und Weise, in der die chromatische Tonleiter entwickelt wurde.
Musiker begannen zunächst mit Intervallen und versuchten, eine Tonleiter mit einer minimalen Anzahl Noten zu finden, die diese Intervalle erzeugen würde.
Die Erfordernis einer minimalen Anzahl von Noten ist offensichtlich wünschenswert, weil diese die Anzahl der Tasten, Saiten, Löcher, usw. bestimmt, die für die Konstruktion eines Musikinstruments notwendig sind.
Intervalle sind notwendig, denn wenn man mehr als eine Note gleichzeitig spielen möchte, werden diese Noten für das Ohr unangenehme Dissonanzen erzeugen, außer wenn sie harmonische Intervalle bilden.
Der Grund, warum Dissonanzen so unangenehm für das Ohr sind, hat eventuell etwas mit der Schwierigkeit zu tun, mit dem Gehirn dissonante Informationen zu verarbeiten.
Es ist sicherlich hinsichtlich des Gedächtnisses und Verständnisses leichter, sich mit harmonischen Intervallen als mit Dissonanzen zu befassen, wobei es bei einigen davon für die meisten Menschen fast unmöglich ist, herauszufinden, ob zwei dissonante Noten gleichzeitig gespielt werden.
Deshalb wird es, wenn das Gehirn mit der Aufgabe komplexe Dissonanzen zu erkennen überlastet ist, unmöglich zu entspannen und die Musik zu genießen oder die musikalische Idee zu verfolgen.
Sicherlich muß jede Tonleiter gute Intervalle erzeugen, wenn wir fortgeschrittene, komplexe Musik komponieren sollen, die mehr als eine Note gleichzeitig erfordert.

\textbf{Die optimale Anzahl Noten in einer Tonleiter stellte sich als 12 heraus.
Leider gibt es keine zwölfnotige Tonleiter, die überall reine Intervalle erzeugt.
Musik würde besser klingen, wenn eine Tonleiter, die nur aus reinen Intervallen besteht, gefunden werden könnte.}
Viele solcher Versuche wurden bereits unternommen, hauptsächlich durch das Erhöhen der Notenanzahl je Oktave und besonders bei Gitarren und Orgeln, aber keine dieser Tonleitern hat eine Akzeptanz erreicht\footnote{zumindest in der \enquote{westlichen} Musik, in der Musik anderer Kulturen gibt es durchaus Tonleitern mit mehr als 20 Tönen je Oktave}.
Es ist relativ leicht, die Zahl der Noten mit einem gitarrenähnlichen Instrument zu erhöhen, weil man nur Saiten und Bünde hinzufügen muß.
Die neuesten Verfahren, die heute entwickelt werden, beziehen computergenerierte Tonleitern mit ein, bei denen der Computer die Frequenzen bei jeder Transposition justiert; dieses Verfahren wird adaptives Stimmen (Sethares) genannt.

\textbf{Das grundlegendste Konzept, das benötigt wird, um Temperaturen zu verstehen, ist das Konzept des Quintenzirkels.}
Nehmen Sie, um einen Quintenzirkel zu beschreiben, eine beliebige Oktave.
Beginnen Sie mit der tiefsten Note, und gehen Sie in Quinten aufwärts.
Nach zwei Quinten kommen Sie über diese Oktave hinaus.
Wenn das geschieht, gehen Sie eine Oktave nach unten, so daß Sie weiter in Quinten aufwärts gehen können und immer noch in der ursprünglichen Oktave bleiben.
Machen Sie das für zwölf Quinten, und Sie werden bei der höchsten Note der Oktave ankommen!
D.h. wenn Sie mit C4 anfangen, kommen Sie am Ende zu C5, und deshalb wird es ein Zirkel\footnote{lat. circulus = Kreis} genannt.
Nicht nur das, sondern jede Note, auf die Sie treffen, wenn Sie die Quinten spielen, ist eine andere Note.
Das bedeutet, daß der Quintenzirkel jede Note trifft, und das nur einmal, was eine nützliche Schlüsseleigenschaft für das Stimmen der Tonleiter ist und dafür, sie mathematisch zu untersuchen.
 

\label{c2_2_hist}

\textbf{Historische Entwicklungen sind ein zentrales Thema der Diskussionen über Temperaturen, weil die Musik aus einer Zeit mit der Temperatur aus dieser Zeit verbunden ist.
Pythagoras wird zugeschrieben, daß er ungefähr 550 v. Chr. unter Benutzung des Quintenzirkels die \enquote{pythagoreische Stimmung} erfunden hat, bei der die chromatische Tonleiter durch das Stimmen mit reinen Quinten erzeugt wird.}
Die zwölf reinen Quinten im Quintenzirkel bilden keinen exakten Faktor 2.
Deshalb ist die letzte Note, die man bekommt, nicht genau die Oktavnote, sondern ist in der Frequenz um den Wert zu hoch, den man \enquote{pythagoreisches Komma} nennt, d.h. ungefähr 23 Cent (ein Cent ist ein Hundertstel eines Halbtonschritts).
Da eine Quarte und eine Quinte eine Oktave bilden, resultiert die pythagoreische Stimmung in einer Tonleiter mit reinen Quarten und Quinten, wobei man allerdings am Ende eine sehr schlechte Dissonanz bekommt.
Es stellt sich heraus, daß mit reinen Quinten zu stimmen, zu unreinen Terzen führt.
Das ist ein weiterer Nachteil der pythagoreischen Stimmung.
Wenn nun jemand stimmen sollte, indem er jede Quinte um 23/12 Cent zusammenzieht, dann hätte er am Ende genau eine Oktave, und das ist eine Möglichkeit, eine \hyperref[et1]{ET-Tonleiter} zu stimmen.
Wir werden im Abschnitt über das Stimmen tatsächlich \hyperref[c2_6_et]{solch eine Methode benutzen}.
Die ET-Tonleiter war schon ca. 100 Jahre nach der Erfindung der pythagoreischen Stimmung bekannt.
Deshalb ist die ET keine \enquote{moderne Temperatur}.

Alle neueren Temperaturen, die auf die Einführung der pythagoreischen Stimmung folgten, waren Bemühungen, diese zu verbessern.
Die erste Methode war, das pythagoreische Komma zu halbieren und auf die letzten beiden Quinten zu verteilen.
\textbf{Eine wichtige Entwicklung war die mitteltönige Stimmung, bei der die Terzen statt der Quinten rein gemacht wurden.}
Musikalisch spielen Terzen eine bedeutendere Rolle als die Quinten, so daß die mitteltönige Stimmung sinnvoll war, besonders in einer Zeit, in der die Musik mehr Gebrauch von den Terzen machte.
Unglücklicherweise hat die mitteltönige Stimmung eine Wolfsquinte, die schlimmer als die der pythagoreischen Stimmung ist.
 

\label{c2_2_wtk2}

Der nächste Meilenstein wird von Bachs \enquote{Das Wohltemperirte Clavier} markiert, in dem er Musik für verschiedene Wohltemperierte Stimmungen (WT) geschrieben hat.
Das waren Temperaturen, die einen Kompromiß zwischen mitteltöniger und pythagoreischer Stimmung darstellten.
Dieses Konzept funktionierte, weil die pythagoreische Stimmung zu Noten führt, die zu \textit{hoch} sind, während die mitteltönige zu Noten führt, die zu \textit{tief} sind.
Außerdem boten die WT nicht nur die Möglichkeit guter Terzen, sondern auch von guten Quinten.
\textbf{Die einfachste WT wurde von Kirnberger, einem Schüler Bachs, entworfen.
Der größte Vorteil der Temperatur von Kirnberger ist ihre Einfachheit.
Bessere WTs wurden von Werckmeister und von Young entwickelt.
Wenn wir die Stimmungen allgemein in mitteltönig, WT und pythagoreisch einteilen, dann ist ET eine WT, weil ET weder erhöht noch erniedrigt ist.}
Es gibt keine Aufzeichnungen darüber, welche Temperatur(en) Bach benutzte.
Wir können die Temperatur(en) nur anhand der Harmonien in seinen Kompositionen vermuten, insbesondere seines \enquote{Wohltemperierten Klaviers}, und diese Studien zeigen, daß im Grunde alle Details des Temperierens bereits zu Bachs Zeiten (vor 1700) bekannt waren, und daß Bach eine Temperatur benutzte, die sich von der von Werckmeister nicht sehr unterschied.

Die Violine scheint einen Vorteil aus ihrem einzigartigen Aufbau zu ziehen, um diese Temperaturprobleme zu umgehen.
Die leeren Saiten bilden miteinander Quintintervalle, so daß sie von Natur aus pythagoreisch gestimmt ist.
Da die Terzen immer rein gespielt werden können, hat sie alle Vorteile der pythagoreischen, mitteltönigen und WT-Stimmung, und weit und breit ist keine Wolfsquinte in Sicht!

In den letzten ca. 100 Jahren wurde ET fast überall akzeptiert.
Deshalb werden die anderen Temperaturen im allgemeinen als \enquote{historische Temperaturen} eingestuft, was klar ein falsche Bezeichnung ist.
Der historische Gebrauch der WT führte zu dem Konzept der Tonartfarbe, bei dem jede Tonart in Abhängigkeit von der Stimmung der Musik besondere Farbe verlieh, und zwar hauptsächlich durch die kleinen Verstimmungen, die \enquote{Spannung} und andere Effekte erzeugen.
Das komplizierte die Lage sehr, weil die Musiker sich nun nicht nur mit reinen Intervallen und Wolfsquinten befassen mußten, sondern auch mit Farben, die nicht so leicht zu definieren waren.
Das Ausmaß, in dem die Farben herausgebracht werden können, hängt vom Klavier, dem Pianisten und dem Zuhörer genauso ab wie vom Stimmer.
Beachten Sie, daß der Stimmer die Streckung (s. \enquote{\hyperref[c2_5_stre]{Was ist Streckung?}} am Ende von Abschnitt 5) mit der Temperatur verbinden kann, um die Farbe zu kontrollieren.
Nachdem man Musik gehört hat, die auf einem Klavier gespielt wird, das WT gestimmt ist, klingt ET eher trüb und farblos.
Deshalb ist die Farbe der Tonart wichtig.
Wichtiger sind die wundervollen Klänge von reinen (gestreckten) Intervallen bei WT.
Auf der anderen Seite gibt es in den WTs immer eine Art von Wolfsquinte, die bei der ET reduziert ist.

Zum Spielen der meisten Musik, die um die Zeit von Bach, Mozart und Beethoven komponiert wurde, ist WT am besten geeignet.
So hat Beethoven z.B. für die dissonanten Nonen im ersten Satz seiner Mondschein-Sonate Akkorde gewählt, die in WT am wenigsten dissonant und in ET viel schlechter sind.
Diese großen Komponisten waren sich der Temperatur genauestens bewußt.
Die meisten Werke aus der Zeit von Chopin oder Liszt wurden im Hinblick auf ET komponiert, so daß die Tonartfarbe kein Thema ist.
Obwohl diese Kompositionen für das geschulte Ohr in ET und WT unterschiedlich klingen, ist nicht klar, daß gegen WT etwas einzuwenden ist, weil reine Intervalle immer besser klingen als verstimmte.

Meine persönliche Ansicht hinsichtlich des Klaviers ist, daß wir von ET abkommen sollten, weil sie uns eines der angenehmsten Aspekte der Musik beraubt - reinen Intervallen.
Sie werden eine dramatische Demonstration davon erleben, wenn Sie den letzten Satz von Beethovens Waldstein-Sonate in ET und WT hören.
Mitteltönige Stimmung kann ziemlich extrem sein, es sei denn Sie spielen Musik dieser Periode (vor Bach), so daß uns die WTs bleiben.
Hinsichtlich der Einfachheit und der leichten Stimmbarkeit ist Kirnberger nicht zu schlagen.
Ich glaube, daß wenn Sie sich an WT gewöhnt haben, ET nicht genauso gut klingen wird.
Deshalb sollte die Welt die WTs zum Standard erheben.
Welche man auswählt, macht für die meisten Menschen keinen großen Unterschied, weil diejenigen, die nicht in den Temperaturen ausgebildet sind, im allgemeinen keinen großen Unterschied zwischen den hauptsächlichen Temperaturen bemerken, geschweige denn zwischen den unterschiedlichen WTs.
Das soll nicht heißen, daß wir alle Kirnberger benutzen sollten, sondern daß wir in den Temperaturen ausgebildet werden und eine Wahl haben sollten, anstatt in die Zwangsjacke der farblosen ET gesteckt zu werden.
Das ist nicht nur eine Frage des Geschmacks oder die Frage, ob die Musik besser klingt.
Wir sprechen darüber, unsere musikalische Sensibilität zu entwickeln und zu wissen, wie man diese wirklich reinen Intervalle benutzt.
Ein Nachteil von WT ist, daß es hörbar wird, wenn das Klavier auch nur ein wenig verstimmt ist.
Es würde mich jedoch freuen, wenn alle Klavierschüler ihre Sensibilität bis zu dem Punkt entwickeln würden, an dem sie bereits erkennen können, wenn das Klavier auch nur ein wenig verstimmt ist.



