% File: c1ii5

\label{c1ii5}

% zuletzt geändert 05.09.2009

\subsection{Die schwierigen Abschnitte zuerst üben}

Kehren wir zu \enquote{Für Elise} zurück.
Es gibt zwei schwierige Abschnitte mit 16 und 23 Takten.
\textbf{Fangen Sie an, das Stück zu lernen, indem Sie die schwierigsten Abschnitte zuerst üben.}
Es wird am längsten dauern, diese zu lernen, deshalb sollten Sie die meiste Übungszeit darauf verwenden.
Darum fangen wir damit an, dass wir diese beiden schwierigen Abschnitte in Angriff nehmen.
Da das Ende der meisten Stücke im Allgemeinen am schwierigsten ist, werden Sie wahrscheinlich von den meisten Stücken das Ende zuerst lernen.
 

\subsection{Schwierige Passagen kürzen - In kleinen Portionen üben (taktweise)}
\label{c1ii6}

\textbf{Ein sehr wichtiger Lerntrick ist, einen kurzen Ausschnitt für das Üben zu wählen.}
Dieser Trick hat aus vielen Gründen vielleicht die größte Auswirkung auf das Reduzieren der Übungszeit.

\textbf{Innerhalb eines schwierigen Abschnitts von sagen wir zehn Takten, gibt es typischerweise nur wenige Notenkombinationen, die Sie in die Klemme bringen.
Es ist nicht notwendig, etwas anderes als diese Noten zu üben.}
Lassen Sie uns die zwei schwierigen Abschnitte in \enquote{Für Elise} untersuchen und die schwierigsten Stellen finden.
Das können der erste Takt oder die letzten fünf Takte der ersten Unterbrechung (Takte 45 bis 56\footnote{25 bis 36, wenn die Wiederholungen mit Voltenklammern notiert werden und die Takte ohne die Wiederholungen zu berücksichtigen einfach fortlaufend durchgezählt werden}) oder das letzte Arpeggio der zweiten Unterbrechung (Takte 82 bis 105\footnote{62 bis 85}) sein.
In allen schwierigen Ausschnitten ist es von entscheidender Bedeutung, die Fingersätze zu beachten.
Bei den letzten fünf Takten der ersten Unterbrechung liegt die Schwierigkeit in der rechten Hand, wobei die Finger 1 und 5 die meiste Arbeit haben.
In Takt 52\footnote{32} (mit dem Doppelschlag) ist der Fingersatz 2321231, und in Takt 53\footnote{33} ist er 251515151525.
Benutzen Sie für das Arpeggio in der zweiten Unterbrechung den Fingersatz 1231354321 usw.
Sowohl Daumenuntersatz als auch Daumenübersatz (s. \hyperref[c1iii5]{Abschnitt III.5}) wird funktionieren, weil diese Passage nicht übermäßig schnell ist, aber ich bevorzuge den Daumenübersatz, weil der Daumenuntersatz eine Bewegung des Ellbogens erfordert und diese zusätzliche Bewegung zu Fehlern führen kann.

\textbf{Kurze Ausschnitte zu üben gestattet es Ihnen, diese  innerhalb von Minuten dutzende, ja hunderte, Male zu üben.}
Der Gebrauch dieser schnellen Wiederholungen ist der schnellste Weg, um Ihrer Hand neue Bewegungen beizubringen.
Wenn die schwierigen Noten als Teil eines längeren Abschnitts gespielt werden, kann der längere Abstand zwischen den Wiederholungen und dem Spielen von anderen Noten dazwischen die Hand durcheinander bringen und dazu führen, dass Sie langsamer lernen.
Diese höhere Lerngeschwindigkeit wird in \hyperref[c1iv5]{Abschnitt IV.5} mengenmäßig berechnet, und diese Berechnung ist die Basis für die Behauptung in diesem Buch, dass diese Methoden tausendmal schneller als die intuitiven Methoden sein können.

Wir wissen alle, dass es abträglich ist, schneller zu spielen, als es Ihre Technik erlaubt.
Jedoch, \textbf{je kürzer der Ausschnitt ist, den Sie wählen, desto schneller können Sie ihn ohne schädliche Auswirkungen üben}, weil er so viel einfacher zu spielen ist.
Deshalb können Sie die meiste Zeit \textit{mit der endgültigen Geschwindigkeit oder schneller} spielen, was der Idealzustand ist, da es so viel Zeit spart.
Mit der intuitiven Methode üben Sie hingegen die meiste Zeit mit niedriger Geschwindigkeit.
 

\subsection{Die Hände getrennt (einhändig, \hyperref[HsHt]{HS}) üben - Erlernen der Spieltechnik}
\label{c1ii7}

\textbf{Im Grunde wird die Entwicklung der Technik zu 100\% durch das getrennte Üben der Hände erreicht.}
Versuchen Sie nicht, Finger- oder Hand-Technik mit beiden Händen zusammen zu entwickeln, weil es viel schwieriger, zeitaufwendiger und \textit{gefährlicher} ist, wie später im einzelnen erklärt wird.

Wählen Sie zwei kurze Passagen, jeweils eine für die rechte Hand und eine für die linke Hand.
\textbf{Üben Sie mit der rechten Hand, bis sie anfängt müde zu werden.
Wechseln Sie dann zur linken Hand.
Wechseln Sie alle 5 bis 15 Sekunden, bevor entweder die ruhende Hand abkühlt und träge wird oder die arbeitende Hand müde wird.}
Wenn Sie die Erholungspause gerade richtig wählen, werden Sie feststellen, dass die ausgeruhte Hand förmlich darauf wartet, etwas zu tun.
Üben Sie nicht, wenn die Hand müde ist, weil das zu Stress und schlechten Angewohnheiten führt.
Wer mit dem getrennten Üben der Hände nicht vertraut ist, hat im Allgemeinen eine \hyperref[c1ii20]{schwächere linke Hand}.
Geben Sie in diesem Fall der linken Hand mehr Arbeit.
Auf diese Weise können Sie 100\% der Zeit hart üben, werden aber nie mit ermüdeten Händen üben!

Üben Sie die zwei schwierigen Abschnitte von \enquote{Für Elise} mit getrennten Händen, bis Sie die Abschnitte mit jeder einzelnen Hand zufriedenstellend schneller als mit der endgültigen Geschwindigkeit spielen können, bevor Sie die Hände zusammen nehmen.
Dies kann in Abhängigkeit Ihrer Spielstärke ein paar Tage bis einige Wochen dauern.
Sobald Sie einhändig ziemlich gut spielen können, versuchen Sie es beidhändig, um zu überprüfen, dass der Fingersatz funktioniert.

\textbf{Es sollte betont werden, dass das getrennte Üben der Hände nur für schwierige Passagen gedacht ist, die Sie nicht spielen können.}
Wenn Sie die Passage angemessen beidhändig spielen können, können Sie den einhändigen Teil selbstverständlich übergehen!
Der eigentliche Zweck dieses Buchs ist, dass Sie, wenn Sie das Klavierspielen beherrschen, schnell in der Lage sind, praktisch ohne einhändig zu üben, beidhändig zu spielen.
Der Zweck ist nicht, eine Abhängigkeit vom einhändigen Spielen zu pflegen.
Spielen Sie nur einhändig, wenn es notwendig ist, und versuchen Sie, es allmählich zu reduzieren, wenn sich Ihre Technik verbessert.
Sie werden jedoch nur in der Lage sein, mit wenig einhändigem Üben beidhändig zu spielen, nachdem Sie sehr fortgeschritten sind - die meisten Schüler werden fünf bis zehn Jahre vom einhändigen Üben abhängig sein und seinen Gebrauch nie ganz aufgeben.
Der Grund dafür ist, dass die ganze Technik am schnellsten einhändig erworben wird.
Für die Option, das einhändige Üben auszulassen, gibt es eine Ausnahme.
Das ist das Auswendiglernen; aus mehreren wichtigen Gründen (siehe \enquote{\hyperref[c1iii6]{Auswendiglernen}} in Abschnitt III.6) sollten Sie alles einhändig auswendig lernen.
Obwohl Sie vielleicht nicht einhändig üben müssen, sollten Sie deshalb einhändig auswendig lernen, außer wenn Sie ein fortgeschrittener Klavierspieler mit einem guten \hyperref[c1ii12mental]{mentalen Spielen} sind.
Solche fortgeschrittenen Themen besprechen wir später.

\textbf{Anfänger sollten alles, was sie lernen, stets einhändig üben, um diese entscheidend wichtige Methode so schnell wie möglich zu beherrschen.}
Mit dem einhändigen Üben erwerben Sie die Finger- und Handtechnik; beim nachfolgenden beidhändigen Üben müssen Sie dann nur noch lernen, die beiden Hände zu koordinieren.
Indem Sie diese Aufgaben voneinander trennen, lernen Sie beides besser und schneller.
Wenn man die einhändige Methode erst einmal beherrscht, sollte man damit experimentieren, beidhändig zu spielen ohne vorher einhändig zu spielen.
Die meisten Schüler sollten in der Lage sein, die einhändigen Methoden in zwei bis drei Jahren zu beherrschen.
\textbf{Die einhändige Methode trennt nicht bloß die Hände.
Was wir im Folgenden lernen werden, sind die Myriaden von Lerntricks, die Sie benutzen können, wenn die Hände erst getrennt sind.}

\textbf{Das getrennte Üben der Hände ist lange nachdem Sie ein Stück gelernt haben wertvoll.}
Sie können Ihre Technik einhändig viel weiter vorantreiben als beidhändig.
Und es macht viel Spaß!
Sie können Finger, Hände und Arme wirklich trainieren.
Die einhändige Methode ist allem überlegen, was \hyperref[c1iii7h]{Hanon} oder andere Übungen zur Verfügung stellen können.
Das ist der Zeitpunkt, an dem Sie \enquote{unglaubliche Arten} herausfinden können, ein Stück zu spielen.
Dabei können Sie Ihre Technik \textit{wirklich} verbessern.
Das anfängliche Lernen einer Komposition dient nur dazu, die Finger mit der Musik vertraut zu machen.
Die Menge der Zeit, die man mit dem Spielen von Stücken verbringt, die man vollständig beherrscht, unterscheidet den erfahrenen Pianisten vom Amateur.
Deshalb können erfahrene Pianisten \hyperref[c1iii14]{vorspielen}, aber die meisten Amateure können nur für sich selbst spielen.


\subsection{Die Kontinuitätsregel}
\label{c1ii8}

\textbf{Wenn Sie einen Ausschnitt üben, beziehen Sie immer den Anfang des folgenden Ausschnitts mit ein.}
Diese Kontinuitätsregel stellt sicher, dass Sie zwei aufeinanderfolgende Ausschnitte, die sie gelernt haben, auch zusammen spielen können.
Sie ist für jeden Ausschnitt anwendbar, den Sie zum Üben isolieren, wie einen Takt, einen ganzen Satz oder sogar Ausschnitte kleiner als einen Takt.
\textbf{Eine Verallgemeinerung der Kontinuitätsregel ist, dass jede Passage für das Üben in kurze Ausschnitte aufgeteilt werden kann, dass diese Ausschnitte sich aber überlappen müssen.
Die überlappende Note oder Gruppe von Noten wird Verbindung genannt.}
Wenn Sie das Ende des ersten Satzes üben, dann schließen Sie einige Takte des zweiten Satzes mit ein.
Während eines Konzerts werden Sie froh sein, dass Sie so geübt haben; es könnte Ihnen sonst passieren, dass Sie plötzlich nicht mehr wissen, wie Sie den zweiten Satz anfangen müssen!

Wir können nun die Kontinuitätsregel auf diese schwierigen Unterbrechungen in \enquote{Für Elise} anwenden.
Um Takt 53 zu üben, fügen Sie die erste Note von Takt 54 hinzu (E gespielt mit Finger 1), welche die Verbindung ist.
Da alle schwierigen Abschnitte für die rechte Hand sind, sollten Sie etwas Material für die linke, sogar aus anderen Musikstücken, zum Üben finden, um der rechten durch das Abwechseln der Hände periodische Pausen zu geben.


\subsection{Der Akkord-Anschlag}
\label{c1ii9}

Angenommen, Sie möchten mit der linken Hand ein CGEG-Quadrupel (Alberti-Begleitung) viele Male sehr schnell hintereinander spielen (wie im dritten Satz von Beethovens Mondschein-Sonate).
Die Folge, die Sie üben, ist CGEGC, wobei das letzte C die Verbindung ist.
Da die Verbindung dieselbe wie die erste Note ist, können Sie dieses Quadrupel undendlich \hyperref[c1iii2]{zirkulieren} ohne aufzuhören.
Wenn Sie es einhändig langsam üben und schrittweise schneller werden, treffen Sie auf eine \enquote{Geschwindigkeitsbarriere}, eine Geschwindigkeit, nach der alles zusammenbricht und Stress entsteht.
\textbf{Die Möglichkeit, diese Barriere zu durchbrechen, ist, das Quadrupel als einen einzigen Akkord zu spielen (CEG).
Sie sind von langsamer Geschwindigkeit zu unendlicher Geschwindigkeit übergegangen!
Das wird als Akkord-Anschlag bezeichnet.}
Nun müssen Sie nur noch lernen, langsamer zu werden, was einfacher ist als schneller zu werden, weil es keine Geschwindigkeitsbarriere gibt, wenn Sie langsamer werden.
Die Kernfrage ist: Wie wird man langsamer?

\textbf{Spielen Sie zunächst den Akkord, und lassen Sie die Hand in der Frequenz auf und ab springen\footnote{gemeint ist \enquote{wie ein Ball}}, in der das Quadrupel wiederholt wird} (sagen wir zwischen ein- und zweimal je Sekunde); das lehrt die Hand, das Handgelenk, den Arm, die Schulter usw., was sie für die schnellen Wiederholungen tun müssen und trainiert die entsprechenden Muskeln.
Beachten Sie, dass die Finger jetzt in der richtigen Position für ein schnelles Spielen sind; sie ruhen bequem auf den Tasten und sind leicht gebogen.
Senken und erhöhen Sie die Spring-Frequenz (sogar über die erforderliche Geschwindigkeit hinaus!).
Beachten Sie dabei, wie Sie die Positionen und Bewegungen des Handgelenks, des Arms, der Finger usw. verändern müssen, um die Bequemlichkeit zu maximieren und Ermüdung zu vermeiden.
Wenn Sie sich nach einer Weile müde fühlen, machen Sie entweder etwas falsch oder Sie haben sich noch nicht die Technik angeeignet, die Akkorde wiederholt zu spielen.
Üben Sie es, bis Sie spielen können, ohne zu ermüden, denn wenn Sie es nicht für einen Akkord tun können, dann werden Sie es auch nie für Quadrupel können.

Behalten Sie die Finger nahe über oder auf den Tasten, wenn Sie die Geschwindigkeit steigern.
Beziehen Sie den ganzen Körper mit ein: Schultern, Ober- und Unterarme, Handgelenk.
Das Gefühl ist, aus den Schultern und Armen heraus zu spielen, nicht den Fingerspitzen.
Wenn Sie das leise, entspannt, schnell und ohne jedes Müdigkeitsgefühl spielen können, dann haben Sie Fortschritte gemacht.
Achten Sie darauf, dass die Akkorde perfekt sind (alle Noten beginnen zur gleichen Zeit), denn ohne diese Art von Empfindlichkeit werden Sie nie die Genauigkeit haben, um schnell zu spielen.\footnote{\enquote{Digital-Pianisten} haben hier zwar einen Vorteil, weil sie ihr Spiel \hyperref[c1iii13MIDI]{aufnehmen und die MIDI-Signale ansehen} können, sollten aber trotzdem die Kontrolle durch das Gehör trainieren.}
\textbf{Es ist wichtig, langsam zu üben, weil Sie so an der Genauigkeit und der \hyperref[c1ii14]{Entspannung} arbeiten können.
Die Genauigkeit verbessert sich schneller bei den geringeren Geschwindigkeiten.}
Es ist jedoch absolut wesentlich, dass Sie zu schnelleren Geschwindigkeiten kommen (selbst wenn es nur kurz ist), bevor Sie langsamer werden.
Wenn Sie dann langsamer werden, versuchen Sie, die gleichen Bewegungen beizubehalten, die bei höherer Geschwindigkeit erforderlich waren, weil sie letzten Endes \textit{diese} Bewegungen üben müssen.



