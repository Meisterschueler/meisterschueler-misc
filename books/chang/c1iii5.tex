% File: c1iii5

\subsection{Schnell spielen: Tonleitern, Arpeggios und chromatische Tonleitern (Chopins Fantaisie Impromptu und Beethovens Mondschein-Sonate, 3. Satz)}
\label{c1iii5}

\subsubsection{Tonleitern: Daumenuntersatz, Daumenübersatz}
\label{c1iii5a}

\textbf{Tonleitern und Arpeggios sind die grundlegendsten Klavierpassagen; trotzdem wird die wichtigste Methode sie zu spielen oft überhaupt nicht gelehrt!}
Arpeggios sind einfach erweiterte Tonleitern und können deshalb ähnlich wie Tonleitern behandelt werden; darum werden wir zunächst die Tonleitern besprechen und dann beschreiben, wie ähnliche Regeln auf Arpeggios angewendet werden können.
Es gibt einen fundamentalen Unterschied darin, wie man ein Arpeggio im Vergleich zu einer Tonleiter spielen muss (mit einem flexiblen Handgelenk); wenn man diesen Unterschied erst einmal gelernt hat, werden Arpeggios viel einfacher, sogar für kleine Hände.

\textbf{Es gibt zwei Arten, eine Tonleiter zu spielen: Die erste ist der wohlbekannte Daumenuntersatz und die zweite der Daumenübersatz.}
Beim Untersatz wird der Daumen unter die Hand gebracht, damit der dritte oder vierte Finger zum Spielen der Tonleiter vorbeigehen kann.
Dieser Vorgang wird durch zwei Eigenschaften des Daumens erleichtert: Er ist kürzer als die anderen Finger und befindet sich unter der Handfläche.
\textbf{Beim Übersatz wird der Daumen wie die anderen vier Finger behandelt, was die Bewegung in hohem Maß vereinfacht.
Beide Methoden sind erforderlich, um eine Tonleiter zu spielen, aber jede wird unter verschiedenen Umständen benötigt}; der Übersatz wird für schnelle, technisch schwierige Passagen benötigt, und der Untersatz ist nützlich für langsame Legatopassagen oder wenn einige Noten gehalten werden müssen, während andere gespielt werden.

Den Begriff Daumenübersatz habe ich gewählt, weil es keine bessere Bezeichnung für die Methode gibt.
Es ist offensichtlich eine unzutreffende Bezeichnung und erschwert Anfängern eventuell, zu verstehen, wie man ihn spielt.
Ich habe es mit anderen Namen versucht, aber keiner davon ist besser als Daumenübersatz.
Der einzige mögliche Vorteil ist, dass diese offensichtliche Fehlbezeichnung die Aufmerksamkeit auf die Existenz des Daumenübersatzes lenkt.

Vielen Klavierlehrern war der Daumenübersatz überhaupt nicht bekannt.
Das war kaum ein Problem, solange die Schüler nicht auf fortgeschrittene Stufen kamen.
Tatsächlich ist es mit genügend Anstrengung und Arbeit möglich, ziemlich schwierige Passagen unter Verwendung des Daumenuntersatzes zu spielen, und \textbf{es gibt vollendete Pianisten, die glauben, dass der Untersatz die einzige Methode ist, die sie benötigen.
In Wahrheit haben sie unbewusst (durch sehr hartes Arbeiten) gelernt, den Untersatz für ausreichend schnelle Passagen so zu verändern, dass er  dem Übersatz nahe kommt.}
Diese Änderung ist notwendig, weil es für solch schnelle Tonleitern körperlich unmöglich ist, sie mit dem Untersatz zu spielen.
Deshalb ist es für die Schüler wichtig, mit dem Lernen des Übersatzes anzufangen, sobald sie über das Anfängerstadium hinaus sind, bevor die Angewohnheit verfestigt ist, den Untersatz für Passagen zu benutzen, die mit Übersatz gespielt werden sollten.

\textbf{Viele Schüler benutzen die Methode, am Anfang langsam zu spielen und die Geschwindigkeit dann zu steigern.
Sie kommen bei niedriger Geschwindigkeit gut mit dem Daumenuntersatz zurecht, gewöhnen sich infolgedessen den Untersatz an, und stellen, wenn sie zur vorgegebenen Geschwindigkeit kommen, fest, dass sie zum Daumenübersatz wechseln müssen.}
Dieser Wechsel kann eine sehr schwierige, frustrierende und zeitraubende Aufgabe sein, nicht nur für Tonleitern sondern auch für jeden schnellen Lauf -- ein weiterer Grund, warum die Methode der schrittweisen Steigerung der Geschwindigkeit in diesem Buch nicht empfohlen wird.
\textbf{Die Daumenuntersatzbewegung ist eine der am meisten verbreiteten Ursachen von Geschwindigkeitsbarrieren und Spielfehlern.
Wenn der Daumenübersatz erst einmal gelernt ist, sollte er deshalb immer benutzt werden, um Läufe zu spielen, außer wenn der Untersatz bessere Ergebnisse liefert.}

Die wichtigsten Muskeln des Daumens für das Klavierspielen sind, so wie die der anderen vier Finger, im Unterarm.
Der Daumen besitzt jedoch weitere Muskeln in der Hand, die benutzt werden, um den Daumen beim Untersatz seitwärts zu bewegen.
Das Einbeziehen dieser zusätzlichen Muskeln für die Untersatzbewegung macht diese zu einer komplexeren Operation und verlangsamt somit die maximal zu erreichende Geschwindigkeit.
Die zusätzliche Komplikation verursacht ebenfalls Fehler.
\textbf{Lehrer, die den Daumenübersatz lehren, behaupten über diejenigen, die ausschließlich den Daumenuntersatz benutzen, dass 90\% ihrer Fehler ihren Ursprung in der Untersatzbewegung haben.}

Man kann den Nachteil des Untersatzes demonstrieren, indem man den Verlust der Beweglichkeit des Daumens in seiner eingeschlagenen Position beobachtet.
Strecken Sie zunächst Ihre Finger so aus, dass alle Finger in derselben Ebene liegen.
Sie werden feststellen, dass alle Finger, einschließlich des Daumens, nach oben und unten beweglich sind (die Bewegung, die man zum Klavierspielen braucht).
Wackeln Sie nun mit dem Daumen schnell auf und ab -- Sie werden sehen, dass sich der Daumen vertikal leicht 3 oder 4 cm und ziemlich schnell bewegen kann (ohne den Unterarm zu drehen).
Ziehen Sie dann den Daumen, während Sie mit derselben schnellen Frequenz weiterwackeln, schrittweise unter die Hand -- Sie werden sehen, dass der Daumen, während er unter die Hand geht, seine vertikale Beweglichkeit verliert, bis er unbeweglich, fast gelähmt wird, wenn er unter dem Mittelfinger ist.

Hören Sie nun mit dem Wackeln auf, und stoßen Sie den Daumen nach unten (ohne das Handgelenk zu bewegen) -- er bewegt sich nach unten!
Das kommt daher, dass Sie nun eine andere Muskelgruppe benutzen.
Versuchen Sie dann, unter Benutzung dieser neuen Muskeln, den Daumen so schnell Sie können auf und ab zu bewegen -- Sie sollten finden, dass diese neuen Muskeln viel schwerfälliger sind und die Auf- und Abwärtsbewegung langsamer ist als die Wackelrate des Daumens, als er ausgestreckt war.
Damit Sie in der Lage sind, den Daumen in seiner eingeschlagenen Position zu bewegen, müssen Sie deshalb nicht nur diese neue Muskelgruppe benutzen, sondern diese Muskeln sind zusätzlich langsamer.
Es ist die Einführung dieser schwerfälligen Muskeln, die beim Daumenuntersatz Fehler verursacht und das Spielen verlangsamt.
Der Daumenübersatz eliminiert diese Probleme.

Tonleitern und Arpeggios gehören zu den in der Klavierpädagogik am meisten missbrauchten Übungen -- Anfängern wird nur der Untersatz gelehrt, wodurch sie unfähig bleiben, sich die richtigen Techniken für Läufe und Arpeggios anzueignen.
Nicht nur das, sondern wenn die Tonleiter beschleunigt wird, beginnt mysteriöserweise der Stress sich aufzubauen.
Schlimmer noch: Der Schüler baut ein großes Repertoire mit falschen Angewohnheiten auf, die mühevoll korrigiert werden müssen.
\textbf{Der Übersatz ist leichter zu lernen als der Untersatz, weil er nicht die Seitwärtsdrehungen von Daumen, Hand, Arm und Ellbogen erfordert.}
Der Daumenübersatz sollte gelehrt werden, sobald schnellere Tonleitern benötigt werden -- innerhalb der ersten beiden Unterrichtsjahre.
Anfängern sollte der Untersatz zuerst gelehrt werden, weil er für langsame Passagen notwendig ist und es länger dauert, ihn zu lernen.
Talentierten Schülern muss der Übersatz innerhalb der ersten Monate ihres Unterrichts gelehrt werden oder sobald sie den Untersatz beherrschen.

Da es zwei Möglichkeiten gibt, die Tonleitern zu spielen, gibt es zwei Schulen hinsichtlich der Art, zu lehren wie man sie spielt.
Die Daumenuntersatzschule (Czerny, Leschetizky) behauptet, dass der Untersatz die einzige Art ist, wie man Legato-Tonleitern spielen kann, und dass man mit genügend Übung Tonleitern bei jeder Geschwindigkeit mit Daumenuntersatz spielen kann.
Die Daumenübersatzschule (\hyperref[Whiteside]{Whiteside}, \hyperref[Sandor]{Sandor}) hat nach und nach die Oberhand gewonnen, und die hartnäckigeren Anhänger \textit{verbieten} die Benutzung des Untersatzes unter allen Umständen.
Sehen Sie dazu im \hyperref[reference]{Quellenverzeichnis} weitere Diskussionen über das Lehren von Daumenuntersatz gegenüber Daumenübersatz.
Beide extremen Schulen liegen falsch, weil man beide Fertigkeiten benötigt.

Die Daumenübersatzlehrer sind verständlicherweise über die Tatsache verärgert, dass fortgeschrittene Schüler, die von privaten Lehrern an sie weitergereicht werden, oft den Daumenübersatz nicht kennen und es sechs Monate oder länger dauert, allein die Stunden von Repertoire zu korrigieren, die sie auf die falsche Art gelernt haben.
Ein Nachteil davon, sowohl den Untersatz als auch den Übersatz zu lernen, ist, dass man beim \hyperref[c1iii11]{Spielen vom Blatt}\index{Spielen vom Blatt} mit dem Daumen durcheinanderkommen kann und nicht weiß, welchen Weg man nehmen soll.
Diese Verwirrung ist ein Grund, warum manche Lehrer der Übersatzschule tatsächlich den Gebrauch des Untersatzes verbieten.
Ich empfehle, dass Sie als Standard den Daumenübersatz benutzen und den Daumenuntersatz als Ausnahme der Regel.
Beachten Sie, dass Chopin beide Methoden lehrte (\hyperref[Eigeldinger]{Eigeldinger}, Seite 37).

Obwohl der Daumenübersatz durch Whiteside und andere wiederentdeckt wurde, geht der früheste Bericht über seinen Gebrauch mindestens auf Franz Liszt zurück (Fay).
Liszt ist dafür bekannt, dass er im Alter von ungefähr 20 Jahren für mehr als ein Jahr nicht mehr auftrat und seine Technik weiterentwickelte.
Er war mit seiner Technik unzufrieden (insbesondere wenn er Tonleitern spielte), wenn er sie mit den wunderbaren Darbietungen Paganinis auf der Geige verglich, und experimentierte mit dem Verbessern seiner Technik.
Am Ende dieser Periode war er mit seinen neuen Fertigkeiten zufrieden, konnte aber anderen nicht genau erklären, was er getan hatte, um sich zu verbessern -- er konnte es nur am Klavier vorführen (das galt für die meisten von Liszts \enquote{Lehren}).
Amy Fay erkannte jedoch, dass er die Tonleitern jetzt anders spielte; anstatt den Daumenuntersatz zu benutzen, \enquote{rollte er die Hand über den passierten Finger}, sodass der Daumen auf die nächste Taste fiel.
Es dauerte offensichtlich mehrere Monate, bis Fay diese Methode imitieren konnte, aber ihrer Aussage nach \enquote{änderte es ihre Art zu spielen völlig}, und sie behauptete, dass es allgemein zu einer deutlichen Verbesserung ihrer Technik führte, nicht nur beim Spielen von Tonleitern, weil der Daumenübersatz bei jedem Lauf und auch bei Arpeggios anwendbar ist.


\subsubsection{Daumenübersatz: Bewegung, Erklärung und Video}
\label{c1iii5b}

Lassen Sie uns damit anfangen, dass wir den grundlegenden Fingersatz für Tonleitern analysieren.
Nehmen Sie die C-Dur-Tonleiter der rechten Hand.
Wir beginnen mit dem leichtesten Teil, das heißt mit der absteigenden Tonleiter der rechten Hand, die 5432132,1432132,1 usw. gespielt wird.
Da der Daumen (1) unter der Hand ist, rollen der Mittelfinger (3) oder der Ringfinger (4) leicht über den Daumen, faltet sich der Daumen ganz natürlich unter diese Finger, und dieser Fingersatz für die absteigende Tonleiter funktioniert bestens.
Diese Bewegung ist im Grunde die Daumenuntersatzbewegung; die Bewegung für den absteigenden Übersatz ist ähnlich, wir werden jedoch eine leichte aber entscheidende Änderung vornehmen müssen, damit es zu einem wahren Übersatz wird; diese Änderung ist allerdings subtil und wird später besprochen.

Nehmen Sie nun die aufsteigende C-Dur-Tonleiter der rechten Hand.
Diese wird 1231234 usw. gespielt.
\textbf{Beim Daumenübersatz wird der Daumen genau so wie die Finger 3 und 4 bei der absteigenden Tonleiter gespielt; das heißt er wird einfach gehoben und gesenkt, ohne die Seitwärtsbewegung unter die Handfläche wie beim Daumenuntersatz.}
Da der Daumen kürzer als die anderen Finger ist, kann er fast parallel zu (und direkt hinter) dem passierten Finger herunterbewegt werden, ohne mit diesem zu kollidieren.
Um mit dem Daumen die richtige Taste zu treffen, werden Sie die Hand bewegen und mit dem Handgelenk ganz leicht \enquote{zucken} müssen.
Bei Tonleitern wie C-Dur sind sowohl der Daumen als auch der passierte Finger auf weißen Tasten und sind sich zwangsläufig ein wenig im Weg.
\textbf{Um jede Möglichkeit einer Kollision zu vermeiden, sollte der Arm fast 45 Grad zur Tastatur stehen (wobei er nach links zeigt), und die Hand rollt über den passierten Finger, indem sie diesen als Drehpunkt benutzt.
Der Finger 3 oder 4 muss dann schnell wegbewegt werden, wenn der Daumen herunterkommt.}
Beim Übersatz ist es, anders als beim Untersatz, nicht möglich, den Finger 3 oder 4 unten zu halten bis der Daumen spielt.
Wenn man den Übersatz das erste Mal ausprobiert, wird die Tonleiter ungleichmäßig sein, und es mag eine \enquote{Lücke} geben, wenn man mit dem Daumen spielt.
Deshalb muss der Übergang sehr schnell sein, auch wenn die Tonleiter langsam gespielt wird.
Wenn Sie Fortschritte machen, werden Sie merken, dass eine schnelle Rollung/Drehung des Handgelenks/Arms hilfreich ist.\footnote{Eine Anleitung für das Erlernen des Daumenübersatzes finden sie (in Englisch) unter anderem auch auf \enquote{http://www.pianostreet.com/smf/index.php/topic,7226.msg72166.html\#msg72166}.}
Anfänger finden den Übersatz üblicherweise leichter als den Untersatz, aber diejenigen, die den Untersatz viele Jahre gelernt haben, werden den Übersatz zunächst schwierig und ungleichmäßig finden.
Drehen Sie auch den Unterarm ein wenig im Uhrzeigersinn (Chopin nannte es die \enquote{Glissando-Position}, siehe unten in \hyperref[c1iii5c]{\autoref{c1iii5c}}), was den Daumen automatisch vorwärts bringt.
Die aufsteigende Tonleiter der rechten Hand ist schwieriger als die absteigende.
Bei der absteigenden Tonleiter dreht man mit dem Daumen und rollt über ihn, was einfach ist.
Bei der aufsteigenden Tonleiter rollt man aber über Finger 3 oder 4, und es gibt Finger oberhalb dieses Fingers, die beim Rollen stören können.

Die Logik hinter dem Daumenübersatz ist die folgende.
\textbf{Der Daumen wird wie jeder andere Finger benutzt.}
Der Daumen bewegt sich nur auf und ab.
Das vereinfacht die Fingerbewegungen, und außerdem müssen die Hände, Arme und Ellbogen nicht zum Anpassen an die Untersatzbewegungen verdreht werden.
So bleiben die Hand und der Arm immer im optimalen Winkel zur Tastatur und gleiten einfach mit der Tonleiter auf und ab.
Ohne diese Vereinfachung können technisch schwierige Passagen unmöglich werden, besonders weil man immer noch neue Handbewegungen hinzufügen muss, um solche Geschwindigkeiten zu erreichen, und die meisten dieser Bewegungen sind mit dem Daumenuntersatz nicht kompatibel.
Am wichtigsten ist, dass \textbf{die Bewegung des Daumens in seine korrekte Position hauptsächlich von der Hand gesteuert wird}, während es beim Untersatz die kombinierte Bewegung des Daumens und der Hand ist, die die Position des Daumens bestimmt.
Weil die Handbewegung weich ist, wird der Daumen genauer positioniert als beim Untersatz, was fehlende und falsch angeschlagene Noten reduziert und dem Daumen gleichzeitig eine bessere Klangkontrolle verleiht.
Auch wird die aufsteigende Tonleiter der absteigenden ähnlich, weil man die Finger zum Vorbeigehen immer \textit{über}setzt.
Das macht es auch einfacher, \hyperref[c1ii25]{beidhändig}\index{beidhändig} zu spielen, da alle Finger beider Hände immer übersetzen.
Ein weiterer Pluspunkt ist, dass der Daumen nun eine schwarze Taste spielen kann.
Es sind diese vielen Vereinfachungen, das Eliminieren des aus dem \enquote{gelähmten} Daumen resultierenden Stresses und noch mehr Vorteile, die weiter unten besprochen werden, welche die Möglichkeiten für Fehler reduzieren und ein schnelleres Spielen ermöglichen.
Es gibt natürlich Ausnahmen: Langsame Legatopassagen oder einige Tonleitern, die schwarze Tasten enthalten, usw. werden mit einer untersatzähnlichen Bewegung leichter ausgeführt.
\textbf{Die meisten Schüler, die nur den Daumenuntersatz benutzen, haben es am Anfang furchtbar schwer, wenn sie versuchen zu verstehen, wie jemand mit Daumenübersatz spielen kann}.
Das ist der deutlichste Hinweis auf den Schaden, der dadurch angerichtet werden kann, dass man den Übersatz nicht so früh wie möglich lernt; für diese Schüler ist der Daumen nicht \enquote{frei}.
Wir werden sehen, dass der freie Daumen ein vielseitiger Finger ist.
Aber verzweifeln Sie nicht, denn es stellt sich heraus, dass die meisten fortgeschrittenen Untersatzschüler bereits wissen, wie man den Übersatz spielt -- sie wissen es nur nicht.

Bei der linken Hand ist es umgekehrt wie bei der rechten; der Übersatz wird für die absteigende Tonleiter benutzt, und die aufsteigende Tonleiter ist dem Untersatz ziemlich ähnlich.
Wenn Ihre rechte Hand weiter fortgeschritten ist als die linke, führen Sie die Ausflüge zu höheren Geschwindigkeiten mit der rechten Hand durch, bis Sie sich genau entscheiden, was sie tun.
Nehmen Sie diese Bewegung dann für die linke Hand.

Da Schüler ohne Lehrer Schwierigkeiten haben, sich den Daumenübersatz vorzustellen, untersuchen wir einen Videoclip, der den Übersatz mit dem Untersatz vergleicht.
Ich habe dieses Video in zwei Formaten hochgeladen, da nicht jede Software alle Formate abspielen kann.
Falls Sie diesen Text nur als Ausdruck vorliegen haben, müssen Sie die URLs von Hand eingeben.
Öffnen Sie dann zuerst Ihr Videoprogramm, und suchen Sie anschließend nach dem Menüpunkt, mit dem Sie einen Link manuell eingeben können -- meistens unter \enquote{Datei}.
Nachfolgend finden Sie zwei URLs; eine davon sollte funktionieren.
\textit{[Es gibt keine spezifische Notation in HTML, mit der Dateien beim Anklicken automatisch heruntergeladen oder direkt in der auf dem PC zugewiesenen Anwendung gestartet werden.
(Fast) jedes Browser-Programm bietet aber anhand des Dateityps die eine oder andere Möglichkeit an.
Benutzen Sie ggf. zum Herunterladen die entsprechenden Funktionen Ihres Browsers, zum Beispiel einen \enquote{Rechtsklick} auf den Link machen und dann \enquote{Speichern unter} auswählen.
Falls Ihr Video-Programm die Möglichkeit bietet, schauen Sie sich das Video auch in Zeitlupe oder sogar bildweise an.  Die Unterschiede in den Bewegungen werden dann deutlicher.]}

\begin{itemize} 
 \item \hyperref[http://www.pianopractice.org/TOscale.mp4]{http://www.pianopractice.org/TOscale.mp4} (extern; ca. 3,1 MB)
 \item \hyperref[http://www.pianopractice.org/TOscale.wmv]{http://www.pianopractice.org/TOscale.wmv} (extern; ca. 1,6 MB)
 \end{itemize}
Das Video zeigt die rechte Hand beim Spielen von zwei Oktaven mit Daumenübersatz -- zweimal auf- und abwärts.
Das wird dann mit dem Daumenuntersatz wiederholt.
Für diejenigen, die keine Klavierspieler sind, mag das im Grunde dasselbe sein, obwohl die Untersatzbewegung leicht übertrieben ist.
Das zeigt, warum Videos von Bewegungen beim Klavierspielen nicht so hilfreich sind, wie man denken könnte.
Die aufsteigenden Übersatzbewegungen sind im Grunde korrekt.
Die absteigende Übersatzbewegung hat einen Fehler -- das Nagelglied des Daumens wird leicht gebeugt.
Bei diesen langsamen Geschwindigkeiten beeinflusst dieses leichte Beugen nicht das Spielen, aber beim strengen Übersatz sollte der Daumen sowohl beim aufsteigenden als auch beim absteigenden Spielen gerade bleiben.
Dieses Beispiel zeigt, wie wichtig es ist, den Übersatz so früh wie möglich zu lernen.
Meine Neigung, das Nagelglied zu beugen, ist das Ergebnis davon, dass ich über viele Jahrzehnte hinweg nur den Untersatz benutzt hatte, bevor ich den Übersatz lernte.
Eine wichtige Schlussfolgerung ist hier, \textbf{den Daumen beim Übersatz die ganze Zeit gerade zu halten}.



