% File: c1iii5b

\label{c1iii5e}
\subsubsection{Arpeggios (Chopin, Wagenradbewegung, \enquote{gespreizte} Finger)}
\label{Arpeggios}

\textbf{Arpeggios korrekt zu spielen ist technisch komplex.
Deshalb eignen sich Arpeggios besonders gut für das Lernen einiger wichtiger \hyperref[c1iii4]{Handbewegungen}\index{Handbewegungen}, wie Schub, Zug und die Wagenradbewegung.}
\enquote{Arpeggio}, so wie es hier benutzt wird, schließt gebrochene Akkorde und Kombinationen von kurzen arpeggioartigen Passagen ein.
Wir werden diese Konzepte hier verdeutlichen, indem wir den 3. Satz von Beethovens Mondschein-Sonate für den Schub und Zug und Chopins Fantaisie Impromptu (FI) für die Wagenradbewegung benutzen.
Erinnern Sie sich daran, dass die Geschmeidigkeit der Hände, insbesondere im Handgelenk, für das Spielen von Arpeggios entscheidend ist.
Die technische Komplexität der Arpeggios kommt von der Tatsache, dass in den meisten Fällen diese Geschmeidigkeit mit allem anderen kombiniert werden muss: Schub, Zug, Wagenradbewegung, Glissandobewegung (oder \enquote{gespreizte} Finger) und Daumenuntersatz oder Daumenübersatz.
Ein Warnhinweis: Die Mondschein-Sonate ist wegen der erforderlichen Geschwindigkeit schwierig.
Viele Kompositionen von Beethoven können nicht verlangsamt werden, weil sie so eng mit dem Rhythmus verbunden sind.
Außerdem erfordert dieser Satz, dass Sie mindestens eine None bequem greifen können.
Diejenigen mit kleineren Händen werden größere Schwierigkeiten haben, dieses Stück zu lernen, als diejenigen mit einer angemessenen Reichweite.

Lassen Sie uns zunächst besprechen, wie man Arpeggios mit Daumenübersatz spielt.
Arpeggios, die über mehrere Oktaven gehen, werden wie Tonleitern mit Übersatz gespielt.
Deshalb wissen Sie, wenn Sie Tonleitern mit Übersatz spielen können, im Prinzip, wie man Arpeggios mit Übersatz spielt.
Arpeggios mit Übersatz zu spielen, ist jedoch ein extremeres Beispiel für die Übersatzbewegung als Tonleitern und dient deshalb als das deutlichste Beispiel für diese Bewegung.
Wir haben oben festgestellt, dass die einfachste Übersatzbewegung jene ist, die beim Spielen von chromatischen Tonleitern benutzt wird (1313123131312 usw. für die rechte Hand).
Die chromatische Übersatzbewegung ist einfach, weil die horizontale Bewegung des Daumens gering ist.
Die nächste, etwas schwierigere Bewegung ist die zum Spielen der H-Dur-Tonleiter.
Diese Übersatzbewegung ist einfach, weil man die gesamte Tonleiter mit flachen Fingern spielen kann, sodass es kein Kollisionsproblem mit dem vorbeigehenden Daumen gibt.
Die nächstschwierigere ist die C-Dur-Tonleiter; sie ist schwieriger, weil alle Finger im engen Bereich der weißen Tasten zusammengedrängt sind.
Die schwierigste Bewegung ist schließlich das Arpeggio mit Übersatz, bei dem die Hand schnell und exakt bewegt werden muss.
Diese Bewegung erfordert eine leichte Beugung und eine kurze, schnelle Drehung des Handgelenks, die manchmal als \enquote{Wurfbewegung} bezeichnet wird.
Das Schöne am Aneignen der Übersatzbewegung für das Arpeggios ist, dass man, sobald man sie gelernt hat, einfach eine kleinere Version derselben Bewegung machen muss, um die leichteren Übersatzbewegung zu spielen.

Der Standard-Fingersatz für das aufsteigende Arpeggio CEGCEG...C mit der rechten Hand ist 123123...5, mit der linken Hand 5421421...1 und umgekehrt für die absteigenden Arpeggios.
In \enquote{Michael Aaron, Adult Piano Course, Book Two} finden Sie die Fingersätze aller Arpeggios und Tonleitern.

\textbf{Weil Arpeggios mehrere Noten übergehen, spreizen die meisten die Finger, um die Noten zu erreichen.
Bei schnellen Arpeggios ist das ein Fehler, weil das Spreizen der Finger ihre Bewegung verlangsamt.}
Der Schlüssel zu schnellen Arpeggios ist, die Hand zu bewegen, anstatt die Finger zu spreizen.
Wenn Sie die Hand und das Handgelenk entsprechend bewegen, werden Sie feststellen, dass es nicht notwendig ist, die Finger zu spreizen.
Diese Methode vereinfacht auch das \hyperref[c1ii14]{Entspannen}\index{Entspannen}.


\paragraph{Die Wagenradbewegung (Chopins FI)}
\label{c1iii5wagen}

Um die Wagenradbewegung zu verstehen, legen Sie Ihre linke Handfläche flach auf die Tasten, und spreizen Sie die Finger wie die Speichen eines Rades.
Beachten Sie, dass die Fingerspitzen vom kleinen Finger bis zum Daumen ungefähr auf einen Halbkreis fallen.
Halten Sie nun den kleinen Finger über die C3-Taste und parallel dazu; Sie müssen die Hand drehen, sodass der Daumen näher zu Ihnen kommt.
Bewegen Sie dann die Hand zur Klappe hin, sodass der kleine Finger die Klappe berührt; achten Sie darauf, dass die Hand stets fest gespreizt ist.
Wenn der vierte Finger zu lang ist und die Klappe zuerst berührt, drehen Sie die Hand weit genug, sodass der kleine Finger die Klappe berührt, aber halten Sie den kleinen Finger so parallel wie möglich zur C3-Taste.
\textbf{Drehen Sie nun die Hand wie ein Rad gegen den Uhrzeigersinn (von oben gesehen), sodass jeder nachfolgende Finger die Klappe (ohne Gleiten) berührt, bis Sie den Daumen erreichen.
Das ist die Wagenradbewegung in der horizontalen Ebene.
Wenn Ihre normale Reichweite mit ausgestreckten Fingern eine Oktave beträgt, werden Sie feststellen, dass die Wagenradbewegung fast zwei Oktaven abdeckt!}
Sie erhalten eine zusätzliche Reichweite, weil diese Bewegung die Tatsache ausnutzt, dass die mittleren drei Finger länger sind als der kleine Finger oder der Daumen und der Umfang eines Halbkreises viel größer ist als der Durchmesser.
Wiederholen Sie nun die Bewegung mit vertikaler Hand (die Handfläche parallel zur Klappe), sodass die Finger abwärts zeigen.
Beginnen Sie mit dem senkrecht stehenden kleinen Finger und senken Sie die Hand, um C3 zu spielen.
Wenn Sie nun die Hand zum C4 aufwärts rollen (machen Sie sich keine Sorgen, wenn es sich sehr unbeholfen anfühlt), wird jeder Finger die Note \enquote{spielen}, die er berührt.
Wenn Sie den Daumen erreichen, werden Sie wieder feststellen, dass Sie eine Entfernung überdecken, die fast das Doppelte Ihrer normalen Reichweite beträgt.
\textbf{In diesem Absatz haben wir drei Dinge gelernt:}

\begin{enumerate} 
 \item \textbf{Wie man mit der Hand \enquote{ein Wagenrad schlägt}.}
 \item \textbf{Diese Bewegung erweitert Ihre effektive Reichweite, ohne dass Sie Sprünge ausführen.}
 \item \textbf{Die Bewegung kann benutzt werden, um die Tasten zu \enquote{spielen}, ohne die Finger relativ zur Hand zu bewegen.}
\end{enumerate}
Beim tatsächlichen Üben wird das Wagenrad so benutzt, dass die Hand irgendwo zwischen der vertikalen und der horizontalen Position ist und die Finger in \hyperref[c1iii4b]{Pyramidenhaltung}\index{Pyramidenhaltung} oder leicht gebogen sind.
Obwohl diese Wagenradbewegung einen Beitrag zur Tastenbewegung leistet, werden Sie zum Spielen auch die Finger bewegen müssen.

Es ist kaum zu glauben, \textbf{aber Sie können die Reichweite sogar noch weiter ausdehnen, indem Sie die Finger \enquote{spreizen} (Fraser), was eine Form der Glissandobewegung ist}.
Stellen Sie sich vor, dass Sie eine übertriebene Glissandobewegung auf das aufsteigende Arpeggio CEGCEG... mit der rechten Hand anwenden;
Sie können nun den Abstand zwischen den Fingern gegenüber dem Wagenrad vergrößern.
Um das zu zeigen, bilden Sie ein V mit den Fingern 2 und 3, und legen Sie das V so an die Kante einer ebenen Fläche, dass nur das V auf der Fläche liegt.
Spreizen Sie das V so weit Sie es leicht und bequem können.
Drehen Sie dann Ihren Arm und die Hand im Uhrzeigersinn um 90 Grad, sodass die Finger nun die Fläche mit den Seiten berühren.
Das ist eine übertriebene Glissandobewegung.
Sie können nun die Finger noch weiter spreizen.
Das funktioniert mit jedem Fingerpaar.

Indem Sie eine Kombination aus \hyperref[c1iii5a]{Daumenübersatz}\index{Daumenübersatz}, \hyperref[c1iii4b]{flachen Fingerhaltungen}\index{flachen Fingerhaltungen}, Wagenradbewegung und \enquote{gespreizten} Fingern benutzen, können Sie deshalb leicht schnelle Arpeggios mit geringer Belastung für die Streckmuskeln greifen und spielen.
Beachten Sie, dass diese komplexe Kombination von Bewegungen durch ein lockeres Handgelenk ermöglicht wird.
Wenn diese Kombination von Bewegungen erst einmal leicht für Sie ist, verfügen Sie über genügend Kontrolle, sodass Sie die Gewissheit erlangen, nie eine Note zu verfehlen.
Üben Sie das CEG-Arpeggio mit diesen Bewegungen.

Wir wenden diese Methode auf die gebrochenen Akkorde in der linken Hand von Chopins FI an.
In Abschnitt III.2 haben wir die \hyperref[c1iii2]{Anwendung des Zirkulierens}\index{Anwendung des Zirkulierens} beim Üben der linken Hand besprochen.
Wir werden dem Zirkulieren nun die Wagenradbewegung usw. hinzufügen.
Zirkulieren Sie die ersten 6 (oder 12) Noten der linken Hand von Takt 5 (bei dem die rechte Hand zum ersten Mal einfällt).
Beginnen wir zunächst nur mit der Wagenradbewegung.
Wenn Sie die Hand fast waagrecht halten, dann muss praktisch der ganze Tastenweg durch die Fingerbewegung zurückgelegt werden.
Wenn Sie jedoch die Hand mehr und mehr zur Vertikalen anheben, trägt die Wagenradbewegung mehr zum Tastenweg bei, und Sie brauchen weniger Fingerbewegung zum Spielen.
\textbf{Die Wagenradbewegung ist besonders für diejenigen mit kleinen Händen nützlich, weil sie automatisch die Reichweite ausdehnt.
Ein Wagenrad zu schlagen vereinfacht es auch zu entspannen, weil es weniger notwendig ist, die Finger weit auseinander gespreizt zu halten.
Sie werden auch feststellen, dass Ihre Kontrolle gesteigert wird, weil die Bewegungen nun zum Teil von den großen Bewegungen der Hand gesteuert werden, was das Spielen weniger abhängig von der Bewegung der einzelnen Finger macht und zu einheitlicheren, gleichmäßigeren Ergebnissen führt.}
Benutzen Sie soviel flache Fingerhaltungen wie notwendig, und fügen Sie ein wenig Glissandobewegung hinzu.  


Die rechte Hand ist sogar eine noch größere Herausforderung.
Die meisten schnellen Läufe sollten mit dem \hyperref[c1iii1a1]{Basisanschlag}\index{Basisanschlag} (langsam) und den \hyperref[c1ii11]{parallelen Sets}\index{parallelen Sets} (schnell) geübt werden.
Der mit Takt 13 beginnende Teil sollte erst wie ein Tremolo geübt werden (\hyperref[c1iii3b]{\autoref{c1iii3b}}) und dann mit parallelen Sets.
Das heißt, üben Sie zunächst (langsam), indem Sie nur die Finger benutzen und ohne Handbewegungen.
Benutzen Sie dann hauptsächlich eine Drehung des Arms und der Hand, um Takt 15 zu spielen.
Übertreiben Sie diese Bewegungen während Sie langsam üben; steigern Sie dann schrittweise die Geschwindigkeit, indem Sie jede dieser Bewegungen vermindern, und kombinieren Sie anschließend die Bewegungen, um die Geschwindigkeit noch weiter zu steigern.
Fügen Sie dann die parallelen Sets hinzu, wobei Sie alle vier Noten während einer Abwärtsbewegung der Hand spielen.
Spielen Sie die weißen Tasten mit \hyperref[c1ii2]{gebogenen Fingern}\index{gebogenen Fingern} und die schwarzen Tasten mit \hyperref[c1iii4b]{flachen Fingerhaltungen}\index{flachen Fingerhaltungen}.
Benutzen Sie die Muskeln zum Spreizen der Handflächen (\hyperref[c1iii7e]{\autoref{c1iii7e}}) statt denen zum Spreizen der Finger, und üben Sie das schnelle \hyperref[c1ii14]{Entspannen}\index{Entspannen} nach dem Spielen jeder der Oktaven in Takt 15.

\label{c1iii5f}
\subsubsection{Schub und Zug, Beethovens Mondschein-Sonate, 3. Satz}
\label{c1iii5SchubZug}

Für diejenigen, die Beethovens Mondschein-Sonate das erste Mal lernen, ist das beidhändige Arpeggio-Ende des dritten Satzes (Takte 196-198; dieser Satz hat 200 Takte) der schwierigste Abschnitt.
Indem wir darstellen, wie man diese schwierige Passage übt, können wir zeigen, wie Arpeggios geübt werden sollten.
Lassen Sie uns die rechte Hand zuerst versuchen.
Um das Üben zu vereinfachen, überspringen wir die erste Note in Takt 196 und üben nur die vier folgenden aufsteigenden Noten (E, G\#, C\#, E), die wir zirkulieren.
\textbf{Machen Sie beim \hyperref[c1iii2]{Zirkulieren}\index{Zirkulieren} mit der Hand eine elliptische Bewegung im Uhrzeigersinn (von oben gesehen).}
Wir teilen diese Ellipse in zwei Teile auf: Der obere Teil ist die Hälfte zum Klavier hin, und der untere Teil ist die Hälfte zu Ihrem Körper hin.
Wenn Sie die obere Hälfte spielen, \enquote{schieben} Sie Ihre Hand zum Klavier hin, und wenn Sie die untere Hälfte spielen, \enquote{ziehen} Sie die Hand vom Klavier weg.
Spielen Sie die vier Noten zuerst während der oberen Hälfte, und führen Sie die Hand mit der unteren Hälfte in ihre ursprüngliche Position zurück.
Das ist die Schubbewegung für das Spielen dieser vier Noten.
Ihre Finger neigen dazu, auf das Klavier zu zu gleiten, während Sie die einzelnen Noten spielen.
Machen Sie nun mit der Hand eine Bewegung gegen den Uhrzeigersinn, und spielen Sie dieselben vier aufsteigenden Noten während der unteren Hälfte der Ellipse.
Jeder Finger neigt dazu, vom Klavier weg zu gleiten, während Sie jede Note spielen.
Diejenigen, die nicht beide Bewegungen geübt haben, finden wahrscheinlich die eine viel unhandlicher als die andere.
Fortgeschrittene Spieler sollten beide Bewegungen gleich bequem finden.

Die obige Anleitung war für das aufsteigende Arpeggio der rechten Hand.
Lassen Sie uns für das absteigende Arpeggio der rechten Hand die ersten vier Noten des nächsten Takts benutzen (die gleichen Noten wie im vorangegangenen Absatz, nur eine Oktave höher und in umgekehrter Reihenfolge).
Die Zugbewegung wird wieder für die untere Hälfte der Bewegung im Uhrzeigersinn benutzt und der Schub für die obere Hälfte der Bewegung gegen den Uhrzeigersinn.
Üben Sie sowohl für aufsteigende als auch für absteigende Arpeggios sowohl den Schub als auch den Zug, bis Sie damit zufrieden sind.
Sehen Sie nun, ob Sie die entsprechenden Übungen für die linke Hand selbst herausfinden können.
\textbf{Beachten Sie, dass diese Zyklen alle \hyperref[c1ii11]{parallele Sets}\index{parallele Sets} sind und deshalb extrem schnell gespielt werden können.}

Nachdem Sie nun gelernt haben, was die Schub- und Zugbewegungen sind, mögen Sie zu Recht fragen: \enquote{Warum brauche ich sie?}
Zunächst sollte darauf hingewiesen werden, dass \textbf{für die Schub- und Zugbewegungen völlig verschiedene Muskelgruppen benutzt werden.
Deshalb muss bei einer bestimmten Anwendung eine Bewegung besser sein als die andere.}
Wir werden unten lernen, dass eine Bewegung stärker als die andere ist.
Schüler, die mit diesen Bewegungen nicht vertraut sind, werden, ohne die geringste Ahnung, was sie getan haben, zufällig eine davon auswählen oder zwischen den beiden wechseln.
Das kann zu unerwarteten Spielfehlern, unnötigem Stress oder Geschwindigkeitsbarrieren führen.
Die Existenz des Schubs und Zugs ist der Situation mit \hyperref[c1iii5b]{Daumenübersatz}\index{Daumenübersatz} und Daumenuntersatz analog.
Erinnern Sie sich daran, dass Sie erst durch das Lernen des Untersatzes \textit{und} des Übersatzes alle Fähigkeiten des Daumens völlig ausnutzen.
Besonders bei hohen Geschwindigkeiten wird der Daumen auf eine Art benutzt, die ungefähr in der Mitte zwischen Untersatz und Übersatz liegt; das wichtige, das man in Erinnerung behalten muss, ist jedoch, dass die Daumenbewegung auf der Übersatzseite des genauen Mittelpunkts sein muss.
Wenn Sie nur ein wenig auf der Untersatzseite sind, dann treffen Sie auf eine Geschwindigkeitsbarriere.

Die Analogie von Schub und Zug zu Untersatz und Übersatz geht sogar noch weiter, weil Schub und Zug ebenfalls eine neutrale Bewegung haben, so wie es eine Reihe von Bewegungen gibt, die zwischen Untersatz und Übersatz liegen.
\textbf{Man bekommt die neutrale Bewegung durch das Reduzieren der kleineren Achse der Ellipse zu Null}; das heißt man verschiebt einfach die Hand nach rechts und links ohne jegliche \textit{offensichtliche} elliptische Bewegung.
Aber hier macht es wieder einen großen Unterschied, ob man sich der neutralen Position von der Schubseite oder der Zugseite nähert, weil die scheinbar ähnlichen neutralen Bewegungen in Wahrheit mit unterschiedlichen Muskelgruppen gespielt werden müssen.
Lassen Sie mich das an einem mathematischen Beispiel verdeutlichen.
Mathematiker werden entsetzt sein, wenn man ihnen sagt, dass 0 = 0 ist, was auf den ersten Blick auf triviale Weise  richtig erscheint.
Die Realität schreibt jedoch vor, dass wir sehr vorsichtig sein müssen.
Das kommt daher, dass wir die wahre Bedeutung von Null kennen müssen, das heißt wir brauchen eine mathematische Definition von Null.
Sie ist definiert als die Zahl 1/N, wobei N gegen unendlich geht.
Man bekommt \enquote{dieselbe} Zahl Null, egal ob N positiv oder negativ ist!
Unglücklicherweise bekommt man, wenn man durch Null dividiert, 1/0, ein unterschiedliches Ergebnis, je nachdem ob N positiv oder negativ ist: \enquote{1/0 = +unendlich} wenn N positiv ist und \enquote{1/0 = -unendlich} wenn N negativ ist!
Wenn Sie angenommen haben, dass die beiden Nullen dasselbe sind, könnte Ihr Fehler nach der Division so groß wie \enquote{2 * unendlich} sein, je nachdem welche Null Sie benutzt haben!
Auf ähnliche Weise sind \enquote{dieselben} neutralen Positionen, die beim Beginnen aus dem Daumenuntersatz oder Daumenübersatz heraus erreicht werden, grundlegend verschieden, und ähnlich ist es bei Schub und Zug.
Unter bestimmten Bedingungen ist entweder eine von der Schubseite oder eine von der Zugseite erreichte neutrale Position besser.
Der Unterschied im Gefühl ist beim Spielen nicht zu verkennen.
Deshalb muss man beide lernen.

Dieser Punkt ist von solch allgemeiner Wichtigkeit, insbesondere für die Geschwindigkeit, dass ich ein weiteres Beispiel anführe.
Das Leben eines Samurais hängt von der Geschwindigkeit seines Schwerts ab.
Um diese Geschwindigkeit zu maximieren, muss das Schwert stets in Bewegung sein.
Wenn der Samurai das Schwert einfach hebt, stoppt und es senkt, ist die Bewegung zu langsam, und sein Leben ist in Gefahr.
Das Schwert muss kontinuierlich auf einer kreisförmigen, elliptischen oder gekrümmten Bahn bewegt werden, auch wenn es so aussieht, als ob es nur angehoben und gesenkt wird.
Das ist eine der ersten Lektionen des Schwertkampfs.
Die Anwendung der im Grunde bogenförmigen Bewegungen zur Steigerung der Geschwindigkeit hat allgemeine Gültigkeit (Aufschlag beim Tennis, Schmettern beim Badminton usw.) und somit auch für das Klavierspielen.

Nun gut, wir haben also festgestellt, dass sowohl Schub als auch Zug notwendig sind, aber wie wissen wir, wann wir was benutzen müssen?
Im Fall des Unter- und Übersatzes waren die Regeln klar; bei langsamen Passagen kann man beide benutzen, und in bestimmten Legato-Situationen braucht man den Untersatz; bei allen anderen sollte man den Übersatz benutzen.
Bei Arpeggios lautet die Regel, dass man die starken Bewegungen als erste Wahl benutzt und die schwachen Bewegungen als zweite Wahl.
Jeder Einzelne hat eine andere starke Bewegung, sodass Sie zunächst experimentieren sollten, um zu sehen, welche für Sie die stärkste ist.
Die Zugbewegungen sollten stärker sein, da die Armmuskeln für den Zug stärker als die für den Schub sind.
Bei den Zugbewegungen werden auch die fleischigen Teile der Finger benutzt, während bei den Schubbewegungen eher die Fingerspitzen benutzt werden, wobei man sich leicht die Fingerspitzen oder das Nagelbett verletzen kann.

Man könnte die Frage stellen: \enquote{Warum nicht immer neutral spielen -- weder Schub noch Zug?}
Oder nur eine Bewegung lernen (nur Schub) und einfach sehr gut darin werden?
Hier werden wir wieder an die Tatsache erinnert, dass es zwei Möglichkeiten gibt, neutral zu spielen, je nachdem, ob man sich von der Schubseite oder der Zugseite nähert, und für eine bestimmte Anwendung ist die eine üblicherweise besser als die andere.
Beachten Sie bei der zweiten Frage, dass eine zweite Bewegung wegen der Ausdauer nützlich sein könnte, da eine andere Gruppe von Muskeln benutzt wird.
Nicht nur das, sondern um die starken Bewegungen gut zu spielen, muss man wissen, wie die schwachen Bewegungen gespielt werden.
Das heißt, Sie spielen am besten, wenn die Hand in dem Sinne ausgewogen ist, dass Sie beide Bewegungen spielen können.
Deshalb sollten Sie, egal ob Sie sich entscheiden, für eine bestimmte Passage Schub oder Zug zu benutzen, immer auch die andere Bewegung üben.
Nur so können Sie wissen, welche Bewegung für Sie die beste ist.
\textbf{Wenn Sie zum Beispiel den Schluss der Beethoven-Sonate üben, sollten Sie feststellen, dass Sie einen schnelleren technischen Fortschritt machen, wenn Sie jeden Zyklus sowohl mit Schub als auch mit Zug üben.}
Am Ende sollten die meisten Schüler sehr nah an der neutralen Bewegung spielen, obwohl sich ein paar dafür entscheiden werden, übertriebene Schub- und Zugbewegungen zu benutzen.

Es gibt viel mehr neues Material, das wir in diesem dritten Satz üben sollten, bevor wir beidhändig spielen, sodass Sie in diesem Stadium wahrscheinlich nichts HT üben müssen -- außer als Experiment, um zu sehen, was Sie tun können und was nicht.
Insbesondere ist, beidhändig mit den höchsten Geschwindigkeiten zu versuchen, kontraproduktiv und nicht zu empfehlen.
Einen kurzen Ausschnitt beidhändig zu zirkulieren kann jedoch sehr nützlich sein; aber dieser sollte nicht zu viel geübt werden, wenn man ihn noch nicht zufriedenstellend einhändig spielen kann.
Die Hauptschwierigkeiten in diesem Satz sind in den Arpeggios und Alberti-Begleitungen (\enquote{do-so-mi-so}-Typ)  konzentriert; haben Sie diese gemeistert, haben Sie 90\% des Satzes bezwungen.
Diejenigen ohne genügende technische Fertigkeiten sollten zufrieden sein, wenn sie \textit{vivace}-Geschwindigkeit (120) erreichen.
Wenn Sie den ganzen Satz zufriedenstellend mit dieser Geschwindigkeit spielen können, dann können Sie die zusätzliche Anstrengung für den Versuch in Richtung \textit{presto} (über 160) auf sich nehmen.
Es ist wahrscheinlich kein Zufall, dass beim 4/4-Takt \textit{presto} mit der schnellen Herzschlagrate einer sehr aufgeregten Person übereinstimmt.
Beachten Sie, dass in Takt 1 die Begleitung in der linken Hand tatsächlich wie ein schlagendes Herz klingt.

Wir werden nun unseren \enquote{Schlachtplan} für das Lernen dieses Satzes skizzieren.
Wir begannen mit dem schwierigsten Teil, dem beidhändigen Arpeggio am Ende.
Die meisten Schüler werden mit der linken Hand mehr Schwierigkeiten als mit der rechten haben; fangen Sie deshalb, sobald die rechte Hand ziemlich zufriedenstellend ist, damit an, das rechtshändige Arpeggio der ersten beiden Takte dieses Satzes zu üben, während Sie weiterhin den linkshändigen Teil des Schlusses üben.
Eine wichtige Regel für das Spielen schneller Arpeggios ist, die Finger soviel wie möglich über den Tasten zu halten und diese fast zu berühren.
Heben Sie die Finger nicht weit von den Tasten.
Erinnern Sie sich daran: Benutzen Sie die flachen Haltungen für die schwarzen Tasten und die gebogene Haltung für die weißen Tasten.
Deshalb wird in den ersten beiden Takten dieses dritten Satzes nur die Note D mit gebogenen Fingern gespielt.
Diese Angewohnheit, für jedes ansteigende Arpeggio nur bestimmte Finger zu beugen, entwickelt man am besten durch das Zirkulieren paralleler Sets.
Natürlich ist die Fähigkeit, mit jedem Finger schnell und unabhängig von den anderen Fingern von einer flachen zur gebogenen Haltung zu wechseln, eine wichtige Fertigkeit, die Sie lernen müssen.

Das Pedal wird in diesem Stück nur in zwei Situationen benutzt:

\begin{enumerate} 
 \item beim doppelten Staccato-Akkord am Ende des zweiten Takts und in allen weiteren ähnlichen Situationen.
 \item in den Takten 165 und 166, in denen das Pedal eine entscheidende Rolle spielt.
\end{enumerate}
Der nächste zu übende Abschnitt ist der tremoloartige Abschnitt der rechten Hand, der in Takt 9 beginnt.
Arbeiten Sie sorgfältig am Fingersatz für die linke Hand -- diejenigen mit kleineren Händen können eventuell den fünften Finger nicht über die gesamte Dauer der beiden Takte unten halten.
Wenn Sie Schwierigkeiten damit haben, den \hyperref[c1iii1b]{Rhythmus}\index{Rhythmus} dieses Abschnitts zu interpretieren, hören Sie sich verschiedene Aufnahmen an, um ein paar Anregungen zu erhalten.
Dann kommen die Alberti-Begleitung der linken Hand, die in Takt 21 beginnt, und ähnliche Teile der rechten Hand, die später auftreten.
Die Alberti-Begleitung kann, wie es ab \hyperref[c1ii8]{\autoref{c1ii8}} erklärt wird, mit \hyperref[c1iii7b]{parallelen Sets}\index{parallelen Sets} geübt werden.
Der nächste schwierige Abschnitt ist der Triller der rechten Hand in Takt 30.
Dieser erste Triller wird am besten mit dem Fingersatz 3,5 ausgeführt, und der zweite erfordert 4,5.
Falls Sie kleine Hände haben, sind diese Triller genauso schwierig wie die Arpeggios am Schluss und sollten von Anfang an geübt werden, wenn Sie beginnen diesen Satz zu lernen.
Das sind die grundlegenden technischen Erfordernisse dieses Stücks.
Die Kadenz von Takt 186 ist eine interessante Kombination einer \enquote{Tonleiter} und eines Arpeggios; wenn Sie Schwierigkeiten damit haben, sie zu interpretieren, hören Sie sich wieder verschiedene Aufnahmen an, um ein paar Anregungen zu erhalten.
Beachten Sie, dass die Takte 187 und 188 \textit{adagio} sind.

Beginnen Sie das beidhändige Üben, nachdem alle diese technischen Probleme einhändig gelöst sind.
\textbf{Es besteht keine Notwendigkeit, den Gebrauch des Pedals zu üben, bis Sie mit dem beidhändigen Üben anfangen.}
Beachten Sie, dass die Takte 163 und 164 ohne Pedal gespielt werden.
Dann gibt die Anwendung des Pedals bei den Takten 165 und 166 diesen beiden letzten Takten eine Bedeutung.
Wegen des schnellen Tempos besteht die Neigung, zu laut zu üben.
Das ist nicht nur musikalisch unkorrekt, sondern auch technisch schädlich.
\textbf{Zu laut zu üben kann zu Ermüdung und Geschwindigkeitsbarrieren führen; der Schlüssel zur Geschwindigkeit ist \hyperref[c1ii14]{Entspannung}\index{Entspannung}.}
Es sind die \textit{p}-Abschnitte, die den größten Teil der Spannung erzeugen.
So ist zum Beispiel das \textit{ff} in Takt 33 nur eine Vorbereitung für das nachfolgende \textit{p}, und es gibt tatsächlich im ganzen Satz sehr wenige \textit{ff}.
Der ganze Abschnitt von Takt 43 bis 48 wird \textit{p} gespielt und führt zu einem einzigen Takt, 50, der \textit{f} gespielt wird.

Schließlich sollten Sie, wenn Sie richtig geübt haben, bestimmte Geschwindigkeiten finden, bei denen es einfacher ist, schneller zu spielen als langsamer zu spielen.
Das ist am Anfang völlig natürlich und ist eines der besten Zeichen, dass Sie die Lektionen dieses Buchs gut gelernt haben.
Selbstverständlich sollten Sie, wenn Sie erst die Technik beherrschen, in der Lage sein, bei jeder Geschwindigkeit mit der gleichen Leichtigkeit zu spielen.
 

\label{c1iii5g}
\subsubsection{Der Daumen: Der vielseitigste Finger}
\label{Daumen}
\textbf{Der Daumen ist der vielseitigste Finger; er lässt uns Tonleitern, Arpeggios und breite Akkorde spielen (wenn Sie es nicht glauben, versuchen Sie, eine Tonleiter ohne den Daumen zu spielen!).}
Die meisten Schüler lernen nicht, wie man den Daumen richtig benutzt, bis sie Tonleitern üben.
Deshalb ist es wichtig, Tonleitern so früh wie möglich zu üben.
Die C-Dur-Tonleiter ständig zu wiederholen, ist, auch wenn man die H-Dur-Tonleiter einschließt, nicht die richtige Art Tonleitern zu üben.
Es ist wichtig, alle Dur- und Molltonleitern und -arpeggios zu üben; das Ziel ist, die richtigen Fingersätze aller Tonleitern sozusagen in den Fingern zu verinnerlichen. 

Spielen Sie mit der Spitze des Daumens, nicht mit dem ersten Gelenk.
Das macht den Daumen effektiv so lang wie möglich, was notwendig ist, weil er der kürzeste Finger ist.
Um eine gleichmäßige Tonleiter zu erzeugen, müssen alle Finger so ähnlich wie möglich sein.
Damit Sie mit der Daumenspitze spielen können, müssen Sie das Handgelenk vielleicht ein wenig anheben.
Die Daumenspitze zu benutzen, ist bei hohen Geschwindigkeiten, für eine bessere Kontrolle und wenn man Arpeggios und Akkorde spielt hilfreich.
Mit der Spitze zu spielen, erleichtert den \hyperref[c1iii5b]{Daumenübersatz}\index{Daumenübersatz} und die \enquote{Glissandobewegung}, bei der die Finger von der Bewegungsrichtung der Hand weg zeigen.
Übertreiben Sie die Glissandobewegung nicht, Sie brauchen nur ein wenig davon.

Es ist sehr wichtig, den Daumen zu befreien, indem man den Daumenübersatz und ein sehr flexibles Handgelenk übt.
Außer beim Daumenuntersatz ist der Daumen immer gerade, wird gespielt, indem man das am Handgelenk befindliche Glied dreht, und wird durch Bewegungen der Hand und des Handgelenks in Position gebracht.
Eine von Liszts signifikantesten technischen Verbesserungen geschah, als er lernte, den Daumen korrekt zu verwenden.

\subsubsection{Schnelle chromatische Tonleitern}
\label{c1iii5h}

\textbf{Die chromatische Tonleiter besteht aus Halbtonschritten.
Die wichtigste Überlegung gilt dem Fingersatz, weil es so viele Möglichkeiten dafür gibt.}
Die Standard-Fingersätze für eine aufsteigende Oktave sind -- beginnend mit C -- 1313123131345 für die rechte Hand und 1313132131321 für die linke Hand (der Fingersatz für die oberen Noten ist für eine Wendung) und jeweils das gleiche rückwärts für eine absteigende Oktave.
Es ist schwierig, diese Fingersätze schnell zu spielen, weil sie aus den kürzest möglichen parallelen Sets aufgebaut sind und deshalb eine große Anzahl Verbindungen enthalten; meistens begrenzen die Verbindungen die Geschwindigkeit. 
Ihr größter Vorteil ist ihre Einfachheit, die sie praktisch auf alle chromatischen Folgen anwendbar macht, egal mit welcher Note man beginnt, und man kann sie sich am leichtesten merken.
Eine Variation davon ist 1212123121234, was ein wenig mehr Geschwindigkeit und Legato ermöglicht und bei großen Händen bequemer ist.

In dem Bestreben, die chromatische Tonleiter zu beschleunigen, wurden verschiedene Folgen mit längeren parallelen Sets erdacht; alle \enquote{akzeptierten} Folgen vermeiden die Benutzung des Daumens bei einer schwarzen Taste.
Die am meisten verwendete ist -- beginnend mit E -- 123123412312 (Hauer, Czerny, Hanon).
Eine Schwierigkeit mit diesem Fingersatz ist, dass der Anfang der Folge in Abhängigkeit von der ersten Note angepasst werden sollte, um die Geschwindigkeit zu maximieren.
Auch unterscheiden sich die rechte und die linke Hand voneinander; diese Folge benutzt vier parallele Sets.
Man kann sie auf drei parallele Sets verkürzen, indem man -- beginnend mit C -- 123412312345 spielt.
Mit guter Daumenübersatztechnik mag diese Tonleiter spielbar sein, aber sogar mit Übersatz benutzen wir selten einen Übergang mit 51 oder 15, weil das schwierig ist.
Ohne Frage begrenzt die Einschränkung, den Daumen auf einer schwarzen Taste zu vermeiden, die Wahl des Fingersatzes und macht es komplizierter, weil der Fingersatz von der ersten Note abhängig wird.

\textbf{Wenn wir einen Daumen auf einer schwarzen Taste zulassen, ist eine gute Tonleiter -- beginnend mit C -:}

\begin{itemize} 
 \item \textbf{1234,1234,1234; 1234,1234,12345 für 2 aufsteigende Oktaven mit der rechten Hand}
 \item \textbf{5432,1432,1432; 1432,1432,14321 für 2 aufsteigende Oktaven mit der linken Hand}
 \end{itemize}
\textbf{mit dem Daumen bei beiden Händen auf G\# und drei identischen parallelen Sets je Oktave -- die einfachste und schnellste mögliche Konfiguration.}
Wieder jeweils das gleiche rückwärts für die absteigenden Oktaven.
Ich nenne das die \textbf{\enquote{vierfingrige chromatische Tonleiter}}; soweit ich weiß, wurde dieser Fingersatz wegen des Daumens auf einer schwarzen Taste gefolgt von einem Passieren des vierten Fingers in der Literatur nicht besprochen.
Zusätzlich zur Geschwindigkeit ist die Einfachheit der größte Vorteil; Sie benutzen denselben Fingersatz, egal wo Sie beginnen (benutzen Sie Finger 3, um mit der rechten Hand auf D zu beginnen), aufsteigend und absteigend, der Fingersatz ist für beide Hände der gleiche, nur umgekehrt, der Daumen und Finger 3 sind synchronisiert, und der Anfang und das Ende sind immer 1,5.  
Mit guter Daumenübersatztechnik ist diese Tonleiter unschlagbar; Sie müssen nur auf das 14 oder 41 achten, wenn die 1 auf dem G\# ist.
Versuchen Sie das beim letzten chromatischen Lauf im Grave von Beethovens Pathetique, und Sie sollten eine merkliche Abnahme der Fehler und schließlich eine deutliche Steigerung der Geschwindigeit feststellen.
Haben Sie das für diesen Lauf gelernt, wird es auch bei jedem anderen chromatischen Lauf funktionieren.
Um einen flüssigen Lauf zu lernen, üben Sie mit dem Schlag auf jeder Note, dann auf jeder zweiten Note, jeder dritten usw.

Obwohl die meisten Übungen nicht hilfreich sind, nimmt das Üben von Tonleitern, Arpeggios und der vierfingrigen chromatischen Tonleiter einen besonderen Platz beim Aneignen der Klaviertechnik ein.
Da man mit ihnen so viele grundlegende technische Fertigkeiten erlernen kann, müssen sie ein Teil des täglichen Lernprogramms eines Klavierspielers sein.


[Ab hier wird der Text noch überarbeitet.]



