% File: c1iii7

\subsection{Übungen}
\label{c1iii7}

\subsubsection{Einführung}
\label{c1iii7a}

Aufgrund einer überwältigenden Zahl von Nachteilen (s. \hyperref[c1iii7h]{Abschnitt 7h}) \textbf{sind die meisten Fingerübungen nicht nützlich}.
Ein Einwand ist, daß sie viel Zeit verschwenden.
Wenn man so übt, daß man schwierige Stücke spielen kann, warum sollte man dann die Zeit mit Fingerübungen verbringen, anstelle die schwierigen Stücke zu üben?
Ein weiterer Einwand ist, daß die meisten Übungen sich zu sehr wiederholen und keinen musikalischen Aufwand erfordern, so daß man sie mit abgeschaltetem musikalischen Teil des Gehirns spielen kann, was gemäß eines jeden sachkundigen Klavierlehrers die schlechteste Art ist, das Klavierspielen zu üben.
\textbf{Stupides Üben ist schädlich}.
Übungen sind dazu da, die Ausdauer zu steigern - die meisten von uns besitzen jedoch eine Menge physischer Ausdauer zum Spielen aber eine ungenügende Ausdauer des Gehirns, weshalb stupide wiederholende Übungen unsere gesamte musikalische Ausdauer verringern können.
Wenn die Schüler nicht sorgfältig angeleitet werden, üben sie diese Wiederholungen mechanisch und lassen das Klavierüben für jeden, der unglücklich genug ist zuhören zu müssen, als eine Strafe erscheinen.
Das ist ein Weg, \enquote{Stille-Kämmerlein-Pianisten} zu erzeugen, die nur üben können, wenn niemand zuhört, da sie nie geübt haben, Musik zu machen.
Einige vollendete Klavierspieler benutzen Übungen routinemäßig zum Aufwärmen, aber diese Angewohnheit resultiert aus ihrer früheren Ausbildung, und Konzertpianisten benötigen sie für ihre Übungssitzungen nicht.

Statt dieser schädlichen Übungen bespreche ich hier eine völlig andere Klasse von Übungen, die Ihnen dabei helfen, Ihre technischen Defizite zu diagnostizieren, die für das Beseitigen dieser Defizite erforderliche Technik zu erwerben und musikalisch zu spielen.
In \hyperref[c1iii7b]{Abschnitt 7b} bespreche ich die Übungen für das Erwerben der Technik, insbesondere der Geschwindigkeit.
\hyperref[c1iii7c]{Abschnitt 7c} zeigt, wann und wie man sie benutzt.
In den \hyperref[c1iii7d]{Abschnitten 7d} bis \hyperref[c1iii7g]{7g} bespreche ich andere nützliche Übungen.
Ich habe die meisten Einwände gegen Übungen vom Hanon-Typ in \hyperref[c1iii7h]{Abschnitt 7h} zusammengetragen.
In der Vergangenheit wurden diese Übungen vom Hanon-Typ wegen mehrerer falscher Vorstellungen weithin akzeptiert:

\begin{enumerate}[label={\roman*.}] 
 \item Man kann Technik dadurch erwerben, daß man eine begrenzte Zahl einfacher Übungen lernt.
 \item Musik und Technik können getrennt gelernt werden.
 \item Technik erfordert hauptsächlich eine Entwicklung der Muskeln ohne eine Entwicklung des Gehirns.
 \item Technik erfordert eine Stärke der Finger.
\end{enumerate}

Solche Übungen wurden bei vielen Lehrern populär, denn wenn sie funktionierten, konnten die Schüler von den Lehrern mit wenig Aufwand in der Technik unterwiesen werden!
Das ist nicht der Fehler dieser Lehrer, weil diese falschen Vorstellungen über Generationen hinweg weitergereicht wurden, auch von berühmten Lehrern, wie Czerny, Hanon und vielen anderen.
Die Wahrheit ist, daß Klavierpädagogik ein herausfordernder, zeitintensiver und wissensbasierter Beruf ist.

%: bisher 7i
\textbf{Wenn wir Technik einfach als die Fähigkeit zu spielen definieren, dann besteht sie mindestens aus drei Komponenten.}
Sie hat eine innere Technikkomponente, die einfach Ihre Fertigkeitsstufe ist.
Die Fertigkeit zu haben bedeutet jedoch nicht, daß man spielen kann.
Wenn Sie z.B. für einige Tage nicht gespielt haben und die Finger eiskalt sind, werden Sie wahrscheinlich nicht in der Lage sein, irgend etwas zufriedenstellend zu spielen.
Somit wäre die zweite Komponente das Maß, in dem die Finger \enquote{aufgewärmt} sind.
Es gibt noch eine dritte Komponente, die wir hier \enquote{Konditionierung} nennen wollen.
Wenn Sie z.B. eine Woche lang große Bäume gefällt haben oder nichts anderes getan haben als tagelang Pullover zu stricken, werden die Hände in keiner guten Verfassung zum Klavierspielen sein.
Die Hände haben sich körperlich an eine andere Aufgabe angepaßt.
Wenn Sie auf der anderen Seite monatelang jeden Tag mindestens drei Stunden Klavier üben, werden Ihre Hände Dinge tun, die sogar Sie erstaunen.
Das Konditionieren bezieht größtenteils den ganzen Körper und wahrscheinlich das Gehirn mit ein und sollte deshalb nicht \enquote{Handkonditionierung} genannt werden.

Übungen können etwas zu allen drei Komponenten der Technik (innere, Aufwärmen und Konditionierung) beitragen, und Schüler verwechseln häufig Übungen zum Aufwärmen oder andere ineffiziente Übungen mit dem Erwerben innerer Technik.
Diese Verwechslung tritt auf, weil praktisch jede Übung zum Aufwärmen und zur Konditionierung beitragen kann, der Schüler dies aber leicht als innere Verbesserung mißverstehen kann, wenn er sich der drei Komponenten nicht bewußt ist.
Dieses Mißverständnis kann von Nachteil sein, falls der Schüler zuviel Aufwand in Übungen steckt und deshalb nicht all die anderen, wichtigeren Arten der Entwicklung innerer Technik lernt.
Dieses Wissen über die Komponenten der Technik ist auch bei der \hyperref[c1iii14]{Vorbereitung auf Konzerte} wichtig, weil man in diesem Fall fragen muß: \enquote{Was ist die beste Art, die Hände aufzuwärmen und zu konditionieren?}

Die innere Fertigkeitsstufe und das Aufwärmen der Hände sind leicht zu verstehen, aber das Konditionieren ist sehr komplex.
Die wichtigsten Faktoren, die das Konditionieren kontrollieren, sind die Dauer und die Häufigkeit des Übens und der Zustand des aus Gehirn, Nerven und Muskeln bestehenden Systems.
\textbf{Um die Hände in ihrer besten Kondition für das Spielen zu halten, müssen die meisten Menschen jeden Tag üben.}
Lassen Sie das Üben ein paar Tage ausfallen, wird die Konditionierung merklich nachlassen.
\textbf{Obwohl an anderer Stelle bemerkt wurde, daß man mit mindestens drei Tagen Üben in der Woche einen deutlichen Fortschritt erzielen kann, wird das deshalb sicherlich nicht zur besten Konditionierung führen.}
Konditionierung ist ein weitaus größerer Effekt als einigen Menschen klar ist.
Fortgeschrittene Klavierspieler achten stets genau auf die Konditionierung, da sie ihre Fähigkeit zum musikalischen Spielen beeinflußt.
Sie ist wahrscheinlich mit physiologischen Veränderungen verbunden, wie z.B. einer Erweiterung der Blutgefäße und der Ansammlung bestimmter Stoffe an spezifischen Stellen des Nerven- und Muskelsystems.
Dieser Faktor der Konditionierung wird wichtiger, wenn Ihre Fertigkeitsstufe steigt und wenn Sie anfangen, sich routinemäßig mit den höheren musikalischen Konzepten zu befassen, wie z.B. der Farbe oder das Charakteristische verschiedener Komponisten herauszubringen.
Unnötig zu sagen, daß sie entscheidend wird, wenn man technisch anspruchsvolles Material spielt.
Deshalb muß sich jeder Klavierspieler der Konditionierung bewußt sein, um zu wissen, was zu einer bestimmten Zeit gespielt oder geübt werden kann.

Ein schwerer zu bestimmender Faktor, der die Konditionierung beeinflußt, ist der Zustand des Gehirns bzw. des Nervensystems.
\textbf{Sie können deshalb ohne offensichtlichen Grund \enquote{gute} Tage und \enquote{schlechte} Tage haben.}
Das ist wahrscheinlich den \enquote{Löchern} analog, in die Athleten fallen.
Tatsächlich kann man für ausgedehnte Perioden \enquote{schlechte Tage} haben.
Indem man sich dieses Phänomens bewußt ist und durch Experimentieren kann dieser Faktor in einem gewissen Ausmaß kontrolliert werden.
Das bloße Bewußtsein, daß solch ein Faktor existiert, kann einem Schüler helfen, besser mit diesen \enquote{schlechten} Tagen zurechtzukommen.
Professionelle Athleten, wie z.B. Golfer, diejenigen, die Meditation praktizieren, usw. wußten schon lange von der Wichtigkeit mentaler Konditionierung.
Die allgemeinen Ursachen solcher schlechten Tage zu kennen, wäre sogar noch hilfreicher.
Die häufigste Ursache ist \hyperref[fpd]{FPD (Schnellspiel-Abbau)}, der am Ende von Abschnitt II.25 besprochen wurde.
Eine weitere häufige Ursache ist ein Abweichen von den Grundlagen: Genauigkeit, Timing, Rhythmus, korrekte Ausführung der Ausdrucksbezeichnungen usw.
Zu schnell zu spielen oder mit zuviel Ausdruck kann der Konditionierung abträglich sein.
Mögliche Abhilfen sind, sich eine gute Aufnahme anzuhören, ein Metronom zu Hilfe zu nehmen oder sich die Notenblätter noch einmal anzusehen.
\textbf{Eine Komposition einmal langsam zu spielen bevor man aufhört, ist eine der effektivsten präventiven Maßnahmen gegen zukünftiges unerklärliches \enquote{schlechtes Spielen} dieser Komposition.}
Deshalb hängt die Konditionierung nicht nur davon ab, wie oft Sie üben, sondern auch davon, was und wie Sie üben.
%: bisher 7i Ende
Solides \hyperref[c1ii12]{mentales Spielen} kann Löcher vermeiden; zumindest können Sie mit ihm erkennen, daß Sie sich in einem Loch befinden, \textit{bevor} sie spielen.
Noch besser ist, daß Sie es dazu benutzen können, aus dem Loch herauszukommen.
Wir benutzen alle ein gewisses Maß an mentalem Spielen, ob wir es wissen oder nicht.
Wenn Sie das mentale Spielen nicht bewußt benutzen, dann kommen und gehen die Löcher, scheinbar ohne Grund, in Abhängigkeit vom Zustand Ihres mentalen Spielens.
Deshalb ist das mentale Spielen so wichtig, wenn Sie \hyperref[c1iii14]{vorspielen}.


\paragraph{Schnelle und langsame Muskeln}
\label{c1iii7aMuskeln}

Für die Technik ist es wichtig, den Unterschied zwischen Kontrolle und Geschwindigkeit einerseits sowie Fingerstärke andererseits zu verstehen.
\textbf{Alle Muskelstränge bestehen hauptsächlich aus schnellen und langsamen Muskeln.}
Die langsamen Muskeln dienen der Kraft und Ausdauer.
Die schnellen Muskeln sind für die Technik notwendig.
Je nachdem wie sie üben, wächst die eine Gruppe zu Lasten der anderen.
Offensichtlich möchte man, wenn man für die Technik übt, daß die schnellen Muskeln wachsen und die langsamen abnehmen.
\textbf{Deshalb sollte man isometrische oder Kraftübungen vermeiden.
Man möchte alle Bewegungen schnell ausführen, und die einzelnen Finger entspannen, sobald sie ihre Arbeit verrichtet haben.}
Deshalb kann jeder Klavierspieler einem Sumoringer auf der Tastatur \enquote{davonlaufen}, obwohl der Ringer mehr Muskeln hat.
Merken Sie sich dieses Konzept der schnellen Muskeln, da es dieser grundlegende schnelle Fingerschlag (auf- oder abwärts) ist, den Sie bei jeder hier besprochenen Übung trainieren müssen; sehen Sie dazu den \enquote{\hyperref[c1iii7i]{schnellen Anschlag}} in Abschnitt (i).

Natürlich brauchen wir eine gewisse Balance der schnellen und langsamen Muskeln, damit die Finger, Hände usw. richtig funktionieren, aber die Forschung auf diesem Gebiet ist für das Klavierspielen beklagenswert unzureichend.
Da diejenigen, die in der Vergangenheit Übungen entwickelten, kaum eine Vorstellung davon hatten oder über Forschungsergebnisse darüber verfügten, was Übungen bewirken müssen, waren die meisten dieser Übungen nur geringfügig hilfreich, und wie hilfreich sie waren hing mehr davon ab, wie man sie benutzte, als von ihrem ursprünglichen Aufbau.
Der Grundgedanke hinter den meisten Übungen war z.B., daß man Fingerstärke benötigt; wir wissen nun, daß das völlig falsch ist.
Ein weiteres Konzept war, daß man um so mehr Technik lernt, je schwieriger die Übung ist.
Das ist offensichtlich falsch; richtig ist nur, daß man, wenn man fortgeschritten ist, schwieriges Material spielen kann.
Einige der leichtesten Übungen können Ihnen die fortgeschrittensten Techniken lehren, und das ist die Art von Übungen, die am nützlichsten sind.



