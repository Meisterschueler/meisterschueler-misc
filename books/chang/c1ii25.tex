% File: c1ii25

\section{Mit beiden Händen zusammen (\hyperlink{HsHt}{HT}) üben und mental spielen}\hypertarget{c1ii25}{}

\subsection{Einführung}\hypertarget{c1ii25a}{}

\textbf{Wir können endlich damit anfangen, die Hände zusammenzuführen!
Dabei bekommen einige Schüler die meisten Probleme, besonders in den ersten paar Jahren des Klavierunterrichts.}
Obwohl die hier vorgestellten Methoden Ihnen sofort helfen sollten, sich die Technik schneller anzueignen, wird es ungefähr zwei Jahre dauern, bis Sie in der Lage sind, wirklich einen Vorteil aus allem zu ziehen, das die Methoden dieses Buchs zu bieten haben.

Beidhändig zu spielen ist fast wie zu versuchen, an zwei unterschiedliche Dinge gleichzeitig zu denken.
Es gibt keine bekannte, vorprogrammierte Koordination der zwei Hände, wie wir sie zwischen unseren beiden Augen (für das Abschätzen von Entfernungen), unseren Ohren (für die Bestimmung der Richtung eines ankommenden Geräuschs) oder unseren Armen und Beinen (zum Gehen) haben.
Deshalb wird es ein wenig Arbeit erfordern, das genaue Koordinieren der Finger der beiden Hände zu lernen.
Die vorangegangene Beschäftigung mit dem \hyperlink{c1ii7}{einhändigen Spielen} macht das Lernen dieser Koordination viel leichter, weil wir uns nur auf das Koordinieren konzentrieren müssen und nicht gleichzeitig auf das Koordinieren \textit{und} auf das Entwickeln von Finger- und Handtechnik.

Die gute Nachricht ist, dass es nur ein grundlegendes \enquote{Geheimnis} dafür gibt, das beidhändige Spielen schnell zu lernen.
Dieses \enquote{Geheimnis} ist ein angemessenes \hyperlink{c1ii7}{Arbeiten mit getrennten Händen}, Sie kennen es also bereits!
\textbf{Die ganze Technik muss einhändig erworben werden;
versuchen Sie nicht, Technik beidhändig zu erwerben, die Sie einhändig erwerben können.}
Mittlerweile sollten die Gründe offensichtlich sein.
Wenn Sie versuchen, Technik beidhändig zu erlangen, die Sie sich einhändig aneignen können, werden Sie Probleme bekommen wie:

\begin{enumerate}[label={\arabic*.}] 
 \item Stress entwickeln,
 \item Ungleichgewicht der Hände (die rechte Hand tendiert dazu stärker zu werden),
 \item \hyperlink{c1ii22}{schlechte Angewohnheiten} aneignen,
 \item Geschwindigkeitsbarrieren erzeugen, usw.
 \end{enumerate}

Beachten Sie, dass alle Geschwindigkeitsbarrieren \textit{erzeugt} werden; sie resultieren aus unkorrektem Spielen oder Stress.
Vorzeitiges beidhändiges Üben kann eine beliebige Zahl von Geschwindigkeitsbarrieren erzeugen.
Falsche Bewegungen sind ein weiteres Hauptproblem; einige Bewegungen scheinen keine Probleme zu bereiten, wenn man langsam beidhändig spielt, werden aber unmöglich, wenn man die Geschwindigkeit steigert.
Das beste Beispiel ist das Spielen mit \hyperlink{c1iii5a}{untersetztem Daumen}.

Als erstes benötigen Sie ein Kriterium für die Entscheidung, wann Sie ausreichend einhändig geübt haben.
Ein gutes Kriterium ist die einhändige Geschwindigkeit.
Typischerweise ist die maximale beidhändige Geschwindigkeit, mit der Sie spielen können, 50 bis 90 Prozent der langsameren einhändigen Geschwindigkeit, entweder die der rechten oder der linken Hand.
Angenommen, Sie können mit der rechten Hand mit der Geschwindigkeit 10 und mit der linken Hand mit der Geschwindigkeit 9 spielen.
Dann mag Ihre maximale beidhändige Geschwindigkeit 7 sein.
Die schnellste Möglichkeit, die beidhändige Geschwindigkeit auf 9 zu erhöhen, wäre, die Geschwindigkeit der rechten Hand auf 12 zu erhöhen und die der linken Hand auf 11.
Als eine allgemeine Regel gilt: Bringen Sie die einhändige Geschwindigkeit um einiges über die endgültige Geschwindigkeit.
Deshalb ist das Kriterium, nach dem wir gesucht haben: Wenn Sie einhändig mit 110 bis 150 Prozent der endgültigen Geschwindigkeit entspannt und kontrolliert spielen können, dann sind Sie bereit zum beidhändigen Üben.


\hypertarget{notenweise}{}

Falls Sie immer noch Probleme haben, benutzen Sie das \hyperlink{c1iii8}{Konturieren}.
Angenommen, Sie können zufriedenstellend einhändig spielen.
Vereinfachen Sie nun eine oder beide Hände, sodass Sie leicht beidhändig spielen können, und fügen Sie dann die gelöschten Noten schrittweise hinzu.
Dafür gibt es viele Möglichkeiten, und Sie können in Abhängigkeit Ihrer Kenntnisse der Musiktheorie wirklich mächtige Methoden entwickeln, weshalb das Konturieren in Abschnitt III.8 detaillierter besprochen wird.
Sie benötigen jedoch keine Theorie, um das Konturieren zu benutzen; ein Beispiel ist die Methode, Noten hinzuzufügen: Nehmen Sie einen kurzen Ausschnitt des schwierigen Abschnitts.
Spielen Sie den Ausschnitt wiederholt einhändig mit der schwierigeren Hand (siehe \enquote{\hyperlink{c1iii2}{Zyklisch spielen}} in Abschnitt III.2).
Fangen Sie nun an, die leichtere Hand Note für Note hinzuzufügen.
Fügen Sie zuerst nur eine Note hinzu, und üben Sie, bis Sie es zufriedenstellend spielen können.
Fügen Sie dann eine weitere hinzu usw., bis der Abschnitt vollständig ist.
Achten Sie darauf, dass Sie beim Hinzufügen der Noten denselben Fingersatz wie beim einhändigen Üben benutzen.
Sehr oft ist der Grund, warum Sie nicht beidhändig spielen können, obwohl Sie einhändig spielen können, dass irgendwo noch ein Fehler ist.
Dieser Fehler liegt häufig im \hyperlink{c1iii1b}{Rhythmus}.
Versuchen Sie deshalb beim Hinzufügen der Noten herauszufinden, ob bei einer Hand ein Fehler vorliegt;
sehen Sie dazu am besten in den Noten nach.

Es gibt jede Menge Unterschiede in der Art und Weise, wie das Gehirn Aufgaben mit einer Hand bewältigt und wie Aufgaben, die eine Koordination der beiden Hände erfordern, weshalb es sich auszahlt, beides getrennt voneinander zu lernen.
Einhändiges Üben neigt nicht dazu, Angewohnheiten zu formen, die nicht direkt vom Gehirn kontrolliert werden, weil das Gehirn jede Hand direkt kontrolliert.
Auf der anderen Seite können beidhändige Bewegungen nur durch Wiederholungen entwickelt werden; diese erzeugen eine reflexartige Angewohnheit und beziehen eventuell Nervenzellen außerhalb des Gehirns mit ein.
Ein Anzeichen dafür ist die Tatsache, dass es länger dauert, beidhändige Bewegungen zu lernen.
\textbf{Deshalb sind schlechte beidhändige Angewohnheiten die schlimmsten, weil es sehr lange dauert, sie zu eliminieren.
Um die Technik schnell zu erwerben, müssen Sie diese Kategorie schlechter Angewohnheiten vermeiden.}

Das \hyperlink{c1ii12mental}{mentale Spielen} ist sowohl für das beidhändige als auch das einhändige Spielen notwendig, aber das beidhändige mentale Spielen ist natürlich für Anfänger schwieriger.
Wenn Sie das mentale Spielen erst einmal beherrschen, ist es einhändig und beidhändig gleich einfach.
Da Sie das mentale Spielen bereits einhändig kennen (siehe Abschnitt 12), ist die verbleibende Hauptaufgabe, es beidhändig zu lernen.
Beim einhändigen Auswendiglernen für das mentale Spielen werden Sie in jeder Komposition auf Stellen gestoßen sein, die Sie am Klavier prüfen mussten - Sie können sie auf dem Klavier spielen aber nicht in Gedanken -, diese Stellen haben Sie sich noch nicht vollständig gemerkt.
Das sind die Stellen, an denen Sie während einer Aufführung Gedächtnisblockaden haben könnten.
Um zu testen, ob Ihr mentales Spielen solide ist, prüfen Sie, ob Sie drei Dinge in Gedanken können:

\begin{enumerate}[label={\arabic*.}] 
 \item Können Sie an einer beliebigen Stelle im Stück beginnen und beidhändig spielen?
 \item Können Sie bei einem beliebigen Abschnitt, den Sie mit der einen Hand spielen, den Abschnitt der anderen Hand hinzufügen?
 \item Können Sie mit beiden Händen gleichzeitig spielen?
 \end{enumerate}
Sie sollten feststellen, dass Sie, wenn Sie das in Gedanken können, es auch leicht auf dem Klavier können.

Zeigen wir nun das beidhändige Üben an praktischen Beispielen.
Ich habe drei Beispiele gewählt, um die beidhändigen Methoden zu veranschaulichen, angefangen mit dem einfachsten, dem \hyperlink{c1ii25b}{ersten Satz von Beethovens Mondschein-Sonate}, dann \hyperlink{c1ii25c}{Mozarts Rondo Alla Turca}, und zum Schluss die anspruchsvolle \hyperlink{c1ii25d}{Fantaisie-Impromptu (FI) von Chopin}.
Sie sollten sich das Stück aussuchen, das Ihrer Fertigkeitsstufe am nächsten kommt.
Sie könnten auch Bachs Inventionen versuchen, die in den Abschnitten \hyperlink{c1iii6l}{III.6l} und \hyperlink{c1iii19}{III.19} detailliert behandelt werden.
Ich überlasse es Ihnen, es mit dem oben besprochenen \enquote{Für Elise} selbst zu versuchen, da es ziemlich kurz und relativ geradlinig ist.
Für viele Klavierspieler ist \enquote{Für Elise} zu \enquote{bekannt} und oft schwierig zu spielen; spielen Sie es in diesem Fall in gedämpfter Weise, konzentrieren Sie sich auf die Genauigkeit statt auf die Emotionen (kein Rubato), und lassen Sie die Musik für sich selbst sprechen.
Mit dem richtigen Publikum kann das sehr wirkungsvoll sein.
Dieses \enquote{unbeteiligte} Spielen kann bei populären, bekannten Stücken nützlich sein.

Die hier ausgewählten drei Kompositionen stellen ihre besonderen Anforderungen.
Die Mondschein-Sonate erfordert Legato, \textit{pp} und die Musik von Beethoven.
Das Alla Turca muss nach Mozart klingen, ist ziemlich schnell und erfordert eine genaue, unabhängige Kontrolle der Hände sowie ein solides Oktavspiel.
Die FI erfordert die Fähigkeit, beidhändig 4 gegen 3 und 2 gegen 3 zu spielen, extrem schnelle Fingersätze für die rechte Hand, die Romantik von Chopin und genaues Pedalieren.
Alle drei können relativ leicht beidhändig mental gespielt werden, weil die linke Hand hauptsächlich eine Begleitung der rechten Hand ist; in Bachs Inventionen spielen beide Hände die Hauptrollen, und es ist schwieriger, beidhändig mental zu spielen.
Das zeigt, dass Bach wahrscheinlich das mentale Spielen gelehrt und absichtlich anspruchsvolle Stücke für seine Schüler komponiert hat.
Diese erhöhte Schwierigkeit erklärt auch, warum einige Schüler die Inventionen - ohne eine entsprechende Anleitung (wie dieses Buch) - extrem schwierig auswendig zu lernen und mit der richtigen Geschwindigkeit zu spielen finden.


\subsection{Beethovens Mondschein-Sonate, 1. Satz, Op. 27, No. 2}\hypertarget{c1ii25b}{}

Die bedeutendste Auseinandersetzung über diesen Satz dreht sich um den Gebrauch des Pedals.
Beethovens Anweisung \enquote{senza sordini} - übersetzt \enquote{ohne Dämpfer} - bedeutet, dass das Haltepedal vom Anfang bis zum Ende getreten sein sollte.
Die meisten Klavierspieler folgen dieser Anweisung \textit{nicht}, weil das Sustain moderner Konzertflügel so lang ist (viel länger als bei Beethovens Klavier), dass das Vermischen all dieser Noten ein Hintergrundgetöse erzeugt, das in der konventionellen Klavierpädagogik als roh angesehen wird.
Sicherlich wird kein Klavierlehrer einem Schüler das erlauben!
Beethoven war jedoch nicht nur Extremist, sondern liebte es auch, die Regeln zu brechen.
Die Mondschein-Sonate basiert auf dem Kontrast.
Der erste Satz ist langsam, legato, mit Pedal und leise.
Der dritte Satz ist das genaue Gegenteil; er ist einfach eine Variation des ersten Satzes, die sehr schnell und agitato gespielt wird - das wird durch die Beobachtung bestätigt, dass die oberste Doppeloktave in Takt 2 des dritten Satzes eine verkürzte Form des dreinotigen Hauptthemas des ersten Satzes ist, das unten besprochen wird (in \hyperlink{Arpeggios}{Abschnitt III.5e} finden Sie weitere Informationen zum dritten Satz).
Es besteht auch ein starker Kontrast zwischen den Dissonanzen und den klaren Harmonien, die diesem ersten Satz seine berühmte Qualität verleihen.
Die Hintergrunddissonanz wird durch das Pedal erzeugt, sowie durch die Nonen usw.
Die Dissonanzen haben den Zweck, die Harmonien wie einen funkelnden Diamanten auf einem dunklen Samtuntergrund hervorzuheben.
Als Extremist wählte Beethoven das harmonischste mögliche Thema: eine Note, die dreimal wiederholt wird (Takt 5)!
Deshalb ist meine Interpretation, dass das Pedal während des ganzen Satzes unten sein sollte, so wie Beethoven es angab.
Bei den meisten Klavieren sollte das keine Probleme verursachen;
bei Konzertflügeln wird das jedoch schwierig, weil der Hintergrundlärm während des Spielens lauter wird und man immer noch \textit{pp} (\enquote{sempre pianissimo}) spielen muss; in diesem Fall könnten Sie den Hintergrund ein wenig reduzieren, bringen Sie ihn aber nie völlig zum Verstummen, denn er ist ein Teil der Musik.
Auf diese Art werden Sie es nicht auf Aufnahmen hören, bei denen der Schwerpunkt üblicherweise auf den klaren Harmonien liegt und der Hintergrund eliminiert wird - die \enquote{Standardkonvention} für den \enquote{korrekten} Pedalgebrauch.
Beethoven mag sich hierbei jedoch dafür entschieden haben, diese Regel zu brechen.
Deshalb setzte er im ganzen Satz keine Pedalzeichen - weil man es nie anheben soll.
\textbf{Da wir uns entschieden haben, das Haltepedal die ganze Zeit getreten zu lassen, ist die erste in diesem Stück zu lernende Regel, dass man das Pedal nicht benutzt, bis man zufriedenstellend beidhändig spielen kann.}
Das wird Sie in die Lage versetzen, zu lernen, wie man legato spielt, was man nur ohne das Pedal üben kann.
Obwohl es sehr leise gespielt wird, besteht keine Notwendigkeit, in diesem Stück das Dämpferpedal zu benutzen; außerdem ist bei den meisten Übungsklavieren die Mechanik mit getretenem Dämpferpedal nicht genügend leichtgängig, um bei \textit{pp} die gewünschte Kontrolle zu ermöglichen.

Fangen Sie damit an, dass Sie, sagen wir die Takte 1-5, einhändig auswendig lernen und anschließend sofort mental spielen.
Achten Sie auf alle Ausdruckszeichen.
Das Stück ist im Halbetakt, aber die ersten beiden Takte sind wie eine Einleitung und haben jeweils nur eine Oktavnote in der linken Hand; der Rest wird strenger im Halbetakt gespielt.
Beethoven sagt uns sofort, in Takt 2, dass die Dissonanz eine wichtige Komponente dieses Satzes sein wird, indem er die H-Oktave in der linken Hand einfügt und das Publikum mit einer Dissonanz schockt.
Fahren Sie mit dem Auswendiglernen abschnittsweise bis zum Ende fort.

Die Oktaven der linken Hand müssen \textit{gehalten} werden.
Spielen Sie zum  Beispiel die C\#-Oktave der linken Hand von Takt 1 mit den Fingern 51; ersetzen Sie die 5, indem Sie sofort Finger 4 und dann Finger 3 auf das untere C\# gleiten lassen und dabei das untere C\# unten halten.
Sie werden die Oktave mit 31 halten, bevor Sie Takt 2 erreichen.
Halten Sie nun die 3, während Sie die H-Oktave von Takt 2 mit 51 spielen.
Auf diese Weise behalten sie beim \textit{Abwärtsgehen} völlig das Legato in der linken Hand bei.
Mit diesem Verfahren können Sie das Legato mit Finger 1 nicht ganz aufrechterhalten, halten Sie diesen aber so lange wie Sie können.
Beim Übergang von Takt 3 zu 4 muss die Oktave der linken Hand \textit{aufwärts gehen}.
Spielen Sie in diesem Fall das F\# von Takt 3 mit 51, halten Sie dann die 5 und spielen Sie die nächste G\#-Oktave mit 41.
Spielen Sie ähnlich von Takt 4 zu 5 die zweite G\#-Oktave von Takt 4 mit 51, und ersetzen Sie dann Finger 1 mit 2, während Sie ihn unten halten (Sie müssen eventuell Finger 5 anheben), sodass Sie den folgenden Akkord von Takt 5 mit den Fingern 521 spielen und das Legato aufrechterhalten können.
Die allgemeine Idee ist, so viele Noten wie möglich zu halten, besonders die untere Note der linken und die obere Note der rechten Hand.
Es gibt gewöhnlich verschiedene Möglichkeiten, dieses \enquote{Halten} auszuführen, Sie sollten deshalb mit Ihnen experimentieren, um zu sehen, welche in einer bestimmten Situation am besten passt.
Die Wahl des jeweiligen Halteverfahrens hängt hauptsächlich von der Größe Ihrer Hand ab.
Die Oktave der linken Hand von Takt 1 könnte zum Beispiel mit 41 oder 31 gespielt werden, sodass Sie keine Finger ersetzen müssen; das hat den Vorteil der Einfachheit, hat aber den Nachteil, dass man sich daran erinnern muss, wenn man das Stück beginnt.
Benutzen Sie das \enquote{Ersetzen der Finger} während des Stücks, um so viel Legato wie möglich durchzuhalten.
\textbf{Sie müssen sich für eine bestimmte Methode des Ersetzens entscheiden, wenn Sie das Stück zum ersten Mal auswendig lernen, und dann immer dieselbe benutzen.}

Warum soll man die Noten legato halten, wenn man schließlich doch alle Noten mit dem Pedal aushält?
Erstens hängt wie Sie die Taste herunterdrücken davon ab, wie Sie sie unten halten; deshalb können Sie durch das Halten ein konsistenteres und zuverlässigeres Legato spielen.
Zweitens lässt der Fänger, wenn Sie die Taste anheben und die Note mit dem Pedal halten, den Hammer frei, was diesem gestattet umherzuspringen, und diese \enquote{Lockerheit} der Mechanik ist hörbar - die Natur des Klangs ändert sich.
Als Herr des Klaviers möchten Sie immer, dass der Fänger den Hammer festhält, sodass Sie die völlige Kontrolle über die gesamte Klaviermechanik haben.
Dieser Grad der Kontrolle ist extrem wichtig, wenn man \textit{pp} spielt - man kann das \textit{pp} nicht kontrollieren, wenn der Hammer umherspringt.
Ein weiterer Grund für das Halten ist, dass es zu einer absoluten Genauigkeit führt, weil Ihre Hand nie die Tastatur verlässt und die gehaltene Note als Referenz zum Finden der nachfolgenden Noten dient.

Musik - wie macht man Musik?
Takt 1 ist nicht nur eine Folge von vier Triolen.
Sie müssen logisch \textit{verbunden} werden; achten Sie deshalb auf die Verbindung zwischen der obersten Note jeder Triole und der untersten Note der nächsten Triole.
Diese Verbindung ist besonders wichtig, wenn man von einem Takt zum nächsten übergeht, und die unterste Note hat oft eine melodische Bedeutung, wie in den Takten 4-5, 9-10 usw.
Die rechte Hand von Takt 5 beginnt mit der tiefsten Note, E, und die Musik steigt kontinuierlich bis zum G\# des dreinotigen Themas.
Dieses Thema sollte nicht \enquote{alleine} gespielt werden, sondern ist der Höhepunkt des arpeggioartigen Anstiegs der vorangegangenen Triole.
Wenn Sie in Takt 8 Schwierigkeiten haben, die None mit der rechten Hand zu greifen, spielen Sie die untere Note mit der linken Hand - ähnlich in Takt 16.
In diesen Fällen können Sie das Legato in der linken Hand nicht völlig beibehalten, aber das Legato in der rechten Hand ist wichtiger, und das Anheben der linken Hand kann weniger hörbar gemacht werden, wenn Sie später das Pedal benutzen.
Wenn Sie die None jedoch leicht greifen können, sollten Sie versuchen, sie nur mit der rechten Hand zu spielen, weil Ihnen das erlaubt, mit der linken Hand mehr Noten zu halten.
Obwohl die erste Note des dreinotigen Themas eine G\#-Oktave ist, sollte die obere Note von der unteren Note unterschieden und lauter als diese gespielt werden.
Die Takte 32-35 sind eine Folge aufsteigender Triolen mit ansteigender Spannung.
Die Takte 36-37 sollten verbunden werden, weil sie ein weiches Lösen dieser Spannung sind.

Der Anfang ist \textit{pp} bis Takt 25, dort folgt ein Crescendo, dann ein Decrescendo zum \textit{p} in Takt 28 und schließlich die Rückkehr zum \textit{pp} in Takt 42.
Bei den meisten \textit{cresc.} und \textit{decresc.} sollte der größte Teil des Anstiegs oder Abfalls dem Ende zu erfolgen, nicht dem Anfang zu.
In Takt 48 gibt es ein unerwartetes Crescendo und bei der ersten Note von Takt 49 einen abrupten Sprung zu \textit{p}.
Das ist das deutlichste Zeichen, dass Beethoven eine klare Harmonie wollte, die einen vom Pedal erzeugten dissonanten Hintergrund überlagert.

Der \enquote{Schluss} beginnt bei Takt 55.
Beachten Sie den Halbetakt; betonen Sie besonders den ersten und dritten Schlag von Takt 57.
Was als ein normaler Schluss erscheint, wird durch die (falschen) Akzentzeichen auf dem vierten Schlag von Takt 58 und dem dritten Schlag von Takt 59 angezeigt.
Der erste Akkord von Takt 60 ist ein falscher Schluss.
Die meisten Komponisten würden das Stück hier beendet haben; es ist derselbe Akkord wie der erste Akkord dieses Satzes - ein Merkmal eines Standardschlusses.
Beethoven benutzte jedoch oft den doppelten Schluss, was den richtigen Schluss \enquote{endgültiger} macht.
Er nimmt den Schlag sofort wieder auf und führt Sie zu dem wahren Ende, indem er eine mit der linken Hand und \textit{pp} gespielte nostalgische Reprise des dreinotigen Themas benutzt.
Die letzten beiden Akkorde sollten die leisesten Noten des ganzen Satzes sein; das ist schwierig, weil sie so viele Noten beinhalten.

Beim beidhändigen Spielen bereitet dieser Satz keine Probleme.
Das einzige neue Element ist das Halten der Noten für das Legato, das eine zusätzliche gleichzeitige Kontrolle der Hände erfordert.

Wenn Sie den ganzen Satz auswendig gelernt haben und ihn zufriedenstellend beidhändig spielen können, fügen Sie das Pedal hinzu.
Wenn Sie sich dafür entscheiden, das Pedal die ganze Zeit getreten zu halten, kann die Melodie der oberen Noten in den Takten 5-9 als ätherische Erscheinung gespielt werden, die eine durch die Akkordprogression erzeugte Hintergrunddissonanz überlagert.
Beethoven wählte diese Konstruktion wahrscheinlich, um den vollen Klang der neuen Klaviere seiner Zeit zu demonstrieren und ihre Fähigkeiten zu erkunden.
Diese Beobachtung stützt die These, dass der dissonante Hintergrund nicht völlig durch das wohlüberlegte Anheben des Pedals eliminiert werden sollte.


\subsection{Mozarts Rondo Alla Turca, aus Sonate K300 (KV331)}\hypertarget{c1ii25c}{}

Ich gehe davon aus, dass Sie \hyperlink{c1ii7}{einhändig} bereits die Hausaufgaben gemacht haben, und beginne mit dem beidhändigen Teil, insbesondere weil das einhändige Spielen bei den meisten Stücken von Mozart relativ einfach ist.
Ich werde hauptsächlich die technischen Schwierigkeiten und \enquote{wie man es wie Mozart klingen lässt} behandeln.
Bevor wir mit den Einzelheiten anfangen, besprechen wir die Form der ganzen Sonate, denn wenn Sie den letzten Abschnitt lernen, entscheiden Sie sich vielleicht dafür, das ganze Stück zu lernen - es gibt keine einzige Seite dieser Sonate, die nicht faszinierend ist.

Der Begriff Sonate wurde für so viele Arten von Musik verwandt, dass er keine bestimmte Definition hat; er hat sich im Laufe der Zeit weiterentwickelt und verändert.
In frühester Zeit bedeutete Sonate einfach Musikstück oder Lied.
\textbf{Vor und während Mozarts Zeit bedeutete \enquote{Sonate} eine Instrumentalmusik mit einem bis vier Teilen, bestehend aus Sonatenhauptsatz, Menuett, Trio, Rondo usw.
Eine Sonatine ist eine kleine Sonate.
Der Sonatenhauptsatz wurde zunächst als erster Teil einer Sonate, einer Symphonie, eines Konzerts usw. entwickelt; er bestand im Allgemeinen aus einer Einführung\footnote{eventuell wiederholt}, einer Entwicklung und einer Reprise\footnote{ABA- oder AABA-Format}.}
Der Sonatenhauptsatz ist historisch wichtig, weil diese Grundstruktur schrittweise in die meisten Kompositionen einfloss. 
Seltsamerweise ist kein Teil dieser Mozart-Sonate (No. 16, K300) in der Sonatenhauptsatzform (Hinson, Seite 552).
Sie beginnt mit einem Thema und sechs Variationen.
Variation V ist Adagio und sollte nicht zu schnell gespielt werden.
Danach kommt eine Unterbrechung, die dem mittleren oder langsamen Satz einer Beethoven-Sonate entspricht.
Diese Unterbrechung hat die Form eines Menuett-Trios, einer Tanzform.
Das Menuett war ursprünglich ein französischer Hoftanz mit drei Schlägen und war der Vorläufer des Walzers.
Das Walzerformat schließt auch die Mazurkas ein; diese waren ursprünglich polnische Tänze, weshalb Chopin so viele Mazurkas komponierte.
Sie unterscheiden sich von den (Wiener) Walzern, die den Akzent auf dem ersten Schlag haben, dadurch, dass ihr Akzent auf dem zweiten oder dritten Schlag liegen kann.
Der Walzer begann unabhängig davon in Deutschland als langsamerer Tanz mit drei betonten Schlägen; er entwickelte sich dann zu dem populären Tanz, den wir nun als \enquote{Wiener} bezeichnen.
Die Trios traten schrittweise in den Hintergrund, als die Quartette an Popularität gewannen.
Sowohl das Menuett als auch das Trio in unserer Sonate haben die Taktart 3/4.
Deshalb trägt jede erste Note den betonten Schlag; zu wissen, dass es im Tanzformat (Walzer) ist, vereinfacht es, das Menuett-Trio richtig zu spielen.
Das Trio sollte einen vom Menuett völlig verschiedenen Eindruck vermitteln (eine Konvention in Mozarts Zeit); diese Veränderung des Eindrucks verleiht dem Übergang ein erfrischendes Gefühl.
\enquote{Trio} bezieht sich im Allgemeinen auf ein Stück, das mit drei Instrumenten gespielt wird; deshalb gibt es in dem Trio drei Stimmen, die man einer Violine, einer Viola und einem Cello zuordnen kann.
Vergessen Sie nicht das \enquote{Menuetto D. C.} (Da Capo, das heißt zum Anfang zurück gehen) am Ende des Trios; Sie müssen deshalb Menuett-Trio-Menuett spielen.
Der letzte Abschnitt ist das Rondo.
Rondos haben den allgemeinen Aufbau ABACADA usw., was einen Ohrwurm, A, ausgiebig benutzt.


Unser Rondo hat den Aufbau (BB')A(CC')A(BB')A'-Coda, eine sehr symmetrische Struktur.
Die Taktart ist ein lebhafter Halbetakt; können Sie die Tonart des BB'-Teils herausfinden?
Der Rest dieses Rondos steht in A-Dur, der angegebenen Tonart dieser Sonate.
Die ganze Sonate wird manchmal als Variation eines einzigen Themas bezeichnet, was wahrscheinlich falsch ist, obwohl das Rondo 
der Variation III und das Trio der Variation IV gleicht.
Das Rondo beginnt mit der B-Struktur, die aus einer kurzen Einheit von nur fünf Noten besteht, die in den Takten 1-3 zweimal, mit einer Pause dazwischen, wiederholt werden; sie wird in Takt 4 mit der doppelten Geschwindigkeit wiederholt; Mozart benutzte am Ende von Takt 3 dieselbe Einheit geschickt als Verbindung zwischen diesen Wiederholungen.
Sie wird mit halber Geschwindigkeit in den folgenden Takten 7 und 8 noch einmal wiederholt, und die letzten zwei Takte bilden das Ende.
Takt 9 ist der gleiche wie Takt 8, außer dass die letzte Note abwärts statt aufwärts geht; diese abrupte Änderung des Wiederholungsmusters ist eine einfache Möglichkeit, ein Ende anzuzeigen.
Die Einheiten mit halber Geschwindigkeit werden durch das Hinzufügen von zwei Vorschlagsnoten am Anfang verschleiert, sodass wir, wenn der gesamte Abschnitt B mit der richtigen Geschwindigkeit gespielt wird, nur die Melodie hören, ohne die einzelnen Wiederholungseinheiten zu erkennen.
Die Effizienz seines Kompositionsprozesses ist erstaunlich - er wiederholte dieselbe Einheit siebenmal in neun Takten mit drei Geschwindigkeiten, um eine seiner berühmten Melodien zu komponieren.
Die ganze Sonate besteht sogar aus diesen wiederholten Abschnitten, die acht bis zehn Takte lang sind.
Es gibt mehrere Abschnitte, die 16 oder 32 Takte lang sind, aber diese sind nur Vervielfältigungen der achttaktigen Basisabschnitte.
Weitere Beispiele zur Analyse der Mikrostruktur von Mozart und Beethoven finden Sie in \hyperlink{c1iv4}{Abschnitt IV.4}.
Diese Art der Analyse kann für das \hyperlink{c1iii6}{Auswendiglernen} und das \hyperlink{c1ii12mental}{mentale Spielen} hilfreich sein - schließlich hat er die Sonate mental komponiert!

Die technisch anspruchsvollen Teile sind:

\begin{enumerate}[label={\arabic*.}] 
 \item der schnelle Triller der rechten Hand in Takt 25,
 \item die schnellen Läufe der rechten Hand in den Takten 36-60 - achten Sie auf einen guten \hyperlink{c1ii18}{Fingersatz},
 \item die schnellen gebrochenen Oktaven der rechten Hand in den Takten 97-104 und 
 \item die schnelle Alberti-Begleitung der linken Hand in den Takten 119-125.
 \end{enumerate}

Ermitteln Sie, welcher dieser Teile Ihnen am meisten Schwierigkeiten bereitet, und beginnen Sie das Üben mit diesem Teil.
Die Reihe der gebrochenen Oktaven der Takte 97-104 ist nicht einfach nur eine Folge gebrochener Oktaven, sondern es sind zwei Melodien, die eine Oktave und einen Halbtonschritt voneinander entfernt sind und sich gegenseitig jagen.
Üben Sie alles einhändig, ohne Pedal, bis es zufriedenstellend ist, bevor Sie mit dem beidhändigen Üben beginnen.
Die \hyperlink{c1iii7b}{Übungen für parallele Sets} sind der Schlüssel zur Entwicklung der zum Spielen dieser Elemente notwendigen Technik, und \hyperlink{c1iii7b1}{Übung \#1} (Wiederholung von Quadrupeln, III.7b) ist die wichtigste, besonders zum Lernen der \hyperlink{c1ii14}{Entspannung}.
Informationen zu \hyperlink{c1iii3}{schnellen Trillern} finden Sie in Abschnitt III.3a.
Die gebrochenen Akkorde in der linken Hand (Takt 28 usw. und in der Coda) sollten sehr schnell gespielt werden, fast wie eine einzelne Note, und sich mit den Noten der rechten Hand decken.
Das beidhändige Spielen sollte zunächst ohne Pedal geübt werden, bis es zufriedenstellend ist.

Wie spielt man, damit die Musik wie Mozart klingt?
Es gibt kein Geheimnis - die Anweisungen waren die ganze Zeit da!
Es sind die Ausdrucksbezeichnungen auf dem Notenblatt; für Mozart hatte jedes Zeichen eine präzise Bedeutung, und wenn Sie \textit{jedes} einzelne davon befolgen, einschließlich der Taktart usw., wird die Musik zu einer vertraulichen, komplizierten Konversation.
Das \enquote{einzige}, das Sie tun müssen, ist, den Drang zu unterdrücken, einen eigenen Ausdruck hinzuzufügen.
Es gibt kein besseres Beispiel dafür, als die letzen drei Akkorde am Ende.
Es ist so einfach, dass es fast unglaublich ist (ein Kennzeichen von Mozart): Der erste Akkord ist ein Staccato, und die restlichen zwei sind legato.
Dieses einfache Mittel erzeugt einen überzeugenden Schluss; wenn man es anders spielt, wird das Ende zum Flop.
Deshalb sollten diese drei letzten Akkorde ohne Pedal gespielt werden, obwohl einige Ausgaben (Schirmer) Pedalzeichen haben.
Bessere Klavierspieler spielen meistens das gesamte Rondo ohne Pedal.

Lassen Sie uns die ersten acht Takte dieses Rondos genauer untersuchen.<br>
\textbf{Rechte Hand}: Das erste viernotige Thema (Takt 1) wird legato gespielt; darauf folgt eine Achtelnote und eine exakte Achtelpause.
Die Note und die Pause werden benötigt, um dem Publikum die Einführung dieser Einheit \enquote{darzureichen}.
Dieses Konstrukt wird wiederholt, dann wird das viernotige Thema mit doppelter Geschwindigkeit (zwei je Takt) in Takt 4 wiederholt und gipfelt im fest gespielten und mit den zwei folgenden Staccato-Noten verbindenden C6.
Diese Verdopplung der Geschwindigkeit ist ein Mittel, das von Komponisten zu jeder Zeit benutzt wurde.
In den Takten 5-7 spielt die rechte Hand staccato und hält so die Spannung aufrecht.
Die Folge der fallenden Noten in den Takten 8-9 bringt diesen Abschnitt zum Abschluss, wie jemand, der auf die Bremse eines Autos tritt.<br>
\textbf{Linke Hand: Die einfache Begleitung mit der linken Hand bietet ein festes Gerüst; ohne dieses würden die ganzen neun Takte wie eine durchgeweichte Nudel umherwackeln.}
Die geschickte Anordnung der Bögen (zwischen der ersten und zweiten Note von Takt 2 usw.) betont nicht nur den Halbetakt, sondern stellt auch die rhythmische Idee der Exposition heraus; \textbf{es klingt wie ein Foxtrott-Tanzschritt - langsam, langsam, schnell-schnell-langsam in den Takten 2-5, wiederholt in den Takten 6-9}.
Da in den Takten 6-8 jede Note staccato sein muss, ist die einzige Möglichkeit, den \hyperlink{c1iii1b}{Rhythmus} zu betonen, den Akzent jeweils auf die erste Note jedes Takts zu legen.

Beide Noten von Takt 9 (beide Hände) sind legato und etwas leiser, um ein Ende darzustellen, und beide Hände heben sich im selben Moment.
Es ist klar, dass wir nicht nur wissen müssen, was die Ausdrucksbezeichnungen sind, sondern auch \textit{warum} sie dort sind.
Natürlich ist keine Zeit, über diese komplizierten Erklärungen nachzudenken; die Musik sollte sich darum kümmern - der Künstler \textit{fühlt} einfach die Wirkung dieser Zeichen.
Das strategische Setzen von Legato, Staccato, Bögen und Akzenten ist der Schlüssel zum Spielen dieses Stücks, während man den Rhythmus genau beibehält.
Ich hoffe, Sie sind nun in der Lage, die Analyse für den Rest dieses Stücks fortzuführen und Musik zu erzeugen, die eindeutig Mozart ist.

Das beidhändige Spielen ist etwas schwieriger als bei der \hyperlink{c1ii25b}{Mondschein-Sonate}, weil dieses Stück schneller ist und eine höhere Genauigkeit erfordert.
Der schwierigste Teil ist vielleicht das Koordinieren des \hyperlink{c1iii3}{Trillers} in der rechten Hand mit der linken Hand in Takt 25.
Versuchen Sie nicht, das zu lernen, indem Sie es langsamer spielen.
Stellen Sie nur sicher, dass die einhändige Arbeit komplett erledigt ist, indem Sie die Takte 25 und 26 als einen einzigen Übungsausschnitt benutzen und dann die beiden Hände bei der endgültigen Geschwindigkeit zusammenführen.
\textbf{Versuchen Sie beim beidhändigen Spielen immer erst, alles mit (oder nahe an) der endgültigen Geschwindigkeit zu kombinieren, 
und benutzen Sie die langsameren Geschwindigkeiten nur als letzten Ausweg, denn wenn Sie es schaffen, sparen Sie eine Menge Zeit und vermeiden es, sich \hyperlink{c1ii22}{schlechte Angewohnheiten} anzueignen.}
Fortgeschrittene Klavierspieler müssen beim Zusammenführen der Hände fast nie langsamer werden.

Fügen Sie das Pedal hinzu, nachdem Sie zufriedenstellend beidhändig ohne das Pedal spielen können.
Im mit Takt 27 beginnenden Abschnitt erzeugt die Kombination der gebrochenen Akkorde der linken Hand, der Oktaven der rechten Hand und des Pedals ein Gefühl der Erhabenheit, die charakteristisch dafür ist, wie Mozart mit relativ einfachen Mitteln Erhabenheit erzeugen konnte.
Halten Sie die letzte Note dieses Abschnitts etwas länger als aufgrund des Rhythmus erforderlich (Tenuto, Takt 35), besonders nach der Wiederholung, bevor Sie mit dem nächsten Abschnitt beginnen.
Wie bereits gesagt, hat Mozart keine Pedalzeichen angegeben; deshalb sollten Sie, nachdem Sie beidhändig ohne Pedal geübt haben, das Pedal \textit{nur} dort hinzufügen, wo Sie glauben, dass es die Musik verbessert.
Besonders bei schwierigem Material, wie das von Rachmaninoff, wird weniger Pedal von der Klavierspielergemeinde als Anzeichen überlegener Technik angesehen.


\hypertarget{FI}{}
\subsection{Chopins Fantaisie-Impromptu, Op. 66}\hypertarget{c1ii25d}{}

Dieses Beispiel wurde ausgewählt, weil:

\begin{enumerate}[label={\arabic*.}] 
 \item jeder diese Komposition mag,
 \item sie ohne gute Lernmethoden als unmöglich zu lernen erscheinen kann,
 \item das Hochgefühl, wenn man plötzlich in der Lage ist, sie zu spielen, unvergleichlich ist,
 \item die Herausforderungen des Stücks ideal zur Veranschaulichung sind und
 \item das die Art von Stück ist, an dem Sie Ihr ganzes Leben arbeiten werden, um \enquote{unglaubliche Dinge} damit zu machen, sodass Sie genauso gut \textit{jetzt} damit anfangen können!
 \end{enumerate}
Die meisten Schüler, die Schwierigkeiten damit haben, haben sie, weil Sie den Einstieg nicht finden, und die anfängliche Hürde erzeugt eine mentale Blockade, die sie ihre Fähigkeit das Stück zu spielen anzweifeln lässt.
Es gibt keine bessere Demonstration der Wirksamkeit der Methoden dieses Buchs, als zu zeigen, wie man diese Komposition lernen kann.
Da dieses Stück jedoch ziemlich schwierig ist, sollten Sie \hyperlink{c1iii1}{Abschnitt III} lesen, bevor Sie es lernen.

Sie werden ungefähr 2 Jahre Klavierunterricht benötigen, bevor Sie mit diesem Stück beginnen können.
Etwas einfachere Stücke sind die oben besprochenen - \hyperlink{c1ii25b}{Mondschein-Sonate} und das \hyperlink{c1ii25c}{Rondo} - sowie \hyperlink{c1iii6l2}{Bachs Inventionen} in Abschnitt III.6l.
Finden Sie stets die Tonart heraus, bevor Sie anfangen.
Tipp: Nach der \enquote{Ankündigung} G\# beginnt die Komposition in Takt 3 mit einem C\# und endet mit einem C\#, und das Largo beginnt mit einem Db (\footnote{enharmonisch verwechselt} dieselbe Note wie C\#!); aber sind die beiden in einer Dur- oder Molltonart?
\textbf{Wegen der großen Zahl Kreuze und Be's, wie in dieser Komposition, sind Anfänger häufig besorgt;
die schwarzen Tasten sind jedoch einfacher zu spielen als die weißen, wenn Sie erst einmal die \hyperlink{c1iii4b}{flachen Fingerhaltungen} (siehe III.4b) und den \hyperlink{c1iii5b}{Daumenübersatz} (siehe III.5) kennen.
Chopin mag diese \enquote{weit entfernten} Tonarten aus diesem Grund gewählt haben, weil die Tonleiter bei der \hyperlink{et1}{Gleichschwebenden Temperatur}, die er wahrscheinlich benutzte, nichts ausmacht (siehe \hyperlink{c2_2c}{Kapitel 2, Abschnitt 2c}).}

Wir fangen an, indem wir die vorbereitenden Hausaufgaben mit dem \hyperlink{c1ii7}{einhändigen Üben} und dem \hyperlink{c1ii12mental}{mentalen Spielen} noch einmal durchgehen.
Deshalb sollten Sie mit dem Ziel beidhändig üben, die beiden Hände sehr genau zu synchronisieren.
Obwohl die letzte Seite die schwierigste sein mag, werden wir die Regel über das Beginnen mit dem Ende durchbrechen und am Anfang beginnen, weil es schwierig ist, dieses Stück richtig anzufangen.
Wenn man aber erst einmal angefangen hat, geht es wie von selbst.
Sie brauchen einen starken, zuversichtlichen Anfang.
Wir werden deshalb mit den ersten beiden Seiten anfangen, bis zum langsamen Cantabile-Teil.
Das Strecken der linken Hand und das ständige Training machen die \hyperlink{c1ii21}{Ausdauer} (das heißt die \hyperlink{c1ii14}{Entspannung}) zu einem Hauptthema.
Diejenigen ohne genügende Erfahrung und besonders diejenigen mit kleineren Händen brauchen vielleicht für Wochen zusätzliche Arbeit an der linken Hand, bevor sie zufriedenstellend ist.
Glücklicherweise ist die linke Hand nicht so schnell, sodass die Geschwindigkeit kein einschränkender Faktor ist und die meisten Schüler in der Lage sein sollten, die linke Hand in weniger als zwei Wochen einhändig schneller als mit der endgültigen Geschwindigkeit, völlig entspannt und ohne Ermüdung zu spielen.

Für Takt 5, in dem die rechte Hand zum ersten Mal einsetzt, ist der vorgeschlagene Fingersatz für die linke Hand 532124542123.
Sie können damit anfangen, dass Sie Takt 5 mit der linken Hand üben und ihn fortlaufend \hyperlink{c1iii2}{zyklisch spielen}, bis Sie ihn gut spielen können.
Sie sollten während des Spielens die \textit{Handfläche} und nicht die \textit{Finger} strecken, sonst kann es zu Stress und Verletzungen kommen.
Sehen Sie dazu in \hyperlink{c1iii7e}{Abschnitt III.7e}, wie man seine Handflächen dehnt.

\textbf{Üben Sie ohne das Pedal.}
Üben Sie in kleinen Abschnitten.
Die vorgeschlagenen Abschnitte sind: Takte 1-4, 5-6, erste Hälfte von 7, zweite Hälfte von 7, 8, 10 (überspringen Sie 9, der der gleiche ist wie 5), 11, 12, 13-14, 15-16, 19-20, 21-22, 30-32, 33-34, dann zwei Akkorde in 35.
Wenn Sie den zweiten Akkord nicht greifen können, spielen Sie ihn als einen schnellen aufsteigenden, gebrochenen Akkord mit der Betonung auf der obersten Note.
Nachdem jeder Abschnitt auswendig gelernt und zufriedenstellend ist, verbinden Sie sie paarweise.
Spielen Sie dann die ganze linke Hand aus dem Gedächtnis, wobei sie mit dem Anfang beginnen und die Abschnitte hinzufügen.
Bringen Sie sie bis zur endgültigen Geschwindigkeit, und prüfen Sie Ihr mentales Spielen.

Wenn Sie diesen ganzen Abschnitt (nur mit der linken Hand) zweimal hintereinander entspannt und ohne sich müde zu fühlen spielen können, haben Sie die notwendige Ausdauer.
An diesem Punkt macht es viel Spaß, schneller zu werden als die endgültige Geschwindigkeit.
Gehen Sie zur Vorbereitung der beidhändigen Arbeit ungefähr bis zur 1,5-fachen endgültigen Geschwindigkeit.
Heben Sie das Handgelenk leicht, wenn Sie mit dem kleinen Finger spielen, und senken Sie es, wenn Sie zum Daumen kommen.
Beim Heben des Handgelenks werden Sie feststellen, dass Sie mehr Kraft in den kleinen Finger legen können, und durch das Senken des Handgelenks vermeiden Sie das Verpassen der Daumennote.
\textbf{In Chopins Musik sind die Noten des kleinen Fingers und des Daumens (aber besonders die des kleinen Fingers) die wichtigsten.}
Üben Sie deshalb, diese beiden Finger mit Autorität zu spielen.
Die in Abschnitt III.5 beschriebene \hyperlink{c1iii5wagen}{Wagenradbewegung} kann hierbei hilfreich sein.

Wenn Sie damit zufrieden sind, fügen Sie das Pedal hinzu; grundsätzlich sollte das Pedal mit jedem Akkordwechsel angehoben werden, der im Allgemeinen ein- oder zweimal je Takt vorkommt.
Das Pedal ist eine schnelle Auf-und-Ab-Bewegung (die den Ton \enquote{abschneidet}) auf dem ersten Schlag, aber Sie können das Pedal für besondere Effekte früher heben.
Für die weite Streckung der linken Hand in der zweiten Hälfte von Takt 14 (beginnend mit E2) ist der Fingersatz 532124, wenn Sie ihn bequem ausführen können.
Wenn nicht, benutzen Sie 521214.

Gleichzeitig sollten Sie die rechte Hand geübt haben, wobei Sie die Hände wechseln, sobald die arbeitende Hand sich ein wenig müde anfühlt.
\textbf{Die Vorgehensweisen sind fast identisch mit denen für die linke Hand, einschließlich des Übens ohne das Pedal.}
Teilen Sie zunächst Takt 5 in zwei Hälften, lernen Sie jede Hälfte bis zur endgültigen Geschwindigkeit, und fügen Sie sie dann zusammen.
Benutzen Sie für das ansteigende Arpeggio in Takt 7 den \hyperlink{c1iii5a}{Daumenübersatz}, weil es zu schnell ist, um es mit Untersatz zu spielen.
Der Fingersatz sollte so sein, dass beide Hände den kleinen Finger oder den Daumen nach Möglichkeit gleichzeitig spielen; das vereinfacht das beidhändige Spielen.
Deshalb ist es keine gute Idee, mit den Fingersätzen der linken Hand herumzualbern - benutzen Sie die Fingersätze, die in den Noten stehen.

Üben Sie nun beidhändig.
Sie können entweder mit der ersten oder der zweiten Hälfte von Takt 5 anfangen, bei dem die rechte Hand das erste Mal dazukommt.
Die zweite Hälfte ist wahrscheinlich einfacher - wegen der geringeren Streckung der linken Hand und weil es (im Gegensatz zur ersten Hälfte) kein Timing-Problem mit der ersten fehlenden Note in der rechten Hand gibt.
Lassen Sie uns deshalb mit der zweiten Hälfte anfangen.
\textbf{Der einfachste Weg das 3,4-Timing zu lernen, ist, es von Anfang an mit der endgültigen Geschwindigkeit zu tun.
Versuchen Sie nicht, langsamer zu werden und herauszufinden, wo jede Note hingehört, weil zu viel davon eine Ungleichmäßigkeit in Ihr Spielen einführen wird, die später eventuell nicht mehr korrigiert werden kann.}
Hier benutzen wir das \enquote{Zirkulieren} - sehen Sie dazu \enquote{\hyperlink{c1iii2}{Zyklisch spielen}} in Abschnitt III.2.
Zirkulieren Sie zunächst fortlaufend ohne zu stoppen die sechs Noten der linken Hand.
Wechseln Sie dann die Hände und machen Sie das gleiche mit den acht Noten der rechten Hand mit dem gleichen (endgültigen) Tempo, wie Sie es mit der linken Hand getan haben.
Zirkulieren Sie als nächstes mehrere Male nur mit der linken Hand, und lassen Sie dann die rechte Hand einstimmen.
Am Anfang müssen Sie nur die ersten Noten genau zur Deckung bringen; machen Sie sich nichts daraus, wenn die anderen nicht so ganz stimmen.
Nach ein paar Versuchen sollten Sie in der Lage sein, es ziemlich gut beidhändig zu spielen.
Wenn nicht, hören Sie auf, und fangen Sie noch einmal von vorne mit dem einhändigen Zirkulieren an.
Da fast die ganze Komposition aus Stücken wie dem Abschnitt, den Sie gerade geübt haben, besteht, zahlt es sich aus, diesen gut zu üben, bis Sie sehr zufrieden sind.
Ändern Sie die Geschwindigkeit, um dieses zu erreichen.
Werden Sie sehr schnell, dann sehr langsam.
Während Sie langsamer werden können Sie sehen, wo all die Noten in Bezug zueinander hingehören.
Sie werden feststellen, dass schnell nicht notwendigerweise schwierig ist und langsamer nicht immer einfacher.
\textbf{Das 3,4-Timing ist ein mathematisches Mittel, das Chopin benutzt hat, um die Illusion von Hyper-Geschwindigkeit in diesem Stück zu erzeugen.}
Die mathematischen Erklärungen und zusätzlichen herausragenden Punkte dieser Komposition werden ausführlicher unter \enquote{\hyperlink{c1iii2}{Zyklisch spielen}} in Abschnitt III.2 besprochen.
Sie werden wahrscheinlich diese Komposition jahrelang mit getrennten Händen üben, nachdem Sie zum ersten Mal mit dem Stück fertig sind, weil es so viel Spaß macht, mit dieser faszinierenden Komposition zu experimentieren.
Fügen Sie nun das Pedal hinzu.
Das ist der Punkt, an dem Sie die Angewohnheit entwickeln sollten, das Pedal exakt zu \enquote{pumpen}.

Wenn Sie mit der zweiten Hälfte von Takt 5 zufrieden sind, wiederholen Sie die gleiche Prozedur für die erste Hälfte.
Fügen Sie dann die beiden Hälften zusammen.
Ein Nachteil des Ansatzes, erst einhändig und dann beidhändig zu lernen ist, dass praktisch der ganze Technikerwerb einhändig erreicht wird, was möglicherweise zu schwach synchronisiertem beidhändigen Spielen führt.
Sie haben nun die meisten Werkzeuge, um den Rest dieser Komposition selbst zu lernen!

Der Cantabile-Abschnitt ist einfach viermal die gleiche Sache mit zunehmender Komplexität.
Lernen und merken Sie sich deshalb zunächst den ersten Teil, weil er der einfachste ist.
Lernen Sie dann den vierten Teil, weil er der schwerste ist.
Normalerweise sollte man den schwierigsten Teil zuerst lernen, aber in diesem Fall mag es für einige Schüler zu lange dauern, mit dem vierten Teil zu beginnen, und die einfachste Variante zuerst zu lernen kann es sehr vereinfachen, den vierten Teil zu lernen, weil sie einander ähnlich sind.
Wie  bei vielen Stücken von Chopin, ist, die linke Hand auswendig zu lernen, der schnellste Weg, um ein festes Fundament für das Auswendiglernen zu bilden, weil die linke Hand üblicherweise eine einfachere Struktur hat, die leichter zu analysieren, auswendig zu lernen und zu spielen ist.
Zudem erzeugte Chopin oft mehrere Versionen für die rechte Hand, während er im Grunde die gleichen Noten in der linken Hand wiederholte, wie er es in diesem Fall tat (dieselben Akkordprogressionen);
deshalb kennen Sie, nachdem Sie den ersten Teil gelernt haben, bereits das meiste für die linke Hand im vierten Teil, sodass Sie diese letzte Wiederholung schnell lernen können.

Der Triller im ersten Takt des letzten Teils, kombiniert mit dem 2,3-Timing, macht die zweite Hälfte dieses Takts schwierig.
Da es vier Teile gibt, könnten Sie den zweiten halben Takt im ersten Teil ohne Triller, im zweiten Teil den Triller als invertierten Mordent, einen kurzen Triller im dritten und im letzten Teil einen längeren Triller spielen.

Der dritte Abschnitt (Presto!) ist dem ersten Abschnitt ähnlich.
Wenn Sie es also geschafft haben, den ersten Abschnitt zu lernen, sind Sie fast schon fein raus.
Dieses Mal ist er jedoch schneller als beim ersten Mal (Allegro) - 
Chopin möchte offensichtlich, dass Sie das mit unterschiedlichen Geschwindigkeiten spielen.
Möglicherweise, weil er sah, dass die Abschnitte sehr unterschiedlich klingen können, wenn man die Geschwindigkeit ändert;
warum sollte es unterschiedlich klingen, und auf welche Weise?
Die Physik und die Psychologie dieser Geschwindigkeitsänderung werden in \hyperlink{c1iii2}{Abschnitt III.2} besprochen.
Beachten Sie, dass ungefähr in den letzten 20 Takten der kleine Finger und der Daumen der rechten Hand die ganze Strecke über bis zum Ende Noten von bedeutendem thematischen Wert tragen.
Dieser Abschnitt kann eine Menge einhändiges Üben mit der rechten Hand erfordern.


\hypertarget{fpd}{}

\textbf{Wenn Sie irgendeine Komposition zu oft mit voller Geschwindigkeit (oder schneller) spielen, kann es sein, dass Sie erleiden, was ich \enquote{Schnellspiel-Abbau} (FPD = Fast Play Degradation) nenne.}
Am nächsten Tag werden Sie vielleicht feststellen, dass Sie sie nicht mehr so gut spielen oder beim Üben keinen Fortschritt machen können.
Das geschieht meistens beim beidhändigen Spielen.
Das Spielen mit getrennten Händen ist eher gegen FPD immun und kann sogar benutzt werden, um ihn zu korrigieren.
FPD tritt wahrscheinlich auf, weil der menschliche Spielmechanismus (Hände, Gehirn usw.) bei solchen Geschwindigkeiten durcheinander gerät, und tritt deshalb nur bei so komplexen Handlungen wie dem beidhändigen Spielen von vorstellungsmäßig oder technisch schwierigen Stücken auf.
Leichte Stücke leiden nicht unter FPD.
FPD kann enorme Probleme bei komplexer Musik, wie Bachs oder Mozarts Kompositionen, erzeugen.
Schüler, die versuchen, sie beidhändig auf Geschwindigkeit zu bringen, können auf alle Arten von Problemen stoßen, und die Standardlösung war, einfach immer langsam zu üben.
Es gibt jedoch eine tolle Lösung für dieses Problem: Üben Sie mit getrennten Händen!
Und denken Sie daran, dass Sie, wann immer Sie schnell spielen, im Allgemeinen unter FPD leiden werden, wenn Sie nicht mindestens einmal vor dem Aufhören langsam spielen.
FPD kann auch ein Zeichen dafür sein, dass Ihr mentales Spielen nicht solide ist oder noch nicht die richtige Geschwindigkeit erreicht hat.


\section{Zusammenfassung}\hypertarget{c1ii26}{}

Damit kommen wir zum Schluss des grundlegenden Abschnitts.
Sie haben alles Notwendige für das Ausarbeiten von Abläufen, mit denen Sie praktisch jedes neue Stück lernen können.
Dieses ist der minimale Satz von Anweisungen, die Sie zum Anfangen benötigen.
In Abschnitt III werden wir noch mehr Anwendungen dieser grundlegenden Schritte erforschen, ebenso werden wir noch mehr Ideen dafür vorstellen, wie man einige weitverbreitete Probleme löst.




