% File: preface

\section*{Vorwort}
\label{preface}

% zuletzt geändert 09.08.2009

\textbf{Dieses ist das beste Buch, das jemals darüber geschrieben wurde, wie man am Klavier übt!}
Das Buch offenbart, dass es hocheffiziente Übungsmethoden gibt, die Ihre Lernrate beschleunigen können - bis zu einem Faktor von 1000, wenn Sie die effizientesten Übungsmethoden bisher nicht gelernt haben (siehe \hyperref[c1iv5]{Berechnen der Lernrate}).
Das Überraschende ist, dass diese Methoden, obwohl sie seit der frühesten Zeit des Klaviers bekannt sind, selten gelehrt wurden, weil nur wenige Lehrer sie kannten und diese sachkundigen Lehrer sich nie die Mühe gemacht haben, ihr Wissen zu verbreiten.

\textbf{Ich erkannte in den 1960ern, dass es kein gutes Buch darüber gab, wie man am Klavier übt.}
Das Beste, das ich finden konnte, war Whitesides Buch, was aber eine völlige Enttäuschung war (siehe die Besprechung des Buchs im \hyperref[Whiteside]{Quellenverzeichnis}).
Als Student der Cornell University, der bis 2:00 Uhr morgens lernte, damit er mit einigen der klügsten Studenten aus aller Welt mithalten konnte, hatte ich wenig Zeit, Klavier zu üben.
Ich musste wissen, was die besten Übungsmethoden sind, besonders weil alles, was ich benutzte, nicht funktionierte, obwohl ich während meiner Jugend 7 Jahre lang eifrig Klavierstunden genommen hatte.
Wie Konzertpianisten so spielen können, wie sie es tun, war ein absolutes Mysterium für mich.
War es einfach eine Frage der genügenden Anstrengung, der Zeit und des Talents, wie viele Menschen anscheinend meinen?
Wenn die Antwort \enquote{Ja} wäre, dann wäre es niederschmetternd für mich gewesen, weil es bedeutet hätte, dass meine musikalische Talentstufe so niedrig war, dass ich ein hoffnungsloser Fall war, weil ich, zumindest während meiner Jugend, genügend Anstrengung und Zeit hineingesteckt hatte, indem ich an Wochenenden bis zu 8 Stunden täglich geübt hatte.

Die Antworten kamen mir schrittweise in den 1970ern, als ich bemerkte, dass die Klavierlehrerin unserer beiden Töchter einige erstaunlich effiziente Übungsmethoden lehrte, die sich von den von der Mehrheit der Klavierlehrer gelehrten Methoden unterschieden.
\textbf{Ich habe diese effizienten Übungsmethoden über einen Zeitraum von mehr als 10 Jahren verfolgt und kam zu der Erkenntnis, dass der wichtigste Faktor für das Lernen des Klavierspielens die \textit{Übungsmethoden} sind.}
Anstrengung, Zeit und Talent waren bloße zweitrangige Faktoren!
In Wahrheit ist \enquote{Talent} schwierig zu definieren und unmöglich zu messen; es ist ein nebulöses Wort, das wir häufig benutzen, das aber keine definierbare Bedeutung hat.
Tatsächlich \textbf{können die richtigen Übungsmethoden praktisch jeden in einen \enquote{talentierten} Musiker verwandeln!}
Ich habe das jedes Mal bei hunderten von Schülerkonzerten und Klavierwettbewerben gesehen, die ich besucht habe.

\textbf{Es wird nun zunehmend erkannt, dass \enquote{Talent}, \enquote{Wunderkinder} und \enquote{Genialität} eher erzeugt werden als angeboren sind} (siehe Olson) - Mozart ist vielleicht das prominenteste Beispiel des \enquote{Mozart-Effekts}.
Einige haben diesen in \enquote{Beethoven-Effekt} umbenannt, was angebrachter sein mag, weil Mozart ein paar Schwächen in der Persönlichkeit usw. hatte, die manchmal seine ansonsten herrliche Musik beeinträchtigten, während psychologisch gesehen, Beethoven die am meisten erleuchtende Musik komponierte.
Sich Musik anzuhören ist nur eine Komponente des komplexen Mozart-Effekts.
Für Klavierspieler hat \textit{Musik zu machen} eine größere Auswirkung auf die geistige Entwicklung.
Deshalb werden gute Übungsmethoden nicht nur die Lernrate beschleunigen, sondern auch dabei helfen, das musikalische Gehirn zu entwickeln sowie das Ausmaß der Intelligenz zu erhöhen, besonders bei jungen Menschen.
Die Lernrate wird, verglichen mit der bei den langsameren Methoden, beschleunigt (es ist wie der Unterschied zwischen einem Fahrzeug, das beschleunigt, und einem, das mit konstanter Geschwindigkeit fährt).
Darum werden Schüler ohne die richtigen Übungsmethoden innerhalb weniger Jahre hoffnungslos zurückbleiben.
Das lässt die Schüler mit guten Übungsmethoden weitaus talentierter erscheinen als sie wirklich sind, weil sie in Minuten oder Tagen das lernen können, wofür die anderen Monate oder Jahre benötigen.
\textbf{Der wichtigste Aspekt des Klavierspielenlernens ist die Entwicklung des Gehirns und eine größere Intelligenz.
Das Gedächtnis ist eine Komponente der Intelligenz, und wir wissen, wie man das Gedächtnis verbessern kann (siehe \hyperref[c1iii6]{\autoref{c1iii6}}).
Dieses Buch lehrt uns auch, wie man die Musik in Gedanken spielt - das wird \enquote{\hyperref[c1ii12mental]{mentales Spielen}} genannt, das natürlich zum \hyperref[c1iii12]{absoluten Gehör} und zur Fähigkeit führt, Musik zu \hyperref[c1iii12blatt]{komponieren}.}
Das sind die Fertigkeiten, durch die sich die größten Musiker von anderen unterschieden, und weswegen wir sie als Genies bezeichneten; wir zeigen hier aber, dass diese Fertigkeiten nicht schwierig zu erlernen sind.
Bis jetzt war die Welt der Musiker für die wenigen \enquote{begabten} Künstler reserviert; wir wissen nun, dass sie ein Universum ist, in dem wir uns alle bewegen können.

\textbf{Die Übungsmethoden können bei jungen Schülern, die mit Leib und Seele dabei sind, innerhalb von weniger als 10 Jahren den Unterschied zwischen einer lebenslangen Zwecklosigkeit und einem Konzertpianisten ausmachen.}
Wenn man die richtigen Übungsmethoden benutzt, dauert es für einen fleißigen Schüler eines beliebigen Alters nur ein paar Jahre, bis er bedeutende Werke berühmter Komponisten spielen kann.
Die traurige Wahrheit der letzten beiden Jahrhunderte ist, dass, obwohl die meisten dieser Übungsmethoden entdeckt und tausende Male wieder entdeckt wurden, diese niemals dokumentiert wurden und die Schüler entweder gezwungen waren, sie selbst wieder zu entdecken oder, wenn sie Glück hatten, von Lehrern zu lernen, die einige dieser Methoden kannten.
Das beste Beispiel für diesen Mangel an Dokumentation sind die \enquote{Lehren} von Franz Liszt.
Es gibt ein Dutzend Franz-Liszt-Gesellschaften, und sie haben hunderte von Publikationen herausgegeben.
\textbf{Zahlreiche Bücher wurden über Liszt geschrieben (siehe \hyperref[Eigeldinger]{Eigeldinger} usw. im Quellenverzeichnis), und tausende von Lehrern haben - unter Angabe der Abstammungslinie - behauptet, die \enquote{Franz-Liszt-Methode} zu lehren.
Und doch gibt es keine einzige Publikation, die beschreibt, was diese Methode ist!}
Es gibt endlose Berichte von Liszts Fähigkeiten und technischem Können, jedoch gibt es keine einzige Quelle über die Einzelheiten, wie er dazu kam.
Zeugnisse in der Literatur zeigen, dass Liszt nicht beschreiben konnte, wie er die Technik erworben hatte; er konnte nur demonstrieren, wie er spielte.
Da es der Klavierpädagogik nicht gelungen ist, zu verfolgen, wie der größte Pianist seine Technik erlangte, wundert es wenig, dass wir kein Lehrbuch für das Klavierspielenlernen hatten.
Können Sie sich vorstellen, Mathematik, Wirtschaftswissenschaften, Physik, Geschichte, Biologie oder irgendetwas anderes ohne ein Lehrbuch zu lernen, und das (wenn Sie Glück haben) nur mit dem Gedächtnis Ihres Lehrers als Führung?
Ohne Lehrbücher und Dokumentation wäre unsere Zivilisation nicht über die der Dschungelstämme hinausgekommen, deren Wissensbasis durch wörtliche Überlieferung weitergegeben wurde.
Das ist im Grunde die Stufe, auf der sich die Klavierpädagogik während 200 Jahren befunden hat!

Es gibt viele Bücher über das Klavierspielenlernen (siehe \hyperref[reference]{Quellenverzeichnis}), jedoch kann keines davon als Lehrbuch für Übungsmethoden gelten, was man als Schüler aber benötigt.
Diese Bücher sagen Ihnen, welche Fertigkeiten Sie benötigen (\hyperref[c1iii5]{Tonleitern, Arpeggios}, \hyperref[c1iii3]{Triller} usw.), und die weiter fortgeschrittenen Bücher beschreiben die Fingersätze, Handpositionen, Bewegungen usw., mit denen man sie spielt, aber keines davon liefert einen hinreichend vollständigen, systematischen Satz Anweisungen darüber, wie man übt.
Die meisten Musikbücher für Anfänger bieten ein paar solcher Anweisungen, aber einige dieser Anweisungen sind falsch - ein gutes Beispiel ist die amateurhafte Anpreisung in der Einführung zur Hanon-Serie, wie man \enquote{mit 60 Übungen ein Virtuose wird} (siehe \hyperref[c1iii7h]{Abschnitt III.7h in Kapitel 1)}.
\textbf{Bevor dieses Buch hier geschrieben wurde, fehlte in der Klavierpädagogik das wichtigste Werkzeug für den Schüler: ein grundlegender Satz an Anweisungen dafür, wie man übt.}

Ich erkannte nicht, wie revolutionär die Methoden dieses Buch waren, bevor ich nicht 1994 die erste Ausgabe meines Buchs beendet hatte.
Die Methoden waren besser als jene, die ich zuvor benutzt hatte, und jahrelang hatte ich sie mit guten, aber nicht bemerkenswerten Resultaten angewandt.
Das Erwachen kam erst nachdem ich dieses Buch beendet hatte, als ich mein eigenes Buch wirklich las, die Methoden systematisch befolgte und ihre unglaubliche Effizienz erlebte.
Was war nun der Unterschied zwischen dem Kennen der Bestandteile der Methode und dem Lesen des Buchs?
Beim Schreiben des Buchs musste ich die verschiedenen Teile nehmen und sie in eine organisierte Struktur bringen, die einem bestimmten Zweck diente und bei der keine wesentlichen Komponenten fehlten.
Ich wusste, dass das Material in einer logischen Struktur anzuordnen der einzige Weg war, ein nützliches Handbuch zu schreiben.
Es ist in der Wissenschaft wohlbekannt, dass die meisten Entdeckungen während des Schreibens der Forschungsberichte gemacht werden und nicht während der Durchführung der Forschung.
Es war so, als wenn ich die meisten Teile eines sagenhaften Autos gehabt hätte, aber ohne einen Mechaniker, der das Auto zusammenbaut, die fehlenden Teile findet und das Auto richtig einstellt, eigneten sich diese Teile nicht gut für den Transport.
Ich wurde von dem Potential dieses Buchs überzeugt, den Klavierunterricht zu revolutionieren, und entschied mich 1999, es der Welt kostenlos im Internet zur Verfügung zu stellen;
so kann es aktualisiert werden, wenn meine Untersuchungen voranschreiten, und das Geschriebene ist sofort der Öffentlichkeit zugänglich.
Zurückblickend ist dieses Buch die Zusammenfassung von mehr als 50 Jahren Forschung, die ich seit meinen ersten Klavierstunden über das Klavierüben durchgeführt habe.

Warum sind diese Übungsmethoden so revolutionär?
Für detaillierte Antworten werden Sie das Buch lesen müssen.
Ich werde hier eine kurze Übersicht geben, wie diese wunderbaren Resultate erreicht werden, und erklären, warum die Methoden funktionieren.
\textbf{Die meisten grundlegenden Ideen in diesem Buch stammen nicht von mir.}
Sie wurden während der letzten 200 Jahre unzählige Male von allen erfolgreichen Pianisten erfunden oder wieder erfunden; diese hätten sonst nicht einen solchen Erfolg gehabt.
\textbf{Das Grundgerüst für dieses Buch wurde unter Verwendung der Lehren von Yvonne Combe erstellt}, der Lehrerin unserer beiden Töchter, die vollendete Klavierspielerinnen wurden (sie haben viele erste Preise bei Klavierwettbewerben gewonnen, und jede hat viele Jahre lang im Durchschnitt an mehr als 10 Konzerten teilgenommen; beide haben ein \hyperref[c1iii12]{absolutes Gehör} und \hyperref[c1iii12blatt]{komponieren} gerne).
Andere Teile dieses Buchs wurden aus der Literatur und aus den Ergebnissen meiner Nachforschungen im Internet zusammengestellt.
\textbf{Mein Beitrag ist das Zusammentragen dieser Ideen, sie in eine Struktur zu bringen und etwas zum Verständnis beizutragen, warum sie funktionieren.
Dieses Verständnis ist für den Erfolg der Methode entscheidend.}
Klavierspielen wurde oft wie Religion gelehrt: Glaube, Hoffnung, Liebe (Wohltätigkeit).
Glaube daran, dass wenn man von einem \enquote{Meister}-Lehrer vorgeschlagene Verfahren befolgt, diese auch funktionieren.
Hoffnung, dass \enquote{üben, üben, üben} ins Paradies führt.
Wohltätigkeit, sodass die gebrachten Opfer und die geleisteten Beiträge Wunder wirken.
Dieses Buch ist anders - \textbf{eine Methode ist nicht akzeptabel, solange die Schüler nicht verstehen, warum sie funktioniert, und sie deshalb nicht an ihre besonderen Bedürfnisse anpassen können}.
Das richtige Verständnis zu finden ist nicht einfach, weil man nicht bloß eine Erklärung aus der Luft greifen kann (sie wird falsch sein) - man braucht genügend Fachkenntnis auf diesem Gebiet, um zu der richtigen Erklärung zu kommen.
Indem man eine korrekte Erklärung bietet, filtert man automatisch die falschen Methoden heraus.
Das mag erklären, warum sogar erfahrene Klavierlehrer, deren Ausbildung stark auf die Musik ausgerichtet war, Schwierigkeiten damit haben können, das richtige Verständnis zu vermitteln, und oftmals sogar die falschen Erklärungen für die richtigen Methoden liefern werden.
In dieser Hinsicht waren mein Beruf und meine Ausbildung für die Lösung technischer Probleme, in Materialwissenschaften (Metalle, Halbleiter, Isolatoren), Optik, Akustik, Physik, Elektronik, Chemie, meine wissenschaftlichen Veröffentlichungen (ich habe über 100 geprüfte Artikel in den großen Wissenschaftsmagazinen veröffentlicht, und es wurden mir sechs Patente erteilt) usw. für das Schreiben dieses Buchs von unschätzbarem Wert.
Diese verschiedenen Erfordernisse könnten erklären, warum sonst niemand diese Art von Buch geschrieben hat.
Als Wissenschaftler habe ich mir den Kopf darüber zerbrochen, wie man \enquote{Wissenschaft} präzise definiert, und endlos mit Wissenschaftlern und Nichtwissenschaftlern über diese Definition debattiert.
Da der wissenschaftliche Ansatz für dieses Buch so grundlegend ist, habe ich einen Abschnitt (Kapitel 1, IV.2) darüber hinzugefügt.
Die Wissenschaft ist nicht bloß die theoretische Welt der intelligentesten Genies; sie ist der effizienteste Weg, unser Leben zu vereinfachen.
Wir brauchen Genies, um die Wissenschaft voranzubringen; wenn sie jedoch entwickelt wurden, sind es die Massen, die von den Fortschritten profitieren.

Was sind einige dieser zauberhaften Ideen, von denen erwartet wird, dass sie den Klavierunterricht revolutionieren?
Lassen Sie uns mit der Tatsache anfangen, dass wenn man berühmten Pianisten beim Auftritt zusieht, sie zwar unglaublich schwierige Stücke spielen, diese aber so aussehen lassen, als wenn sie einfach wären.
Wie machen sie das?
Tatsache ist, dass sie für sie leicht \textit{sind}!
Deshalb sind viele der hier besprochenen Lerntricks Methoden dafür, Schweres leicht zu machen - nicht nur leicht, sondern oft trivial einfach.
Das wird dadurch erreicht, dass man \hyperref[c1ii7]{mit beiden Händen getrennt übt} und \hyperref[c1ii6]{kleine Abschnitte zum Üben herausgreift}, manchmal bis zu einer oder zwei Noten herab.
Man kann die Dinge nicht einfacher machen als das!
Vollendete Pianisten können auch unglaublich schnell spielen - wie üben \textit{wir}, um in der Lage zu sein, schnell zu spielen?
Einfach!
Indem wir den \enquote{\hyperref[c1ii9]{Akkord-Anschlag}} (Abschnitt II.9) benutzen.
\textbf{Ein Schlüssel zum Erfolg der hier besprochenen Methoden ist deshalb die Anwendung einfallsreicher \textit{Lerntricks}, die zur Lösung bestimmter Probleme notwendig sind.}

Auch mit den hier beschriebenen Methoden müssen Sie eventuell schwierige Passagen hunderte Male und manchmal bis zu 10.000-mal üben, bevor Sie die schwierigsten Passagen mit Leichtigkeit spielen können.
Wenn Sie nun eine Beethoven-Sonate, sagen wir mit halber Geschwindigkeit (Sie lernen sie gerade), üben müssten, würde es ungefähr eine Stunde dauern, sie durchzuspielen.
Deshalb würde es 30 Jahre oder fast ein halbes Leben dauern, sie 10.000-mal zu wiederholen, wenn Sie eine Stunde täglich zum Üben hätten und 7 Tage die Woche nur diese Sonate üben würden.
Klar ist das nicht die richtige Art, die Sonate zu lernen, obwohl viele Schüler Übungsmethoden benutzen, die sich nicht sehr davon unterscheiden.
Dieses Buch beschreibt Methoden dafür, nur die wenigen Noten zu identifizieren, die man üben muss, und diese dann im Bruchteil einer Sekunde zu spielen, sodass man sie innerhalb weniger Wochen (oder bei leichterem Material sogar Tage) 10.000-mal wiederholen kann, und das bei einer Übungszeit von nur ungefähr 10 Minuten täglich an 5 Tagen die Woche - wir haben die Übungszeit von einem halben Leben auf ein paar Wochen reduziert.

Dieses Buch bespricht viele weitere Effizienz-Prinzipien, zum Beispiel gleichzeitig zu üben und \hyperref[c1iii6a]{auswendig zu lernen}.
\textbf{Während des Übens muss man jede Passage viele Male wiederholen, und Wiederholung ist die beste Art auswendig zu lernen; deshalb macht es keinen Sinn, während des Übens nicht auswendig zu lernen, insbesondere da dies der schnellste Weg zum Lernen ist.}
Haben Sie sich je gefragt, wie jeder Konzertpianist ein Repertoire von mehreren Stunden auswendig lernen kann?
Die Antwort ist ziemlich einfach.
\textbf{Studien mit Gedächtniskünstlern (wie denjenigen, die ganze Telefonbuchseiten auswendig lernen können) haben gezeigt, dass sie auswendig lernen können, weil sie Gedächtnisalgorithmen entwickelt haben, auf die sie das auswendig zu lernende Material schnell abbilden können.
Für Klavierspieler ist die Musik ein solcher Algorithmus.}
Sie können das beweisen, indem Sie einen Klavierspieler bitten, nur eine Seite zufälliger Noten auswendig zu lernen und sich jahrelang daran zu erinnern.
Das ist (ohne einen Algorithmus) unmöglich, obwohl dieser Klavierspieler vielleicht keine Schwierigkeit damit hat, sich mehrere 20 Seiten lange Beethoven-Sonaten zu merken und sie 10 Jahre später immer noch spielen zu können.
So stellt sich das, was wir für ein besonderes Talent der Konzertpianisten hielten, als etwas heraus, das jeder kann.
Schüler, die die Methoden dieses Buches benutzen, lernen - außer wenn sie das \hyperref[c1iii11]{Spielen vom Blatt} üben - alles auswendig, was sie lernen.
Darum empfiehlt dieses Buch keine Übungen wie \hyperref[c1iii7h]{Hanon} und Czerny, die nicht dazu gedacht sind, aufgeführt zu werden; aus demselben Grund sind die Chopin-Etüden allerdings empfehlenswert.
\textbf{Etwas zu üben, das nicht zur Aufführung gedacht ist, ist nicht nur eine Zeitverschwendung, sondern zerstört auch jeden Sinn für die Musik, den man ursprünglich hatte.}
Wir besprechen alle wichtigen Methoden des Auswendiglernens, die den Klavierspieler dazu befähigen, Kunststücke vorzuführen, die die meisten Menschen nur von \enquote{begnadeten Musikern} erwarten, wie 
\hyperref[c1ii12mental]{die Komposition im Kopf zu spielen}, ohne Klavier, oder sogar die ganze Komposition aus dem Gedächtnis niederzuschreiben.
Wenn Sie jede Note der Komposition aus dem Gedächtnis spielen können, gibt es keinen Grund, warum Sie sie nicht alle aufschreiben können!
Solche Fähigkeiten dienen nicht der Show oder zur Prahlerei, sondern sie sind für das \hyperref[c1iii14]{Vorspielen} ohne Fehler und Gedächtnislücken entscheidend, und sie ergeben sich fast als automatisches Nebenprodukt dieser Methoden, sogar für uns gewöhnliche Sterbliche mit einem gewöhnlichen Gedächtnis.
Viele Schüler können komplette Kompositionen spielen, sie aber nicht niederschreiben oder in Gedanken spielen - solche Schüler haben die Kompositionen nur zum Teil und auf eine Art auswendiggelernt, die für Auftritte unzureichend ist.
Ein unzureichendes Gedächtnis und ein Mangel an Selbstvertrauen sind die Hauptursachen für \hyperref[c1iii15]{Nervosität}.
Sie fragen sich, warum sie Lampenfieber bekommen und warum das fehlerfreie Vorspielen eine solch entmutigende Aufgabe ist, während Mozart sich einfach hinsetzen und spielen konnte.

\textbf{Weitere Beispiele hilfreichen Wissens sind die \hyperref[c1ii14]{Entspannung} und der Gebrauch der Schwerkraft.}
Das Gewicht des Arms ist nicht nur als Basis für gleichmäßiges und ausgeglichenes Spielen wichtig (die Schwerkraft ist immer konstant), sondern auch zum Testen des Grades der Entspannung.
\textbf{Das Klavier wurde mit der Schwerkraft als Referenz konstruiert (\hyperref[c1ii10]{Kapitel 1, Abschnitt II.10}), weil der menschliche Körper sich genau passend zur Schwerkraft entwickelte}, was bedeutet, dass die zum Klavierspielen notwendige Kraft ungefähr dem Gewicht des Arms entspricht.
Wenn wir schwierige Tätigkeiten ausführen, zum Beispiel eine anspruchsvolle Klavierpassage zu spielen, ist es unsere natürliche Neigung, uns zu verspannen, sodass der ganze Körper zu einer einzigen zusammengezogenen Muskelmasse wird.
Zu versuchen, die Finger unter solchen Bedingungen unabhängig voneinander und schnell zu bewegen, ist so, als ob man einen Sprint mit Gummibändern um beide Beine versuchen wollte.
Wenn Sie alle unnötigen Muskeln entspannen können und nur die erforderlichen Muskeln bloß für die Augenblicke benutzen, in denen sie gebraucht werden, dann können Sie längere Zeit ohne Anstrengung, ohne zu ermüden und mit mehr Kraftreserven als notwendig sind, um die lautesten Töne zu erzeugen, extrem schnell spielen. 

\textbf{Wir werden sehen, dass viele \enquote{etablierte Lehrmethoden} Mythen sind, die dem Schüler unbeschreibliches Elend verursachen können.}
Solche Mythen überleben aufgrund eines Mangels an gründlicher wissenschaftlicher Untersuchung.
Diese Methoden sind unter anderem: die gebogene Fingerhaltung, der Daumenuntersatz zum Spielen von Tonleitern, die meisten Fingerübungen, eine hohe Sitzposition, \enquote{ohne Fleiß kein Preis}, die Geschwindigkeit langsam steigern und der großzügige Gebrauch des Metronoms.
Wir erklären nicht nur, warum sie schädlich sind, sondern zeigen auch die korrekten Alternativen auf, welche jeweils folgende sind: \hyperref[c1iii4b]{flache Fingerhaltungen}, \hyperref[c1iii5a]{Daumenübersatz}, \hyperref[c1ii11]{parallele Sets}, eine niedrigere \hyperref[c1ii3]{Sitzposition}, \hyperref[c1ii14]{Entspannung}, schnelles Erreichen der Geschwindigkeit durch ein Verständnis der \enquote{Geschwindigkeitsbarrieren} und Aufzeigen besonders nützlicher Anwendungen des \hyperref[c1ii19]{Metronoms}.
\textbf{Auf \textit{Geschwindigkeitsbarrieren} trifft man, wenn man versucht, eine Passage schneller zu spielen, aber eine Maximalgeschwindigkeit erreicht, die man nicht mehr steigern kann, egal wie hart man übt.}
Was verursacht Geschwindigkeitsbarrieren, wie viele gibt es, und wie vermeidet oder eliminiert man sie?
Die Antworten: \textbf{Geschwindigkeitsbarrieren sind das Resultat Ihrer Versuche, das Unmögliche zu tun (das heisst Sie errichten die Geschwindigkeitsbarrieren selbst, indem Sie die falschen Übungsmethoden benutzen!), es gibt im Grunde eine unendliche Anzahl, und man vermeidet sie, indem man die richtigen Übungsmethoden benutzt.}
Eine Möglichkeit, Geschwindigkeitsbarrieren zu vermeiden, ist, sie gar nicht erst aufzubauen, indem man ihre Ursachen kennt (Stress, falscher Fingersatz oder \hyperref[c1iii1b]{Rhythmus}, Mangel an Technik, \hyperref[c1ii13]{zu schnelles Üben}, mit beiden Händen zusammen üben, bevor man dazu bereit ist, usw.).
\textbf{Eine weitere Möglichkeit ist, von unendlicher Geschwindigkeit aus mit der Geschwindigkeit abwärts zu gehen, indem man die \hyperref[c1ii11]{parallelen Sets} (oder den \hyperref[c1ii9]{Akkord-Anschlag}) benutzt, anstatt die Geschwindigkeit schrittweise zu steigern.}
Wenn Sie mit Geschwindigkeiten beginnen können, die oberhalb der Geschwindigkeitsbarrieren liegen, dann gibt es keine Geschwindigkeitsbarriere, wenn Sie die Geschwindigkeit verringern.

Dieses Buch behandelt oft einen wichtigen Punkt - dass die besten Übungsmethoden für das Klavierspielen überraschend kontraintuitiv sind.
Dieser Punkt ist in der Klavierpädagogik von größter Wichtigkeit, weil er der hauptsächliche Grund dafür ist, warum die falschen Übungsmethoden oft von den Schülern und den Lehrern benutzt werden.
Wenn sie nicht so kontraintuitiv wären, dann wäre dieses Buch nicht notwendig gewesen.
Folglich behandeln wir nicht nur, was man tun sollte, sondern auch, was man nicht tun sollte.
Diese negativen Abschnitte sind nicht dazu gedacht, diejenigen zu kritisieren, die die falschen Methoden benutzen, sondern sind notwendige Komponenten des Lernprozesses.
Der Grund, warum die Intuition falsch liegt, ist, dass die Aufgaben beim Klavierspielen so komplex sind und es so viele Möglichkeiten gibt, sie zu erfüllen, dass die Wahrscheinlichkeit, die richtige Methode zu treffen, nahe null ist, wenn man die einfachsten, offensichtlichen auswählt.
Dazu vier Beispiele kontraintuitiver Übungsmethoden:

\begin{enumerate}[label={\arabic*.}] 
\item \hyperref[c1ii7]{Die Hände beim Üben zu trennen} ist kontraintuitiv, weil man zunächst mit jeder einzelnen Hand üben muss, dann \hyperref[c1ii25]{mit beiden zusammen}, sodass es so aussieht, als müsste man dreimal üben anstatt nur einmal mit beiden Händen zusammen.
Warum soll man die Hände getrennt üben, also etwas, das man zum Schluss nie benutzen wird?
Ungefähr 80\% dieses Buchs handeln davon, warum man die Hände getrennt üben \textit{muss}.
\textbf{Die Hände getrennt zu üben ist der einzige Weg, schnell die Geschwindigkeit und die Kontrolle zu steigern, ohne in Schwierigkeiten zu kommen.}
Es erlaubt Ihnen, 100\% der Zeit bei jeder Geschwindigkeit ohne Ermüdung, Stress oder Verletzungen hart zu arbeiten, weil diese Methode darauf basiert, die Hände zu wechseln, sobald die arbeitende Hand anfängt müde zu werden.
\textbf{Die Hände getrennt zu üben ist die einzige Möglichkeit, wie Sie experimentieren können, um die korrekten Handbewegungen für die Geschwindigkeit und den Ausdruck zu finden, und es ist der schnellste Weg, um zu lernen \hyperref[c1ii14]{wie man entspannt}.}
Zu versuchen, sich die Technik mit beiden Händen zusammen anzueignen, ist die Hauptursache für Geschwindigkeitsbarrieren, schlechte Angewohnheiten, Verletzungen und Stress.

\item Man neigt intuitiv dazu, langsam \hyperref[c1ii25]{mit beiden Händen zusammen} zu üben und die Geschwindigkeit schrittweise zu steigern, das ist aber eine der schlechtesten Arten zu üben, weil sie so viel Zeit verschwendet und man die Hände dazu trainiert, langsame Bewegungen auszuführen, die sich von denen unterscheiden, die Sie bei der endgültigen Geschwindigkeit brauchen.
\textbf{Einige Schüler verschlimmern das Problem, indem sie das Metronom ständig  als Richtschnur benutzen, um die Geschwindigkeit zu steigern oder den \hyperref[c1iii1b]{Rhythmus} zu halten.
Das ist einer der schwersten Fälle von Missbrauch des Metronoms.
Ein Metronom sollten Sie nur kurz benutzen, um das Timing (Geschwindigkeit und Rhythmus) zu prüfen.}
Wenn Sie es zu viel benutzen, kann das zum Verlust Ihres internen Rhythmus, zum Verlust der Musikalität und zu biophysikalischen Schwierigkeiten führen, die entstehen können, wenn man starren Wiederholungen zu lange ausgesetzt ist (das Gehirn kann sogar anfangen, dem Metronomklick entgegenzuwirken, und man hört das Klicken eventuell nicht oder zur falschen Zeit).
\textbf{Die für die Geschwindigkeit notwendige Technik eignet man sich durch das Entdecken von neuen Handbewegungen an, nicht indem man eine langsame Bewegung beschleunigt}; das heisst die Handbewegungen für langsames Spielen und für schnelles Spielen unterscheiden sich voneinander.
Deshalb führt der Versuch, eine langsame Bewegung zu beschleunigen zu Geschwindigkeitsbarrieren - weil man versucht, das Unmögliche zu tun.
Langsames Spielen zu beschleunigen ist genau so, als ob man ein Pferd dazu bringen wollte, das Gehen auf die Geschwindigkeit des Galopps zu bringen - es kann es nicht.
Ein Pferd muss die Bewegung vom Gehen zum Trott, Kanter und dann zum Galopp ändern.
Wenn man das Pferd dazu zwingt, mit der Geschwindigkeit des Kanters zu gehen, dann wird es auf eine Geschwindigkeitsbarriere treffen und sich wahrscheinlich dadurch verletzen, dass es sich selbst die Hufe zertritt.

\item Um richtig auswendig zu lernen und in der Lage zu sein, gut vorzuspielen, muss man langsam üben, sogar nachdem man das Stück leicht mit der endgültigen Geschwindigkeit spielen kann.
Das ist kontraintuitiv, weil man beim Auftritt immer mit der endgültigen Geschwindigkeit spielt; warum soll man also langsam üben und so viel Zeit verschwenden?
Schnell zu spielen kann sowohl für das Vorspielen als auch für das Gedächtnis schädlich sein.
Schnell zu spielen kann \enquote{\hyperref[fpd]{Schnellspiel-Abbau}} verursachen, und die beste Möglichkeit, Ihr Gedächtnis zu testen, ist, langsam zu spielen.
\textbf{Deshalb wird es zu einem schlechten Auftritt führen, wenn man die Konzertstücke am Tag des Konzerts mit voller Geschwindigkeit übt.}
Das ist eine der kontraintuitivsten Regeln und deshalb schwer zu befolgen.
Wie oft haben Sie schon den Satz gehört: \enquote{Ich habe während der Unterrichtsstunde schrecklich gespielt, obwohl ich heute Morgen so gut gespielt habe!}?
Obwohl ein großer Teil dieses Buchs darauf ausgerichtet ist, zu lernen, mit der richtigen Geschwindigkeit zu spielen, ist deshalb der richtige Gebrauch der langsamen Geschwindigkeit für ein genaues Gedächtnis und ein fehlerfreies Vorspielen entscheidend.
Langsam zu üben ist jedoch gar nicht so einfach, weil man nicht langsam üben sollte, bevor man nicht schnell spielen kann!
Ansonsten hätten Sie keine Vorstellung davon, ob Ihre Bewegungen beim langsamen Spielen richtig oder falsch sind.
Dieses Problem wird gelöst, indem man die Hände getrennt übt und schnell die endgültige Geschwindigkeit erreicht.
Nachdem man die Handbewegungen für das schnelle Spielen kennt, kann man jederzeit langsam üben.

\item Den meisten Menschen fällt es schwer, etwas auswendig zu lernen, das sie nicht spielen können, weshalb sie instinktiv ein Stück zuerst lernen und \textit{dann} versuchen, es auswendig zu lernen.
Es stellt sich heraus, dass \textbf{man jede Menge Zeit sparen kann, indem man zuerst auswendig lernt und dann aus dem Gedächtnis heraus übt} (wir sprechen über technisch anspruchsvolle Musik, die zu schwierig ist, um sie vom Blatt zu spielen).
Außerdem behalten jene, die auswendig lernen nachdem sie das Stück gelernt haben, aus Gründen, die im Buch erläutert werden, niemals so gut.
Sie werden stets von Gedächtnisproblemen geplagt.
Deshalb müssen gute Methoden zum Auswendiglernen ein integraler Bestandteil jeder Übungsprozedur sein; Auswendiglernen ist eine Notwendigkeit, kein Luxus.
\end{enumerate}

Diese vier Beispiele sollten dem Leser eine gewisse Vorstellung davon geben, was ich mit kontraintuitiven Übungsmethoden meine.
Das Überraschende ist, dass \textit{die Mehrzahl} der guten Übungsmethoden für die meisten Menschen kontraintuitiv ist.
Glücklicherweise haben die Genies, die vor uns kamen, bessere Übungsmethoden gefunden.

Warum führt die Tatsache, dass die korrekten Methoden kontraintuitiv sind, zur Katastrophe?
Sogar Schüler, die die korrekten Methoden gelernt haben (denen aber nie beigebracht wurde, was man nicht tun darf), können in die intuitiven Methoden zurückfallen, weil ihnen ihr Gehirn einfach weiterhin sagt, sie sollten die intuitiven Methoden benutzen (das ist die \textit{Definition} von intuitiven Methoden).
Das geschieht Lehrern natürlich genauso.
Eltern tappen jedes Mal in diese Falle!
Dadurch kann eine Beteiligung der Eltern manchmal kontraproduktiv sein; die Eltern müssen ebenfalls \textit{informiert} sein.
Deshalb unternimmt dieses Buch jede Anstrengung, um die Torheiten der intuitiven Methoden zu ermitteln und herauszustellen.
Darum raten viele Lehrer von einer Beteiligung der Eltern ab, es sei denn, die Eltern können ebenfalls am Unterricht teilnehmen.
Wenn sie sich selbst überlassen werden, zieht es die Mehrzahl der Schüler, Lehrer und Eltern zu den intuitiven (falschen) Methoden.
Das ist der Hauptgrund, warum heutzutage so viele falsche Methoden gelehrt werden und warum Schüler informierte Lehrer und vernünftige Lehrbücher brauchen.
Alle Klavierlehrer sollten ein Lehrbuch benutzen, das Übungsmethoden erklärt; das wird sie davon befreien, die Mechanismen des Übens lehren zu müssen und ihnen gestatten, sich auf die Musik zu konzentrieren, wobei die Lehrer am meisten benötigt werden.
Die Eltern sollten das Lehrbuch ebenfalls lesen, weil Eltern sehr leicht in die Fallen der intuitiven Methoden geraten.

Klavierlehrer lassen sich im Allgemeinen in drei Kategorien einteilen:

\begin{enumerate}[label={\alph*.}] 
\item private Lehrer, die nicht unterrichten können,
\item private Lehrer, die sehr gut sind, und
\item Lehrer an Universitäten und Konservatorien.
\end{enumerate}

Die letzte Gruppe ist üblicherweise sehr gut, weil sie in ihrem Umfeld miteinander kommunizieren müssen.
Sie sind in der Lage, die schlimmsten Lehrmethoden schnell zu identifizieren und sie zu eliminieren.
Unglücklicherweise sind die meisten Schüler an Konservatorien bereits ziemlich fortgeschritten, und somit ist es zu spät, um ihnen grundlegende Übungsmethoden beizubringen.
Die Gruppe (a) besteht hauptsächlich aus \enquote{Einzelkämpfern}, die sich nicht so sehr mit anderen Lehrern austauschen und hauptsächlich die intuitiven Methoden benutzen; das erklärt, warum sie nicht unterrichten können.
Sie können viele der schlechten Lehrer umgehen, wenn Sie nur Lehrer auswählen, die eine Website unterhalten, weil diese zumindest gelernt haben zu kommunizieren.
Die Gruppen (b) und (c) sind mit den korrekten Übungsmethoden ziemlich vertraut, obwohl wenige sie alle kennen, weil es kein standardisiertes Lehrbuch gab; auf der anderen Seite wissen die meisten von Ihnen eine Menge nützlicher Details, die nicht in diesem Buch enthalten sind.
Es gibt herzlich wenige Lehrer der Gruppe (b) und die Lehrer der Gruppe (c) akzeptieren im Allgemeinen nur fortgeschrittene Schüler.
Das Problem mit dieser Situation ist, dass die meisten Schüler mit Lehrern der Gruppe (a) anfangen und nie über das Anfänger- oder Mittelstufenniveau hinauskommen und sich deshalb niemals für die Lehrer der Gruppe (c) qualifizieren.
Deshalb gibt die Mehrzahl der Anfänger frustriert auf, obwohl praktisch alle von ihnen das Potential haben, ein vollendeter Musiker zu werden.
Mehr noch, dieser Mangel an Vorwärtskommen nährt das allgemeine Missverständnis, dass Klavierspielen zu lernen ein lebenslanges fruchtloses Bemühen bedeutet, das die Mehrzahl der Eltern und Kinder davon abhält, über Klavierunterricht nachzudenken.

Es gibt eine innige Beziehung zwischen Musik und Mathematik.
Musik ist in vielerlei Hinsicht eine Form der Mathematik, und die großen Komponisten haben diese Beziehung untersucht und ausgenutzt.
Die meisten grundlegenden Theorien der Musik können mit mathematischen Termen ausgedrückt werden.
Die Harmonie ist eine Reihe von Verhältnissen, und die Harmonie führt zur chromatischen Tonleiter, die eine logarithmische Gleichung ist.
Die meisten Tonleitern sind Teilmengen der chromatischen Tonleiter, und Akkordprogressionen sind die einfachsten Beziehungen unter diesen Teilmengen.
Ich bespreche einige konkrete Beispiele für den Gebrauch der Mathematik in einigen der berühmtesten Kompositionen (\hyperref[c1iv4]{Kapitel 1, Abschnitt IV.4}) und schließe alle \hyperref[c1iv6]{Themen für zukünftige Untersuchungen} in der Musik (sowohl mathematische als auch andere) in Kapitel 1, Abschnitt IV ein.
Es macht keinen Sinn, zu fragen, ob Musik Kunst oder Mathematik ist; beide sind Eigenschaften der Musik.
Mathematik ist einfach eine Möglichkeit, etwas quantitativ zu messen; deshalb kann alles, was in der Musik quantifiziert werden kann (zum Beispiel das Taktmaß, die thematische Struktur usw.), mathematisch behandelt werden.
Obwohl die Mathematik für einen Künstler nicht notwendig ist, sind deshalb Musik und Mathematik untrennbar verbunden.
Das Wissen um diese Beziehung kann oft nützlich sein (wie von jedem großen Komponisten gezeigt wird) und wird immer nützlicher werden, je mehr sich das musikalische Verständnis der Mathematik der Musik schrittweise nähert und die Künstler lernen, einen Nutzen aus der Mathematik zu ziehen.
Die Kunst ist eine Abkürzung, bei der das menschliche Gehirn dazu benutzt wird, Ergebnisse zu erzielen, die auf anderem Wege nicht zu erzielen sind.
Ein wissenschaftliches Herangehen an die Musik beschäftigt sich nur mit den einfacheren Ebenen der Musik, die analytisch behandelt werden können: Die Wissenschaft unterstützt die Kunst.
Die Annahme ist falsch, dass die Wissenschaft irgendwann die Kunst ersetzen wird oder, das andere Extrem, dass man für Musik nur die Kunst benötigt; die Kunst sollte so frei sein, alles einzufügen, das der Künstler wünscht, und die Wissenschaft kann eine unschätzbare Hilfe bieten.

Zu viele Klavierspieler wissen nicht, wie das Klavier funktioniert und was es bedeutet, \hyperref[c2_2c]{temperiert zu stimmen}, oder was es bedeutet, ein Klavier zu \hyperref[c2_7_hamm]{intonieren}.
Das ist besonders überraschend, weil die Wartung des Klaviers einen direkten Einfluss auf die Fähigkeit Musik zu machen und auf die Entwicklung der Technik hat.
Es gibt viele Konzertpianisten, die nicht den Unterschied zwischen \hyperref[et1]{gleichschwebender Temperatur} und wohltemperierten Stimmungen kennen, obwohl einige der Kompositionen, die sie spielen (zum Beispiel Bach), den Gebrauch der einen oder der anderen ausdrücklich verlangen.
Wann man ein elektronisches Klavier benutzen soll, wann man zu einem Klavier oder Flügel höherer Qualität wechseln soll und wie man bei einem Klavier Qualität erkennt, sind in der Karriere eines jeden Klavierspielers wichtige Fragen.
Deshalb enthält dieses Buch einen Abschnitt über den \hyperref[c1iii17e]{Kauf eines Klaviers} und ein Kapitel über das \hyperref[c2_1]{Stimmen des eigenen Klaviers}.
So wie die elektronischen Klaviere bereits immer richtig gestimmt sind, so müssen auch die akustischen Klaviere in naher Zukunft dauerhaft richtig gestimmt sein, zum Beispiel indem man den temperaturabhängigen Ausdehnungskoeffizienten der Saiten benutzt, um das Klavier elektronisch zu stimmen (siehe Gilmore, http://home.kc.rr.com/eromlignod, self-tuning piano).
Heutzutage sind praktisch alle Heimklaviere fast die ganze Zeit aus der Stimmung, weil das Klavier anfängt aus der Stimmung zu gehen, sobald der Stimmer das Haus verlässt oder sich die Temperatur oder Feuchtigkeit im Raum ändert.
Das ist eine unannehmbare Situation.
Bei zukünftigen Klavieren wird man einen Schalter umlegen können, und das Klavier stimmt sich innerhalb von Sekunden selbst.
Wenn sie massenweise produziert werden, sind die Kosten eines sich selbst stimmenden Klaviers im Vergleich zu einem Qualitätsklavier gering.
Man könnte meinen, dass dies die Klavierstimmer arbeitslos machen würde, aber das wird nicht der Fall sein, weil die Zahl der Klaviere (durch dieses Buch) zunehmen wird, der Mechanismus zum Selbststimmen gewartet werden muss und bei Klavieren mit einer solch perfekten Stimmung das regelmäßige Intonieren der Hämmer und Einstellen (beides wird heute oft vernachlässigt) eine bedeutsame Verbesserung des musikalischen Ergebnisses bewirkt.
Durch die gestiegene Anzahl der fortgeschrittenen Klavierspieler entsteht eine größere Nachfrage nach diesen gehobenen Wartungsarbeiten.
Sie könnten plötzlich erkennen, dass es das Klavier war, nicht Sie selbst, das die technische Entwicklung und das musikalische Ergebnis begrenzt hat (bei abgenutzten Hämmern ist das immer der Fall!).
Was denken Sie, warum Konzertpianisten so viel Aufhebens um ihr Klavier machen?

Zusammengefasst: Dieses Buch stellt ein einmaliges Ereignis in der Geschichte der Klavierpädagogik dar und revolutioniert den Klavierunterricht.
Überraschenderweise ist wenig in diesem Buch grundlegend neu.
Wird verdanken die meisten wichtigen Konzepte Combe, Liszt, Chopin, Beethoven, Mozart, Bach usw.
Combe und Liszt gaben uns das \hyperref[c1ii7]{Üben mit getrennten Händen}, das \hyperref[c1ii6]{Üben in kleinen Portionen} und die \hyperref[c1ii14]{Entspannung};
Liszt und Chopin gaben uns den \hyperref[c1iii5b]{Daumenübersatz} und befreiten uns von \hyperref[c1iii7h]{Hanon} und Czerny;
Mozart lehrte uns das \hyperref[c1iii6]{Auswendiglernen} und das \hyperref[c1ii12mental]{mentale Spielen};
Bach wusste alles über \hyperref[c1ii11]{parallele Sets}, \hyperref[ruhig]{ruhige Hände} und die Wichtigkeit des \hyperref[c1iii14d]{musikalischen Übens},
und sie alle (besonders Beethoven) zeigten uns die Beziehungen zwischen Mathematik und Musik.
Die enorme Menge an Zeit und Anstrengung, die, um das Rad neu zu erfinden und beim nutzlosen Wiederholen von Fingerübungen, in der Vergangenheit in jeder Pianistengeneration verschwendet wurde, übersteigt alle Vorstellungen.
Indem das in diesem Buch zusammengestellte Wissen dem Schüler vom ersten Tag des Klavierunterrichts an zugänglich gemacht wird, läuten wir ein neues Zeitalter im Erlernen des Klavierspielenlernens ein.
Dieses Buch ist nicht das Ende der Straße - es ist nur ein Anfang.
Die zukünftige Erforschung der Übungsmethoden wird zweifellos Verbesserungen zu Tage fördern; das liegt in der Natur des wissenschaftlichen Vorgehens.
Es garantiert, dass wir nie wieder nützliche Informationen verlieren und immer nur voranschreiten werden, und dass jeder Lehrer Zugang zu den besten verfügbaren Informationen haben wird.
Wir verstehen bislang noch nicht die biologischen Veränderungen, die den Erwerb der Technik begleiten, und wie sich das menschliche (besonders das kindliche) Gehirn entwickelt.
Diese zu verstehen wird uns erlauben, sie direkt hervorzubringen, statt dass wir etwas 10.000-mal wiederholen müssen.
Seit Bachs Zeit gab es in der Klavierpädagogik einen Stillstand der Entwicklung; wir haben nun die Hoffnung, das Klavierspielen von einem scheinbar unerreichbaren Traum in eine Kunst zu verwandeln, an der sich jeder erfreuen kann.

PS: Dieses Buch ist mein Geschenk an die Gesellschaft.
Die Übersetzer haben ebenfalls ihre kostbare Zeit dazu beigetragen.
Zusammen leisten wir Pionierarbeit dafür, kostenlos Web-basierte Ausbildung von höchstem Format zur Verfügung zu stellen, etwas, das hoffentlich zu einem Vorboten der Zukunft wird.
Es gibt keinen Grund, warum Ausbildung nicht kostenlos sein sollte.
Eine solche Umwälzung mag so erscheinen, als ob sie die Jobs einiger Lehrer gefährden könnte, aber mit verbesserten Lehrmethoden wird das Klavierspielen viel populärer werden, was zu einer höheren Nachfrage nach Lehrern führen wird, die unterrichten können, weil Schüler mit einem guten Lehrer immer schneller lernen werden als wenn sie alleine sind.
Die ökonomischen Auswirkungen dieser verbesserten Lernmethoden können beträchtlich sein.
Dieses Buch wurde 1994 zuerst gedruckt, und die Website\footnote{das heisst die von Chuan C. Chang} startete 1999.
Ich schätze, dass bis zum Jahr 2002 mehr als 10.000 Schüler diese Methode gelernt hatten.
Nehmen wir an, dass 10.000 ernsthafte Klavierschüler durch diese Methoden 5 Stunden je Woche einsparen, dass sie 40 Wochen pro Jahr üben, und dass ihre Zeit einem Wert von 5\$ je Stunde entspricht; dann ist die gesamte jährliche Ersparnis:

(5 Stunden / (Woche * Schüler)) * (40 Wochen / Jahr) * (\$5 / Stunde) * (10.000 Schüler) = \$10.000.000 / Jahr für 2002 oder \$1.000 / Jahr und Schüler.

10 Millionen Dollar pro Jahr sind nur die Einsparung der Schüler; wir haben die Auswirkungen für die Lehrer sowie die Klavier- und Musikindustrie nicht berücksichtigt.
Jedes Mal, wenn die Übernahme wissenschaftlicher Methoden solche Sprünge in der Effektivität erzeugt hat, hat das jeweilige Gebiet in der Vergangenheit einen Aufschwung erlebt, der anscheinend grenzenlos war und jedem genutzt hat.
Bei der heutigen (2007) Weltbevölkerung von mehr als 6,6 Milliarden Menschen können wir davon ausgehen, dass der Anteil der Klavierspieler schließlich ein Prozent oder mehr als 66 Millionen betragen wird, sodass die potentielle ökonomische Auswirkung dieses Buchs mehrere Milliarden Dollar pro Jahr übersteigen könnte.
Ein solcher wirtschaftlicher Nutzen in einem kleinen Sektor war in der Vergangenheit eine unüberwindliche Kraft, und dieser Motor wird den Umschwung beim Klavierspielen weiter antreiben.
Noch wichtiger ist, dass Musik und jeder Zuwachs bei der geistigen Entwicklung eines Kindes unbezahlbar sind.
 



