% File: c1iv3

\section{Die Thermodynamik des Klavierspielens}\hypertarget{c1iv3}{}

Ein wichtiges Feld der Mathematik ist das Studium großer Anzahlen.
Auch wenn einzelne Ereignisse eines bestimmten Typs nicht vorhersagbar sind, verhalten sich große Anzahlen solcher Ereignisse oft gemäß strikter Gesetze.
Obwohl die Energien einzelner Wassermoleküle in einem Glas Wasser erheblich voneinander abweichen können, kann die Wassertemperatur sehr konstant bleiben und mit sehr großer Genauigkeit gemessen werden.
Hat Klavierspielen eine analoge Situation, die es uns erlaubt, die Gesetze großer Anzahlen anzuwenden und dabei einige nützliche Schlüsse zu ziehen?


\hypertarget{canonic}{}

Klavierspielen ist wegen der großen Anzahl von Variablen, die in das Erzeugen von Musik eingehen, ein komplexer Vorgang.
Das Studium großer Anzahlen wird durch das Zählen der \enquote{Anzahl der Zustände} eines Systems erreicht.
Die Gesamtzahl der so gezählten bedeutungsvollen Zustände könnte die \enquote{\hyperlink{ueb-canonic}{kanonische Gesamtheit}} genannt werden, eine bedeutungsvolle Ansammlung, die zusammen ein Lied singt, das wir entziffern können.
Deshalb müssen wir nur die kanonische Gesamtheit berechnen, und wenn wir damit fertig sind, wenden wir einfach die bekannten mathematischen Regeln großer Systeme (z.B. der Thermodynamik) an und voilà!
Wir sind fertig!

Die fraglichen Variablen sind hier klar die unterschiedlichen Bewegungen des menschlichen Körpers, besonders jener Teile, die beim Klavierspielen wichtig sind.
Unsere Aufgabe ist es, all die Arten zu zählen, in denen der Körper beim Klavierspielen bewegt werden kann; das ist sicherlich eine sehr große Zahl; die Frage ist, ob sie groß genug für eine bedeutungsvolle kanonische Gesamtheit ist.

Da niemand jemals versucht hat, diese kanonische Gesamtheit zu berechnen, erforschen wir hier ein neues Gebiet, und ich werde nur eine ungefähre Schätzung abgeben.
Das Schöne an den kanonischen Gesamtheiten ist, daß am Ende, wenn die Berechnungen richtig sind (eine berechtigte Sorge bei etwas so neuem), die Methode, die benutzt wurde, um dorthin zu gelangen, üblicherweise unerheblich ist - man kommt immer zu denselben Antworten.
Wir berechnen die Gesamtheit, indem wir alle relevanten Variablen auflisten und den gesamten Parameterraum dieser Variablen zählen.
Fangen wir also an.

Beginnen wir mit den Fingern.
Finger können sich auf, ab und seitwärts bewegen sowie gekrümmt und gestreckt werden (drei Variablen).
Sagen wir, es gibt 10 meßbare verschiedene Positionen für jede Variable (Parameterraum = 10); wenn wir nur die Anzahl der 10er-Gruppen berücksichtigen, haben wir bisher 4, einschließlich der Tatsache, daß wir 10 Finger haben.
In Wahrheit gibt es viel mehr Variablen (wie das Rotieren der Finger um jede Fingerachse) und mehr als 10 meßbare Parameter je Variable, aber wir zählen nur die Zustände, die vernünftig benutzt werden können, um ein bestimmtes Stück auf dem Klavier zu spielen.
Der Grund für diese Einschränkung ist, daß wir die Resultate dieser Berechnungen benutzen werden, um zu vergleichen, wie zwei Personen dasselbe Stück spielen oder wie eine Person es zweimal hintereinander spielt.
Das wird später klarer werden.

Nun können die Handflächen angehoben oder gesenkt, seitwärts gebeugt und um die Achse des Unterarms gedreht werden.
Das sind 3 weitere 10er-Gruppen, womit wir insgesamt 7 hätten.
Der Unterarm kann angehoben oder gesenkt und seitwärts geschwungen werden; die neue Summe ist 9.
\footnote{Anatomisch genau genommen ist die Drehung des Handgelenks ja eine Drehung des Unterarms im Ellbogen, aber das ist hier egal, weil wir trotzdem jetzt bei 9 sind.}
Der Oberarm kann vor oder zurück und seitwärts geschwungen werden; die neue Summe ist 11.
Der Körper kann vorwärts oder rückwärts und seitwärts bewegt werden; die neue Summe ist 13.
Dann gibt es noch die Variablen der Kraft, Geschwindigkeit und Beschleunigung; macht insgesamt mindestens 16.
Somit umfaßt der gesamte Parameterraum eines Klavierspielers viel mehr als 10 hoch 16 Zustände (eine 1 gefolgt von 16 Nullen!).
Die tatsächliche Anzahl für ein bestimmtes Musikstück ist um mehrere Zehnerpotenzen höher, weil die obige Berechnung nur für eine Note gilt und ein typisches Musikstück tausende oder zehntausende Noten enthält.
Der daraus resultierende Parameterraum ist deshalb ungefähr 10 hoch 20.
Das nähert sich dem Gesamtheitsraum für Moleküle; so hat z.B. ein Kubikzentimeter Wasser 10 hoch 23 Moleküle, von denen jedes viele Freiheitsgrade in der Bewegung und viele mögliche Energiezustände hat.
Da die Thermodynamik auf Wasservolumina von wesentlich weniger als 0,0001 Kubikzentimeter anwendbar ist, kommt die kanonische Gesamtheit eines Klavierspielers thermodynamischen Bedingungen sehr nah.

Wenn die kanonische Gesamtheit des Klavierspielers fast thermodynamische Eigenschaften hat, welchen Schluß können wir daraus ziehen?
Das wichtigste Ergebnis ist, daß jeder einzelne Punkt in diesem Phasenraum völlig irrelevant ist, weil die Wahrscheinlichkeit, daß man diesen Punkt reproduzieren kann, nahe Null ist.
Aus diesem Ergebnis können wir sofort einige Schlüsse ziehen:

\textbf{Erstes Gesetz der Klavierdynamik}: Keine zwei Personen können dasselbe Musikstück auf exakt die gleiche Weise spielen.
Eine Folgerung aus diesem ersten Gesetz ist, daß dieselbe Person, die dasselbe Musikstück zweimal spielt, es niemals exakt auf die gleiche Weise spielen wird.

Na und? Nun, das bedeutet, daß die Vorstellung, man könnte, wenn man jemandem beim Spielen zuhört, seine Kreativität dadurch verringern, daß man diesen Künstler imitiert, keine gültige Vorstellung ist, da ein exaktes Imitieren niemals möglich ist.
Es unterstützt wirklich die Lehrmeinung, nach der behauptet wird, daß einem guten Künstler zuzuhören nicht schädlich sein kann.
Jeder Pianist ist ein einzigartiger Künstler, und niemand wird jemals seine Musik reproduzieren.
Die Folgerung bietet eine wissenschaftliche Erklärung für den Unterschied zwischen dem Anhören einer Aufnahme (die eine Aufführung exakt reproduziert) und dem Zuhören bei einem Live-Konzert, das niemals exakt reproduziert werden kann (außer als Aufnahme).

\textbf{Zweites Gesetz der Klavierdynamik}: Wir können niemals jeden Aspekt davon, wie wir ein bestimmtes Stück spielen, völlig kontrollieren.

Dieses Gesetz ist nützlich für das Verständnis dafür, wie wir uns unbewußt schlechte Angewohnheiten aneignen können und wie die Musik wenn wir auftreten ihr Eigenleben bekommt und uns auf manche Arten außer Kontrolle gerät.
Die mächtigen Gesetze der Klavierdynamik übernehmen in diesen Fällen die Führung, und es ist nützlich, unsere Grenzen und die Quellen unserer Schwierigkeiten zu kennen, um sie soviel wie möglich zu kontrollieren.
Es ist ein wahrhaft demütigender Gedanke, nach langem, harten Üben festzustellen, daß wir ohne den leisesten Hauch einer Ahnung jede beliebige Zahl von schlechten Gewohnheiten angenommen haben könnten.
Das mag in der Tat eine Erklärung dafür bieten, warum es so nützlich ist, beim letzten Durchlauf während des Übens langsam zu spielen.
Indem man langsam und exakt spielt, verringert man den Gesamtheitsraum in hohem Maß und schließt die \enquote{schlechten} Bewegungen aus, die weit von dem Raum der \enquote{richtigen} entfernt sind.
Wenn diese Prozedur tatsächlich bestimmte schlechte Angewohnheiten eliminiert und sich der Effekt von Übungseinheit zu Übungseinheit kumuliert, dann kann das auf lange Sicht einen großen Unterschied in der Rate erzeugen, mit der Sie die Technik erwerben.



