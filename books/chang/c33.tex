% File: c33

\section{Was ist die Wissenschaftliche Methode?}
\label{c3_3}

\subsection{Einleitung}
\label{c3_3a}

Eine häufige falsche Vorstellung ist, daß Klavierspielen eine Kunst ist und deshalb der wissenschaftliche Ansatz nicht möglich und nicht anwendbar sei.
Diese falsche Vorstellung ist auf ein falsches Verständnis dafür zurückzuführen, was Wissenschaft ist.
Es mag viele Menschen überraschen, daß Wissenschaft in Wahrheit eine Kunst ist; Wissenschaft und Kunst können nicht voneinander getrennt werden, so wie Klaviertechnik und musikalisches Spielen nicht voneinander getrennt werden können.
Wenn Sie es nicht glauben, gehen Sie einfach zu irgendeiner großen Universität.
Sie wird immer eine herausragende Abteilung besitzen: die Abteilung für Kunst und Wissenschaft.
Beide erfordern Vorstellungskraft, Originalität und die Fähigkeit zur Ausführung.
Zu sagen, daß eine Person die Wissenschaft nicht kenne und deshalb einen wissenschaftlichen Ansatz nicht benutzen könne, ist so, als ob man sagen würde, daß man, wenn man weniger weiß, weniger lernen sollte.
Das macht keinen Sinn, weil es genau die Person, die weniger weiß, ist, die mehr lernen muß.
Offensichtlich müssen wir klar definieren, was Wissenschaft ist.


\subsection{Definition}
\label{c3_3b}

\textbf{Die einfachste Definition der wissenschaftlichen Methode ist, daß sie jede Methode ist, die funktioniert}.
Die wissenschaftliche Methode ist eine, die in völliger Harmonie mit der Realität oder Wahrheit ist.
Wissenschaft ist Befähigung.
Deshalb ist zu sagen, daß \enquote{Wissenschaft nur etwas für Wissenschaftler ist}, so, als ob man sagen würde, daß Jumbo Jets nur etwas für Luftfahrtingenieure sind.
Es ist wahr, daß  Flugzeuge nur von Luftfahrtingenieuren gebaut werden können, aber das hindert nicht einen von uns daran, Flugzeuge für unsere Reisen zu benutzen -- in Wahrheit sind diese Flugzeuge für uns gebaut worden.
Genauso ist der Zweck der Wissenschaft, das Leben für alle leichter zu machen, nicht nur für Wissenschaftler.

Obwohl kluge Wissenschaftler benötigt werden, um die Wissenschaft voran zu bringen, kann jeder von der Wissenschaft profitieren.
Deshalb \textbf{ist eine weitere Möglichkeit, Wissenschaft zu definieren, daß sie zuvor unmögliche Aufgaben ermöglicht und schwierige Aufgaben vereinfacht.}
Von diesem Standpunkt aus nützt Wissenschaft den Unwissenderen unter uns mehr als den besser Informierten, die Dinge selbst herausfinden können.
Dazu ein Beispiel: Wenn ein Analphabet gebeten würde, zwei sechsstellige Zahlen zu addieren, hätte er keine Möglichkeit, es von selbst zu tun.
Jeder Drittklässler jedoch, der Rechnen gelernt hat, kann diese Aufgabe ausführen, wenn man ihm einen Stift und Papier gibt.
Heute kann man dem Analphabeten innerhalb weniger Minuten beibringen, diese Zahlen auf einem Taschenrechner zu addieren.
Nachweislich hat die Wissenschaft eine zuvor unmögliche Aufgabe für jeden leicht gemacht.

Die obigen Definitionen der wissenschaftlichen Methode liefern keine direkte Information darüber, wie man ein wissenschaftliches Projekt durchführt.
\textbf{Eine praktische Definition des wissenschaftlichen Ansatzes ist, daß er eine Gruppe von eindeutig definierten Objekten und deren Beziehungen zueinander ist}.
Eine der nützlichsten Beziehungen ist ein Klassifizierungsschema, das Objekte in Klassen und Unterklassen einteilt.
Beachten Sie, daß das Wort \enquote{definieren} eine sehr spezielle Bedeutung bekommt.
Objekte müssen in einer solchen Art definiert werden, daß sie nützlich sind und auf eine solche Art, daß die Beziehungen zwischen ihnen präzise beschrieben werden können.
Und all diese Definitionen und Beziehungen müssen wissenschaftlich korrekt sein (hierbei bekommen Nichtwissenschaftler Probleme).

Lassen Sie uns ein paar Beispiele ansehen.
Musiker haben grundlegende Objekte, wie z.B. \hyperref[c1iii5a]{Tonleitern}\index{Tonleitern}, \hyperref[c1iii7e]{Akkorde}\index{Akkorde}, Harmonien, Verzierungen usw., definiert.
In diesem Buch wurden wichtige Konzepte, wie z.B. \hyperref[c1ii7]{Üben mit getrennten Händen}\index{Üben mit getrennten Händen}, \hyperref[c1ii9]{Akkord-Anschlag}\index{Akkord-Anschlag}, \hyperref[c1ii11]{parallele Sets}\index{parallele Sets}, \hyperref[c1ii5]{abschnittsweises Üben}\index{abschnittsweises Üben}, \hyperref[c1ii15]{automatische Verbesserung der Fähigkeiten nach dem Üben (PPI)}\index{automatische Verbesserung der Fähigkeiten nach dem Üben (PPI)} usw., präzise definiert.
Damit diese wissenschaftliche Methode, dieses Buch zu schreiben, funktioniert (d.h. damit ein nützliches Lehrbuch herauskommt), ist es notwendig, alle nützlichen Beziehungen zwischen diesen Objekten zu kennen.
Insbesondere ist es wichtig, vorauszusehen was der Leser \textit{benötigt}.
Der Akkord-Anschlag wurde als Antwort auf eine Notwendigkeit zur Lösung eines Geschwindigkeitsproblems definiert.
Man kann hier sehen, warum die Physik nicht so wichtig ist wie die menschliche Befähigung.
Ich habe verschiedene Bücher gelesen, die das Staccato besprechen, ohne es jemals zu definieren.
Die Wissenschaft spielt bereits auf den grundlegendsten Stufen der Definitionen, Erklärungen und Anwendungen eine Rolle.
Der Autor muß bestens mit den besprochenen Themen vertraut sein, damit er keine fehlerhaften Aussagen macht.
Das ist der Kern der Wissenschaft, nicht Mathematik oder Physik.

Eines der Probleme mit \hyperref[Whiteside]{Whitesides Buch} ist der Mangel an präzisen Definitionen.
Sie benutzt viele Worte und Konzepte, wie z.B. \hyperref[c1iii1b]{Rhythmus}\index{Rhythmus} und \hyperref[c1iii8]{Konturieren}\index{Konturieren}, ohne sie zu definieren.
Das macht es für den Leser schwierig, zu verstehen was sie sagt oder ihre Anweisungen anzuwenden.
Natürlich mag es zunächst unmöglich erscheinen, diese komplexen Konzepte, auf die wir in der Musik oft treffen, zu definieren, besonders wenn man alle Komplexitäten und Nuancen im Umfeld eines schwierigen Konzeptes einschließen möchte.
Es ist jedoch die normale wissenschaftliche Vorgehensweise, Bestimmungsgrößen zu benutzen, um die Definition zu begrenzen, wenn man bestimmte Beispiele benutzt und andere Bestimmungsgrößen, um die Definition auf andere Möglichkeiten auszudehnen.
Es ist nur eine Frage sowohl des Verständnisses des Themas als auch der Bedürfnisse des Lesers.
\hyperref[Fink]{Finks} und \hyperref[Sandor]{Sandors} Buch bieten Beispiele von ausgezeichneten Definitionen.
Was ihnen fehlt, sind die Beziehungen: ein systematischer, strukturierter Ansatz, wie man diese Definitionen benutzt, um die Technik Schritt für Schritt zu erwerben.
Sie haben auch ein paar der wichtigen Definitionen vergessen, die in diesem Buch enthalten sind.

Der Hauptbestandteil der wissenschaftlichen Methode ist Wissen, aber Wissen alleine ist nicht genug.
Dieses Wissen muß in eine Struktur gebracht werden, so daß wir die Beziehungen zwischen den Objekten sehen, verstehen und ausnutzen können.
Ohne diese Beziehungen weiß man nicht, ob man alle notwendigen Teile hat oder gar wie man sie benutzt.
So sind z.B. \hyperref[c1ii11]{parallele Sets}\index{parallele Sets} ziemlich nutzlos, solange man das HS-Üben nicht kennt.
Die häufigste Methode, diesen Überbau herzustellen, ist ein Klassifizierungsschema.
In diesem Buch werden die verschiedenen Verfahren in Anfängermethoden, mittlere Stufen des Lernens, Methoden zum Auswendiglernen, Methoden zur Steigerung der Geschwindigkeit, schlechte Angewohnheiten usw. eingeteilt.
Hat man erst einmal die Definitionen und das Klassifizierungsschema, muß man anschließend die Details darüber, wie alles zusammengehört und ob es fehlende Elemente gibt, hinzufügen.
Wir besprechen nun einige besondere Komponenten der wissenschaftlichen Methode.


\subsection{Forschung}
\label{c3_3c}

Ein Handbuch über das Klavierspielen ist im Grunde eine Liste von Entdeckungen, wie man einige technische Probleme löst.
Es ist ein Produkt der Forschung.
In der wissenschaftlichen Forschung führt man Experimente durch, sammelt die Daten und schreibt die Resultate auf eine Art nieder, daß andere verstehen können, was man getan hat, und die Resultate reproduzieren können.
Klavierspielen zu lehren ist nicht anders.
Man muß zunächst verschiedene Übungsmethoden erforschen, die Resultate sammeln und sie aufschreiben, so daß andere davon profitieren können.
Klingt trivial einfach.
Aber wenn man sich umschaut, ist das \textit{nicht} das, was in bezug auf den Klavierunterricht geschehen ist.
Liszt hat seine Übungsmethoden niemals schriftlich festgehalten.
Die \enquote{intuitive Methode} (wie sie in diesem Buch beschrieben wird) erfordert keine Forschung; sie ist die am wenigsten informierte Art zu üben.
Deshalb war \hyperref[Whiteside]{Whitesides Buch} so erfolgreich -- sie führte Forschungen durch und hielt ihre Ergebnisse fest.
Leider hatte sie keine wissenschaftliche Ausbildung und versagte bei wichtigen Aspekten, wie z.B. einem klaren, kurzen Schreibstil (besonders bei den Definitionen) und der Ordnung (Klassifizierung und Beziehungen).
Wenn es uns gelingt, diese Unzulänglichkeiten zu korrigieren, dann besteht natürlich einige Hoffnung, daß wir wissenschaftliche Methoden auf das Lehren des Klavierspielens anwenden können.
Offensichtlich wurde von allen großen Pianisten ein enormes Maß an Forschung durchgeführt.
Unglücklicherweise wurde sehr wenig davon dokumentiert; es fiel dem unwissenschaftlichen Ansatz der Klavierpädagogik zum Opfer.


\subsection{Dokumentation und Kommunikation}
\label{c3_3d}

Das oberste Ziel der Dokumentation ist die Aufzeichnung allen Wissens auf einem Gebiet -- es ist ein unschätzbarer Verlust, daß Bach, Chopin, Liszt usw. ihre Übungsmethoden nicht niedergeschrieben haben.
Eine weitere Funktion der wissenschaftlichen Dokumentation ist das Eliminieren von Fehlern.
Eine korrekte Idee, die von einem großen Meister formuliert und mündlich von den Lehrern an die Schüler weitergegeben wurde, ist fehleranfällig und völlig unwissenschaftlich.
Wenn die Idee niedergeschrieben ist, kann man sie auf ihre Genauigkeit überprüfen und alle Fehler beseitigen sowie neue Erkenntnisse hinzufügen.
D.h., Dokumentation erzeugt eine Einbahnstraße, bei der sich eine Idee im Laufe der Zeit in ihrer Genauigkeit nur verbessern kann.

Eine Erkenntnis, die sogar Wissenschaftler überrascht hat, ist, daß ungefähr die Hälfte aller neuen Entdeckungen nicht während der Durchführung der Forschungen gemacht werden, sondern wenn die Resultate niedergeschrieben werden.
Aus diesem Grund hat sich das wissenschaftliche Schreiben zu einem Gebiet mit besonderen Erfordernissen entwickelt, die so beschaffen sind, daß nicht nur die Fehler minimiert werden, sondern auch der Entdeckungsprozeß maximiert wird.
Während des Schreibens dieses Buchs entdeckte ich die Erklärung für die \hyperref[c1iv2b]{Geschwindigkeitsbarrieren}\index{Geschwindigkeitsbarrieren}.
Ich war damit konfrontiert, etwas über Geschwindigkeitsbarrieren zu schreiben und begann mich natürlich zu fragen, was sie sind und was sie erzeugt.
Es ist wohlbekannt, daß man, wenn man erst einmal die richtigen Fragen stellt, auf dem besten Weg ist, eine Antwort zu finden.
Ähnlich wurde das Konzept der \hyperref[c1iii7b]{parallelen Sets}\index{parallelen Sets} mehr während des Schreibens entwickelt als während meiner Forschungen (Bücher lesen, mit Lehrern sprechen und das Internet benutzen) und persönlicher Experimente am Klavier.
Das Konzept der parallelen Sets wurde jedesmal benötigt, wenn bestimmte Übungsverfahren zu Schwierigkeiten führten.
Deshalb wurde es notwendig, dieses Konzept präzise zu definieren, damit man es wiederholt bei so vielen Gelegenheiten benutzen kann.

Es ist wichtig, mit allen anderen Wissenschaftlern, die ähnliche Arbeiten durchführen, zu kommunizieren und jegliche neuen Resultate der Forschung offen zu diskutieren.
In dieser Hinsicht war die Klavierwelt beklagenswert unwissenschaftlich.
Die meisten Bücher über das Klavierspielen haben nicht einmal ein Quellenverzeichnis (einschließlich der ersten Ausgabe meines Buchs, weil es innerhalb einer begrenzten Zeit geschrieben wurde -- diese Unzulänglichkeit wurde in dieser zweiten Ausgabe korrigiert), und sie bauen selten auf den bisherigen Arbeiten von anderen auf.
Lehrer an den bedeutenden Musikinstitutionen kommen der Aufgabe zu kommunizieren besser nach als private Lehrer, weil sie an einer Institution versammelt sind und zwangsläufig in Kontakt kommen.
Als Folge davon ist die Klavierpädagogik an einer solchen Institution der der meisten privaten Lehrer überlegen.
Zu viele Klavierlehrer sind in bezug auf das Annehmen oder Erforschen verbesserter Lehrmethoden inflexibel und stehen oftmals allem kritisch gegenüber, das von \textit{ihren} Methoden abweicht.
Das ist eine sehr unwissenschaftliche Situation.

Beispiele der offenen Kommunikation in meinem Buch sind das miteinander Verflechten der Konzepte von: den \hyperref[c1ii10]{Armgewichtsmethoden}\index{Armgewichtsmethoden} und der \hyperref[c1ii14]{Entspannung}\index{Entspannung} (Ansatz nach der Art von Taubman), Ideen aus \hyperref[Whiteside]{Whitesides Buch} (Kritik an den Übungen der Art von \hyperref[c1iii7h]{Hanon}\index{Hanon} und der Methode des Daumenuntersatzes), Einschluß der verschiedenen von \hyperref[Sandor]{Sandor} usw. beschriebenen Handbewegungen.
Da das Internet die absolute Form der offenen Kommunikation ist, ist das Aufkommen des Internets eventuell das wichtigste Ereignis, das am Ende dazu führen wird, daß die Klavierpädagogik wissenschaftlicher durchgeführt wird.
Dafür gibt es kein besseres Beispiel als dieses Buch.

Ein Mangel an Kommunikation ist offensichtlich die Hauptursache, warum so viele Klavierlehrer immer noch die intuitive Methode lehren, obwohl die meisten der in diesem Buch beschriebenen Methoden während der letzten zweihundert Jahre von dem einen oder anderen Lehrer gelehrt wurden.
Wenn der wissenschaftliche Ansatz der völlig offenen Kommunikation und der richtigen Dokumentation von der Klavierlehrergemeinde früher angenommen worden wäre, dann wäre die jetzige Situation sicher eine ganz andere und eine große Zahl Klavierschüler würde mit Raten lernen, die im Vergleich zu den heutigen Standards unglaublich erscheinen.

Beim Schreiben der ersten Ausgabe meines Buchs wurde mir die Wichtigkeit der richtigen Dokumentation und des Ordnens der Ideen durch die Tatsache demonstriert, daß ich, obwohl ich die meisten Ideen in meinem Buch bereits ungefähr 10 Jahre kannte, nicht in vollem Umfang von ihnen profitieren konnte, bis ich dieses Buch fertiggestellt hatte.
Mit anderen Worten: Nachdem ich das Buch fertiggestellt hatte, las ich es erneut und probierte es systematisch aus.
Dann erst erkannte ich, wie effektiv die Methode war!
Obwohl ich die meisten Bestandteile der Methode kannte, gab es offenbar einige Lücken, die erst gefüllt wurden, als ich damit konfrontiert wurde, alle Ideen in eine nützliche und organisierte Struktur zu bringen.
Es ist so, als ob ich alle Einzelteile eines Autos hätte, sie aber solange nutzlos wären, bis ein Mechaniker sie zusammenbaut und das Auto einstellt.

So verstand ich z.B. nicht ganz, warum die Methode so schnell war (1000mal schneller als die intuitive Methode), bevor ich nicht die Berechnung der Lernrate durchgeführt hatte \hyperref[c1iv5]{(s. Kapitel 1, Abschnitt IV.5)}.
Ich führte die Berechnungen zunächst aus Neugierde aus, weil ich hoffte, ein Kapitel über die Lerntheorie zu schreiben.
Tatsächlich dauert es fast ein Jahr, bis ich mich selbst überzeugen konnte, daß die Berechnung ungefähr richtig war -- eine Lernrate von 1000mal schneller schien zunächst ein lachhaft absurdes Ergebnis zu sein, bis ich feststellte, daß Schüler, die die intuitive Methode benutzen, oftmals während ihres ganzen Lebens nicht über die Mittelstufe hinauskommen, während andere in weniger als zehn Jahren zu Konzertpianisten werden können.
Die meisten Menschen neigten dazu, solche Unterschiede der Lernrate dem Talent zuzuschreiben, was nicht zu meinen Beobachtungen paßte.
Ein Nebenprodukt dieser Berechnung war ein besseres Verständnis dafür, \textit{warum} die Methode schneller war, weil man keine Gleichung schreiben kann, ohne zu wissen, welche physikalischen Prozesse beteiligt sind.
Als die mathematischen Formeln mir verrieten, welche Teile die Lernrate am meisten beschleunigten, konnte ich effektivere Übungsmethoden entwickeln.

Ein erstklassiges Beispiel einer neuen Entdeckung, die aus dem Schreiben dieses Buchs resultierte, ist das Konzept der \hyperref[c1ii11]{parallelen Sets}\index{parallelen Sets}.
Ohne dieses Konzept fand ich es unmöglich, alle Ideen auf eine stimmige Weise zusammenzustellen.
Als das Konzept der parallelen Sets eingeführt war, führte es natürlich zu den \hyperref[c1iii7b]{Übungen für parallele Sets}\index{Übungen für parallele Sets}.
Nichts davon wäre geschehen, wenn ich das Buch nicht geschrieben hätte, obwohl ich Übungen für parallele Sets die ganze Zeit benutzt hatte, ohne es bewußt wahrzunehmen.
Das kommt daher, daß der \hyperref[c1ii9]{Akkord-Anschlag}\index{Akkord-Anschlag} eine primitive Form der Übungen für parallele Sets ist; sogar \hyperref[Whiteside]{Whiteside} beschreibt Methoden für das Üben des \hyperref[c1iii3]{Trillers}\index{Trillers}, die im Grunde Übungen für parallele Sets sind.


\subsection{Konsistenzprüfungen}
\label{c3_3e}

Viele wissenschaftliche Entdeckungen werden als Resultat von Konsistenzprüfungen gemacht.
Diese Prüfungen funktionieren folgendermaßen.
Nehmen Sie an, Sie würden 10 Fakten über Ihr Experiment kennen, und Sie entdecken ein elftes.
Sie haben nun die Möglichkeit, dieses neue Ergebnis gegen alle alten Resultate zu prüfen, und oftmals führt diese Prüfung zu einer weiteren Entdeckung.
Eine einzige Entdeckung kann ohne jegliche weitere Experimente potentiell zu 10 weiteren Ergebnissen führen.
Die neuen Methoden dieses Buchs brachten z.B. ein viel schnelleres Lernen hervor, was dann darauf schließen ließ, daß die intuitive Methode Übungsverfahren beinhalten muß, die in Wahrheit das Lernen behindern.
Mit diesem Wissen wurde es eine einfache Sache, Gesichtspunkte der intuitiven Methode zu finden, die den Fortschritt verlangsamen.
Diese Aufdeckung der Schwächen der intuitiven Methode wären fast unmöglich gewesen, wenn man nur die intuitive Methode gekannt hätte.
Das ist eine Konsistenzprüfung, denn wenn beide Methoden korrekt wären, müßten Sie gleich effektiv sein.
Solch ein geistiger Prozeß, automatisch von allem auf das man trifft die Konsistenz zu prüfen, mag vielen Menschen nicht selbstverständlich erscheinen.
Als Wissenschaftler hatte ich das jedoch während meiner Laufbahn aus schierer Notwendigkeit bewußt getan.

Konsistenzprüfungen sind der ökonomischste und schnellste Weg, Fehler zu finden und neue Entdeckungen zu machen, weil man neue Ergebnisse erhält, ohne weitere Experimente durchzuführen.
Es kostet wenig extra, außer Ihrer Zeit.
Sie können nun sehen, warum der Prozeß des Dokumentierens so produktiv sein kann.
Jedesmal, wenn ein neues Konzept eingeführt wird, kann es gegen alle anderen bekannten Konzepte des Klavierübens geprüft werden, um potentiell zu neuen Ergebnissen zu führen.
Die Methode ist wegen der großen Zahl der Fakten, die bereits bekannt sind, mächtig.
Lassen Sie uns annehmen, daß man diese bekannten Wahrheiten zählen könnte und es 1000 wären.
Dann bedeutet eine neue Entdeckung, daß man nun 1000 weitere Möglichkeiten hat, um zu prüfen, ob sich neue Entdeckungen daraus ergeben!

Konsistenzprüfungen sind für das Eliminieren von Fehlern am wichtigsten und wurden benutzt, um Fehler in diesem Buch zu minimieren.
Langsames Üben ist z.B. sowohl nützlich als auch schädlich.
Diese Inkonsistenz muß irgendwie beseitigt werden; das geschieht durch sorgfältiges Definieren derjenigen Bedingungen, die langsames Üben erfordern (Auswendiglernen, HT-Üben), und der Bedingungen, unter denen langsames Üben abträglich ist (intuitive Methode ohne HS-Üben).
Klar ist jedes Pauschalurteil, das sagt \enquote{Langsames Üben ist gut, weil immer schnell zu spielen zu Problemen führt.}, nicht mit allen bekannten Fakten konsistent.
Wann immer ein Autor eine falsche Behauptung aufstellt, ist eine Konsistenzprüfung oft der leichteste Weg, diesen Fehler herauszufinden.


\subsection{Grundlegende Theorie}
\label{c3_3f}

Wissenschaftliche Resultate müssen immer auf einer Theorie oder einem Prinzip basieren, das durch andere verifiziert werden kann.
Sehr wenige Konzepte stehen allein, unabhängig von allem anderen.
Mit anderen Worten: Für alles, von dem jemand behauptet, daß es funktioniert, muß es eine gute Erklärung geben, warum es funktioniert; anderenfalls ist es suspekt.
Erklärungen wie \enquote{Es hat bei mir funktioniert.} oder \enquote{Ich habe das 30 Jahre lang so unterrichtet.} oder sogar \enquote{Das ist, wie Liszt es getan hat.} sind einfach nicht gut genug.
Wenn ein Lehrer ein Verfahren 30 Jahre unterrichtet hat, sollte er genügend Zeit gehabt haben, herauszufinden, warum es funktioniert.
Die \textit{Erklärungen} sind oft wichtiger als die Verfahren, die sie erklären.
HS-Üben funktioniert z.B., weil es eine schwierige Aufgabe vereinfacht.
Wenn dieses Prinzip der Vereinfachung eingeführt ist, kann man nach weiteren Dingen dieser Art Ausschau halten, wie z.B. schwierige Passagen zu kürzen oder das \hyperref[c1iii8]{Konturieren}\index{Konturieren}.
Ein Beispiel für eine grundlegende Erklärung ist der Zusammenhang zwischen der Schwerkraft und der Armgewichtsmethode und ihrer Beziehung zum Tastengewicht.
Im Beispiel der schweren Hand des Sumoringers und der leichten Hand des Kindes \hyperref[c1ii10]{(\autoref{c1ii10})} müssen beide bei einem korrekten Anschlag mit Freiem Fall einen Ton gleicher Intensität erzeugen, wenn ihre Hände aus der gleichen Höhe auf das Klavier herunterfallen.
Das ist offensichtlich  für den Sumoringer wegen seiner Neigung, sich auf das Klavier zu stützen, um seine schwerere Hand anzuhalten, schwieriger.
Deshalb ist der korrekte Freie Fall für den Sumoringer schwieriger auszuführen.
Diese Feinheiten auf theoretischer Grundlage zu verstehen führt zur Ausführung eines wirklich korrekten Freien Falls.
Mit anderen Worten: Bei einem korrekten Freien Fall darf man sich nicht auf dem Klavier abstützen, um die Hand anzuhalten, bis der Anschlag vollständig ist.
Man braucht ein sehr geschmeidiges Handgelenk, um diese Meisterleistung zu vollbringen.

Selbstverständlich gibt es immer ein paar Konzepte, die sich der Erklärung widersetzen, und es ist extrem wichtig, sie klar als \enquote{gültige Prinzipien ohne Erklärungen} zu klassifizieren.
Wie können wir in diesen Fällen wissen, daß sie gültig sind?
Sie können nur als gültig angesehen werden, nachdem man eine unbestreitbare Aufzeichnung der experimentellen Überprüfung erstellt hat.
Es ist wichtig, diese klar zu kennzeichnen, weil Verfahren ohne Erklärungen schwieriger anzuwenden sind und diese Verfahren sich während wir dazulernen und sie besser verstehen ändern.
Das beste an den Methoden, für die es gute Erklärungen gibt, ist, daß man uns nicht jedes Detail, wie man das Verfahren durchführt, sagen muß -- wir können die Details oft anhand unseres eigenen Verständnisses der Methode selbst einfügen.

Leider ist die Geschichte der Klavierpädagogik voller Verfahren für das Erwerben der Technik, die keine theoretische Grundlage haben, die aber trotzdem eine breite Akzeptanz erfahren haben.
Die \hyperref[c1iii7h]{Hanon-Übungen}\index{Hanon-Übungen} sind das beste Beispiel dafür.
Die meisten Anweisungen, wie man etwas tun soll, die ohne eine Erklärung dafür gegeben werden, warum sie funktionieren, haben in einem wissenschaftlichen Ansatz einen geringen Wert.
Das nicht nur wegen der hohen Wahrscheinlichkeit, daß solche Verfahren falsch sind, sondern auch, weil es die Erklärung ist, die dabei hilft, das Verfahren korrekt anzuwenden.
Weil es keine theoretische Grundlage für die Hanon-Übungen gibt, wenn er uns ermahnt, \enquote{die Finger stark anzuheben} und \enquote{eine Stunde täglich zu üben}, können wir in keinster Weise wissen, ob diese Verfahren tatsächlich hilfreich sind.
In jedem Verfahren des täglichen Lebens ist es für jeden fast unmöglich, alle notwendigen Schritte eines Verfahrens für alle denkbaren Fälle zu beschreiben.
Es ist ein Verständnis dafür, warum es funktioniert, das jedem gestattet, das Verfahren abzuändern, damit es den besonderen Bedürfnissen des einzelnen und der sich ändernden Umstände gerecht wird.

So empfehlen z.B. Lehrer, die die intuitive Methode benutzen, daß man das Spielen langsam und genau anfängt und die Geschwindigkeit schrittweise steigert.
Andere Lehrer mögen das langsame Spielen so weit wie möglich zu unterbinden suchen, weil es eine solche Zeitverschwendung ist.
Keines dieser Extreme ist das Beste.
Das \hyperref[c1ii16]{langsame Spielen des intuitiven Ansatzes}\index{langsame Spielen des intuitiven Ansatzes} ist unerwünscht, weil man eventuell Bewegungen verfestigt, die das schnellere Spielen stören.
Auf der anderen Seite ist langsames Spielen, wenn man erst einmal mit der endgültigen Geschwindigkeit spielen kann, sehr nützlich für das \hyperref[c1iii6h]{Auswendiglernen}\index{Auswendiglernen} und für das Üben der \hyperref[c1ii14]{Entspannung}\index{Entspannung} und Genauigkeit.
Deshalb ist die einzige Möglichkeit, die richtige Übungsgeschwindigkeit auszuwählen, im Detail zu verstehen, warum man diese Geschwindigkeit nehmen muß.
In diesem Zeitalter der Informationstechnologie und des Internets sollte es fast keinen Platz mehr für blindes Vertrauen geben.

Das heißt nicht, daß es Regeln ohne Erklärungen nicht gibt.
Schließlich gibt es immer noch viele Dinge in dieser Welt, die wir nicht verstehen.
Beim Klavierspielen ist die Regel, \hyperref[c1ii17]{vor dem Aufhören langsam zu spielen}\index{vor dem Aufhören langsam zu spielen}, ein Beispiel dafür.
Es muß eine gute Erklärung geben, aber ich habe noch keine gehört, die ich für zufriedenstellend halte.
In der Wissenschaft sind Paulis Ausschließungsprinzip\footnote{oder kurz Pauli-Prinzip} (zwei Fermionen können nicht die gleichen Quantenzahlen haben) und die Heisenbergsche Unschärferelation Beispiele von Regeln, die nicht von einem tieferen Prinzip abgeleitet werden können.
Deshalb ist es genauso wichtig, etwas zu verstehen, wie zu wissen, was wir nicht verstehen.
Die sachkundigsten Physikprofessoren sind diejenigen, die alle Dinge benennen können, die wir immer noch nicht verstehen.


\subsection{Dogma und Lehre}
\label{c3_3g}

Wir wissen alle, daß man nicht jede Regel brechen kann, von der man glaubt, sie brechen zu können, und immer noch musikalisch spielen kann, es sei denn, man hat Initialen wie LvB.
Die dogmatischen Lehrmethoden, die in der Klavierpädagogik so weit verbreitet sind, haben sich in diesem restriktiven Umfeld der Schwierigkeit, Schüler zum Erzeugen von Musik anzuleiten, entwickelt.
Um es zynisch zu sagen: Der dogmatische Ansatz ist ein angenehmer Weg, die Unwissenheit des Lehrers dadurch zu verbergen, daß alles unter den Dogma-Teppich gekehrt wird.
Alle großen Vorträge, die ich von berühmten Künstlern gehört habe, sind voller exzellenter wissenschaftlicher Erklärungen, warum man auf eine bestimmte Art vorspielen oder nicht vorspielen sollte.
Es sind jedoch nicht alle großen Künstler auch gute Lehrer oder in der Lage, zu erklären was sie tun.
Die Lektion daraus ist für die Schüler, daß sie im allgemeinen nichts akzeptieren sollten, das sie nicht verstehen können; das wird dazu führen, daß die Ausbildungsstufe, die sie erreichen, ansteigt.
Ich bin überzeugt, daß sogar die Interpretation der Musik mit der Zeit ebenfalls wissenschaftlicher wird, genauso wie die Alchemie sich schließlich zur Chemie entwickelte.

Leider ist ein dogmatischen Herangehen an das Unterrichten nicht immer ein Zeichen für einen schlechteren Lehrer.
In Wahrheit scheint es, vermutlich aus historischen Gründen, eher das Gegenteil zu sein.
Zum Glück sind viele gute junge Lehrer, und besonders diejenigen an großen Institutionen, weniger dogmatisch -- sie können erklären.
Wenn die Lehrer besser ausgebildet sind, sollten sie in der Lage sein, Dogma vermehrt durch ein tieferes Verständnis für die zugrunde liegenden Prinzipien zu ersetzen.
Das sollte die Effizienz und die Leichtigkeit des Lernens für den Schüler deutlich verbessern.

Den meisten Menschen ist bewußt, daß Wissenschaftler ihr ganzes Leben lang lernen müssen, nicht nur wenn sie an der Universität für ihre Abschlüsse arbeiten.
Den meisten ist jedoch nicht bewußt, in welchem Ausmaß Wissenschaftler ihre Zeit der Ausbildung widmen, nicht nur um zu lernen, sondern auch um alle anderen zu unterrichten, insbesondere andere Wissenschaftler.
Tatsächlich muß, um das Maß der Entdeckungen zu maximieren, die Ausbildung zu einer ganztägigen, alles verschlingenden Passion werden.
Wissenschaftler entwickeln sich deshalb oftmals mehr zu Lehrern als z.B. Klavier- oder Schullehrer, sowohl wegen des breiteren Bereichs an \enquote{Schülern}, auf die sie treffen, als auch wegen der Breite der Themen, die sie abdecken müssen.
Es ist wirklich erstaunlich, wie viel man wissen muß, um nur eine kleine neue Entdeckung zu machen.
Deshalb muß ein notwendiger Teil der wissenschaftlichen Dokumentation die höchsten Fertigkeiten des Unterrichtens einschließen.
Ein wissenschaftlicher Forschungsbericht ist nicht so sehr eine Dokumentation dessen, was getan wurde, als vielmehr ein Lehrbuch darüber, wie man das Experiment reproduziert und die zugrunde liegenden Prinzipien versteht.
Deshalb ist die wissenschaftliche Methode für das Unterrichten ideal.
Und es ist eine Lehrmethode, die zur dogmatischen Methode diametral verschieden ist.


\subsection{Schlußfolgerungen}
\label{c3_3h}

Der wissenschaftliche Ansatz ist mehr als nur eine präzise Art, die Ergebnisse eines Experiments zu dokumentieren.
Er ist ein Verfahren, das zur Beseitigung von Fehlern und zur Erzeugung von Entdeckungen entwickelt wurde.
Vor allem ist er im Grunde ein Mittel zur Befähigung des Menschen.
Wenn der wissenschaftliche Ansatz früher übernommen worden wäre, dann wäre die Klavierpädagogik heute aller Wahrscheinlichkeit nach völlig anders.
Das Internet wird sicherlich die Übernahme von wissenschaftlicheren Vorgehensweisen in das Lernen des Klavierspielens beschleunigen.
 

\section{Theorie des Lernens}
\label{c3_4}

\footnote{Abschnitt 4 ist im Original z.Zt. (30.5.2005) noch \enquote{preliminary draft} also ein \enquote{Rohentwurf}.}

Ist es nicht seltsam, daß wir, wenn wir auf die Universität gehen, finden, daß \enquote{101 Lernen} kein erforderlicher Kurs ist (wenn er überhaupt existiert!)?
Von Colleges und Universitäten erwartet man, daß sie Lernzentren sind.
Psychologische Abteilungen haben oft einführende Kurse über Studiengewohnheiten usw., aber man sollte meinen, daß die Wissenschaft des Lernens der erste Punkt auf der Tagesordnung an jedem Lernzentrum wäre.
Beim Schreiben dieses Buchs fand ich, daß es notwendig ist, über den Lernprozeß nachzudenken und eine -- wie auch immer näherungsweise -- Gleichung für die Lernrate abzuleiten.
 


