% File: c1iii14

\section{Vorbereitung auf Auftritte und Konzerte}
\label{c1iii14} 

\subsection{Nutzen und Risiken von Auftritten und Konzerten}
\label{c1iii14a}

Der Nutzen und die Risiken der Auftritte bestimmen unser tägliches Programm für das Lernen des Klavierspielens.
Für den Amateurklavierspieler ist der Nutzen von Auftritten, auch von zufälligen, unermeßlich.
Der wichtigste Nutzen ist, daß die Technik nie richtig erworben wird, bis man sie in einem Auftritt zeigen kann.
Für junge Schüler ist der Nutzen sogar noch fundamentaler.
Sie lernen, was es bedeutet, eine richtige Aufgabe zu beenden, und sie lernen, was \enquote{Musik machen} bedeutet.
Die meisten Kinder (die keinen Musikunterricht erhalten) lernen diese Fertigkeiten nicht, bis sie aufs College gehen; Klavierschüler müssen sie, unabhängig von ihrem Alter, bei ihrem \textit{ersten Konzert} lernen.
Schüler sind nie so selbstmotiviert wie bei der Vorbereitung auf ein Konzert.
Lehrer, die Konzerte veranstaltet haben, kennen den enormen Nutzen.
Ihre Schüler werden konzentriert, selbstmotiviert und ergebnisorientiert; sie hören dem Lehrer aufmerksam zu und versuchen wirklich, die Bedeutung der Anweisungen des Lehrers zu verstehen.
Die Schüler meinen es todernst damit, alle Fehler zu eliminieren und alles korrekt zu lernen - es ist ein privates Unternehmertum in Vollendung, weil es \textit{ihr} Konzert ist.
Lehrer ohne Konzerte haben am Ende oft Schüler, die vielleicht ein paarmal unmittelbar vor dem Unterrichtstag üben.

Da die Psychologie und Soziologie des Klavierspielens nicht gut entwickelt sind, gibt es Risiken, die wir ernsthaft bedenken müssen.
Das wichtigste ist die \hyperref[c1iii15]{Nervosität} und ihre Auswirkungen auf den Geist, besonders bei Kindern.
Nervosität kann Konzerte zu einer furchtbaren Erfahrung machen, die eine sorgfältige Beachtung erfordert, um nicht nur unglückliche Erfahrungen, sondern auch bleibende psychologische Schäden zu vermeiden.
Die Nervosität zu reduzieren wird zumindest den Streß und die Furcht abschwächen.
Konzerte zu einer angenehmen Erfahrung werden zu lassen und Spannung und Streß zu reduzieren, wird nicht genug Beachtung geschenkt, insbesondere bei Klavierwettbewerben.
Dieses ganze Thema wird im Abschnitt über Nervosität vollständiger behandelt.
Der Punkt ist hier, daß jede Abhandlung des Auftretens eine Diskussion des Lampenfiebers einschließen muß.
Sogar große Künstler haben aus dem einen oder anderen Grund für längere Zeit aufgehört aufzutreten, und einige der Gründe standen zweifellos im Zusammenhang mit dem Streß.
Obwohl gute Klavierlehrer stets Konzerte ihrer Schüler abhalten und sie an Wettbewerben teilnehmen lassen, sind sie oft keine guten Soziologen und Psychologen, konzentrieren sich nur auf das Klavierspielen und ignorieren die Nervosität.
Es ist für jede Person, die Kinder durch Konzerte und Wettbewerbe begleitet, wichtig, die Grundlagen dessen zu lernen, was Nervosität verursacht, wie man mit ihr umgeht und was ihre psychologischen Konsequenzen sind.
Wenn die Lehrer in dieser Hinsicht versagen, ist es die Aufgabe der Eltern, das soziale und psychologische Wohl ihrer Kinder im Auge zu behalten; darum ist der folgende Abschnitt (15) über Nervosität ein notwendiger Begleiter dieses Abschnitts.

Es gibt zahlreiche weitere psychologische und soziologische Gesichtspunkte bei Konzerten und Wettbewerben.
Die Bewertungssysteme in Musikwettbewerben sind bekanntermaßen unfair, und das Bewerten ist eine schwierige und undankbare Aufgabe.
Deshalb müssen Schüler, die an einem Wettbewerb teilnehmen sollen, über diese Unzulänglichkeiten des Systems informiert werden, damit die wahrgenommene Ungerechtigkeit und die Enttäuschung nicht zu psychologischen Problemen führen.
Es ist für Schüler schwierig aber möglich, zu verstehen, daß das Dabeisein das wichtigste Element von Wettbewerben ist, nicht daß man gewinnt.
Es wird zuviel Wert auf die technische Schwierigkeit gelegt und nicht genug auf die Musikalität.
Das System ermutigt nicht die Kommunikation zwischen den Lehrern, um die Lehrmethoden zu verbessern.
Es ist kein Wunder, daß es eine Denkrichtung gibt, die das Abschaffen der Wettbewerbe befürwortet.
Es steht außer Frage, daß Konzerte und Wettbewerbe notwendig sind; aber die derzeitige Situation kann sicher verbessert werden.
Mehr dazu in \hyperref[c1iii15]{Abschnitt (15)}.


\subsection{Grundlagen fehlerfreien Vorspielens}
\label{c1iii14b}

Die grundlegenden Voraussetzungen für ein fehlerfreies Vorspielen sind: technische Vorbereitung, musikalische Interpretation, \hyperref[c1ii12mental]{mentales Spielen} und ein gutes Verfahren für die Vorbereitung auf den Auftritt.
Wenn alle diese Elemente zusammenkommen, ist ein perfekter Auftritt im Grunde garantiert.

Natürlich gibt es viele Entschuldigungen dafür, daß man nicht auftreten kann.
Diese Entschuldigungen zu kennen, ist eine der Voraussetzungen dafür, zu lernen wie man auftritt.
Die vielleicht am häufigsten vorgebrachte Entschuldigung ist, daß man immer neue Stücke lernt, so daß ungenügend Zeit vorhanden ist, um ein Stück wirklich abzuschließen oder die fertigen Stücke in spielbarem Zustand zu halten.
Wir haben gesehen, daß ein neues Stück zu lernen die beste Art ist, die alten Stücke zu verschlechtern.
Für diejenigen, die niemals aufgetreten sind, ist der zweite wichtige Grund, daß sie wahrscheinlich nie irgendein Stück wirklich zu Ende gebracht haben.
In jedem \enquote{interessanten} Stück, das es wert ist aufgeführt zu werden, gibt es immer diesen einen schwierigen Abschnitt, den man nicht richtig bewältigen kann.
Eine weitere Entschuldigung ist, daß Stücke, die leicht für Sie sind, irgendwie immer uninteressant sind.
Beachten Sie, daß die Lernmethoden dieses Buchs so konzipiert sind, daß sie jeder dieser Entschuldigungen entgegenwirken und zwar hauptsächlich durch die Beschleunigung des Lernprozesses und durch die Förderung des \hyperref[c1iii6]{Auswendiglernens},
so daß in dem Moment, in dem Sie ein Stück gut in Gedanken spielen können, keine dieser Entschuldigungen mehr berechtigt ist.
Somit befinden sich alle für ein fehlerfreies Vorspielen notwendigen Elemente in diesem Buch.
Wir besprechen nun ein paar weitere Gesichtspunkte des Lernens, wie man auftritt.


\subsection{Für Auftritte üben}
\label{c1iii14c}

Direkt vor dem Auftritt benutzen die meisten Pianisten zur Vorbereitung auf den Auftritt eine Übungsgeschwindigkeit, die etwas niedriger als die Aufführungsgeschwindigkeit ist.
Diese Geschwindigkeit gestattet das exakte Üben, ohne daß man unerwartete schlechte Gewohnheiten annimmt, und erzeugt ein klares geistiges Bild der Musik.
Sie konditioniert auch die Hand dafür, bei der schnelleren Aufführungsgeschwindigkeit mit Kontrolle zu spielen, und verbessert die Technik.
Diese langsamere Geschwindigkeit ist nicht notwendigerweise einfacher als die Aufführungsgeschwindigkeit.
Der Grund für die zwei Geschwindigkeiten ist, daß es während des Vorspielens leichter ist, den Ausdruck herauszubringen, wenn man etwas schneller spielt als man beim letzten Mal gespielt hat.
Wenn Sie dieselbe Komposition zweimal hintereinander spielen (oder am selben Tag), kommt die Musik beim zweiten Mal flach heraus, außer wenn sie schneller als beim ersten Mal gespielt wird, weil das langsamere Spielen weniger aufregend klingt und dieses Gefühl - zusätzlich zum \hyperref[fpd]{FPD (Schnellspiel-Abbau)} - eine negative Rückkopplung erzeugt.
Nach solchem wiederholten Spielen (eigentlich nach jedem Vorspielen), sollten Sie so schnell wie möglich langsam spielen, um den FPD zu verhindern und die Musik in Ihrem Kopf \enquote{zurückzusetzen}.
Bei Computern gibt es einen ähnlichen Vorgang: Nach längerem Gebrauch sind die Daten auf der Festplatte zunehmend fragmentiert und müssen defragmentiert werden, damit sie wieder zusammenhängend gespeichert sind.

Wer keine Erfahrung im Vorspielen hat, spielt wegen der \hyperref[c1iii15]{Nervosität} während des Konzerts oft schneller als er aufgrund seiner Fertigkeitsstufe kann. 
Solche unpassenden Geschwindigkeiten kann man mit Hilfe von Videoaufnahmen leicht erkennen.
Für den Fall, daß Ihnen dieser Fehler während des Auftritts unterläuft, ist es deshalb während des routinemäßigen Übens (nicht unmittelbar vor einem Auftritt) wichtig, mit Geschwindigkeiten zu üben, die schneller als die Aufführungsgeschwindigkeit sind.
Offensichtlich muß die Aufführungsgeschwindigkeit niedriger als Ihre höchste Geschwindigkeit sein.
Erinnern Sie sich daran, daß das Publikum dieses Stück nicht wie Sie während des Übens unzählige Male gehört hat und somit nicht genauso mit jedem Detail vertraut ist; es ist wahrscheinlich, daß es für das Publikum viel schneller klingt als für Sie, und Ihre \enquote{endgültige Geschwindigkeit} kann für das Publikum zu schnell sein.
Ein Stück, das mit sorgfältiger Beachtung jeder Note gespielt wird, kann schneller klingen als ein Stück, das mit höherer Geschwindigkeit gespielt wird aber mit nicht deutlich zu erkennenden Noten.
Sie müssen dem Publikum jede Note \enquote{mundgerecht servieren}, da es diese sonst nicht hört.

Üben Sie, über Fehler hinwegzukommen.
Besuchen Sie Konzerte von Schülern und beobachten Sie, wie diese auf ihre Fehler reagieren; Sie werden leicht die richtigen und die falschen Reaktionen erkennen.
Ein Schüler, der nach einem Fehler seine Frustration zeigt oder seinen Kopf schüttelt, macht aus einem Fehler drei: den ursprünglichen Fehler, eine falsche Reaktion, und er vermittelt dem Publikum, daß ein Fehler begangen wurde.
Mehr dazu unten in \hyperref[c1iii14g]{Abschnitt g}.


\hypertarget{c1iii14musikalisch}{}
\subsection{Musikalisch üben}
\label{c1iii14d}

Was bedeutet es, musikalisch zu spielen?
Diese Frage kann nur durch eine Unzahl von Mikroregeln definitiv beantwortet werden, die auf bestimmte Abschnitte in bestimmten Kompositionen anwendbar sind; hierbei kann Ihnen ein Lehrer zeigen, was Sie tun müssen.
Wenn Sie die gesamte musikalische Notation inklusive der Zeichen in Ihre Musik einbeziehen, haben Sie eine solide Grundlage.
Es gibt ein paar allgemeine Regeln für das musikalische Spielen:

\begin{enumerate}[label={\roman*.}] 
 \item Verbinden Sie jeden Takt sorgfältig mit dem vorangegangenen Takt (oder dem Schlag oder der Phrase).
Diese Takte bzw. Schläge stehen nicht alleine da; einer fließt logisch in den nächsten, und sie unterstützen sich alle gegenseitig.
Sie sind ebenso rhythmisch wie konzeptionell verbunden.
Man könnte meinen, daß dieser Punkt in trivialer Weise offensichtlich ist; aber wenn Sie dies bewußt tun, könnten Sie von der entscheidenden Verbesserungen Ihrer Musik überrascht werden.

 \item Es muß immer eine Unterhaltung zwischen der RH und LH vorhanden sein.
Sie spielen nicht unabhängig voneinander.
Und sie werden nicht automatisch miteinander reden, nur weil Sie sie zeitlich perfekt aufeinander abgestimmt haben.
Man muß eine Unterhaltung der beiden Hände oder Stimmen bewußt erzeugen.

 \item \enquote{cresc.} bedeutet, daß das meiste der Passage leise gespielt werden sollte; nur die letzten paar Noten sind laut, d.h. es ist wichtig leise anzufangen.
Ähnlich ist es bei den anderen Markierungen dieser Art (rit., accel., dim. usw.); stellen Sie sicher, daß Sie Platz für das Stattfinden der Aktion reservieren, und fangen Sie sie nicht sofort an, warten Sie bis zum letzten Moment.
Diese \enquote{Ausdrucksmittel} sollten ein geistiges Bild erzeugen; wenn Sie z.B. ein Crescendo schrittweise steigern, ist es so, als ob man eine Treppe hinaufsteigt, während es, wenn Sie bis zum letzten Moment warten und exponentiell steigern, so ist, als ob man in die Luft geworfen wird, was einen größeren Effekt erzielt.

 \item Streben Sie mehr nach Genauigkeit als nach einem ausdrucksstarken Rubato; \textit{rubato} ist oft zu einfach, inkorrekt und nicht im Einklang mit dem Publikum.
Hier ist der richtige Zeitpunkt, das Timing und den Rhythmus mit dem Metronom zu prüfen!

 \item Wenn Sie im Zweifel sind, beginnen und beenden Sie jede Phrase leise, mit den lauteren Noten nahe der Mitte.
Es ist üblicherweise falsch, die lauten Noten am Anfang zu haben; selbstverständlich kann man aber auch Musik machen, indem man diese Regel bricht.
\end{enumerate}

Musikalität hat keine Grenze - Sie können sich unabhängig davon verbessern, wo Sie sich auf der Skala der Musikalität befinden.
Der angsteinflößende Teil davon ist die Kehrseite.
Wenn man nicht achtgibt, kann man unmusikalische Spielgewohnheiten entwickeln, die fortwährend die Musikalität zerstören können.
Darum ist es so wichtig, sich auf die Musikalität zu konzentrieren und nicht nur auf die Technik; es kann den Unterschied ausmachen, ob Sie auftreten können oder nicht.

Hören Sie (beim Üben) stets Ihrer eigenen Musik zu, und führen Sie Ihre Musik mit dem \hyperref[c1ii12mental]{mentalen Spielen} - das ist der einzige Weg, die Aufmerksamkeit des Publikums anzuziehen.
Werden Sie bei einem Fehler nicht bedrückt, weil das es erschweren würde, gut zu spielen.
Wenn Sie einen guten Start haben, wird das Publikum jedoch hineingezogen, die Musik trägt sich selbst, und der Auftritt wird leichter.
Somit wird das Spielen zu einer Rückkopplungsschleife zweier Vorgänge, die sich gegenseitig unterstützen müssen: die Musik mittels des mentalen Spielens führen und der vom Klavier ausgehenden Musik zuhören.

Viele Schüler hassen es, zu üben wenn andere dabei sind, die zuhören; manche sind sogar der Ansicht, daß intensives Klavierüben notwendigerweise unangenehm und eine Strafe für das Ohr sei.
Das sind Symptome verbreiteter falscher Vorstellungen, die aus ineffizienten Übungsmethoden resultieren, und Zeichen einer schwachen mentalen Ausdauer.
Mit den richtigen Übungsmethoden und dem musikalischen Spielen sollte an den Übungseinheiten nichts Unangenehmes sein.
\textbf{Das beste Kriterium dafür, ob Sie richtig üben, ist die Reaktion der anderen - wenn Ihr Üben gut für sie klingt oder sie zumindest nicht stört, dann machen Sie es richtig.}
Das musikalische Spielen verbessert die mentale Ausdauer.



