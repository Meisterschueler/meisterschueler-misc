% File: c1iii21

\subsection{Klavierspielen und die Psychologie}
\label{c1iii21}

Uns allen ist bewußt, daß die Psychologie nicht nur in der Musik, sondern auch beim Erlernen des Klavierspielens eine wichtige Rolle spielt.
Es gibt zahlreiche Möglichkeiten, einen Vorteil aus unserem Verständnis der Psychologie zu ziehen, und wir werden einige dieser Methoden in diesem Abschnitt besprechen.
Die wichtigere, sofort zu erledigende Aufgabe ist jedoch, die psychologischen Fallen aufzudecken, die zu scheinbar unüberwindlichen Hindernissen für das Lernen des Klavierspielens führen, wie z.B. \enquote{Mangel an Talent} oder \enquote{\hyperref[c1iii15]{Nervosität}} beim \hyperref[c1iii14]{Auftreten}.
Ein weiteres Beispiel ist das Phänomen der \hyperref[c1iii16e]{Unfähigkeit großer Künstler zu unterrichten}, wie oben in Abschnitt 16.e beschrieben.
Dieses Phänomen wurde durch die psychologische Herangehensweise der Künstler an das Unterrichten erklärt, die ihr Herangehen an die Musik widerspiegelt.
Da die Psychologie der Musik nur minimal verstanden wird, erzeugen die Komponisten die Musik in ihrem Geist quasi \enquote{aus dem Nichts} -- es gibt keine Formel für das Erzeugen von Musik.
Auf ähnliche Weise haben sie Ihre Technik dadurch erworben, daß sie sich das musikalische Ergebnis vorgestellt haben und  die Hände eine Möglichkeit finden ließen, dieses zu erreichen.
Es ist eine unheimliche Verkürzung des Wegs zu einem komplexen Ziel, wenn es funktioniert.
Für die meisten Schüler ist es jedoch ein höchst ineffizienter Weg, sich die Technik anzueignen, und wir wissen nun, daß es bessere Vorgehensweisen gibt.
Offensichtlich ist die Psychologie sowohl für das Lernen, das Üben und das Auftreten als auch für das Komponieren von Musik wichtig.

Die Psychologie wird hauptsächlich durch das Wissen kontrolliert, und es ist oft schwierig, zwischen Psychologie und Wissen zu unterscheiden.
In den meisten Fällen kontrolliert das Wissen, wie wir an ein Thema herangehen.
Aber die Psychologie bestimmt, wie wir dieses Wissen anwenden.
Es ist nun an der Zeit, einige besondere Punkte zu untersuchen.

Der vielleicht wichtigste Punkt ist, wie wir das Klavierspielenlernen sehen, oder unsere generelle Haltung dem Prozeß des Spielenlernens gegenüber.
Die Methoden dieses Buchs sind von den \enquote{intuitiven} Methoden diametral verschieden.
Wenn ein Schüler z.B. beim Lernen versagte, war es gemäß des alten Systems wegen eines Mangels an Talent, so daß das Versagen die Schuld des Schülers war.
Im System dieses Buchs ist das Versagen die Schuld des Lehrers, weil es die Aufgabe des Lehrers ist, alle für den Erfolg notwendigen Informationen zur Verfügung zu stellen.
Es gibt kein blindes Vertrauen mehr darin, daß jeden Tag eine Stunde lang \hyperref[c1iii7h]{Hanon} zu üben jemanden in einen Virtuosen verwandelt.
Es sollte tatsächlich nichts aufgrund bloßen Vertrauens angenommen werden, und es ist die Verpflichtung des Lehrers, jede Methode zu erklären, so daß der Schüler sie versteht.
Das erfordert, daß der Lehrer sich in einer breiten Vielzahl von Disziplinen auskennt, in der Kunst und in den Wissenschaften.
Wir sind an einem Punkt in der Geschichte angekommen, an dem Kunstlehrer die Wissenschaft nicht mehr ignorieren können.
Deshalb erfordert die Psychologie des Klavierspielenlernens sowohl für den Schüler als auch den Lehrer tiefgreifende Veränderungen der Grundhaltung.

Für die Schüler -- besonders für diejenigen, die nach dem alten System mit Regeln ausgebildet wurden -- reicht das Erleben des Übergangs vom alten zum neuen System von \enquote{sehr leicht} bis zur völligen Verwirrung.
Einige Schüler werden sofort die neue Befähigung und Freiheit genießen und -- innerhalb einer Woche -- den vollen Nutzen aus den Methoden ziehen.
Auf der anderen Seite wird es Schüler geben, die erkennen, daß die alten Regeln nicht mehr gelten und nach \enquote{neuen Regeln} Ausschau halten, die sie befolgen können.
Sie haben jede Menge Fragen: Wenn ich eine Hand \hyperref[c1iii2]{zirkuliere}, sind zehnmal genug oder muß ich es 10.000 mal tun?
Zirkuliere ich so schnell wie ich kann oder mit einer langsameren, genaueren Geschwindigkeit?
Ist HS-Üben auch dann notwendig, wenn ich bereits HT spielen kann?
Bei einfacher Musik kann HS-Üben schrecklich langweilig sein -- warum brauche ich es?
Solche Fragen enthüllen das Ausmaß, in dem der Schüler sich an die neue Denkweise angepaßt hat oder nicht.
Lassen Sie uns zur Verdeutlichung die letzte Frage näher analysieren.
Um solch eine Frage zu stellen, muß diese Person blind geübt haben, weil sie gelesen hat, daß es notwendig ist, HS zu üben.
Mit anderen Worten: Sie hat eine Regel blind befolgt.
Das ist nicht die Methode dieses Buchs.
Wir definieren ein Ziel und benutzen dann das HS-Üben, um dieses Ziel zu erreichen.
Dieses Ziel kann ein sichereres \hyperref[c1iii6]{Gedächtnis} sein, um Gedächtnisblockaden während der Aufführung zu vermeiden, oder die technische Entwicklung, so daß man, wenn man HT spielt, hören kann, daß das Spielen auf überlegenen technischen Fertigkeiten basiert.
Wenn diese Ziele erreicht werden, ist das Üben überhaupt nicht langweilig!

Für den Lehrer steht außer Frage, daß alles in der modernen Gesellschaft auf einer breiten Ausbildung basiert.
Es ist nicht notwendig, ein Wissenschaftler zu werden oder fortgeschrittene Konzepte der Psychologie zu studieren.
Erfolg in der realen Welt ist nicht an akademische Leistungen gebunden; die meisten erfolgreichen Unternehmer sind keine diplomierten Wirtschaftswissenschaftler.
Der vielleicht wichtigste Fortschritt der modernen Gesellschaft ist, daß all diese Konzepte, die als Spezialwissen fortgeschrittener Gebiete angesehen wurden, leichter verständlich werden; nicht weil sie sich geändert haben, sondern weil ein besseres Verständnis die Dinge immer vereinfacht und die Lehrmethoden immer besser werden.
Außerdem werden wir vertrauter mit ihnen, weil wir sie zunehmend in unserem täglichen Leben benötigen.
Es wird einfacher, auf die Informationen zuzugreifen.
Deshalb muß ein Lehrer nur neugierig sein und gewillt, mit anderen zu kommunizieren, und die Ergebnisse werden automatisch folgen.

Viele von uns brauchen ein psychologisches Mittel, um die unbegründete Angst vor der Unfähigkeit \hyperref[c1iii6]{auswendig zu lernen} zu überwinden.
In diesem Buch sprechen wir nicht darüber, nur \enquote{Für Elise} auswendig zu lernen.
Wir sprechen über ein Repertoire von mehr als 5 Stunden richtiger Musik, wobei Sie sich bei den meisten Stücken einfach ans Klavier setzen und sie sofort spielen können.
Um ein solches Repertoire zu \hyperref[c1iii6k]{pflegen}, ist es nur erforderlich, daß Sie jeden Tag auf dem Klavier spielen.
Einige Menschen haben keine Schwierigkeiten mit dem Auswendiglernen, aber die meisten haben die vorgefaßte Meinung, daß ein bedeutendes Repertoire auswendig zu lernen nur etwas für die wenigen \enquote{Begabten} ist.
Der Hauptgrund für diese unbegründete Angst ist die frühere Angewohnheit, Schülern zuerst beizubringen, ein Stück gut zu spielen, und erst danach, es auswendig zu lernen, was die schwierigste Art auswendig zu lernen ist (wie in Abschnitt III.6 beschrieben).
Für Schüler, die von Anfang an richtig unterrichtet wurden, ist das Auswendiglernen zur zweiten Natur geworden; es ist ein integraler  Bestandteil davon, eine neue Komposition zu lernen.
Diese Vorgehensweise wird Sie automatisch zu einem guten Auswendiglernenden machen, obwohl das bei älteren Menschen viele Jahre dauern kann.

\hyperref[c1iii15]{Nervosität} ist eine besonders schwer zu überwindende psychologische Barriere.
Um erfolgreich zu sein, muß man verstehen, daß Nervosität ein rein psychologischer Prozeß ist.
Das derzeitige System, junge Schüler ohne richtige \hyperref[c1iii14]{Vorbereitung} in Konzerte zu hetzen, ist kontraproduktiv und erzeugt im allgemeinen Schüler, die für Probleme mit der Nervosität anfälliger sind als zu Beginn ihres Unterrichts.
Wenn ein Schüler beim Klavierspielen erst einmal intensive Nervosität erfahren muß, kann es einen negativen Einfluß auf alle ähnlichen Situationen haben, wie in Theaterstücken zu spielen oder jede andere Art öffentlichen Auftretens.
Deshalb ist das jetzige System für die psychologische Gesundheit allgemein schlecht.
Wie oben in Abschnitt 15 besprochen, ist die Nervosität für die meisten Menschen ein gut zu lösendes Problem, und ein gutes Programm für das Überwinden der Nervosität wird wegen des Stolzes, der Freude und dem Gefühl der Vollendung, das man hat, zur Stärkung der Persönlichkeit beitragen.

Die Psychologie zieht sich durch alles, was wir im Zusammenhang mit dem Klavier tun, von der Motivation der Schüler bis zu den Grundlagen der Musik und des Komponierens von Musik.
Schüler motiviert man am besten, indem man Übungsmethoden lehrt, deren Nutzen so groß ist, daß die Schüler nicht aufhören möchten.
Wettbewerbe und Konzerte sind große Motivatoren, aber sie müssen mit Vorsicht und mit einem richtigen Verständnis der Psychologie durchgeführt werden.
Am interessantesten sind die psychologischen Aspekte der Grundlagen von Musik.
Bach benutzte das einfachste thematische Material, \hyperref[c1iii20ps]{parallele Sets}, und zeigte, daß sie benutzt werden können, um die tiefste jemals geschriebene Musik zu komponieren und uns gleichzeitig zu lehren, wie man übt.
\hyperref[c1iv4]{Mozart benutzte eine Formel} zur Massenproduktion von Musik;
wir verstehen nun, wie er innerhalb eines so kurzen Lebens so viel schreiben konnte.
Beethoven benutzte Konzepte der Gruppentheorie als tragende Säule seiner Musik.
Er zeigte, wie man die Aufmerksamkeit des Publikums mit einer eingängigen Melodie in der RH aufrecht erhalten kann, während man die Emotionen mit der LH kontrolliert, so wie es die Fernsehindustrie heute auch tut, indem Sie uns ein aufregendes Video zeigt und dabei die Emotionen mit Klangeffekten kontrolliert.
Chopin, der wie kein anderer für seinen Romantizismus und seine einzigartige Musikalität bekannt ist, benutzte in seiner Fantaisie Impromptu mathematische Mittel, um Musik zu schreiben, die die \enquote{Klangmauer} durchbricht (s. \hyperref[c1iii2]{Abschnitt III.2}) und besondere Effekte im Gehirn erzeugt, die das Publikum zu fesseln vermögen.
Die \hyperref[c2_2]{chromatische Tonleiter} wurde von Intervallen abgeleitet, und die Musik folgt Akkordprogressionen, weil diese Beziehungen der hörbaren Frequenzen die Verfahren für die Informationsverarbeitung und das Gedächtnis im Gehirn vereinfachen.
Die Technik kann nicht von der Musik getrennt werden, und die Musik kann nicht von der Psychologie getrennt werden; deshalb ist Klavierüben nicht mit dem Aufbau von Fingermuskeln oder mit wiederholenden Übungen gleichzusetzen: Bei der Technik dreht sich letztendlich alles um das menschliche Gehirn.
Die Kunst und die Künstler bringen uns das alles nahe, lange bevor wir analytisch erklären können, warum es funktioniert.


\subsection{Zusammenfassung der Methoden}
\label{c1iii22}

Die Methoden basieren auf sieben Hauptprinzipien:

\begin{enumerate}[label={\arabic*.}] 
 \item \hyperref[c1ii7]{mit getrennten Händen üben} (HS, Abschnitt II.7),
 \item \hyperref[c1ii6]{abschnittsweise üben} (II.6),
 \item Entspannung (\hyperref[c1ii10]{\autoref{c1ii10}} und \hyperref[c1ii14]{14}),
 \item parallele Sets (\hyperref[c1ii11]{\autoref{c1ii11}}, \hyperref[c1iii7b]{\autoref{c1iii7b}}, \hyperref[c1iv2a]{\autoref{c1iv2a}})
 \item \hyperref[c1iii6]{Auswendiglernen} (III.6),
 \item mentales Spielen (\hyperref[c1iii6tastatur]{\autoref{c1iii6tastatur}} und \hyperref[c1iii12]{12})
 \item und musikalisches Spielen (im ganzen Buch).
\end{enumerate}

\begin{itemize} 
\item Lernen Sie nur musikalische Kompositionen, kein \hyperref[c1iii7h]{Hanon}, Czerny usw., aber Tonleitern, Arpeggios und die chromatische Tonleiter sind wichtig (siehe \hyperref[c1iii5a]{\autoref{c1iii5a}}).
Ihr erstes Klavier sollte ein \hyperref[c1iii17b]{Digitalpiano} mit gewichteten Tasten sein; kaufen Sie dann so bald wie möglich einen qualitativ guten \hyperref[c1iii17d]{Flügel}.

\item Hören Sie sich Auftritte und Aufnahmen an.
Es ist nicht möglich, andere exakt nachzuahmen, und es wird Ihnen einige Anregungen geben, die Ihnen beim musikalischen Üben helfen.

\item \hyperref[c1iii6g]{Üben Sie alte, fertige Stücke kalt} (ohne Aufwärmen, siehe Abschnitt III.6g), um Ihre Fähigkeiten für das \hyperref[c1iii14]{Auftreten} zu stärken.

\item Wenn Sie ein neues Stück beginnen, schauen Sie sich die Notenblätter an, um die schwierigen Stellen zu ermitteln, und \hyperref[c1ii5]{üben Sie die schwierigsten Abschnitte zuerst}; dann:

 \begin{enumerate}[label={\alph*.}] 
 \item Üben Sie \hyperref[c1ii7]{mit getrennten Händen} und mit sich überschneidenden Abschnitten (\hyperref[c1ii8]{Kontinuitätsregel}, Abschnitt II.8); wechseln Sie oft die Hände, wenn notwendig, alle fünf Sekunden.
Die gesamte technische Entwicklung sollte HS erfolgen.
 
 \item Lernen Sie das Stück zuerst HS \hyperref[c1iii6]{auswendig}, und beginnen Sie erst dann mit dem Üben für die Technik; versuchen Sie, so schnell wie möglich \hyperref[c1iii7i]{die endgültige Geschwindigkeit zu erreichen}.
Auswendiglernen nachdem man gelernt hat, das Stück gut zu spielen, funktioniert nicht.
Lernen Sie das \hyperref[c1ii12mental]{mentale Spielen}, sobald Sie mit dem Auswendiglernen beginnen, und benutzen Sie es, um sich ein \hyperref[c1iii12]{relatives und abolutes Gehör} anzueignen (Abschnitt III.12).
 
 \item Benutzen Sie die \hyperref[c1ii11]{parallelen Sets}, um Ihre Schwachstellen zu diagnostizieren; \hyperref[c1iii2]{zirkulieren} (Abschnitt III.2) Sie die parallelen Sets, um diese Schwächen zu eliminieren und um schnell zur endgültigen Geschwindigkeit zu kommen.
 
 \item \hyperref[c1ii6]{Teilen Sie schwierige Passagen} in kleine Abschnitte auf, die leicht zu spielen sind, und benutzen Sie diese Abschnitte für die \hyperref[c1ii14]{Entspannung} und die Geschwindigkeit.
\end{enumerate}

\item Spielen Sie den letzten Durchlauf jeder wiederholenden Übung immer \hyperref[c1ii17]{langsam}, bevor Sie die Hände wechseln oder zu einem neuen Abschnitt übergehen.

\item Üben Sie stets das \hyperref[c1ii14]{Entspannen}, besonders HS; das schließt den gesamten Körper ein, inklusive der \hyperref[c1ii21]{Atmung und des Schluckens} (Abschnitt II.21).

\item \hyperref[c1ii22]{Spielen Sie durch Fehler hindurch}; halten Sie nicht an, um sie zu korrigieren, weil Sie dadurch ein Stottern entwickeln.
Korrigieren Sie die Fehler später, indem Sie im Bereich der Fehler abschnittsweise üben.

\item Benutzen Sie das \hyperref[c1ii19]{Metronom} kurz (üblicherweise ein paar Sekunden), um den \hyperref[c1iii1b]{Rhythmus} und die Geschwindigkeit zu prüfen; benutzen Sie es nicht, um die Geschwindigkeit schrittweise zu steigern oder für längere Zeit (mehr als einige Minuten).

\item Benutzen Sie das \hyperref[c1ii23]{Pedal} nur, wenn es in den Noten angezeigt ist; üben Sie ohne Pedal, bis Sie zufriedenstellend HT spielen können, und fügen Sie erst dann das Pedal hinzu.

\item Um das \hyperref[c1ii25]{beidhändige Spielen} (Abschnitt II.25) zu lernen, üben Sie HS bis zu einer Geschwindigkeit, die höher als die endgültige HT-Geschwindigkeit ist, bevor Sie mit dem HT-Üben beginnen.
Um schwierige Passagen HT zu üben, nehmen Sie einen kurzen Abschnitt davon, spielen Sie mit der schwierigeren Hand, und fügen Sie nach und nach die Noten der anderen Hand hinzu.

\item Üben Sie musikalisch, ohne Lautstärke aber mit Festigkeit, Autorität und Ausdruck.
Das Klavierspielenüben ist kein Training der Fingerstärke; es ist die Entwicklung von Fertigkeiten des Gehirns und von Nervenverbindungen für die Kontrolle und die Geschwindigkeit.
Lernen Sie bei Fortissimo-Passagen zuerst das \hyperref[c1ii14]{Entspannen}, die Technik und die Geschwindigkeit, und fügen Sie erst dann das Fortissimo hinzu.
Die Kraft für das Fortissimo kommt aus dem Körper und den Schultern, nicht aus den Armen.

\item Bevor Sie mit dem Üben aufhören, spielen Sie das, was Sie gerade geübt haben, \hyperref[c1ii17]{langsam}, um eine korrekte \hyperref[c1ii15]{automatische Verbesserung nach dem Üben (PPI, Abschnitt II.15)} zu garantieren, die hauptsächlich während des Schlafes stattfindet.
Das letzte, das Sie gebrauchen können, ist, daß die PPI Ihre Fehler einschließt (besonders die aus dem \hyperref[fpd]{Schnellspiel-Abbau} resultierenden -- Abschnitt II.25).
\end{itemize}




