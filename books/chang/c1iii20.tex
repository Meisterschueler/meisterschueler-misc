% File: c1iii20

\subsection{Bach: der größte Komponist und Lehrer (15 Inventionen)}
\label{c1iii20} 

Wir analysieren kurz Bachs fünfzehn zweistimmige Inventionen aus einfacher struktureller Sicht, um zu untersuchen, wie und warum er sie komponierte.
Das Ziel ist, besser zu verstehen, wie man Bachs Kompositionen übt und von ihnen profitiert.
Als Nebenprodukt können wir diese Ergebnisse benutzen, um darüber zu spekulieren, was Musik ist und wie Bach aus (wie wir zeigen werden) grundlegendem \enquote{Unterrichtsmaterial}, das sich nicht von Czerny oder Cramer-Bülow unterscheiden sollte, eine solch unglaubliche Musik erzeugte.
Klar hat Bach fortgeschrittene musikalische Konzepte der Harmonie, des Kontrapunkts usw. benutzt, die von Musiktheoretikern bis  zum heutigen Tag diskutiert werden, während andere \enquote{Unterrichtsmusik} hauptsächlich wegen ihres Werts für das Fingertraining geschrieben haben.
Hier untersuchen wir die Inventionen nur auf der einfachsten strukturellen Stufe.
Sogar auf dieser grundlegenden Stufe gibt es ein paar lehrreiche und faszinierende Ideen, die wir erforschen können und zu der Erkenntnis führen, daß Musik und Technik untrennbar sind.
 
Es gibt einen guten Aufsatz über Bachs Inventionen, ihre Geschichte usw. von Dr. Yo Tomita von der Queen's University in Belfast, Irland (www.music.qub.ac.uk/~tomita/essay/inventions.html).
Jede Invention benutzt eine andere der Tonleitern, die in den zu Bachs Zeiten favorisierten \hyperref[c2_2_wtk2]{Wohltemperierten Stimmungen} wichtig waren.
Die Inventionen wurden -- ungefähr 1720 -- ursprünglich für seinen ältesten Sohn Wilhelm Friedemann Bach geschrieben, als dieser 9 Jahre alt war.
Sie wurden später verändert und anderen Schülern gelehrt.

Die bemerkenswerteste Eigenschaft aller Inventionen ist, daß jede sich auf eine kleine Zahl \hyperref[c1iii7b]{paralleler Sets} konzentriert, gewöhnlich weniger als drei.
Nun könnte man sagen: \enquote{Das ist nicht fair -- da praktisch jede Komposition in parallele Sets zerlegt werden kann, müssen die Inventionen natürlich aus lauter parallelen Sets bestehen. Was ist also neu daran?}
Das neue Element ist, daß jede Invention nur auf einem bis drei bestimmten parallelen Sets basiert, die Bach zum Üben ausgewählt hat.
Um das zu zeigen, listen wir diese parallelen Sets weiter unten für jede Invention auf.
Um sich ganz auf einfache parallele Sets zu konzentrieren, vermied Bach völlig den Gebrauch von Terzen und komplexeren Intervallen (in einer Hand), die \hyperref[c1iii7h]{Hanon} in seinen Übungen mit höheren laufenden Nummern benutzt hat.
Bach wollte, daß seine Schüler die parallelen Sets vor den Intervallen beherrschen.

Einzelne parallele Sets sind aus technischer Sicht fast trivial.
Deshalb sind sie so nützlich -- sie sind leicht zu lernen.
Jeder mit einiger Erfahrung am Klavier kann lernen, sie sehr schnell zu spielen.
Die wahre technische Herausforderung tritt auf, wenn man zwei davon mit einer Verbindung dazwischen vereinen muß.
Bach wußte das offensichtlich und benutzte deshalb nur Kombinationen paralleler Sets als Bausteine.
Somit lehren uns die Inventionen, wie man parallele Sets und Verbindungen spielt -- parallele Sets zu lernen macht keinen Sinn, wenn man sie nicht verbinden kann.
Im folgenden benutze ich den Begriff \enquote{lineare} parallele Sets, um Sets zu beschreiben, in denen die Finger nacheinander spielen (z.B. 12345), und \enquote{alternierende} Sets, wenn die Finger abwechselnd spielen (z.B. 132435).
Diese verbundenen parallelen Sets bilden das, was man in diesen Inventionen normalerweise \enquote{Motiv} nennt.
Die Tatsache, daß sie aus den grundlegendsten parallelen Sets erzeugt wurden, läßt jedoch darauf schließen, daß die \enquote{Motive} nicht wegen ihres musikalischen Gehalts ausgewählt wurden, sondern wegen ihres pädagogischen Werts, und die Musik wurde dann durch die Genialität Bachs hinzugefügt.
Dadurch konnte nur Bach eine solche Meisterleistung vollbringen; das erklärt, warum Hanon scheiterte.
Der Hauptgrund für das Scheitern Hanons ist natürlich, daß er im Gegensatz zu Bach die guten Übungsmethoden nicht kannte.
Ich habe im folgenden jeweils nur eine repräsentative Kombination der parallelen Sets für jede Invention angeführt; Bach benutzte sie in vielen Variationen, wie z.B. umgekehrt, gespiegelt usw.
Beachten Sie, daß Hanon seine Übungen im Grunde aus den gleichen parallelen Sets aufbaute, obwohl er das wahrscheinlich eher dem Zufall verdankte, daß er diese Motive aus Bachs Werken entnahm.
Vielleicht ist der überzeugendste Beweis dafür, daß Bach die parallelen Sets kannte, die mit steigender Nummer der Inventionen zunehmende Komplexität der von ihm ausgewählten parallelen Sets.


\label{c1iii20ps}

<h3><br>Liste der parallelen Sets in den einzelnen Inventionen (für die RH)</h3>

\begin{enumerate}[label={\arabic*.}] 

\item \label{c1iii20ps01}

1234 und 4231 (linear gefolgt von alternierend); das war ein Fehler, weil die erste Invention nur die einfachsten (linearen) Sets behandeln sollte.
Dementsprechend ersetzte Bach in einer späteren Änderung dieser Invention das alternierende Set 4231 durch zwei lineare Sets, 432 und 321.
Diese Änderung ist ein, meine These unterstützendes, Indiz dafür, daß Bach die parallelen Sets als grundlegende Einheiten für das Strukturstudium benutzte.
Die Reihenfolge der Inventionen in bezug auf die Schwierigkeit mag jedoch für die meisten von uns nicht dieselbe wie die in bezug auf die Komplexität der parallelen Sets sein, weil die strukturelle Einfachheit der parallelen Sets nicht immer ein leichteres Spielen bedeutet.


\item \label{c1iii20ps02}

Lineare Sets wie in \hyperref[c1iii20ps01]{\#1} aber mit einer größeren Vielfalt in den Verbindungen.
Eine zusätzliche Komplexität ist, daß das gleiche Motiv an den verschiedenen Stellen unterschiedliche Fingersätze erfordert.
Somit behandeln die ersten beiden Inventionen hauptsächlich lineare Sets, aber die zweite ist komplexer.


\item \label{c1iii20ps03}

324 und 321 (alternierend gefolgt von linear).
Ein kurzes alternierendes Set wird eingeführt.


\item \label{c1iii20ps04}

12345 und 54321 mit einer ungewöhnlichen Verbindung.
Diese längeren linearen Sets mit der ungewöhnlichen Verbindung erhöhen die Schwierigkeit.


\item \label{c1iii20ps05}

4534231.
Völlig alternierende Sets.


\item \label{c1iii20ps06}

545, 434, 323 usw.
Das einfachste Beispiel der grundlegenden zweinotigen parallelen Sets, die mit einer Verbindung verknüpft werden; diese sind schwierig, wenn schwache Finger einbezogen werden.
Obwohl sie einfach sind, sind sie extrem wichtige technische Grundelemente, und sie zwischen den beiden Händen zu wechseln, ist eine großartige Möglichkeit, zu lernen sie zu kontrollieren (indem man \hyperref[c1ii20]{eine Hand benutzt, um die andere zu unterrichten}, s. Abschnitt II.20).
Sie führt auch die arpeggioartigen Sets ein.


\item \label{c1iii20ps07}

543231.
Das ist wie eine Kombination von \hyperref[c1iii20ps03]{\#3} und \hyperref[c1iii20ps04]{\#4} und ist deshalb komplexer als beide.


\item \label{c1iii20ps08}

14321 und die erste Einführung der \enquote{Alberti}-artigen Kombination 2434.
Hier wird die Steigerung der Schwierigkeit durch die Tatsache erzeugt, daß die 14 am Anfang nur einen oder zwei Halbtöne entfernt sind, was es für Kombinationen, die schwache Finger einbeziehen, schwieriger macht.
Es ist erstaunlich, daß Bach nicht nur alle Kombinationen mit den schwachen Fingern kannte, sondern auch, wie er sie in reale Musik einband.
Darüber hinaus wählte er Situationen, in denen er den schwierigen Fingersatz benutzen mußte.


\item \label{c1iii20ps09}

Der Lehrstoff ist hier ähnlich dem in \hyperref[c1iii20ps02]{\#2} (lineare Sets) aber schwieriger.


\item \label{c1iii20ps10}

Dieses Stück besteht fast vollständig aus arpeggioartigen Sets.
Da die Finger bei arpeggioartigen Sets eine größere Strecke zwischen den Noten zurücklegen müssen, stellen sie eine weitere Steigerung in der Schwierigkeit dar.


\item \label{c1iii20ps11}

Ähnlich wie \hyperref[c1iii20ps02]{\#2} und \hyperref[c1iii20ps09]{\#9}.
Die Schwierigkeit wird erneut gesteigert, indem das Motiv gegenüber den vorangegangenen Stücken verlängert ist.
Beachten Sie, daß es in allen anderen Stücken nur ein kurzes Motiv gibt, dem ein einfacher Kontrapunktabschnitt folgt, was es vereinfacht, sich auf die parallelen Sets zu konzentrieren.


\item \label{c1iii20ps12}

Diese kombiniert lineare und arpeggioartige Sets und wird schneller gespielt als die vorherigen Stücke.


\item \label{c1iii20ps13}

Arpeggioartige Sets, wird schneller gespielt als \hyperref[c1iii20ps10]{\#10}.


\item \label{c1iii20ps14}

12321, 43234.
Eine schwierigere Version von \hyperref[c1iii20ps03]{\#3} (fünf Noten statt drei und schneller).


\item \label{c1iii20ps15}

3431, 4541.
Schwierige Kombinationen, die Finger 4 einbeziehen.
Diese Fingerkombinationen sind besonders schwer zu spielen, wenn mehrere von ihnen aneinandergereiht werden.
\end{enumerate}

Die obige Liste zeigt, daß:

\begin{enumerate}[label={\roman*.}] 
\item es eine systematische Einführung in zunehmend komplexere parallele Sets gibt.
\item die Tendenz zu einer schrittweisen Steigerung der Schwierigkeit besteht, mit der Betonung der Entwicklung der schwächeren Finger.
\item die \enquote{Motive} in Wahrheit sorgfältig ausgewählte parallele Sets und Verbindungen sind, die wegen ihres technischen Werts ausgewählt wurden.
\end{enumerate}

Die Tatsache, daß Motive, die einfach wegen ihrer technischen Nützlichkeit ausgewählt wurden, benutzt werden können, um einige der größten Musikstücke, die jemals komponiert wurden, zu erzeugen, ist faszinierend.
Diese Tatsache ist für Komponisten nichts Neues.
Für den durchschnittlichen Musikliebhaber, der von Bachs Musik begeistert ist, scheinen diese Motive wegen der Vertrautheit, die sie  bei mehrfachem Anhören erzeugen, eine besondere Bedeutung mit scheinbar tiefem musikalischen Wert zu gewinnen.
In Wahrheit sind es nicht die Motive selbst, sondern wie sie in der Komposition benutzt werden, was den Zauber erzeugt.
Wenn Sie sich einfach die reinen Grundmotive ansehen, können sie kaum einen Unterschied zwischen Hanon und Bach feststellen, und trotzdem wird niemand die Hanon-Übungen als Musik ansehen.
Das ganze Motiv besteht eigentlich nur aus den parallelen Sets und dem angefügten Kontrapunktabschnitt, der so genannt wird, weil er als Kontrapunkt zu dem wirkt, was von der anderen Hand gespielt wird.
Bachs geschickte Verwendung des Kontrapunkts dient offensichtlich mehreren Zwecken, von denen einer das Erzeugen der Musik ist.
Es mag so erscheinen, als ob der Kontrapunkt (der in den Hanon-Übungen fehlt) keine technischen Lehren hinzufügt (der Grund, warum Hanon ihn ignorierte), aber Bach benutzte ihn, um Fertigkeiten wie z.B. Triller, Verzierungen, Staccatos, Unabhängigkeit der Hände usw. zu üben, und der Kontrapunkt macht es sicher viel einfacher, die Musik zu komponieren und ihre Schwierigkeitsstufe anzupassen.

Musik wird so durch eine \enquote{logische} Reihenfolge von Noten oder Gruppen von Noten erzeugt, die vom Gehirn erkannt werden, so wie Ballett, schöne Blumen oder eine herrliche Landschaft visuell erkannt werden.
Was ist diese \enquote{Logik}?
Ein großer Teil davon ist wie im visuellen Fall eine automatisierte, fast so etwas wie eine festverdrahtete, Datenverarbeitung des Gehirns; sie beginnt mit einer angeborenen Komponente (neugeborene Babys schlafen ein, wenn sie ein Wiegenlied hören), aber eine große Komponente kann erlernt werden (z.B. Bach von Rock 'n' Roll zu unterscheiden).
Aber sogar der erlernte Teil ist größtenteils automatisiert.
Mit anderen Worten: Wenn ein beliebiger Ton das Gehör erreicht, fängt das Gehirn sofort damit an, die Klänge zu verarbeiten und zu interpretieren, egal ob wir bewußt versuchen, die Information zu verarbeiten oder nicht.
Ein sehr großer Teil dieser automatisierten Verarbeitung geschieht, ohne daß wir Notiz davon nehmen, wie Tiefenwahrnehmung, Fokussierung der Augen, Bestimmung der Richtung und Quellen von Tönen, Bewegungen zum Gehen und Halten der Balance, Töne in beängstigende und beruhigende zu unterscheiden usw.
Das meiste dieser Verarbeitung ist angeboren oder erlernt, ist aber im Grunde außerhalb unserer bewußten Kontrolle.
Das Ergebnis dieser mentalen Verarbeitung ist, was wir Sinn für Musik nennen.
Akkordprogressionen und andere Elemente der Musiktheorie geben uns eine Vorstellung davon, was diese Logik ist.
Aber das meiste dieser \enquote{Theorie} ist heutzutage eine einfache Zusammenstellung von verschiedenen Eigenschaften wirklich existierender Musik.
Diese bilden keine ausreichend grundlegende Theorie, um uns zu gestatten, Musik zu erzeugen, obwohl sie uns erlauben, Fallen zu vermeiden und eine Komposition zu erweitern bzw. zu vervollständigen, wenn man erst einmal auf beliebige Weise ein realisierbares Motiv erzeugt hat.
Es scheint so, als ob die Musiktheorie heute immer noch sehr unvollständig wäre.
Hoffentlich können wir, indem wir weiterhin Musik von großen Meistern analysieren, langsam, Schritt für Schritt das Ziel erreichen, ein besseres Verständnis der Musik zu entwickeln.



