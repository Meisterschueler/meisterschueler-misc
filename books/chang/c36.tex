% File: c36

\subsection{Das Unterbewußtsein}
\label{c3_6}

\subsubsection{Einleitung}
\label{c3_6a}

Das Gehirn hat einen bewußten und einen unterbewußten Teil.
Die meisten Menschen haben es nicht gelernt, das Unterbewußtsein zu benutzen, aber das Unterbewußtsein ist wichtig, weil es

\begin{itemize} 
 \item die Emotionen kontrolliert,
 \item 24 Stunden am Tag funktioniert, egal ob man wach ist oder schläft, und
 \item einige Dinge tun kann, die das Bewußtsein nicht kann, einfach weil es eine andere Art Gehirn ist.
 \end{itemize}
Obwohl es schwierig ist, das bewußte Gehirn mit dem unterbewußten zu vergleichen, weil sie verschiedene Funktionen ausüben und verschiedene Fähigkeiten haben, könnten wir statistisch vermuten, daß das Unterbewußtsein bei der Hälfte der menschlichen Bevölkerung cleverer ist als das bewußte.
Zusätzlich zur Tatsache, daß man eine zusätzliche Fähigkeit des Gehirns hat, macht es deshalb keinen Sinn, diesen Teil des Gehirns, der eventuell cleverer ist als der bewußte Teil, nicht zu benutzen.
In diesem Abschnitt präsentiere ich meine Ideen dazu, wie das Unterbewußte eventuell funktioniert und zeige, wie wir mit Hilfe des Unterbewußtseins einige erstaunliche Leistungen vollbringen können.


\subsubsection{Emotionen}
\label{c3_6b}

Das Unterbewußtsein kontrolliert Emotionen auf mindestens zwei Arten.
Die erste ist eine schnelle Kampf- oder Fluchtreaktion -- das Erzeugen von sofortiger Wut oder Furcht.
Wenn solche Situationen aufkommen, muß man schneller reagieren können als man denken kann, so daß das bewußte Gehirn durch etwas umgangen werden muß, das für eine sofortige Reaktion fest verdrahtet und vorprogrammiert ist.
Die zweite ist ein sehr langsames, schrittweises Erkennen einer tiefen oder grundlegenden Situation.
Ob der erste und der zweite Teil des unterbewußten Gehirns Teile desselben Unterbewußtseins sind, ist eine akademische Frage, da wir fast mit Sicherheit viele Arten eines unterbewußten Verhaltens besitzen.
Gefühle der Depression während einer Midlife-Krise könnten das Ergebnis von Vorgängen der zweiten Art von Unterbewußtsein sein: Das unterbewußte Gehirn hat während man älter wird Zeit gehabt, alle negativen Situationen herauszufinden, die sich entwickeln, und die Zukunft fängt an weniger hoffnungsvoll auszusehen.
Solch ein Prozeß erfordert die Auswertung von Myriaden guter und schlechter Möglichkeiten, die die Zukunft bringen mag.
Wenn man versuchen wollte, solch eine zukünftige Situation zu bewerten, müßte das bewußte Gehirn alle diese Möglichkeiten auflisten, jede bewerten und versuchen, sie zu behalten.
Das Unterbewußtsein funktioniert anders.
Es bewertet verschiedene Situationen auf eine unsystematische Weise; wie es eine bestimmte Situation für die Beurteilung auswählt, unterliegt nicht unserer Kontrolle; das wird mehr von alltäglichen Ereignissen kontrolliert.
Das Unterbewußtsein speichert seine Schlußfolgerungen auch in etwas, was man \enquote{Emotionsfach} nennen könnte.
Für jede Emotion gibt es ein Fach und jedesmal, wenn das Unterbewußtsein zu einem Entschluß kommt, sagen wir zu einem glücklichen, dann deponiert es den Entschluß in einem \enquote{Glücklichfach}.
Der Füllgrad jedes Fachs bestimmt Ihren emotionalen Zustand.
Das erklärt, warum Menschen oftmals spüren können, was richtig oder falsch ist oder ob eine Situation gut oder schlecht ist, ohne daß sie genau wissen, was die Gründe dafür sind.
So beeinflußt das Unterbewußtsein unser Leben viel mehr als die meisten von uns merken.


\subsubsection{Das Unterbewußtsein benutzen}
\label{c3_6c}

Üblicherweise geht das Unterbewußtsein seine eigenen Wege; man kontrolliert normalerweise nicht, welche Möglichkeiten es in Betracht zieht, weil die meisten von uns nicht gelernt haben, mit ihm zu kommunizieren.
Die Ereignisse, denen man im täglichen Leben begegnet, machen es jedoch in der Regel ziemlich deutlich, welche Faktoren wichtig sind und welche unwichtig, und es zieht das Unterbewußtsein ganz natürlich zu den wichtigsten Ideen.
Wenn diese wichtigen Ideen zu wichtigen Schlüssen führen, wird es interessierter.
Wenn sich eine genügende Zahl solcher wichtiger Schlüsse aufstapeln, wird es sich mit Ihnen in Verbindung setzen.
Das erklärt, warum manchmal plötzlich eine unerwartete Eingebung in unserem Bewußtsein aufblitzt.
Darum ist hier die wichtige Frage, wie man am besten mit seinem Unterbewußtsein kommunizieren kann.

Jede Idee, von der man sich selber überzeugen kann, daß sie wichtig ist, oder jedes Rätsel oder Problem, das man mit großer Mühe zu lösen versucht hat, wird offensichtlich ein Kandidat zur Überprüfung durch das Unterbewußtsein sein.
Das ist deshalb eine Art, wie man sein Problem dem Unterbewußtsein präsentieren kann.
Außerdem muß das Unterbewußtsein, um in der Lage zu sein, ein Problem zu lösen, alle notwendigen Informationen besitzen.
Deshalb ist es wichtig, alles zu untersuchen und so viele Informationen über das Problem zu sammeln wie man kann.
Im College habe ich auf diese Art viele Probleme meiner Hausaufgaben gelöst, die meine klügeren Klassenkameraden nicht lösen konnten.
Sie haben versucht, sich einfach hinzusetzen, ihre Aufgabe zu bearbeiten und hofften, das Problem zu lösen.
Probleme in einer schulischen Umgebung sind solche, die immer mit den Informationen, die im Klassenzimmer oder Lehrbuch gegeben wurden, lösbar sind.
Man muß nur die richtigen Teile zusammenfügen, um auf die Antwort zu kommen.
Ich habe mir deshalb keine Gedanken darüber gemacht, ob ich in der Lage wäre, das Problem sofort zu lösen, sondern habe nur intensiv darüber nachgedacht, um sicherzustellen, daß ich das ganze Kursmaterial studiert hatte.
Wenn ich ein Problem nicht sofort lösen konnte, wußte ich, daß mein Unterbewußtsein weiter daran arbeiten würde, so daß ich das Problem einfach vergessen und später dazu zurückkehren konnte.
Somit war es nur erforderlich, daß ich nicht bis zur letzten Minute wartete, um zu versuchen solche Probleme zu lösen.
Einige Zeit danach würde die Antwort plötzlich in meinem Kopf auftauchen, oftmals zu merkwürdigen und unerwarteten Gelegenheiten.
Sie tauchten meistens am frühen Morgen auf, wenn mein Geist erholt und frisch war.
Man kann also der Erfahrung nach sowohl lernen, dem Unterbewußtsein das Material zu präsentieren, als auch die Schlußfolgerungen daraus zu empfangen.
Im allgemeinen kam die Antwort nicht, wenn ich mein Unterbewußtsein absichtlich darum bat, sondern sie kam, wenn ich etwas tat, das mit dem Problem nicht in Zusammenhang stand.
Man kann das Unterbewußtsein auch benutzen, um sich an etwas zu erinnern, das man vergessen hat.
Versuchen Sie zunächst, sich so gut Sie können daran zu erinnern, und bemühen Sie sich dann für eine Weile überhaupt nicht mehr.
Nach einiger Zeit wird sich Ihr Gehirn oftmals für Sie daran erinnern.

Selbstverständlich kennen wir bis jetzt noch keinen direkten Weg, uns mit unserem Unterbewußtsein zu unterhalten.
Und diese Kommunikationskanäle sind von Person zu Person sehr verschieden, so daß jede Person experimentieren muß, um herauszufinden was am besten funktioniert.
Klar kann man die Kommunikation mit ihm sowohl verbessern, als auch die Kommunikationskanäle blockieren.
Viele meiner clevereren Freunde im College wurden sehr frustriert, wenn sie herausfanden, daß ich die Antwort ohne Anstrengung gefunden hatte, während sie es nicht konnten; und sie wußten, daß sie cleverer waren.
Diese Art der Frustration kann jegliche Kommunikation zwischen den verschiedenen Teilen des Gehirns blockieren.
Es ist besser, eine entspannte, positive Einstellung aufrechtzuerhalten und das Gehirn seine Sache erledigen zu lassen.
Das ist wahrscheinlich der Grund, warum Techniken wie Meditation und Qi Gong so gut funktionieren.
Das sind effektive, lange Zeit getestete, Methoden der Kommunikation mit den verschiedenen Teilen des Gehirns und des Körpers.
Beachten Sie, daß die verschiedenen Teile des Gehirns viele Körperfunktionen direkt kontrollieren, wie z.B. die Herzfrequenz, den Blutdruck, die Atmung, Verdauung, Speichelbildung, die Funktion der inneren Organe, sexuelle Reaktionen usw.
Das sind mächtige Funktionen, die große Mengen von Energie erzeugen oder verschwenden können, so daß wie die Teile reibungslos zusammenarbeiten oder gegeneinander agieren einen wichtigen Effekt auf Ihre allgemeine Gesundheit und geistige Funktionen hat.
Eine weitere wichtige Methode, einen maximalen Nutzen aus dem Unterbewußtsein zu ziehen, ist, es ohne Störung durch das bewußte Gehirn sich selbst zu überlassen, nachdem man ihm das Problem präsentiert hat.
Mit anderen Worten: Sie sollten das Problem vergessen und sich sportlich betätigen, ins Kino gehen oder etwas anderes tun, das Ihnen Spaß macht, und das Unterbewußtsein wird seine Aufgabe besser erfüllen, weil es ein völlig anderer Teil Ihres Gehirns ist.
Wenn Sie die ganze Zeit bewußt über das Problem nachdenken, dann beeinflussen Sie das Unterbewußtsein und erlauben ihm nicht, seine eigenen freien Forschungen zu betreiben.

Das Gehirn hat viele Teile, und es ist von Vorteil, jedes Teil zu kennen und zu lernen, wie man es benutzt.
Das unterbewußte Gehirn ist wahrscheinlich eines der am meisten zu wenig genutzten Teile unseres Gehirns, weil zu vielen von uns seine Existenz nicht bewußt ist.
Es muß bestimmt noch viele andere nützliche Teile unseres Gehirns geben.
So gibt es z.B. zahlreiche automatische Gehirnprozesse, die unser tägliches Leben beeinflussen.
Wenn wir ein Bild mit unseren Augen sehen, geschehen viele Dinge sofort und automatisch.
Wenn man ein Bild empfängt, wird das Gehirn vorübergehend mit der Informationsverarbeitung überladen, so daß es andere Aufgaben nicht gut ausführen kann.
Deshalb spürt man auch mit geöffneten Augen weniger Schmerzen als mit geschlossenen.
Ein ähnlicher Effekt tritt bei Geräuschen auf.
Deshalb vermindert Schreien bei Schmerzen tatsächlich die Schmerzen.
Der angenehme Klang der Musik ist eine weitere automatische Reaktion, genauso ist es bei Reaktionen auf visuelle Eingaben wie schönen Blumen, beruhigenden Panoramablicken von Bergen oder Seen, oder der Wirkung von unangenehmen oder angenehmen Düften.
Es ist eine dieser automatischen Reaktionen, die wir aufrufen, wenn wir Musik anhören;  trotzdem, gerade so wie wir nicht genau erklären können, warum eine schöne Blume schön aussieht, können wir nicht genau erklären, warum Musik so gut klingt.
Vielleicht ist es eine von diesen festverdrahteten unterbewußten Reaktionen.

Die Identifizierung der verschiedenen Teile des Gehirn muß sicherlich eine der zukünftigen Revolutionen sein.
Die medizinische Wissenschaft schreitet immer schneller voran, und das Gehirn zu verstehen wird einer der größten Durchbrüche sein, angefangen damit, wie es sich in der Kindheit entwickelt und wie wir diese Entwicklung erleichtern können.
Deshalb ist es voll und ganz möglich, daß Mozart kein musikalisches Genie war, sondern ein Genie, das durch die Musik erzeugt wurde.
 





