% File: c1i1

\part{Klaviertechnik}\hypertarget{c1i1}{}

\chapter{Einführung} 

% zuletzt geändert 15.08.2009
\section{Zweck dieses Buchs}

Der Zweck dieses Buchs ist es, die besten bekannten Methoden zum Üben des Klavierspielens vorzustellen.
Für Schüler bedeutet das 
Kennen dieser Methoden eine Verringerung der zum Lernen notwendigen Zeit, die einen wesentlichen Teil der Lebenszeit ausmacht, und eine Zunahme der Zeit, die für das Musizieren genutzt werden kann, anstatt sie mit dem Kampf mit der Spieltechnik zu verbringen.
Viele Schüler verbringen 100 Prozent ihrer Zeit damit, neue Stücke zu lernen, und da dieser Vorgang so lange dauert, bleibt keine Zeit für das Üben der Kunst Musik zu machen übrig.
Dieser bedauerliche Umstand ist das größte Hindernis für die Entwicklung der Spieltechnik, weil das Musizieren für die technische Entwicklung notwendig ist.
\textbf{Das Ziel ist hier, den Lernprozess so zu beschleunigen, dass wir 10 Prozent der Übungszeit auf die technische Arbeit und 90 Prozent zum Musizieren verwenden.}

Wie musizieren Musiker?
\textbf{Egal ob man Musik komponiert oder ein Instrument spielt, die ganze Musik muss aus dem Gehirn des Künstlers kommen.}
Mit genügend Übung können wir sicher unser Gehirn abschalten und mechanisch aus dem Gedächtnis spielen.
Das ist absolut die falsche Art Musik zu machen, weil die resultierende Musik auf einer niedrigen Stufe sein wird.
Viele Klavierspieler nehmen irrtümlicherweise an, dass ein teurer, großer Konzertflügel seinen eigenen Klang mit seiner charakteristischen Musik erzeugt und wir deshalb zum Lernen des Klavierspielens unsere Finger trainieren müssen.
Das menschliche Gehirn ist aber hinsichtlich der Musikalität weitaus komplexer als alle mechanischen Apparate und diesen überlegen.
Das Gehirn hat nicht die Beschränkungen von Holz, Filz und Metall.
Deshalb ist es wichtiger, das Gehirn zu trainieren als die Fingermuskeln, insbesondere da jede Fingerbewegung von einem Nervenimpuls des Gehirns angeregt werden muss.
Die Antwort auf die obige Frage ist, was wir in diesem Buch als \hyperlink{c1ii12mental}{mentales Spielen} bezeichnen werden.
Das mentale Spielen ist einfach der Prozess, sich die Musik in Gedanken vorzustellen oder sie sogar auf einem imaginären Klavier zu spielen.
Wir werden sehen, dass das mentale Spielen praktisch alles kontrolliert, was wir in der Musik tun, vom Lernprozess (Technik) bis zu dem \hyperlink{c1iii6}{Auswendiglernen}, dem \hyperlink{c1iii12}{absoluten Gehör}, dem \hyperlink{c1iii14}{Auftreten}, dem \hyperlink{c1iii12blatt}{Komponieren}, der \hyperlink{c1iii15}{Kontrolle von Nervosität} usw.
Es ist so allumfassend, dass es nicht möglich ist, es nur in einem Abschnitt zu erklären; es wird praktisch in jedem Abschnitt dieses Buchs besprochen.
Eine ziemlich ausführliche Besprechung finden Sie in \hyperlink{c1iii6tastatur}{Abschnitt III.6j}.

Das mentale Spielen machte Mozart (und alle großen Musiker) zu dem, was er war; er wird zum Teil wegen seiner Fähigkeit zum mentalen Spielen als einer der größten Genies angesehen.
Die wunderbare Nachricht ist, \textbf{dass man es lernen kann}.
Die traurige historische Tatsache ist, dass man das mentale Spielen zu vielen Schülern nie gelehrt hat;
in diesem Buch wurde dem mentalen Spielen vielleicht zum ersten Mal ein offizieller Name (Definition) gegeben, obwohl Sie, wenn Sie ein \enquote{talentierter} Musiker sind, es sich vielleicht schon irgendwie auf wundersame Weise aneignen mussten.
\textbf{Das mentale Spielen sollte bereits im ersten Jahr des Klavierunterrichts gelehrt werden und ist bei den jüngsten Kindern besonders effektiv;
die offensichtlichste Möglichkeit, das Unterrichten zu beginnen, ist, die Fertigkeiten für das Auswendiglernen und das absolute Gehör zu lehren.}
Das mentale Spielen ist die Kunst, den Geist des Publikums durch die Musik, die man spielt, zu kontrollieren, und deshalb funktioniert das am besten, wenn es von Ihrem Geist ausgeht.
Das Publikum sieht Ihre Fähigkeit, mental zu spielen, als etwas 
Außerordentliches, was nur ein paar ausgewählten begabten Musikern mit einer Intelligenz weit über dem Durchschnitt gegeben ist.
Mozart war sich fast mit Sicherheit dessen bewusst und benutzte das mentale Spielen, um sein Ansehen sehr zu vergrößern.
Das mentale Spielen hilft Ihnen auf unzählige Arten, das Klavierspielen zu lernen, wie im ganzen Buch gezeigt wird.
Da man das mentale Spielen ohne Klavier ausführen kann, können Sie zum Beispiel Ihre Übungszeit effektiv verdoppeln oder verdreifachen, 
indem Sie das mentale Spielen benutzen, wenn kein Klavier verfügbar ist.
Beethoven und Einstein erschienen oft geistesabwesend, weil sie die meiste wache Zeit mit dem mentalen Spielen verbrachten.\footnote{Wobei Einstein nicht nur Musik mental spielte, sondern zum Beispiel auch: \enquote{Was würde geschehen, wenn die Straßenbahn, in der ich gerade unterwegs bin, mit Lichtgeschwindigkeit fahren würde?}}

Somit ist das mentale Spielen nichts Neues;
nicht nur die großen Musiker und Künstler, sondern praktisch alle heutigen Spezialisten, wie Athleten, trainierte Soldaten, Geschäftsmänner usw. müssen ihr eigenes mentales Spielen pflegen, um im Wettbewerb erfolgreich zu sein.
Tatsächlich benutzt es jeder von uns ständig!
Wenn wir morgens aufstehen und die für den Tag geplanten Aktivitäten auf die Schnelle durchgehen, führen wir mentales Spielen aus, und die Komplexität dieses mentalen Spielens übersteigt wahrscheinlich die einer Mazurka von Chopin.
Trotzdem tun wir das innerhalb eines Augenblicks, sogar ohne es als mentales Spielen aufzufassen, weil wir es seit frühester Kindheit geübt haben.
Können Sie sich vorstellen, welche Desaster geschehen würden, wenn wir keinen mentalen Plan für den Tag hätten?
Aber das tun wir im Grunde, wenn wir auf eine Bühne gehen und ohne Training im mentalen Spielen ein Konzert geben.
Kein Wunder, dass man beim Auftreten so nervös wird!
Wie wir sehen werden, ist das mentale Spielen vielleicht das beste \hyperlink{c1iii15}{Mittel gegen das Lampenfieber} - bei Mozart funktionierte es bestimmt.


\section{Was ist Klaviertechnik?}\hypertarget{c1i2}{}

Wir müssen verstehen, was Technik ist, weil sie nicht zu verstehen zu falschen Übungsmethoden führt.
Wichtiger noch: Das richtige Verstehen kann uns dabei helfen, die richtigen Übungsmethoden zu entwickeln.
Das am meisten verbreitete Missverständnis ist, dass Technik eine vererbte Fingerfertigkeit sei.
Sie ist es nicht.
\textbf{Die angeborene Geschicklichkeit von vollendeten Pianisten und von Durchschnittsbürgern ist gar nicht so unterschiedlich.}
Das bedeutet, dass praktisch jeder lernen kann, gut Klavier zu spielen.
Es gibt zahlreiche Beispiele von geistig Behinderten mit eingeschränkter Koordination (Savants, Inselbegabte), die ein erstaunliches musikalisches Talent beweisen.
Viele von uns sind wesentlich geschickter, können jedoch leider die musikalischen Passagen aus einem Mangel an ein paar einfachen aber entscheidenden Informationen nicht bewältigen.
\textbf{Der Erwerb der Technik ist größtenteils ein Prozess der Entwicklung des Gehirns und der Nerven, nicht der Entwicklung von Fingerstärke.}

Technik ist die Fähigkeit, millionen verschiedener Passagen auszuführen; deshalb ist sie keine Geschicklichkeit, sondern eine Ansammlung vieler Fertigkeiten.
Das Wundersame an der Klaviertechnik und \textbf{die wichtigste Botschaft dieses Buchs ist, dass diese Fertigkeiten innerhalb kurzer Zeit erlernt werden können, wenn die richtigen Lernmethoden angewandt werden.}
Diese Fertigkeiten werden in zwei Phasen erlangt:

\begin{enumerate}[label={\arabic*.}] 
\item entdecken, wie Finger, Hände, Arme usw. bewegt werden müssen, und
\item das Gehirn, die Nerven und die Muskeln so zu konditionieren, dass sie diese Bewegungen einfach und kontrolliert ausführen können.
 \end{enumerate}
Viele Schüler denken, dass Klavierspielen zu üben aus stundenlanger Fingergymnastik besteht, weil ihnen nie die eigentliche Bedeutung der Technik beigebracht wurde.
\textbf{In Wahrheit verbessern Sie Ihr Gehirn, wenn Sie das Klavierspielen lernen!}
Sie machen sich selbst klüger und verbessern Ihr Gedächtnis; deshalb hat es so viele nützliche Auswirkungen, wenn Sie das Klavierspielen richtig lernen, zum Beispiel Erfolg in der Schule, die Fähigkeit, besser mit alltäglichen Problemen fertigzuwerden und die Fähigkeit, sich trotz zunehmenden Alters das Gedächtnis länger zu erhalten.
Deshalb ist das \hyperlink{c1iii6}{Auswendiglernen} ein untrennbarer Bestandteil des Technikerwerbs.

Wir müssen unsere eigene Anatomie verstehen und lernen, wie wir die korrekte Technik entdecken und uns aneignen können.
Dies stellt eine fast unmögliche Aufgabe für das durchschnittliche menschliche Gehirn dar, es sei denn, Sie widmen ihr von Kindheit an Ihr ganzes Leben.
Selbst dann werden die meisten keinen Erfolg haben.
Der Grund ist, dass der Klavierspieler ohne die richtige Anleitung die korrekten Bewegungen usw. durch Ausprobieren herausfinden muss.
Man hängt von der geringen Wahrscheinlichkeit ab, dass die Hand bei dem Versuch, diese schwierige Passage schneller zu spielen, zufällig in eine funktionierende Bewegung verfällt.
Wenn Sie Pech haben, entdeckt Ihre Hand diese Bewegung nie, und Sie bleiben ewig hängen - ein Phänomen, das man \enquote{Geschwindigkeitsbarriere} nennt.
Die meisten Anfänger unter den Klavierschülern haben nicht die geringste Vorstellung von den komplexen Bewegungen, die die Finger, Hände und Arme ausführen können.
Zum Glück haben die vielen Genies vor uns die meisten nützlichen Entdeckungen bereits gemacht (sonst wären sie keine so großen Künstler gewesen), was zu effizienten Übungsmethoden führt.

Eine weitere falsche Vorstellung von der Technik ist, dass man, wenn die Finger erst einmal genügend geschickt sind, alles spielen kann.
Fast jede einzelne Passage, die sich von den anderen unterscheidet, ist ein neues Abenteuer; sie muss neu gelernt werden.
Erfahrene Pianisten sind \textit{scheinbar} in der Lage, fast alles zu spielen, weil

\begin{enumerate}[label={\arabic*.}] 
\item sie fast alles geübt haben, das man oft vorfindet, und
\item sie wissen, wie man Neues sehr schnell lernt.
 \end{enumerate}
Es gibt große Klassen von Passagen, wie zum Beispiel Tonleitern, die häufig auftreten.
Das Wissen, wie diese zu spielen sind, wird bedeutende Teile der meisten Kompositionen abdecken.
Wichtiger ist jedoch, dass es allgemeine Lösungen für große Problemklassen und spezielle Lösungen für spezielle Probleme gibt.


\section{Technik, Musik und mentales Spielen}\hypertarget{c1i3}{}

Wenn wir uns nur auf die Entwicklung der \enquote{Fingertechnik} konzentrieren und die Musik während des Übens vernachlässigen, können wir unmusikalische Spielgewohnheiten annehmen.
\textbf{Unmusikalisches Spielen ist stets absolut verboten, weil es ein Fehler ist.}
Ein verbreitetes Symptom dieses Fehlers ist die Unfähigkeit, die Übungsstücke zu spielen, wenn der Lehrer (oder sonst jemand!) zuhört.
Wenn Publikum dabei ist, machen diese Schüler seltsame Fehler, die sie während des \enquote{Übens} nicht gemacht haben.
Das geschieht, weil die Schüler ohne Beachtung der Musik geübt hatten und plötzlich erkennen, dass sie nun die Musik hinzufügen müssen, weil jemand zuhört.
Leider haben sie bis zur Unterrichtsstunde niemals wirklich musikalisch geübt!
Ein weiteres Symptom unmusikalischen Übens ist, dass die Schüler sich unwohl fühlen, wenn andere sie beim Üben hören können.
\textbf{Klavierlehrer wissen, dass Schüler musikalisch üben müssen, um sich die Technik anzueignen.
Was für die Ohren und das Gehirn richtig ist, stellt sich als für den menschlichen Spielapparat richtig heraus.}
Sowohl Musikalität als auch Technik benötigen Genauigkeit und Kontrolle.
Praktisch jeder technische Makel kann in der Musik wahrgenommen werden.
Die Musik ist die schwierigste Probe, ob die Technik richtig oder falsch ist.
Wie wir das ganze Buch hindurch sehen werden, gibt es mehr Gründe, warum Musik niemals von der Technik getrennt werden sollte.
Nichtsdestoweniger neigen viele Schüler dazu, beim Üben die Musik zu vernachlässigen und ziehen es vor, zu \enquote{arbeiten}, wenn niemand dabei ist, der zuhört.
Solche Übungsmethoden erzeugen \enquote{Stille-Kämmerlein-Pianisten}, die gerne spielen aber nicht vorspielen können.
\textbf{Wenn Schülern beigebracht wird, immer musikalisch zu üben, dann wird diese Art von Problem gar nicht existieren; vorspielen und üben sind ein und dasselbe.}
Dieses Buch enthält viele Vorschläge für das Üben des \hyperlink{c1iii14}{Auftretens}, wie zum Beispiel sein Spielen von Anfang an \hyperlink{c1iii13}{auf Video aufzunehmen}.

\textbf{Viele Schüler denken zu Unrecht, dass die Finger die Musik kontrollieren, und sie warten darauf, dass das Klavier diesen großartigen Sound erzeugt.}
Das wird zu einer eintönigen Vorführung und unvorhersehbaren Ergebnissen führen.
Die Musik muss aus dem Geist kommen, und der Klavierspieler muss das Klavier dazu bringen, das zu erzeugen, was er möchte.
Das ist das oben vorgestellte \hyperlink{c1ii12mental}{mentale Spielen};
wenn Sie das mentale Spielen noch nie geübt haben, werden Sie feststellen, dass es eine Stufe des Auswendiglernens erfordert, die Sie noch nie erreicht haben - aber \textit{genau} das brauchen Sie für einen fehlerfreien, respekteinflößenden Auftritt.
Zum Glück sind es nur ein paar Schritte von den in diesem Buch geschilderten Verfahren für das \hyperlink{c1iii6}{Auswendiglernen} zum mentalen Spielen, aber es bedeutet einen großen Fortschritt in Ihren musikalischen Fertigkeiten, nicht nur für die Technik und das Musikmachen, sondern auch für das Lernen eines \hyperlink{c1iii12}{absoluten Gehörs}, das Komponieren und jeden Aspekt des Klavierspielens.
So sind Technik, Musik und mentales Spielen untrennbar miteinander verflochten.
Sobald Sie sich mit dem mentalen Spielen intensiv beschäftigen, werden Sie entdecken, dass es ohne das absolute Gehör nicht wirklich funktioniert.
Diese Erörterungen bieten eine solide Grundlage für das Identifizieren der Fertigkeiten, die wir lernen müssen.
Dieses Buch liefert die Übungsmethoden, die man braucht, um sie zu lernen.


\section{Generelles Vorgehen, Interpretation, Musikunterricht, Absolutes Gehör}\hypertarget{c1i4}{}

Die Lehrer spielen eine wichtige Rolle dabei, den Schülern zu zeigen, wie man musikalisch spielt und übt.
Zum Beispiel beginnen und enden die meisten Musikstücke mit demselben Akkord, eine etwas geheimnisvolle Regel, die eigentlich aus den Grundregeln für Akkordprogressionen resultiert.
Ein Verständnis der Akkordprogressionen ist für das \hyperlink{c1iii6}{Auswendiglernen} sehr nützlich.
Eine musikalische Phrase beginnt und endet im Allgemeinen mit leiseren Noten, mit lauteren Noten dazwischen; wenn Sie im Zweifel sind, dann ist dies eine gute Grundregel.
Das ist vielleicht ein Grund, warum so viele Kompositionen mit einem unvollständigen Takt beginnen - der erste Schlag trägt in der Regel den Akzent und ist zu laut.
Es gibt viele Bücher, die sich mit der musikalischen Interpretation beschäftigen (\hyperlink{Gieseking}{Gieseking}, \hyperlink{Sandor}{Sandor}), und es gibt zahlreiche Beispiele in diesem Buch.

Ein musikalisches Training lohnt sich in jüngsten Jahren am meisten.
Die meisten Babys, die häufig ein perfekt gestimmtes Klavier hören, entwickeln \textit{automatisch} ein \hyperlink{c1iii12}{absolutes Gehör} - das ist nichts Außergewöhnliches.
Niemand wird mit einem absoluten Gehör geboren, da es zu 100 Prozent eine erlernte Fertigkeit ist (die exakten Frequenzen der Tonleitern sind willkürliche menschliche Konventionen - es gibt kein Naturgesetz, das besagt, dass das mittlere A bei 440 Hz sein muss; die meisten Orchester stimmen auf 442 Hz, und bevor das A standardisiert wurde, gab es einen viel größeren Bereich erlaubter Frequenzen).
Wenn dieses absolute Gehör nicht gepflegt wird, dann wird es im späteren Leben verloren gehen.
\textbf{Klavierunterricht für junge Kinder kann im Alter von ungefähr drei bis vier Jahren beginnen.
Es ist vorteilhaft, wenn Jüngere früh (ab der Geburt) klassische Musik hören, weil klassische Musik den höchsten musikalischen Gehalt (komplex, logisch) aller verschiedenen Arten von Musik hat.}
Einige Formen der zeitgenössischen Musik könnten durch das Überbetonen bestimmter beschränkter Aspekte - wie Lautstärke oder zu einfache Musikstrukturen, die das Gehirn nicht stimulieren - die musikalische Entwicklung eeinträchtigen, indem sie die Entwicklung des Gehirns stören.

Obwohl man musikalisch begabt sein muss, um Musik zu komponieren, ist die Fähigkeit, Klavier zu spielen, nicht so vom musikalischen Verstand abhängig.
In Wahrheit sind die meisten von uns musikalischer als wir uns selbst zutrauen, und es ist der Mangel an Technik, der unsere musikalische Ausdrucksfähigkeit am Klavier einschränkt.
Wir haben bereits alle die Erfahrung gemacht, berühmten Pianisten zuzuhören und zu erkennen, dass sie sich voneinander unterscheiden - das ist mehr musikalische Sensibilität als wir jemals benötigen, um mit dem Klavierspielen zu beginnen.
Man muss nicht acht Stunden täglich üben; einige berühmte Pianisten haben Übungszeiten von weniger als einer Stunde empfohlen.
Sie können Fortschritte machen, wenn Sie drei- oder viermal die Woche für jeweils eine Stunde üben.

Schließlich sollte eine umfassende musikalische Ausbildung (\hyperlink{c1iii5a}{Tonleitern}, Taktarten, Hörschule - einschließlich absolutem Gehör -, Diktate, Theorie usw.) ein integraler Bestandteil davon sein, das Klavierspielen zu lernen, weil alle Teile, die man lernt, für die anderen Teile hilfreich sind.
Letzten Endes ist eine umfassende musikalische Ausbildung der einzige Weg, das Klavierspielen zu lernen.
Leider stehen den meisten angehenden Klavierspielern nicht die Mittel oder die Zeit zur Verfügung, um diesen Weg zu verfolgen.
Dieses Buch ist dazu gedacht, dem Schüler eine Ausgangsbasis zu geben, indem er lernt, wie man sich die Technik schnell aneignet, sodass er sich überlegen kann, alle die anderen nützlichen Themen zu studieren.
\textbf{In der Regel komponieren Schüler, die glänzende Klavierspieler sind, am Ende fast immer ihre eigene Musik.}
Das Studium der Kompositionslehre ist keine Voraussetzung für das Komponieren.
Einige Lehrer halten nicht viel davon, zu viel Kompositionstheorie zu lernen, bevor man mit dem Komponieren seiner eigenen Musik beginnt, weil einen das davon abhalten kann, seinen eigenen Stil zu entwickeln.

Was sind einige der herausragenden Merkmale der Methoden dieses Buchs?

\begin{enumerate}[label={\arabic*.}] 
\item Diese Methoden sind nicht so übermäßig anstrengend wie ältere Methoden, die den Schülern für den Klavierunterricht einen hingebungsvollen Lebensstil abverlangen.
Die Schüler erhalten die Möglichkeit, sich eine bestimmte Prozedur auszusuchen, mit der man ein definiertes Ziel innerhalb einer abschätzbaren Zeitspanne erreichen kann.
Wenn die Methoden \textit{wirklich} funktionieren, sollten sie kein lebenslanges blindes Vertrauen erfordern, um Können zu erlangen!

\item Jede Prozedur dieser Methoden hat eine körperliche Grundlage (wenn sie funktioniert, hat sie immer eine; die früheren Probleme der Klavierpädagogik lagen im Finden der richtigen Erklärungen); sie muss außerdem die folgenden erforderlichen Elemente enthalten:

\begin{enumerate}[label={\alph*.}] 
\item \textbf{Ziel:} Techniken, die erworben werden sollen, das heißt wenn Sie nicht schnell genug oder keine Triller spielen können, wenn Sie auswendig spielen möchten, usw.

\item \textbf{Dann ist zu tun:} das heißt \hyperlink{c1ii7}{mit getrennten Händen üben}, den \hyperlink{c1ii9}{Akkord-Anschlag} benutzen, während des Übens auswendig lernen usw.

\item \textbf{Weil:} die physiologischen, psychologischen, mechanischen usw. Erklärungen dafür, warum diese Methoden funktionieren; zum Beispiel vereinfacht mit getrennten Händen zu üben schwierige Passagen.

\item \textbf{Wenn nicht:} Probleme, die entstehen, wenn auf Unkenntnis beruhende Methoden benutzt werden.
Ohne dieses \enquote{Wenn nicht} können die Schüler jede andere Methode wählen - warum also diese?
Wir müssen wissen, was wir nicht tun dürfen, denn schlechte Angewohnheiten und falsche Methoden, nicht ungenügende Übung, sind die Hauptgründe für einen Mangel an Fortschritt.

 \end{enumerate}

\item 
Dieses Buch bietet einen vollständigen, gegliederten Satz an Lernwerkzeugen, die Sie mit einem Minimum an Aufwand in das Wunderland des \hyperlink{c1ii12mental}{mentalen Spielens} bringen.
Gute Reise!

 \end{enumerate}


