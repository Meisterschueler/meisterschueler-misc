% File: contents

\hypertarget{Inhalt}{}

<h2 align=\enquote{center}>Inhaltsverzeichnis</h2>

Das Buch ist in folgenden Sprachen verfügbar:
 \hyperref[http://www.pianopractice.org]{Englisch} (extern; das Original),
 \hyperref[http://bbs.popiano.org/viewthread.php?tid=81448&amp;extra=page\%3D3]{Einfaches Chinesisch} (extern),
 \hyperref[http://www.pianogarden.tw]{Traditionelles Chinesisch} (extern),
 \hyperref[./index.html]{Deutsch},
 \hyperref[http://pagesperso-orange.fr/musico/documents/textes/pianopratique/tabledesmatieres_fr.htm]{Französisch} (extern),
 \hyperref[http://web.tiscali.it/pianobook]{Italienisch} (extern),
 \hyperref[http://pianofundamental.sakura.ne.jp]{Japanisch} (extern),
 \hyperref[http://pianoart.eu.interia.pl]{Polnisch} (extern),
 \hyperref[http://www.pianopractice.org/spanish.pdf]{Spanisch} (extern)
 .

Ich suche Freiwillige, die das Buch in jede andere Sprache übersetzen - siehe \hyperlink{HinUeber}{Hinweise für Übersetzer} am Ende dieses Buchs.
Senden Sie bitte eine E-Mail an \hyperref[mailto:cc88m@aol.com?subject=foppde:\%20Translation\%20request]{cc88m@aol.com}, um die Einzelheiten zu besprechen.

\footnote{Wenn diese Website neu für Sie ist, sollten Sie zunächst die ersten beiden Abschnitte der \hyperlink{anmerkungen}{Anmerkungen} lesen.}


<h2><br>\hyperlink{preface}{Vorwort} \textbf{\textit{[09.08.2009]}}</h2>

<h2><br>\hyperlink{c1i1}{Kapitel 1: Klaviertechnik}</h2>

\chapter{Einführung \textbf{\textit{[15.08.2009]}}}

\begin{enumerate} 
 \item \hyperlink{c1i1}{Zweck dieses Buchs}
 \item \hyperlink{c1i2}{Was ist Klaviertechnik?}
 \item \hyperlink{c1i3}{Technik, Musik und mentales Spielen}
 \item \hyperlink{c1i4}{Generelles Vorgehen, Interpretation, Musikunterricht, Absolutes Gehör}
 \end{enumerate}

\chapter{Grundlegende Verfahren des Klavierübens}

\begin{enumerate} 
 \item \hyperlink{c1ii1}{Der Übungsablauf} \textbf{\textit{[22.08.2009]}}
 \item \hyperlink{c1ii2}{Position der Finger}
 \item \hyperlink{c1ii3}{Höhe der Sitzbank und ihr Abstand zum Klavier}
 \item \hyperlink{c1ii4}{Ein neues Stück - Anhören und analysieren (\enquote{Für Elise})}
 \item \hyperlink{c1ii5}{Die schwierigen Abschnitte zuerst üben}  \textbf{\textit{[05.09.2009]}}
 \item \hyperlink{c1ii6}{Schwierige Passagen kürzen - In kleinen Portionen üben (taktweise)}
 \item \hyperlink{c1ii7}{Die Hände getrennt (einhändig, HS) üben - Erlernen der Spieltechnik}
 \item \hyperlink{c1ii8}{Die Kontinuitätsregel}
 \item \hyperlink{c1ii9}{Der Akkord-Anschlag}
 \item \hyperlink{c1ii10}{Freier Fall, Akkord-Übung und Entspannung}
 \item \hyperlink{c1ii11}{Parallele Sets}
 \item \hyperlink{c1ii12}{Lernen, Auswendiglernen und mentales Spielen} \textbf{\textit{[20.09.2009]}}
 \item \hyperlink{c1ii13}{Spielgeschwindigkeit beim Üben}
 \item \hyperlink{c1ii14}{Wie man entspannt}
 \item \hyperlink{c1ii15}{Automatische Verbesserung nach dem Üben (PPI)}
 \item \hyperlink{c1ii16}{Gefahren des langsamen Spielens - Fallstricke der \enquote{Intuitiven Methode}}
 \item \hyperlink{c1ii17}{Die Wichtigkeit des langsamen Spielens} \textbf{\textit{[04.10.2009]}}

 \item \hyperlink{c1ii18}{Fingersatz}
 \item \hyperlink{c1ii19}{Akkurates Tempo und das Metronom}
 \item \hyperlink{c1ii20}{Die schwache linke Hand - Eine Hand unterrichtet die andere}
 \item \hyperlink{c1ii21}{Ausdauer aufbauen, Atmung}

 \item \hyperlink{c1ii22}{Schlechte Angewohnheiten: Der größte Feind des Pianisten} \textbf{\textit{[31.10.2009]}}
 \item \hyperlink{c1ii23}{Haltepedal}\footnote{Anmerkungen zu den Bezeichnungen der Pedale finden Sie \hyperlink{Pedale}{hier}}
 
 \item \hyperlink{c1ii24}{Dämpferpedal, Timbre und Eigenschwingungen vibrierender Saiten}
 \item \hyperlink{c1ii25}{Mit beiden Händen zusammen (HT) üben und mental spielen} \textbf{\textit{[05.12.2009]}}
  \begin{enumerate}[label={\alph*.}] 
   <li>\hyperlink{c1ii25a}{Einführung}
   \item \hyperlink{c1ii25b}{Beethovens Mondschein-Sonate, 1. Satz, Op. 27, No. 2}
   \item \hyperlink{c1ii25c}{Mozarts Rondo Alla Turca, aus Sonate K300 (KV331)}
   \item \hyperlink{c1ii25d}{Chopins Fantaisie-Impromptu, Op. 66}
   \end{enumerate}
 </li>
 \item \hyperlink{c1ii26}{Zusammenfassung}
 \end{enumerate}

\chapter{Ausgewählte Themen des Klavierübens}

\begin{enumerate} 
 \item \hyperlink{c1iii1}{Klang, Rhythmus, Legato, Staccato} \textbf{\textit{[14.02.2010]}}
  \begin{enumerate}[label={\alph*.}] 
   <li>\hyperlink{c1iii1a}{Was ist ein \enquote{Guter Klang}?}
   <ul type=\enquote{circle}>
    <li>\hyperlink{c1iii1a1}{Der Basisanschlag}
    \item \hyperlink{c1iii1a2}{Klang: Einzelne gegenüber mehreren Noten, Pianissimo, Fortissimo}
   </ul>
   </li>
   \item \hyperlink{c1iii1b}{Was ist Rhythmus? (Beethovens Sturm-Sonate und Appassionata)}
   \item \hyperlink{c1iii1c}{Legato, Staccato}
   \end{enumerate}
 </li>
 \item \hyperlink{c1iii2}{Zyklisch spielen (Chopins Fantaisie Impromptu, Op. 66)} \textbf{\textit{[21.03.2010]}}

 \item \hyperlink{c1iii3}{Triller und Tremolos (Beethovens Pathétique, 1. Satz)}
  <ol type=\enquote{a}>
   <li>\hyperlink{c1iii3}{Triller}
   \item \hyperlink{c1iii3b}{Tremolos (Beethovens Pathétique, 1. Satz)}
   \end{enumerate}
 </li>
 \item \hyperlink{c1iii4}{Bewegungen der Hand, der Finger und des Körpers} \textbf{\textit{[24.10.2010]}}
  \begin{enumerate}[label={\alph*.}] 
   <li>\hyperlink{c1iii4}{Bewegungen der Hand}
   \item \hyperlink{c1iii4b}{Mit flachen (gestreckten) Fingern spielen}
   \item \hyperlink{c1iii4c}{Bewegungen des Körpers}
   \end{enumerate}
 </li>
 \item \hyperlink{c1iii5}{Schnell spielen: Tonleitern, Arpeggios und chromatische Tonleitern (Chopins Fantaisie Impromptu und Beethovens Mondschein-Sonate, 3. Satz)} \textbf{\textit{[21.08.2011]}}
  \begin{enumerate}[label={\alph*.}] 
   <li>\hyperlink{c1iii5a}{Tonleitern: Daumenuntersatz, Daumenübersatz}
   \item \hyperlink{c1iii5b}{Daumenübersatz: Bewegung, Erklärung und Video}
   \item \hyperlink{c1iii5c}{Daumenübersatz üben: Geschwindigkeit, Glissandobewegung}
   \item \hyperlink{c1iii5d}{Tonleitern: Herkunft, Namensgebung, Fingersätze}
    <ul type=\enquote{circle}>
     <li>\hyperlink{table}{Fingersatztabelle}
    </ul>
   </li>
   \item \hyperlink{c1iii5e}{Arpeggios (Chopin, Wagenradbewegung, \enquote{gespreizte} Finger)}
   \item \hyperlink{c1iii5f}{Schub und Zug, Beethovens Mondschein-Sonate, 3. Satz}
   \item \hyperlink{c1iii5g}{Der Daumen: Der vielseitigste Finger}
   \item \hyperlink{c1iii5h}{Schnelle chromatische Tonleitern}
   \end{enumerate}<br><br>[Ab hier wird der Text noch überarbeitet.]<br><br>
 </li>
 \item \hyperlink{c1iii6}{Auswendiglernen}
  \begin{enumerate}[label={\alph*.}] 
   <li>\hyperlink{c1iii6a}{Warum auswendig lernen?}
   \item \hyperlink{c1iii6b}{Wer kann auswendig lernen, was und wann?}
   \item \hyperlink{c1iii6c}{Auswendiglernen und Pflege des Gelernten}
   \item \hyperlink{c1iii6d}{Hand-Gedächtnis}
   \item \hyperlink{c1iii6e}{Wie fängt man an?}
   \item \hyperlink{c1iii6f}{Auffrischung des Gedächtnisses}
   \item \hyperlink{c1iii6g}{Kaltstart}
   \item \hyperlink{c1iii6h}{Langsam spielen}
   \item \hyperlink{c1iii6i}{Vorausschauend spielen}
   \item \hyperlink{c1iii6j}{Langzeitgedächtnis aufbauen}
    <ul type=\enquote{circle}>
     <li>\hyperlink{c1iii6hand}{Hand-Gedächtnis}
     \item \hyperlink{c1iii6musik}{Musik-Gedächtnis}
     \item \hyperlink{c1iii6foto}{Fotografisches Gedächtnis}
     \item \hyperlink{c1iii6tastatur}{Tastatur-Gedächtnis}
     \item \hyperlink{c1iii6theorie}{Theoretisches Gedächtnis}
    </ul>
   </li>
   \item \hyperlink{c1iii6k}{Pflege}
   \item \hyperlink{c1iii6l}{Blattspieler und Auswendiglernende (Bachs Inventionen)}
    <ul type=\enquote{circle}>
     <li>\hyperlink{c1iii6l}{Blattspieler und Auswendiglernende}
     \item \hyperlink{c1iii6l2}{Bachs Inventionen}
     \item \hyperlink{ruhig}{Ruhige Hände}
    </ul>
   </li>
   \item \hyperlink{c1iii6m}{Funktion des menschlichen Gedächtnisses}
   \item \hyperlink{c1iii6n}{Ein guter Auswendiglernender werden}
   \item \hyperlink{c1iii6o}{Zusammenfassung}
   \end{enumerate}
 </li>
 \item \hyperlink{c1iii7}{Übungen}
  \begin{enumerate}[label={\alph*.}] 
   <li>\hyperlink{c1iii7a}{Einführung}
    <ul type=\enquote{circle}>
      <li>\hyperlink{c1iii7aMuskeln}{Schnelle und langsame Muskeln}
    </ul>
   </li> 
   \item \hyperlink{c1iii7b}{Parallele Sets}
   \item \hyperlink{c1iii7c}{Wie verwendet man die Übungen für parallele Sets (Appassionata, 3. Satz)?}
   \item \hyperlink{c1iii7d}{Tonleitern, Arpeggios, Unabhängigkeit der Finger und Anheben der Finger}
   \item \hyperlink{c1iii7e}{(Große) Akkorde spielen, Dehnung der Handflächen}
   \item \hyperlink{c1iii7f}{Sprünge}
   \item \hyperlink{c1iii7g}{Weitere Übungen}
   \item \hyperlink{c1iii7h}{Probleme mit Hanons Übungen}
   \item \hyperlink{c1iii7i}{Die Geschwindigkeit steigern}
    <ul type=\enquote{circle}>
     <li>\hyperlink{c1iii7iAnschlag}{Schneller Anschlag, Entspannung}
     \item \hyperlink{c1iii7iAndere}{Andere Geschwindigkeitsmethoden}
     \item \hyperlink{c1iii7iMusik}{Geschwindigkeit und Musik}
    </ul>
   </li>
   \end{enumerate}
 </li>
 \item \hyperlink{c1iii8}{Konturieren (Beethovens Sonate \#1)}
 \item \hyperlink{c1iii9}{Ein Stück auf Hochglanz bringen - Fehler beseitigen}
 \item \hyperlink{c1iii10}{Kalte Hände, rutschende Finger, Krankheiten, Handverletzungen, Gehörschäden}
  \begin{enumerate}[label={\alph*.}] 
   <li>\hyperlink{c1iii10}{Kalte Hände}
   \item \hyperlink{c1iii10rutschen}{Rutschende (trockene oder schwitzende) Finger}
   \item \hyperlink{c1iii10krank}{Krankheiten}
   \item \hyperlink{c1iii10ungesund}{Gesundes und ungesundes Üben}
   \item \hyperlink{c1iii10hand}{Verletzungen der Hand}
   \item \hyperlink{c1iii10gehoer}{Gehörschäden}
   \end{enumerate}
 </li>
 \item \hyperlink{c1iii11}{Blattspiel}
 \item \hyperlink{c1iii12}{Absolutes Gehör und relatives Gehör (vom Blatt singen)}
   <ul type=\enquote{circle}>
    <li>\hyperlink{c1iii12tonhoehe}{Verfahren zum Lernen der relativen und absoluten Tonhöhenerkennung}
    \item \hyperlink{c1iii12blatt}{Vom Blatt singen und komponieren}
   </ul>
 </li>
 \item \hyperlink{c1iii13}{Filmen und Aufnehmen des eigenen Spielens,\footnote{MIDI, Digitalpianos, Keyboards usw.}}
     <ul type=\enquote{circle}>
      <li>\hyperlink{c1iii13MIDI}{\footnote{MIDI, Digitalpianos, Keyboards usw.}}
    </ul>
 </li>
 \item \hyperlink{c1iii14}{Vorbereitung auf Auftritte und Konzerte}
  \begin{enumerate}[label={\alph*.}] 
   <li>\hyperlink{c1iii14a}{Nutzen und Risiken von Auftritten und Konzerten}
   \item \hyperlink{c1iii14b}{Grundlagen fehlerfreien Vorspielens}
   \item \hyperlink{c1iii14c}{Für Auftritte üben}
   \item \hyperlink{c1iii14d}{Musikalisch üben}
   \item \hyperlink{c1iii14e}{Zwangloses Vorspielen}
   \item \hyperlink{c1iii14f}{Vorbereitung auf Konzerte}
   \item \hyperlink{c1iii14g}{Während des Konzerts}
   \item \hyperlink{c1iii14h}{Das ungewohnte Klavier}
   \item \hyperlink{c1iii14i}{Nach dem Konzert}
   \end{enumerate}
 </li>
 \item \hyperlink{c1iii15}{Ursachen und Kontrolle von Nervosität}
 \item \hyperlink{c1iii16}{Unterrichten}
  \begin{enumerate}[label={\alph*.}] 
   <li>\hyperlink{c1iii16a}{Lehrer}
   \item \hyperlink{c1iii16b}{Kinder unterrichten, Eltern einbeziehen}
   \item \hyperlink{c1iii16c}{Blattspiel, Auswendiglernen, Theorie}
   \item \hyperlink{c1iii16d}{Einige Elemente des Klavierunterrichts}
   \item \hyperlink{c1iii16e}{Warum die größten Pianisten nicht unterrichten konnten}
   \end{enumerate}
 </li>
 \item \hyperlink{c1iii17}{Klaviere, Flügel und elektronische Klaviere; Kauf und Wartung}
  \begin{enumerate}[label={\alph*.}] 
   <li>\hyperlink{c1iii17a}{Flügel, akustisches oder elektronisches Klavier?}
   \item \hyperlink{c1iii17b}{Elektronische Klaviere}
   \item \hyperlink{c1iii17c}{Klaviere}
   \item \hyperlink{c1iii17d}{Flügel}
   \item \hyperlink{c1iii17e}{Ein akustisches Klavier kaufen}
   \item \hyperlink{c1iii17f}{Pflege und Wartung des Klaviers}
   \item \hyperlink{c1iii17g}{\footnote{Anmerkungen zu Digitalpianos}}
   \end{enumerate}
 </li>
 \item \hyperlink{c1iii18}{Wie man das Klavierspielenlernen beginnt - vom jüngsten Kind bis zum ältesten Erwachsenen}
     \begin{enumerate}[label={\alph*.}] 
        <li>\hyperlink{c1iii18a}{Benötigt man einen Lehrer?}
        \item \hyperlink{c1iii18b}{Bücher für Anfänger; Keyboards}
        \item \hyperlink{c1iii18c}{Anfänger im Alter von 0 bis über 65}
            <ul type=\enquote{circle}>
               <li>\hyperlink{c1iii18c0}{Von 0 bis 6}
               \item \hyperlink{c1iii18c3}{Von 3 bis 12}
               \item \hyperlink{c1iii18c13}{Von 13 bis 19}
               \item \hyperlink{c1iii18c20}{Von 20 bis 35}
               \item \hyperlink{c1iii18c35}{Von 35 bis 45}
               \item \hyperlink{c1iii18c45}{Von 45 bis 65}
               \item \hyperlink{c1iii18c65}{Über 65}
            </ul>
        </li>
      \end{enumerate}
 </li>
 \item \hyperlink{c1iii19}{Der \enquote{ideale} Übungsablauf (Bachs Invention \#4)}
     \begin{enumerate}[label={\alph*.}] 
        <li>\hyperlink{c1iii19a}{Die Regeln lernen}
        \item \hyperlink{c1iii19b}{Ein neues Stück lernen (Invention \#4)}
        \item \hyperlink{c1iii19c}{\enquote{Normale} Übungsabläufe und Bachs Lehren}
      \end{enumerate}
 </li>
 \item \hyperlink{c1iii20}{Bach: der größte Komponist und Lehrer (15 Inventionen)}
   <ul type=\enquote{circle}>
    <li>\hyperlink{c1iii20ps}{Liste der parallelen Sets in den einzelnen Inventionen (für die RH)}
   </ul>
 </li>
 \item \hyperlink{c1iii21}{Klavierspielen und die Psychologie}
 \item \hyperlink{c1iii22}{Zusammenfassung der Methoden}

</ol>

\chapter{Mathematische Theorien des Klavierspielens}

\begin{enumerate} 
 \item \hyperlink{c1iv1}{Wozu braucht man mathematische Theorien?}
 \item \hyperlink{c1iv2}{Die Theorie der Fingerbewegung}
   \begin{enumerate}[label={\alph*.}] 
    <li>\hyperlink{c1iv2a}{Serielles und paralleles Spielen}
    \item \hyperlink{c1iv2b}{Geschwindigkeitsbarrieren}
    \item \hyperlink{c1iv2c}{Die Geschwindigkeit steigern}
    \end{enumerate}
 </li>
 \item \hyperlink{c1iv3}{Die Thermodynamik des Klavierspielens}
 \item \hyperlink{c1iv4}{Mozarts Formel, Beethoven und Gruppentheorie}
 \item \hyperlink{c1iv5}{Berechnen der Lernrate}
 \item \hyperlink{c1iv6}{Noch zu erforschende Themen}
   <ol type=\enquote{a}>
    <li>\hyperlink{c1iv6a}{Impulstheorie des Klavierspielens}
    \item \hyperlink{c1iv6b}{Die Physiologie der Technik}
    \item \hyperlink{c1iv6c}{Gerhirnforschung (HS- und HT-Spielen usw.)}
    \item \hyperlink{c1iv6d}{Was verursacht Nervosität?}
    \item \hyperlink{c1iv6e}{Ursachen von und Mittel gegen Tinnitus}
    \item \hyperlink{c1iv6f}{Was ist Musik?}
    \item \hyperlink{c1iv6g}{In welchem Alter soll bzw. darf man mit dem Klavierspielen anfangen?}
    \item \hyperlink{c1iv6h}{Die Zukunft des Klavierspielens}
    \item \hyperlink{c1iv6i}{Die Zukunft des Unterrichts}
    \end{enumerate}
 </li>
</ol>

<h2><br>Kapitel 2: Stimmen des Klaviers</h2> 

\begin{enumerate} 
 \item \hyperlink{c2_1}{Einleitung}
 \item \hyperlink{c2_2}{Chromatische Tonleiter und Temperaturen}
  \begin{enumerate}[label={\alph*.}] 
   <li>\hyperlink{c2_2a}{Einleitung}
   \item \hyperlink{c2_2b}{Mathematische Behandlung}
   \item \hyperlink{c2_2c}{Temperatur und Musik}
   \end{enumerate}
 </li>
 \item \hyperlink{c2_3}{Werkzeuge zum Stimmen}
 \item \hyperlink{c2_4}{Vorbereitung}
 \item \hyperlink{c2_5}{Wie man anfängt}
  <ol type=\enquote{a}>
   <li>\hyperlink{c2_5a}{Einleitung}
   \item \hyperlink{c2_5_hamm}{Einsetzen und Bewegen des Stimmhammers}
   \item \hyperlink{c2_5_wirb}{Den Wirbel einstellen}
   \item \hyperlink{c2_5_unis}{Unisono stimmen}
   \item \hyperlink{c2_5_mits}{Mitschwingung}
   \item \hyperlink{c2_5_infi}{Diese letzte infinitesimale Bewegung ausführen}
   \item \hyperlink{c2_5_span}{Ausgleich der Saitenspannung}
   \item \hyperlink{c2_5_disk}{Wiegen im Diskant}
   \item \hyperlink{c2_5_bass}{Grollen im Bass}
   \item \hyperlink{c2_5_harm}{Harmonisches Stimmen}
   \item \hyperlink{c2_5_stre}{Was ist Streckung?}
   \item \hyperlink{c2_5_prae}{Präzision, Präzision, Präzision}
   \end{enumerate}
 </li>
 \item \hyperlink{c2_6}{Stimmverfahren}
  \begin{enumerate}[label={\alph*.}] 
   <li>\hyperlink{c2_6a}{Einleitung}
   \item \hyperlink{c2_6_gabe}{Das Klavier nach der Stimmgabel stimmen}
   \item \hyperlink{c2_6_kirn}{Kirnberger II}
   \item \hyperlink{c2_6_et}{Gleichschwebende Temperatur (gleichstufige Temperatur, gleichmäßige Temperatur)}
   \end{enumerate}
 </li>
 \item \hyperlink{c2_7}{Kleinere Reparaturen durchführen}
  \begin{enumerate}[label={\alph*.}] 
   <li>\hyperlink{c2_7_hamm}{Intonieren der Hämmer}
   \item \hyperlink{c2_7_pilo}{Polieren der Piloten}
   \end{enumerate}
 </li>
</ol> 

<h2><br>\hyperlink{c3_1}{Kapitel 3: Wissenschaftliche Methode, Theorie des Lernens, Das Gehirn}</h2>

\footnote{Abschnitt 4 ist im Original zurzeit (26.5.2003) noch \enquote{preliminary draft} also ein \enquote{Rohentwurf}.}


\begin{enumerate} 
 \item \hyperlink{c3_1}{Einleitung}
 \item \hyperlink{c3_2}{Der wissenschaftliche Ansatz}
  \begin{enumerate}[label={\alph*.}] 
   <li>\hyperlink{c3_2a}{Einleitung}
   \item \hyperlink{c3_2b}{Lernen}
   \end{enumerate}
 </li>
 \item \hyperlink{c3_3}{Was ist die Wissenschaftliche Methode?}
  <ol type=\enquote{a}>
   <li>\hyperlink{c3_3a}{Einleitung}
   \item \hyperlink{c3_3b}{Definition}
   \item \hyperlink{c3_3c}{Forschung}
   \item \hyperlink{c3_3d}{Dokumentation und Kommunikation}
   \item \hyperlink{c3_3e}{Konsistenzprüfungen}
   \item \hyperlink{c3_3f}{Grundlegende Theorie}
   \item \hyperlink{c3_3g}{Dogma und Lehre}
   \item \hyperlink{c3_3h}{Schlussfolgerungen}
   \end{enumerate}
 </li>
 \item \hyperlink{c3_4}{Theorie des Lernens}
 \item \hyperlink{c3_5}{Was Träume erzeugt und Methoden zu ihrer Kontrolle}
  \begin{enumerate}[label={\alph*.}] 
   <li>\hyperlink{c3_5a}{Einleitung}
   \item \hyperlink{c3_5b}{Der Fall-Traum}
   \item \hyperlink{c3_5c}{Der Unfähig-zu-laufen-Traum}
   \item \hyperlink{c3_5d}{Der Zu-spät-zur-Prüfung-kommen- oder Sich-verlaufen-Traum}
   \item \hyperlink{c3_5e}{Die Lösung für meinen langen und komplexen Traum}
   \item \hyperlink{c3_5f}{Die Kontrolle der Träume}
   \item \hyperlink{c3_5g}{Was uns diese Träume über das Gehirn lehren}
   \end{enumerate}
</li>
 \item \hyperlink{c3_6}{Das Unterbewusstsein}
  \begin{enumerate}[label={\alph*.}] 
   <li>\hyperlink{c3_6a}{Einleitung}
   \item \hyperlink{c3_6b}{Emotionen}
   \item \hyperlink{c3_6c}{Das Unterbewusstsein benutzen}
   \end{enumerate}
</li>
</ol>

<h2><br>\hyperlink{reference}{Quellenverzeichnis}</h2>

<h3>\underline{\hyperlink{reference}{Buchbesprechungen}}</h3>
\begin{itemize} 
 \item \hyperlink{allgemein}{Allgemeine Schlussfolgerungen}
 \item \hyperlink{Bree}{Bree, Malwine: \textit{The Leschetizky Method}}
 \item \hyperlink{Bruser}{Bruser, Madeline: \textit{The Art of Practicing}}
 \item \hyperlink{Chang}{Chang, Chuan C.: \textit{Fundamentals of Piano Practice}, erste Ausgabe}
 \item \hyperlink{Eigeldinger}{Eigeldinger, Jean-Jacques: \textit{Chopin, pianist and teacher as seen by his pupils}}
 \item \hyperlink{Fink}{Fink, Seymour: \textit{Mastering Piano Technique}}
 \item \hyperlink{Gieseking}{Gieseking, Walter und Leimer, Karl: \textit{Modernes Klavierspiel}}
 \item \hyperlink{Green}{Green, Barry, und Gallwey, Timothy: \textit{The Inner Game of Music}}
 \item \hyperlink{Hofman}{Hofman, Josef: \textit{Piano Playing, With Piano Questions Answered}}
 \item \hyperlink{Lhevine}{Lhevine, Josef: \textit{Basic Principles in Piano Playing}}
 \item \hyperlink{Prokop}{Prokop, Richard: \textit{Piano Power, a Breakthrough Approach to Improving your Technique}}
 \item \hyperlink{Richman}{Richman, Howard: \textit{Super Sight-Reading Secrets}}
 \item \hyperlink{Sandor}{Sandor, Gyorgy: \textit{On Piano Playing}}
 \item \hyperlink{Sherman}{Sherman, Russell: \textit{Piano Pieces}}
 \item \hyperlink{Suzuki}{Suzuki, Shinichi (et al): \textit{The Suzuki Concept: An Introduction to a Successful Method for Early Music Education} und<br>\textit{HOW TO TEACH SUZUKI PIANO}}
 \item \hyperlink{Walker}{Walker, Alan: \textit{Franz Liszt, The Virtuoso Years, 1811-1847}}
 \item \hyperlink{Werner}{Werner, Kenney: \textit{Effortless Mastery}}
 \item \hyperlink{Whiteside}{Whiteside, Abby: \textit{On Piano Playing}}
 \item \hyperlink{American}{Weinreich, G.:\textit{The Coupled Motions of Piano Strings}}
 \item \hyperlink{Lectures}{Verschiedene: \textit{Five Lectures on the Acoustics of the Piano}}
 \end{itemize}


<h3>\underline{\hyperlink{Websites}{Websites, Bücher, Videos}}</h3>

\footnote{Im \hyperref[http://www.pianopractice.org]{Original} (extern) folgt hier unter anderem eine Zusammenfassung der Links.
Da diese Liste in der übersetzten Seite wegen der unklaren deutschen Rechtslage nicht wiedergegeben wird, ist hier die Zusammenfassung nicht aufgeführt.}


<h2><br>\hyperlink{anmerkungen}{Anmerkungen}</h2>

<h2><br>\hyperlink{testimonials}{Leserkommentare}</h2>
Probleme, Sorgen und Erfolge von Klavierspielern; hilfreiche Kommentare von Lehrern und Lesern.


<h2><br>Anhang</h2>

\begin{itemize} 
 \item \hyperlink{ueberset}{Anmerkungen zur Übersetzung}
 \item \hyperlink{AbkFarben}{Im Text verwendete Abkürzungen und Farben}
 \item \hyperlink{Danke}{Danke!}
 \end{itemize}




