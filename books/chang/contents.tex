% File: contents

\label{Inhalt}

<h2 align=\enquote{center}>Inhaltsverzeichnis</h2>

Das Buch ist in folgenden Sprachen verfügbar:
 \hyperref[http://www.pianopractice.org]{Englisch} (extern; das Original),
 \hyperref[http://bbs.popiano.org/viewthread.php?tid=81448&amp;extra=page\%3D3]{Einfaches Chinesisch} (extern),
 \hyperref[http://www.pianogarden.tw]{Traditionelles Chinesisch} (extern),
 \hyperref[./index.html]{Deutsch},
 \hyperref[http://pagesperso-orange.fr/musico/documents/textes/pianopratique/tabledesmatieres_fr.htm]{Französisch} (extern),
 \hyperref[http://web.tiscali.it/pianobook]{Italienisch} (extern),
 \hyperref[http://pianofundamental.sakura.ne.jp]{Japanisch} (extern),
 \hyperref[http://pianoart.eu.interia.pl]{Polnisch} (extern),
 \hyperref[http://www.pianopractice.org/spanish.pdf]{Spanisch} (extern)
 .

Ich suche Freiwillige, die das Buch in jede andere Sprache übersetzen -- siehe \hyperref[HinUeber]{Hinweise für Übersetzer}\index{Hinweise für Übersetzer} am Ende dieses Buchs.
Senden Sie bitte eine E-Mail an \hyperref[mailto:cc88m@aol.com?subject=foppde:\%20Translation\%20request]{cc88m@aol.com}, um die Einzelheiten zu besprechen.

\footnote{Wenn diese Website neu für Sie ist, sollten Sie zunächst die ersten beiden Abschnitte der \hyperref[anmerkungen]{Anmerkungen}\index{Anmerkungen} lesen.}


<h2><br>\hyperref[preface]{Vorwort}\index{Vorwort} \textbf{\textit{[09.08.2009]}}</h2>

<h2><br>\hyperref[c1i1]{Kapitel 1: Klaviertechnik}</h2>

\section{Einführung \textbf{\textit{[15.08.2009]}}}

\begin{enumerate} 
 \item \hyperref[c1i1]{Zweck dieses Buchs}\index{Zweck dieses Buchs}
 \item \hyperref[c1i2]{Was ist Klaviertechnik?}\index{Was ist Klaviertechnik?}
 \item \hyperref[c1i3]{Technik, Musik und mentales Spielen}\index{Technik, Musik und mentales Spielen}
 \item \hyperref[c1i4]{Generelles Vorgehen, Interpretation, Musikunterricht, Absolutes Gehör}\index{Generelles Vorgehen, Interpretation, Musikunterricht, Absolutes Gehör}
\end{enumerate}

\section{Grundlegende Verfahren des Klavierübens}

\begin{enumerate} 
 \item \hyperref[c1ii1]{Der Übungsablauf}\index{Der Übungsablauf} \textbf{\textit{[22.08.2009]}}
 \item \hyperref[c1ii2]{Position der Finger}\index{Position der Finger}
 \item \hyperref[c1ii3]{Höhe der Sitzbank und ihr Abstand zum Klavier}\index{Höhe der Sitzbank und ihr Abstand zum Klavier}
 \item \hyperref[c1ii4]{Ein neues Stück -- Anhören und analysieren (\enquote{Für Elise})}\index{Ein neues Stück -- Anhören und analysieren (\enquote{Für Elise})}
 \item \hyperref[c1ii5]{Die schwierigen Abschnitte zuerst üben}\index{Die schwierigen Abschnitte zuerst üben}  \textbf{\textit{[05.09.2009]}}
 \item \hyperref[c1ii6]{Schwierige Passagen kürzen -- In kleinen Portionen üben (taktweise)}\index{Schwierige Passagen kürzen -- In kleinen Portionen üben (taktweise)}
 \item \hyperref[c1ii7]{Die Hände getrennt (einhändig, HS) üben -- Erlernen der Spieltechnik}\index{Die Hände getrennt (einhändig, HS) üben -- Erlernen der Spieltechnik}
 \item \hyperref[c1ii8]{Die Kontinuitätsregel}\index{Die Kontinuitätsregel}
 \item \hyperref[c1ii9]{Der Akkord-Anschlag}\index{Der Akkord-Anschlag}
 \item \hyperref[c1ii10]{Freier Fall, Akkord-Übung und Entspannung}\index{Freier Fall, Akkord-Übung und Entspannung}
 \item \hyperref[c1ii11]{Parallele Sets}\index{Parallele Sets}
 \item \hyperref[c1ii12]{Lernen, Auswendiglernen und mentales Spielen}\index{Lernen, Auswendiglernen und mentales Spielen} \textbf{\textit{[20.09.2009]}}
 \item \hyperref[c1ii13]{Spielgeschwindigkeit beim Üben}\index{Spielgeschwindigkeit beim Üben}
 \item \hyperref[c1ii14]{Wie man entspannt}\index{Wie man entspannt}
 \item \hyperref[c1ii15]{Automatische Verbesserung nach dem Üben (PPI)}\index{Automatische Verbesserung nach dem Üben (PPI)}
 \item \hyperref[c1ii16]{Gefahren des langsamen Spielens -- Fallstricke der \enquote{Intuitiven Methode}}\index{Gefahren des langsamen Spielens -- Fallstricke der \enquote{Intuitiven Methode}}
 \item \hyperref[c1ii17]{Die Wichtigkeit des langsamen Spielens}\index{Die Wichtigkeit des langsamen Spielens} \textbf{\textit{[04.10.2009]}}

 \item \hyperref[c1ii18]{Fingersatz}\index{Fingersatz}
 \item \hyperref[c1ii19]{Akkurates Tempo und das Metronom}\index{Akkurates Tempo und das Metronom}
 \item \hyperref[c1ii20]{Die schwache linke Hand -- Eine Hand unterrichtet die andere}\index{Die schwache linke Hand -- Eine Hand unterrichtet die andere}
 \item \hyperref[c1ii21]{Ausdauer aufbauen, Atmung}\index{Ausdauer aufbauen, Atmung}

 \item \hyperref[c1ii22]{Schlechte Angewohnheiten: Der größte Feind des Pianisten}\index{Schlechte Angewohnheiten: Der größte Feind des Pianisten} \textbf{\textit{[31.10.2009]}}
 \item \hyperref[c1ii23]{Haltepedal}\index{Haltepedal}\footnote{Anmerkungen zu den Bezeichnungen der Pedale finden Sie \hyperref[Pedale]{hier}\index{hier}}
 
 \item \hyperref[c1ii24]{Dämpferpedal, Timbre und Eigenschwingungen vibrierender Saiten}\index{Dämpferpedal, Timbre und Eigenschwingungen vibrierender Saiten}
 \item \hyperref[c1ii25]{Mit beiden Händen zusammen (HT) üben und mental spielen}\index{Mit beiden Händen zusammen (HT) üben und mental spielen} \textbf{\textit{[05.12.2009]}}
  \begin{enumerate}[label={\alph*.}] 
   <li>\hyperref[c1ii25a]{Einführung}\index{Einführung}
   \item \hyperref[c1ii25b]{Beethovens Mondschein-Sonate, 1. Satz, Op. 27, No. 2}\index{Beethovens Mondschein-Sonate, 1. Satz, Op. 27, No. 2}
   \item \hyperref[c1ii25c]{Mozarts Rondo Alla Turca, aus Sonate K300 (KV331)}\index{Mozarts Rondo Alla Turca, aus Sonate K300 (KV331)}
   \item \hyperref[c1ii25d]{Chopins Fantaisie-Impromptu, Op. 66}\index{Chopins Fantaisie-Impromptu, Op. 66}
  \end{enumerate}
 </li>
 \item \hyperref[c1ii26]{Zusammenfassung}\index{Zusammenfassung}
 \end{enumerate}

\section{Ausgewählte Themen des Klavierübens}

\begin{enumerate} 
 \item \hyperref[c1iii1]{Klang, Rhythmus, Legato, Staccato}\index{Klang, Rhythmus, Legato, Staccato} \textbf{\textit{[14.02.2010]}}
  \begin{enumerate}[label={\alph*.}] 
   <li>\hyperref[c1iii1a]{Was ist ein \enquote{Guter Klang}?}\index{Was ist ein \enquote{Guter Klang}?}
   <ul type=\enquote{circle}>
    <li>\hyperref[c1iii1a1]{Der Basisanschlag}\index{Der Basisanschlag}
    \item \hyperref[c1iii1a2]{Klang: Einzelne gegenüber mehreren Noten, Pianissimo, Fortissimo}\index{Klang: Einzelne gegenüber mehreren Noten, Pianissimo, Fortissimo}
   </ul>
   </li>
   \item \hyperref[c1iii1b]{Was ist Rhythmus? (Beethovens Sturm-Sonate und Appassionata)}\index{Was ist Rhythmus? (Beethovens Sturm-Sonate und Appassionata)}
   \item \hyperref[c1iii1c]{Legato, Staccato}\index{Legato, Staccato}
  \end{enumerate}
 </li>
 \item \hyperref[c1iii2]{Zyklisch spielen (Chopins Fantaisie Impromptu, Op. 66)}\index{Zyklisch spielen (Chopins Fantaisie Impromptu, Op. 66)} \textbf{\textit{[21.03.2010]}}

 \item \hyperref[c1iii3]{Triller und Tremolos (Beethovens Pathétique, 1. Satz)}\index{Triller und Tremolos (Beethovens Pathétique, 1. Satz)}
  <ol type=\enquote{a}>
   <li>\hyperref[c1iii3]{Triller}\index{Triller}
   \item \hyperref[c1iii3b]{Tremolos (Beethovens Pathétique, 1. Satz)}\index{Tremolos (Beethovens Pathétique, 1. Satz)}
   \end{enumerate}
 </li>
 \item \hyperref[c1iii4]{Bewegungen der Hand, der Finger und des Körpers}\index{Bewegungen der Hand, der Finger und des Körpers} \textbf{\textit{[24.10.2010]}}
  \begin{enumerate}[label={\alph*.}] 
   <li>\hyperref[c1iii4]{Bewegungen der Hand}\index{Bewegungen der Hand}
   \item \hyperref[c1iii4b]{Mit flachen (gestreckten) Fingern spielen}\index{Mit flachen (gestreckten) Fingern spielen}
   \item \hyperref[c1iii4c]{Bewegungen des Körpers}\index{Bewegungen des Körpers}
   \end{enumerate}
 </li>
 \item \hyperref[c1iii5]{Schnell spielen: Tonleitern, Arpeggios und chromatische Tonleitern (Chopins Fantaisie Impromptu und Beethovens Mondschein-Sonate, 3. Satz)}\index{Schnell spielen: Tonleitern, Arpeggios und chromatische Tonleitern (Chopins Fantaisie Impromptu und Beethovens Mondschein-Sonate, 3. Satz)} \textbf{\textit{[21.08.2011]}}
  \begin{enumerate}[label={\alph*.}] 
   <li>\hyperref[c1iii5a]{Tonleitern: Daumenuntersatz, Daumenübersatz}\index{Tonleitern: Daumenuntersatz, Daumenübersatz}
   \item \hyperref[c1iii5b]{Daumenübersatz: Bewegung, Erklärung und Video}\index{Daumenübersatz: Bewegung, Erklärung und Video}
   \item \hyperref[c1iii5c]{Daumenübersatz üben: Geschwindigkeit, Glissandobewegung}\index{Daumenübersatz üben: Geschwindigkeit, Glissandobewegung}
   \item \hyperref[c1iii5d]{Tonleitern: Herkunft, Namensgebung, Fingersätze}\index{Tonleitern: Herkunft, Namensgebung, Fingersätze}
    <ul type=\enquote{circle}>
     <li>\hyperref[table]{Fingersatztabelle}\index{Fingersatztabelle}
    </ul>
   </li>
   \item \hyperref[c1iii5e]{Arpeggios (Chopin, Wagenradbewegung, \enquote{gespreizte} Finger)}\index{Arpeggios (Chopin, Wagenradbewegung, \enquote{gespreizte} Finger)}
   \item \hyperref[c1iii5f]{Schub und Zug, Beethovens Mondschein-Sonate, 3. Satz}\index{Schub und Zug, Beethovens Mondschein-Sonate, 3. Satz}
   \item \hyperref[c1iii5g]{Der Daumen: Der vielseitigste Finger}\index{Der Daumen: Der vielseitigste Finger}
   \item \hyperref[c1iii5h]{Schnelle chromatische Tonleitern}\index{Schnelle chromatische Tonleitern}
   \end{enumerate}<br><br>[Ab hier wird der Text noch überarbeitet.]<br><br>
 </li>
 \item \hyperref[c1iii6]{Auswendiglernen}\index{Auswendiglernen}
  \begin{enumerate}[label={\alph*.}] 
   <li>\hyperref[c1iii6a]{Warum auswendig lernen?}\index{Warum auswendig lernen?}
   \item \hyperref[c1iii6b]{Wer kann auswendig lernen, was und wann?}\index{Wer kann auswendig lernen, was und wann?}
   \item \hyperref[c1iii6c]{Auswendiglernen und Pflege des Gelernten}\index{Auswendiglernen und Pflege des Gelernten}
   \item \hyperref[c1iii6d]{Hand-Gedächtnis}\index{Hand-Gedächtnis}
   \item \hyperref[c1iii6e]{Wie fängt man an?}\index{Wie fängt man an?}
   \item \hyperref[c1iii6f]{Auffrischung des Gedächtnisses}\index{Auffrischung des Gedächtnisses}
   \item \hyperref[c1iii6g]{Kaltstart}\index{Kaltstart}
   \item \hyperref[c1iii6h]{Langsam spielen}\index{Langsam spielen}
   \item \hyperref[c1iii6i]{Vorausschauend spielen}\index{Vorausschauend spielen}
   \item \hyperref[c1iii6j]{Langzeitgedächtnis aufbauen}\index{Langzeitgedächtnis aufbauen}
    <ul type=\enquote{circle}>
     <li>\hyperref[c1iii6hand]{Hand-Gedächtnis}\index{Hand-Gedächtnis}
     \item \hyperref[c1iii6musik]{Musik-Gedächtnis}\index{Musik-Gedächtnis}
     \item \hyperref[c1iii6foto]{Fotografisches Gedächtnis}\index{Fotografisches Gedächtnis}
     \item \hyperref[c1iii6tastatur]{Tastatur-Gedächtnis}\index{Tastatur-Gedächtnis}
     \item \hyperref[c1iii6theorie]{Theoretisches Gedächtnis}\index{Theoretisches Gedächtnis}
    </ul>
   </li>
   \item \hyperref[c1iii6k]{Pflege}\index{Pflege}
   \item \hyperref[c1iii6l]{Blattspieler und Auswendiglernende (Bachs Inventionen)}\index{Blattspieler und Auswendiglernende (Bachs Inventionen)}
    <ul type=\enquote{circle}>
     <li>\hyperref[c1iii6l]{Blattspieler und Auswendiglernende}\index{Blattspieler und Auswendiglernende}
     \item \hyperref[c1iii6l2]{Bachs Inventionen}\index{Bachs Inventionen}
     \item \hyperref[ruhig]{Ruhige Hände}\index{Ruhige Hände}
    </ul>
   </li>
   \item \hyperref[c1iii6m]{Funktion des menschlichen Gedächtnisses}\index{Funktion des menschlichen Gedächtnisses}
   \item \hyperref[c1iii6n]{Ein guter Auswendiglernender werden}\index{Ein guter Auswendiglernender werden}
   \item \hyperref[c1iii6o]{Zusammenfassung}\index{Zusammenfassung}
   \end{enumerate}
 </li>
 \item \hyperref[c1iii7]{Übungen}\index{Übungen}
  \begin{enumerate}[label={\alph*.}] 
   <li>\hyperref[c1iii7a]{Einführung}\index{Einführung}
    <ul type=\enquote{circle}>
      <li>\hyperref[c1iii7aMuskeln]{Schnelle und langsame Muskeln}\index{Schnelle und langsame Muskeln}
    </ul>
   </li> 
   \item \hyperref[c1iii7b]{Parallele Sets}\index{Parallele Sets}
   \item \hyperref[c1iii7c]{Wie verwendet man die Übungen für parallele Sets (Appassionata, 3. Satz)?}\index{Wie verwendet man die Übungen für parallele Sets (Appassionata, 3. Satz)?}
   \item \hyperref[c1iii7d]{Tonleitern, Arpeggios, Unabhängigkeit der Finger und Anheben der Finger}\index{Tonleitern, Arpeggios, Unabhängigkeit der Finger und Anheben der Finger}
   \item \hyperref[c1iii7e]{(Große) Akkorde spielen, Dehnung der Handflächen}\index{(Große) Akkorde spielen, Dehnung der Handflächen}
   \item \hyperref[c1iii7f]{Sprünge}\index{Sprünge}
   \item \hyperref[c1iii7g]{Weitere Übungen}\index{Weitere Übungen}
   \item \hyperref[c1iii7h]{Probleme mit Hanons Übungen}\index{Probleme mit Hanons Übungen}
   \item \hyperref[c1iii7i]{Die Geschwindigkeit steigern}\index{Die Geschwindigkeit steigern}
    <ul type=\enquote{circle}>
     <li>\hyperref[c1iii7iAnschlag]{Schneller Anschlag, Entspannung}\index{Schneller Anschlag, Entspannung}
     \item \hyperref[c1iii7iAndere]{Andere Geschwindigkeitsmethoden}\index{Andere Geschwindigkeitsmethoden}
     \item \hyperref[c1iii7iMusik]{Geschwindigkeit und Musik}\index{Geschwindigkeit und Musik}
    </ul>
   </li>
   \end{enumerate}
 </li>
 \item \hyperref[c1iii8]{Konturieren (Beethovens Sonate \#1)}\index{Konturieren (Beethovens Sonate \#1)}
 \item \hyperref[c1iii9]{Ein Stück auf Hochglanz bringen -- Fehler beseitigen}\index{Ein Stück auf Hochglanz bringen -- Fehler beseitigen}
 \item \hyperref[c1iii10]{Kalte Hände, rutschende Finger, Krankheiten, Handverletzungen, Gehörschäden}\index{Kalte Hände, rutschende Finger, Krankheiten, Handverletzungen, Gehörschäden}
  \begin{enumerate}[label={\alph*.}] 
   <li>\hyperref[c1iii10]{Kalte Hände}\index{Kalte Hände}
   \item \hyperref[c1iii10rutschen]{Rutschende (trockene oder schwitzende) Finger}\index{Rutschende (trockene oder schwitzende) Finger}
   \item \hyperref[c1iii10krank]{Krankheiten}\index{Krankheiten}
   \item \hyperref[c1iii10ungesund]{Gesundes und ungesundes Üben}\index{Gesundes und ungesundes Üben}
   \item \hyperref[c1iii10hand]{Verletzungen der Hand}\index{Verletzungen der Hand}
   \item \hyperref[c1iii10gehoer]{Gehörschäden}\index{Gehörschäden}
   \end{enumerate}
 </li>
 \item \hyperref[c1iii11]{Blattspiel}\index{Blattspiel}
 \item \hyperref[c1iii12]{Absolutes Gehör und relatives Gehör (vom Blatt singen)}\index{Absolutes Gehör und relatives Gehör (vom Blatt singen)}
   <ul type=\enquote{circle}>
    <li>\hyperref[c1iii12tonhoehe]{Verfahren zum Lernen der relativen und absoluten Tonhöhenerkennung}\index{Verfahren zum Lernen der relativen und absoluten Tonhöhenerkennung}
    \item \hyperref[c1iii12blatt]{Vom Blatt singen und komponieren}\index{Vom Blatt singen und komponieren}
   </ul>
 </li>
 \item \hyperref[c1iii13]{Filmen und Aufnehmen des eigenen Spielens,\footnote{MIDI, Digitalpianos, Keyboards usw.}\index{Filmen und Aufnehmen des eigenen Spielens,\footnote{MIDI, Digitalpianos, Keyboards usw.}}
     <ul type=\enquote{circle}>
      <li>\hyperref[c1iii13MIDI]{\footnote{MIDI, Digitalpianos, Keyboards usw.}}
    </ul>
 </li>
 \item \hyperref[c1iii14]{Vorbereitung auf Auftritte und Konzerte}\index{Vorbereitung auf Auftritte und Konzerte}
  \begin{enumerate}[label={\alph*.}] 
   <li>\hyperref[c1iii14a]{Nutzen und Risiken von Auftritten und Konzerten}\index{Nutzen und Risiken von Auftritten und Konzerten}
   \item \hyperref[c1iii14b]{Grundlagen fehlerfreien Vorspielens}\index{Grundlagen fehlerfreien Vorspielens}
   \item \hyperref[c1iii14c]{Für Auftritte üben}\index{Für Auftritte üben}
   \item \hyperref[c1iii14d]{Musikalisch üben}\index{Musikalisch üben}
   \item \hyperref[c1iii14e]{Zwangloses Vorspielen}\index{Zwangloses Vorspielen}
   \item \hyperref[c1iii14f]{Vorbereitung auf Konzerte}\index{Vorbereitung auf Konzerte}
   \item \hyperref[c1iii14g]{Während des Konzerts}\index{Während des Konzerts}
   \item \hyperref[c1iii14h]{Das ungewohnte Klavier}\index{Das ungewohnte Klavier}
   \item \hyperref[c1iii14i]{Nach dem Konzert}\index{Nach dem Konzert}
   \end{enumerate}
 </li>
 \item \hyperref[c1iii15]{Ursachen und Kontrolle von Nervosität}\index{Ursachen und Kontrolle von Nervosität}
 \item \hyperref[c1iii16]{Unterrichten}\index{Unterrichten}
  \begin{enumerate}[label={\alph*.}] 
   <li>\hyperref[c1iii16a]{Lehrer}\index{Lehrer}
   \item \hyperref[c1iii16b]{Kinder unterrichten, Eltern einbeziehen}\index{Kinder unterrichten, Eltern einbeziehen}
   \item \hyperref[c1iii16c]{Blattspiel, Auswendiglernen, Theorie}\index{Blattspiel, Auswendiglernen, Theorie}
   \item \hyperref[c1iii16d]{Einige Elemente des Klavierunterrichts}\index{Einige Elemente des Klavierunterrichts}
   \item \hyperref[c1iii16e]{Warum die größten Pianisten nicht unterrichten konnten}\index{Warum die größten Pianisten nicht unterrichten konnten}
   \end{enumerate}
 </li>
 \item \hyperref[c1iii17]{Klaviere, Flügel und elektronische Klaviere; Kauf und Wartung}\index{Klaviere, Flügel und elektronische Klaviere; Kauf und Wartung}
  \begin{enumerate}[label={\alph*.}] 
   <li>\hyperref[c1iii17a]{Flügel, akustisches oder elektronisches Klavier?}\index{Flügel, akustisches oder elektronisches Klavier?}
   \item \hyperref[c1iii17b]{Elektronische Klaviere}\index{Elektronische Klaviere}
   \item \hyperref[c1iii17c]{Klaviere}\index{Klaviere}
   \item \hyperref[c1iii17d]{Flügel}\index{Flügel}
   \item \hyperref[c1iii17e]{Ein akustisches Klavier kaufen}\index{Ein akustisches Klavier kaufen}
   \item \hyperref[c1iii17f]{Pflege und Wartung des Klaviers}\index{Pflege und Wartung des Klaviers}
   \item \hyperref[c1iii17g]{\footnote{Anmerkungen zu Digitalpianos}}
   \end{enumerate}
 </li>
 \item \hyperref[c1iii18]{Wie man das Klavierspielenlernen beginnt -- vom jüngsten Kind bis zum ältesten Erwachsenen}\index{Wie man das Klavierspielenlernen beginnt -- vom jüngsten Kind bis zum ältesten Erwachsenen}
     \begin{enumerate}[label={\alph*.}] 
        <li>\hyperref[c1iii18a]{Benötigt man einen Lehrer?}\index{Benötigt man einen Lehrer?}
        \item \hyperref[c1iii18b]{Bücher für Anfänger; Keyboards}\index{Bücher für Anfänger; Keyboards}
        \item \hyperref[c1iii18c]{Anfänger im Alter von 0 bis über 65}\index{Anfänger im Alter von 0 bis über 65}
            <ul type=\enquote{circle}>
               <li>\hyperref[c1iii18c0]{\autoref{c1iii18c0}}
               \item \hyperref[c1iii18c3]{\autoref{c1iii18c3}}
               \item \hyperref[c1iii18c13]{\autoref{c1iii18c13}}
               \item \hyperref[c1iii18c20]{\autoref{c1iii18c20}}
               \item \hyperref[c1iii18c35]{\autoref{c1iii18c35}}
               \item \hyperref[c1iii18c45]{\autoref{c1iii18c45}}
               \item \hyperref[c1iii18c65]{Über 65}\index{Über 65}
            </ul>
        </li>
      \end{enumerate}
 </li>
 \item \hyperref[c1iii19]{Der \enquote{ideale} Übungsablauf (Bachs Invention \#4)}\index{Der \enquote{ideale} Übungsablauf (Bachs Invention \#4)}
     \begin{enumerate}[label={\alph*.}] 
        <li>\hyperref[c1iii19a]{Die Regeln lernen}\index{Die Regeln lernen}
        \item \hyperref[c1iii19b]{Ein neues Stück lernen (Invention \#4)}\index{Ein neues Stück lernen (Invention \#4)}
        \item \hyperref[c1iii19c]{\enquote{Normale} Übungsabläufe und Bachs Lehren}\index{\enquote{Normale} Übungsabläufe und Bachs Lehren}
      \end{enumerate}
 </li>
 \item \hyperref[c1iii20]{Bach: der größte Komponist und Lehrer (15 Inventionen)}\index{Bach: der größte Komponist und Lehrer (15 Inventionen)}
   <ul type=\enquote{circle}>
    <li>\hyperref[c1iii20ps]{Liste der parallelen Sets in den einzelnen Inventionen (für die RH)}\index{Liste der parallelen Sets in den einzelnen Inventionen (für die RH)}
   </ul>
 </li>
 \item \hyperref[c1iii21]{Klavierspielen und die Psychologie}\index{Klavierspielen und die Psychologie}
 \item \hyperref[c1iii22]{Zusammenfassung der Methoden}\index{Zusammenfassung der Methoden}

</ol>

\section{Mathematische Theorien des Klavierspielens}

\begin{enumerate} 
 \item \hyperref[c1iv1]{Wozu braucht man mathematische Theorien?}\index{Wozu braucht man mathematische Theorien?}
 \item \hyperref[c1iv2]{Die Theorie der Fingerbewegung}\index{Die Theorie der Fingerbewegung}
   \begin{enumerate}[label={\alph*.}] 
    <li>\hyperref[c1iv2a]{Serielles und paralleles Spielen}\index{Serielles und paralleles Spielen}
    \item \hyperref[c1iv2b]{Geschwindigkeitsbarrieren}\index{Geschwindigkeitsbarrieren}
    \item \hyperref[c1iv2c]{Die Geschwindigkeit steigern}\index{Die Geschwindigkeit steigern}
   \end{enumerate}
 </li>
 \item \hyperref[c1iv3]{Die Thermodynamik des Klavierspielens}\index{Die Thermodynamik des Klavierspielens}
 \item \hyperref[c1iv4]{Mozarts Formel, Beethoven und Gruppentheorie}\index{Mozarts Formel, Beethoven und Gruppentheorie}
 \item \hyperref[c1iv5]{Berechnen der Lernrate}\index{Berechnen der Lernrate}
 \item \hyperref[c1iv6]{Noch zu erforschende Themen}\index{Noch zu erforschende Themen}
   <ol type=\enquote{a}>
    <li>\hyperref[c1iv6a]{Impulstheorie des Klavierspielens}\index{Impulstheorie des Klavierspielens}
    \item \hyperref[c1iv6b]{Die Physiologie der Technik}\index{Die Physiologie der Technik}
    \item \hyperref[c1iv6c]{Gerhirnforschung (HS- und HT-Spielen usw.)}\index{Gerhirnforschung (HS- und HT-Spielen usw.)}
    \item \hyperref[c1iv6d]{Was verursacht Nervosität?}\index{Was verursacht Nervosität?}
    \item \hyperref[c1iv6e]{Ursachen von und Mittel gegen Tinnitus}\index{Ursachen von und Mittel gegen Tinnitus}
    \item \hyperref[c1iv6f]{Was ist Musik?}\index{Was ist Musik?}
    \item \hyperref[c1iv6g]{In welchem Alter soll bzw. darf man mit dem Klavierspielen anfangen?}\index{In welchem Alter soll bzw. darf man mit dem Klavierspielen anfangen?}
    \item \hyperref[c1iv6h]{Die Zukunft des Klavierspielens}\index{Die Zukunft des Klavierspielens}
    \item \hyperref[c1iv6i]{Die Zukunft des Unterrichts}\index{Die Zukunft des Unterrichts}
    \end{enumerate}
 </li>
</ol>

<h2><br>Kapitel 2: Stimmen des Klaviers</h2> 

\begin{enumerate} 
 \item \hyperref[c2_1]{Einleitung}\index{Einleitung}
 \item \hyperref[c2_2]{Chromatische Tonleiter und Temperaturen}\index{Chromatische Tonleiter und Temperaturen}
  \begin{enumerate}[label={\alph*.}] 
   <li>\hyperref[c2_2a]{Einleitung}\index{Einleitung}
   \item \hyperref[c2_2b]{Mathematische Behandlung}\index{Mathematische Behandlung}
   \item \hyperref[c2_2c]{Temperatur und Musik}\index{Temperatur und Musik}
  \end{enumerate}
 </li>
 \item \hyperref[c2_3]{Werkzeuge zum Stimmen}\index{Werkzeuge zum Stimmen}
 \item \hyperref[c2_4]{Vorbereitung}\index{Vorbereitung}
 \item \hyperref[c2_5]{Wie man anfängt}\index{Wie man anfängt}
  <ol type=\enquote{a}>
   <li>\hyperref[c2_5a]{Einleitung}\index{Einleitung}
   \item \hyperref[c2_5_hamm]{Einsetzen und Bewegen des Stimmhammers}\index{Einsetzen und Bewegen des Stimmhammers}
   \item \hyperref[c2_5_wirb]{Den Wirbel einstellen}\index{Den Wirbel einstellen}
   \item \hyperref[c2_5_unis]{Unisono stimmen}\index{Unisono stimmen}
   \item \hyperref[c2_5_mits]{Mitschwingung}\index{Mitschwingung}
   \item \hyperref[c2_5_infi]{Diese letzte infinitesimale Bewegung ausführen}\index{Diese letzte infinitesimale Bewegung ausführen}
   \item \hyperref[c2_5_span]{Ausgleich der Saitenspannung}\index{Ausgleich der Saitenspannung}
   \item \hyperref[c2_5_disk]{Wiegen im Diskant}\index{Wiegen im Diskant}
   \item \hyperref[c2_5_bass]{Grollen im Bass}\index{Grollen im Bass}
   \item \hyperref[c2_5_harm]{Harmonisches Stimmen}\index{Harmonisches Stimmen}
   \item \hyperref[c2_5_stre]{Was ist Streckung?}\index{Was ist Streckung?}
   \item \hyperref[c2_5_prae]{Präzision, Präzision, Präzision}\index{Präzision, Präzision, Präzision}
   \end{enumerate}
 </li>
 \item \hyperref[c2_6]{Stimmverfahren}\index{Stimmverfahren}
  \begin{enumerate}[label={\alph*.}] 
   <li>\hyperref[c2_6a]{Einleitung}\index{Einleitung}
   \item \hyperref[c2_6_gabe]{Das Klavier nach der Stimmgabel stimmen}\index{Das Klavier nach der Stimmgabel stimmen}
   \item \hyperref[c2_6_kirn]{Kirnberger II}\index{Kirnberger II}
   \item \hyperref[c2_6_et]{Gleichschwebende Temperatur (gleichstufige Temperatur, gleichmäßige Temperatur)}\index{Gleichschwebende Temperatur (gleichstufige Temperatur, gleichmäßige Temperatur)}
   \end{enumerate}
 </li>
 \item \hyperref[c2_7]{Kleinere Reparaturen durchführen}\index{Kleinere Reparaturen durchführen}
  \begin{enumerate}[label={\alph*.}] 
   <li>\hyperref[c2_7_hamm]{Intonieren der Hämmer}\index{Intonieren der Hämmer}
   \item \hyperref[c2_7_pilo]{Polieren der Piloten}\index{Polieren der Piloten}
   \end{enumerate}
 </li>
</ol> 

<h2><br>\hyperref[c3_1]{Kapitel 3: Wissenschaftliche Methode, Theorie des Lernens, Das Gehirn}</h2>

\footnote{Abschnitt 4 ist im Original zurzeit (26.5.2003) noch \enquote{preliminary draft} also ein \enquote{Rohentwurf}.}


\begin{enumerate} 
 \item \hyperref[c3_1]{Einleitung}\index{Einleitung}
 \item \hyperref[c3_2]{Der wissenschaftliche Ansatz}\index{Der wissenschaftliche Ansatz}
  \begin{enumerate}[label={\alph*.}] 
   <li>\hyperref[c3_2a]{Einleitung}\index{Einleitung}
   \item \hyperref[c3_2b]{Lernen}\index{Lernen}
  \end{enumerate}
 </li>
 \item \hyperref[c3_3]{Was ist die Wissenschaftliche Methode?}\index{Was ist die Wissenschaftliche Methode?}
  <ol type=\enquote{a}>
   <li>\hyperref[c3_3a]{Einleitung}\index{Einleitung}
   \item \hyperref[c3_3b]{Definition}\index{Definition}
   \item \hyperref[c3_3c]{Forschung}\index{Forschung}
   \item \hyperref[c3_3d]{Dokumentation und Kommunikation}\index{Dokumentation und Kommunikation}
   \item \hyperref[c3_3e]{Konsistenzprüfungen}\index{Konsistenzprüfungen}
   \item \hyperref[c3_3f]{Grundlegende Theorie}\index{Grundlegende Theorie}
   \item \hyperref[c3_3g]{Dogma und Lehre}\index{Dogma und Lehre}
   \item \hyperref[c3_3h]{Schlussfolgerungen}\index{Schlussfolgerungen}
   \end{enumerate}
 </li>
 \item \hyperref[c3_4]{Theorie des Lernens}\index{Theorie des Lernens}
 \item \hyperref[c3_5]{Was Träume erzeugt und Methoden zu ihrer Kontrolle}\index{Was Träume erzeugt und Methoden zu ihrer Kontrolle}
  \begin{enumerate}[label={\alph*.}] 
   <li>\hyperref[c3_5a]{Einleitung}\index{Einleitung}
   \item \hyperref[c3_5b]{Der Fall-Traum}\index{Der Fall-Traum}
   \item \hyperref[c3_5c]{Der Unfähig-zu-laufen-Traum}\index{Der Unfähig-zu-laufen-Traum}
   \item \hyperref[c3_5d]{Der Zu-spät-zur-Prüfung-kommen- oder Sich-verlaufen-Traum}\index{Der Zu-spät-zur-Prüfung-kommen- oder Sich-verlaufen-Traum}
   \item \hyperref[c3_5e]{Die Lösung für meinen langen und komplexen Traum}\index{Die Lösung für meinen langen und komplexen Traum}
   \item \hyperref[c3_5f]{Die Kontrolle der Träume}\index{Die Kontrolle der Träume}
   \item \hyperref[c3_5g]{Was uns diese Träume über das Gehirn lehren}\index{Was uns diese Träume über das Gehirn lehren}
   \end{enumerate}
</li>
 \item \hyperref[c3_6]{Das Unterbewusstsein}\index{Das Unterbewusstsein}
  \begin{enumerate}[label={\alph*.}] 
   <li>\hyperref[c3_6a]{Einleitung}\index{Einleitung}
   \item \hyperref[c3_6b]{Emotionen}\index{Emotionen}
   \item \hyperref[c3_6c]{Das Unterbewusstsein benutzen}\index{Das Unterbewusstsein benutzen}
   \end{enumerate}
</li>
</ol>

<h2><br>\hyperref[reference]{Quellenverzeichnis}</h2>

<h3>\underline{\hyperref[reference]{Buchbesprechungen}\index{Buchbesprechungen}}</h3>
\begin{itemize} 
 \item \hyperref[allgemein]{Allgemeine Schlussfolgerungen}\index{Allgemeine Schlussfolgerungen}
 \item \hyperref[Bree]{Bree, Malwine: \textit{The Leschetizky Method}\index{Bree, Malwine: \textit{The Leschetizky Method}}
 \item \hyperref[Bruser]{Bruser, Madeline: \textit{The Art of Practicing}\index{Bruser, Madeline: \textit{The Art of Practicing}}
 \item \hyperref[Chang]{Chang, Chuan C.: \textit{Fundamentals of Piano Practice}\index{Chang, Chuan C.: \textit{Fundamentals of Piano Practice}, erste Ausgabe}
 \item \hyperref[Eigeldinger]{Eigeldinger, Jean-Jacques: \textit{Chopin, pianist and teacher as seen by his pupils}}
 \item \hyperref[Fink]{Fink, Seymour: \textit{Mastering Piano Technique}}
 \item \hyperref[Gieseking]{Gieseking, Walter und Leimer, Karl: \textit{Modernes Klavierspiel}}
 \item \hyperref[Green]{Green, Barry, und Gallwey, Timothy: \textit{The Inner Game of Music}\index{Green, Barry, und Gallwey, Timothy: \textit{The Inner Game of Music}}
 \item \hyperref[Hofman]{Hofman, Josef: \textit{Piano Playing, With Piano Questions Answered}\index{Hofman, Josef: \textit{Piano Playing, With Piano Questions Answered}}
 \item \hyperref[Lhevine]{Lhevine, Josef: \textit{Basic Principles in Piano Playing}\index{Lhevine, Josef: \textit{Basic Principles in Piano Playing}}
 \item \hyperref[Prokop]{Prokop, Richard: \textit{Piano Power, a Breakthrough Approach to Improving your Technique}}
 \item \hyperref[Richman]{Richman, Howard: \textit{Super Sight-Reading Secrets}\index{Richman, Howard: \textit{Super Sight-Reading Secrets}}
 \item \hyperref[Sandor]{Sandor, Gyorgy: \textit{On Piano Playing}}
 \item \hyperref[Sherman]{Sherman, Russell: \textit{Piano Pieces}\index{Sherman, Russell: \textit{Piano Pieces}}
 \item \hyperref[Suzuki]{Suzuki, Shinichi (et al): \textit{The Suzuki Concept: An Introduction to a Successful Method for Early Music Education}\index{Suzuki, Shinichi (et al): \textit{The Suzuki Concept: An Introduction to a Successful Method for Early Music Education} und<br>\textit{HOW TO TEACH SUZUKI PIANO}}
 \item \hyperref[Walker]{Walker, Alan: \textit{Franz Liszt, The Virtuoso Years, 1811-1847}\index{Walker, Alan: \textit{Franz Liszt, The Virtuoso Years, 1811-1847}}
 \item \hyperref[Werner]{Werner, Kenney: \textit{Effortless Mastery}\index{Werner, Kenney: \textit{Effortless Mastery}}
 \item \hyperref[Whiteside]{Whiteside, Abby: \textit{On Piano Playing}}
 \item \hyperref[American]{Weinreich, G.:\textit{The Coupled Motions of Piano Strings}\index{Weinreich, G.:\textit{The Coupled Motions of Piano Strings}}
 \item \hyperref[Lectures]{Verschiedene: \textit{Five Lectures on the Acoustics of the Piano}\index{Verschiedene: \textit{Five Lectures on the Acoustics of the Piano}}
 \end{itemize}


<h3>\underline{\hyperref[Websites]{Websites, Bücher, Videos}\index{Websites, Bücher, Videos}}</h3>

\footnote{Im \hyperref[http://www.pianopractice.org]{Original} (extern) folgt hier unter anderem eine Zusammenfassung der Links.
Da diese Liste in der übersetzten Seite wegen der unklaren deutschen Rechtslage nicht wiedergegeben wird, ist hier die Zusammenfassung nicht aufgeführt.}


<h2><br>\hyperref[anmerkungen]{Anmerkungen}\index{Anmerkungen}</h2>

<h2><br>\hyperref[testimonials]{Leserkommentare}\index{Leserkommentare}</h2>
Probleme, Sorgen und Erfolge von Klavierspielern; hilfreiche Kommentare von Lehrern und Lesern.


<h2><br>Anhang</h2>

\begin{itemize} 
 \item \hyperref[ueberset]{Anmerkungen zur Übersetzung}\index{Anmerkungen zur Übersetzung}
 \item \hyperref[AbkFarben]{Im Text verwendete Abkürzungen und Farben}\index{Im Text verwendete Abkürzungen und Farben}
 \item \hyperref[Danke]{Danke!}\index{Danke!}
 \end{itemize}




