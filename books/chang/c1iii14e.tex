% File: c1iii14e

\subsubsection{Zwangloses Vorspielen}
\label{c1iii14e}

Gewöhnliche Arten zwanglosen Vorspielens sind Stücke zu spielen, um in Geschäften Klaviere zu testen, oder bei Partys für Freunde zu spielen usw.
Diese unterscheiden sich von formellen Konzerten aufgrund ihrer größeren Freiheit und dem reduzierten mentalen Druck.
Es gibt üblicherweise kein festgelegtes Programm, Sie picken sich das heraus, was Sie im Moment für angemessen halten.
Es kann voller Änderungen und Unterbrechungen sein.
\hyperref[c1iii15]{Nervosität} ist in der Regel kein Thema, und das zwanglose Vorspielen ist sogar eine der besten Möglichkeiten, Methoden zur Vermeidung von Nervosität zu üben.
Trotz dieser abschwächenden Faktoren ist das am Anfang nicht leicht.
Um einen leichten Start zu bekommen, spielen Sie kleine Auszüge (kurze Ausschnitte einer Komposition).
Beginnen Sie mit leichten Stücken; picken Sie nur die am besten klingenden Abschnitte heraus.
Wenn es nicht so gut funktioniert, beginnen Sie mit einem anderen. Dasselbe wenn Sie hängenbleiben.
Sie können jederzeit anfangen und aufhören.
Das ist eine großartige Möglichkeit zu experimentieren und herauszufinden wie Sie vorspielen und welche  Auszüge funktionieren.
Neigen Sie dazu, zu schnell zu spielen?
Es ist besser, zu langsam anzufangen und schneller zu werden als umgekehrt.
Können Sie ein schönes Legato spielen, oder ist Ihr Klang zu schrill?
Können Sie sich an ein anderes Klavier anpassen, insbesondere an eines, das verstimmt oder schwer zu spielen ist?
Können Sie der Reaktion des Publikums folgen?
Reagiert das Publikum auf Ihr Spielen?
Können Sie die passende Art von Auszügen für die Gelegenheit auswählen?
Können Sie sich selbst in die richtige Gemütsverfassung zum Spielen bringen?
Wie nervös sind Sie, können Sie es kontrollieren?
Können Sie Fehler überspielen ohne sich von ihnen stören zu lassen?
Eine weitere Möglichkeit, das Vorspielen zu üben, ist, Kinder, die nie Klavierunterricht hatten, in das Klavierspielen einzuführen.
Bringen Sie ihnen die Tonleiter, \enquote{Alle meine Entchen} oder \enquote{Happy Birthday} bei.

Auszüge zu spielen hat einen interessanten Vorteil. Das Publikum ist meistens von Ihrer Fähigkeit beeindruckt, irgendwo in der Mitte eines Stücks anzufangen und aufzuhören.
Die meisten Menschen nehmen an, daß alle Amateurklavierspieler Stücke vom Anfang bis zum Ende mit einem \hyperref[c1iii6d]{Hand-Gedächtnis} lernen, und daß die Fähigkeit Auszüge zu spielen ein besonderes Talent erfordert.
Fangen Sie mit kurzen Auszügen an, und versuchen Sie dann schrittweise längere.
Haben Sie dieses zwanglose Auszüge-Vorspielen erst einmal bei vier oder fünf verschiedenen Gelegenheiten gemacht, können Sie Ihre Fähigkeiten zum Vorspielen gut einschätzen.
Offensichtlich sollte das Spielen von Auszügen eines der Dinge sein, die sie regelmäßig \hyperref[c1iii6g]{\enquote{kalt} üben}.

Es gibt ein paar Regeln für die Vorbereitung auf das Auszüge-Vorspielen.
Spielen Sie kein Stück, das Sie gerade gelernt haben.
Lassen Sie es für mindestens sechs Monate schmoren; am besten ein Jahr (üben Sie während dieser Zeit das Auszüge-Vorspielen).
Wenn Sie gerade zwei Wochen damit verbracht haben, ein schwieriges neues Stück zu lernen, dann erwarten Sie nicht, daß Sie in der Lage sind Auszüge zu spielen, die Sie in diesen zwei Wochen überhaupt nicht gespielt haben -- seien Sie auf alle Arten von Überraschungen, wie z.B. Gedächtnisblockaden, vorbereitet.
Üben Sie die Auszüge an dem Tag, an dem Sie sie vorführen werden, nicht schnell.
Sie sehr langsam zu üben wird hilfreich sein.
Können Sie sie immer noch HS spielen?
Sie können eine Menge dieser Regeln bei sehr kurzen Auszügen brechen.
Prüfen Sie vor allen Dingen, ob Sie sie \hyperref[c1ii12mental]{in Gedanken (ohne das Klavier) spielen} können -- das ist der ultimative Test, ob Sie bereit sind.

Allgemein gesagt: Erwarten Sie nicht, daß Sie etwas gut darbieten können, egal ob informell oder nicht, solange Sie das Stück nicht mindestens dreimal vorgeführt haben; manche behaupten mindestens fünfmal.
Abschnitte, die Sie für einfach hielten, können sich als schwierig vorzuspielen erweisen und umgekehrt.
Deshalb ist der erste Punkt der Tagesordnung, daß Sie Ihre Erwartungen ein wenig verringern und anfangen zu planen, wie Sie das Stück spielen werden, besonders wenn etwas unerwartetes geschieht.
Es wird sicherlich nicht so wie Ihr bester Durchgang beim Üben werden.
Ohne diese mentale Vorbereitung kann es Ihnen passieren, daß Sie schließlich nach jedem Versuch, etwas vorzuspielen, enttäuscht sind und psychologische Probleme bekommen.

Ein paar Fehler oder fehlende Noten werden beim Üben nicht wahrgenommen, und Ihre Einschätzung darüber, wie es während des Übens klingt, ist wahrscheinlich viel optimistischer als Ihre eigene Beurteilung, wenn Sie auf die gleiche Art vor einem Publikum gespielt hätten.
Nach dem Üben neigt man dazu, sich nur an die guten Teile zu erinnern, und nach der Aufführung neigt man dazu, sich nur an die Fehler zu erinnern.
Normalerweise ist man selbst sein schlimmster Kritiker; jeder Ausrutscher klingt für einen selbst viel schlimmer als für das Publikum.
Meistens bekommt das Publikum die Hälfte der Fehler nicht mit und vergißt die meisten, die es mitbekommt, nach kurzer Zeit wieder.
Das zwanglose Vorspielen geht wesentlich entspannter vonstatten, und es bietet eine einfache Möglichkeit, Ihnen schrittweise den Weg zum formellen Auftreten zu ebnen und Sie so auf Konzerte vorzubereiten.

Klassische Musik ist für das formlose Vorspielen nicht immer die beste Wahl.
Deshalb sollte jeder Klavierspieler Pop-Musik, Jazz, Cocktail-Musik, Musik aus \enquote{Fake Books} und das Improvisieren lernen.
Das sind einige der besten Möglichkeiten, für formelle Konzerte zu üben.
Mehr dazu in Abschnitt 23.


\subsubsection{Vorbereitung auf Konzerte}
\label{c1iii14f}

Auch wenn ein Schüler während des Übens perfekt spielen kann, kann er während des Konzerts alle Arten von Fehlern machen und mit der Musikalität ringen, wenn die Vorbereitung inkorrekt ist.
Die meisten Schüler üben in der Woche vor dem Konzert und insbesondere am Tag des Konzerts intuitiv hart und mit voller Geschwindigkeit.
Um das Konzert zu simulieren, stellen sie sich ein Publikum vor und spielen sich die Seele aus dem Leib, indem sie das Stück mehrmals vom Anfang bis zum Ende spielen.
Diese Übungsmethode ist die größte Ursache von Fehlern und schlechten Auftritten.
Die vielsagendste Bemerkung, die ich oft höre, ist: \enquote{Merkwürdig, ich habe den ganzen Morgen so gut gespielt, aber während des Konzerts habe ich Fehler gemacht, die ich während des Übens nicht gemacht habe!}
Für einen erfahrenen Lehrer ist das ein Schüler, der ohne Kontrolle übt und ohne Anleitung über die richtigen und falschen Methoden zur Vorbereitung auf Konzerte.

Lehrer, die jene Konzerte veranstalten, in denen die Schüler wunderbar spielen, halten ihre Schüler an der kurzen Leine und kontrollieren sorgsam deren Übungsablauf.
Wozu die ganze Aufregung?
Weil während eines Konzerts das am meisten angespannte Element das Gehirn ist, nicht der Spielmechanismus.
Und diese Anspannung kann mit keiner Art von simuliertem Auftritt nachgebildet werden.
Deshalb muß das Gehirn für einen einmaligen Auftritt ausgeruht und voll geladen sein; es darf nicht dadurch entladen sein, daß man sich die Seele aus dem Leib gespielt hat.
Alle Fehler haben ihren Ursprung im Gehirn.
Die ganze notwendige Information muß in geordneter Weise, ohne Durcheinander, im Gehirn gespeichert sein.
Deshalb spielen nicht richtig vorbereitete Schüler während eines Konzerts immer schlechter als während des Übens.
Wenn man mit voller Geschwindigkeit übt, dann wird ein großes Maß an Unordnung in das Gedächtnis gebracht.
Die Umgebung ist beim Konzert anders als beim Üben und kann sehr ablenkend sein.
Deshalb müssen Sie ein einfaches, fehlerfreies Gedächtnis des Stücks haben, das trotz aller zusätzlichen Ablenkungen abgerufen werden kann.
Darum ist es schwierig, dasselbe Stück zweimal am selben Tag aufzuführen, ja sogar an aufeinanderfolgenden Tagen.
Die zweite Aufführung ist ausnahmslos schlechter als die erste, obwohl man intuitiv erwarten würde, daß die zweite Aufführung besser wäre, weil man eine zusätzliche Erfahrung in der Aufführung des Stücks hat.
Wie sonst in diesem Abschnitt, ist diese Art von Anmerkungen nur auf Schüler anwendbar.
Professionelle Musiker sollten in der Lage sein, alles zu jeder Zeit beliebig oft vorzuspielen; diese Fertigkeit kommt von den ständigen Auftritten und dem ständigen Feilen an den richtigen Regeln zur Vorbereitung.

Durch Versuch und Irrtum haben erfahrene Lehrer funktionierende Übungsabläufe gefunden.
Die wichtigste Regel ist, das Maß an Übung am Konzerttag zu begrenzen, damit der Geist frisch bleibt.
Das Gehirn ist am Konzerttag völlig unempfänglich.
Es kann nur durcheinandergeraten.
Nur eine kleine Minderheit erfahrener Klavierspieler hat genügend \enquote{starke} musikalische Gehirne, um am Konzerttag etwas neues aufzunehmen.
Das gilt übrigens auch für Tests und Examen in der Schule.
Meistens wird man in einem Examen besser abschneiden, wenn man am Abend vorher ins Kino geht, als wenn man versucht, sich etwas einzutrichtern.
\textbf{Ein typischer empfohlener Übungsablauf beim Klavierspielen ist, einmal fast mit voller Geschwindigkeit zu spielen, dann einmal mit mittlerer Geschwindigkeit und zum Schluß einmal langsam.}
Das war's! Kein weiteres Üben!
Spielen Sie nie schneller als mit Konzertgeschwindigkeit.
Beachten Sie, wie kontraintuitiv das ist.
Da Eltern und Freunde fast immer intuitive Methoden benutzen, ist es für den Lehrer wichtig, sicherzustellen, daß jeder, der mit dem Schüler zu tun hat, die Regeln ebenfalls kennt.
Das gilt insbesondere bei den jüngeren Schülern.
Ansonsten werden die Schüler, trotz allem was der Lehrer sagt, wenn sie zum Konzert kommen, den ganzen Tag mit voller Geschwindigkeit geübt haben, weil ihre Eltern es so wollten.

Selbstverständlich ist das nur der Anfang.
Der Ablauf kann an die Umstände angepaßt werden.
Dieser Ablauf ist für den typischen Schüler und nicht für professionelle Künstler gedacht, die weitaus detailliertere Abläufe haben, die nicht nur von der Art der gespielten Musik abhängen, sondern auch von dem bestimmten Komponisten oder dem bestimmten zu spielenden Stück.
Klar muß, damit dieser Ablauf funktioniert, das Stück einige Zeit vor dem Auftritt fertig sein.
Jedoch ist dies sogar dann der beste Ablauf für den Konzerttag, wenn das Stück noch nicht perfektioniert wurde und mit mehr Übung verbessert werden kann.
Wenn Sie einen Fehler machen, von dem Sie wissen, daß er hartnäckig ist und der fast mit Sicherheit während des Konzerts auftreten wird, fischen Sie die paar Takte heraus, die den Fehler enthalten, und üben Sie diese mit den angemessenen Geschwindigkeiten (enden Sie immer mit langsamem Spielen), wobei Sie schnelles Spielen so weit wie möglich vermeiden.
Wenn Sie sich nicht sicher sind, daß das Stück völlig auswendiggelernt ist, spielen Sie es mehrere Male sehr langsam.
Die Wichtigkeit eines sicheren \hyperref[c1ii12mental]{mentalen Spielens} muß noch einmal betont werden -- es ist der ultimative Test für das Gedächtnis und ob Sie zum Auftreten bereit sind.
Üben Sie das mentale Spielen mit jeder Geschwindigkeit und so oft Sie möchten; es kann auch ein nervöses Zittern beruhigen.

Vermeiden Sie auch extreme Anstrengungen, wie z.B. ein Fußballspiel oder etwas schweres zu heben oder zu schieben (wie z.B. einen Konzertflügel!).
Das kann plötzlich die Antwort Ihrer Muskeln auf ein Signal des Gehirns ändern, und Sie machen am Ende beim Spielen völlig unerwartete Fehler.
Selbstverständlich können leichte Aufwärmübungen, Dehnen, Gymnastik, Tai Chi, Yoga usw. sehr nützlich sein.

\textbf{Spielen Sie in der Woche vor dem Konzert immer mit mittlerer Geschwindigkeit und danach mit langsamer Geschwindigkeit, bevor Sie mit dem Üben aufhören}.
Sie können die langsame Geschwindigkeit durch die mittlere ersetzen, wenn Ihnen die Zeit knapp wird, das Stück besonders einfach ist oder wenn Sie ein erfahrenerer Künstler sind.
Übrigens ist diese Regel auf jede Übungseinheit anwendbar, aber sie ist vor einem Konzert besonders entscheidend.
Das langsame Spielen tilgt alle schlechten Angewohnheiten, die Sie eventuell angenommen haben, und stellt das entspannte Spielen wieder her.
Konzentrieren Sie sich deshalb während dieses mittleren bzw. langsamen Spielens auf die Entspannung.
Es gibt keine feste Zahl wie bei der halben Geschwindigkeit, um mittlere und langsame Geschwindigkeit zu definieren, obwohl die mittlere im allgemeinen ungefähr 3/4 der endgültigen Geschwindigkeit ist und die langsame ungefähr 1/2.
Allgemeiner gesagt: Mittlere Geschwindigkeit ist die Geschwindigkeit, mit der man bequem, entspannt und mit viel Zeitersparnis spielen kann.
Langsam ist die Geschwindigkeit, bei der Sie jeder einzelnen Note Beachtung schenken müssen.

Sie können bis zum letzten Tag vor dem Konzert an der Verbesserung des Stücks arbeiten -- besonders an der musikalischen.
Aber während der letzten Woche ist es nicht zu empfehlen, neues Material hinzuzufügen oder das Stück zu ändern (z.B. den Fingersatz), obwohl Sie es als Trainingsexperiment versuchen könnten, um zu sehen, wie weit Sie sich treiben können.
In der Lage zu sein, während der letzten Woche etwas Neues hinzuzufügen, ist ein Zeichen, daß Sie starke Fähigkeiten zum Auftreten haben; tatsächlich ist es ein gutes Training für das Auftreten, absichtlich etwas auf die letzte Minute zu ändern.
Vermeiden Sie beim Arbeiten an einem langen Stück, wie z.B. einer Beethoven-Sonate, es viele Male ganz durchzuspielen.
Es ist am besten, es in kleine Abschnitte von wenigen Seiten zu zerteilen und diese Abschnitte zu üben.
HS zu üben ist ebenfalls eine ausgezeichnete Idee, weil jeder sich immer technisch verbessern kann.
Obwohl zu schnelles Spielen in der letzten Woche nicht empfehlenswert ist, können Sie mit jeder Geschwindigkeit HS üben.
Vermeiden Sie es, während dieser letzten Woche neue Stücke zu lernen.
Das bedeutet nicht, daß Sie auf die Konzertstücke beschränkt sind; Sie können weiterhin jedes Stück üben, das Sie zuvor gelernt haben.
Neue Stücke werden oft dazu führen, daß Sie neue Fertigkeiten erwerben, die die Art, wie Sie das Konzertstück spielen, beeinflussen oder ändern.
Im allgemeinen werden Sie es nicht merken, daß dies geschehen ist, bis Sie das Konzertstück spielen und sich fragen, wie sich ein paar neue Fehler eingeschlichen haben.

Machen Sie es sich zur Angewohnheit, Ihre Konzertstücke \enquote{\hyperref[c1iii6g]{kalt}} (ohne Aufwärmen) zu spielen, wenn Sie eine Übungseinheit beginnen.
Die Hände werden sich nach einem oder zwei Stücken erwärmen, so daß Sie eventuell die Reihenfolge der Konzertstücke bei jeder Übungseinheit ändern müssen, wenn Sie viele Stücke spielen.
\enquote{Kalt spielen} muß natürlich innerhalb eines vernünftigen Rahmens stattfinden.
Wenn die Finger von der Untätigkeit völlig träge sind, können Sie nicht, und sollten es auch nicht versuchen, schwieriges Material mit der vollen Geschwindigkeit spielen; es wird zu Streß und sogar Verletzungen führen.
Einige Stücke können nur gespielt werden, wenn die Hände völlig aufgewärmt sind,
insbesondere, wenn man sie musikalisch spielen möchte.
Die Schwierigkeit, musikalisch zu spielen, darf jedoch keine Entschuldigung dafür sein, nicht kalt zu spielen, weil in diesem Fall der Aufwand wichtiger als das Ergebnis ist.
Sie müssen herausfinden, welche Stücke Sie kalt mit voller Geschwindigkeit spielen können und welche nicht.
Verringern Sie die Geschwindigkeit so weit, daß Sie mit kalten Händen spielen können; Sie können aber stets mit der endgültigen Geschwindigkeit spielen, nachdem die Hände aufgewärmt sind. 

Üben Sie mehrere Tage vor dem Konzert nur die ersten paar Takte.
Wann immer Sie Zeit haben, tun Sie so, als ob der Moment des Konzerts wäre, und spielen Sie die ersten paar Takte. 
Wählen Sie die ersten zwei bis fünf Takte, und üben Sie jedesmal eine andere Anzahl.
Halten Sie nicht am Ende eines Taktes an, sondern spielen Sie immer bis zur Note des nächsten Takts.



