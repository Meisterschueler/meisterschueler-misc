% File: c1iii11

\subsection{Blattspiel}
\label{c1iii11}

\textbf{Es ist nützlich, das Spielen vom Blatt in drei Stufen zu unterteilen, damit wir wissen, worüber wir sprechen}, weil der Begriff \enquote{Blattspiel} für verschiedene Vorgänge benutzt wird.
Auf der Anfängerstufe bedeutet Blattspiel, Kompositionen zu spielen, die wir nicht auswendig gelernt haben, und die wir spielen, während wir auf das Notenblatt sehen.
Wir sind vielleicht schon mit den Melodien der Komposition vertraut und haben sie bereits gespielt.
In der Mittelstufe können wir Musik vom Blatt spielen, die wir noch nicht kennen und die wir noch nicht geübt haben.
Diese Stufe wird im allgemeinen als das richtige Blattspiel angesehen und ist das Thema dieses Abschnitts.\footnote{Siehe auch \hyperref[c030530]{Prima-Vista-Spiel} im Quellenverzeichnis.}
Auf der fortgeschrittenen Stufe schließt das Blattspiel die Anwendung der grundlegenden Musiktheorie ein, wie z.B. Akkordprogressionen und Harmonien, sowie die Interpretation der Musik.
Es folgen die Grundregeln des Blattspiels (siehe auch \hyperref[Richman]{Richman}):

\begin{enumerate}[label={\arabic*.}] 
\item \textbf{Blicken Sie stets auf die Noten; schauen Sie nicht auf die Tastatur oder Finger.}
Schauen Sie gelegentlich auf die Hände, wenn es für große \hyperref[c1iii7f]{Sprünge} notwendig ist.
Versuchen Sie, einen \enquote{Randblick} auf die Tastatur entwickeln, so daß Sie eine ungefähre Vorstellung davon haben, wo die Hände sind, während Sie immer noch auf die Noten schauen.
Mit dem Randblick können Sie beide Hände gleichzeitig im Auge behalten.
\textbf{Gewöhnen Sie sich an, die Tasten vor dem Spielen zu erfühlen.}
Obwohl diese Regel unabhängig davon anwendbar ist, ob man vom Blatt spielt oder nicht, wird sie beim Blattspiel entscheidend.
Die Tasten vor dem Spielen zu erfühlen hilft auch, bei Sprüngen \enquote{vorzeitig in der richtigen Position zu sein}, s. Abschnitte \hyperref[c1iii7e]{7e} und \hyperref[c1iii7f]{7f} oben; deshalb sollten Sie das Üben der Sprungbewegungen mit dem Üben des Blattspiels verbinden.


\item \textbf{Spielen Sie durch Fehler hindurch, und machen Sie sie so unhörbar wie möglich.}
Am besten lassen Sie die Musik so klingen, als ob Sie etwas geändert hätten - dann weiß das Publikum nicht, ob Sie einen Fehler gemacht oder es geändert haben.
Deshalb haben Schüler mit einer Grundausbildung in Musiktheorie solch einen Vorteil beim Blattspiel.
Drei Möglichkeiten, Fehler weniger hörbar zu machen, sind:

\begin{enumerate}[label={\roman*.}] 
<li>den Rhythmus intakt halten
\item eine fortlaufende Melodie beibehalten (falls Sie nicht alles lesen können, behalten Sie die Melodie bei, und lassen Sie die Begleitung weg)
\item üben, die Teile zu vereinfachen, die zum Ablesen zu schwierig sind
 \end{enumerate}
Als erstes müssen Sie die Angewohnheit loswerden, bei jedem Fehler anzuhalten und zurückzugehen (Stottern), falls Sie diese Angewohnheiten bereits haben.
\textbf{Die beste Zeit, die Fertigkeit zu entwickeln, nicht bei jedem Fehler anzuhalten, ist, wenn Sie Ihre ersten Unterrichtsstunden beginnen.}
Wenn sich die Angewohnheit zu stottern erst einmal verfestigt hat, erfordert es viel Arbeit, sie zu eliminieren.
Für diejenigen mit der Angewohnheit zu stottern ist es das beste, sich dafür zu entscheiden, nie wieder zurückzugehen (egal, ob es gelingt oder nicht) - sie wird langsam anfangen zu verschwinden.
Spielfehler voraussehen zu lernen ist eine große Hilfe und wird weiter unten besprochen.
Das wirksamste Werkzeug ist die Fähigkeit, die Musik zu vereinfachen.
Eliminieren Sie Verzierungen, fischen Sie die Melodie aus schnellen Läufen, usw.

</li>
\item \textbf{Lernen Sie alle allgemeinen musikalischen Konstrukte: Alberti-Begleitungen, Dur- und Moll-\hyperref[c1iii5]{Tonleitern} und ihre \hyperref[table]{Fingersätze} genauso wie die zugehörigen \hyperref[Arpeggios]{Arpeggios}, einfache Akkorde und Akkordumstellungen, Triller, Verzierungen usw.}
Beim Spielen vom Blatt sollten Sie die Konstrukte erkennen und nicht die einzelnen Noten lesen.
Lernen Sie die Positionen der sehr hohen und sehr tiefen Noten auf dem Notenblatt auswendig, so daß Sie sie sofort erkennen können.
Lernen Sie zunächst die Oktav-Cs, und fügen Sie dann die anderen Noten hinzu, beginnend mit den Noten, die den Cs am nächsten sind.


\item \textbf{Schauen Sie dem voraus, was Sie spielen; mindestens einen Takt voraus oder auch mehr, wenn Sie die Fertigkeit entwickeln, die Musikstruktur zu lesen.}
Versuchen Sie, eine Struktur voraus zu lesen.
Durch das Vorausschauen können Sie sich nicht nur vorzeitig vorbereiten, sondern auch Spielfehler voraussehen, bevor sie auftreten.
Sie können auch Probleme mit dem Fingersatz vorhersehen und können vermeiden, sich selbst in unmögliche Situationen zu bringen.
Obwohl Vorschläge zum Fingersatz in den Noten im allgemeinen hilfreich und vielleicht die besten Fingersätze sind, sind sie oft nutzlos, weil Sie sie eventuell nicht ohne Übung benutzen können.
Deshalb sollten Sie Ihren eigenen Vorrat an Fingersätzen entwickeln.


\item \enquote{Üben, üben, üben}.
\textbf{Obwohl das Blattspiel relativ leicht zu lernen ist, muß es jeden Tag geübt werden, um es zu verbessern.
Bei den meisten Schülern erfordert es ein bis zwei Jahre} eifriges Üben, um gut darin zu werden.
Da das Spielen vom Blatt so stark vom Erkennen von Strukturen abhängt, ist es nah mit dem \hyperref[c1iii6]{Auswendiglernen} verwandt.
Das bedeutet, daß Sie die Fähigkeit zum Blattspiel verlieren können, wenn Sie aufhören zu üben.
Wie beim Auswendiglernen bleibt Ihnen die Fähigkeit jedoch ein Leben lang erhalten, wenn Sie bereits in jungen Jahren ein guter Blattspieler geworden sind.
Versuchen Sie nach dem Üben des Blattspiels, einige Strukturen des Stücks, die häufig vorkommen, in Gedanken zu spielen (siehe \hyperref[c1iii6j]{Abschnitt III.6j}).


\end{enumerate}
Sie sollten stets weitere \enquote{Verkaufstricks} hinzufügen, wenn Sie besser werden.
Üben Sie die Kunst, eine Komposition zu überfliegen, bevor Sie sie vom Blatt spielen, um ein Gefühl dafür zu bekommen, wie schwierig sie ist.
Dann können Sie bereits vorzeitig herausfinden, wie Sie um die \enquote{unmöglichen} Abschnitte herumkommen.
Sie können sie sogar schnell üben, indem Sie eine verdichtete Version der Lerntricks benutzen (HS, schwierige Abschnitte verkürzen, parallele Sets benutzen usw.) und zwar gerade so viel, um es passabel klingen zu lassen.
Ich habe Blattspieler getroffen, die sich eine Weile mit mir über einige Abschnitte eines neuen Stücks unterhielten und es dann ohne Schwierigkeiten durchspielten.
Ich habe später herausgefunden, daß sie diese Abschnitte in den paar Sekunden geübt hatten, in denen sie mich mit ihrer \enquote{Besprechung} abgelenkt hatten.

Nehmen Sie ein paar Bücher mit leichten Stücken.
Da es am Anfang leichter ist, das Blattspiel mit bekannten Stücken zu üben, können Sie die gleiche Komposition mehrere Male benutzen, um das Blattspiel zu üben, z.B. erneut nach einer Woche oder mehr.
Bücher mit \enquote{Sonatinas}, Mozarts leichtere Sonaten und Bücher für Anfänger mit leichten beliebten Liedern sind für das Üben gut geeignet.
Als leichteste Stücke könnten Sie Beyer benutzen, die Bücher für Anfänger in Abschnitt III.18c oder die leichtesten Stücke von Bach. 
Obwohl Sie mit bereits bekannten Stücken eine ganze Menge Fertigkeiten zum Spielen vom Blatt erlernen können, sollten Sie jedoch auch mit Stücken üben, die Sie nie zuvor gesehen haben, um die wahre Fertigkeit des Blattspiels zu entwickeln.
Die nützlichste Fertigkeit, die eine Hilfe beim wahren Blattspiel sein kann, ist das Singen vom Blatt, mit dem wir uns nun beschäftigen.


\subsection{Absolutes Gehör und relatives Gehör (vom Blatt singen)}
\label{c1iii12}

\textbf{Relatives Gehör ist die Fähigkeit, eine Note in bezug auf eine Vergleichsnote zu erkennen.
Absolutes Gehör ist die Fähigkeit, eine Note ohne eine vorgegebene Vergleichsnote zu erkennen.}
Die Qualität Ihres absoluten Gehörs wird dadurch bestimmt, wie genau Sie eine Tonhöhe wiedergeben können, wie schnell Sie eine Note erkennen können und wie viele Noten Sie erkennen können, wenn sie gleichzeitig gespielt werden.
Menschen mit einem guten absoluten Gehör können sofort (innerhalb von 3 bis 5 Sekunden) 10 gleichzeitig gespielte Noten erkennen.
Beim Standardtest für das absolute Gehör werden zwei Klaviere benutzt: Der Tester sitzt an dem einem und der Schüler an dem anderen Klavier, und der Schüler versucht, die vom Tester gespielten Noten zu wiederholen.
Wenn nur ein Klavier benutzt wird, benennt der Schüler die vom Tester gespielte Note (do, re, mi usw. oder C, D, E usw.).
Benutzen Sie in den folgenden Übungen zunächst CDE, weil die meisten Lehrbücher diese Namensgebung verwenden.
Es ist jedoch in Ordnung, wenn Sie do-re-mi benutzen, weil das für Sie besser funktioniert.
\textbf{Niemand wird mit einem relativen oder absoluten Gehör geboren; beides sind erlernte Fertigkeiten}, weil die chromatische Tonleiter eine menschliche Erfindung ist - es gibt keine physische Beziehung zwischen den Tonhöhen der chromatischen Tonleiter und der Natur.
Die einzige physische Verbindung zwischen der chromatischen Tonleiter und dem Ohr ist, daß beide mit einer logarithmischen Skala arbeiten, um einen großen Frequenzbereich abzudecken.
Wir wissen, daß das Ohr mit einer logarithmischen Skala arbeitet, weil Harmonien eine besondere Bedeutung haben, Harmonien Verhältnisse sind und Verhältnisse am leichtesten auf einer logarithmischen Skala zu handhaben sind.
Deshalb ist uns das Erkennen von Harmonien angeboren, obwohl uns ein absolutes Gehör nicht angeboren ist.
Der Effekt des logarithmischen menschlichen Hörens ist, daß das Ohr einen großen Unterschied in der Tonhöhe zwischen 40 und 42,4 Hz hört (ein Halbton oder 100 Cent) aber fast keinen Unterschied zwischen 2000 Hz und 2002,4 Hz (ungefähr 2 Cent) hört, also den gleichen Unterschied von 2,4 Hz.
Das menschliche Ohr reagiert auf alle Frequenzen innerhalb seines Bereichs und ist nicht von Geburt an auf eine absolute Skala geeicht.
Hierin unterscheidet es sich vom Auge, das auf Farbe auf einer absoluten Skala reagiert (jeder sieht ab seiner Geburt ohne Übung Rot als Rot, und diese Wahrnehmung ändert sich mit dem Alter nicht), weil das Erkennen der Farben durch chemische Reaktionen erreicht wird, die auf spezifische Wellenlängen des Lichts reagieren.
Einige Menschen, die bestimmte Tonhöhen mit bestimmten Farben verbinden, können ein absolutes Gehör durch die Farben erwerben, die von den Tönen hervorgerufen werden.
Sie eichen das Ohr auf einen absoluten Bezugswert.
Die meisten Schüler lernen das absolute Gehör durch das assoziative Gedächtnis.

\textbf{Absolutes und relatives Gehör lernt man in frühester Jugend am besten.}
Babys, die kein einziges Wort verstehen können, reagieren entsprechend auf eine beruhigende Stimme, ein Schlaflied oder ein Angst einflößendes Geräusch, was ihre Bereitschaft zu einer musikalischen Ausbildung zeigt.
\textbf{Die beste Möglichkeit für Kleinkinder, ein absolutes Gehör zu erwerben, ist, von Geburt an fast täglich einem gut gestimmten Klavier zuzuhören.}
Deshalb sollten alle Eltern, die ein Klavier besitzen, es gestimmt halten und in Gegenwart des Babys darauf spielen.
Dann sollten sie ab und zu prüfen, ob das Kind ein absolutes Gehör hat.
Dieser Test kann ausgeführt werden, indem man eine Note spielt (wenn das Kind nicht hinsieht) und es dann bittet, die Note auf dem Klavier zu finden.
Natürlich müssen Sie dem Kind zuerst den Tonumfang des Klavier erklären: Fangen Sie mit der C-Dur-Tonleiter in der Mitte des Klaviers an, und erklären Sie danach die Tatsache, daß alle anderen Noten mit dieser Tonleiter im Oktavabstand verbunden sind.
Wenn das Kind die Note nach ein paar Versuchen finden kann, hat es ein relatives Gehör; wenn es die Note jedesmal sofort findet, hat es ein absolutes Gehör.
Die jeweilige Stimmung des Klaviers (\hyperref[c2_6_et]{gleichschwebend}, \hyperref[c2_2_wtk2]{wohltemperiert} usw.) ist nicht wichtig; in der Tat wissen die meisten Menschen mit absolutem Gehör nichts über Stimmungen, und wenn Noten auf Klavieren mit unterschiedlichen Stimmungen gespielt werden, haben sie keine Schwierigkeiten, die Noten zu erkennen, weil sich die meisten Frequenzen durch die verschiedenen Stimmungen um weniger als 5\% ändern und niemand ein absolutes Gehör mit einer solchen Genauigkeit hat.
Absolutes und relatives Gehör können später im Leben erworben werden, aber es wird in einem höheren Alter als 20 bis 30 schwieriger.
Tatsächlich \textbf{fangen diejenigen mit absolutem Gehör ungefähr im Alter von 20 Jahren an, es langsam wieder zu verlieren, wenn es nicht gepflegt wird.}
Viele Klavierschulen lehren allen Schülern routinemäßig ein absolutes Gehör mit einer Erfolgsquote von über 90\%.
Das Problem beim Unterrichten einer Gruppe älterer Schülern ist, daß es meistens einen gewissen Prozentsatz \enquote{tonhöhenbehinderte} Schüler gibt, die nie auf Tonhöhen trainiert wurden und deshalb sogar eventuell Schwierigkeiten haben, ein relatives Gehör zu erlernen.
Eine Anleitung, wie man sehr jungen Kindern ein absolutes Gehör lehrt, finden Sie in \hyperref[c1iii16c]{Abschnitt 16c}, weil es ziemlich einfach ist und ein integraler Bestandteil des Unterrichts für sehr junge Kinder; eine Anleitung für Erwachsene folgt weiter unten in diesem Abschnitt.

\textbf{Ein absolutes Gehör zu haben, ist sicherlich von Vorteil.}
Es ist eine große Hilfe für das \hyperref[c1iii6]{Auswendiglernen}, \hyperref[c1iii11]{Blattspiel}, beim Überwinden von Erinnerungsblockaden und für das Komponieren.
Sie können eine Stimmpfeife für Ihren Chor sein und leicht eine Geige oder ein Blasinstrument stimmen.
Es macht viel Spaß, weil Sie sagen können, wie schnell ein Auto fährt, wenn Sie das Singen der Reifen hören.
Sie können den Unterschied zwischen verschiedenen Autohupen und Pfeifen von Lokomotiven insbesondere dadurch erkennen, ob sie Terzen oder Quinten benutzen.
Sie können sich Telefonnummern leicht anhand ihrer Töne merken.
Es gibt jedoch Nachteile.
Musik, die außerhalb der Stimmung gespielt wird, kann unangenehm sein.
Da so viel Musik außerhalb der Stimmung gespielt wird, kann das ein ziemliches Problem darstellen.
Manchmal können Menschen heftig auf solch eine Musik reagieren; körperliche Reaktionen wie z.B. tränende Augen oder feuchte Haut können auftreten.
Transponierte Musik ist in Ordnung, weil jede Note immer noch korrekt ist.
Es wird schwierig, verstimmte Klaviere zu spielen.
\textbf{Ein absolutes Gehör ist ein zweischneidiges Schwert.}

\textbf{Es gibt eine Methode, die das Erlernen des absoluten und relativen Gehörs beschleunigt und vereinfacht!}
Diese Methode wird im allgemeinen nicht an Musikschulen oder in der Literatur gelehrt, obwohl sie von denjenigen mit einem absoluten Gehör seit den Anfängen der Musik benutzt wurde (meistens ohne zu wissen, wie sie es erworben haben).
Mit der hier beschriebenen Methode werden die Fähigkeiten zur Erkennung von Tonhöhen einfach zu Nebenprodukten des \hyperref[c1iii6]{Auswendiglernens}.
Es erfordert wenig zusätzlichen Aufwand, sich die Tonhöhenerkennung anzueignen, weil das Auswendiglernen ohnehin erforderlich ist, wie in Abschnitt III.6 erklärt wurde.
In diesem Abschnitt haben wir gesehen, daß das endgültige Ziel des Auswendiglernens die Fähigkeit ist, die Musik in Gedanken zu spielen (\hyperref[c1iii6tastatur]{mentales Spielen}).
Wenn man während des Übens des mentalen Spielens auf die relative und absolute Tonhöhen achtet, dann erwirbt man die Fertigkeiten zur Tonhöhenerkennung wie von selbst!
\textbf{Spielen Sie die Musik nicht nur in Gedanken, sondern spielen Sie sie auch mit der korrekten Tonhöhe.}
Das ist absolut sinnvoll, weil Sie, wenn Sie nicht mit der korrekten Tonhöhe spielen, viele der Vorteile des mentalen Spielens verschenken.
Umgekehrt wird das mentale Spielen nicht gut funktionieren, wenn es nicht mit der absoluten Tonhöhe ausgeführt wird, weil das mentale Spielen eine Funktion des Gedächtnisses, das Gedächtnis assoziativ und die absolute Tonhöhe eine der wichtigsten Assoziationen ist - die absolute Tonhöhe gibt der Musik ihre wahre Melodielinie, Farbe, Ausdruck usw.
Für die meisten genügt das Auswendiglernen von zwei bedeutsamen Kompositionen, um sich ein absolutes Gehör mit der Auflösung eines Halbtons anzueignen, was schneller ist als jede heute gelehrte Methode; bei den meisten sollte das ein paar Wochen oder ein paar Monate dauern.
Junge Kinder werden das ohne zusätzlichen Aufwand, fast automatisch, erreichen (s. \hyperref[c1iii16c]{Abschnitt 16c}); wenn man älter wird, wird man wegen all der verwirrenden Klänge, die bereits im Gedächtnis gespeichert sind, mehr Aufwand benötigen.

Zwei nützliche Kompositionen für das Üben des relativen und absoluten Gehörs sind Bachs Invention \#1 und der erste Satz von Beethovens Mondscheinsonate.
Durch Bach erhalten Sie das mittlere C (die erste Note des Stücks) und die C-Dur-Tonleiter; das sind die nützlichste Note und Tonleiter, die man mit den absoluten Tonhöhen lernen kann.
Die Mondscheinsonate hat bezaubernde Melodien, die das Auswendiglernen einfach und unterhaltsam machen.
Außerdem erzeugen die komplexen Modulationen der Tonarten eine Vielzahl von Noten und Intervallen, und die Komplexität verhindert, daß man die Noten erraten kann - es erfordert ein beträchtliches Maß an Übung und Wiederholungen, bevor man das Stück perfekt in Gedanken spielen kann.
Es ist auch für jeden technisch hinreichend einfach.
Beide Kompositionen sollten zunächst für das Üben der Tonhöhen HS geübt werden und erst später HT.

Wenn Sie sich die Noten vorstellen, versuchen Sie nicht, sie zu summen oder zu singen, weil der Tonumfang des Klaviers viel größer als der Ihrer Stimmbänder ist und Sie Ihr Gedächtnis darauf trainieren müssen, mit diesen höheren und tieferen Noten zurechtzukommen.
Auch muß das Abbild jeder Note im Gedächtnis zunächst alles einschließen - die Obertöne, das Timbre und andere Eigenschaften Ihres Klaviers.
Sie brauchen so viele Assoziationen wie möglich, um den Gedächtnisprozeß zu beschleunigen.
Benutzen Sie deshalb dasselbe Klavier, bis Sie das Gefühl haben, sie haben das absolute Gehör, und versuchen Sie sich jede Eigenschaft des Klangs Ihres Klaviers einzuprägen.
Wenn Sie kein elektronisches Klavier besitzen, sorgen Sie dafür, daß das Klavier gestimmt ist.
Wenn Sie ein sicheres absolutes Gehör erworben haben, wird es mit jeder Tonquelle funktionieren.
Solange Sie kein ausgebildeter Sänger sind, der mit der richtigen Tonhöhe singen kann (in diesem Fall müßten Sie keine Erkennung der absoluten Tonhöhen üben), werden Sie die Tonhöhen nicht genau singen können.
Der daraus resultierende falsche Klang wird das Gehirn verwirren und jegliche Fähigkeit zur Tonhöherkennung zerstören, die Sie vielleicht schon erworben haben.
So wie das Spielen in Gedanken den Klavierspieler von den Einschränkungen durch das Klavier befreit, befreit Sie das mentale Vorstellen der Tonhöhe (im Gegensatz zum Singen) von den Beschränkungen der Stimmbänder.


\paragraph{Verfahren zum Lernen der relativen und absoluten Tonhöhenerkennung}
\label{c1iii12tonhoehe}

Nachdem Sie das Stück von Bach vollständig auswendiggelernt haben und das ganze Stück vollständig in Gedanken spielen können, beginnen Sie damit, die relative Tonhöhenerkennung zu lernen.
Spielen Sie die erste Note (C4) auf dem Klavier, benutzen Sie sie als Referenz, um die ersten ein oder zwei Takte in Gedanken zu spielen, und prüfen Sie die letzte Note mit dem Klavier.
Die meisten Anfänger werden sich die Intervalle fast richtig vorstellen, weil das Gehirn automatisch versucht, Schritte in gesungenen Intervallen zu machen.
Aufsteigende Noten werden so zu tief und absteigende Noten zu hoch gesungen.
Beginnen Sie mit einem oder zwei Takten, korrigieren Sie alle Fehler, und wiederholen Sie es so lange, bis die Fehler verschwinden.
Fügen Sie dann mehr Takte hinzu, usw.
Wenn Sie das ganze Stück auf diese Art durchgearbeitet haben, sollte Ihre relative Tonhöhenerkennung ziemlich gut sein.
Fangen Sie dann mit der absoluten Tonhöhenerkennung an.
Spielen Sie die ersten paar Takte ohne Referenznote vom Klavier in Gedanken, und prüfen Sie anschließend, ob Ihre Vorstellung von C4 richtig ist.
Jeder hat seine höchste und tiefste Note, die er summen kann.
Summen Sie nun zunächst ohne das Klavier bis zur maximalen Note aufwärts und dann bis zur minimalen Note abwärts; prüfen Sie danach noch einmal Ihre Vorstellung von C4 mit dem Klavier.
Wiederholen Sie das, bis Ihr C4 maximal um einen Halbton abweicht.
Ab diesem Punkt hängt der weitere Fortschritt von der Übung ab; versuchen Sie jedes Mal, wenn Sie am Klavier vorbeikommen, sich das C4 vorzustellen (indem Sie die ersten Takte der Invention \#1 benutzen), und prüfen Sie es.
Sie können das C4 direkt finden, indem Sie sich darauf konzentrieren, wie es auf dem Klavier klingt, aber mit richtiger Musik es ist einfacher, weil Musik aus mehr Assoziationen besteht.
Wenn das C4 ziemlich korrekt ist, beginnen Sie damit, Noten einer beliebigen Stelle des Klaviers zu testen, und raten Sie, welche es sind (nur die weißen Tasten).
Am Anfang werden Sie vielleicht ziemlich daneben liegen.
Es gibt einfach zu viele Noten auf dem Klavier.
Um die Erfolgsquote zu erhöhen, raten Sie die Noten durch das Herstellen des Bezugs zur Oktave von C4 bis C5; so ist z.B. C2 wie C4, nur zwei Oktaven tiefer.
Auf diese Art reduzieren Sie die Aufgabe, 88 Noten auf der Tastatur auswendig zu lernen, auf das Lernen von lediglich 8 Noten und eines Intervalls (Oktave).
Diese Vereinfachung ist möglich, weil die \hyperref[c2_2]{chromatische Tonleiter} logarithmisch aufgebaut ist; eine weitere Vereinfachung der Noten innerhalb der Oktave wird durch die Intervalle erreicht (Halbton, Terz, Quarte, Quinte).
Machen Sie sich mit allen Noten des Klaviers vertraut, indem Sie sie oktavweise spielen und das Gehirn darauf trainieren, alle Noten im Oktavabstand zu erkennen - alle Cs, Ds usw.
Bis Sie sich ein rudimentäres absolutes Gehör antrainiert haben, üben Sie die absolute Tonhöhenerkennung am Klavier, so daß Sie sich sofort korrigieren können, wenn Sie von der richtigen Tonhöhe abweichen.
Üben Sie nicht längere Zeit in Gedanken mit der falschen Tonhöhe; Sie sollten immer das Klavier in der Nähe haben, um sich selbst zu korrigieren.
Beginnen Sie erst mit dem Üben ohne Klavier, wenn Ihre absolute Tonhöhenerkennung höchstens um zwei Halbtöne abweicht.

Lernen Sie dann den ganzen ersten Satz der Mondscheinsonate auswendig, und fangen Sie an, mit den schwarzen Tasten zu arbeiten.
Der Erfolg mit der absoluten Tonhöhenerkennung hängt davon ab, wie Sie sich überprüfen.
Denken Sie sich mehrere Möglichkeiten aus; ich zeige Ihnen ein paar Beispiele.
Benutzen wir zunächst die ersten drei Noten der RH der Mondscheinsonate.
Merken Sie sich den absoluten Klang dieser Noten, und prüfen Sie ihn mehrere Male am Tag.
Prüfen Sie, ob Sie jedesmal die erste Note (G\#3) richtig treffen, wenn Sie am Klavier sind.
Üben Sie die relative Tonhöhe, indem Sie die zweite Note prüfen - C\#4, eine Quarte von G\#3 -, gehen Sie dann in Gedanken einen Halbton nach unten zum C4, und prüfen Sie wieder.
Gehen Sie zur dritten Note, E4, prüfen Sie sie, dann in Gedanken abwärts zu C4, und prüfen Sie.
Von G\#3 in Gedanken einen Halbton abwärts, dann aufwärts zu C4.
Springen Sie nun zu einer beliebigen Stelle dieses Satzes, und wiederholen Sie den Vorgang in ähnlicher Weise.

Der Fortschritt mag Ihnen zunächst langsam vorkommen, aber Ihre Vorstellungen sollten mit zunehmender Übung dem richtigen Klang immer näher kommen.
Am Anfang braucht das Identifizieren der Noten seine Zeit, weil Sie Ihre Vorstellung durch Summen zu Ihrer höchsten oder tiefsten Note überprüfen müssen oder indem Sie sich an den Anfang der Invention oder der Mondscheinsonate erinnern.
Eines Tages sollten Sie dann plötzlich die wundervolle Erfahrung machen, daß Sie jede Note direkt, ohne Zwischenschritte,  identifizieren können.
Sie haben das wahre absolute Gehör erworben!
Dieses anfängliche absolute Gehör ist zerbrechlich, und Sie können es mehrmals verlieren und wieder zurückgewinnen.
Der nächste Schritt ist, Ihr absolutes Gehör zu stärken, indem Sie üben, die Noten so schnell wie möglich zu identifizieren.
Die Stärke Ihres absoluten Gehörs wird von der Geschwindigkeit bestimmt, mit der Sie Noten identifizieren können.
Beginnen Sie danach mit zwei gleichzeitig gespielten Noten, dann mit Akkorden aus drei Noten usw.
Wenn Sie ein starkes absolutes Gehör haben, üben Sie, die Noten mit der richtigen Tonhöhe zu summen und zu singen und das Singen vom Blatt mit der richtigen Tonhöhe.
Gratulation, Sie haben es geschafft!

Der biologische Mechanismus, der dem absoluten Gehör zugrunde liegt, wird noch nicht ganz verstanden.
Er scheint vollständig eine Funktion des Gedächtnisses zu sein.
Um das absolute Gehör wirklich zu erwerben, müssen Sie deshalb Ihre geistigen Gewohnheiten ändern, so wie Sie es tun müssen, um ein guter Auswendiglernender zu werden.
Beim Auswendiglernen war die notwendige Veränderung, die Angewohnheit zu entwickeln, ständig Assoziationen zu erfinden (je ungeheuerlicher oder schockierender, desto besser!) und sie automatisch im Gehirn zu wiederholen.
Bei guten Auswendiglernenden geschieht dieser Vorgang von selbst, oder ohne Aufwand, und deshalb sind sie gut.
Die Gehirne von schlechten Auswendiglernenden werden entweder untätig, wenn Sie nicht gebraucht werden, oder schweifen zu logischen oder anderen Interessen ab, statt für das Auswendiglernen zu arbeiten.
Menschen mit absolutem Gehör neigen dazu, ständig in Gedanken Musik zu machen; in ihren Köpfen spielt ständig Musik, egal ob es ihre eigene Kompositionen sind oder Musik, die sie gehört haben.
Deshalb beginnen die meisten Menschen mit absolutem Gehör automatisch damit, Musik zu komponieren.
Das Gehirn kehrt immer zur Musik zurück, wenn es mit nichts anderem beschäftigt ist.
Das ist wahrscheinlich eine Voraussetzung dafür, ein permanentes absolutes Gehör zu erwerben.
Beachten Sie, daß ein absolutes Gehör Sie nicht zu einem Komponisten macht, das mentale Spielen tut es.
Deshalb ist das mentale Spielen viel wichtiger als ein absolutes Gehör; diejenigen mit einem starken mentalen Spielen können leicht die absolute Tonhöhenerkennung lernen, sie aufrechterhalten und alle in diesem Abschnitt besprochenen Vorteile genießen.
Wie beim Auswendiglernen ist der schwerste Teil des Erwerbs eines permanenten absoluten Gehörs nicht das Üben, sondern die Änderung Ihrer geistigen Gewohnheiten.
Im Prinzip ist es einfach: Spielen Sie soviel sie können in Gedanken, und überprüfen Sie es für das absolute Gehör am Klavier.

Im Rahmen der Gedächtnispflege muß man sich regelmäßig mit dem absoluten Gehör und dem Auswendiglernen mit Hilfe des mentalen Spielens beschäftigen.
Diese Vorgehensweise pflegt automatisch die Tonhöhenerkennung; prüfen Sie einfach hin und wieder, ob Ihr mentales Spielen noch mit der richtigen Tonhöhe arbeitet.
Das sollte ebenfalls automatisch geschehen, weil Sie zumindest den Anfang jedes Stücks mental spielen sollten, bevor Sie es am Klavier spielen.
Indem Sie es zuerst in Gedanken spielen stellen Sie sicher, daß die Geschwindigkeit, der Rhythmus und der Ausdruck korrekt sind.
Musik klingt aufregender, wenn sie mental geführt wird, und weniger aufregend, wenn man sie spielt und darauf wartet, daß das Klavier die Musik erzeugt.
Die Kombination des absoluten Gehörs, mentalen Spielens und Tastatur-Gedächtnisses führt zu einem mächtigen Satz an Werkzeugen, die das Komponieren von Musik einfach werden lassen, sowohl das Komponieren in Gedanken als auch das Spielen auf dem Klavier.

Konventionelle Methode für das Lernen der absoluten Tonhöhenerkennung benötigen viel Zeit, oft mehr als sechs Monate und üblicherweise einiges mehr, und das resultierende absolute Gehör ist schwach.
Eine Möglichkeit anzufangen ist, sich eine Note zu merken.
Sie könnten z.B. das A mit 440 Hz nehmen, weil Sie es jedesmal hören, wenn Sie in ein Konzert gehen und sich vielleicht am einfachsten daran erinnern können.
Das A ist jedoch keine brauchbare Note, um zu den verschiedenen Intervallen der C-Dur-Tonleiter zu kommen, die die nützlichste Tonleiter ist, die man sich merken sollte.
Wählen Sie deshalb das C, E oder G, je nachdem, welches Sie sich am besten merken können; C ist wahrscheinlich am besten.
Die übliche Vorgehensweise für das Lernen der absoluten Tonhöhenerkennung ist in Musikschulen das Solfège (Gesangsübungen).
Solfège-Bücher können über den Buchhandel oder das Internet bezogen werden.
Es besteht aus zunehmend komplexen Folgen von Übungen mit verschiedenen Tonleitern, Intervallen, Taktarten, Rhythmen, Vorzeichen usw. für das Gesangstraining.
Es deckt auch die Tonhöhenerkennung und Diktate ab.
Solfège-Bücher verwendet man am besten in einer Klasse mit einem Lehrer.
Die absolute Tonhöhenerkennung wird als Nebenprodukt zu den Übungen gelehrt, indem man lernt, diese mit der richtigen Tonhöhe zu singen.
Deshalb gibt es für den Erwerb des absoluten Gehörs keine speziellen Methoden - man wiederholt alles so lange, bis die richtige Tonhöhe im Gedächtnis verankert ist.
Da die absolute Tonhöhenerkennung mit vielen anderen Dingen zusammen gelernt wird, ist der Fortschritt langsam.

Kurz gesagt, muß jeder Klavierspieler die absolute Tonhöhenerkennung lernen, weil es so einfach, nützlich und in vielen Situationen sogar notwendig ist.
Wir haben oben gezeigt, daß die absolute Tonhöhenerkennung mit Musik einfacher zu lernen ist als durch reines Auswendiglernen.
Die absolute Tonhöhenerkennung ist untrennbar mit dem mentalen Spielen verbunden, was Sie von den mechanischen Beschränkungen von Musikinstrumenten befreit.
Diese Fähigkeiten zum mentalen Spielen und zur absoluten Tonhöhenerkennung qualifizieren Sie gemäß der gängigen Vorstellungen automatisch als \enquote{talentiert} oder sogar als \enquote{Genie}, aber eine solche Beurteilung ist hauptsächlich für das Publikum wichtig; für Sie selbst ist es beruhigend zu wissen, daß Sie Fähigkeiten erworben haben, die notwendig sind, um ein vollendeter Musiker zu werden.


\paragraph{Vom Blatt singen und komponieren}
\label{c1iii12blatt}

Ein relatives und absolutes Gehör befähigen Sie nicht automatisch dazu, Musik, die Sie gerade gehört haben, sofort niederzuschreiben oder auf dem Klavier zu spielen.
Diese Fertigkeiten müssen genauso geübt werden, wie Sie für die Technik, das Blattspiel oder das Auswendiglernen üben, und braucht eine Weile, bis Sie es gelernt haben; ein relatives und absolutes Gehör zu entwickeln sind die erste Schritte zum Erreichen dieser Ziele.
Um in der Lage zu sein, ein Stück oder Ihre Komposition niederzuschreiben, ist es offensichtlich notwendig, Diktate zu lernen und zu üben.
Ein schneller Weg, Diktate zu üben, ist, das Singen vom Blatt zu üben.
Nehmen Sie ein beliebiges Stück, lesen Sie ein paar Takte, und singen Sie es oder spielen es in Gedanken (nur eine Stimme).
Prüfen Sie es anschließend am Klavier.
Wenn Sie das mit genügend Musik durchführen, die Sie nie zuvor gehört haben, dann werden Sie das Singen vom Blatt lernen und den größten Teil der Fertigkeiten entwickeln, die Sie für Diktate brauchen.
Um zu üben, jede Melodie auf dem Klavier zu spielen, üben Sie das Blattspiel.
Wenn Sie im Blattspiel ziemlich gut geworden sind (das wird mehr als sechs Monate benötigen), fangen Sie an, Ihre eigenen Melodien auf dem Klavier zu spielen.
Der Zweck für das Lernen des Blattspiels ist, daß Sie sich mit allgemeinen Läufen, Akkorden, Begleitungen usw. vertraut machen, so daß Sie sie schnell auf dem Klavier finden können.
Eine andere Möglichkeit ist, mit dem Spielen nach \enquote{Fake Books} anzufangen und das Improvisieren (Abschnitt 23) zu lernen.
Machen Sie sich beim Komponieren keine Sorgen, wenn Sie es zunächst schwierig finden, ein Stück anzufangen oder es zu beenden - das sind die schwierigsten Elemente des Komponierens.
Sammeln Sie nur einige Ideen, die Sie später zu einer Komposition zusammenfügen können.
Sorgen Sie sich nicht darum, daß Sie nie Unterricht im Komponieren hatten; es ist am besten, zuerst den eigenen Stil zu entwickeln und dann das Komponieren zu lernen, um den eigenen Stil weiterzuentwickeln.
Die Musik kommt nie \enquote{auf Befehl}, was frustrierend sein kann; deshalb müssen Sie, wenn die Ideen auftauchen, sofort an ihnen arbeiten.
Musik anzuhören, die Sie mögen, oder an einem guten Konzertflügel zu komponieren, kann inspirierend sein.
Obwohl Digitalpianos für das Komponieren von Popmusik und das Üben von Jazz-Improvisationen ausreichend sind, kann ein qualitativ guter Flügel sehr hilfreich sein, wenn man klassische Musik auf hoher Ebene komponiert.



