% File: c26

\section{Stimmverfahren}
\label{c2_6}

\subsection{Einleitung}
\label{c2_6a}

Stimmen besteht aus dem \enquote{Einstellen der Bezugsnoten} in einer Oktave in der Nähe des mittleren C und daraus, diese Oktave in geeigneter Weise auf alle anderen Tasten zu \enquote{kopieren}.
Sie werden verschiedene harmonische Stimmungen benötigen, um die Bezugsnoten einzustellen, und zunächst wird nur die mittlere Saite jeder Note der \enquote{Bezugsoktave} gestimmt.
Das \enquote{Kopieren} wird durch das Stimmen in Oktaven durchgeführt.
Wenn eine Saite jeder Note auf diese Art gestimmt ist, werden die restlichen Saiten jeder Note unisono gestimmt.

Beim Einstellen der Bezugsnoten müssen wir uns entscheiden, welche Temperatur wir benutzen möchten.
Wie oben in \hyperref[c2_2]{\autoref{c2_2}} erklärt wurde, sind die meisten Klaviere heutzutage auf \hyperref[et1]{gleichschwebende Temperatur (ET)}\index{gleichschwebende Temperatur (ET)} gestimmt, aber die historischen Temperaturen, insbesondere die \hyperref[c2_2_wtk2]{Wohltemperierten Stimmungen (WT)}\index{Wohltemperierten Stimmungen (WT)} erfreuen sich zunehmender Beliebtheit.
Deshalb habe ich ET und eine WT, \hyperref[c2_6_kirn]{Kirnberger II (K-II)}\index{Kirnberger II (K-II)}, für dieses Kapitel ausgewählt.
K-II ist eine der am leichtesten zu stimmenden Temperaturen; deshalb werden wir diese zuerst ansehen.
Die meisten, die nicht mit den verschiedenen Temperaturen vertraut sind, werden zunächst keinen Unterschied zwischen ET und K-II bemerken; sie werden beide im Vergleich zu einem verstimmten Klavier hervorragend klingen.
Auf der anderen Seite sollten die meisten Klavierspieler einen deutlichen Unterschied hören und in der Lage sein, eine Meinung oder eine Vorliebe zu entwickeln, wenn man ihnen bestimmte Musikstücke vorspielt und die Unterschiede aufzeigt.
Der einfachste Weg für Außenstehende, sich die Unterschiede anzuhören, ist, ein modernes elektronisches Klavier zu benutzen, das alle diese Temperaturen eingebaut hat, und dasselbe Stück mit jeder der Temperaturen zu spielen.
Benutzen Sie als ein leichtes Teststück z.B. den ersten Satz von Beethovens Mondscheinsonate; als ein schwierigeres Stück können Sie den dritten Satz seiner Waldstein-Sonate benutzen.
Probieren Sie auch ein paar Ihrer Lieblingsstücke von Chopin aus.
Mein Vorschlag für einen Anfänger ist, zuerst K-II zu lernen, so daß man ohne zu viele Schwierigkeiten anfangen kann, und dann ET zu lernen, wenn man schwierigere Aufgaben in Angriff nehmen kann.
Ein Nachteil dieses Plans ist, daß man eventuell K-II so sehr gegenüber ET bevorzugt, daß man sich nie dazu entschließt, ET zu lernen.
Wenn man sich an K-II gewöhnt hat, wird ET ein wenig ungenügend oder \enquote{schmutzig} klingen.
Man kann jedoch nicht wirklich als Stimmer angesehen werden, bevor man nicht ET stimmen kann.
Auch gibt es viele WTs, auf die Sie vielleicht einen Blick werfen möchten, die in verschiedener Hinsicht K-II überlegen sind.

WT-Stimmungen sind wünschenswert, weil sie perfekte Harmonien haben, die der Kern der Musik sind.
Sie haben jedoch einen großen Nachteil.
Weil die perfekten Harmonien so schön sind, treten die Dissonanzen in den \enquote{Wolfs}-Tonleitern hervor und sind sehr unangenehm.
Nicht nur das, sondern jede Saite, die ein wenig aus der Stimmung ist, ist sofort zu erkennen.
Deshalb erfordern WT-Stimmungen ein viel häufigeres Stimmen als ET.
Man könnte meinen, daß ein leichtes Verstimmen der Unisono-Saiten bei ET genauso unangenehm wäre, aber offenbar sind, wenn die Intervalle wie bei der ET aus der Stimmung sind, die geringfügigen Abweichungen in der Stimmung der Unisono-Saiten bei ET weniger wahrnehmbar.
Deshalb kann für Klavierspieler, die ein sensibles Gehör für das Stimmen haben, WT ziemlich unangenehm sein, solange sie ihr Klavier nicht selbst stimmen können.
Das ist ein wichtiger Punkt, weil die meisten Klavierspieler, die die Vorteile der WT hören können, empfindlich auf die Stimmung reagieren.
Die Erfindung des selbststimmenden Klaviers kann vielleicht der Retter der WT sein, weil das Klavier immer richtig gestimmt sein wird.
Deshalb wird WT eventuell nur durch elektronische und selbststimmende Klaviere (wenn sie verfügbar werden -- s. \hyperref[c1iv6h]{Abschnitt IV.6h \enquote{Die Zukunft des Klaviers}}) eine breite Zustimmung finden.

Sie können das Stimmen in ET überall beginnen, aber die meisten Stimmer benutzen die Stimmgabel A440 um anzufangen, weil Orchester im allgemeinen nach A440 stimmen.
Das Ziel bei K-II ist, C-Dur und so viele Tonarten \enquote{in der Nähe} wie möglich rein zu haben (mit reinen Intervallen), weshalb das Stimmen mit dem mittleren C (C4 = 261,6 -- die meisten Stimmer benutzen die C523,3-Stimmgabel um das mittlere C zu stimmen) begonnen wird.
Nun ist das aus K-II resultierende A, wenn man vom richtigen C aus stimmt, nicht das A440.
Deshalb benötigen Sie zwei Stimmgabeln (A und C), um sowohl ET als auch K-II stimmen zu können.
Alternativ können Sie nur mit einer C-Gabel beginnen und fangen das Stimmen in ET bei C an.
Zwei Stimmgabeln zu haben ist ein Vorteil, denn egal ob Sie von C oder von A aus starten, können Sie sich selbst überprüfen, wenn Sie bei ET bei der anderen Note ankommen.


\label{c2_6b}
\subsection{Das Klavier nach der Stimmgabel stimmen}
\label{c2_6_gabe}

Einer der schwierigsten Schritte beim Stimmvorgang ist das Stimmen des Klaviers nach der Stimmgabel.
Diese Schwierigkeit hat zwei Ursachen:

\begin{enumerate}[label={\arabic*.}] 
 \item Die Stimmgabel hat eine andere -- üblicherweise kürzere -- Aushaltezeit (Sustain) für den Ton als das
 Klavier, so daß die Gabel ausklingt, bevor Sie einen genauen Vergleich machen können.
 \item Die Gabel erzeugt eine reine Sinuswelle ohne die lauten Obertöne der Klaviersaiten.
\end{enumerate}

Deshalb kann man keine Schwebungen mit höheren Obertönen benutzen, um die Genauigkeit des Stimmens zu erhöhen, wie man es mit zwei Klaviersaiten tun kann.
Ein Vorteil von elektronischen Stimmgeräten ist, daß sie so programmiert werden können, daß sie Referenztöne mit Rechteckschwingungen liefern, die eine große Anzahl von höheren harmonischen Obertönen beinhalten.
Diese hohen harmonischen Obertöne (die notwendig sind, um die scharfen Ecken der Rechteckschwingungen zu erzeugen) sind für eine höhere Stimmgenauigkeit nützlich.
Wir müssen deshalb diese beiden Probleme lösen, damit wir das Klavier genau nach der Stimmgabel stimmen können.

Beide Schwierigkeiten können beseitigt werden, wenn wir das Klavier als Stimmgabel benutzen können und diesen Übergang von der Stimmgabel zum Klavier durchführen, indem wir einige hohe harmonische Obertöne des Klaviers benutzen.
Finden Sie, um diesen Übergang zu erreichen, eine Note innerhalb der gedämpften Noten, die laute Schwebungen mit der Gabel erzeugt.
Wenn Sie keine Note finden können, benutzen Sie die Note einen Halbton höher oder tiefer; benutzen Sie z.B. für eine Stimmgabel A das Ab oder A\# auf dem Klavier.
Wenn diese Schwebungsfrequenzen etwas zu hoch sind, versuchen Sie die gleichen Noten eine Oktave tiefer.
Stimmen Sie nun das A auf dem Klavier so, daß es Schwebungen der gleichen Frequenz mit diesen Bezugsnoten erzeugt (Ab, A\#, oder jede andere Note, die Sie gewählt haben).
Die beste Möglichkeit, die Stimmgabel zu hören, ist, sie \hyperref[c2_3_gabel]{wie oben beschrieben}\index{wie oben beschrieben} gegen den Tragus zu drücken oder sie auf irgendeine große, harte, flache Oberfläche zu drücken.
 

\label{c2_6c}
\subsection{Kirnberger II}
\label{c2_6_kirn}

\begin{itemize} 
 \item Dämpfen Sie alle Nebensaiten von F3 bis F4.
 \item Stimmen Sie C4 (das mittlere C) nach der Gabel.
 \item Benutzen Sie dieses C4, um G3 (Quarte), E4 (Terz), F3 (Quinte), und F4 (Quarte) zu stimmen.
 \item Benutzen Sie G3, um D4 (Quinte) und H3 (Terz) zu stimmen.
 \item Benutzen Sie H3, um F\#3 (Quarte) zu stimmen.
 \item Benutzen Sie F\#3, um Db4 (Quinte) zu stimmen.
 \item Benutzen Sie F3, um B3 (Quarte) zu stimmen.
 \item Benutzen Sie B3, um Eb4 (Quarte) zu stimmen.
 \item Benutzen Sie Eb4, um Ab3 (Quinte) zu stimmen.
 \item Alle Stimmungen bis hierhin sind \textit{rein}.<br>
Stimmen Sie nun A3 so, daß die Schwebungsfrequenzen von F3-A3 und A3-D4 die gleichen sind.
\end{itemize}

Sie sind fertig mit dem Einstellen der Bezugsnoten!


\label{c2_6_kirn2}

Stimmen Sie nun in \textit{reinen} Oktaven aufwärts, bis zu den höchsten Noten.
Stimmen Sie dann abwärts, bis zu den tiefsten Noten.
Beginnen Sie dabei mit der Bezugsoktave als Referenz.
Stimmen Sie bei all diesen Stimmungen nur eine neue Oktavsaite, während Sie die anderen dämpfen.
Stimmen Sie dann die eine bzw. zwei verbleibenden Saiten mit der neu gestimmten Saite \hyperref[c2_5_unis]{unisono}\index{unisono}.

Das ist ein Moment, in dem Sie die Regel \enquote{Stimmen Sie eine Saite nach einer anderen.} brechen sollten.
Wenn z.B. Ihre Referenznote eine (gestimmte) 3-saitige Note ist, benutzen Sie sie wie sie ist, ohne eine Saite davon zu dämpfen.
Das dient als ein Test der Qualität Ihres Stimmens.
Wenn es Ihnen schwer fällt, sie zu benutzen, um eine neue einzelne Saite zu stimmen, dann war u.U. Ihr Unisono-Stimmen der Referenznote nicht genügend genau, und Sie sollten zurückgehen und sie bereinigen.
Wenn Sie auch nach erheblicher Mühe nicht 3 gegen 1 stimmen können, haben Sie selbstverständlich keine andere Chance, als zwei der drei Saiten zu dämpfen, damit Sie vorwärtskommen.
Sie gefährden jedoch die Qualität des Stimmens.
Wenn alle Noten im Diskant und Baß fertig sind, dann sind die einzigen ungestimmten Noten jene, die Sie für das Einstellen der Bezugsnoten gedämpft haben.
Stimmen Sie diese -- mit der tiefsten Note beginnend -- unisono mit ihrer mittleren Saite, indem Sie vom Filz jeweils eine Schleife wegziehen.
 

\label{c2_6d}
\label{c2_6_et}

\subsection{Gleichschwebende Temperatur \hyperref[et]{(gleichstufige Temperatur, gleichmäßige
 Temperatur)}\index{(gleichstufige Temperatur, gleichmäßige
 Temperatur)}}

Ich zeige hier das leichteste, angenäherte Verfahren für die gleichschwebende Temperatur.
Genauere Algorithmen kann man in der Literatur finden (Reblitz, Jorgensen).
Kein professioneller Stimmer, der etwas auf sich hält, würde dieses Schema benutzen; wenn man jedoch gut darin wird, kann man eine annehmbare gleichschwebende Temperatur erzeugen.
Bei einem Anfänger werden die vollständigeren und präziseren Schemata nicht notwendigerweise zu besseren Ergebnissen führen.
Mit den komplexeren Methoden kann ein Anfänger schnell durcheinander kommen, ohne eine Vorstellung davon zu haben, was er falsch gemacht hat.
Mit der hier gezeigten Methode kann man schnell die Fähigkeit entwickeln, herauszufinden, was man falsch gemacht hat:

\begin{itemize} 
 \item Dämpfen Sie die Nebensaiten von G3 bis C\#5.
 \item Stimmen Sie A4 nach der A440 Gabel.
 \item Stimmen Sie A3 nach A4.
 \item Stimmen Sie dann mit verkürzten Quinten von A3 aus aufwärts, bis Sie nicht mehr weiter aufwärts gehen können, ohne den gedämpften Bereich zu verlassen, dann eine Oktave tiefer, und wiederholen Sie dieses \enquote{aufwärts in Quinten und eine Oktave abwärts}-Verfahren bis Sie zu A4 kommen.
Sie beginnen z.B. mit einer verkürzten A3-E4, dann einer verkürzten E4-H4.
Die nächste Quinte würde Sie über die höchste gedämpfte Note, C\#5, hinausführen; stimmen Sie deshalb eine Oktave abwärts, H4-H3.
\end{itemize}

Alle Oktaven sind selbstverständlich \textit{rein}.
Die verkürzten Quinten sollten am unteren Ende des gedämpften Bereichs mit etwas weniger als 1 Hz schweben und ungefähr 1,5 Hz am oberen Ende.
Die Schwebungsfrequenzen der Quinten zwischen dieser oberen und unteren Grenze sollten langsam mit zunehmender Tonhöhe steigen.

Wenn Sie in Quinten aufwärts gehen, stimmen Sie von \textit{rein} zu \textit{tiefer}, um eine verkürzte Quinte zu erzeugen.
Deshalb können Sie mit \textit{rein} beginnen und \textit{tiefer} stimmen, um gleichzeitig die Schwebungsfrequenz auf den gewünschten Wert zu steigern und \hyperref[c2_5_wirb]{den Wirbel richtig einzustellen}\index{den Wirbel richtig einzustellen}.
Wenn Sie alles perfekt getan haben, sollte das letzte D4-A4 ohne neu zu stimmen eine verkürzte Quinte mit einer Schwebungsfrequenz von 1 Hz sein.
Dann sind Sie fertig.
Sie haben gerade einen \enquote{Quintenzirkel} beendet.
Das Wunder des Quintenzirkels ist, daß er jede Note einmal stimmt, ohne irgendeine in der A3-A4-Oktave zu überspringen!

Wenn die abschließende D4-A4 nicht richtig ist, haben Sie irgendwo einen Fehler begangen.
Kehren Sie in diesem Fall die Prozedur um; beginnen Sie bei A4, gehen Sie in verkürzten Quinten abwärts und in Oktaven aufwärts, bis Sie A3 erreichen, wobei die abschließende A3-E4 eine verkürzte Quinte mit einer Schwebungsfrequenz von etwas weniger als 1 Hz sein sollte.
Um in Quinten abwärts zu gehen, erzeugen Sie eine verkürzte Quinte, indem Sie von \textit{rein} nach \textit{höher} stimmen.
Dieser Schritt des Stimmens wird jedoch den Wirbel nicht einstellen.
Um den Wirbel korrekt einzustellen, müssen Sie deshalb zunächst \textit{zu hoch} stimmen und dann die Schwebungsfrequenz auf den gewünschten Wert vermindern.
Deshalb ist in Quinten abwärts zu gehen eine schwierigere Operation als in Quinten aufwärts zu gehen.

Eine alternative Methode ist, mit A anzufangen, mit Quinten aufwärts bis zum C zu stimmen und dieses C mit einer Stimmgabel zu prüfen.
Wenn Ihr C \textit{zu hoch} ist, waren Ihre Quinten nicht ausreichend verkürzt und umgekehrt.
Eine weitere Variation ist, in Quinten von A3 aus etwas mehr als die Hälfte der Strecke aufwärts zu stimmen und dann von A4 bis zur letzten Note, die Sie beim Aufwärtsgehen gestimmt haben, abwärts zu stimmen.

Wenn die Bezugsnoten eingestellt sind, fahren Sie wie oben im Abschnitt über \hyperref[c2_6_kirn2]{Kirnberger II}\index{Kirnberger II} beschrieben fort.



