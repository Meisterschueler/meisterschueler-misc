% File: c1iii7i

\subsubsection{Die Geschwindigkeit steigern}
\label{c1iii7i}

Beim Klavierspielen dreht sich alles um eine ausgezeichnete Fingerkontrolle.
Wenn wir die Geschwindigkeit steigern, wird eine solche Kontrolle zunehmend schwierig, da die menschlichen Hände ursprünglich nicht für solche Geschwindigkeiten gemacht sind.
Die Hände sind jedoch komplex und anpassungsfähig, und wir wissen aus der Geschichte, daß solch schnelles Spielen möglich ist.
Deshalb werden wir -- so wie im Rest dieses Buchs -- versuchen, die richtigen, oder besten, Methoden für das Erreichen  unseres Geschwindigkeitsziels zu finden.


\paragraph{Schneller Anschlag, Entspannung}
\label{c1iii7iAnschlag}

\textbf{Es scheint offensichtlich, daß eine schnelle Anschlagsbewegung der Schlüssel zum schnellen Spielen ist, obwohl sie nicht immer gelehrt wird.
Der wichtigste Punkt für die Geschwindigkeit ist die Fingerbewegung im Knöchelgelenk.}
Jeder Finger besteht aus drei Knochen.
Das Knöchelgelenk ist das Gelenk zwischen Finger und Handfläche.
Beim Daumen ist das Knöchelgelenk sehr nah am Handgelenk.
Denken Sie sich beim schnellen Spielen jeden Finger als eine Einheit, und bewegen Sie ihn einfach am Knöchelgelenk.
Diese Bewegung hat zahllose Vorteile.
Sie benutzt für das Bewegen der Finger nur die eine Muskelgruppe, die auch die schnellste ist.
Den Finger am Knöchel zu bewegen, ist für den Daumen besonders wichtig.
Sie können nichts schnell spielen, wenn der Daumen nicht mit den anderen Fingern Schritt halten kann.
Andere Muskeln zum Beugen der Finger einzubeziehen würde die Bewegung stark verkomplizieren, was zu Verzögerungen der Nervenimpulse auf dem Weg vom Gehirn führen würde.
Das ist die Erklärung, warum der Daumenuntersatz beim schnellen Spielen nicht funktioniert -- beim Daumenuntersatz muß man die beiden anderen Gelenke des Daumens beugen, was eine langsamere Bewegung ist.
Das erklärt auch, warum \hyperref[c1iii4b]{flache Fingerhaltungen} schneller sind als die gebogenen.
Konzentrieren Sie sich deshalb, wenn Sie schnell spielen, nicht auf die Fingerspitzen, sondern benutzen Sie das Gefühl, daß die Finger sich an den Knöcheln bewegen.
Die Bewegung an den Knöcheln fördert auch sehr die \hyperref[c1ii14]{Entspannung}.
Wie im folgenden besprochen, müssen Sie beim schnellen Spielen auch das schnelle Entspannen üben.

\textbf{Wir müssen nun jede der drei Komponenten des \hyperref[c1iii1a1]{Basisanschlags} (siehe 1a) beschleunigen.}
Beim schnellen Spielen muß der Abschlag so schnell wie möglich sein, aber in dem Sinn kontrolliert, daß die Noten gleichmäßig sind und die gewünschte Lautstärke haben.
Das Halten ist wichtig, weil Sie während des Haltens sofort entspannen müssen, den Finger aber nicht anheben dürfen, so daß der Fänger nicht vorzeitig gelöst wird.
Dann müssen Sie das Anheben genau zum richtigen Zeitpunkt beginnen; dieses Anheben muß genauso beschleunigt sein.
Oben in \hyperref[c1iii7a]{Abschnitt 7a} haben wir gesehen, daß alle Muskelbündel aus langsamen und schnellen Muskeln bestehen; deshalb müssen wir, wenn wir für die Geschwindigkeit üben, schnelle Muskeln und schnelle Nervenreaktionen entwickeln und die Zahl der langsamen Muskeln reduzieren.
Das bedeutet, daß einfach mit ganzer Kraft stundenlang zu üben kontraproduktiv sein wird.
Schneller zu spielen funktioniert auch nicht, weil es nur erschwert, eine dieser Anschlagkomponenten zu üben, und man am Ende hauptsächlich die falschen Bewegungen übt.
Es bedeutet auch, daß es einige Zeit dauert, die Geschwindigkeit zu entwickeln, weil dazu körperliche Veränderungen im Gehirn, den Muskeln und den Nerven notwendig sind.
Dann muß man alle für die Geschwindigkeit notwendigen Bewegungen lernen.
So möchten Sie sich z.B. nicht angewöhnen, sich in das Klavier zu lehnen, um die Tasten während des Haltens unten zu halten, da keine langsamen Muskeln wachsen sollen -- Sie müssen den Fingerdruck sorgfältig kontrollieren, wenn Sie \enquote{für einen guten Klang tief in das Klavier spielen}.
Für die Geschwindigkeit müssen wir jede Komponente des Anschlags separat üben, und wenn sie alle richtig beschleunigt werden, kann man sie zusammenfügen.
Das bedeutet, jede Note langsam zu üben aber jede Komponente schnell auszuführen.
Wenn Sie viele Noten schnell spielen, werden Sie es \textit{niemals} richtig hinbekommen.

Die einfachste Möglichkeit, den schnellen Anschlag zu üben, ist, die fünf Noten von C bis G in Folge zu spielen und dabei jede Komponente des Anschlags sorgfältig zu üben.
Üben Sie die Abwärtsbewegung so schnell Sie können, bewahren Sie aber die Fähigkeit, die Lautstärke zu kontrollieren, den Druck für die Haltekomponente ständig aufrechtzuerhalten und sofort zu entspannen.
Das ist dem Basisanschlag ähnlich, außer daß nun alles beschleunigt sein muß.
Üben Sie während des Übergangs zum Halten das sofortige Entspannen, während Sie genügend Druck aufrechterhalten, um den Fänger in seiner Position zu halten.
Heben Sie für die Aufwärtskomponente den Finger danach schnell und gleichzeitig mit dem Ausführen des Abschlags durch den nächsten Finger.
Alle nicht spielenden Finger sollten nur die Tastenoberflächen berühren und nicht hoch über den Tasten schweben.
Es ist vermutlich einfacher, die Noten erst paarweise zu üben: 121212..., dann 232323... usw.
Spielen Sie zunächst nur eine oder zwei Noten je Sekunde, und werden Sie schrittweise schneller, wenn Sie besser werden.
Übertreiben Sie die Aufwärtsbewegung, da die Streckmuskeln bei den meisten zu schwach sind und zusätzliches Training benötigen.
Beziehen Sie den ganzen Körper mit ein, während Sie entspannt bleiben; das Gefühl ist, daß jede Note ihren Ursprung unten im Bauch hat.
Für diese Übungen sind schnelle Auf- und Abwärtsbewegungen das Ziel, nicht wie schnell Sie aufeinanderfolgende Noten spielen können.

Schnelles Spielen wird nicht durch das Lernen von nur einer Fertigkeit erreicht; es ist eine Kombination vieler Fertigkeiten, und das ist ein weiterer Grund, warum es lange dauert, es zu lernen.
Geschwindigkeit ist wie eine Kette, und die Maximalgeschwindigkeit wird durch das schwächste Glied der Kette begrenzt; deshalb müssen Sie die schnellen Muskeln wachsen lassen, indem Sie die schnellen Komponenten des Anschlags üben, so daß die Geschwindigkeit der Finger nicht zum begrenzenden Faktor wird.
Wenn die Geschwindigkeit steigt, wird es offensichtlich, daß man den Basisanschlag ändern muß, damit man schneller als mit einer bestimmten Geschwindigkeit spielen kann.
Die erste Änderung ist, das Halten wegzulassen, da es nur Zeit verschwendet.
Wir haben jedoch eine wichtige Lektion gelernt, die wir nicht vergessen dürfen -- die Entspannung.
Zwischen dem Abschlag und dem Anheben muß ein Moment der Entspannung sein.
Mit anderen Worten: Man möchte nicht in irgend eine der unerwünschten Situationen geraten, die Streß erzeugen.
Einige Schüler werden z.B. diese Bewegung \enquote{vereinfachen}, indem sie alle Streckmuskeln anspannen (alle Finger anheben) und schnell spielen, indem einfach die Beugemuskeln stärker angespannt werden als die Streckmuskeln.
Das baut eindeutig Streß auf und erzeugt eine Geschwindigkeitsbarriere, weil die eigenen Muskeln gegeneinander arbeiten.
Die Lektion, die wir beim Basisanschlag gelernt haben, daß beim Abschlag nur der Beugemuskel und beim Anheben nur der Streckmuskel aktiviert wird, ist für die Geschwindigkeit und die Entspannung entscheidend.


\paragraph{Andere Geschwindigkeitsmethoden}
\label{c1iii7iAndere}

Fügen Sie nun alle anderen Bewegungen hinzu, die zur Geschwindigkeit führen.
Wir befassen uns hier mit \textit{allgemeinen} Geschwindigkeitstricks; es gibt noch mehr \textit{spezielle} Tricks für praktisch jede schwierige, schnelle Passage.
Deshalb sind Übungen wie \hyperref[c1iii7h]{Hanon} so schädlich -- sie halten Sie vom Lernen dieser speziellen Tricks ab, indem sie Sie zu dem falschen Glauben verleiten, daß Hanon alle diese allgemeinen und speziellen Probleme lösen wird.
Ein Beispiel für einen speziellen Geschwindigkeitstrick ist der ungewöhnliche Fingersatz der RH ab Takt 20 des dritten Satzes von Beethovens Appassionata (eigentlich gibt es mehrere mögliche Fingersätze.
Es folgen ein paar allgemeine Tricks, die auf viele Arten von Fällen anwendbar sind.

Die \hyperref[c1ii11]{parallelen Sets} lehren Ihnen, alle Finger gleichzeitig zu bewegen, so daß aufeinander folgende Noten viel schneller als mit der Geschwindigkeit jedes einzelnen Fingers gespielt werden können.
Aber ohne zunächst einen soliden Basisanschlag aufzubauen, können die parallelen Sets zu zahlreichen schlechten Angewohnheiten und zu unordentlichem Spielen führen.
Die parallelen Sets alleine trainieren Sie nicht automatisch darauf, den Finger zur richtigen Zeit zu heben, um die Dauer der Noten exakt zu kontrollieren.
Und sie lehren Ihnen nicht notwendigerweise schnelle Anschläge, weil schnelle Anschläge am besten geübt werden, indem man den Basisanschlag langsam übt.
Parallele Sets müssen auch einem anderen Zweck dienen: dem Gehirn die Vorstellung von der Geschwindigkeit zu vermitteln.
Bis man wirklich schnell spielen kann, hat das Gehirn keine Vorstellung davon, was es bedeutet, körperlich schnell zu spielen -- man ist also in einem Dilemma: Man kann nicht schnell spielen, aber man muß schnell spielen, um das Gehirn so zu trainieren, daß man schnell spielen kann.
An dieser Stelle kommen die phasengekoppelten parallelen Sets ins Spiel.
Bei den phasengekoppelten parallelen Sets ordnet man einfach die Finger so an, daß einer etwas höher als der andere ist, so daß wenn man die ganze Hand senkt, die einzelnen Finger automatisch nacheinander spielen.
Wenn man die vertikalen Abstände der Finger sehr gering hält, kann man die Noten so schnell spielen, wie man möchte, fast unendlich schnell.
Diese schnellen parallelen Sets sind das, was Sie benötigen, um dem Gehirn die Vorstellung von Geschwindigkeit zu lehren.
Aber die Bewegung der phasengekoppelten parallelen Sets verstößt völlig gegen die Regeln der Bewegungen des Basisanschlags.
Deshalb ist die Idee der parallelen Sets, von zwei extremen Positionen ausgehend zu beginnen -- dem Basisanschlag (korrekte Bewegung aber langsam) und den phasengekoppelten parallelen Sets (schnell aber die falschen Bewegungen) -- und so zu üben, daß man ungefähr die Mitte trifft, bei der man sowohl die Geschwindigkeit als auch die Genauigkeit hat sowie die Kontrolle, die Unabhängigkeit der Finger, \hyperref[ruhig]{ruhige Hände} und die Entspannung.
Deshalb haben Sie, wenn Sie sehr schnelle parallele Sets nur phasengekoppelt spielen können, nur einen kleinen Teil dessen gelernt, was Sie zur Vollendung benötigen.

Die \hyperref[c1iii4b]{flachen Fingerhaltungen} können schneller als die gebogenen sein, weil sie die Krümmungslähmung vermeiden, und die Fingerspitzen ausgestreckter Finger können sich schneller bewegen als die Spitzen gekrümmter Finger.
Durch das Entspannen der letzten beiden Glieder an der Fingerspitze vereinfachen Sie auch die Bewegung, so daß Sie die schnellen Muskeln in einer geringeren Zahl von Muskelsträngen entwickeln müssen.
Gewöhnen Sie sich an, die flachen und gebogenen Fingerhaltungen fingerweise zu kombinieren (jeder einzelne Finger kann je nach Bedarf gerade oder gebogen sein), weil Ihnen das einen schnelleren Zugriff auf die Tasten geben kann.
Die allgemeine Regel ist: Benutzen Sie die flachen Fingerhaltungen, wenn Sie nur schwarze oder nur weiße Tasten spielen; wenn beide Farben benötigt werden, benutzen Sie die flachen Haltungen für die schwarzen und die gebogenen für die weißen Tasten.
Diejenigen mit langen Fingern müssen eventuell den Daumen und den kleinen Finger flach halten und die Finger 2 bis 4 gebogen.

Geschwindigkeit ist, nach der Musikalität, die am schwersten zu erwerbende Fertigkeit.
Es ist ein weit verbreitetes intuitives Mißverständnis, daß man das schnelle Spielen üben muß, um sich die Geschwindigkeit anzueignen.
Erfahrene Lehrer wissen um die Zwecklosigkeit eines solchen vereinfachten Ansatzes und haben versucht, Methoden für das Aneignen der Geschwindigkeit zu entwickeln.
Ein verbreiteter Ansatz ist, den Schülern vom schnellen Spielen abzuraten -- dieser Ansatz wird zumindest alle Arten potentieller, irreversibler Probleme verhindern: psychologische, körperliche, musikalische, technische usw., geht aber das Geschwindigkeitsproblem nicht direkt an und kann den Lernprozeß unnötig verlangsamen.

Die falsche Vorstellung, daß man \enquote{Klaviermuskeln} aufbauen muß, um schnell zu spielen, hat bei vielen dazu geführt, daß sie lauter spielen als es notwendig wäre.
Geschwindigkeit bedeutet Fertigkeit, nicht Kraft.
Deshalb muß man die Lautstärke von der Technik trennen; erwerben Sie zum Spielen lauter Passagen erst die Technik, und fügen Sie dann die Lautstärke hinzu.
Schwierige Passagen führen beim Üben häufig zu Streß und Ermüdung.
Leise zu spielen reduziert beides und beschleunigt dadurch den Technikerwerb.
Wenn man mehrere Jahre Klavier spielt, wird man jedes Jahr stärker, und es kann sein, daß man schließlich entsprechend lauter spielt, ohne es zu merken.
Das lautere Spielen erschwert auch das musikalische Spielen.
Unter Pianisten herrscht Einigkeit darüber, daß es schwierig ist, gleichzeitig leise und schnell zu spielen -- und der Grund ist einfach: Man benötigt mehr Technik dazu.

Ein \hyperref[c1iii1a]{guter Klang} wird erzeugt, indem man \enquote{tief in das Klavier hineingeht}.
Man muß aber auch entspannen.
Es ist nicht notwendig, dauernd nach unten zu drücken.
Dieser konstante Abwärtsdruck verschwendet nicht nur Energie (was ermüdet), sondern verhindert auch, daß die Finger sich schnell bewegen.
Es besteht oft die Neigung, sich in das Klavier hineinzulehnen, um \enquote{tief zu spielen}, und nachdem man das mehrere Jahre getan hat, kann es sein, daß man schließlich ohne es zu merken mit einer enormen Kraft nach unten drückt (siehe \hyperref[testimonials06]{Punkt 6 der Leserkommentare}).
Sogar Schüler der Armgewichtsmethode, die den korrekten Armdruck lehrt, haben manchmal am Ende einen ungeeigneten Abwärtsdruck.
Deshalb prüfen Armgewichtslehrer immer den Abwärtsdruck, indem sie die Entspannung des Arms kontrollieren.
Trotz des Begriffs \enquote{Armgewicht} ist das Gewicht des Arms in der Regel nicht die richtige abwärts gerichtete Kraft.
Die Armgewichtsmethode verlangt, daß man genügend entspannt ist, so daß man das Gewicht des Arms spüren kann.
Der optimale Abwärtsdruck hängt von mehreren Faktoren ab (Geschwindigkeit, Lautstärke, staccato-legato usw.).
Eine Möglichkeit, den richtigen Abwärtsdruck zu prüfen, ist, ihn zu verringern, bis man anfängt, Noten auszulassen.
Fügen Sie dann soviel Abwärtsdruck hinzu, daß Sie keine mehr auslassen -- das sollte der richtige Wert sein, bei dem Sie für die Technik und die Geschwindigkeit üben -- Sie werden vielleicht feststellen, daß Ihr ursprünglicher Druck zu hoch war.
Haben Sie den richtigen Druck gefunden, sollten Sie schneller spielen können.
Die \hyperref[c1iii3]{Triller} und Verzierungen werden auch schneller und klarer.
Das Pianissimo wird sich auch verbessern.
Den Abwärtsdruck zu reduzieren bedeutet nicht, daß man wie staccato spielt oder die Finger über dem unteren Punkt des Anschlags \enquote{schweben}, ohne daß die Fänger greifen, da dieses gegen die Regeln des Basisanschlags verstößt.
Mit dem optimalen Abwärtsdruck wird der \enquote{schwache vierte Finger} -- wegen der Reduktion des Stresses und weil der vierte Finger nicht ständig mit den stärkeren Fingern konkurrieren muß -- weniger ein Problem sein.
Das richtige Anheben der Finger, das Staccato, Legato usw. werden alle beeinflußt, wenn man die Lautstärke und den Abwärtsdruck ändert.
Trennen Sie deshalb immer die Technik von der Lautstärke, und üben Sie leise aber bestimmt für die Technik.

\hyperref[c1iii1b]{Rhythmus} ist extrem wichtig.
Nicht nur der Rhythmus der Musik, wie sie von den Fingerspitzen gespielt wird, aber auch vom ganzen Körper, so daß sich nicht ein Teil gegen einen anderen bewegt.
Weitere Probleme sind unnötige Bewegungen und solche, die nicht zum Rhythmus passen.
Eine erforderliche Bewegung in einer Hand kann eine unbeabsichtigte Bewegung an einer anderen Stelle des Körpers verursachen.
\textbf{Viele dieser unerwünschten Bewegungen werden sichtbar, wenn man \hyperref[c1iii13]{sich selbst auf Video aufnimmt}.}
Selbstverständlich muß man zunächst einmal den richtigen Rhythmus im Kopf haben; lassen Sie sich den Rhythmus nicht vom Klavier diktieren, da der Rhythmus genauso ein Teil der Musik ist wie die Melodie; man muß ihn bewußt durch das \hyperref[c1ii12]{mentale Spielen} kontrollieren.
Der Rhythmus besteht nicht nur aus der zeitlichen Abfolge, sondern auch aus der Kontrolle des Klangs und der Lautstärke.

Balance ist ein weiterer wichtiger Faktor.
Nicht nur die Balance Ihres Körpers auf der Bank, sondern auch der Schwerpunkt jeder spielenden Hand und der gemeinsame Schwerpunkt beider Hände.
Achten Sie beim HS-Üben darauf, wo der Schwerpunkt der Hand liegt (von wo die Abwärtskraft ausgeht).
Versuchen Sie, diesen Punkt entlang einer Linie zu plazieren, die gerade durch den Arm verläuft.
Das ist nur wichtig, wenn man sehr schnell spielt, da während des langsamen Spielens alle Impulse vernachlässigbar sind und der Schwerpunkt genau in dem Finger liegt, der gerade den Abschlag ausführt, so daß man ihn nicht umherbewegen kann.
Wenn der Schwerpunkt nicht am richtigen Ort ist, muß man zum Ausgleichen zusätzliche Muskeln benutzen, was zu Streß und Ermüdung führt.

In Abhängigkeit von der Situation werden Sie die an anderer Stelle besprochenen Methoden benötigen, wie den \hyperref[c1iii5b]{Daumenübersatz}, die \hyperref[c1iii5wagen]{Wagenradbewegung}, \hyperref[c1iii4b]{mit flachen Fingern spielen}, \hyperref[c1iii8]{Konturieren} usw.
Um auf alle diese Faktoren zu achten, werden Sie häufig mit moderaten oder langsamen Geschwindigkeiten üben müssen.


\paragraph{Geschwindigkeit und Musik}
\label{c1iii7iMusik}

Ein Schlüssel zum Verständnis, wie man für die Geschwindigkeit üben muß, ist die Frage: \enquote{Warum ist die Geschwindigkeit ein ungeeignetes Kriterium, um den Erfolg zu messen?}
Die Antwort ist, daß Geschwindigkeit allein, ohne die richtige Technik, die Musik ruinieren wird.
Deshalb sollten wir die Musik als Kriterium für den Erwerb der Geschwindigkeit benutzen.
D.h. um die Geschwindigkeit zu erwerben, muß man musikalisch spielen.
Die Musikalität ist jedoch nur eine notwendige Bedingung; sie ist keine hinreichende Bedingung.
Musikalisch zu spielen garantiert nicht automatisch die Geschwindigkeit.
Aber zumindest sind wir halbwegs am Ziel!
Wir wissen nun, daß wir schnell spielen können, aber nur bis zu Geschwindigkeiten, bei denen wir die Musikalität aufrechterhalten können.
Eine Lösung ist, nur Kompositionen zu spielen, die so leicht sind, daß man sie musikalisch spielen kann.
Deshalb ist es so wichtig, daß Sie Ihre fertigen Stücke spielen -- üben Sie nicht immer nur schwieriges Material und ignorieren Ihre fertigen Stücke.
Die Lösung, nur leichte Stücke zu spielen, ist nicht durchführbar, weil die besten Schüler Stücke spielen möchten, die sie herausfordern, und sie bereit sind, dafür zu arbeiten. 
Viel wichtiger ist vielleicht, daß herausfordernde Stücke Ihnen dabei helfen können, schneller Fortschritte zu machen.
In diesem Fall müssen Sie mehrfach wiederholen: Lernen Sie das Stück zunächst mit langsamerer Geschwindigkeit, so daß Sie es noch musikalisch spielen können; Benutzen Sie dann parallele Sets usw., um schnellere Geschwindigkeiten zu ermöglichen (hauptsächlich HS) und dann das musikalische Spielen mit diesen höheren Geschwindigkeiten zu üben.
Wiederholen Sie dann die gesamte Prozedur, d.h. üben Sie mit verschiedenen Geschwindigkeiten.
Außerdem müssen Sie wissen, wie man den größten Nutzen aus der \hyperref[c1ii15]{automatischen Verbesserung nach dem Üben (PPI)} zieht.

Musikalität kann man nicht in ein paar kurzen Sätzen definieren.
Das ist angesichts der Tatsache, daß auch der Begriff der Musik nicht einfach zu definieren ist, nicht verwunderlich.
Viele Schüler sind darüber verzweifelt, daß sie nicht musikalisch sind.
Auf der anderen Seite verfügen wir alle in dem Sinn über genügend Musikalität, daß wir musikalische Qualität auf sehr hohen Stufen beurteilen können -- denken Sie an die häufigen Bemerkungen (auch von Nichtklavierspielern) über die Unzulänglichkeiten oder feinen Unterschiede von Klavierspielern oder berühmten Künstlern.
Wenn es aber darum geht, selbst Musik zu machen, wird es unerklärlicherweise etwas anderes.
Warum?
Die Antwort ist einfach: Es ist nicht so, daß wir nicht musikalisch wären; wir haben nur nicht die Fertigkeit erlernt, musikalisch zu spielen.
Das musikalische Spielen fällt nicht leicht, besonders wenn man keinen guten Lehrer hat, der einem erklären kann, was man falsch macht.
Eine der besten Methoden zur Entwicklung der Musikalität ist, das eigene Spielen \hyperref[c1iii13]{aufzunehmen oder zu filmen}, sich die Aufnahme kritisch anzuhören oder anzusehen und dabei jene hohe Stufe des musikalischen Urteilsvermögens zu benutzen, über die wir alle verfügen.
Das Aufnehmen sollte -- unabhängig vom Alter -- bereits im ersten Unterrichtsjahr beginnen.
Man muß sich auch professionelle Aufnahmen der Stücke anhören, die man spielt.
Anfänger werden Schwierigkeiten damit haben, Aufnahmen ihrer einfachen Übungsstücke zu finden; bitten Sie in diesem Fall den Lehrer, sie zu spielen, so daß Sie sie aufnehmen können.
Die meisten Klavierspieler hören sich eine hinreichende Menge Musik an, aber der entscheidende Punkt ist hier, daß Sie sich Aufführungen der Stücke, die Sie spielen, anhören müssen.
Der grundlegendste Teil der Musikalität ist Genauigkeit (Taktart usw.) und das Befolgen der Ausdrucksbezeichnungen in den Noten.
\textbf{Fehler beim Notenlesen, besonders beim \hyperref[c1iii1b]{Rhythmus}, können es unmöglich machen, ein Stück auf die endgültige Geschwindigkeit zu bringen.}

Die meisten nehmen einfach an, daß sie, wenn sie üben, auch üben \hyperref[c1iii14]{vorzuspielen}.
Für die meisten Menschen ist das absolut falsch.
Der mentale Zustand für das Üben und das Vorspielen sind üblicherweise zwei völlig voneinander verschiedene Zustände.
Beim Üben möchte der Geist die Technik erwerben und das Stück lernen.
Beim Vorspielen ist dessen einzige Aufgabe, Musik zu erzeugen.
Für einige ist es unmöglich, sich während des Übens in den Zustand des Vorspielens zu versetzen, weil das Gehirn weiß, daß kein Publikum anwesend ist.
Deshalb ist die Videoaufzeichnung oder das Aufnehmen so wichtig; außerdem kann man seine eigenen Stärken und Schwächen sehen und hören.
Machen Sie nicht nur Aufnahmen zu Übungszwecken, sondern auch zur dauerhaften Archivierung des Erreichten -- ein Album all Ihrer fertigen Stücke.
Eine gute Möglichkeit dafür ist das Veröffentlichen im Internet.\footnote{Auf deutschen Servern Copyright usw. beachten!} 
Wenn die Aufnahmen nicht für das dauerhafte Archivieren gedacht sind, dann werden sie einfach zu einer weiteren Übungsaufnahme, die sich nicht sehr von den routinemäßigen Übungsaufnahmen unterscheidet.
Alle guten Klavierlehrer veranstalten Konzerte ihrer Schüler; diese Konzerte lehren ihnen den Zustand des Vorspielens.
Sie werden überrascht sein, wie schnell Sie Fortschritte machen, wenn Sie eine Aufnahme bis zu einem bestimmten Termin fertig haben oder auf einem Konzert vorspielen müssen.
Die meisten schreiben diesen Fortschritt dem Druck zu, das Lernen eines Stücks bis zu einem bestimmten Datum abzuschließen, was nur teilweise stimmt.
Eine große Komponente des Fortschritts kann der Psychologie des musikalischen Übens zugeschrieben werden.
Das beweist, daß wir alle wissen, was \enquote{musikalisch spielen} bedeutet.
Aber es mangelt uns an der mentalen Disziplin, musikalisch zu üben.

Es gibt zwei entgegengesetzte Arten, sich der Musikalität zu nähern.
Eine ist die \enquote{künstlerische} Herangehensweise, bei der im Geist ein musikalischer Ausdruck erzeugt wird und die Hände \enquote{einfach alles ausführen}, um den gewünschten Effekt zu erzielen.
Leider können das die meisten Menschen mit normaler Intelligenz nicht -- es erfordert wirkliches \enquote{Talent}.
Die andere Art ist der analytische Ansatz, bei dem die Person jede einzelne zur Erzeugung des endgültigen Effekts notwendige anatomische Bewegung lernt.
Leider haben wir keine vollständige Liste all dieser notwendigen Bewegungen.
Wir befinden uns alle irgendwo zwischen diesen beiden gegensätzlichen Arten.
Mit anderen Worten: Erfolgreiche Pianisten beider Extreme werden letzten Endes im Grunde dasselbe tun, so daß es keinen \enquote{korrekten} oder bevorzugten Ansatz gibt -- jeder kann von beiden profitieren.

Die Schlußfolgerung ist, daß man die Geschwindigkeit nicht erwerben kann, indem man die Finger zwingt, schneller zu spielen als sie es gemäß ihrer technischen Stufe können, da man die Entspannung verliert, schlechte Angewohnheiten entwickelt und Geschwindigkeitsbarrieren aufbaut.
Der Basisanschlag muß auch bei hohen Geschwindigkeiten aufrechterhalten werden.
Die beste Möglichkeit, innerhalb Ihrer technischen Grenzen zu bleiben, ist, musikalisch zu spielen.
Sie können kurzzeitig parallele Sets, \hyperref[c1iii2]{Zirkulieren} usw. benutzen, um die Geschwindigkeit unter einer geringeren Beachtung der Musikalität schnell zu steigern, aber Sie sollten das die Ausnahme sein lassen, nicht die Regel.
Wenn Sie es notwendig finden, längere Zeit zu zirkulieren, sollte das musikalisch geübt werden.
Das ist ein weiterer Grund, warum das HS-Üben so effektiv ist; man kann HS mit höheren Geschwindigkeiten musikalisch spielen als HT.
Als nächstes müssen Sie alle analytischen Methoden für das Steigern der Geschwindigkeit verinnerlichen, wie die Entspannung, die verschiedenen Hand- und Fingerhaltungen, den \hyperref[c1iii5b]{Daumenübersatz}, den korrekten Abwärtsdruck usw.
Letzten Endes wird das Üben der Geschwindigkeit um der Geschwindigkeit willen kontraproduktiv; wenn Sie Klavier spielen, müssen Sie Musik machen.
Das befreit Sie von dem Geschwindigkeitsdämon und führt Sie in die sagenhafte Welt des wundervollen Klavierklangs.



