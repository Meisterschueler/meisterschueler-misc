% File: c1iv4

\subsection{Mozarts Formel, Beethoven und Gruppentheorie}
\label{c1iv4}

Es gibt eine enge, wenn nicht sogar unabdingbare, Beziehung zwischen Mathematik und Musik.
Zumindest sind ihnen eine große Zahl der fundamentalsten Eigenschaften gemeinsam, angefangen mit der Tatsache, daß die chromatische Tonleiter eine einfache logarithmische Gleichung ist \hyperref[c2_2]{(s. Kapitel Zwei, Abschnitt 2)}, und daß die Grundintervalle Verhältnisse der kleinsten ganzen Zahlen sind.
Bis jetzt interessieren sich wenige Musiker um der Mathematik willen für die Mathematik.
Praktisch jeder ist jedoch neugierig und hat sich von Zeit zu Zeit gefragt, ob die Mathematik irgendwie in das Erzeugen von Musik verwickelt ist.
Gibt es ein tiefes, zugrunde liegendes Prinzip, dem sowohl die Mathematik als auch die Musik unterworfen ist?
Außerdem gibt es die feststehende Tatsache, daß jedesmal, wenn wir die Mathematik erfolgreich auf ein Gebiet angewandt haben, wir auf diesem Gebiet mit gewaltigen Schritten vorangekommen sind.
Eine Möglichkeit, diese Beziehung zu untersuchen, ist, die Arbeit der größten Komponisten von einem mathematischen Standpunkt aus zu studieren.

Die folgenden Analysen beinhalten keine Eingaben seitens der Musiktheorie.
Als ich das erste Mal von Mozarts Formel hörte, war ich sehr begeistert, weil ich dachte, daß es ein wenig Licht in die Musiktheorie und die Musik an sich bringen könnte.
Sie mögen zunächst enttäuscht sein, so wie ich es war, wenn Sie herausfinden, daß Mozarts Formel sich als streng strukturell erweist.
Strukturelle Analysen haben bisher noch nicht viel Information darüber erbracht, wie man berühmte Melodien hervorbringt; aber dann macht das die Musiktheorie genauso wenig.
Die heutige Musiktheorie hilft nur dabei, \enquote{korrekte} Musik zu komponieren oder sie weiter auszuführen, wenn man erst einmal eine musikalische Idee bekommen hat.
Musiktheorie ist eine Klassifikation von Notenfamilien und ihren Arrangements in bestimmte Muster.
Wir können noch nicht die Möglichkeit ausschließen, daß Musik letzten Endes auf bestimmten nachweisbaren Arten von strukturellen Mustern basiert.
Ich habe zuerst von Mozarts Formel in einer Vorlesung eines Musikprofessors erfahren.
Ich habe die Quelle verloren - wenn jemand, der dieses Buch liest, eine Quelle kennt, lassen Sie es mich bitte wissen.

Es ist nun bekannt, daß Mozart praktisch seine ganze Musik, seit er sehr jung war, nach einer einzigen Formel komponiert hat, die seine Musik um einen Faktor von mehr als zehn erweiterte.
D.h., wann immer er sich eine neue Melodie ausdachte, die eine Minute dauerte, wußte er, daß seine endgültige Komposition mindestens 10 Minuten lang sein würde.
Manchmal war sie \textit{viel} länger.
Der erste Teil seiner Formel war, jedes Thema zu wiederholen.
Diese Themen waren im allgemeinen sehr kurz - nur 4 bis 10 Noten, viel kürzer als man annehmen würde, wenn man an ein musikalisches Thema denkt.
Diese Themen, die viel kürzer sind als die gesamte Melodie, verschwinden einfach in der Melodie, weil sie zu kurz sind, um erkannt zu werden.
Deshalb nehmen wir sie normalerweise nicht wahr, und das ist fast mit Sicherheit ein absichtliches Konstrukt des Komponisten.
Das Thema wird dann zwei- oder dreimal verändert und noch einmal wiederholt, um das zu erzeugen, was das Publikum als fortlaufende Melodie wahrnimmt.
Diese Änderungen bestehen aus der Anwendung verschiedener mathematischer und musikalischer Symmetrien wie Inversionen, Umkehrungen, harmonischen Veränderungen, geschickte Positionierung von Verzierungen usw.
Diese Wiederholungen werden zu einem Abschnitt zusammengestellt, und der ganze Abschnitt wird wiederholt.
Die erste Wiederholung trägt einen Faktor von zwei bei, die verschiedenen Veränderungen bringen einen weiteren Faktor von zwei bis sechs (oder mehr) und die Wiederholung des ganzen Abschnitts am Schluß bringt einen weiteren Faktor von zwei, was mindestens zu 2x2x2 = 8 führt.
Auf diese Art war er in der Lage, große Kompositionen mit einem Minimum an thematischem Material zu schreiben.
Zusätzlich folgten seine Änderungen des ursprünglichen Themas einer bestimmten Reihenfolge, so daß gewisse Stimmungen oder Färbungen der Musik in jeder Komposition in derselben Reihenfolge angeordnet waren.

Wegen dieser vorherbestimmten Struktur war er in der Lage, seine Kompositionen von überall in der Mitte beginnend oder Stimme für Stimme niederzuschreiben, weil er bereits im voraus wußte, wo jeder Teil hingehört.
Und er mußte nicht das ganze Stück niederschreiben, bevor nicht das letzte Teil des Puzzles an seinem Platz war.
Er konnte auch mehrere Stücke gleichzeitig komponieren, weil sie alle dieselbe Struktur hatten.
Diese Formel ließ ihn als ein größeres Genie erscheinen als er tatsächlich war.
Das führt uns natürlich zu der Frage, wieviel von seinem angeblichen \enquote{Genie} einfach eine Illusion war, die von solchen Manipulationen hervorgerufen wurde.
Das soll seine Genialität nicht in Frage stellen - die Musik beweist diese schließlich!
Viele der wundervollen Dinge, die diese Genies taten, waren jedoch das Ergebnis von relativ einfachen Mitteln, und wir können alle einen Vorteil daraus ziehen, indem wir die Details dieser Mittel herausfinden.
So vereinfacht Mozarts Formel zu kennen es z. B., seine Kompositionen zu zergliedern und auswendig zu lernen.
Der erste Schritt zu einem Verständnis seiner Formel ist die Fähigkeit, seine Wiederholungen zu analysieren.
Es sind keine einfachen Wiederholungen; Mozart benutzte sein Genie, um die Wiederholungen zu verändern und zu tarnen, so daß sie Musik erzeugten und, wichtiger noch, die Tatsache der Wiederholung nicht wahrgenommen wird.


\label{KV525}

Lassen Sie uns als Beispiel für die Wiederholungen die berühmte Melodie im Allegro seiner \enquote{Eine Kleine Nachtmusik}\footnote{KV525} untersuchen.
Das ist die Melodie, die am Anfang des Films \enquote{Amadeus} von Salieri gespielt und dem Pastor erkannt wurde.
Diese Melodie ist eine Wiederholung, die als eine Frage und eine Antwort gesetzt ist.
Die Frage ist die einer männlichen Stimme \hyperref[ueb-KV525]{\enquote{Oh, mein Schatz, kommst Du nachher zu mir?}} und die Antwort ist eine weibliche Stimme: \hyperref[ueb-KV525]{\enquote{Ja, oh ja, ich komm nachher zu Dir!}}
Die männliche Aussage wird mit nur zwei Noten erzeugt, die eine gebieterische Quarte bilden und dreimal wiederholt werden (sechs Noten), und die Frage wird erzeugt, indem am Ende drei ansteigende Noten hinzugefügt werden (das scheint für die meisten Sprachen universell zu sein - Fragen werden durch das Anheben der Stimme am Ende des Satzes gestellt).
Somit besteht der erste Teil aus 9 Noten.
Die Wiederholung ist eine Antwort in einer weiblichen Stimme, weil die Tonhöhe höher ist, und besteht wieder aus zwei Noten, dieses Mal in einer sanfteren kleinen Terz, die (Sie haben es geahnt!) dreimal wiederholt wird (sechs Noten).
Es ist eine Antwort, weil die letzten drei Noten sich abwärts schlängeln.
Wieder sind es insgesamt 9 Noten.
Die Effizienz, mit der er dieses Konstrukt erzeugt hat, ist erstaunlich.
Es ist sogar noch unglaublicher, wie er die Wiederholung tarnt, so daß man es nicht als Wiederholung ansieht, wenn man sich das Ganze anhört.
Praktisch seine ganze Musik kann auf diese Art analysiert werden, d.h. meistens als Wiederholungen.
Wenn Sie noch nicht überzeugt sind, nehmen Sie ein beliebiges seiner Stücke und analysieren es, und Sie werden dieses Muster finden.

Lassen Sie uns ein anderes Beispiel betrachten: die Sonate No. 16 in A, K 300 (KV 331, mit dem Rondo \enquote{Alla Turca} am Ende).
Die Grundeinheit des Anfangsthemas ist eine Viertelnote, gefolgt von einer Achtelnote.
Die erste Einführung dieser Einheit wird durch das Hinzufügen der Sechzehntelnote getarnt, auf die die Grundeinheit folgt.
Auf diese Art wird die Einheit im ersten Takt zweimal wiederholt.
Er übersetzt danach (in der Tonhöhe) die ganze verdoppelte Einheit des ersten Takts und wiederholt sie im zweiten Takt.
Der dritte Takt besteht nur aus einer zweimaligen Wiederholung der Grundeinheit.
Im vierten Takt tarnt er wieder die erste Einheit mit Hilfe der Sechzehntelnoten.
Die Takte 1 bis 4 werden dann mit kleinen Änderungen in den Takten 5 bis 8 wiederholt.
Von einem strukturellen Standpunkt aus ist jeder der ersten 8 Takte dem Muster des ersten Takts nachgebildet.
Von einem melodischen Standpunkt aus erzeugen diese 8 Takte zwei lange Melodien mit ähnlichen Anfängen aber verschiedenen Endungen.
Da alle 8 Takte wiederholt werden, hat er im Grunde seine anfängliche Idee, die im ersten Takt enthalten ist, mit 16 multipliziert!
Wenn man in Begriffen der Grundeinheit denkt, hat er sie mit 32 multipliziert.
Aber dann fährt er damit fort, diese Grundeinheit zu nehmen und unglaubliche Variationen zu erzeugen, um die ganze Sonate zu schaffen, so daß der endgültige Multiplikationsfaktor sogar noch größer ist.
Das ist mit der Feststellung gemeint, daß er Wiederholungen von Wiederholungen benutzt.
Indem er die Wiederholungen der veränderten Einheiten aneinanderreiht, erzeugt er am Ende eine Melodie, die sich wie eine lange Melodie anhört, bis man sie in Ihre Komponenten zerlegt.

In der zweiten Hälfte dieser Exposition führt er neue Veränderungen der Grundeinheit ein.
In Takt 10 fügt er zunächst eine Verzierung mit melodischem Wert hinzu, um die Wiederholung zu tarnen und führt danach eine weitere Änderung ein, indem er die Grundeinheit als Triole spielt.
Nachdem die Triole eingeführt ist, wird sie zweimal in Takt 11 wiederholt.
Takt 12 ist Takt 4 ähnlich; er ist eine Wiederholung der Grundeinheit aber auf eine solche Weise strukturiert, daß er als Verbindung zwischen den in Zusammenhang dazu stehenden vorangegangenen und nachfolgenden drei Takten fungiert.
Somit sind die Takte 9 bis 16 den Takten 1 bis 8 ähnlich, es steht aber eine andere musikalische Idee dahinter.
Die letzten zwei Takte (17 und 18) bilden das Ende der Exposition.
Mit diesen Analysen als Beispiele sollten Sie nun in der Lage sein, den Rest des Stückes zu zergliedern.
Sie werden dasselbe Muster von Wiederholungen das ganze Stück hindurch finden.
Wenn Sie mehr von seiner Musik analysieren, werden Sie weitere Komplexitäten einbeziehen müssen; er mag drei- oder sogar viermal wiederholen und andere Änderungen dazumischen, um die Wiederholungen zu tarnen.
Klar ist, daß er ein Meister der Tarnung ist, so daß die Wiederholungen und andere Strukturen üblicherweise nicht offensichtlich sind, wenn man nur der Musik ohne jegliche Absicht zur Analyse zuhört.

Mozarts Formel wurde wahrscheinlich hauptsächlich dazu entwickelt, seine Produktivität zu steigern.
Trotzdem mag er bestimmte magische (hypnotische?, \enquote{süchtig machende}?) Kräfte in den Wiederholungen der Wiederholungen gefunden haben, und er hatte wahrscheinlich seine eigenen musikalischen Gründe, die Stimmungen seiner Themen in der von ihm benutzten Reihenfolge anzuordnen.
D.h., wenn man seine Themen weiter nach den Stimmungen, die sie hervorrufen, klassifiziert, dann findet man, daß er die Stimmungen immer in derselben Reihenfolge anordnet.
Es stellt sich hier die Frage: \enquote{Wenn wir tiefer und tiefer graben, werden wir nur mehr von diesen einfachen strukturellen bzw. mathematischen Mitteln finden, die einfach aufeinander gestapelt sind, oder steckt mehr in der Musik?}
Es muß fast mit Sicherheit mehr dahinter stecken, aber bis jetzt hat noch niemand Hand daran gelegt, nicht einmal die großen Komponisten selbst - zumindest, soweit sie es uns gesagt haben.
Deshalb scheint es so, als ob das einzige, das wir Normalsterblichen tun können, weitergraben ist.

Der oben erwähnte Musikprofessor, der eine Vorlesung über Mozarts Formel hielt, behauptete auch, daß die Formel so streng befolgt wurde, daß man sie benutzen kann, um Mozarts Kompositionen zu identifizieren.
Elemente dieser Formel sind jedoch unter Komponisten wohlbekannt.
Somit ist Mozart nicht der Erfinder dieser Formel und ähnliche Formeln wurden wahrscheinlich von den Komponisten seiner Zeit ausgiebig benutzt.
Insbesondere einige von Salieris Kompositionen befolgen eine sehr ähnliche Formel; vielleicht war das ein Versuch Salieris, Mozart nachzuahmen.
Deshalb muß man einige Details der Mozart eigenen Formel kennen, damit man sie benutzen kann, um seine Kompositionen zu identifizieren.

Es gibt wenig Zweifel daran, daß eine starke Wechselwirkung zwischen Musik und Genie besteht.
Wir wissen nicht einmal, ob Mozart ein Komponist war, weil er ein Genie war, oder ob der Umstand, daß er von Geburt an beträchtlichen Kontakt mit Musik hatte, die Genialität erzeugte.
Die Musik trug sicherlich zur Entwicklung seines Gehirns bei.
Es kann sehr gut sein, daß Wolfgang Amadeus selbst das beste Beispiel für den \enquote{Mozart-Effekt} war, obwohl er nicht den Nutzen aus seinen eigenen Meisterwerken hatte.
In diesen ersten paar Jahren des neuen Jahrtausends fangen wir gerade an, einige Geheimnisse der Funktion des Gehirns zu verstehen.
Zum Beispiel dachte man bis vor kurzem teilweise zu unrecht, daß ein bestimmter Teil der geistig behinderten Menschen ein ungewöhnliches musikalisches Talent hätte.
Es stellt sich heraus, daß Musik einen starken Effekt auf die tatsächliche Funktionsweise des Gehirns und seiner motorischen Kontrolle hat.
Das ist einer der Gründe, warum wir immer Musik beim Tanzen oder Trainieren benutzen.
Der beste Beweis dafür kommt von Alzheimer-Patienten, die ihre Fähigkeit sich selber anzuziehen verloren haben, weil sie die verschiedenen Arten der Kleidungsstücke nicht erkennen können.
Es wurde entdeckt, daß diese Patienten sich oft selbst anziehen können, wenn man diese Prozedur mit der richtigen Musik begleitet!
\enquote{Richtige Musik} ist normalerweise Musik, die sie in früher Jugend gehört haben oder ihre Lieblingsmusik.
Deshalb können geistig behinderte Menschen, die extrem unbeholfen sind, wenn sie alltägliche Verrichtungen ausführen, sich plötzlich hinsetzen und Klavier spielen, wenn die Musik von der richtigen Art ist, die ihr Gehirn stimuliert.
Sie müssen deshalb nicht musikalisch talentiert sein; es ist die Musik, die ihnen neue Fähigkeiten verleiht.
In einem größeren Maßstab ist es natürlich nicht nur die Musik, die diese magischen Auswirkungen auf das Gehirn hat, wie es von einigen Behinderten bewiesen wird, die unglaubliche Mengen an Informationen auswendig lernen können oder mathematische Kunststücke ausführen können, die normale Menschen nicht ausführen können.
Es existiert ein grundlegenderer interner Rhythmus im Gehirn, den die Musik anregen kann.
Ich weiß nicht, was dieser Mechanismus ist, aber er muß irgendwie dem Taktzyklus von Computerchips analog sein.
Ohne den Taktzyklus würden diese Chips nicht arbeiten, und das Maß für ihre Leistung ist die Taktrate - 3 GHz-Chips sind besser als 1 GHz-Chips.

Wenn Musik solch nachhaltige Wirkungen auf Behinderte erzeugen kann, stellen Sie sich vor, was sie mit dem Gehirn eines aufblühenden Genies tun könnte, insbesondere während der Entwicklung des Gehirns in früher Kindheit.
Diese Auswirkungen gelten für jeden, der Klavier spielt, nicht nur für Behinderte oder Genies.
Haben Sie jemals gute Tage und schlechte Tage gehabt?
Haben Sie jemals bemerkt, daß wenn Sie ein Stück das erste Mal lernen, Sie es plötzlich unglaublich gut spielen können, und es dann verlieren, wenn Sie weiterüben, oder daß Sie viel besser spielen, wenn Sie mit guten Spielern - z.B. in einer Kammermusikgruppe - spielen?
Haben Sie es schwer gefunden aufzutreten, wenn das Publikum aus Leuten besteht, die das Stück besser spielen können als Sie es können?
Spielen Sie besser, wenn die Musik gut herauskommt und schlechter, sobald Sie Fehler machen?
Wahrscheinlich ja, und die Antworten auf diese Fragen liegen in der Beziehung zwischen der Musik und dem Gehirn.
Deshalb sollte uns das Verständnis dieser Beziehung sehr dabei helfen, einige dieser Schwierigkeiten zu überwinden.


\label{c1iv4Gruppe}

Die Benutzung der mathematischen Mittel ist tief in Beethovens Musik eingebettet.
Deshalb ist das einer der besten Plätze, um nach Informationen über die Beziehung zwischen Mathematik und Musik zu graben.
Ich sage nicht, daß andere Komponisten keine mathematischen Mittel benutzen.
Praktisch jede musikalische Komposition hat mathematische Grundlagen.
Beethoven war jedoch in der Lage, diese mathematischen Mittel ins Extreme zu erweitern.
Durch das Analysieren dieser Extremfälle können wir überzeugendere Beweise dafür finden, welche Arten von Mitteln er benutzte.

Wir wissen alle, daß Beethoven niemals wirklich höhere Mathematik studiert hat.
Trotzdem nahm er eine erstaunliche Menge Mathematik in seine Musik auf und das auf sehr hohen Stufen.
Der Anfang seiner Fünften Symphonie ist ein erstklassiger Fall, aber Beispiele wie dieses sind überaus zahlreich.
Er \enquote{benutzte} Konzepte der Gruppentheorie, um diese berühmte Symphonie zu komponieren.
Tatsächlich benutzte er, was Kristallographen die Raumgruppe der Symmetrie-Transformationen nennen!
Diese Gruppe bestimmt viele fortgeschrittene Technologien wie die Quantenmechanik, Kernphysik und Kristallographie, die die Fundamente der heutigen technischen Revolution sind.
Auf dieser Abstraktionsstufe \textbf{sind ein Diamantkristall und Beethovens 5. Symphonie ein und dasselbe!}
Ich werde diese bemerkenswerte Beobachtung im folgenden erklären.

Die Raumgruppe, die Beethoven \enquote{benutzte} (er hatte sicher einen anderen Namen dafür) wurde angewandt, um Kristalle zu charakterisieren, wie z.B. Silizium oder Diamant.
Es sind die Eigenschaften der Raumgruppe, die Kristallen gestatten, makellos zu wachsen, und deshalb ist die Raumgruppe die absolute Grundlage für die Existenz von Kristallen.
Da Kristalle durch die Raumgruppe charakterisiert sind, führt ein Verständnis der Raumgruppe zu einem grundlegenden Verständnis der Kristalle.
Das war hervorragend für Materialforscher, die daran arbeiteten, Kommunikationsprobleme zu lösen, weil die Raumgruppe den Rahmen bot, von dem aus sie ihre Studien starten konnten.
Es ist so, als ob die Physiker von New York nach San Francisco fahren müßten und die Mathematiker ihnen eine Straßenkarte geben würden!
So haben wir den Siliziumtransistor perfektioniert, der zu integrierten Schaltkreisen und der Computer-Revolution geführt hat.
Was ist also die Raumgruppe? Und warum war diese Gruppe so nützlich, um diese Symphonie zu komponieren?

Gruppen werden durch eine Reihe von Eigenschaften definiert.
Mathematiker haben herausgefunden, daß Gruppen, die auf diese Art definiert werden, mathematisch manipuliert werden können, und Physiker fanden sie nützlich: d.h., diese bestimmten Gruppen, die Mathematiker und Wissenschaftler interessierten, bieten uns einen Pfad zur Realität.
Eine der Eigenschaften von Gruppen ist, daß Sie aus Elementen und Operationen bestehen.
Eine weitere Eigenschaft ist, daß wenn man eine Operation auf ein Element ausführt, man ein anderes Element derselben Gruppe erhält.
Eine vertraute Gruppe ist die Gruppe der ganzen Zahlen: \-1, 0, 1, 2, 3 usw.
Eine Operation dieser Gruppe ist die Addition: 2 + 3 = 5.
Beachten Sie, daß die Anwendung der Operation + auf die Elemente 2 und 3 zu einem anderen Element der Gruppe, 5, führt.
Da die Operationen auch ein Element in ein anderes transformieren, werden sie auch Transformationen genannt.
Ein Element der Raumgruppe kann alles in jedem Raum sein: ein Atom, ein Frosch oder eine Note in jeder musikalischen Raumdimension wie Tonhöhe, Geschwindigkeit oder Lautstärke.
Die Operationen der Raumgruppe, die für die Kristallographie relevant sind, sind Translation, Rotation, Spiegelung, Inversion und die Unäre Operation.
Diese sind fast selbsterklärend (Translation bedeutet, daß man das Element eine bestimmte Entfernung im Raum bewegt), außer bei der Unären Operation, welche das Element im Grunde unverändert läßt.
Diese ist jedoch ein wenig subtil, weil sie nicht das gleiche ist wie die Gleichheitstransformation, und wird deshalb in den Lehrbüchern immer am Schluß aufgelistet.
Unäre Operationen sind im allgemeinen mit dem speziellsten Element der Gruppe verbunden, das wir das Unäre Element nennen könnten; in der oben erwähnten Gruppe der ganzen Zahlen wäre das die 0 für die Addition und die 1 für die Multiplikation (5+0 = 5x1 = 5).

Lassen Sie mich demonstrieren, wie man diese Raumgruppe im täglichen Leben benutzen könnte.
Können Sie erklären, warum beim Blick in den Spiegel die linke Hand zur rechten herumgedreht wird (und umgekehrt) aber Ihr Kopf nicht hinunter zu Ihren Füßen rotiert?
Die Raumgruppe sagt uns, daß man nicht die rechte Hand drehen und eine linke Hand bekommen kann, weil links-rechts eine Spiegelungsoperation ist, keine Rotation.
Beachten Sie, daß dies eine merkwürdige Transformation ist: obwohl Ihre rechte Hand im Spiegel Ihre linke Hand ist, ist die Warze auf Ihrer rechten Hand nun auf der linken Hand Ihres Spiegelbildes.
Die Spiegelungsoperation ist der Grund, warum die rechte Hand zur linken Hand wird, wenn man in einen flachen Spiegel schaut; ein Spiegel kann jedoch keine Rotation ausführen, deshalb bleibt Ihr Kopf oben und die Füße bleiben unten.
Gekrümmte Spiegel, die optische Streiche spielen (wie die Umkehrung der Position von Kopf und Füßen) sind komplexere Spiegel, die zusätzliche Operationen der Raumgruppe durchführen können, und die Gruppentheorie ist beim Analysieren von Bildern in einem gekrümmten Spiegel genauso hilfreich.
Die Lösung des Problems des Bildes im flachen Spiegel erschien ziemlich einfach, weil wir einen Spiegel zu Hilfe hatten und wir so vertraut mit Spiegeln sind.
Dasselbe Problem kann noch einmal anders dargelegt werden, und es wird sofort viel schwieriger, so daß die Notwendigkeit der Gruppentheorie für die Hilfe zur Lösung des Problems offensichtlicher wird.
Wenn Sie einen rechtshändigen Handschuh von innen nach außen kehren, wird er ein rechtshändiger bleiben oder wird er zu einem linkshändigen Handschuh?
Ich überlasse es Ihnen, das herauszufinden (Tip: benutzen Sie einen Spiegel).

Lassen Sie uns sehen, wie Beethoven sein intuitives Verständnis der räumlichen Symmetrie benutzte, um seine 5. Symphonie zu komponieren.
Dieser berühmte erste Satz ist zu einem großen Teil aus einem einzigen musikalischen Thema aufgebaut, das aus vier Noten besteht, von denen die ersten drei Wiederholungen derselben Note sind.
Da die vierte Note anders ist, wird sie Überraschungsnote genannt und trägt den Schlag.
Dieses musikalische Thema kann schematisch durch die Folge 555\textbf{3} repräsentiert werden, wobei \textbf{3} die Überraschungsnote ist.
Das ist eine auf der Tonhöhe basierende Raumgruppe; Beethoven benutzte einen Raum mit 3 Dimensionen: Tonhöhe, Zeit und Lautstärke.
Ich werde in der folgenden Diskussion nur die Tonhöhen- und Zeitdimension berücksichtigen.
Beethoven beginnt seine Fünfte Symphonie indem er zunächst ein Element seiner Gruppe vorstellt: drei wiederholte Noten und eine Überraschungsnote, 555\textbf{3}.
Nach einer kurzen Pause, um uns Zeit zu geben, sein Element zu erkennen, führt er eine Translationsoperation aus: 444\textbf{2}.
Jede Note wird nach unten verschoben.
Das Ergebnis ist ein weiteres Element derselben Gruppe.
Nach einer weiteren Pause, so daß wir seinen Translationsoperator erkennen können, sagt er \enquote{Ist das nicht interessant? Laß uns Spaß haben!} und demonstriert das Potential dieses Operators mit einer Serie von Translationen, die Musik erzeugt.
Um sicherzustellen, daß wir sein Konstrukt verstehen, mischt er dieses Mal keine anderen, komplizierteren Operatoren darunter.
In der darauffolgenden Serie von Takten fügt er zuerst den Rotationsoperator hinzu, was \textbf{3}555 erzeugt, und anschließend den Spiegelungsoperator, was zu \textbf{7}555 führt.
Irgendwo nahe der Mitte des ersten Satzes führt er schließlich das ein, was als Unäres Element interpretiert werden kann: 555\textbf{5}.
Beachten Sie, daß die Noten einfach wiederholt werden, was die Unäre Operation ist.

In den letzten schnellen Takten kehrt er zur selben Gruppe zurück, benutzt aber nur das Unäre Element, und zwar auf eine Weise, die eine Stufe komplexer ist.
Es wird immer dreimal wiederholt.
Das seltsame daran ist, daß diesem eine vierte Sequenz folgt - eine Überraschungssequenz 765\textbf{4}, die kein Element ist.
Zusammen mit dem dreifach wiederholten Unären Element bildet die Überraschungssequenz eine Supergruppe der ursprünglichen Gruppe.
Er hat sein Gruppenkonzept verallgemeinert!
Die Supergruppe besteht nun aus drei Elementen und einem Nichtelement der anfänglichen Gruppe, was die Bedingungen der anfänglichen Gruppe erfüllt (drei Wiederholungen und eine Überraschung).

Somit liest sich der Anfang von Beethovens Fünfter Symphonie, wenn man ihn in eine mathematische Sprache übersetzt, fast Satz für Satz wie das erste Kapitel eines Lehrbuchs für Gruppentheorie!
Erinnern Sie sich daran, daß die Gruppentheorie eine der höchsten Formen von Mathematik ist.
Das Material wird sogar in der richtigen Reihenfolge präsentiert, wie es in Lehrbüchern auftritt, von der Einführung des Elements bis zum Gebrauch der Operatoren, angefangen mit dem einfachsten, der Translation und am Schluß der subtilste, der Unäre Operator.
Beethoven demonstriert sogar die Allgemeingültigkeit des Konzepts, indem er eine Supergruppe aus der ursprünglichen Gruppe erzeugt.

Beethoven war von diesem 4-notigen Thema besonders angetan und benutzte es in vielen seiner Kompositionen, so z.B. im ersten Satz der Klaviersonate \enquote{Appassionata}, siehe LH in Takt 10.
Als Meister seines Fachs vermied er bei der Appassionata sorgsam die auf der Tonhöhe basierende Raumgruppe und benutzte andere Räume - er transformierte das Thema in einem Temporaum und einem Lautstärkenraum (Takte 234 bis 238).
Das ist eine weitere Unterstützung der Vorstellung, daß er einen intuitiven Begriff von der Gruppentheorie gehabt haben mußte und bewußt zwischen diesen Räumen unterschieden hat.
Es scheint eine mathematische Unmöglichkeit zu sein, daß diese vielen Übereinstimmungen seiner Konstrukte mit der Gruppentheorie nur durch Zufall entstanden sind und ist quasi ein Beweis, daß er irgendwie mit diesen Konzepten gespielt hat.

Warum war dieses Konstrukt in dieser Einführung so nützlich?
Es bietet mit Sicherheit eine einheitliche Plattform, an die man seine Musik anknüpfen kann.
Die Einfachheit und Einheitlichkeit gestatten dem Publikum, sich ohne Ablenkung nur auf die Musik zu konzentrieren.
Es hat auch einen süchtig machenden Effekt.
Diese unterschwelligen Wiederholungen (mal angenommen, das Publikum weiß nicht, daß er dieses bestimmte Mittel einsetzte) können einen großen emotionalen Effekt erzeugen.
Es ist wie bei einem Zaubertrick - er hat einen viel größeren Effekt, wenn wir nicht wissen, wie der Zauberer es macht.
Es ist eine Möglichkeit, das Publikum ohne sein Wissen zu kontrollieren.
So wie Beethoven ein intuitives Verständnis dieses gruppenartigen Konzepts hatte, so können wir alle spüren, daß irgendeine Art von Muster existiert, ohne daß wir es ausdrücklich erkennen.
Mozart erreichte durch die Verwendung von Wiederholungen einen ähnlichen Effekt.

Das Wissen um diese gruppenartigen Mittel, die er benutzt, ist für das Spielen seiner Musik sehr nützlich, weil es Ihnen genau sagt, was Sie tun sollten und was nicht.
Ein weiteres Beispiel davon findet man im dritten Satz seiner Waldstein-Sonate, in der der ganze Satz auf einem 3-notigen Thema basiert, das durch 15\textbf{5} repräsentiert wird (das erste CG\textbf{G} am Anfang).
Er macht das gleiche mit dem anfänglichen Arpeggio des ersten Satzes der Appassionata, mit einem Thema, das durch 53\textbf{1} repräsentiert wird (das erste CAb\textbf{F}).
In beiden Fällen verliert die Musik ihre Struktur, Tiefe und Spannung, sofern man nicht den Schlag auf der letzten Note beibehält.
Das ist bei der Appassionata besonders interessant, weil man bei einem Arpeggio normalerweise den Schlag auf die erste Note setzt, und viele Schüler machen tatsächlich diesen Fehler.
Wie in der Waldstein-Sonate wird dieses anfängliche Thema den ganzen Satz hindurch wiederholt und wird zunehmend offensichtlich, wenn der Satz voranschreitet.
Aber zu diesem Zeitpunkt ist das Publikum süchtig danach und merkt nicht einmal, daß es die Musik dominiert.
Diejenigen, die es interessiert, können gegen Ende des ersten Satzes der Appassionata nachsehen, wo Beethoven das Thema zu 31\textbf{5} transformiert und es in Takt 240 auf eine extreme und fast lächerliche Stufe anhebt.
Trotzdem wird der größte Teil des Publikums keine Vorstellung davon haben, welches Mittel Beethoven benutzt hat, und den wilden Höhepunkt genießen, der offensichtlich lächerlich extrem ist aber inzwischen eine mysteriöse Vertrautheit in sich trägt, weil das Konstrukt dasselbe ist und Sie es hunderte Male gehört haben.
Beachten Sie, daß dieser Höhepunkt viel von seinem Effekt verliert, wenn der Pianist nicht das Thema herausstellt (das im ersten Takt eingeführt wurde!) und die Schlagnote betont.

Beethoven liefert uns die Begründung für das unerklärliche 53\textbf{1}-Arpeggio am Anfang der Appassionata, wenn sich das Arpeggio in Takt 35 in das Hauptthema umwandelt.
Dann entdecken wir, daß das Arpeggio am Anfang eine von seinem Hauptthema abgeleitete Inversion ist und warum der Schlag dort ist, wo er ist.
Deshalb ist der Anfang dieses Stücks, bis Takt 35, eine psychologische Vorbereitung auf eines der schönsten Themen, das er komponiert hat.
Er wollte die Vorstellung des Themas in unser Gehirn einpflanzen, noch bevor wir es hören!
Das mag eine Erklärung dafür sein, warum dieses fremdartige Arpeggio am Anfang unter Benutzung einer unlogischen Akkordprogression zweimal wiederholt wird.
Durch eine Analyse dieser Art wird die Struktur des ganzen ersten Satzes offenbar, was uns dabei hilft, das Stück auswendig zu lernen, zu interpretieren und korrekt zu spielen.

Die Benutzung gruppentheoretischer Konzepte könnte eine zusätzliche Dimension sein, die Beethoven in seine Musik eingeflochten hat, vielleicht um uns wissen zu lassen, wie schlau er war, falls wir die Botschaft immer noch nicht empfangen haben.
Es kann der Mechanismus sein oder nicht sein, mit dem er die Musik generiert hat.
Deshalb gibt uns die obige Analyse nur einen kleinen Blick auf die mentalen Prozesse, die Musik inspirieren.
Einfach diese Mittel zu benutzen, führt nicht zu Musik.
Oder kommen wir nahe an etwas heran, daß Beethoven wußte aber niemandem verraten hat?



