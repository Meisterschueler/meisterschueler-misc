% File: c1iii4c

\subsubsection{Bewegungen des Körpers}
\label{c1iii4c}

Viele Lehrer unterstützen \enquote{den Gebrauch des ganzen Körpers für das Klavierspielen} (siehe \hyperref[Whiteside]{Whiteside}).
Was bedeutet das?
Gibt es besondere Körperbewegungen, die für die Technik erforderlich sind?
Nicht wirklich; die Technik liegt in den Händen und in der \hyperref[c1ii14]{Entspannung}\index{Entspannung}.
Da jedoch die Hände mit dem Körper verbunden sind und durch ihn unterstützt werden, kann man nicht einfach in einer Position sitzen und hoffen zu spielen.
Wenn man in den höheren Lagen spielt, sollte der Körper den Händen folgen, und Sie könnten sogar ein Bein in die Gegenrichtung strecken, um den Körper auszubalancieren, wenn es nicht für die Pedale gebraucht wird.
Auch erfordert selbst die kleinste Bewegung eines Fingers die Aktivierung einer Reihe von Muskeln -- mindestens bis zur Körpermitte (nahe des Brustbeins) hin, wenn nicht sogar bis zu den Beinen und anderen Körperteilen, die den Körper unterstützen.
Die Entspannung ist, wegen der schieren Größe der einbezogenen Muskeln, im Körper genauso wichtig wie in den Händen und den Fingern.
Obwohl die meisten der erforderlichen Körperbewegungen einfach mit dem gesunden Menschenverstand zu verstehen sind und nicht derart wichtig erscheinen, sind die Körperbewegungen nichtsdestoweniger für das Klavierspielen absolut notwendig.
Lassen Sie uns diese Bewegungen besprechen, von denen einige nicht völlig offensichtlich sind.

Der wichtigste Aspekt ist die Entspannung.
Es ist die gleiche Art Entspannung, die Sie in den Händen und den Armen brauchen -- benutzen Sie nur die Muskeln, die für das Spielen erforderlich sind, und nur für die kurzen Momente, während denen sie gebraucht werden.
Entspannung bedeutet auch freies Atmen; wenn Ihre Kehle nach hartem Üben trocken ist, dann schlucken Sie nicht richtig, ein sicheres Zeichen von Anspannung.
\textbf{Entspannung ist eng mit der Unabhängigkeit eines jeden Teils des Körpers verbunden.
Als erstes müssen Sie, bevor Sie über nützliche Körperbewegungen nachdenken, sicherstellen, dass die Hände und Finger völlig vom Körper entkoppelt sind.
Wenn Sie nicht entkoppelt sind, dann wird der \hyperref[c1iii1b]{Rhythmus}\index{Rhythmus} unordentlich, und man kann alle Arten von unerwarteten Fehlern machen.
Wenn man außerdem nicht merkt, dass der Körper und die Hände gekoppelt sind, dann wird man sich fragen, warum man so viele merkwürdige Fehler macht, für die man keinen Grund findet.}
Dieses Entkoppeln ist beim \hyperref[c1ii25]{beidhändigen Spielen}\index{beidhändigen Spielen} besonders wichtig, weil das Koppeln die Unabhängigkeit der beiden Hände stört.
Koppeln ist eine der Ursachen der Fehler: Eine Bewegung in einer Hand erzeugt zum Beispiel durch den Körper eine unfreiwillige Bewegung in der anderen.
Das bedeutet nicht, dass man das Entkoppeln des Körpers während des \hyperref[c1ii7]{einhändigen Übens}\index{einhändigen Übens} ignorieren kann; im Gegenteil, das Entkoppeln sollte während der einhändigen Arbeit bewusst geübt werden.
Beachten Sie, dass das Entkoppeln ein einfaches Konzept und leicht auszuführen ist, wenn man es erst einmal gelernt hat, dass es aber körperlich ein komplexer Prozess ist.
Jede Bewegung in einer Hand erzeugt \textit{notwendigerweise} eine gleiche und entgegengesetzte Reaktion im Körper, die automatisch zur anderen Hand übertragen wird.
Deshalb erfordert das Entkoppeln einen aktiven Aufwand; es ist keine bloße passive Entspannung.
Glücklicherweise sind unsere Gehirne genügend entwickelt, sodass wir leicht das Konzept des Entkoppelns begreifen können.
Darum muss das Entkoppeln aktiv geübt werden.
Wenn Sie irgendeine neue Komposition lernen, wird immer ein wenig Kopplung vorhanden sein, bis Sie diese heraustrainieren.
Die schlimmste Art der Kopplung ist jene, die man während des Übens erwirbt, wenn man mit Stress übt oder versucht, etwas zu spielen, das zu schwierig ist.
Während der intensiven Bemühungen, die notwendig sind, um schwieriges Material zu spielen, kann ein Schüler jede Zahl von unnötigen Bewegungen verinnerlichen -- besonders während des beidhändigen Übens -, was schließlich das Spielen stört, wenn die Geschwindigkeit gesteigert wird.
Indem Sie einhändig auf Geschwindigkeit kommen, können Sie die meisten dieser aus dem beidhändigen Koppeln resultierenden Fehler vermeiden.

Der Körper wird wie oben beschrieben benutzt, um mittels der Schultern fortissimo zu spielen.
Er wird auch zum leisen Spielen benutzt, denn um leise zu spielen, braucht man eine stabile, konstante Plattform, von der aus man diese kleinen, kontrollierten Kräfte erzeugen kann.
Die Hand und der Arm haben selbst zu viele mögliche Bewegungen, um als stabile Plattform zu dienen.
Wenn sie sicher mit einem stabilen Körper verbunden sind, hat man eine viel stabilere Bezugsplattform.
Deshalb sollte die Ruhe des Pianissimo vom Körper ausgehen, nicht von den Fingerspitzen.
Und um mechanischen \enquote{Lärm} aufgrund von zusätzlichen Fingerbewegungen zu reduzieren, sollten die Finger soviel wie möglich auf den Tasten sein.
Tatsächlich bietet das Erfühlen der Tasten einen weiteren stabilen Bezugspunkt, von dem aus man spielen kann.
Wenn der Finger die Taste verlässt, hat man diesen wertvollen Bezugspunkt verloren, und der Finger kann nun überallhin wandern, was es schwierig macht, die nächste Note genau zu kontrollieren.


% zuletzt geändert 21.08.2011


